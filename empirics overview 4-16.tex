\documentclass[12pt,english]{article}
\usepackage{lmodern}
\usepackage[T1]{fontenc}
\usepackage[latin9]{inputenc}
\usepackage{geometry}
\usepackage{amsthm}
\usepackage{verbatim}
\geometry{verbose,tmargin=1in,bmargin=1in,lmargin=1in,rmargin=1in}
\usepackage{setspace}
%\usepackage{esint}
\onehalfspacing
\usepackage{babel}
\usepackage{amsmath}

\theoremstyle{remark}
\newtheorem*{remark}{Remark}
\begin{document}

\title{Empirics of my model: overview}
\author{Nicolas Fernandez-Arias}
\maketitle

\section{Introduction}
It would be nice to actually solve my model. And nest it and the standard model in a general model, and then the discrepancies between it and the standard model can be used for identification.


\section{Stylized facts}


\section{Identification}
We can imagine a model with 7 parameters: $\{\lambda,\nu,\chi,p,\xi,\beta,\rho \}$. This assumes the same innovation technology for entrants and incumbents. There will still be R\&D by incumbents in equilibrium since free entry does not occur immediately. 

\subsection{Level of innovation intensity}

\subsection{Relative innovation intensities}
One key prediction of my model is on the time-path of the ratio of innovation effort by incumbents and entrants. This time path is determined by $p,\lambda,\nu$. 

Incumbents and entrants have R\&D technology given by:
\begin{align*}
R(z) &= \chi z \phi(z) \\
\hat{R}(z;\bar{z}) &= \chi z \phi(\chi) \\
\phi(z) &= z^{-p}
\end{align*}

and given a choice of curvature and level productivity of this function, $\lambda$ is identified by the extent to which entrants innovate relative to incumbents. In equilibrium, entrant innovation effort is given by
\begin{align*}
\hat{z}(m) &= \xi \min(m,M) \\ 
M &= \Big[ \frac{\tilde{\beta}}{\lambda V(0)} \Big]^{-1/p}
\end{align*}

and incumbent innovation effort is
\begin{align*}
z(m) = \Big[ \frac{\tilde{\beta} - \nu W(m) - \nu V'(m)}{(1-p)(\lambda V(0) - V(m))} \Big]^{-1/p}
\end{align*}

This implies that, for all $m$, we have
\begin{align}
\frac{z(m)}{\hat{z}(m)} &= \Big[ \frac{\tilde{\beta} - \nu W(m) - \nu V'(m)}{\tilde{\beta}} \times \frac{\lambda V(0)}{(1-p)(\lambda V(0) - V(m))} \Big]^{-1/p} \label{relative_innovation_intensities} \\
 &= D(m) (1-p)^{1/p}
\end{align}

In addition, the fact that $V'(m) = 0,W(m) = 0 \text{ if } m \ge M$ implies that, for $m > M$, $D(m) = D(M)$ and so
\begin{align}
\frac{z(m)}{\hat{z}(m)} = D(M) (1-p)^{1/p}
\end{align}

\paragraph{Identification of p} 
First, consider equation (\ref{relative_innovation_intensities}). From here we see that $p$ shifts the ratio for all $m$ in a similar way. In particular, taking logs, get
\begin{align}
\log(z(m)) &= \log(\hat{z}(m)) - \frac{1}{p} \log \big( D(m) \big) - \frac{\log(1-p)}{p} \\
		   &= \log(\hat{z}(m)) - \underbrace{\frac{1}{p} \big(D(m) - D(M) \big)}_{\text{Slope in }m} - \underbrace{\frac{\log(1-p) + D(M)}{p}}_{\text{Level shift}}
\end{align}

Hence, taking other parameters as given, $p$ determines:\footnote{I am only keeping track of \textit{direct effects} of changing $p$. Changing $p$ also changes the function $D$ through its effect on equilibrium values $V,W$. Still thinking about how to make this all work.}
\begin{enumerate}
	\item First term: higher $p$ attenuates the rate of change of ratio in $m$
	\item Second term: $p$ shifts level of ratio. Direction?
\end{enumerate}

\paragraph{Identification of $\lambda,\nu$}
We have defined
\begin{align*}
D(m) = \Big( \frac{\tilde{\beta}-\nu W(m) - \nu V'(m)}{\tilde{\beta}} \times \frac{\lambda V(0)}{(1-p)(\lambda V(0) - V(m))} \Big)^{-1/p}
\end{align*}







\section{Extensions}













\end{document}