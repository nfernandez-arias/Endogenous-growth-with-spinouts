\documentclass[12pt,english]{article}
\usepackage{lmodern}
\usepackage[T1]{fontenc}
\usepackage[latin9]{inputenc}
\usepackage{geometry}
\usepackage{amsthm}
\usepackage{verbatim}
\geometry{verbose,tmargin=1in,bmargin=1in,lmargin=1in,rmargin=1in}
\usepackage{setspace}
%\usepackage{esint}
\onehalfspacing
\usepackage{babel}
\usepackage{amsmath}

\theoremstyle{remark}
\newtheorem*{remark}{Remark}
\begin{document}

\title{Nesting my model in a framework which includes the standard model}
\author{Nicolas Fernandez-Arias}
\maketitle

Moll's suggestion was to nest my model in a general framework which nests the standard model for some parameter value (or at least, in the limit of some parameter value).

The obvious way to do this is to specify that, with some intensity $\theta$, the good $j$ switches to free entry. The only difference in the analysis of the model is that the mass of agents immediately jumps to $M$ upon the realization of this shock. The question is then: how does this affect the predictions of the model?

Hard to know this without actually solving the model. That seems to be one of the main things I should try to do before next week.










\end{document}
