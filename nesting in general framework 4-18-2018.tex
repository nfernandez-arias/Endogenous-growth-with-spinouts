\documentclass[12pt,english]{article}
\usepackage{lmodern}
\usepackage[T1]{fontenc}
\usepackage[latin9]{inputenc}
\usepackage{geometry}
\usepackage{amsthm}
\usepackage{verbatim}
\geometry{verbose,tmargin=1in,bmargin=1in,lmargin=1in,rmargin=1in}
\usepackage{setspace}
%\usepackage{esint}
\onehalfspacing
\usepackage{babel}
\usepackage{amsmath}

\theoremstyle{remark}
\newtheorem*{remark}{Remark}
\begin{document}

\title{Nesting my model in a framework which includes the standard model}
\author{Nicolas Fernandez-Arias}
\maketitle

Moll's suggestion was to nest my model in a general framework which nests the standard model for some parameter value (or at least, in the limit of some parameter value).

The obvious way to do this is to specify that, with some intensity $\theta$, the good $j$ switches to free entry. The only difference in the analysis of the model is that the mass of agents immediately jumps to $M$ upon the realization of this shock. The question is then: how does this affect the predictions of the model?

Hard to know this without actually solving the model, since $\theta$ will affect the equilibrium values of $V,W$. But, intuitively, what will be the effect of increasing $\theta$? Because I am not taking into account the effects on $V,W$ from this change, these are essentially partial equilibrium results. However, I suspect the ones related to comparing activities of entrants to activities of incumbents will hold true, since general equilibrium effects tend to affect both values $V,W$ equally (e.g., the total rate of creative destruction reduces $V,W$ similarly). 

Still, my conjectures are:

\begin{enumerate}
	\item Monopolies will be destroyed more often
	\item Entrant share of R\&D will be higher
	\item Non-spinout share of entrant R\&D will be higher
	\item Share of entrants that are spinouts will be lower?
\end{enumerate}

The deeper problem is: what do the entrants in my model represent in the data? If I cannot observe them (or their R\&D spending) directly, I cannot directly use the above moments (or the ones discussed in the ``empirics overview" document). 

What can I observe for sure? Well, I can observe firms that have been around a while, but what I am interested in is \textit{products} that have been around for a while in their current form. 













\end{document}
