
\documentclass[12pt,english]{article}
\usepackage{lmodern}
\usepackage[T1]{fontenc}
\usepackage[latin9]{inputenc}
\usepackage{geometry}
\usepackage{amsthm}
\usepackage{verbatim}
\geometry{verbose,tmargin=1in,bmargin=1in,lmargin=1in,rmargin=1in}
\usepackage{setspace}
%\usepackage{esint}
\onehalfspacing
\usepackage{babel}
\usepackage{amsmath}

\theoremstyle{remark}
\newtheorem*{remark}{Remark}
\begin{document}

\title{Implementation of Empirics for "A Model of Endogenous Growth with Employee Spinouts"}
\author{Nicolas Fernandez-Arias}
\maketitle

\section{Introduction}

\section{Preliminary Analysis}

Problem with preliminary analysis: RD user cost affects spinouts, too. So no exclusion restriction. 

Why do we need this? I guess because otherwise the noise in $Z_{it}$ is correlated with the noise in $Y_{it}$, and the noise in $Z_{it}$ is present in the estimate of $X_{it}$ put into the second stage. 

In our case, we will predict of course more R\&D when the policy is enacted, but we won't know if the effect is coming from that or from the fact that the policy was enacted.

Only options are:
\begin{enumerate}
	\item Ignore, arguing that R\&D tax incentives are likely to affect R\&D more than the decision to form a spinout, which requires an idea in the first place
	\item Somehow obtain R\&D information at the state-industry level, \textsl{}
\end{enumerate}

\paragraph{Computing $RD_{it}$}

Don't have  yearly series for this (check again), so will aggregate up using Compustat firms, using a geographic / industry weighting potentially informed by the patent data.

\subsection{Specification 1: OLS}

Have
\begin{align*}
	Y_{it} &= \alpha_i + \gamma_t + \beta RD_{it} + \epsilon_{it}
\end{align*}

where $Y_{it}$ is the outcome of interest in $i,t$, where $i$ indexes either state or industry. 

\subsection{Specification 2: Reduced Form}

Have
\begin{align*}
	Y_{it} &= \alpha_i + \gamma_t + \beta (RD user cost)_{it} + \epsilon_{it}
\end{align*}









\end{document}