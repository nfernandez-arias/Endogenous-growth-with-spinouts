\documentclass[12pt,english]{article}
%\usepackage{lmodern}
\linespread{1.05}
\usepackage{mathpazo}
%\usepackage{mathptmx}
%\usepackage{utopia}
\usepackage{microtype}
\usepackage[T1]{fontenc}
\usepackage[latin9]{inputenc}
\usepackage[dvipsnames]{xcolor}
\usepackage{geometry}
\usepackage{amsthm}
\usepackage{amsfonts}
\usepackage{amssymb}

\usepackage{courier}
\usepackage{verbatim}
\usepackage[round]{natbib}
\bibliographystyle{plainnat}


\definecolor{red1}{RGB}{128,0,0}
%\geometry{verbose,tmargin=1.25in,bmargin=1.25in,lmargin=1.25in,rmargin=1.25in}
\geometry{verbose,tmargin=1in,bmargin=1in,lmargin=1in,rmargin=1in}
\usepackage{setspace}

\usepackage[colorlinks=true, linkcolor={red!70!black}, citecolor={blue!50!black}, urlcolor={blue!80!black}]{hyperref}
%\usepackage{esint}
\onehalfspacing
\usepackage{babel}
\usepackage{amsmath}
\usepackage{graphicx}

\theoremstyle{remark}
\newtheorem{remark}{Remark}
\begin{document}
	
\title{Solving for the static equilibrium allocation}
\author{Nicolas Fernandez-Arias}
\maketitle

\tableofcontents

\section{Equilibrium intermediate goods prices $p_j$, demand $k_j$}
(Still need to show derivation, easy) Final goods maximization implies inverse demand
\begin{align*}
	p_j &= L_F^{\beta} q_j^{\beta} k_j^{-\beta}
\end{align*}

Marginal cost of producing intermediate variety $j$ is $\overline{w}/\overline{q}$. 

Profit maximization for monopolist $j$ is 
\begin{align*}
	\pi(q_j) = \max_{k_j \ge 0} \Big\{ L_F^{\beta} q_j^{\beta} k_j^{1-\beta} - \frac{\overline{w}}{\overline{q}} \Big\} 
\end{align*}

This yields
\begin{align*}
	k_j &= \Big[ \frac{(1-\beta) \overline{q}}{\overline{w}} L_F q_j  \Big] \\
	p_j &= \frac{\overline{w}}{(1-\beta) \overline{q}}
\end{align*}

\section{Calculating production wage $\overline{w}$}

Final goods production:
\begin{align*}
	Y &= \frac{L_F^{\beta}}{1-\beta} \int_0^1 q_j^{\beta} k_j^{1-\beta} dj 
\end{align*}

First order condition with respect to labor demand:
\begin{align*}
	\frac{\beta}{1-\beta}L_F^{\beta-1} \int_0^1 q_j^{\beta} k_j^{1-\beta} dj &= \overline{w}
\end{align*}

Substitute intermediate goods demand:
\begin{align*}
	\frac{\beta}{1-\beta}L_F^{\beta-1} \int_0^1 q_j^{\beta} \Big[ \Big( \frac{(1-\beta)\overline{q}}{\overline{w}} \Big)^{\frac{1}{\beta}} L_F q_j \Big]^{1-\beta} dj = \overline{w}
\end{align*}

Simplify:
\begin{align*}
	\frac{\beta}{1-\beta} \Big[ \Big( \frac{(1-\beta) \overline{q}}{\overline{w}}\Big)^{\frac{1}{\beta}} \Big]^{1-\beta} \overline{q} &= \overline{w} \\
	\underbrace{\frac{\beta}{1-\beta} (1-\beta)^{\frac{1-\beta}{\beta}}}_{\tilde{C}(\beta)} \overline{w}^{\frac{\beta-1}{\beta}} \overline{q}^{\frac{1-\beta}{\beta}} \overline{q} &= \overline{w} \\
	\tilde{C}(\beta) \overline{q}^{\frac{1}{\beta}} &= \overline{w}^{\frac{1}{\beta}} \\
	\tilde{C}(\beta)^{\beta} \overline{q} &= \overline{w}
\end{align*}

Finally, define $C(\beta) = \tilde{C}(\beta)^{\beta}$ and we have
\begin{align*}
	\overline{w} &= C(\beta) \overline{q}
\end{align*}

\section{Equilibrium output}
Have
\begin{align*}
	Y &= \frac{L_F^{\beta}}{1-\beta} \int_0^1 q_j^{\beta} k_j^{1-\beta} dj
\end{align*}

Substitute in expression derived above for equilibrium $k_j$ and $\bar{q}/\bar{w} = C(\beta)^{-1}$. Get
\begin{align*}
	Y &= \frac{L_F^{\beta}}{1-\beta} \int_0^1 q_j^{\beta} \Big( \frac{(1-\beta)\bar{q}}{\bar{w}}L_F q_j \Big)^{1-\beta} dj \\
	  &= \frac{L_F^{\beta}}{1-\beta} \int_0^1 q_j^{\beta} \Big( (1-\beta)C(\beta)^{-1} L_F q_j \Big)^{1-\beta} dj \\
	  &= \Big((1-\beta)C(\beta)^{-1} \Big)^{1-\beta} \frac{L_F^{\beta}}{1-\beta} L_F^{1-\beta} \int_0^1 q_j^{\beta} q_j^{1-\beta} dj \\
	Y(L_F,\bar{q}) &= \frac{\Big((1-\beta)C(\beta)^{-1} \Big)^{1-\beta} }{1-\beta}L_F \bar{q}
\end{align*}

\section{Equilibrium welfare}
Define
\begin{align*}
	\textrm{Welfare} &= \int_0^{\infty} e^{-\rho t}Y(t) dt \\
	                 &= \frac{Y_0}{\rho - g}
\end{align*}
where 
\begin{align*}
	Y_0 &= Y(L_F,1) \\
	    &= \frac{\Big((1-\beta)C(\beta)^{-1} \Big)^{1-\beta} }{1-\beta}L_F
\end{align*}

\section{Equilibrium allocation of labor to final goods production}
Given $k_j,\overline{w}$ and $L_F + L_I  = 1 - L_{RD}$ we can calculate $L_F$ given $L_{RD}$. 

We start by computing $l_j$ as a function of $L_F, q_j,$ and $\overline{w}$:
\begin{align*}
	l_j &= k_j / \overline{q} \\
		&=  \Big[ \frac{1-\beta}{\overline{w}} \Big]L_F   q_j \\
		&= \frac{1-\beta}{C(\beta)} L_F \frac{q_j}{\overline{q}}
\end{align*}

Then we aggregate to obtain $L_I$ as a function of $L_F$:
\begin{align*}
	L_I &= \int_0^1 l_j dj \\
	    &= \int_0^1 \frac{1-\beta}{C(\beta)} L_F \frac{q_j}{\overline{q}} dj \\
	    &= \frac{1-\beta}{C(\beta)} L_F \int_0^1 \frac{q_j}{\overline{q}} dj \\
	    &= \frac{1-\beta}{C(\beta)} L_F 
\end{align*}

Finally, we can compute $L_F$ using $L_I + L_F = L - L_{RD}$: 
\begin{align*}
	L_F &= L - L_{RD} - L_I \\
	    &= L - L_{RD} - \frac{1-\beta}{C(\beta)} L_F \\
	L_F \Big( 1 +  \frac{1-\beta}{C(\beta)} \Big) &= L - L_{RD} \\
	L_F &= \frac{L - L_{RD}}{1 +  \frac{1-\beta}{C(\beta)}}
\end{align*}

\section{Equilibrium intermediate goods producer flow profit $\pi_j$}

Given $L_F(L_{RD})$ we can now compute profits:
\begin{align*}
	\pi_j &= (p_j - c_j) k_j \\
	      &= \Big( \frac{\overline{w}}{(1-\beta)\overline{q}} - \frac{\overline{w}}{\overline{q}} \Big) \Big[ \frac{(1-\beta) \overline{q}}{\overline{w}} L_F q_j  \Big] \\
	      &= \frac{\overline{w}}{\overline{q}} \Big( \frac{1}{1-\beta} - 1 \Big) \times \frac{\overline{q}}{\overline{w}} (1-\beta) L_F q_j \\
	      &= \Big( \frac{1}{1-\beta} - 1 \Big) (1-\beta) L_F q_j \\
	      &= (1 - (1-\beta)) L_F q_j \\
	      &= \beta L_F q_j 
\end{align*}

\section{Stationary distribution $\mu(m)$}

\subsection{Without CNCs}

I calculate $\mu(m)$ using 
\begin{align*}
	\mu(m) &= C_{\mu} e^{-\int_0^m \frac{a'(m') + \tau(m')}{a(m')}}dm' \\
	1 &= \int_0^{M} \mu(m)dm
\end{align*}

\subsection{With CNCs}

With CNCs, there will be a state $M > 0$ such that particles will be stuck there when they arrive until an innovation arrival occurs. Hence, there is a mass of particles at $M > 0$. 

We can compute the stationary mass at $M > 0$ by reducing the problem to a case without a mass point. Consider a modified problem with no mass points. For $m \le M$, everything is the same. For $m > M$, now suppose that $a(m) \equiv 1$ instead of $a(m) \equiv 0$; and assume that $\tau(m) = \tau(M)$. This modified Markov process has a stationary distribution $\tilde{\mu}$ with no mass points, with the property that 
\begin{align*}
\int_M^{\infty} \tilde{\mu}(m) dm = \mu(M)
\end{align*}

where $\mu(M)$ is the mass at $M$ in our original problem. 

The stationary distribution $\tilde{\mu}$ must satisfy the Kolmogorov Forward equation. Hence, 
\begin{align*}
\tilde{\mu}(m) &= \begin{cases}
C_{\mu} e^{-\int_0^m \frac{a'(m') + \tau(m')}{a(m')}dm'} &\textrm{for $m \le M$} \\
\tilde{\mu}(M) e^{-\tau(M)(m-M)} &\textrm{for $m > M$}
\end{cases} \\
1 &= \int_0^1 \mu(m) dm
\end{align*}

To compute, begin by computing 
\begin{align*}
\hat{\mu}(m) &= e^{-\int_0^m \frac{a'(m') + \tau(m')}{a(m')}dm'}
\end{align*}

Then normalize:
\begin{align*}
C_{\mu}^{-1} &= \int_0^M \hat{\mu}(m) dm + \frac{\hat{\mu}(M)}{\tau(M)} 
\end{align*}

Then the stationary distribution $\mu$ has a mass point at $m = M$ of $C_{\mu} \frac{\hat{\mu}(M)}{\tau(M)}$ and for $m \le M$ we have 
\begin{align}
\mu(m) &= C_{\mu} e^{-\int_0^m \frac{a'(m') + \tau(m')}{a(m')}dm'}
\end{align}


\section{Stationary average relative quality given $m$, $\gamma(m)$}

\subsection{Without CNCs}

First, I compute $t(m)$ (equilibrium time to reach state $m$) as 
\begin{align*}
t(m) &= \int_0^m a(m')^{-1} dm' 
\end{align*}

Next, I compute $\tilde{\gamma}(m) = \gamma(m) / C_{\gamma}$ as 
\begin{align*}
	\tilde{\gamma}(m) &= e^{-gt(m)}
\end{align*}

Finally, I compute $C_{\gamma}$ using 
\begin{align*}
	C_{\gamma}^{-1} = \int_0^{M} \tilde{\gamma}(m) \mu(m) dm
\end{align*}

Finally, 
\begin{align*}
	\gamma(m) &= C_{\gamma} \tilde{\gamma}(m) 
\end{align*}

\subsection{With CNCs}

With CNCs, now have a mass point at $M^{CNC}$. For $m < M$, $\gamma(m)$ is the same. For $m > M$, now $\gamma$ is undefined. Finally,
\begin{align}
	\gamma(M) &\equiv E[q/Q | m = M] \nonumber \\
	          &= \int_0^{\infty} \lim_{\varepsilon \to 0} \gamma(M-\varepsilon) e^{-gs} \Gamma(s) ds
\end{align} 

where $\Gamma(s)$ is the probability that a particle spends exactly $s$ years in state $M$.  
Once a particle is in state $M$, it remains there until a Poisson process with constant arrival rate $\tau(M)$ arrives. Hence, $\Gamma(s)$ is the pdf of an exponential distribution with intensity $\tau(M)$, 
\begin{align*}
	\Gamma(s) &= \tau(M) e^{-\tau(M)s}
\end{align*}

Therefore, 
\begin{align}
	\gamma(M) &= \int_0^{\infty} \lim_{\varepsilon \to 0} \gamma(M-\varepsilon) e^{-gs} \Gamma(s) ds \nonumber \\
		      &= \int_0^{\infty} \gamma_{-}(M) e^{-gs} \tau(M) e^{-(g + \tau(M))s} ds \nonumber \\
	          &= \gamma_{-}(M) \tau(M) \int_0^{\infty} e^{-(g+ \tau(M))s} ds \nonumber \\
   \therefore \gamma(M)  &= \frac{\gamma_{-}(M) \tau(M)}{g+\tau(M)}
\end{align}


\section{Stationary distribution $\mu(t)$}

We have the relationship
\begin{align*}
\mu(t)dt &= \mu(m)dm \\
\mu(t) &= \mu(m) \frac{dm}{dt}
\end{align*}

Substituting in
\begin{align*}
\frac{dm}{dt} = \nu a(m(t))
\end{align*}

Get 
\begin{align*}
\mu(t) &= \mu(m) \nu a(m(t))
\end{align*}

To implement, have $m_i = m(t_i)$, so
\begin{align*}
\mu^t_i &= \mu^m_i \nu a^m_i 
\end{align*}



\section{Calculating $L_{RD}$}

I compute
\begin{align*}
	L_{RD} &= \int_0^M \gamma(m) \frac{a(m)}{\nu} \mu(m)  dm
\end{align*}

Note that $\gamma(m) \frac{a(m)}{\nu}$ is average R\&D labor demand for an intermediate good in state $m$.

\section{Calculating $g$}

I compute
\begin{align*}
	g &= (\lambda - 1)\int_0^M \gamma(m) \mu(m) \tau(m) dm
\end{align*}
\end{document}