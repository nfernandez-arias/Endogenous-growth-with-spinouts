\documentclass[11pt,english]{article}
\usepackage{palatino}
\usepackage[T1]{fontenc}
\usepackage[latin9]{inputenc}
\usepackage{geometry}
\usepackage{amsthm}
\usepackage{courier}
\usepackage{verbatim}
\geometry{verbose,tmargin=1in,bmargin=1in,lmargin=1in,rmargin=1in}
\usepackage{setspace}
%\usepackage{esint}
\onehalfspacing
\usepackage{babel}
\usepackage{amsmath}

\theoremstyle{remark}
\newtheorem*{remark}{Remark}
\begin{document}
	
\title{Hack to solve incumbent HJB}
\author{Nicolas Fernandez-Arias}
\maketitle

If I implement the finite difference scheme on its own, it returns a solution which satisfies the HJB to very high precision \textit{except for at $0$}. 

This problem compounds, because even though it has a minor effect on $V(0)$, it completely throws off $V'(0)$, which changes the optimal policy $z_I^*(0)$ dramatically. 

My solution is that when I solve the HJB, I just set $z_1 = z_2$, which is approximately true in equilibrium by continuity. 

This then returns a solution $V$ that is still approximately wrong at $z_1$. While imperceptible to the naked eye, again it completely throws off $V'(0)$ meaning $z_1$ is no longer optimal. At every point $m > 0$, the finite difference version of the HJB is approximately satisfied. At $m = 0$, it is also approximately satisfied \textit{provided that} I use $V'(0) \approx \frac{V_3-V_2}{m_3 - m_2}$, which in equilibrium is approximately equal to $V'(0)$. 

The above definitely does not constitute as rigorous an argument as I would like to make, but I personally am convinced that the object I am arriving at is a very close approximation of a value function that satisfies the HJB of the incumbent. 




\end{document}