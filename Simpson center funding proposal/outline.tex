\documentclass[12pt,english]{article}
\usepackage{lmodern}
\usepackage[T1]{fontenc}
\usepackage[latin9]{inputenc}
\usepackage{geometry}
\usepackage{amsthm}
\usepackage{courier}
\usepackage{verbatim}
\geometry{verbose,tmargin=1in,bmargin=1in,lmargin=1in,rmargin=1in}
\usepackage{setspace}
%\usepackage{esint}
\onehalfspacing
\usepackage{babel}
\usepackage{amsmath}

\theoremstyle{remark}
\newtheorem*{remark}{Remark}
\begin{document}
	
\title{Funding proposal for VentureSource data for ``Endogenous growth with creative destruction by employee spinouts'' project}
\author{Nicolas Fernandez-Arias}
\maketitle

\section{Description of project}

This project consists of developing an model of endogenous growth with a new channel, calibrating it, and using it to analyze comparative statics and policy. 

The key mechanism is that a firm's current employees can, through learning-by-doing, become the future competition of the firm. 

\subsection{Model}

The model I am using is, technically, most closely related to the model in Akcigit \& Kerr, "Growth through Heterogeneous Innovations" (forthcoming in the JPE). The differences are:
\begin{enumerate}
	\item Workers are risk-neutral in my model
	\item In my model, the R\&D input is labor rather than the final good (and made some scaling assumptions on this to preserve a BGP)
	\item I have dropped ``external innovation'' by incumbents (and hence any explicit role for ``firm size'' - my analysis is at the product level)
	\item I have changed the innovation technology of entrants
	\item I have added a new class of potential entrant - an employee spinout - which can only form upon learning by working as an R\&D employee in that product line
\end{enumerate}

The resulting problem for either an incumbent or a spinout in a given product line $j$ then depends on a one-dimensional state variable, $m(j)$, which can be thought of as a measure of how widely disseminated the knowledge of ``how to compete in an race to develop the next step on the quality ladder for product line $j$". As is standard in this type of model, once there is a victory, the state of product $j$ resets -- in this case, $m$ is reset to $m = 0$. 

The resulting model exhibits a BGP with a constant growth rate $g$ in output, wages, etc; and constant distributions of products over states $\mu(m)$, constant $\gamma(m) = E[q / Q | m]$,\footnote{The joint distribution over $(q/Q_t,m)$ is not constant over time, but $E[q/Q_t | m] = \gamma(m)$ is. Only $\gamma(m)$ is necessary for computing aggregates.} and constant normalized R\&D wages by state of product $w(m) / Q$.\footnote{In other words, the R\&D wage for a product in state $m$ is $w(m) * Q$ for some constant function $w(m)$.} 

\subsection{Calibration and analysis}

\subsubsection{Calibration}

To analyze the implications of the model, I will attempt to calibrate its parameters using previous estimates from the literature, aggregate data moments, and causally estimated mechanisms in the micro data (more on this below).

\subsubsection{Analysis} 

Once I have a calibrated model, I can analyze it in a variety of ways. 
\begin{enumerate}
	\item 
\end{enumerate}


\section{Role for Venture Source data}

A key mechanism in my model is that R\&D leads to future competition. The strength of this mechanism is governed by two parameters, $\nu$ and $\xi$. The parameter $\nu$ is the rate at which an employee learns how to compete with the parent firm. The parameter $\xi$ is the size of an individual employee's spinout. 

The parameters $\nu$ and $\xi$ enter into the aggregates of the model as the product $\nu \xi$. Since I am interested in aggregate implications, this means I could in principle calibrate my model using only aggregate data and be fine.  

However, I am interested in using the micro data for several reasons:
\begin{enumerate}
	\item I can directly test the key mechanism in my model and get a truly causal interpretation.\footnote{I would no longer be looking at the data ``through the lens of a model'' which precludes the possibility that spinouts are not created by R\&D.}
	\item I can explore the extent to which my model is able to ``aggregate up'' well measurements from the micro data
	\item More generally, the micro data provides me with more moments against which to check my model's performance.
\end{enumerate}

\subsection{Description of data set}

The VentureSource contains information on VC-funded companies going back to 1986. 

\subsubsection{Features and coverage}

Venture Source contains:
\begin{itemize}
	\item Firm information: description, industry, products, revenue, ...
	\item Employee bios (for founders and some C-level): previous employment, education?
	\item Funding information: dates, valuations, VC firms
\end{itemize}

The coverage continues to be solid into the present day, for both founders and CTOs. I'm attaching a spreadsheet that describes the structure of the dataset, as well as one describing the coverage.

\subsubsection{Cost}

The cost structure is
\begin{enumerate}
	\item \$10,000: One-year academic license containing only basic firm information (name, location, industry) and founder bios.
	\item \$15,000: Above plus valuations at funding dates (useful for extending my analysis)
	\item \$17,000: Above plus product and revenue data (not much coverage)
	\item \$20,000: Full US archive
	\item \$30,000: Full global archive (coverage outside US is not as good)
	\item For multiple-year contracts, 5\% is deducted from the first year cost.
\end{enumerate}

For the project described in this document, I need (1) which  





\subsection{Proposed methodology}

With an estimate of $\xi$ based roughly on the expected size of a spinout, I can estimate $\nu$ from the micro data provided that: 

\begin{enumerate}
	\item I can measure R\&D spending by a given company
	\item I have an instrument for R\&D spending by each potential parent company
	\item I can measure the ``spawning rate'' of spinouts from a given company
\end{enumerate}

I can obtain (1), at least for public firms, using Compustat data (I could do something more indirect using the model + measures of patenting by non-public firms). I can obtain (2) using the instrument in Bloom et al. "Identifying Technology Spillovers...". Finally, I can do (3) using VentureSource data. I will describe (3) below. 

My proposed methodology has two non-trivial components, both of which have already been done.

\subsubsection{Constructing an instrument for R\&D spending}

This has been done by Bloom et al. in 2013. 

\subsubsection{Identifying spinouts}

This has been done by Gompers, Lerner \& Scharfstein in 2005.

There is an issue of lack of information on the employment start date for the cofounders, which is addressed by the same paper by Gompers et al., and I will attempt to handle this in the same way. 

\end{document}