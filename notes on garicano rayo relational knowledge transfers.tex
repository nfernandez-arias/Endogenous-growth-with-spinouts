\documentclass[12pt,english]{article}
\usepackage{lmodern}
\usepackage[T1]{fontenc}
\usepackage[latin9]{inputenc}
\usepackage{geometry}
\usepackage{amsthm}
\usepackage{verbatim}
\geometry{verbose,tmargin=1in,bmargin=1in,lmargin=1in,rmargin=1in}
\usepackage{setspace}
%\usepackage{esint}
\onehalfspacing
\usepackage{babel}
\usepackage{amsmath}

\theoremstyle{remark}
\newtheorem*{remark}{Remark}
\begin{document}

\title{Notes on Garicano \& Rayo AER 2017, "Relational knowledge transfers"}
\author{Nicolas Fernandez-Arias}
\maketitle

\begin{itemize}
	\item Non-competes can potentially be used to mitigate this friction. Not perfectly, but they make the worker's ability to use the knowledge outside the firm lower. Think about these details.
	\item What exactly is their social welfare function? Are they claiming that imposing a restriction on the employer improves the employer's welfare? I find that incredibly hard to believe...the employer has to pay a wage, hence is able to transfer LESS knowledge because the employee is paying the employer on net less..
	\item Think about how this mechanism not only slows down training, but also makes firms
	\begin{enumerate}
		\item Train employees in menial skills early on that won't be useful to them in the long run (to make them productive early on, so they can buy knowledge more quickly)
		\item Have certain roles at the firm done less efficiently. They don't care about the efficiency, because they simply appropriate all of the output. But for the economy, maybe it's less efficient?
	\end{enumerate} 
	\item More generally, go back to the idea of non-competes and inequality - non-competes are essentially a way of helping mitigate the financial friction preventing poor workers from learning on the job, e.g. working for startups in California with low wages 
\end{itemize}





\end{document}