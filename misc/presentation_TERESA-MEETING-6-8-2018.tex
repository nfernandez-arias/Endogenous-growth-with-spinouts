\documentclass[english,usenames,dvipsnames]{beamer}
\usetheme{default}
\usepackage[utf8]{inputenc}
\usepackage{caption}
\usepackage{appendixnumberbeamer}
\usepackage{babel}
\usepackage{amsmath}
\usepackage{geometry}
\usepackage{bbm}
\usepackage{amsthm}
\usepackage{verbatim}
\definecolor{myred1}{RGB}{255,50,0}
\definecolor{myblue1}{RGB}{0,100,255}
\definecolor{mygreen1}{RGB}{34,139,35}


\title{Employee Spinouts: }
\author{Nicolas Fernandez-Arias}
\date[June 8, 2018]{June 8, 2018}

\begin{document}
	
\maketitle	
		
\begin{frame}{Introduction}
\begin{itemize}
	\item Firms with knowledge often have to give this knowledge to employees in order to extract useful labor from them
	\item In particular, R\&D employees, who are working to push the frontier of the firm's knowledge, must be "brought up to speed" in order to be able to make improvements
	\item Employees who have learned may then choose to form firms which compete with their parents, e.g. 
	\begin{itemize}
		\item Compete directly by developing a better version of existing product / service
		\item Compete by pursuing some new application of the technology before the parent is able to 
	\end{itemize}
\end{itemize}
\end{frame}

\begin{frame}{My project}
\begin{itemize}
	\item Theory
	\begin{itemize}
		\item Standard endogenous growth model with quality ladders + creative destruction (Similar to Akcigit \& Kerr 2017 and others based on Grossman \& Helpman 1991), \textbf{plus entry by employee spinouts} 
		\item Spinouts formed by employees who have learned the frontier technology by working as an R\&D employee 
		\item No worker-entrepreneur choice - employees who have learned can effectively sell their idea to a competitive fringe of potential entrants. 
	\end{itemize}
	\item Empirics
	\begin{itemize}
		\item Key parameter: product $\nu \xi$, where $\nu$ is the rate of learning by employees and $\xi$ is the maximum size of an individual spinout
		\item Given the rest of the model, this is identified by, say, the fraction of firms which are spinouts 
		\item However, there are many other predictions of the model 
	\end{itemize}
\end{itemize}
\end{frame}




	
	
	
	
	
	
	
	
\end{document}