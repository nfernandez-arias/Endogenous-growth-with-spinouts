\documentclass[12pt,english]{article}
%\usepackage{lmodern}
\linespread{1.05}
\usepackage{mathpazo}
%\usepackage{mathptmx}
%\usepackage{utopia}
\usepackage{microtype}
\usepackage[T1]{fontenc}
\usepackage[latin9]{inputenc}
\usepackage[dvipsnames]{xcolor}
\usepackage{geometry}
\usepackage{amsthm}
\usepackage{amsfonts}

\usepackage{courier}
\usepackage{verbatim}
\usepackage[round]{natbib}
\bibliographystyle{plainnat}


\definecolor{red1}{RGB}{128,0,0}
%\geometry{verbose,tmargin=1.25in,bmargin=1.25in,lmargin=1.25in,rmargin=1.25in}
\geometry{verbose,tmargin=1in,bmargin=1in,lmargin=1in,rmargin=1in}
\usepackage{setspace}

\usepackage[colorlinks=true, linkcolor={red!70!black}, citecolor={blue!50!black}, urlcolor={blue!80!black}]{hyperref}
%\usepackage{esint}
\onehalfspacing
\usepackage{babel}
\usepackage{amsmath}
\usepackage{graphicx}

\theoremstyle{remark}
\newtheorem{remark}{Remark}
\begin{document}
	
	
	
\title{Endogenous Growth with Spinouts: Bridging Model and Data}
\author{Nicolas Fernandez-Arias}
\maketitle

\section{Taking the model seriously}

Esteban's suggestion is to run the same regression in model and data. This amounts to taking the model seriously, i.e. assuming that the relevant heterogeneity is included in the model so that the regressions are not seriously misleading. The advantage is that I do not need to worry so much about getting the "right" specification: as long as I run the same specification in model and data, I have a moment which can be used to discipline the model.

\section{Multi-product firms}

Taken most literally, the single-product firms in the model correspond to \textbf{products} in the data. A firm with varying R\&D effort (and patents, etc.) in the data can correspond to a single firm in the model with varying R\&D effort (i.e one product with varying R\&D effort over time) \textbf{or} to a multi-firm "firm" that has varying R\&D due to e.g. varying numbers of products in operation. In the model, these two have the same implications for spinout formation, so there should be no issue running the regressions. 

\section{Shocks to R\&D spending}

Of course it would be nice to have this. Need to think about feasibility of this. Don't have that much spare power.

\section{Stages of spinout formation}

The model has implications about the effect of spinouts at different stages on firm incentives for R\&D. The model also has implications for how the rate of the formation of the different kinds of spinouts depends on R\&D 


\section{Variation in non-compete enforcement}

Esteban's point:
\begin{enumerate}
	\item Higher non-compete enforcement should reduce the rate at which R\&D translates into spinout formation, particularly within-industry spinouts
	\item Higher non-compete enforcement should *not* affect the extent to which increased competition in the R\&D race incentivizes parent R\&D spending
	\item Hence this is a stark prediction of my model and can be tested in the data
\end{enumerate}






\end{document}