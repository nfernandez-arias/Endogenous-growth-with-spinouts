




\subsubsection{Numerical simulations}

Because it is difficult to make broad claims about welfare, below I exhibit different parameter settings that lead to different socially optimal $\kappa_c$. 

\subsubsection{title}

\subsubsection{Extension with heterogeneity in $\kappa_e$}

It is possible to extend the model to a case where the cost of entry into good $j$ is $\kappa_{e,jt}$. Assume that every time a good $j$ is successfully improved (by an incumbent, entrant or spinout) a new $\kappa_{ejt}$ is drawn from an independent uniform distribution on $[0,1]$. Along the BGP, for each value of $\kappa_c$ there exists a threshold $\bar{\kappa}_{e}(\kappa_c) \in [0,1]$ such that for all $j$ with $\kappa_{e,jt} > \bar{\kappa}_{e} (\kappa_c) $ the incumbent uses a non-compete. To be consistent with the baseline model's notation, define $\bar{\kappa}_c = \inf \{\kappa_c \ge 0: \bar{\kappa}_e(\kappa_c) = 1 \}$.  

Define $\tilde{V}$ as the value of incumbency prior to the realization of $\kappa_{e,jt}$. Define $\hat{V}(\kappa_e)$ as the ex-post value. These quantities are related by 
\begin{align}
	\tilde{V} &\equiv \mathbb{E}_{\kappa_e} [V(\kappa_e)] = \int_0^1 V(\kappa_e) d\kappa_e
\end{align}

As before, the cost of enforcing noncompetes is given by $\kappa_c \nu \tilde{V}$. The incumbent HJB becomes
\begin{align}
	(r + \tau_E) \hat{V}(\kappa_e) = \tilde{\pi} + \max_{\substack{x \in \{0,1\} \\ z \ge 0}} \Bigg\{ z \Big( &\chi_I(\lambda - 1) \tilde{V} - w_{RD,E} - x \nu \kappa_c \tilde{V} \nonumber \\
	&- (1-x) \nu \big( V(\kappa_e) - (1-\kappa_e) \lambda \tilde{V} \big)   \Big)  \Bigg\} \label{ext:hjb_incumbent}
\end{align}

Suppose that $\kappa_c < \bar{\kappa}_c$, so that a positive measure set of incumbents use non-competes, i.e. those with $\bar{\kappa}_e(\kappa_c) < \kappa_e < 1$. Further suppose we are in an equilibrium where some of these incumbents incur positive R\&D expenditures. Along this range we know that $\hat{V}(\kappa_e) = \hat{\underline{V}}$ is constant, since non-competes are used and the value of $\kappa_e$ is not relevant for the value of the incumbent conditional on innovation (by any agent) due to the fact that draws of $\kappa_e$ are independent over time. Therefore, if any incur expenditures, all do. Also, 
\begin{align}
	\hat{\underline{V}} &= \frac{\chi_I(\lambda-1) \tilde{V} - w_{RD}}{\kappa_c \nu }	
\end{align}

Now consider $\kappa_e \le \bar{\kappa}_e(\kappa_c)$. Holding constant the value $V(\kappa_e)$, it is apparent from (\ref{ext:hjb_incumbent}) that the effective wage paid for R\&D labor is lower for lower values of $\kappa_e$. This is due to the fact that the worker perceives a higher expected value of future spinout formation. The expected benefit to the incumbent from R\&D, on the other hand, does not depend on $\kappa_e$. In order for the FOC to be satisfied on a positive measure set $K \subseteq \{ \kappa_e: \kappa_e < \bar{\kappa}_e \}$, the expected wage must be the same for all such $\kappa_e$. This requires that
\begin{align}
	V(\kappa_e) = \Theta + (1-\kappa_e)\lambda \tilde{V} 
\end{align}

for some constant $\Theta$. However, if indeed it is the case that 

\paragraph{Household problem}

\begin{maxi*}[1]<b>
	{\substack{\{C(t)\}_{t \ge 0} \\ \{\ell_{RD,j}(t)\}_{j \in [0,1]} \\ \{\hat{\ell}_{RD,j}(t)\}_{j \in [0,1]} \\ L_I(t),L_F(t)\ge 0}} {\mathbb{E} \int_0^{\infty} \frac{C(t)^{1-\theta}-1}{1-\theta} dt}{}{}
	\addConstraint{ C(t)}{ \ge 0} {\text{Non-negativity of consumption}}
	\addConstraint{ A'(t)}{ = \overbrace{r_t A(t)}^{\mathclap{\text{Returns from equity holdings}}} + \underbrace{\bar{w}_t(L_I(t) + L_F(t))}_{\mathclap{\text{Production labor earnings}}} - \overbrace{C(t)}^{\mathclap{\text{Consumption}}}} {\text{Asset law of motion}}
	\addConstraint{ }{+ \overbrace{\int_0^1 \big( \underbrace{w_{RD,jt}}_{\mathclap{\text{Wage}}} + \underbrace{(\frac{\bar{q}_{jt}}{Q_t})^{-1} \nu (1-\kappa_e) V(j,t|\lambda \bar{q}_{jt})}_{\mathclap{\text{WSO formation}}} \big) \underbrace{\ell_{RD,j}(t)}_{\mathclap{\text{Labor supply}}} dj}^{\mathclap{\text{Effective R\&D labor earnings incumbent j}}}}
	\addConstraint{ }{+ \int_0^1 \hat{w}_{RD,t} \hat{\ell}_{RD,j}(t) dj}
	\addConstraint{\lim_{t \to \infty} e^{-\int_0^{\infty} r_t dt} A(t) }{\ge 0}{\text{No-ponzi condition}}
	\addConstraint{A(0)}{ = A_0} {\text{Initial asset holdings}}
	\addConstraint{\int_0^1 (\ell_{RD,j}(t) + \hat{\ell}_{RD,j}(t))dj}{ \le \bar{L}_{RD}} {\text{R\&D labor endowment}}
	\addConstraint{L_I(t) + L_F(t)}{\le 1 - \bar{L}_{RD}} {\text{Production labor endowment}}
\end{maxi*}


%%% Stuff about R&D misallocation

\paragraph{Effect on steady-state consumption}

The allocation of R\&D labor also has an effect on steady-state consumption. Restating (\ref{eq:agg_consumption_decomposition}), steady-state consumption is given by
\begin{align} 
\tilde{C} = \tilde{Y} - \big(\hat{\tau} + (1-x) z \nu\big) \kappa_e \lambda \tilde{V}  - x z \nu \bar{\kappa}_c \tilde{V}    
\end{align} 

Consider how the above expression varies in $\hat{z}$. First, a higher $\hat{z}$ implies a higher entrant innovation rate $\hat{\tau}$ and thereby a higher flow cost of creative destruction $(\hat{\tau} + \tau^S)\kappa_e \lambda \tilde{V}$. Furthermore, the increase in $\hat{z}$ reduces $z$ by the R\&D labor market clearing condition (\ref{eq:RD_labor_market_clearing}). If NCAs are used ($x = 1$) then this implies a reduction in the flow cost of NCA enforcement, $x z \kappa_c \nu \tilde{V}$. If NCAs are not used ($x = 0$) then lower $z$ reduces the entry cost paid by WSOs, $(1-x) \nu  z \kappa_e \lambda \tilde{V}$. Finally, (\ref{eq:hjb_incumbent_gordon_formula}) implies that the increase in $\hat{\tau}$ can reduce $\tilde{V}$, although it may be partly or wholly offset by the reduction in the interest rate $r$ stemming from lower growth $g$, via the Euler equation (\ref{eq:interest_rate}). Overall, $\tilde{V}$ may increase or decrease. \textbf{[Put an expression adding up all of these effects showing that the sign is not determined]} 

To make things more clear, consider a model where $\hat{z}$ is set exogenously, i.e. the free entry condition (\ref{eq:free_entry_condition}) does not necessarily hold. The propositions on existence and uniqueness of symmetric BGPS in the previous chapter go through with the natural modifications. When $\psi = 0.5$ as will be the case in the calibration of Section \ref{sec:calibration}, the change of variables $\zeta = \hat{z}^{1/2}$ yields
\begin{align}
\tilde{C} = \tilde{Y} - \Big(\big(\hat{\chi} \zeta \kappa_e \lambda + (1-x) (\bar{L}_{RD} - \zeta^2) \nu\big) \kappa_e \lambda   - x (\bar{L}_{RD} - \zeta^2) \nu \bar{\kappa}_c \Big)\tilde{V} 
\end{align}

where I used the R\&D labor market clearing condition (\ref{eq:RD_labor_market_clearing}) and the definition of $\hat{\tau}$. Differentiating this expression with respect to $\zeta$ yields
\begin{align}
\frac{d\tilde{C}}{d\zeta} &= - \Lambda \frac{d\tilde{V}}{d\zeta}  + \Big(  \hat{\chi} \kappa_e \lambda \big( 1 - 2\frac{1-\kappa_e}{\chi(\lambda -1) - x \kappa_c \nu - (1-x) (1 - (1-\kappa_e) \lambda) \nu }\big) \Big) \tilde{V} \\
\Lambda &= \hat{\chi} \hat{z}^{1/2} \kappa_e \lambda  + x (\bar{L}_{RD} - \hat{z}) \nu \kappa_c + (1-x)  (\bar{L}_{RD} - \hat{z}) \nu \kappa_e \lambda 
\end{align}

where I substituted for $\zeta$ in the first line using (\ref{eq:effort_entrant}) and in the second line using $\zeta = \hat{z}^{1/2}$. 

By (\ref{eq:hjb_incumbent_gordon_formula}),
\begin{align}
\tilde{V} &= \frac{\tilde{\pi}}{\underbrace{\theta (\lambda -1) (\chi (\bar{L}_{RD} - \hat{z}) + \hat{\chi} \hat{z}^{1/2} + (1-x) \nu (\bar{L}_{RD} - \hat{z})) + \rho}_{\mathclap{= r\text{ by (\ref{eq:growth_accounting}), (\ref{eq:interest_rate}) }} }  + \hat{\chi} \hat{z}^{1/2}} 
\end{align}

The value of incumbency is increasing in the growth rate, via the interest rate, and decreasing in the rate of creative destruction by entrants $\hat{\chi} \hat{z}^{1/2}$. These two mechanisms work opposite each other so that in the calibration $\frac{d\tilde{V}}{d\hat{z}}$, and hence $\frac{d\tilde{V}}{d\zeta}$, is relatively small in magntiude. Whether steady-state consumption $\tilde{C}$ increases or decreases from an overallocation to $\hat{z}$ is therefore largely determined by the sign of 
\begin{align}
\frac{d\Lambda}{d\zeta} &= \hat{\chi} \kappa_e \lambda \Big( 1 - 2\frac{1-\kappa_e}{\chi(\lambda -1) - x \kappa_c \nu - (1-x) (1 - (1-\kappa_e) \lambda) \nu }\Big)
\end{align} 

Heuristically, if $\frac{d\Lambda}{d\zeta} < 0$ and $\frac{d\tilde{V}}{d\zeta}$ is small in magnitude, then $\tilde{C}$ decreases when R\&D labor is exogenously reallocated from OI by incumbents to CD by entrants. In the calibration of Section \ref{sec:calibration}, this is the case.

\subsubsection{Effect on welfare}

Because the sign of the change in consumption can be positive or negative depending on parameters, in the decentralized equilibrium it is not possible to say in general whether R\&D is over or underallocated to CD by entrants or OI by incumbents. Numerically, in the calibration of Section \ref{sec:calibration}, there is overallocation of R\&D to CD by entrants. 



%%% Stuff about NCA misallocation


\subsubsection{Effect on steady-state consumption}

Whether the increase in growth in the last paragraph actually increases welfare depends on its effect on steady-state consumption. The change in steady-state consumption from $x = 1$ to $x = 0$ is the net of the reduction in NCA enforcement costs and the reduction in entry costs,
\begin{align}
\Delta \tilde{C} &= z_{x = 1} \nu \bar{\kappa}_c  \tilde{V}_{x = 1} - z_{x = 0} \nu \kappa_e \lambda \tilde{V}_{x = 0} + \hat{\tau}_{x = 1} \kappa_e \lambda \tilde{V}_{x=1} - \hat{\tau}_{x = 0} \kappa_e \lambda \tilde{V}_{x =0}
\end{align}

where the subscript $x = i$ for $i \in \{0,1\}$ denotes the BGP the variable is drawn from. The first two terms are the reductions in NCA costs (positive) and entry costs (negative) associated with WSOs. The last two terms constitute the reduction in entry costs associated entrants. As argued in the last paragraph, $z_{x = 0} = z_{x=1}$ and $\hat{\tau}_{x = 0} = \hat{\tau}_{x = 1}$. By contrast, $\tilde{V}_{x=0} < \tilde{V}_{x=1}$ due to the increased interest rate induced by the higher BGP growth rate (see equation (\ref{eq:interest_rate})). Using the definition of $\bar{\kappa}_c = 1 - (1-\kappa_e) \lambda$ and rearranging yields
\begin{align}
\Delta \tilde{C} &= \tilde{V}_{x = 1} \Bigg( z \nu \Big( 1 - (1-\kappa_e)\lambda - \frac{\tilde{V}_{x = 0}}{\tilde{V}_{x = 1}}\kappa_e \lambda \Big) + \hat{\tau} \kappa_e \lambda \Big( 1 - \frac{\tilde{V}_{x = 0}}{\tilde{V}_{x = 1}}\Big) \Bigg) \label{eq:misallocation_NCA_consumption}
\end{align}

The second term in parentheses is always positive. The first term in parentheses is positive if and only if\footnote{The ratio $\frac{V_{x = 0}}{V_{x = 1}}$ is simply equal to the ratio of the discount factors used in either equilibrium, $\frac{V_{x = 0}}{V_{x = 1}} = \frac{r_{x = 1} + \hat{\tau}}{r_{x = 0} + \hat{\tau}}$ which can be expressed as a closed form,
	\begin{align*}
	\frac{V_{x = 0}}{V_{x = 1}} &= \frac{\theta(\lambda - 1) (\tau+ \hat{\tau}) + \rho + \hat{\tau})}{\theta(\lambda - 1) (\tau+ \hat{\tau} + z\nu )  + \rho + \hat{\tau})}
	\end{align*} 
	What is left is to substitute the equilibrium expressions for $z$, $\tau$, and $\hat{\tau}$}
\begin{align}
\frac{\lambda -1}{\lambda} < \big( 1 - \frac{V_{x = 0}}{V_{x = 1}} \big) \kappa_e  \label{cs:consumption_decreasing_condition}
\end{align}

The first term dominates when $z$ is large compared to $\hat{z}$. Everything is expressible in closed form the but the expressions are unwieldy enough to not provide further insight. In the calibration of Section \ref{sec:calibration}, (\ref{cs:consumption_decreasing_condition}) is violated as $1 - \frac{V_{x = 0}}{V_{x = 1}}$ is on the order of $0.01$ so the RHS of is close to zero, while the LHS is always strictly positive as $\lambda > 1$ so the first term is negative. Furthermore, incumbents do about twice as much R\&D as entrants ($z / \hat{z} \approx 2 $) so the first term in (\ref{eq:misallocation_NCA_consumption}) dominates and $\Delta \tilde{C} < 0$.

\subsubsection{Effect on welfare}

Given that $\Delta \tilde{C}$ may be positive or negative, the effect on welfare from the misallocation in NCA use can in principle be positive or negative. Intuitively, however, the increase in growth dominates for two reasons. First, the reduction in $\tilde{C}$ is the net of two forces of similar magnitude, so it is small. Second, the increase in growth has positive externalities because it improves the effective productivity of the rest of the economy, both in static terms (directly by increasing $\tilde{C}$ and indirectly by improving intermediate goods production technology (\ref{intermediate_goods_production})) and in dynamic terms (entrants work to improve a higher quality technology). In the calibration of Section \ref{sec:calibration}, $\Delta \tilde{C} < 0$ is sufficiently small in magnitude that, for $\varepsilon > 0$ small $\Delta \tilde{W} > 0$ upon a switch from $x = 1$ to $x = 0$. 




%%% Theoretical policy exercises
%%%

%%% cost of ncas
\paragraph{Consumption}\label{cs:consumption1}

Consider again $\kappa_c \in [0, \bar{\kappa}_c)$. As $\tau^S = 0$ and using R\&D labor market clearing (\ref{eq:RD_labor_market_clearing}), consumption is given by 
\begin{align}
\tilde{C} &= \tilde{Y} - \Big( \hat{\chi} (\bar{L}_{RD} - z)^{1-\psi} \kappa_e \lambda + z \nu \kappa_c \Big) \tilde{V} \label{cs:consumption_eq}
\end{align}

As discussed in the preceding section, an increase in $\kappa_c$ reduces $z$ and increases $\hat{z}$. The logic in Section \ref{model:efficiency:misallocationRD} means that, on its own, this may increase or decrease $\tilde{C}$. In addition, in this case $\kappa_c$ is increasing, which adds a term $- z\nu \tilde{V}$ to the derivative of $\tilde{C}$ in $z$. 

As $\kappa_c$ crosses the $\bar{\kappa}_c$ threshold, incumbents no longer use NCAs ($x = 0$). The logic in Section \ref{model:efficiency:misallocationNCAs} applies in this case without modification, as the change in $\kappa_c$ while crossing the threshold is infinitesimal. Therefore the change in steady-state consumption $\tilde{C}$ can be positive or negative, depending on parameters. 

\paragraph{Welfare}

The previous two sections show that an increase in $\kappa_c$ may increase or decrease both growth and welfare, depending on parameters. I postpone discussion of welfare until the parameters have been disciplined by the calibration in Section \ref{sec:calibration}.



%%% NCA cost

at $\kappa_{c0}$ and consider a counterfactual in which the planner has set the cost of NCAs to $\kappa_{c1} > \kappa_{c0}$ instead. 



Next, suppose $\mathbbm{1}^{NCA} = 0$ and $z > 0$ at $\kappa_{c0} < \bar{\kappa}_c$ and consider an increase to $\kappa_{c1}$. If $\kappa_{c1} < \bar{\kappa}_c$ then $\mathbbm{1}^{NCA} = 0$ and there is no change in the equilibrium because the NCA cost is not paid in either case. If $\kappa_{c1}$ is large enough then $x = 1$. Upon crossing the $\bar{\kappa}_c$ threshold, $\tau^S$ jumps from $0$ to $\nu z$ while $z,\hat{z}$ do not jump. As discussed in Section \ref{model:efficiency:misallocationNCAs}, this unequivocally increases grwoth. Specifically, by the growth accounting equation (\ref{eq:growth_accounting}), the growth rate jumps to $g_1 > g_0$. By the Euler equation (\ref{eq:euler}), the interest rate jumps from $r$ to $r_1>r_0$. Equation (\ref{eq:hjb_incumbent_gordon_formula}) then implies that $\tilde{V}$ declines. If $z > 0$ with $\kappa_{c1}$, the incumbent FOC (\ref{eq:hjb_incumbent_foc}) then implies that the R\&D wage declines to $w_{RD1} < w_{RD0}$. If $z = 0$ with $\kappa_{c1}$, then the above changes occur until the threshold $\kappa_{c}'$ such that $z = 0$, beyond which further increases in $\kappa_c$ have no effect on the equilibrium. 

Intuitively, the increase in $\tau^S$ at the jump from $\mathbbm{1}^{NCA} = 1$ to $\mathbbm{1}^{NCA} = 0$ has no direct effect on innovation incentives because at the threshold $\kappa_c = \bar{\kappa}_c$, incumbents' effective cost of R\&D is the same whether $\mathbbm{1}^{NCA} = 1$ or $\mathbbm{1}^{NCA} = 0$. In partial equilibrium, this means that the growth rate increases which, in general equilibrium, increases the interest rate. This reduces the value of incumbency, but as this affects incumbents' and entrants' payoffs to R\&D symmetrically and total R\&D spending is fixed, it has no effect on the allocation of R\&D between incumbents and entrants and therefore growth. Hence, growth increases in general equilibrium as well. The reduced value of incumbency is simply passed through to lower R\&D wages.



%%%  CD tax

The incumbent HJB is can be rearranged to
\begin{align}
(r + \hat{\tau}) \tilde{V} = \tilde{\pi} + \max_{\substack{x \in \{0,1\} \\ z \ge 0}} \Big\{z &\Big( \overbrace{\chi (\lambda - 1) \tilde{V}}^{\mathclap{\mathbb{E}[\textrm{Benefit from R\&D}]}}- \hat{w}_{RD} \\
&-  \underbrace{(1-x)(1 - (1-(1+T_e)\kappa_{e})\lambda)\nu \tilde{V}}_{\mathclap{\text{Net cost from spinout formation}}} - \overbrace{x \kappa_{c} \nu \tilde{V}}^{\mathclap{\text{Direct cost of NCA}}}\Big) \Big\} \label{eq:hjb_incumbent_entryTax_2}
\end{align}

By the usual argument, $z > 0$ implies that the incumbent's FOC can be rearranged to
\begin{align}
\tilde{V} &= \frac{\hat{w}_{RD}}{\chi(\lambda -1) - \nu (x\kappa_c + (1-x)(1 - (1-(1+T_e)\kappa_e)\lambda)) } \label{eq:hjb_incumbent_foc_entryTax}
\end{align}




%%%% Regressions



\begin{table}[!htb]
	\scriptsize
	\centering
	{
\def\sym#1{\ifmmode^{#1}\else\(^{#1}\)\fi}
\begin{tabular}{l*{6}{c}}
\toprule
                    &\multicolumn{1}{c}{(1)}&\multicolumn{1}{c}{(2)}&\multicolumn{1}{c}{(3)}&\multicolumn{1}{c}{(4)}&\multicolumn{1}{c}{(5)}&\multicolumn{1}{c}{(6)}\\
                    &\multicolumn{1}{c}{Founders}&\multicolumn{1}{c}{Founders}&\multicolumn{1}{c}{Founders}&\multicolumn{1}{c}{WSO4}&\multicolumn{1}{c}{WSO4}&\multicolumn{1}{c}{WSO4}\\
\midrule
R\&D                &        0.43\sym{***}&        0.44\sym{***}&        0.44\sym{***}&        0.27\sym{***}&        0.26\sym{***}&        0.26\sym{***}\\
                    &      (0.13)         &      (0.13)         &      (0.14)         &     (0.051)         &     (0.053)         &     (0.040)         \\
\addlinespace
hNCA=0 $\times$ R\&D&           0         &           0         &           0         &           0         &           0         &           0         \\
                    &         (.)         &         (.)         &         (.)         &         (.)         &         (.)         &         (.)         \\
\addlinespace
hNCA=1 $\times$ R\&D&        0.56         &        0.56         &        0.56         &       0.080         &       0.078         &       0.078         \\
                    &      (0.44)         &      (0.41)         &      (0.40)         &      (0.13)         &      (0.13)         &      (0.12)         \\
\addlinespace
Firm FE             &         Yes         &         Yes         &         Yes         &         Yes         &         Yes         &         Yes         \\
\addlinespace
Year FE             &         Yes         &          No         &          No         &         Yes         &          No         &          No         \\
\addlinespace
Age FE              &          No         &         Yes         &         Yes         &          No         &         Yes         &         Yes         \\
\addlinespace
Industry-Year FE    &          No         &         Yes         &         Yes         &          No         &         Yes         &         Yes         \\
\addlinespace
State-Year FE       &          No         &         Yes         &         Yes         &          No         &         Yes         &         Yes         \\
\midrule
Clustering          &       gvkey         &       gvkey         &naics4 Statecode         &       gvkey         &       gvkey         &naics4 Statecode         \\
R-squared (adj.)    &        0.67         &        0.68         &        0.68         &        0.66         &        0.64         &        0.64         \\
R-squared (within, adj)&        0.29         &        0.30         &        0.30         &        0.26         &        0.25         &        0.25         \\
Observations        &       59485         &       57956         &       57956         &       59485         &       57956         &       57956         \\
\bottomrule
\multicolumn{7}{l}{\footnotesize Standard errors in parentheses}\\
\multicolumn{7}{l}{\footnotesize \sym{*} \(p<0.1\), \sym{**} \(p<0.05\), \sym{***} \(p<0.01\)}\\
\end{tabular}
}

	\caption{The regressions above relate corporate R\&D, and its interaction with 1-digit NAICS industry, to the  entrepreneurship decisions of employees. The dependent variable is average yearly number of founders joining startups in years $t+1,t+2,t+3$. The independent variables are averages over $t,t-1,t-2$. Firm controls are employment, assets, intangible assets, investment, net income, cumulative citation-weighted patents, and the product of Tobin's Q and Assets (i.e., firm market value). Standard errors are clustered by firm in columns (1)-(3) and (5)-(7). In columns (4) and (8), standard errors are multway clustered by State and 4-digit NAICS industry.}
	\label{table:RDandSpinoutFormation_absolute_founder2_hNCA_l3f3}
\end{table}


\begin{table}[!htb]
	\scriptsize
	\centering
	{
\def\sym#1{\ifmmode^{#1}\else\(^{#1}\)\fi}
\begin{tabular}{l*{6}{c}}
\toprule
                    &\multicolumn{1}{c}{(1)}&\multicolumn{1}{c}{(2)}&\multicolumn{1}{c}{(3)}&\multicolumn{1}{c}{(4)}&\multicolumn{1}{c}{(5)}&\multicolumn{1}{c}{(6)}\\
                    &\multicolumn{1}{c}{Founders}&\multicolumn{1}{c}{Founders}&\multicolumn{1}{c}{Founders}&\multicolumn{1}{c}{WSO4}&\multicolumn{1}{c}{WSO4}&\multicolumn{1}{c}{WSO4}\\
\midrule
naics1=1 $\times$ R\&D&        1.10\sym{***}&                     &                     &        0.17\sym{**} &                     &                     \\
                    &      (0.39)         &                     &                     &     (0.073)         &                     &                     \\
\addlinespace
naics1=2 $\times$ R\&D&      -0.061         &        0.12         &        0.12         &        0.11         &        0.13         &        0.13         \\
                    &      (0.16)         &      (0.22)         &      (0.17)         &     (0.076)         &         (.)         &     (0.098)         \\
\addlinespace
naics1=3 $\times$ R\&D&        0.44\sym{***}&        0.43\sym{***}&        0.43\sym{***}&        0.23\sym{***}&        0.22         &        0.22\sym{***}\\
                    &     (0.084)         &     (0.089)         &      (0.10)         &     (0.049)         &         (.)         &     (0.053)         \\
\addlinespace
naics1=4 $\times$ R\&D&        2.55\sym{***}&        1.61\sym{***}&        1.61\sym{***}&      0.0100\sym{*}  &       0.035         &       0.035\sym{**} \\
                    &      (0.38)         &      (0.59)         &      (0.54)         &    (0.0058)         &         (.)         &     (0.015)         \\
\addlinespace
naics1=5 $\times$ R\&D&        1.68\sym{***}&        1.80\sym{***}&        1.80\sym{***}&        0.69\sym{***}&        0.70         &        0.70\sym{***}\\
                    &      (0.25)         &      (0.21)         &      (0.17)         &     (0.064)         &         (.)         &     (0.056)         \\
\addlinespace
naics1=6 $\times$ R\&D&        2.42         &        6.12\sym{***}&        6.12\sym{***}&      -0.058         &        0.29         &        0.29         \\
                    &      (1.91)         &      (2.12)         &      (2.05)         &     (0.052)         &         (.)         &      (0.23)         \\
\addlinespace
naics1=7 $\times$ R\&D&        0.84         &        0.48         &        0.48         &        0.41\sym{*}  &       -0.33         &       -0.33         \\
                    &      (1.30)         &      (0.81)         &      (0.97)         &      (0.22)         &         (.)         &      (0.32)         \\
\addlinespace
naics1=8 $\times$ R\&D&        2.65\sym{***}&                     &                     &        0.35         &                     &                     \\
                    &      (1.00)         &                     &                     &      (0.25)         &                     &                     \\
\addlinespace
naics1=9 $\times$ R\&D&       -2.28\sym{***}&        0.51         &        0.51\sym{*}  &     -0.0013         &     6.9e-17         &     6.9e-17         \\
                    &      (0.77)         &      (0.52)         &      (0.27)         &    (0.0091)         &         (.)         &   (6.9e-09)         \\
\addlinespace
Firm FE             &         Yes         &         Yes         &         Yes         &         Yes         &         Yes         &         Yes         \\
\addlinespace
Year FE             &         Yes         &          No         &          No         &         Yes         &          No         &          No         \\
\addlinespace
NAICS1-Age FE       &          No         &         Yes         &         Yes         &          No         &         Yes         &         Yes         \\
\addlinespace
Industry-Year FE    &          No         &         Yes         &         Yes         &          No         &         Yes         &         Yes         \\
\addlinespace
NAICS1-State-Year FE&          No         &         Yes         &         Yes         &          No         &         Yes         &         Yes         \\
\midrule
Clustering          &       gvkey         &       gvkey         &naics4 Statecode         &       gvkey         &       gvkey         &naics4 Statecode         \\
R-squared (adj.)    &        0.74         &        0.73         &        0.73         &        0.69         &        0.66         &        0.65         \\
R-squared (within, adj)&        0.43         &        0.38         &        0.38         &        0.34         &        0.28         &        0.28         \\
Observations        &       59485         &       56448         &       56448         &       59485         &       56448         &       56448         \\
\bottomrule
\multicolumn{7}{l}{\footnotesize Standard errors in parentheses}\\
\multicolumn{7}{l}{\footnotesize \sym{*} \(p<0.1\), \sym{**} \(p<0.05\), \sym{***} \(p<0.01\)}\\
\end{tabular}
}

	\caption{The regressions above relate corporate R\&D, and its interaction with 1-digit NAICS industry as well as an indicator for NCA enforcement, to the  entrepreneurship decisions of employees. The dependent variable is average yearly number of founders joining startups in years $t+1,t+2,t+3$. The independent variables are averages over $t,t-1,t-2$. Firm controls are employment, assets, intangible assets, investment, net income, cumulative citation-weighted patents, and the product of Tobin's Q and Assets (i.e., firm market value). Standard errors are clustered by firm in columns (1)-(3) and (5)-(7). In columns (4) and (8), standard errors are multway clustered by State and 4-digit NAICS industry.}
	\label{table:RDandSpinoutFormation_absolute_founder2_naics1_l3f3}
\end{table}

\begin{table}[!htb]
	\scriptsize
	\centering
	{
\def\sym#1{\ifmmode^{#1}\else\(^{#1}\)\fi}
\begin{tabular}{l*{6}{c}}
\toprule
                    &\multicolumn{1}{c}{(1)}&\multicolumn{1}{c}{(2)}&\multicolumn{1}{c}{(3)}&\multicolumn{1}{c}{(4)}&\multicolumn{1}{c}{(5)}&\multicolumn{1}{c}{(6)}\\
                    &\multicolumn{1}{c}{Founders}&\multicolumn{1}{c}{Founders}&\multicolumn{1}{c}{Founders}&\multicolumn{1}{c}{WSO4}&\multicolumn{1}{c}{WSO4}&\multicolumn{1}{c}{WSO4}\\
\midrule
hNCA\_xrd\_l3\_ind\_1dig1&       -1.27         &           0         &           0         &       -0.88         &           0         &           0         \\
                    &      (3.61)         &         (.)         &         (.)         &      (0.56)         &         (.)         &         (.)         \\
\addlinespace
hNCA\_xrd\_l3\_ind\_1dig2&        2.91         &        3.48         &        3.48         &       -0.16         &        0.44         &        0.44         \\
                    &      (3.64)         &      (2.58)         &      (4.05)         &      (0.28)         &         (.)         &      (0.34)         \\
\addlinespace
hNCA\_xrd\_l3\_ind\_1dig3&      -0.047         &      -0.027         &      -0.027         &      -0.052         &      -0.055         &      -0.055         \\
                    &      (0.14)         &      (0.15)         &      (0.14)         &     (0.086)         &         (.)         &     (0.073)         \\
\addlinespace
hNCA\_xrd\_l3\_ind\_1dig4&       0.092         &        0.59         &        0.59         &       0.065         &       -0.14         &       -0.14         \\
                    &      (1.74)         &      (1.73)         &      (1.23)         &     (0.071)         &         (.)         &      (0.15)         \\
\addlinespace
hNCA\_xrd\_l3\_ind\_1dig5&       0.086         &        0.14         &        0.14         &       -0.39\sym{***}&       -0.43         &       -0.43\sym{**} \\
                    &      (0.39)         &      (0.43)         &      (0.52)         &      (0.13)         &         (.)         &      (0.17)         \\
\addlinespace
hNCA\_xrd\_l3\_ind\_1dig6&        2.94         &        2.12         &        2.12         &      -0.022         &       -0.34         &       -0.34         \\
                    &      (4.55)         &      (3.88)         &      (2.31)         &     (0.099)         &         (.)         &      (0.32)         \\
\addlinespace
hNCA\_xrd\_l3\_ind\_1dig7&        6.14         &        25.2\sym{*}  &        25.2\sym{**} &        0.60         &       -1.68         &       -1.68         \\
                    &      (5.95)         &      (13.6)         &      (12.0)         &      (0.53)         &         (.)         &      (1.94)         \\
\addlinespace
hNCA\_xrd\_l3\_ind\_1dig8&       -2.01\sym{*}  &           0         &           0         &       -0.48\sym{**} &           0         &           0         \\
                    &      (1.05)         &         (.)         &(0.00000084)         &      (0.20)         &         (.)         &(0.00000036)         \\
\addlinespace
hNCA\_xrd\_l3\_ind\_1dig9&       -2.71\sym{***}&       -1.80         &       -1.80\sym{***}&       0.025         &    -1.4e-15         &    -1.4e-15         \\
                    &      (0.99)         &      (1.43)         &      (0.61)         &     (0.036)         &         (.)         &   (3.5e-09)         \\
\addlinespace
naics1=1 $\times$ R\&D&        2.27         &                     &                     &        0.97\sym{*}  &                     &                     \\
                    &      (3.34)         &                     &                     &      (0.55)         &                     &                     \\
\addlinespace
naics1=2 $\times$ R\&D&      -0.064         &        0.12         &        0.12         &        0.11         &        0.13         &        0.13         \\
                    &      (0.16)         &      (0.22)         &      (0.19)         &     (0.076)         &         (.)         &     (0.096)         \\
\addlinespace
naics1=3 $\times$ R\&D&        0.46\sym{***}&        0.44\sym{***}&        0.44\sym{***}&        0.25\sym{***}&        0.25         &        0.25\sym{***}\\
                    &      (0.12)         &      (0.13)         &      (0.14)         &     (0.060)         &         (.)         &     (0.064)         \\
\addlinespace
naics1=4 $\times$ R\&D&        2.55\sym{***}&        1.59\sym{***}&        1.59\sym{**} &      0.0075         &       0.041         &       0.041\sym{**} \\
                    &      (0.39)         &      (0.61)         &      (0.60)         &    (0.0054)         &         (.)         &     (0.016)         \\
\addlinespace
naics1=5 $\times$ R\&D&        1.61\sym{***}&        1.69\sym{***}&        1.69\sym{***}&        1.01\sym{***}&        1.04         &        1.04\sym{***}\\
                    &      (0.39)         &      (0.43)         &      (0.31)         &      (0.12)         &         (.)         &      (0.13)         \\
\addlinespace
naics1=6 $\times$ R\&D&        1.69         &        4.98         &        4.98\sym{**} &      -0.054         &        0.47         &        0.47         \\
                    &      (1.39)         &      (3.81)         &      (2.25)         &     (0.056)         &         (.)         &      (0.29)         \\
\addlinespace
naics1=7 $\times$ R\&D&       -0.38         &       -0.13         &       -0.13         &        0.29\sym{***}&       -0.29         &       -0.29         \\
                    &      (0.58)         &      (0.76)         &      (1.38)         &     (0.099)         &         (.)         &      (0.26)         \\
\addlinespace
naics1=8 $\times$ R\&D&        3.34\sym{***}&                     &                     &        0.50\sym{***}&                     &                     \\
                    &      (0.82)         &                     &                     &      (0.19)         &                     &                     \\
\addlinespace
naics1=9 $\times$ R\&D&      -0.064         &        0.65         &        0.65\sym{*}  &      -0.022         &    -4.7e-17         &    -4.7e-17         \\
                    &      (0.51)         &      (0.58)         &      (0.35)         &     (0.029)         &         (.)         &   (2.5e-09)         \\
\addlinespace
Firm FE             &         Yes         &         Yes         &         Yes         &         Yes         &         Yes         &         Yes         \\
\addlinespace
Year FE             &         Yes         &          No         &          No         &         Yes         &          No         &          No         \\
\addlinespace
NAICS1-Age FE       &          No         &         Yes         &         Yes         &          No         &         Yes         &         Yes         \\
\addlinespace
Industry-Year FE    &          No         &         Yes         &         Yes         &          No         &         Yes         &         Yes         \\
\addlinespace
NAICS1-State-Year FE&          No         &         Yes         &         Yes         &          No         &         Yes         &         Yes         \\
\midrule
Clustering          &       gvkey         &       gvkey         &naics4 Statecode         &       gvkey         &       gvkey         &naics4 Statecode         \\
R-squared (adj.)    &        0.74         &        0.73         &        0.73         &        0.69         &        0.66         &        0.66         \\
R-squared (within, adj)&        0.43         &        0.38         &        0.38         &        0.34         &        0.28         &        0.28         \\
Observations        &       59485         &       56448         &       56448         &       59485         &       56448         &       56448         \\
\bottomrule
\multicolumn{7}{l}{\footnotesize Standard errors in parentheses}\\
\multicolumn{7}{l}{\footnotesize \sym{*} \(p<0.1\), \sym{**} \(p<0.05\), \sym{***} \(p<0.01\)}\\
\end{tabular}
}

	\caption{The regressions above relate corporate R\&D, and its interaction with 1-digit NAICS industry as well as an indicator for NCA enforcement, to the  entrepreneurship decisions of employees. The dependent variable is average yearly number of founders joining startups in years $t+1,t+2,t+3$. The independent variables are averages over $t,t-1,t-2$. Firm controls are employment, assets, intangible assets, investment, net income, cumulative citation-weighted patents, and the product of Tobin's Q and Assets (i.e., firm market value). Standard errors are clustered by firm in columns (1)-(3) and (5)-(7). In columns (4) and (8), standard errors are multway clustered by State and 4-digit NAICS industry.}
	\label{table:RDandSpinoutFormation_absolute_founder2_naics1_hNCA_l3f3}
\end{table}

\begin{table}[!htb]
	\scriptsize
	\centering
	{
\def\sym#1{\ifmmode^{#1}\else\(^{#1}\)\fi}
\begin{tabular}{l*{6}{c}}
\toprule
                    &\multicolumn{1}{c}{(1)}&\multicolumn{1}{c}{(2)}&\multicolumn{1}{c}{(3)}&\multicolumn{1}{c}{(4)}&\multicolumn{1}{c}{(5)}&\multicolumn{1}{c}{(6)}\\
                    &\multicolumn{1}{c}{Founders}&\multicolumn{1}{c}{Founders}&\multicolumn{1}{c}{Founders}&\multicolumn{1}{c}{WSO4}&\multicolumn{1}{c}{WSO4}&\multicolumn{1}{c}{WSO4}\\
\midrule
hNCA\_xrd\_l3\_ind\_2dig31&        0.11         &        0.21         &        0.21         &     -0.0097         &     -0.0023         &     -0.0023         \\
                    &      (0.34)         &      (0.29)         &      (0.39)         &    (0.0068)         &         (.)         &    (0.0073)         \\
\addlinespace
hNCA\_xrd\_l3\_ind\_2dig32&       0.090         &       0.086         &       0.086\sym{*}  &       -0.11         &       -0.13         &       -0.13\sym{***}\\
                    &     (0.060)         &     (0.061)         &     (0.046)         &     (0.085)         &         (.)         &     (0.036)         \\
\addlinespace
hNCA\_xrd\_l3\_ind\_2dig33&      -0.060         &      -0.061         &      -0.061         &       0.078         &       0.097         &       0.097         \\
                    &      (0.30)         &      (0.31)         &      (0.23)         &      (0.13)         &         (.)         &      (0.10)         \\
\addlinespace
hNCA\_xrd\_l3\_ind\_2dig51&      -0.010         &       -2.12\sym{*}  &       -2.12\sym{*}  &       -0.45\sym{***}&       -2.17         &       -2.17\sym{***}\\
                    &      (0.42)         &      (1.15)         &      (1.25)         &      (0.14)         &         (.)         &      (0.55)         \\
\addlinespace
hNCA\_xrd\_l3\_ind\_2dig52&        3.84\sym{*}  &        2.12         &        2.12         &        0.56         &        0.79         &        0.79         \\
                    &      (2.03)         &      (2.21)         &      (2.61)         &      (0.39)         &         (.)         &      (0.65)         \\
\addlinespace
hNCA\_xrd\_l3\_ind\_2dig53&        9.94\sym{***}&        0.56         &        0.56         &      -0.056         &     1.3e-16         &     1.3e-16         \\
                    &      (1.89)         &      (3.36)         &      (4.11)         &      (0.12)         &         (.)         &(0.00000028)         \\
\addlinespace
hNCA\_xrd\_l3\_ind\_2dig54&       -0.94         &       -1.66         &       -1.66         &       -0.25         &       -0.55         &       -0.55         \\
                    &      (0.85)         &      (1.22)         &      (1.35)         &      (0.51)         &         (.)         &      (0.42)         \\
\addlinespace
hNCA\_xrd\_l3\_ind\_2dig55&           0         &           0         &           0         &           0         &           0         &           0         \\
                    &         (.)         &         (.)         &(0.000000053)         &         (.)         &         (.)         &(0.00000027)         \\
\addlinespace
hNCA\_xrd\_l3\_ind\_2dig56&        4.82\sym{***}&        1.82         &        1.82         &      -0.060         &       -0.29         &       -0.29         \\
                    &      (1.77)         &      (4.24)         &      (4.46)         &      (0.11)         &         (.)         &      (0.77)         \\
\addlinespace
hNCA\_xrd\_l3\_notIn3or5&       -3.64\sym{***}&       -3.07\sym{***}&       -3.07\sym{***}&      -0.017         &      -0.018         &      -0.018         \\
                    &      (0.70)         &      (0.44)         &      (0.44)         &     (0.026)         &         (.)         &     (0.026)         \\
\addlinespace
naics2\_selected=0 $\times$ R\&D&        0.72         &        0.23         &        0.23         &      0.0055         &       0.015         &       0.015         \\
                    &      (0.52)         &      (0.23)         &      (0.39)         &     (0.022)         &         (.)         &     (0.024)         \\
\addlinespace
naics2\_selected=31 $\times$ R\&D&       -0.17         &       -0.20         &       -0.20         &      0.0036         &      0.0034         &      0.0034         \\
                    &      (0.29)         &      (0.26)         &      (0.32)         &    (0.0041)         &         (.)         &    (0.0051)         \\
\addlinespace
naics2\_selected=32 $\times$ R\&D&        0.39\sym{***}&        0.39\sym{***}&        0.39\sym{***}&        0.39\sym{***}&        0.39         &        0.39\sym{***}\\
                    &     (0.056)         &     (0.065)         &     (0.055)         &     (0.050)         &         (.)         &     (0.044)         \\
\addlinespace
naics2\_selected=33 $\times$ R\&D&        0.57\sym{**} &        0.53\sym{*}  &        0.53\sym{**} &       0.050         &       0.031         &       0.031         \\
                    &      (0.26)         &      (0.28)         &      (0.22)         &     (0.088)         &         (.)         &     (0.029)         \\
\addlinespace
naics2\_selected=51 $\times$ R\&D&        1.30\sym{***}&        1.42\sym{***}&        1.42\sym{***}&        0.99\sym{***}&        1.03         &        1.03\sym{***}\\
                    &      (0.37)         &      (0.34)         &      (0.40)         &      (0.13)         &         (.)         &      (0.13)         \\
\addlinespace
naics2\_selected=52 $\times$ R\&D&       -0.73\sym{*}  &       -0.38         &       -0.38         &       -0.67\sym{*}  &       -0.84         &       -0.84         \\
                    &      (0.44)         &      (0.70)         &      (0.85)         &      (0.39)         &         (.)         &      (0.59)         \\
\addlinespace
naics2\_selected=53 $\times$ R\&D&       -3.79\sym{**} &       -4.41\sym{***}&       -4.41\sym{**} &      -0.012         &     3.0e-17         &     3.0e-17         \\
                    &      (1.89)         &      (1.52)         &      (2.06)         &      (0.12)         &         (.)         &(0.00000013)         \\
\addlinespace
naics2\_selected=54 $\times$ R\&D&        0.48\sym{**} &        0.13         &        0.13         &        0.92\sym{**} &        0.93         &        0.93\sym{***}\\
                    &      (0.20)         &      (0.29)         &      (0.48)         &      (0.37)         &         (.)         &      (0.28)         \\
\addlinespace
naics2\_selected=56 $\times$ R\&D&        1.45         &        7.37\sym{**} &        7.37\sym{**} &       0.049         &       -0.29         &       -0.29         \\
                    &      (1.76)         &      (3.33)         &      (3.63)         &      (0.11)         &         (.)         &      (0.41)         \\
\addlinespace
Firm FE             &         Yes         &         Yes         &         Yes         &         Yes         &         Yes         &         Yes         \\
\addlinespace
Year FE             &         Yes         &          No         &          No         &         Yes         &          No         &          No         \\
\addlinespace
NAICS2*-Age FE      &          No         &         Yes         &         Yes         &          No         &         Yes         &         Yes         \\
\addlinespace
Industry-Year FE    &          No         &         Yes         &         Yes         &          No         &         Yes         &         Yes         \\
\addlinespace
NAICS2*-State-Year FE&          No         &         Yes         &         Yes         &          No         &         Yes         &         Yes         \\
\midrule
Clustering          &       gvkey         &       gvkey         &naics4 Statecode         &       gvkey         &       gvkey         &naics4 Statecode         \\
R-squared (adj.)    &        0.75         &        0.76         &        0.76         &        0.70         &        0.69         &        0.69         \\
R-squared (within, adj)&        0.47         &        0.26         &        0.26         &        0.37         &        0.22         &        0.22         \\
Observations        &       59485         &       56202         &       56202         &       59485         &       56202         &       56202         \\
\bottomrule
\multicolumn{7}{l}{\footnotesize Standard errors in parentheses}\\
\multicolumn{7}{l}{\footnotesize \sym{*} \(p<0.1\), \sym{**} \(p<0.05\), \sym{***} \(p<0.01\)}\\
\end{tabular}
}

	\caption{The regressions above relate corporate R\&D, and its interaction with some 2-digit NAICS industries as well as an indicator for NCA enforcement, to the  entrepreneurship decisions of employees. The dependent variable is average yearly number of founders joining startups in years $t+1,t+2,t+3$. The independent variables are averages over $t,t-1,t-2$. Firm controls are employment, assets, intangible assets, investment, net income, cumulative citation-weighted patents, and the product of Tobin's Q and Assets (i.e., firm market value). Standard errors are clustered by firm in columns (1)-(3) and (5)-(7). In columns (4) and (8), standard errors are multway clustered by State and 4-digit NAICS industry.}
	\label{table:RDandSpinoutFormation_absolute_founder2_naics2_hNCA_l3f3}
\end{table}


\begin{table}[!htb]
	\scriptsize
	\centering
	{
\def\sym#1{\ifmmode^{#1}\else\(^{#1}\)\fi}
\begin{tabular}{l*{6}{c}}
\toprule
                    &\multicolumn{1}{c}{(1)}&\multicolumn{1}{c}{(2)}&\multicolumn{1}{c}{(3)}&\multicolumn{1}{c}{(4)}&\multicolumn{1}{c}{(5)}&\multicolumn{1}{c}{(6)}\\
                    &\multicolumn{1}{c}{Founders}&\multicolumn{1}{c}{Founders}&\multicolumn{1}{c}{Founders}&\multicolumn{1}{c}{WSO4}&\multicolumn{1}{c}{WSO4}&\multicolumn{1}{c}{WSO4}\\
\midrule
hNCA\_xrd\_l3\_ind\_3dig325&       0.077         &       0.078         &       0.078         &       -0.11         &       -0.15\sym{**} &       -0.15\sym{***}\\
                    &     (0.055)         &     (0.056)         &     (0.055)         &     (0.079)         &     (0.076)         &     (0.039)         \\
\addlinespace
hNCA\_xrd\_l3\_ind\_3dig333&       -0.67         &       -0.65         &       -0.65         &       -0.36\sym{***}&       -0.41\sym{***}&       -0.41\sym{***}\\
                    &      (0.54)         &      (0.49)         &      (0.40)         &      (0.12)         &      (0.14)         &      (0.14)         \\
\addlinespace
hNCA\_xrd\_l3\_ind\_3dig334&        0.81\sym{**} &        0.85\sym{**} &        0.85\sym{***}&        0.39\sym{**} &        0.40\sym{**} &        0.40\sym{*}  \\
                    &      (0.34)         &      (0.40)         &      (0.11)         &      (0.19)         &      (0.18)         &      (0.20)         \\
\addlinespace
hNCA\_xrd\_l3\_ind\_3dig336&       0.016         &       0.062         &       0.062         &      -0.063\sym{*}  &      -0.073\sym{*}  &      -0.073         \\
                    &      (0.31)         &      (0.32)         &      (0.33)         &     (0.035)         &     (0.041)         &     (0.055)         \\
\addlinespace
hNCA\_xrd\_l3\_ind\_3dig511&       -2.65\sym{***}&       -4.21\sym{***}&       -4.21\sym{***}&       -1.97\sym{***}&       -2.67\sym{***}&       -2.67\sym{***}\\
                    &      (0.75)         &      (1.28)         &      (0.72)         &      (0.45)         &      (0.85)         &      (0.40)         \\
\addlinespace
hNCA\_xrd\_l3\_ind\_3dig519&        11.4\sym{***}&        11.1\sym{***}&        11.1\sym{***}&        1.42\sym{***}&        0.76         &        0.76         \\
                    &      (2.71)         &      (2.75)         &      (0.87)         &      (0.52)         &      (0.69)         &      (1.14)         \\
\addlinespace
naics3\_selected=0 $\times$ R\&D&        0.12         &       0.097         &       0.097         &       0.056         &       0.069\sym{*}  &       0.069         \\
                    &      (0.28)         &      (0.30)         &      (0.32)         &     (0.034)         &     (0.040)         &     (0.053)         \\
\addlinespace
naics3\_selected=325 $\times$ R\&D&        0.41\sym{***}&        0.43\sym{***}&        0.43\sym{***}&        0.38\sym{***}&        0.38\sym{***}&        0.38\sym{***}\\
                    &     (0.072)         &     (0.097)         &     (0.093)         &     (0.062)         &     (0.083)         &     (0.074)         \\
\addlinespace
naics3\_selected=333 $\times$ R\&D&        0.76\sym{**} &        0.83\sym{***}&        0.83\sym{**} &        0.29\sym{***}&        0.30\sym{***}&        0.30\sym{***}\\
                    &      (0.30)         &      (0.30)         &      (0.37)         &     (0.052)         &     (0.070)         &     (0.072)         \\
\addlinespace
naics3\_selected=334 $\times$ R\&D&        0.35         &        0.35         &        0.35\sym{**} &      -0.042         &      -0.049         &      -0.049         \\
                    &      (0.36)         &      (0.38)         &      (0.17)         &      (0.13)         &      (0.15)         &     (0.052)         \\
\addlinespace
naics3\_selected=511 $\times$ R\&D&        1.08\sym{***}&        1.26\sym{***}&        1.26\sym{***}&        0.83\sym{***}&        0.96\sym{***}&        0.96\sym{***}\\
                    &      (0.30)         &      (0.38)         &      (0.15)         &      (0.22)         &      (0.24)         &     (0.087)         \\
\addlinespace
naics3\_selected=519 $\times$ R\&D&       -3.56         &       -6.36\sym{***}&       -6.36\sym{***}&       -1.79\sym{***}&       -2.25\sym{***}&       -2.25\sym{***}\\
                    &      (2.51)         &      (1.07)         &      (0.46)         &      (0.51)         &      (0.30)         &      (0.28)         \\
\addlinespace
Firm FE             &         Yes         &         Yes         &         Yes         &         Yes         &         Yes         &         Yes         \\
\addlinespace
Year FE             &         Yes         &          No         &          No         &         Yes         &          No         &          No         \\
\addlinespace
NAICS3*-Age FE      &          No         &         Yes         &         Yes         &          No         &         Yes         &         Yes         \\
\addlinespace
Industry-Year FE    &          No         &         Yes         &         Yes         &          No         &         Yes         &         Yes         \\
\addlinespace
NAICS3*-State-Year FE&          No         &         Yes         &         Yes         &          No         &         Yes         &         Yes         \\
\midrule
Clustering          &       gvkey         &       gvkey         &naics4 Statecode         &       gvkey         &       gvkey         &naics4 Statecode         \\
R-squared (adj.)    &        0.75         &        0.76         &        0.75         &        0.72         &        0.72         &        0.72         \\
R-squared (within, adj)&        0.45         &        0.24         &        0.24         &        0.41         &        0.24         &        0.24         \\
Observations        &       59485         &       57149         &       57149         &       59485         &       57149         &       57149         \\
\bottomrule
\multicolumn{7}{l}{\footnotesize Standard errors in parentheses}\\
\multicolumn{7}{l}{\footnotesize \sym{*} \(p<0.1\), \sym{**} \(p<0.05\), \sym{***} \(p<0.01\)}\\
\end{tabular}
}

	\caption{The regressions above relate corporate R\&D, and its interaction with some 3-digit NAICS industries as well as an indicator for NCA enforcement, to the  entrepreneurship decisions of employees. The dependent variable is average yearly number of founders joining startups in years $t+1,t+2,t+3$. The independent variables are averages over $t,t-1,t-2$. Firm controls are employment, assets, intangible assets, investment, net income, cumulative citation-weighted patents, and the product of Tobin's Q and Assets (i.e., firm market value). Standard errors are clustered by firm in columns (1)-(3) and (5)-(7). In columns (4) and (8), standard errors are multway clustered by State and 4-digit NAICS industry.}
	\label{table:RDandSpinoutFormation_absolute_founder2_naics3_hNCA_l3f3}
\end{table}

\begin{table}[!htb]
	\scriptsize
	\centering
	{
\def\sym#1{\ifmmode^{#1}\else\(^{#1}\)\fi}
\begin{tabular}{l*{6}{c}}
\toprule
                    &\multicolumn{1}{c}{(1)}&\multicolumn{1}{c}{(2)}&\multicolumn{1}{c}{(3)}&\multicolumn{1}{c}{(4)}&\multicolumn{1}{c}{(5)}&\multicolumn{1}{c}{(6)}\\
                    &\multicolumn{1}{c}{$\frac{\textrm{Founders}}{\textrm{Assets}}$}&\multicolumn{1}{c}{$\frac{\textrm{Founders}}{\textrm{Assets}}$}&\multicolumn{1}{c}{$\frac{\textrm{Founders}}{\textrm{Assets}}$}&\multicolumn{1}{c}{$\frac{\textrm{WSO4}}{\textrm{Assets}}$}&\multicolumn{1}{c}{$\frac{\textrm{WSO4}}{\textrm{Assets}}$}&\multicolumn{1}{c}{$\frac{\textrm{WSO4}}{\textrm{Assets}}$}\\
\midrule
$\frac{\textrm{R\&D}}{\textrm{Assets}}$&        1.34         &        1.41         &        1.41         &        0.66\sym{*}  &        0.61\sym{+}  &        0.61\sym{**} \\
                    &      (1.12)         &      (1.14)         &      (1.19)         &      (0.40)         &      (0.40)         &      (0.30)         \\
\addlinespace
hNCA=0 $\times$ $\frac{\textrm{R\&D}}{\textrm{Assets}}$&           0         &           0         &           0         &           0         &           0         &           0         \\
                    &         (.)         &         (.)         &         (.)         &         (.)         &         (.)         &         (.)         \\
\addlinespace
hNCA=1 $\times$ $\frac{\textrm{R\&D}}{\textrm{Assets}}$&       -0.56         &       -0.54         &       -0.54         &       -0.30         &       -0.27         &       -0.27         \\
                    &      (1.39)         &      (1.42)         &      (1.36)         &      (0.76)         &      (0.76)         &      (0.47)         \\
\addlinespace
Firm FE             &         Yes         &         Yes         &         Yes         &         Yes         &         Yes         &         Yes         \\
\addlinespace
Year FE             &         Yes         &          No         &          No         &         Yes         &          No         &          No         \\
\addlinespace
Age FE              &          No         &         Yes         &         Yes         &          No         &         Yes         &         Yes         \\
\addlinespace
Industry-Year FE    &          No         &         Yes         &         Yes         &          No         &         Yes         &         Yes         \\
\addlinespace
State-Year FE       &          No         &         Yes         &         Yes         &          No         &         Yes         &         Yes         \\
\midrule
Clustering          &       gvkey         &       gvkey         &naics4 Statecode         &       gvkey         &       gvkey         &naics4 Statecode         \\
R-squared (adj.)    &        0.26         &        0.21         &        0.21         &        0.27         &        0.22         &        0.22         \\
R-squared (within, adj)&      0.0021         &      0.0020         &      0.0020         &     0.00093         &     0.00086         &     0.00086         \\
Observations        &       59477         &       57948         &       57948         &       59477         &       57948         &       57948         \\
\bottomrule
\multicolumn{7}{l}{\footnotesize Standard errors in parentheses}\\
\multicolumn{7}{l}{\footnotesize \sym{++} \(p<0.2\), \sym{+} \(p<0.15\), \sym{*} \(p<0.1\), \sym{**} \(p<0.05\), \sym{***} \(p<0.01\)}\\
\end{tabular}
}

	\caption{The regressions above relate corporate R\&D, and its interaction with 1-digit NAICS industry, to the  entrepreneurship decisions of employees. The dependent variable is average yearly number of founders joining startups in years $t+1,t+2,t+3$. The independent variables are averages over $t,t-1,t-2$. All LHS and RHS variable (except Tobin's Q) are normalized by a trailing 5 year moving average of assets. Firm controls are employment, assets, intangible assets, investment, net income, cumulative citation-weighted patents, and Tobin's Q. Standard errors are clustered by firm in columns (1)-(3) and (5)-(7). In columns (4) and (8), standard errors are multway clustered by State and 4-digit NAICS industry.}
	\label{table:RDandSpinoutFormation_at_founder2_hNCA_l3f3}
\end{table}

\begin{table}[!htb]
	\scriptsize
	\centering
	{
\def\sym#1{\ifmmode^{#1}\else\(^{#1}\)\fi}
\begin{tabular}{l*{6}{c}}
\toprule
                    &\multicolumn{1}{c}{(1)}&\multicolumn{1}{c}{(2)}&\multicolumn{1}{c}{(3)}&\multicolumn{1}{c}{(4)}&\multicolumn{1}{c}{(5)}&\multicolumn{1}{c}{(6)}\\
                    &\multicolumn{1}{c}{$\frac{\textrm{Founders}}{\textrm{Assets}}$}&\multicolumn{1}{c}{$\frac{\textrm{Founders}}{\textrm{Assets}}$}&\multicolumn{1}{c}{$\frac{\textrm{Founders}}{\textrm{Assets}}$}&\multicolumn{1}{c}{$\frac{\textrm{WSO4}}{\textrm{Assets}}$}&\multicolumn{1}{c}{$\frac{\textrm{WSO4}}{\textrm{Assets}}$}&\multicolumn{1}{c}{$\frac{\textrm{WSO4}}{\textrm{Assets}}$}\\
\midrule
xrd\_at\_l3\_industry\_1dig1&        0.13         &        0.57         &        0.57         &        0.17         &        0.39         &        0.39         \\
                    &      (0.36)         &      (1.05)         &      (0.70)         &      (0.21)         &      (0.86)         &      (0.39)         \\
\addlinespace
xrd\_at\_l3\_industry\_1dig2&       -0.14         &       -0.24         &       -0.24         &       0.013         &       0.046         &       0.046         \\
                    &      (0.14)         &      (0.19)         &      (0.17)         &     (0.073)         &      (0.12)         &     (0.066)         \\
\addlinespace
xrd\_at\_l3\_industry\_1dig3&        1.35         &        1.44         &        1.44         &        0.60         &        0.62         &        0.62\sym{***}\\
                    &      (0.87)         &      (0.89)         &      (0.94)         &      (0.38)         &      (0.42)         &      (0.20)         \\
\addlinespace
xrd\_at\_l3\_industry\_1dig4&       -0.33         &        0.38         &        0.38         &       -0.13         &        0.19         &        0.19         \\
                    &      (1.07)         &      (1.10)         &      (1.14)         &      (0.16)         &      (0.38)         &      (0.19)         \\
\addlinespace
xrd\_at\_l3\_industry\_1dig5&       0.034         &      -0.090         &      -0.090         &        0.33         &      -0.099         &      -0.099         \\
                    &      (1.02)         &      (1.07)         &      (1.15)         &      (0.75)         &      (0.90)         &      (0.85)         \\
\addlinespace
xrd\_at\_l3\_industry\_1dig6&        0.22         &        0.95         &        0.95         &       -0.12         &       -0.14         &       -0.14         \\
                    &      (0.57)         &      (0.94)         &      (1.11)         &      (0.16)         &      (0.21)         &      (0.21)         \\
\addlinespace
xrd\_at\_l3\_industry\_1dig7&       -13.8\sym{***}&       -15.8\sym{***}&       -15.8\sym{***}&       -0.31         &       -0.47         &       -0.47         \\
                    &      (5.20)         &      (4.09)         &      (3.44)         &      (0.71)         &      (0.95)         &      (0.86)         \\
\addlinespace
xrd\_at\_l3\_industry\_1dig8&        0.53\sym{*}  &        0.54         &        0.54         &      -0.060         &       -0.43         &       -0.43         \\
                    &      (0.28)         &      (0.79)         &      (0.69)         &      (0.17)         &      (0.33)         &      (0.33)         \\
\addlinespace
xrd\_at\_l3\_industry\_1dig9&       -0.30         &       -0.46         &       -0.46         &      -0.063         &      -0.026         &      -0.026         \\
                    &      (0.38)         &      (0.58)         &      (0.37)         &     (0.075)         &      (0.13)         &     (0.091)         \\
\addlinespace
Firm FE             &         Yes         &         Yes         &         Yes         &         Yes         &         Yes         &         Yes         \\
\addlinespace
Year FE             &         Yes         &          No         &          No         &         Yes         &          No         &          No         \\
\addlinespace
Age FE              &          No         &         Yes         &         Yes         &          No         &         Yes         &         Yes         \\
\addlinespace
Industry-Year FE    &          No         &         Yes         &         Yes         &          No         &         Yes         &         Yes         \\
\addlinespace
State-Year FE       &          No         &         Yes         &         Yes         &          No         &         Yes         &         Yes         \\
\midrule
Clustering          &       gvkey         &       gvkey         &naics4 Statecode         &       gvkey         &       gvkey         &naics4 Statecode         \\
R-squared (adj.)    &        0.26         &        0.21         &        0.21         &        0.27         &        0.22         &        0.22         \\
R-squared (within, adj)&      0.0023         &      0.0022         &      0.0022         &     0.00079         &     0.00083         &     0.00083         \\
Observations        &       59477         &       57948         &       57948         &       59477         &       57948         &       57948         \\
\bottomrule
\multicolumn{7}{l}{\footnotesize Standard errors in parentheses}\\
\multicolumn{7}{l}{\footnotesize \sym{*} \(p<0.1\), \sym{**} \(p<0.05\), \sym{***} \(p<0.01\)}\\
\end{tabular}
}

	\caption{The regressions above relate corporate R\&D, and its interaction with 1-digit NAICS industry as well as an indicator for NCA enforcement, to the  entrepreneurship decisions of employees. The dependent variable is average yearly number of founders joining startups in years $t+1,t+2,t+3$. The independent variables are averages over $t,t-1,t-2$. The independent variables are averages over $t,t-1,t-2$. All LHS and RHS variable (except Tobin's Q) are normalized by a trailing 5 year moving average of assets. Firm controls are employment, assets, intangible assets, investment, net income, cumulative citation-weighted patents, and the product of Tobin's Q. Standard errors are clustered by firm in columns (1)-(3) and (5)-(7). In columns (4) and (8), standard errors are multway clustered by State and 4-digit NAICS industry.}
	\label{table:RDandSpinoutFormation_at_founder2_naics1_l3f3}
\end{table}
1
\begin{table}[!htb]
	\scriptsize
	\centering
	{
\def\sym#1{\ifmmode^{#1}\else\(^{#1}\)\fi}
\begin{tabular}{l*{6}{c}}
\toprule
                    &\multicolumn{1}{c}{(1)}&\multicolumn{1}{c}{(2)}&\multicolumn{1}{c}{(3)}&\multicolumn{1}{c}{(4)}&\multicolumn{1}{c}{(5)}&\multicolumn{1}{c}{(6)}\\
                    &\multicolumn{1}{c}{$\frac{\textrm{Founders}}{\textrm{Assets}}$}&\multicolumn{1}{c}{$\frac{\textrm{Founders}}{\textrm{Assets}}$}&\multicolumn{1}{c}{$\frac{\textrm{Founders}}{\textrm{Assets}}$}&\multicolumn{1}{c}{$\frac{\textrm{WSO4}}{\textrm{Assets}}$}&\multicolumn{1}{c}{$\frac{\textrm{WSO4}}{\textrm{Assets}}$}&\multicolumn{1}{c}{$\frac{\textrm{WSO4}}{\textrm{Assets}}$}\\
\midrule
xrd\_at\_l3\_industry\_1dig1&        0.91         &        6.79         &        6.79         &        0.14         &        4.66         &        4.66         \\
                    &      (1.39)         &      (13.4)         &      (7.06)         &      (0.78)         &      (12.9)         &      (4.35)         \\
\addlinespace
xrd\_at\_l3\_industry\_1dig2&       -0.36         &       0.035         &       0.035         &       -0.21         &      -0.057         &      -0.057         \\
                    &      (0.97)         &      (1.33)         &      (0.99)         &      (0.51)         &      (0.76)         &      (0.49)         \\
\addlinespace
xrd\_at\_l3\_industry\_1dig3&        1.78         &        1.84         &        1.84         &        0.68         &        0.69         &        0.69\sym{**} \\
                    &      (1.36)         &      (1.36)         &      (1.54)         &      (0.47)         &      (0.45)         &      (0.32)         \\
\addlinespace
xrd\_at\_l3\_industry\_1dig4&      -0.040         &        0.53         &        0.53         &       -0.14         &       0.024         &       0.024         \\
                    &      (0.72)         &      (0.96)         &      (1.22)         &      (0.21)         &      (0.43)         &      (0.26)         \\
\addlinespace
xrd\_at\_l3\_industry\_1dig5&       -0.78         &       -0.95         &       -0.95         &        0.79         &        0.27         &        0.27         \\
                    &      (1.05)         &      (1.11)         &      (1.23)         &      (0.70)         &      (0.90)         &      (0.58)         \\
\addlinespace
xrd\_at\_l3\_industry\_1dig6&      -0.029         &        0.11         &        0.11         &      -0.096         &       0.063         &       0.063         \\
                    &      (0.43)         &      (0.57)         &      (0.58)         &      (0.17)         &      (0.20)         &      (0.22)         \\
\addlinespace
xrd\_at\_l3\_industry\_1dig7&       -17.4\sym{***}&       -18.4\sym{***}&       -18.4\sym{***}&       -0.31         &      -0.017         &      -0.017         \\
                    &      (3.52)         &      (2.94)         &      (2.17)         &      (0.86)         &      (0.95)         &      (0.91)         \\
\addlinespace
xrd\_at\_l3\_industry\_1dig8&        0.52\sym{*}  &        0.33         &        0.33         &      -0.057         &       -0.53         &       -0.53         \\
                    &      (0.28)         &      (0.72)         &      (0.80)         &      (0.17)         &      (0.37)         &      (0.44)         \\
\addlinespace
xrd\_at\_l3\_industry\_1dig9&       -0.54         &       -0.80         &       -0.80         &      -0.058         &       0.042         &       0.042         \\
                    &      (0.63)         &      (0.93)         &      (0.79)         &     (0.095)         &      (0.15)         &      (0.10)         \\
\addlinespace
hNCA\_xrd\_at\_l3\_industry\_1dig1&       -0.91         &       -6.60         &       -6.60         &       0.040         &       -4.54         &       -4.54         \\
                    &      (1.40)         &      (13.5)         &      (7.25)         &      (0.80)         &      (12.9)         &      (4.33)         \\
\addlinespace
hNCA\_xrd\_at\_l3\_industry\_1dig2&        0.24         &       -0.29         &       -0.29         &        0.24         &        0.11         &        0.11         \\
                    &      (0.97)         &      (1.36)         &      (0.90)         &      (0.51)         &      (0.75)         &      (0.46)         \\
\addlinespace
hNCA\_xrd\_at\_l3\_industry\_1dig3&       -0.95         &       -0.89         &       -0.89         &       -0.19         &       -0.17         &       -0.17         \\
                    &      (1.68)         &      (1.69)         &      (1.73)         &      (0.90)         &      (0.89)         &      (0.65)         \\
\addlinespace
hNCA\_xrd\_at\_l3\_industry\_1dig4&       -0.81         &       -0.44         &       -0.44         &       0.027         &        0.45         &        0.45         \\
                    &      (2.77)         &      (2.99)         &      (2.83)         &      (0.19)         &      (0.90)         &      (0.77)         \\
\addlinespace
hNCA\_xrd\_at\_l3\_industry\_1dig5&        1.62         &        1.65         &        1.65         &       -0.91         &       -0.70         &       -0.70         \\
                    &      (2.06)         &      (2.08)         &      (2.25)         &      (1.53)         &      (1.68)         &      (1.15)         \\
\addlinespace
hNCA\_xrd\_at\_l3\_industry\_1dig6&        0.58         &        2.25         &        2.25         &      -0.037         &       -0.53         &       -0.53         \\
                    &      (0.97)         &      (2.00)         &      (2.15)         &     (0.083)         &      (0.32)         &      (0.47)         \\
\addlinespace
hNCA\_xrd\_at\_l3\_industry\_1dig7&        18.0\sym{***}&        18.0\sym{***}&        18.0\sym{***}&     -0.0055         &       -2.82         &       -2.82\sym{**} \\
                    &      (3.80)         &      (4.34)         &      (1.61)         &      (1.01)         &      (2.42)         &      (1.28)         \\
\addlinespace
hNCA\_xrd\_at\_l3\_industry\_1dig8&       0.092         &        23.8         &        23.8         &     -0.0032         &        15.4         &        15.4         \\
                    &      (0.80)         &      (39.6)         &      (22.4)         &      (0.51)         &      (18.4)         &      (12.6)         \\
\addlinespace
hNCA\_xrd\_at\_l3\_industry\_1dig9&        0.53         &        0.76         &        0.76         &     -0.0084         &       -0.14         &       -0.14         \\
                    &      (0.62)         &      (0.94)         &      (0.71)         &     (0.058)         &      (0.15)         &      (0.13)         \\
\addlinespace
Firm FE             &         Yes         &         Yes         &         Yes         &         Yes         &         Yes         &         Yes         \\
\addlinespace
Year FE             &         Yes         &          No         &          No         &         Yes         &          No         &          No         \\
\addlinespace
Age FE              &          No         &         Yes         &         Yes         &          No         &         Yes         &         Yes         \\
\addlinespace
Industry-Year FE    &          No         &         Yes         &         Yes         &          No         &         Yes         &         Yes         \\
\addlinespace
State-Year FE       &          No         &         Yes         &         Yes         &          No         &         Yes         &         Yes         \\
\midrule
Clustering          &       gvkey         &       gvkey         &naics4 Statecode         &       gvkey         &       gvkey         &naics4 Statecode         \\
R-squared (adj.)    &        0.26         &        0.21         &        0.21         &        0.27         &        0.22         &        0.22         \\
R-squared (within, adj)&      0.0025         &      0.0023         &      0.0023         &     0.00072         &     0.00070         &     0.00070         \\
Observations        &       59477         &       57948         &       57948         &       59477         &       57948         &       57948         \\
\bottomrule
\multicolumn{7}{l}{\footnotesize Standard errors in parentheses}\\
\multicolumn{7}{l}{\footnotesize \sym{*} \(p<0.1\), \sym{**} \(p<0.05\), \sym{***} \(p<0.01\)}\\
\end{tabular}
}

	\caption{The regressions above relate corporate R\&D, and its interaction with 1-digit NAICS industry as well as an indicator for NCA enforcement, to the  entrepreneurship decisions of employees. The dependent variable is average yearly number of founders joining startups in years $t+1,t+2,t+3$. The independent variables are averages over $t,t-1,t-2$. The independent variables are averages over $t,t-1,t-2$. All LHS and RHS variable (except Tobin's Q) are normalized by a trailing 5 year moving average of assets. Firm controls are employment, assets, intangible assets, investment, net income, cumulative citation-weighted patents, and the product of Tobin's Q. Standard errors are clustered by firm in columns (1)-(3) and (5)-(7). In columns (4) and (8), standard errors are multway clustered by State and 4-digit NAICS industry.}
	\label{table:RDandSpinoutFormation_at_founder2_naics1_hNCA_l3f3}
\end{table}

\begin{table}[!htb]
	\scriptsize
	\centering
	{
\def\sym#1{\ifmmode^{#1}\else\(^{#1}\)\fi}
\begin{tabular}{l*{6}{c}}
\toprule
                    &\multicolumn{1}{c}{(1)}&\multicolumn{1}{c}{(2)}&\multicolumn{1}{c}{(3)}&\multicolumn{1}{c}{(4)}&\multicolumn{1}{c}{(5)}&\multicolumn{1}{c}{(6)}\\
                    &\multicolumn{1}{c}{$\frac{\textrm{Founders}}{\textrm{Assets}}$}&\multicolumn{1}{c}{$\frac{\textrm{Founders}}{\textrm{Assets}}$}&\multicolumn{1}{c}{$\frac{\textrm{Founders}}{\textrm{Assets}}$}&\multicolumn{1}{c}{$\frac{\textrm{WSO4}}{\textrm{Assets}}$}&\multicolumn{1}{c}{$\frac{\textrm{WSO4}}{\textrm{Assets}}$}&\multicolumn{1}{c}{$\frac{\textrm{WSO4}}{\textrm{Assets}}$}\\
\midrule
xrd\_at\_l3\_notIn3or5 &       -0.48         &       -0.53         &       -0.53         &      -0.080         &       0.042         &       0.042         \\
                    &      (0.49)         &      (0.62)         &      (0.50)         &      (0.14)         &      (0.15)         &      (0.15)         \\
\addlinespace
hNCA\_xrd\_at\_l3\_notIn3or5&        0.52         &        0.87         &        0.87         &       0.035         &       -0.13         &       -0.13         \\
                    &      (0.51)         &      (0.73)         &      (0.59)         &     (0.099)         &      (0.15)         &      (0.19)         \\
\addlinespace
xrd\_at\_l3\_industry\_2dig31&       -0.10         &        1.40         &        1.40\sym{**} &      -0.059         &       -0.34         &       -0.34         \\
                    &      (0.47)         &      (1.07)         &      (0.61)         &      (0.15)         &      (0.43)         &      (0.37)         \\
\addlinespace
xrd\_at\_l3\_industry\_2dig32&        0.78         &        0.89         &        0.89\sym{*}  &        1.02         &        1.03         &        1.03         \\
                    &      (0.79)         &      (0.79)         &      (0.46)         &      (0.72)         &      (0.70)         &      (0.62)         \\
\addlinespace
xrd\_at\_l3\_industry\_2dig33&        3.76         &        3.89         &        3.89         &       0.018         &     -0.0035         &     -0.0035         \\
                    &      (3.56)         &      (3.71)         &      (4.00)         &      (0.12)         &      (0.14)         &      (0.26)         \\
\addlinespace
hNCA\_xrd\_at\_l3\_industry\_2dig31&       -0.16         &       -1.78         &       -1.78         &        0.22         &       0.056         &       0.056         \\
                    &      (0.64)         &      (1.82)         &      (1.82)         &      (0.30)         &      (1.03)         &      (1.04)         \\
\addlinespace
hNCA\_xrd\_at\_l3\_industry\_2dig32&       -0.78         &       -0.67         &       -0.67\sym{*}  &       -0.86         &       -0.78         &       -0.78\sym{*}  \\
                    &      (1.42)         &      (1.44)         &      (0.35)         &      (1.26)         &      (1.23)         &      (0.46)         \\
\addlinespace
hNCA\_xrd\_at\_l3\_industry\_2dig33&       -1.37         &       -1.46         &       -1.46         &        1.13         &        1.07         &        1.07         \\
                    &      (4.00)         &      (4.09)         &      (4.87)         &      (1.10)         &      (1.07)         &      (0.99)         \\
\addlinespace
xrd\_at\_l3\_industry\_2dig51&       -1.42         &       -1.56         &       -1.56         &        0.60         &       -0.23         &       -0.23         \\
                    &      (1.38)         &      (1.37)         &      (1.60)         &      (0.55)         &      (0.93)         &      (0.27)         \\
\addlinespace
xrd\_at\_l3\_industry\_2dig52&       -2.27         &       -2.91         &       -2.91\sym{***}&       -0.29         &       -0.25         &       -0.25\sym{*}  \\
                    &      (2.72)         &      (3.56)         &      (1.03)         &      (0.31)         &      (0.28)         &      (0.13)         \\
\addlinespace
xrd\_at\_l3\_industry\_2dig53&     -0.0084         &        0.69         &        0.69         &      -0.074         &       0.067         &       0.067         \\
                    &      (0.34)         &      (0.67)         &      (0.54)         &      (0.15)         &      (0.30)         &      (0.34)         \\
\addlinespace
xrd\_at\_l3\_industry\_2dig54&        0.59         &       -0.16         &       -0.16         &        1.73         &        1.25         &        1.25         \\
                    &      (2.54)         &      (2.88)         &      (2.06)         &      (2.46)         &      (2.75)         &      (1.25)         \\
\addlinespace
xrd\_at\_l3\_industry\_2dig55&           0         &           0         &           0         &           0         &           0         &           0         \\
                    &         (.)         &         (.)         &   (3.4e-09)         &         (.)         &         (.)         &   (1.8e-09)         \\
\addlinespace
xrd\_at\_l3\_industry\_2dig56&       0.058         &        0.45         &        0.45         &      0.0045         &        0.29         &        0.29         \\
                    &      (0.29)         &      (0.42)         &      (0.45)         &      (0.13)         &      (0.30)         &      (0.30)         \\
\addlinespace
hNCA\_xrd\_at\_l3\_industry\_2dig51&        0.27         &        0.21         &        0.21         &       -2.13         &       -1.58         &       -1.58         \\
                    &      (2.44)         &      (2.50)         &      (2.42)         &      (2.00)         &      (2.17)         &      (1.00)         \\
\addlinespace
hNCA\_xrd\_at\_l3\_industry\_2dig52&        1.78         &        2.03         &        2.03\sym{*}  &        0.17         &       0.030         &       0.030         \\
                    &      (2.72)         &      (3.08)         &      (1.08)         &      (0.30)         &      (0.77)         &      (0.86)         \\
\addlinespace
hNCA\_xrd\_at\_l3\_industry\_2dig53&       0.066         &       -0.39         &       -0.39         &       0.053         &        0.29         &        0.29         \\
                    &      (0.29)         &      (0.69)         &      (0.66)         &      (0.11)         &      (0.36)         &      (0.32)         \\
\addlinespace
hNCA\_xrd\_at\_l3\_industry\_2dig54&        8.07         &        8.79         &        8.79\sym{***}&        2.76         &        2.47         &        2.47\sym{*}  \\
                    &      (6.33)         &      (6.01)         &      (3.00)         &      (3.74)         &      (3.91)         &      (1.43)         \\
\addlinespace
hNCA\_xrd\_at\_l3\_industry\_2dig55&           0         &           0         &           0         &           0         &           0         &           0         \\
                    &         (.)         &         (.)         &   (1.4e-10)         &         (.)         &         (.)         &   (2.6e-10)         \\
\addlinespace
hNCA\_xrd\_at\_l3\_industry\_2dig56&       -1.60         &       -2.41         &       -2.41         &       0.061         &        0.51         &        0.51         \\
                    &      (3.28)         &      (3.95)         &      (4.62)         &      (0.13)         &      (0.84)         &      (0.64)         \\
\addlinespace
Firm FE             &         Yes         &         Yes         &         Yes         &         Yes         &         Yes         &         Yes         \\
\addlinespace
Year FE             &         Yes         &          No         &          No         &         Yes         &          No         &          No         \\
\addlinespace
Age FE              &          No         &         Yes         &         Yes         &          No         &         Yes         &         Yes         \\
\addlinespace
Industry-Year FE    &          No         &         Yes         &         Yes         &          No         &         Yes         &         Yes         \\
\addlinespace
State-Year FE       &          No         &         Yes         &         Yes         &          No         &         Yes         &         Yes         \\
\midrule
Clustering          &       gvkey         &       gvkey         &naics4 Statecode         &       gvkey         &       gvkey         &naics4 Statecode         \\
R-squared (adj.)    &        0.27         &        0.22         &        0.21         &        0.27         &        0.22         &        0.22         \\
R-squared (within, adj)&      0.0051         &      0.0047         &      0.0047         &      0.0024         &      0.0021         &      0.0021         \\
Observations        &       59477         &       57948         &       57948         &       59477         &       57948         &       57948         \\
\bottomrule
\multicolumn{7}{l}{\footnotesize Standard errors in parentheses}\\
\multicolumn{7}{l}{\footnotesize \sym{*} \(p<0.1\), \sym{**} \(p<0.05\), \sym{***} \(p<0.01\)}\\
\end{tabular}
}

	\caption{The regressions above relate corporate R\&D, and its interaction with some 2-digit NAICS industries as well as an indicator for NCA enforcement, to the  entrepreneurship decisions of employees. The dependent variable is average yearly number of founders joining startups in years $t+1,t+2,t+3$. The independent variables are averages over $t,t-1,t-2$. The independent variables are averages over $t,t-1,t-2$. All LHS and RHS variable (except Tobin's Q) are normalized by a trailing 5 year moving average of assets. Firm controls are employment, assets, intangible assets, investment, net income, cumulative citation-weighted patents, and the product of Tobin's Q. Standard errors are clustered by firm in columns (1)-(3) and (5)-(7). In columns (4) and (8), standard errors are multway clustered by State and 4-digit NAICS industry.}
	\label{table:RDandSpinoutFormation_at_founder2_naics2_hNCA_l3f3}
\end{table}


\begin{table}[!htb]
	\scriptsize
	\centering
	{
\def\sym#1{\ifmmode^{#1}\else\(^{#1}\)\fi}
\begin{tabular}{l*{6}{c}}
\toprule
                    &\multicolumn{1}{c}{(1)}&\multicolumn{1}{c}{(2)}&\multicolumn{1}{c}{(3)}&\multicolumn{1}{c}{(4)}&\multicolumn{1}{c}{(5)}&\multicolumn{1}{c}{(6)}\\
                    &\multicolumn{1}{c}{Founders}&\multicolumn{1}{c}{Founders}&\multicolumn{1}{c}{Founders}&\multicolumn{1}{c}{WSO4}&\multicolumn{1}{c}{WSO4}&\multicolumn{1}{c}{WSO4}\\
\midrule
hNCA\_xrd\_at\_l3\_ind\_3dig325&       -69.2         &       -65.1         &       -65.1         &       -52.4         &       -64.2         &       -64.2         \\
                    &      (43.4)         &      (50.9)         &      (54.8)         &      (36.9)         &      (40.7)         &      (43.4)         \\
\addlinespace
hNCA\_xrd\_at\_l3\_ind\_3dig333&       582.7         &       789.0\sym{**} &       789.0\sym{***}&       141.9\sym{**} &       183.9\sym{***}&       183.9\sym{***}\\
                    &     (425.5)         &     (306.2)         &     (227.3)         &      (64.0)         &      (69.7)         &      (46.4)         \\
\addlinespace
hNCA\_xrd\_at\_l3\_ind\_3dig334&        52.6         &        97.6         &        97.6         &        14.3         &        15.2         &        15.2         \\
                    &     (181.9)         &     (203.9)         &     (155.1)         &      (54.5)         &      (61.8)         &      (36.1)         \\
\addlinespace
hNCA\_xrd\_at\_l3\_ind\_3dig336&      -378.7         &      -329.0         &      -329.0         &        7.24         &       -4.98         &       -4.98         \\
                    &     (393.9)         &     (350.0)         &     (331.1)         &     (180.1)         &     (185.2)         &      (68.5)         \\
\addlinespace
hNCA\_xrd\_at\_l3\_ind\_3dig511&     -1161.2         &       -18.5         &       -18.5         &      -404.0         &       -65.4         &       -65.4         \\
                    &    (1435.7)         &     (545.9)         &     (233.9)         &     (501.8)         &     (336.4)         &      (88.5)         \\
\addlinespace
hNCA\_xrd\_at\_l3\_ind\_3dig519&      1735.1         &      2426.6\sym{**} &      2426.6\sym{***}&        81.4         &       411.2         &       411.2\sym{**} \\
                    &    (1095.5)         &    (1119.6)         &     (413.7)         &     (190.0)         &     (562.4)         &     (165.8)         \\
\addlinespace
naics3\_selected=0 $\times$ $\frac{\textrm{R\&D}}{\textrm{Assets}}$&        16.7         &       -11.1         &       -11.1         &        29.3\sym{***}&        18.9\sym{*}  &        18.9         \\
                    &      (26.9)         &      (30.0)         &      (35.7)         &      (10.7)         &      (10.3)         &      (17.1)         \\
\addlinespace
naics3\_selected=325 $\times$ $\frac{\textrm{R\&D}}{\textrm{Assets}}$&        29.3         &        41.2         &        41.2         &        43.0\sym{*}  &        68.1\sym{*}  &        68.1\sym{**} \\
                    &      (28.9)         &      (43.2)         &      (38.5)         &      (24.5)         &      (36.8)         &      (30.7)         \\
\addlinespace
naics3\_selected=333 $\times$ $\frac{\textrm{R\&D}}{\textrm{Assets}}$&      -533.8         &      -634.9\sym{**} &      -634.9\sym{***}&      -148.3\sym{**} &      -170.0\sym{**} &      -170.0\sym{***}\\
                    &     (439.0)         &     (295.7)         &     (142.0)         &      (71.4)         &      (67.8)         &      (37.9)         \\
\addlinespace
naics3\_selected=334 $\times$ $\frac{\textrm{R\&D}}{\textrm{Assets}}$&        47.9         &        56.4         &        56.4         &        54.6         &        41.4         &        41.4         \\
                    &     (100.4)         &     (101.8)         &     (127.9)         &      (37.2)         &      (43.8)         &      (40.0)         \\
\addlinespace
naics3\_selected=511 $\times$ $\frac{\textrm{R\&D}}{\textrm{Assets}}$&        15.6         &       -55.6         &       -55.6         &        71.6         &        59.1         &        59.1         \\
                    &     (550.1)         &     (558.4)         &     (328.0)         &     (317.7)         &     (345.1)         &     (138.6)         \\
\addlinespace
naics3\_selected=519 $\times$ $\frac{\textrm{R\&D}}{\textrm{Assets}}$&       -11.4         &     -1422.0         &     -1422.0\sym{***}&      -119.3         &       -95.1         &       -95.1\sym{*}  \\
                    &     (621.9)         &    (1005.3)         &     (140.7)         &     (122.4)         &     (285.2)         &      (50.6)         \\
\addlinespace
Firm FE             &         Yes         &         Yes         &         Yes         &         Yes         &         Yes         &         Yes         \\
\addlinespace
Year FE             &         Yes         &          No         &          No         &         Yes         &          No         &          No         \\
\addlinespace
NAICS3*-Age FE      &          No         &         Yes         &         Yes         &          No         &         Yes         &         Yes         \\
\addlinespace
Industry-Year FE    &          No         &         Yes         &         Yes         &          No         &         Yes         &         Yes         \\
\addlinespace
NAICS3*-State-Year FE&          No         &         Yes         &         Yes         &          No         &         Yes         &         Yes         \\
\midrule
Clustering          &       gvkey         &       gvkey         &naics4 Statecode         &       gvkey         &       gvkey         &naics4 Statecode         \\
R-squared (adj.)    &        0.55         &        0.68         &        0.68         &        0.54         &        0.63         &        0.63         \\
R-squared (within, adj)&       0.017         &      0.0067         &      0.0067         &       0.017         &      0.0091         &      0.0091         \\
Observations        &       59481         &       57145         &       57145         &       59481         &       57145         &       57145         \\
\bottomrule
\multicolumn{7}{l}{\footnotesize Standard errors in parentheses}\\
\multicolumn{7}{l}{\footnotesize \sym{*} \(p<0.1\), \sym{**} \(p<0.05\), \sym{***} \(p<0.01\)}\\
\end{tabular}
}

	\caption{The regressions above relate corporate R\&D, and its interaction with some 3-digit NAICS industries as well as an indicator for NCA enforcement, to the  entrepreneurship decisions of employees. The dependent variable is average yearly number of founders joining startups in years $t+1,t+2,t+3$. The independent variables are averages over $t,t-1,t-2$. The independent variables are averages over $t,t-1,t-2$. All LHS and RHS variable (except Tobin's Q) are normalized by a trailing 5 year moving average of assets. Firm controls are employment, assets, intangible assets, investment, net income, cumulative citation-weighted patents, and the product of Tobin's Q. Standard errors are clustered by firm in columns (1)-(3) and (5)-(7). In columns (4) and (8), standard errors are multway clustered by State and 4-digit NAICS industry.}
	\label{table:RDandSpinoutFormation_at_founder2_naics3_hNCA_l3f3}
\end{table}



\section{Empirics}

\subsection{Further challenges to identification and interpretation}

\textbf{[Remove this discussion and related tables - outside the scope of the paper]}

The above discussion cannot quite be mapped to my model for two basic reasons. First, the relationship might not be causal. That is difficult to address without exogenous variation. There are some instruments for R\&D spending, but I have had issues constructing them (in particular my first stage is perplexingly different from the first stage obtained in \cite{babina_entrepreneurial_2019}). Currently, I prefer the route of simply admitting that the identification isn't perfect and trying to be explicit about what bounds I am putting on the omitted confounding variable in order to be able to reasonably claim that I can map this finding to my model (which does not include any omitted confounding variables for the regression I run). 

Second, 4-digit NAICS industry is, at best, a rough proxy of competition. It could be that spinouts in the same 4-digit NAICS industry are in fact complementary, rather than substitutes, to their parent firms.\footnote{It could also be that the NAICS industry code used is outdated, as I discovered today when looking at Compustat. Many data processing firms from the 1990s for example have a NAICS code starting in 54 which would have been modified to 51 in the 2017 NAICS. Since my NAICS-Venture Source crosswalk is based on the 2017 NAICS, this means I miss a lot of WSOs this way. I e-mailed WRDS to get some clarity on what the naics variable in Compsutat actually refers to.} \cite{babina_entrepreneurial_2019} show some results suggesting this interpretation based on how spinout and parent-firm industries relate in the BEA input-output tables, but since their measure is industry-based, it cannot directly address the possibility of within-industry complementarity. They actually address this by claiming (not showing) that a regression which interacts R\&D with a measure of NCA enforcement does not find that higher enforcement attenuates the effect. This is interpreted as evidence against an "idea stealing" theory.  

\autoref{table:RDandSpinoutFormation_absolute_founder2_hNCA_l3f3} shows this regression in my data and confirms the alluded to finding of \cite{babina_entrepreneurial_2019} that higher NCA enforcement does not reduce the R\&D-spinout correlation, interpreted as evidence that spinouts do not typically compete with their parent. To further explore the nature of this relationship, \autoref{table:RDandSpinoutFormation_absolute_founder2_naics1_hNCA_l3f3} additionally interacts the interacted variable with 1-digit industry codes.\footnote{For reference, \autoref{table:RDandSpinoutFormation_absolute_founder2_naics1_l3f3} shows the regression where R\&D is interacted only with 1-digit NAICS codes.} Focusing on the right-most column, two notable results emerge from this regression. The relationship between WSOs and R\&Ds is statistically indistinguishable across NCA enforcement regimes for most industries, but crucially not for industry 5, where the difference is substantial (high NCA enforcement cuts the relationship in half) and statistically significant at the 5\% level (\textbf{multiple hypothesis testing...}). Also, in industry 5 high NCA enforcement increases the relationship between R\&D and spinout formation  overall, which suggests that spinouts are being formed in other industries as a result of R\&D when NCAs are used. This warrants further exploration, since industry 5 is the industry with by far the most WSOs (and certainly with the most VC-funded startups). Since industry 5 includes several disparate industries (software publishers and data processing services, but also finance / insurance, consulting and management services), it is probably worth looking for heterogeneity within industry 5. Moreover, the fact that industry 3, which contains all manufacturing, is unaffected by NCAs may depend on the specific industry within manufacturing. So my next step is to explore how the effect depends on 2-digit industries.

\autoref{table:RDandSpinoutFormation_absolute_founder2_naics2_hNCA_l3f3} shows the regressions in \autoref{table:RDandSpinoutFormation_absolute_founder2_naics1_hNCA_l3f3} with NAICS industries 3 and 5 expanded to 2-digit industries. Broken out by industry, the effect of high NCA enforcement is typically to reduce the effect of R\&D on WSO spinouts (column 8), though sometimes the reduction is statistically insignificant. Interestingly, the relationship is stronger overall in states with low NCA enforcement, suggesting that the lost WSOs are more than made up for by non-WSO spinouts. This is an interesting finding which further emphasizes how important it is to analyze this relationship at the industry level. 

For now, I conclude that my data offers evidence against the conclusion of \cite{babina_entrepreneurial_2019} that NCA enforcement does not affect the relationship between R\&D and spinout formation. I believe they reached this conclusion because when they assessed the effect of NCA enforcement they did not (1) separate WSOs from non-WSO spinouts or (2) control for the result being driven by industry. Viewed "roughly" through the lens of my model, different industries my have different values of $\nu$ and $\kappa_e$. 



%% footnote about stationary distribution of q

This property is not only analytically convenient -- it is necessary for a BGP to exist. In the baseline model, the expected growth rate of normalized frontier quality $\tilde{\bar{q}}_j = \frac{\bar{q}_j}{Q}$ is constant for all $j \in [0,1]$ and there is no exit of low quality firms (and subsequent injection of "average quality" firms). Running this stochastic process forward in time, the distribution of $\tilde{\bar{q}}_j$ spreads out, i.e. its variance and higher order measures of dispersion increase, which implies that there is no stationary distribution of $\tilde{\bar{q}}_j$. A BGP continues to exist, however, because only the mean of $\tilde{\bar{q}}_j$, $\mathbb{E}[\tilde{\bar{q}}_j] = 1$, is relevant for aggregate variables. This is why, e.g. the growth accounting equation (\ref{eq:growth_accounting}) can be written so simply. If, instead, growth is faster for higher $\tilde{\bar{q}}_j$, as is the case with a fixed entry fee, there is again no stationary distribution of $\tilde{\bar{q}}_j$, as before. However, in addition, there is no BGP, because aggregate variables such as the growth rate and the R\&D wage now depend on the entire distribution of $\tilde{\bar{q}}_j$, which is not stationary.

% naics codes summaries

\begin{table}[]
	\centering
	\captionof{table}{3-digit breakdown of NAICS 31-33}\label{}
	\begin{tabular}{rl}
		\toprule \toprule
		3-digit code & Description \tabularnewline
		\midrule
		311 & Food \tabularnewline 
		312 & Beverage and tobacco  \tabularnewline 
		313-316 & Textiles, apparel and leather  \tabularnewline
		321-323 & Wood, paper manufacturing and printing \tabularnewline  
		324 & Petroleum and coal products \tabularnewline
		325 & Chemical \tabularnewline
		326 & Plastics and rubber \tabularnewline
		327 & Nonmetallic mineral products \tabularnewline 
		331-332 & Primary and fabricated metal  \tabularnewline
		333 & Machinery \tabularnewline
		334 & Computer and electronic product \tabularnewline
		335 & Electrical equipment \tabularnewline
		336 & Transportation equipment \tabularnewline
		337 & Furniture \tabularnewline
		339 & Misc \tabularnewline
		\bottomrule
	\end{tabular}
\end{table}

\begin{table}[]
	\centering
	\captionof{table}{\textbf{[in progress]} 3-digit breakdown of NAICS 51-56}\label{}
	\begin{tabular}{rl}
		\toprule \toprule
		3-digit code & Description \tabularnewline
		\midrule
		511 & Publishing (non-internet) \tabularnewline 
		512 & Motion picture and sound recording \tabularnewline 
		515 & Broadcasting (non-internet) \tabularnewline
		517 & Telecommunications \tabularnewline  
		518 & Data processing, hosting and related services \tabularnewline
		519 & Other information services \tabularnewline
		& Plastics and rubber \tabularnewline
		327 & Nonmetallic mineral products \tabularnewline 
		331-332 & Primary and fabricated metal  \tabularnewline
		333 & Machinery \tabularnewline
		334 & Computer and electronic product \tabularnewline
		335 & Electrical equipment \tabularnewline
		336 & Transportation equipment \tabularnewline
		337 & Furniture \tabularnewline
		339 & Misc \tabularnewline
		\bottomrule
	\end{tabular}
\end{table}