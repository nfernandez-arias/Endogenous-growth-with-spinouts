




\subsubsection{Numerical simulations}

Because it is difficult to make broad claims about welfare, below I exhibit different parameter settings that lead to different socially optimal $\kappa_c$. 

\subsubsection{title}

\subsubsection{Extension with heterogeneity in $\kappa_e$}

It is possible to extend the model to a case where the cost of entry into good $j$ is $\kappa_{e,jt}$. Assume that every time a good $j$ is successfully improved (by an incumbent, entrant or spinout) a new $\kappa_{ejt}$ is drawn from an independent uniform distribution on $[0,1]$. Along the BGP, for each value of $\kappa_c$ there exists a threshold $\bar{\kappa}_{e}(\kappa_c) \in [0,1]$ such that for all $j$ with $\kappa_{e,jt} > \bar{\kappa}_{e} (\kappa_c) $ the incumbent uses a non-compete. To be consistent with the baseline model's notation, define $\bar{\kappa}_c = \inf \{\kappa_c \ge 0: \bar{\kappa}_e(\kappa_c) = 1 \}$.  

Define $\tilde{V}$ as the value of incumbency prior to the realization of $\kappa_{e,jt}$. Define $\hat{V}(\kappa_e)$ as the ex-post value. These quantities are related by 
\begin{align}
	\tilde{V} &\equiv \mathbb{E}_{\kappa_e} [V(\kappa_e)] = \int_0^1 V(\kappa_e) d\kappa_e
\end{align}

As before, the cost of enforcing noncompetes is given by $\kappa_c \nu \tilde{V}$. The incumbent HJB becomes
\begin{align}
	(r + \tau_E) \hat{V}(\kappa_e) = \tilde{\pi} + \max_{\substack{x \in \{0,1\} \\ z \ge 0}} \Bigg\{ z \Big( &\chi_I(\lambda - 1) \tilde{V} - w_{RD,E} - x \nu \kappa_c \tilde{V} \nonumber \\
	&- (1-x) \nu \big( V(\kappa_e) - (1-\kappa_e) \lambda \tilde{V} \big)   \Big)  \Bigg\} \label{ext:hjb_incumbent}
\end{align}

Suppose that $\kappa_c < \bar{\kappa}_c$, so that a positive measure set of incumbents use non-competes, i.e. those with $\bar{\kappa}_e(\kappa_c) < \kappa_e < 1$. Further suppose we are in an equilibrium where some of these incumbents incur positive R\&D expenditures. Along this range we know that $\hat{V}(\kappa_e) = \hat{\underline{V}}$ is constant, since non-competes are used and the value of $\kappa_e$ is not relevant for the value of the incumbent conditional on innovation (by any agent) due to the fact that draws of $\kappa_e$ are independent over time. Therefore, if any incur expenditures, all do. Also, 
\begin{align}
	\hat{\underline{V}} &= \frac{\chi_I(\lambda-1) \tilde{V} - w_{RD}}{\kappa_c \nu }	
\end{align}

Now consider $\kappa_e \le \bar{\kappa}_e(\kappa_c)$. Holding constant the value $V(\kappa_e)$, it is apparent from (\ref{ext:hjb_incumbent}) that the effective wage paid for R\&D labor is lower for lower values of $\kappa_e$. This is due to the fact that the worker perceives a higher expected value of future spinout formation. The expected benefit to the incumbent from R\&D, on the other hand, does not depend on $\kappa_e$. In order for the FOC to be satisfied on a positive measure set $K \subseteq \{ \kappa_e: \kappa_e < \bar{\kappa}_e \}$, the expected wage must be the same for all such $\kappa_e$. This requires that
\begin{align}
	V(\kappa_e) = \Theta + (1-\kappa_e)\lambda \tilde{V} 
\end{align}

for some constant $\Theta$. However, if indeed it is the case that 

\paragraph{Household problem}

\begin{maxi*}[1]<b>
	{\substack{\{C(t)\}_{t \ge 0} \\ \{\ell_{RD,j}(t)\}_{j \in [0,1]} \\ \{\hat{\ell}_{RD,j}(t)\}_{j \in [0,1]} \\ L_I(t),L_F(t)\ge 0}} {\mathbb{E} \int_0^{\infty} \frac{C(t)^{1-\theta}-1}{1-\theta} dt}{}{}
	\addConstraint{ C(t)}{ \ge 0} {\text{Non-negativity of consumption}}
	\addConstraint{ A'(t)}{ = \overbrace{r_t A(t)}^{\mathclap{\text{Returns from equity holdings}}} + \underbrace{\bar{w}_t(L_I(t) + L_F(t))}_{\mathclap{\text{Production labor earnings}}} - \overbrace{C(t)}^{\mathclap{\text{Consumption}}}} {\text{Asset law of motion}}
	\addConstraint{ }{+ \overbrace{\int_0^1 \big( \underbrace{w_{RD,jt}}_{\mathclap{\text{Wage}}} + \underbrace{(\frac{\bar{q}_{jt}}{Q_t})^{-1} \nu (1-\kappa_e) V(j,t|\lambda \bar{q}_{jt})}_{\mathclap{\text{WSO formation}}} \big) \underbrace{\ell_{RD,j}(t)}_{\mathclap{\text{Labor supply}}} dj}^{\mathclap{\text{Effective R\&D labor earnings incumbent j}}}}
	\addConstraint{ }{+ \int_0^1 \hat{w}_{RD,t} \hat{\ell}_{RD,j}(t) dj}
	\addConstraint{\lim_{t \to \infty} e^{-\int_0^{\infty} r_t dt} A(t) }{\ge 0}{\text{No-ponzi condition}}
	\addConstraint{A(0)}{ = A_0} {\text{Initial asset holdings}}
	\addConstraint{\int_0^1 (\ell_{RD,j}(t) + \hat{\ell}_{RD,j}(t))dj}{ \le \bar{L}_{RD}} {\text{R\&D labor endowment}}
	\addConstraint{L_I(t) + L_F(t)}{\le 1 - \bar{L}_{RD}} {\text{Production labor endowment}}
\end{maxi*}