\documentclass[12pt,english]{article}
%\usepackage{lmodern}
\linespread{1.05}
\usepackage{mathpazo}
%\usepackage{mathptmx}
%\usepackage{utopia}
\usepackage{microtype}
\usepackage[T1]{fontenc}
\usepackage[latin9]{inputenc}
\usepackage[dvipsnames]{xcolor}
\usepackage{geometry}
\usepackage{amsthm}
\usepackage{amsfonts}

\usepackage{courier}
\usepackage{verbatim}
\usepackage[round]{natbib}
\bibliographystyle{plainnat}


\definecolor{red1}{RGB}{128,0,0}
%\geometry{verbose,tmargin=1.25in,bmargin=1.25in,lmargin=1.25in,rmargin=1.25in}
\geometry{verbose,tmargin=1in,bmargin=1in,lmargin=1in,rmargin=1in}
\usepackage{setspace}

\usepackage[colorlinks=true, linkcolor={red!70!black}, citecolor={blue!50!black}, urlcolor={blue!80!black}]{hyperref}
%\usepackage{esint}
\onehalfspacing
\usepackage{babel}
\usepackage{amsmath}
\usepackage{graphicx}

\theoremstyle{remark}
\newtheorem{remark}{Remark}
\begin{document}
	
	
	
\title{Endogenous Growth with Spinouts - Progress Report and Issues}
\author{Nicolas Fernandez-Arias}
\maketitle

\section{Progress update}

\subsection{Spinouts dataset}

\begin{itemize}
	\item Coverage: US VC-funded spinouts, 1986-2014
	\item 12,010 parent firm - spinout pairs
	\item 3,483 unique parent firms
	\item 9,280 unique spinout firms
	\item See Figure \ref{TotalSpinoutCount}
\end{itemize}

\begin{figure}[h] \phantomsection 
	\centering
	\includegraphics[scale=0.43]{../empirics/code/plots/totalSpinoutsAndStartupsPerYear.png}
	\caption{Rapid decrease after 2012 in both series (though earlier for spinouts series). Could be due to typical lag between founding and first VC-funding (when a startup first appears in the dataset). I estimated the pattern of founding-funding lags and use that to "correct" the data, but it does not completely remove the decrease. Hence, some (potentially corner) combination of the following is true: (1) the dataset coverage is declining in recent years, (2) the speed with which startups receive VC-funding after being founded has decreased, (3) the number of startups founded each year which eventually receive VC funding is declining in recent years. For now I am planning on doing robustness checks to make sure this isn't affecting results.}
	\label{TotalSpinoutCount}
\end{figure}


\section{Main issues to be resolved}

\subsection{Theory}

\subsubsection{Negative R\&D wage}

\paragraph{Problem}

\begin{itemize}
	\item Model produces far too low R\&D wage, so can't match R\&D spending / sales -- negative for most of the state space, due to the high value of forming a spin out. Adding deadweight loss of creative destruction actually makes this *worse*, since it lowers entry by spinouts in general equilibrium. I think I can fix this by having spinouts not draw from a different pool of ideas, but I need to implement this.
	\item At the end of the day, this might be "unfixable" because workers in reality are risk averse but in the model risk-neutral...but adding worker risk aversion makes my model intractable.
	\item Target 50\% of R\&D - GDP ratio to calibrate, since approximately half of R\&D spending goes to wages
\end{itemize}

\paragraph{Possible solutions}

\begin{itemize}
	\item Employees value continuation payoff lower. Interpretation: have to raise money and give up a stake? But not really in the model.
	\item Deadweight loss of creative destruction -- DOES NOT WORK because reduces entry
\end{itemize}


\subsubsection{Incumbents not spending enough on R\&D}

\paragraph{Problem} 

\begin{itemize}
	\item Can do a decomposition of $\log\Big(\big(\frac{z_E(m)}{z_I(m)}\big)^{\psi}\Big)$ into the various sources of the differences, as
	\begin{align*}
		\Big(\frac{z_E(0)}{z_I(0)}\Big)^{\psi} &= \Bigg(1-\kappa \Bigg) \times \Bigg(\frac{1}{1-\psi} \Bigg) \times \Bigg( \frac{\chi_E}{\chi_I} \Bigg) \times \Bigg( \frac{\lambda}{\lambda-1} \Bigg) \times \Bigg( \frac{w_I}{w_E} \Bigg) 
	\end{align*}
	
	For general $m$, have
	\begin{align*}
	\Big(\frac{z_E(m) + z_S(m)}{z_I(m)}\Big)^{\psi} &=  \Bigg(1-\kappa \Bigg) \times \Bigg(\frac{1}{1-\psi} \Bigg) \times \Bigg( \frac{\chi_E}{\chi_I} \Bigg) \times \Bigg( \frac{\lambda V(0)}{\lambda V(0) - V(m)} \Bigg) \times \Bigg( \frac{w_I^{\textrm{effective}}(m)}{w_E(m)} \Bigg) 
	\end{align*}
	
	In turn, have
	\begin{align*}
		\frac{w_I^{\textrm{effective}}(m)}{w_E(m)} &= \frac{w_I(m) - \nu V'(m)}{w_E(m)} \\
								  				   &= \frac{w_I(m)}{w_E(m)} \times \frac{w_I(m) - \nu V'(m)}{w_I(m)} \\
								  				   &= \frac{\overline{w} - \nu^I W(m)}{\overline{w} - \nu^E W(m)}  \times \frac{w_I(m) - \nu V'(m)}{\overline{w} - \nu^I W(m)}
 	\end{align*}
 	
 	Also,
 	\begin{align*}
 		\frac{\lambda V(0)}{\lambda V(0) - V(m)} &= \frac{\lambda V(0)}{\lambda V(0) - V(0)} \times \frac{\lambda V(0) - V(0)}{\lambda V(0) - V(m)} \\
 		                            &= \frac{\lambda}{\lambda - 1} \times \frac{\lambda V(0) - V(0)}{\lambda V(0) - V(m)}
 	\end{align*}
 	
 	Putting it all together we get (for $m \le M^*$)
 	\begin{align*}
	\Bigg(\frac{z_E(m) + z_S(m)}{z_I(m)}\Bigg)^{\psi} &=  \overbrace{\Bigg(1-\kappa \Bigg) }^{\textrm{DW Loss from creative destruction}}\times \overbrace{\Bigg(\frac{1}{1-\psi} \Bigg)}^{\textrm{DRS in R\&D}} \times \overbrace{\Bigg( \frac{\chi_E}{\chi_I} \Bigg) }^{\textrm{Relative prod.}}\times \overbrace{\Bigg( \frac{\lambda}{\lambda - 1}  \Bigg) }^{\textrm{Business-stealing}} \\ \times \underbrace{\Bigg( \frac{\lambda V(0) - V(0)}{\lambda V(0) - V(m)}\Bigg)}_{\textrm{"Escape competition"}}
	          &\times \underbrace{\Bigg( \frac{\overline{w} - \nu^I W(m)}{\overline{w} - \nu^E W(m)}\Bigg)}_{\textrm{Nominal wage difference}} \times \underbrace{\Bigg( \frac{w_I(m) - \nu V'(m)}{\overline{w} - \nu^I W(m)} \Bigg)}_{\textrm{Cannibalization by spinouts}}
 	\end{align*}
 	
 	Note though that the model spends most of time at $m << M^*$, so to a reasonable approximation, the issue I am having is related to the ratio $z_E(0) / z_E(0)$. 
 	
 	An analogous decomposition can be used for spending on R\&D by entrants/spinouts vs. incumbents. For now, using $w_E(m) \approx w_s(m)$ (this can easily be improved but will do later), get 
 	\begin{align*}
 		\Bigg(\frac{z_E(m)w_E(m) + z_S(m)w_S(m)}{z_I(m) w_I(m)}\Bigg)^{\psi} &\approx \Bigg(\frac{(z_E(m) + z_S(m)) w_E(m) }{z_I(m) w_I(m)}\Bigg)^{\psi} \\  
 		&= \Bigg(  \frac{z_E(m) + z_S(m)}{z_I(m)}\Bigg)^{\psi} \Bigg( \frac{w_E(m)}{w_I(m)}\Bigg)^{\psi}
 	\end{align*}
 	so that, using the previous decomposition, it follows that we can decompose the ratio of R\&D wage bills with 
 	\begin{align*}
 	 \Bigg(\frac{z_E(m)w_E(m) + z_S(m)w_S(m)}{z_I(m) w_I(m)}\Bigg)^{\psi}  &\approx \overbrace{\Bigg(1-\kappa \Bigg) }^{\textrm{DW Loss from creative destruction}}\times  \overbrace{\Bigg(\frac{1}{1-\psi} \Bigg)}^{\textrm{DRS in R\&D}} \times \overbrace{\Bigg( \frac{\chi_E}{\chi_I} \Bigg) }^{\textrm{Relative prod.}} \\ \times \underbrace{\Bigg( \frac{\lambda}{\lambda - 1}  \Bigg) }_{\textrm{Business-stealing}} \times \underbrace{\Bigg( \frac{\lambda V(0) - V(0)}{\lambda V(0) - V(m)}\Bigg)}_{\textrm{"Escape competition"}}
 	&\times \underbrace{\Bigg( \frac{\overline{w} - \nu^I W(m)}{\overline{w} - \nu^E W(m)}\Bigg)^{1-\psi}}_{\textrm{Nominal wage difference}} \times \underbrace{\Bigg( \frac{w_I(m) - \nu V'(m)}{\overline{w} - \nu^I W(m)} \Bigg)}_{\textrm{Cannibalization by spinouts}}
 	\end{align*}
 		
 	The only difference is that now the ratio of nominal wages has an exponent $1-\psi$ on it: raising the incumbent wage diminishes the incumbent wage bill by less than it diminishes the incumbent R\&D effort, since the wage bill is the product of wage and R\&D effort. Intuitive. 	
 	
	The key elements of this decomposition are:
	\begin{itemize}
		\item The first term comes from the fact that incumbents take into account their effect on decreasing returns to R\&D, while entrants and spinouts do not. With an R\&D elasticity of $0.5$, this leads to a factor of 2. 
		\item The second term is the relative productivity term. Self-explanatory. 
		\item $\frac{\lambda}{\lambda-1}$: incumbents have no business-stealing incentive. This is about 20.. With $\psi = 0.5$, this alone accounts for a entrants doing 400 times more innovation than incumbents!!
		\item The fourth 
		\item $\frac{w_I}{w_E}$: incumbent pays a different wage since employees learn faster from them. For most calibrations I'm looking for, this doesnt' make a big difference.
		\item The last term -- cannibalization from spinouts -- does play a significant difference - can be a factor of approx 10 for calibrations i've explored. 
	\end{itemize}

	\begin{remark}
		Exploiting the decomposition to improve the fit of the model is possible, but not trivial. E.g., the decomposition suggests increasing $\lambda$ while decreasing all $\chi$ variables would keep growth steady while reorienting innovation behavior towards incumbents, but it also changes $V$ and particularly $W$, which could affect other terms in the decomposition. 
	\end{remark}

	\begin{remark}
		I care about \emph{both} of these decompositions. The wage bill is important for matching the data on R\&D spending / sales ratio, and the R\&D effort is important for matching the rate of spinout formation by incumbents
	\end{remark}
\end{itemize}



\paragraph{Solution}

The most promising avenues for fixing this are: 
\begin{enumerate}
	\item Adding spinouts which do not compete with parent firms. This reduces the contribution of the last term (without affecting the nominal wage difference).
	\item Increasing $\lambda$ while decreasing all other innovation productivity parameters to keep $g$ and total spending constant
	\item Increasing rate of drift of $m$ so that more time is spent at higher $m$, increasing the "escape competition" effect
	\item Decreasing the ratio of $\chi_E / \chi_E$ further. 
\end{enumerate}



\subsubsection{Non-competes in the model}

\begin{itemize}
	\item As we have discussed before, my model may not be a good model of \textit{why} non-competes are signed. Hence, the model will not do a good job predicting which employment contracts will be bound by non-competes if I add optimal contracting to the model. 
	\item My idea: as in Baslandze 2019, posit increased deadweight loss of forming spinouts in states where non-competes are enforced more heavily, without modeling the contracting problem. I will simply cite evidence from Starr 2019 ("Noncompetes in the U.S. Labor force") that shows that non-competes are prevalent among highly skilled workers. 
	\item I will validate the model by comparing its predictions across enforcement regimes anyway...so shouldn't be a big deal to use this "unrealistic" assumption, no?
\end{itemize}

\subsection{Empirics}

\subsubsection{R\&D and spinout formation}

\begin{enumerate}
	\item Finding some results that suggest R\&D spending leads to spinout formation, even with Parent firm and Naics4-year fixed effects, and appropriate clustering. 
	\item Patent counts and patent applications also predict spinouts
	\item One standard deviation change in R\&D spending has larger effect on expected spinout formation than one standard deviation change in other control variables, though close for some
	\item Some lingering details on these regressions: OLS vs Poisson vs Negative Binomial -- need to implement the alternatives carefully
\end{enumerate}

\subsubsection{Non-compete enforcement}

\begin{enumerate}
	\item Generally, not finding significant results -- appears to be due to low economic significance rather than low statistical power...
\end{enumerate}

\subsubsection{Effect of spinout financing deals on parent firm market value}

\begin{enumerate}
	\item Same as before -- appears to be due to low significance, not low power.
	\item Still no results even when using \textit{abnormal} returns, based on Fama-French 3-factor model.
\end{enumerate}



\end{document}




