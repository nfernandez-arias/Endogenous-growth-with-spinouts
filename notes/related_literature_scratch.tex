This occurs because NCAs increase incentives for own-product innovation, mitigating the equilibrium overallocation of R\&D spending to creative destruction. This misallocation is due chiefly to two negative externalities associated with creative destruction. First, and most importantly, is the business-stealing externality. Creative destruction steals profits from existing firms; therefore, for a given amount of productivity growth generated, creative destruction has a higher private return than own-product innovation. As private returns must be equated in equilibrium, this implies that the marginal growth impact of a unit of R\&D labor allocated to creative destruction is lower. Second, there is a congestion externality: firms engaging in creative destruction do not coordinate with each other and therefore inefficiently duplicate research efforts. 


\cite{franco_covenants_2008} studies a two-period, two-region model with employee spinouts in which the region which does not enforce NCAs initially lags but eventually overtakes the region in which NCAs are enforced. The model formalizes the common narrative -- present in the literature since \cite{saxenian_regional_1994} -- that Silicon Valley in California overtook Route 128 in Massachusetts as the principal innovation hub due to California's extreme lack of enforcement of NCAs. By contrast, this paper instead considers a fully dynamic model in which today's spinouts are tomorrow's incumbents. In addition, it emphasizes the role of R\&D investment in spawning spinout firms and the consequent disincentive to R\&D posed by spinout formation. However, I do not study a model with multiple regions with varying degrees of NCA enforcement.

\cite{shi_restrictions_2018} uses a rich model of contracting disciplined by data on executive noncompete agreements to study the effect of noncompetes on executive mobility and firm investment. She finds that the optimal policy is to somewhat restrict the permitted duration of NCAs. Her approach allows her to study the optimal contracting problem in more detail than in mine. However, she is mainly interested in an environment where the firm's productivity is embodied in the worker and where the parent firm's concern is poaching of their key employee by a rival firm rather than creative destruction by an employee spinout. In addition, she focuses on firm investment in capital expenditures, whereas I focus on R\&D investment. Finally she performs her analysis in a partial equilibrium framework while I consider a fully specified general equilibrium model.