I consider a specification where all variables are divided by a trailing five-year moving average of firm assets, which measures the size of the firm. This amounts to the specification
\begin{align}
	\tilde{WSO}_{it} &= \beta \tilde{R\&D}_{it} + \gamma \tilde{X}_{it} + \alpha_{i} + \xi_{j(i)t} + \sigma_{s(i)t} + \eta_{a(i,t)} + \epsilon_{it},
\end{align}
where a $\tilde{}$ superscript denotes that the variable is normalized by a trailing five-year moving average of assets. The result is displayed in the second column of Table \ref{table:RDandSpinoutFormation_headlingRegs}. The estimate for the coefficient on R\&D spending is significant at the 1\% level and indistinguishable from the previous estimate. To estimate this regression, I use WLS weighting by firm level assets. The purpose of this is on efficiency and consistency grounds. To the extent that large firms conduct various statistically independent R\&D projects, the errors will be smaller for those firms; WLS weighting by assets therefore improves efficiency. Second, because I am interested in the aggregate implications of this mechanism, in the event that the effects are heterogeneous across firms it is preferable for the estimate to be driven by firms with more assets (which conduct more R\&D all else equal, by construction).\footnote{The estimates from OLS (unreported) are larger. Consistent with the efficiency argument, the standard errors are four times larger than when estimating with WLS.}