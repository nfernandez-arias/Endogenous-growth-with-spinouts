\documentclass[11pt,english]{article}
\usepackage{palatino}
\usepackage[T1]{fontenc}
\usepackage[latin9]{inputenc}
\usepackage{geometry}
\usepackage{amsthm}
\usepackage{courier}
\usepackage{verbatim}
\geometry{verbose,tmargin=1in,bmargin=1in,lmargin=1in,rmargin=1in}
\usepackage{setspace}
%\usepackage{esint}
\onehalfspacing
\usepackage{babel}
\usepackage{amsmath}

\theoremstyle{remark}
\newtheorem*{remark}{Remark}
\begin{document}
	
\title{Aggregate constant returns to scale entrant and spinout innovation technology}
\author{Nicolas Fernandez-Arias}
\maketitle

\section{Background}

In AK 2017, entrants -- in the aggregate, not just individually -- are assumed to have constant returns to scale in their innovation production function. This adds tractability, since it imposes a simple condition relating the wage to the value of incumbency, conditional on there being any amount of entry.\footnote{For spinouts, which do not have free entry, the analogous condition is an inequality.} Some questions:

\begin{enumerate}
	\item Does this actually simplify the model?
	\item Would it improve the realism / fit of my model?
	\item Can this be used in my paper?
\end{enumerate}

\section{Analysis}

I believe the answer to (3) above is \textbf{no}. Therefore, (1) and (2) are not really applicable (though still interesting to think about).

The free entry condition for entrants in this case would be something like
\begin{align*}
	\chi_E \lambda V(0) &= w(m)
\end{align*}

for $m \le M$, where $M$ is the point at which ordinary entrants stop entering. This implies that, in equilibrium, 
\begin{align*}
	w(m) &\equiv \hat{w}
\end{align*}

for $m \le M$. Unless
\begin{align*}
	W(m) &\equiv \hat{W}
\end{align*}

this implies that R\&D labor supply will be zero for $m \in [0,M]$ such that 
\begin{align*}
	W(m) < \max_{m' \in [0,M]} W(m')
\end{align*}

But by the Inada conditions on incumbent R\&D productivity, at a finite wage $w(m)$, incumbents will demand some R\&D labor. Hence there is no equilibrium (at least not a BGP) with entry by ordinary entrants. 

Next, I will argue that $W(m)$ cannot be constant for $m < M$ in equilibrium. Suppose that it is constant. Then $w(m)$ is also constant. The HJB is therefore
\begin{align*}
	(\rho + \tau(m)) W(m) &= \overbrace{\nu a(m) W'(m)}^{= 0}+ \overbrace{\frac{z_S(m)}{m}}^{= \xi} \Big( \chi_S  \lambda V(0) - \overbrace{w(m)}^{=\hat{w}} \Big) \\
	                      &= \xi \Big( \chi_S \lambda V(0) - \hat{w} \Big)
\end{align*}

Hence, unless $\tau(m)$ is constant, we arrive at a contradiction. Well, $\tau(m)$ is given by 
\begin{align*}
	\tau(m) &= z_E(m)\chi_E + z_S(m) \chi_S  \\
	        &= z_E(m) \chi_E + \xi m \chi_S
\end{align*}

The constant returns to scale assumption does not pin down $z_E(m)$, hence it is free to vary to make $\tau$ constant. So we are left with:

\begin{align*}
	\hat{\tau} &= z_S(M) \chi_S 
\end{align*}

So I was wrong - it \textbf{is} possible!

\section{Equilibrium with constant returns to scale}







\end{document}