\documentclass[12pt,english]{article}
\usepackage{palatino}
\usepackage[T1]{fontenc}
\usepackage[latin9]{inputenc}
\usepackage{geometry}
\usepackage{amsthm}
\usepackage{courier}
\usepackage{verbatim}
\geometry{verbose,tmargin=1in,bmargin=1in,lmargin=1in,rmargin=1in}
\usepackage{setspace}
%\usepackage{esint}
\onehalfspacing
\usepackage{babel}
\usepackage{amsmath}
\usepackage{graphicx}

\theoremstyle{remark}
\newtheorem*{remark}{Remark}
\begin{document}
	
\title{Endogenous Growth with Creative Destruction by Employee Spinouts}
\author{Nicolas Fernandez-Arias}
\maketitle

if there is bilateral surplus generated by the expected learning of the employee, some will go to the parent firm in equilibrium through a lower wage, strengthening the incentive to innovate. This occurs if, for example, 

(explain tradeoff: firm / worker engage in bilaterally optimal contract, this contract may not be socially optimal because it monopolizes the existing knowledge. But for the same reason, it could be socially optimal, because it could encourage incumbent innovation by strengthening their monopoly on knowledge.

\paragraph{Old stuff}

\paragraph{Why non-constant hazard rate}

\begin{enumerate}
	\item \textbf{Introspection:} makes sense to think that as you leak knowledge, there is an increasing hazard rate of it being used against you
	\item \textbf{Comparative statics:} Without this, there is no way to incorporate into the model the ``escape competition effect'', as in Aghion et al. 2005, "Competition and Innovatin: An Inverted-U Relationship". Although, my model may still be able to generate the inverted-U for different reasons
	\item \textbf{Problem: }how to find this in the data? Well, the obvious way would be to somehow measure the ``time since the last innovation'', which in equilibrium, is correlated with the hazard rate of their being an innovation.
	\item Need to think a bit how to do this -- basically though this adds one more important fact that I need to confirm in the data for my model analysis to be valid
	\item \textbf{Natural idea: }look at patent data
	\item \textbf{Problems}
	\begin{itemize}
		\item Cross-section: lots of cross-sectional heterogeneity that could mean that there is a negative correlation between time since last innovation and hazard rate. I.e., heterogeneity in technological opportunities. Can this be added to my model? Maybe?
		\item Other idea: look at how Aghion et al. 2005 does it. What data do they have that I don't have that allows them to answer this question re: incumbents, not incumbents vs. spinouts?
	\end{itemize}
\end{enumerate}

\paragraph{Difficulty identifying increasing hazard rate}

Let $t_{j,s}$ denote the time since the last innovation, for product-line $\times$ level observation $j$.

First, how do we even observe $t_{j,s}$? Not clear. Probably using patent data.

Secondly, suppose that the hazard rate is $h_j(t) = z_j h(t)$ for some product-line $\times$ level-specific innovation rate $z_j$. The empirical hazard rate is
\begin{align*}
P[t_{j,s+1} = 0|t_{js}] &= E[z_j h(t_{js}) | t_{js}] \\ 
&= E[z_j | t_{js}] h(t_{js})
\end{align*}

The problem: $E[z_j | t_{js}]$ is decreasing in $t_{js}$ - high hazard-rate products are likely to not last a long time between innovations. This makes it difficult to observe a correlation between time since last innovation and hazard rate. 

The only solution is to somehow measure $z_j$, or at least, find some way to argue that $z_j \approx z_{j'}$.  Not sure how to do this.

One other issue: there could easily be other mechanisms creating an increasing hazard rate, i.e. that it takes time to come up with innovations. 

The real arrival rate is not a Poisson process...to control for this, I would want to look also at R\&D expenditures as a function of how long ago the last innovation was. But again, there could be other variation: certain firms might realize that they don't have an edge innovating on something, and so on. There's just so much variation in the data, it's hard to know what's actually what without a natural experiment...

Maybe the solution is to look at a specific industry to find this. Need to look at the literature. E.g., you might think that drug discovery is something like a Poisson process, because it's so unpredictable. Then it's just a matter of controlling for $z_j$. A crude way to do this would be to proxy $z_j$ with how valuable the patent is once it's invented, assuming that more valuable innovations are more difficult. Measure value using patent citations or something. 

MAKE QUESTION I AM ANSWERING MORE PRECISE!!
WHAT QUESTION AM I TRYING TO ANSWER!!?

\end{document}