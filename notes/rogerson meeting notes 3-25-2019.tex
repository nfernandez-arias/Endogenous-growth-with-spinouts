\documentclass[12pt,english]{article}
%\usepackage{lmodern}
\linespread{1.05}
\usepackage{mathpazo}
%\usepackage{mathptmx}
%\usepackage{utopia}
\usepackage{microtype}
\usepackage[T1]{fontenc}
\usepackage[latin9]{inputenc}
\usepackage[dvipsnames]{xcolor}
\usepackage{geometry}
\usepackage{amsthm}
\usepackage{amsfonts}
\usepackage{courier}
\usepackage{verbatim}
\usepackage[round]{natbib}
\bibliographystyle{plainnat}


\definecolor{red1}{RGB}{255,100,200}
\geometry{verbose,tmargin=1in,bmargin=1in,lmargin=1in,rmargin=1in}
\usepackage{setspace}



\usepackage[colorlinks=true, linkcolor={red!70!black}, citecolor={blue!50!black}, urlcolor={blue!80!black}]{hyperref}
%\usepackage{esint}
\onehalfspacing
\usepackage{babel}
\usepackage{amsmath}
\usepackage{graphicx}

\theoremstyle{remark}
\newtheorem{remark}{Remark}
\begin{document}
	
\title{Rogerson meeting notes 3-25-2019}
\author{Nicolas Fernandez-Arias}
\maketitle

\begin{itemize}
	\item What is exact question I want to answer?
	\item Issue of contracting between firm and employee is central to how enforcement affects outcomes. So need to argue that the contracting in my model is realistic. Since won't be allowing completely free contracting, need to argue that friction in reality are such that contracting ends up similar to what is in the model.
	\item Project may evolve into something slightly different focusing on the empirics etc
\end{itemize}





\end{document}