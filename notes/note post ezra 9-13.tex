\documentclass[12pt,english]{article}
%\usepackage{lmodern}
\linespread{1.05}
\usepackage{mathpazo}
%\usepackage{mathptmx}
%\usepackage{utopia}
\usepackage{microtype}
\usepackage[T1]{fontenc}
\usepackage[latin9]{inputenc}
\usepackage[dvipsnames]{xcolor}
\usepackage{geometry}
\usepackage{amsthm}
\usepackage{amsfonts}

\usepackage{courier}
\usepackage{verbatim}
\usepackage[round]{natbib}
\bibliographystyle{plainnat}


\definecolor{red1}{RGB}{128,0,0}
%\geometry{verbose,tmargin=1.25in,bmargin=1.25in,lmargin=1.25in,rmargin=1.25in}
\geometry{verbose,tmargin=1in,bmargin=1in,lmargin=1in,rmargin=1in}
\usepackage{setspace}

\usepackage[colorlinks=true, linkcolor={red!70!black}, citecolor={blue!50!black}, urlcolor={blue!80!black}]{hyperref}
%\usepackage{esint}
\onehalfspacing
\usepackage{babel}
\usepackage{amsmath}
\usepackage{graphicx}

\theoremstyle{remark}
\newtheorem{remark}{Remark}
\begin{document}
	
\title{Notes post Ezra Meeting 9/13/19}
\author{Nicolas Fernandez-Arias}
\maketitle

\begin{itemize}
	\item Work out stuff re: production functions. There should be an equivalence 
	\begin{itemize}
		\item Still confused about this. Changing the exponent on $q_j(t)$ is not just a change of units, because $q_j(t)$ appears in other places. Think about this more, do the exercise in Acemoglu's book (or check the solutions manual, actually, no time for that shit).
		\item Also isn't there a place in the textbook where Acemoglu literally *says* that making a modification along these lines creates firm size dynamics? I.e., does more than change the step size, even allowing the calibration to vary? Think about this.
	\end{itemize}
	\item Ezra agrees about the equivalence of R\&D directly leading to spinouts and the PRODUCTS of R\&D leading to spinouts. Agrees that this is part of why the wage is super low. 
	\begin{itemize}
		\item However, still better to have flexibility that spinouts are not related to productivity improvements whatsoever. 
		\item How to separate these two sources of spinouts? If just look at R\&D regression, miss interpretation of R\&D as a reduced form for the fact that both knowledge inputs and knowledge outputs lead to more spinouts. Could I maybe run the regression using R\&D *and* patents and invert the regression equation to see what fraction of spinouts result from knowledge, and match *this* to the model? 
		\item Ezra suggested using enforcement changes as an instrument. But doesn't this fail the exclusion restriction because enforcement changes also change the number of spinouts independent of R\&D spending.
	\end{itemize}
	\item Agrees re: reinterpretation of model as products not firms. But then need (1) product entry rate, and (2) relative likelihood of spinout and regular entrant products will successfully enter. 
	\begin{itemize}
		\item For (1) talk to Steve Redding, he might have some insight
		\item For (2), compute ratio in likelihood that a regular entering firm and a spinout firm end up getting to "generating revenue" stage or above. (can try similar stuff with IPO, too, but this is a better baseline). This has the implicit assumption that spinouts and entering firms have the same number of employees per attempted product; e.g., if spinouts and entering firms are both the same size and the same number of products. 
	\end{itemize}
	\item Agrees re: R\&D reflecting only "own" innovation as well as only money spent on employees, so not as much of a miss. Agrees that this requires the reinterpretation of firms as products.
	\begin{itemize}
		\item Can maybe try to roughly proxy the fraction of R\&D spending inside and outside the firm based on the average fraction of a firm's patents which cite mostly patents not owned by the firm. Implicitly assumes that they have the same innovation production function for internal and external innovation, but since I have no way of estimating this directly, it is the most reasonable approach.
		\item Maybe worth trying to talk to Ufuk about this...although maybe it's too late at this point and looks bad, honestly. But he is the right person to ask. 
	\end{itemize}
	\item David Argente of Penn State has a paper on Patents and Products
	\item Ignacio Cuesta
	\item Oberfield Buera - do robustness checks, see this for inspiration
\end{itemize}




\end{document}






\end{document}