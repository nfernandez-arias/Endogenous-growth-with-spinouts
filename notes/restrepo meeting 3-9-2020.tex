\documentclass[12pt,english]{article}
\usepackage{lmodern}
\linespread{1.05}
%\usepackage{mathpazo}
%\usepackage{mathptmx}
%\usepackage{utopia}
\usepackage{microtype}
\usepackage[section]{placeins}
\usepackage[T1]{fontenc}
\usepackage[latin9]{inputenc}
\usepackage[dvipsnames]{xcolor}
\usepackage{geometry}
\usepackage{amsthm}
\usepackage{amsfonts}
\usepackage{svg}
\usepackage{booktabs}
\usepackage{caption}
\usepackage{blindtext}
%\renewcommand{\arraystretch}{1.2}
\usepackage{multirow}
\usepackage{float}
\usepackage{rotating}

\usepackage{chngcntr}

% TikZ stuff

\usepackage{tikz}
\usepackage{mathdots}
\usepackage{yhmath}
\usepackage{cancel}
\usepackage{color}
\usepackage{siunitx}
\usepackage{array}
\usepackage{amssymb}
\usepackage{gensymb}
\usepackage{tabularx}
\usetikzlibrary{fadings}
\usetikzlibrary{patterns}
\usetikzlibrary{shadows.blur}

\usepackage[font=small]{caption}
%\usepackage[printfigures]{figcaps}
%\usepackage[nomarkers]{endfloat}


%\usepackage{caption}
%\captionsetup{justification=raggedright,singlelinecheck=false}

\usepackage{courier}
\usepackage{verbatim}
\usepackage[round]{natbib}
\bibliographystyle{plainnat}

\definecolor{red1}{RGB}{128,0,0}
%\geometry{verbose,tmargin=1.25in,bmargin=1.25in,lmargin=1.25in,rmargin=1.25in}
\geometry{verbose,tmargin=1in,bmargin=1in,lmargin=1in,rmargin=1in}
\usepackage{setspace}

\usepackage[colorlinks=true, linkcolor={red!70!black}, citecolor={blue!50!black}, urlcolor={blue!80!black}]{hyperref}
%\usepackage{esint}
\onehalfspacing
\usepackage{babel}
\usepackage{amsmath}
\usepackage{graphicx}

\theoremstyle{remark}
\newtheorem{remark}{Remark}
\begin{document}
	
\title{Restrepo meeting}
\author{Nicolas Fernandez-Arias}
\date{\today}


\begin{enumerate}
	\item Look across industries to see if non-competes more or less common. Use model to estimate $\kappa_j$ by industry (by using industry R\&D spending coupled with industry entry rate, I guess). Test prediction that non-competes will be more commonly used in industries with higher $\kappa_j$. Assume that there is a cost to using a non-compete (proportional to $q$) so that not all firms do it; and industry-specific $\kappa_j$ then makes it all work out
	\item Think about model with ideas generated (as in Esteban's paper) but where the ideas vary along two dimensions: how high their quality is, how easily implementable they are by the parent firm, and how directly they compete with the parent firm. Non-competes can then help provide incentives to innovate by making it easier to screen out ideas which ex-ante do not improve the parent-firm match. These are ideas which compete directly with the parent firm, which are more costly for the entrant to implement, and which are not too high quality. 
	\item Do this stuff *after* getting my current version up and running with calibration, draft written, shown it to Rogerson / Violante / Esteban, etc.
\end{enumerate}













\end{document}

