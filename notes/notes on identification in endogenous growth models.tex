\documentclass[12pt,english]{article}
\usepackage{lmodern}
\usepackage[T1]{fontenc}
\usepackage[latin9]{inputenc}
\usepackage{geometry}
\usepackage{amsthm}
\usepackage{courier}
\usepackage{verbatim}
\geometry{verbose,tmargin=1in,bmargin=1in,lmargin=1in,rmargin=1in}
\usepackage{setspace}
%\usepackage{esint}
\onehalfspacing
\usepackage{babel}
\usepackage{amsmath}

\theoremstyle{remark}
\newtheorem*{remark}{Remark}
\begin{document}
	
\title{Identification and importance of step size $\lambda$}
\author{Nicolas Fernandez-Arias}
\maketitle

\section{Introduction}

We want to calibrate the model in part using moments like ``the fraction of innovations performed by incumbents on their own products", which is ~20\% in the patent data. This therefore requires:

\begin{enumerate}
	\item Low $\lambda$, so that only entrants perceive large benefit to innovation
	\item Not much productivity advantage in R\&D of incumbents vs entrants (they still need some advantage in order to do any innovation in equilibrium)
\end{enumerate}

Given step size $\lambda$, growth rate is determined by frequency of arrivals. Hence, with small $\lambda$, arrivals have to be frequent. 

Hence, if we also have data on typical length of time between innovations, we could determine which case we are in.

In models where profit margins are determined by step size, then that is a direct way to identify. But this is not true in my model / basic creative destruction model. Why? Seems weird to have a model where this is not the case..

What are the implications of these different calibrations? 

More broadly: how important is it for my model to be reasonable to allow spinouts to have a different step size / how important is it to give them a different pool of ideas they draw from? Could I use the distribution of VC valuations to infer the distribution of idea sizes and see how they differ between entrants and spinouts? idk...






 






\end{document}