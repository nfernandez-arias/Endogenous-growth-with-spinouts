\documentclass[12pt,english]{article}
\usepackage{lmodern}
\usepackage[T1]{fontenc}
\usepackage[latin9]{inputenc}
\usepackage{geometry}
\usepackage{amsthm}
\usepackage{courier}
\usepackage{verbatim}
\geometry{verbose,tmargin=1in,bmargin=1in,lmargin=1in,rmargin=1in}
\usepackage{setspace}
%\usepackage{esint}
\onehalfspacing
\usepackage{babel}
\usepackage{amsmath}

\theoremstyle{remark}
\newtheorem*{remark}{Remark}
\begin{document}
	
\title{Notes on model}
\author{Nicolas Fernandez-Arias}
\maketitle

\section{Non-monotonicity of effective R\&D wage}

The effective R\&D wage is defined as 
\begin{align*}
	w_{eff}(m) &= w(m) - \nu V'(m)
\end{align*}
starts out below the production wage, is increasing for low $m$, then switches to decreasing, until reaching the final goods wage for large $m$.

\paragraph{Explanation} There is a kink in the effective R\&D wage at the point where regular entrants stop exiting to compensate for the entering spinouts. This in turn arises from a kink in $V'(m)$. There is no corresponding kink in $W(m)$. 


\section{Final goods labor share related to sales / profit ratio}








\end{document}