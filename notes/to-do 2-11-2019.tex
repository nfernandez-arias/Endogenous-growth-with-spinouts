\documentclass[11pt,english]{article}
\usepackage{palatino}
\usepackage[T1]{fontenc}
\usepackage[latin9]{inputenc}
\usepackage{geometry}
\usepackage{amsthm}
\usepackage{courier}
\usepackage{verbatim}
\geometry{verbose,tmargin=1in,bmargin=1in,lmargin=1in,rmargin=1in}
\usepackage{setspace}
%\usepackage{esint}
\onehalfspacing
\usepackage{babel}
\usepackage{amsmath}

\theoremstyle{remark}
\newtheorem*{remark}{Remark}
\begin{document}
	
\title{To-do list Endogenous growth w spinouts}
\author{Nicolas Fernandez-Arias}
\maketitle


\begin{enumerate}
	\item Think about tractability added when entrant and spinout returns are constant returns to scale at the aggregate level (as in AK 2017). Can talk to Akcigit about the differences.
	\item Think about the fact that there is a firm-specific sensitivity to non-compete enforcement shocks based on the location of the R\&D of the firm across states. This gives lots of data points which could be used to estimate something. Essentially it's like comparing firms inside states that change to firms outside them. And this can be used for any outcome variable. Is there something here or no?
\end{enumerate}




\end{document}