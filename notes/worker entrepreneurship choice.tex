\documentclass[12pt,english]{article}
\usepackage{lmodern}
\usepackage[T1]{fontenc}
\usepackage[latin9]{inputenc}
\usepackage{geometry}
\usepackage{amsthm}
\usepackage{courier}
\usepackage{verbatim}
\geometry{verbose,tmargin=1in,bmargin=1in,lmargin=1in,rmargin=1in}
\usepackage{setspace}
%\usepackage{esint}
\onehalfspacing
\usepackage{babel}
\usepackage{amsmath}

\theoremstyle{remark}
\newtheorem*{remark}{Remark}
\begin{document}
	
\title{Addition of Worker choice into entrepreneurship}
\author{Nicolas Fernandez-Arias}
\maketitle

\section{Math}

Assume that workers can choose whether to work as entrepreneurs or not. 

First, I have to decide whether they can move frictionlessly into and out of entrepreneurship, or whether they choose to commit to working as an entrepreneur until some point - either until they decide once to ``quit'', or when there is a product innovation in that category.

I have done some thinking about the latter two cases, but the first case actually seems like the most reasonable starting point, so let's think about that first. 

First of all, in any case, a worker will only ever work at their lowest $m$, so a worker's state is one-dimensional. 

In the case that movement into and out of entrepreneurship is frictionless, a worker simply works as an entrepreneur when the flow payoff is higher than the highest flow payoff he can achieve is. Let $w^*$ denote this maximum flow payoff. Let $W(m)$ denote the value of a worker with best knowledge in state $m$. 




\end{document}