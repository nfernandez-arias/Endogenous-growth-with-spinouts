\documentclass[12pt,english]{article}
\usepackage{lmodern}
\linespread{1.05}
%\usepackage{mathpazo}
%\usepackage{mathptmx}
%\usepackage{utopia}
\usepackage{microtype}
\usepackage{placeins}
\usepackage[T1]{fontenc}
\usepackage[latin9]{inputenc}
\usepackage[dvipsnames]{xcolor}
\usepackage{geometry}
\usepackage{amsthm}
\usepackage{amsfonts}

\usepackage{booktabs}
\usepackage{caption}
\usepackage{blindtext}
%\renewcommand{\arraystretch}{1.2}
\usepackage{multirow}

%\usepackage{caption}
%\captionsetup{justification=raggedright,singlelinecheck=false}

\usepackage{courier}
\usepackage{verbatim}
\usepackage[round]{natbib}
\bibliographystyle{plainnat}

\definecolor{red1}{RGB}{128,0,0}
%\geometry{verbose,tmargin=1.25in,bmargin=1.25in,lmargin=1.25in,rmargin=1.25in}
\geometry{verbose,tmargin=1in,bmargin=1in,lmargin=1in,rmargin=1in}
\usepackage{setspace}

\usepackage[colorlinks=true, linkcolor={red!70!black}, citecolor={blue!50!black}, urlcolor={blue!80!black}]{hyperref}
%\usepackage{esint}
\onehalfspacing
\usepackage{babel}
\usepackage{amsmath}
\usepackage{graphicx}

\theoremstyle{remark}
\newtheorem{remark}{Remark}
\begin{document}
	
\title{Questions for Mark Aguiar}
\author{Nicolas Fernandez-Arias}
\maketitle

\begin{enumerate}
	\item Modeling: can I abstract from "stock" of spinouts $m$ to get a more tractable model? 
	\begin{itemize}
		\item Loses significant tractability
		\item Adds "escape competition" effect; seems reasonable to have this in there in a model about effects of increased competition on incumbent behavior...
		\item Can't measure $m$ directly, because industrial classification too broad / noisy in the data, so cannot test models prediction for joint distribution of $m$ and R\&D effort...
		\item Moreover, this "escape competition" effect is still much too weak to generate reasonable predictions for incumbent R\&D spending
		\item Finally, if I get rid of this, I can introduce heterogeneity in R\&D spending / sales, and test whether the model matches well the cross-sectional relationship between R\&D spending and spinout formation, look at the firm-size distribution, etc.
	\end{itemize}
	\item Identifying key economic parameters:
	\begin{itemize}
		\item Key parameters governing relevance of non-competes:
		\begin{itemize}
			\item Productivity gap spinouts vs ordinary entrants
			\begin{itemize}
				\item Typically identified based on relative entry rates / R\&D expenditures, but I don't observe R\&D expenditures for spinouts
				\item Absent R\&D expenditures, 
			\end{itemize}
			\item Gap between value to spinouts and harm to incumbents (e.g., due to increased competition, or simply deadweight loss of inefficient replication of complementary assets)
			\begin{itemize}
				\item Liu suggests modelling could get this -- just assume that upon spinout formation a duopoly is formed where no profits are earned until someone obtains a lead, after which they become the leader as before. I.e. abstract from follower once he loses. This is a middle-ground, keeps the model tractable, but imposes a competition penalty when spinouts enter. 
			\end{itemize}
			\item Fraction (innovtion-size weighted) of knowledge spillovers from R\&D that do not compete with parent firm (i.e. are not preventedby non-competes)
		\end{itemize}
	\end{itemize}
	\item Model validation
	\begin{itemize}
		\item Exogenous increase in NCA enforcement should raise R\&D spending
		\item Exogenous increase in NCA enforcement should attenuate relationship between corporate R\&D and WSO formation (but not formation of non-competing spinouts)
	\end{itemize}
	\item Empirical work
	\begin{itemize}
		\item How rigorous does it really need to be? E.g., causal estimates of effect of R\&D to spinout formation? Or can I just look at the relationship in the data and interpret it through the lens in the model?
		\item How to relate estimate to parameters in the model? 
	\end{itemize}
\end{enumerate}










\end{document}