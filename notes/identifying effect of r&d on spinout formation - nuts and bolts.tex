\documentclass[11pt,english]{article}
\usepackage{geometry}
\usepackage{amsthm}
\usepackage{courier}
\usepackage{verbatim}
\geometry{verbose,tmargin=1in,bmargin=1in,lmargin=1in,rmargin=1in}
\usepackage{setspace}
%\usepackage{esint}
\onehalfspacing
\usepackage{babel}
\usepackage{amsmath}

\usepackage{palatino}
\usepackage[T1]{fontenc}
\usepackage[latin9]{inputenc}

\theoremstyle{remark}
\newtheorem*{remark}{Remark}
\begin{document}
	
\title{Identifying the effect of incumbent R\&D on spinout formation}
\author{Nicolas Fernandez-Arias}
\maketitle

\section{Background}

Define $S_{ijlt}$ denote total number of new spinouts of firm $i$ in state $j$ in industry $l$ during year $t$. Similarly, let $E_{jlt}$ denote the total number of non-spinout entrants in state $j$ in industry $l$ during year $t$. Let $\rho_{jt}$ measure the effective subsidy to R\&D in state $j$ at time $t$, taken as exogenous. Let $RD_{ijt}$ denote the relevant RD stock at firm $i$ in state $j$ and time $t$.\footnote{E.g., weighted average of current and lagged state-level R\&D spending and/or employment.}. Let $RD_{i,-j,t}$ denote the same stock but outside of state $j$. Finally let $l(i)$ denote the industry / product-market of firm $i$. 

\paragraph{Note} Need to modify the below to take into account the fact that R\&D depends log-linearly on R\&D user cost.

\subsection{OLS}

Basic OLS would estimate
\begin{align*}
	S_{ijlt} &= \alpha_{ijl}^S + \beta RD_{ijt} + \gamma RD_{i,-j,t} + \epsilon^S_{ijlt}
\end{align*}

The problem is that 
\begin{align*}
	\epsilon^S_{ijlt} &= \varepsilon^S_{ijt} + \sigma^S_{lt} + \eta^S_{ijlt} \\
	RD_{ijt} &= \alpha_{ij}^{RD} + \varepsilon^{RD}_{jt} + \sigma^{RD}_{l(i)t} + \eta^{RD}_{ijlt}
\end{align*}

with $\varepsilon^S_{jt},\varepsilon^{RD}_{jt}$ and $\sigma^S_{lt},\sigma^{RD}_{lt}$ correlated. In words, there are state-time and industry-time shocks to innovation investment opportunities, and these shocks affect both incumbent R\&D and spinout formation in the same direction. This biases $\hat{\beta}_{OLS},\hat{\gamma}_{OLS}$ upwards.

\subsection{IV with federal level R\&D subsidies interacted with firm-specific variables}
The clearest solution is to instrument R\&D by the Federal tax incentive component of the firm-time-varying R\&D user cost $\rho_{it}^F$. As explained in Bloom et al. 2013, $\rho_{it}^F$ has a firm-time-specific component $\xi_{it}^F$ because what qualifies as R\&D for tax purposes depends on a firm-time-specific "base".

If $\xi_{it}^F$ is orthogonal to the time-varying user cost of R\&D for startups, the IV estimate is consistent. And it should be, since we have
\begin{align*}
	\rho_{it}^F &= \rho_{t}^F + \xi_{it}^F
\end{align*}

with $\mathbf{E}\Big[\xi_{it}^F \cdot \rho_t^F \Big] = 0$. Since the time-varying user cost of R\&D for startups is likely to be correlated if anything with $\rho_t^F$, the estimate should be consistent.

\paragraph{Minor caveat} While $\xi_{it}^F$ is likely uncorrelated with $\rho_t^F$, it may be predictable and manipulable by the firm. Could this be a problem? 

\subsection{IV with state-level R\&D subsidies}

It may be possible to perform a similar analysis using the state-level component of the firm-time R\&D user cost, $\rho_{ijt}$. This would be good as it would provide two alternative forms of identification which could be compared. 

Suppose we can decompose
\begin{align*}
	\varepsilon_{ijt}^{RD} &= \zeta_i^{RD} \rho_{jt} + \tilde{\varepsilon}^{RD}_{jt}
\end{align*}

Then a first-stage equation for R\&D is   
\begin{align*}
	RD_{ijt} &= \alpha_{ij}^{RD} + \zeta^{RD} \rho_{jt} + \tilde{\varepsilon}^{RD}_{jt} + \sigma^{RD}_{lt} + \eta^{RD}_{ijlt}
\end{align*}

But the exclusion restriction does not hold, because
\begin{align*}
	\varepsilon_{jt}^S &= \zeta_i^S \rho_{jt} + \tilde{\varepsilon}^S_{jt}
\end{align*}

i.e., $\rho_{jt}$ also affects spinout formation \textit{directly}. Since $\zeta^S$ is likely of the same sign as $\zeta^{RD}$, this biases $\hat{\beta}_{IV}$ upwards.

What about $\hat{\gamma}_{IV}$? If spinout formation in state $j$ does not depend directly on R\&D subsidies in states $-j$, then the exclusion restriction is satisfied. Moreover, even if spinout formation in $j$ is directly affected by beneficial policies in $-j$, it is likely through reallocation of spinouts from $j$ to $-j$. Hence, the exclusion restriction is not satisfied, but biases estimates downwards. Estimates give a lower bound. This leads to the first idea for robust identification, below.

\section{Identifying based on out-of-state spinouts}

This is simply a mathematical formalization of the logic in the previous paragraph.

\subsection{Idea}
The most obvious way is to consider spinout-related outcomes that are not directly affected by R\&D subsidies in state $j$. I.e. if the outcome is spinouts in states other than the one that received the shock. To the extent that this continues to violate the exclusion restriction, it biases estimates downwards -- since one would expect, all else equal, a reduction in the price of R\&D in state $j$ relative to states $-j$ would lead a higher proportion of spinouts ocurring in state $-j$. So it would give a lower bound on this. 

\subsection{Problem}
The problem is that there may not be much statistical power. 

Also, if I am interested in studying the effects of state-level non-compete enforcement, I should be measuring within-state outcomes. The other side of this coin, of course, is that I could measure how non-compete enforcement affects the fraction of spinouts founded in the state vs. out of the state.

\section{Diff-in-diff: identification based on within-state non-spinout entry rates}

The previous approach works in theory but suffers from two related drawbacks. 

First, it only gives us part of the picture. It can provide evidence that R\&D leads to spinouts, but does not provide a reliable estimate of the full magnitude of this effect without additional assumptions (e.g. that the sensitivity of out-of-state spinouts to R\&D is related to the sensitivity of total spinouts to R\&D by the same factor of proportinality as the relative fraction of out-of-state to total spinouts).

Second, if I am interested in exploring the consequences of non-compete enforcement policy, it becomes particularly important to have a reliable direct estimate of the within-state effect of R\&D spending on spinout formation, because this is presumably what will be most affected by the change in policy. Put another way, the ratio of the in-state and out-state coefficients will change with differing non-compete enforcement, so I need a direct measure of within-state sensitivity.

To instead obtain a direct estimate of the effect of R\&D on within-state spinouts, we need to normalize the outcome variable by something which is affected in the same way by the instrument $\rho_{jt}$. 

\subsection{Basic idea} 
The simplest version of this idea begins with the assumption that we may write 
\begin{align*}
	E_{jlt} &= \alpha_{jl}^E + \epsilon^E_{jlt} \\
	\epsilon^E_{jlt} &= \varepsilon^E_{jt} + \sigma^E_{lt} + \eta_{jlt}^E \\
	\varepsilon^E_{jt} &= \zeta^E \rho_{jt} + \tilde{\varepsilon}_{jt} \\
	\zeta^E &= \zeta^S
\end{align*}

If the above holds, then we can simply use $S_{ijlt} - E_{jlt}$ as an outcome variable. 

However, the above does not hold. Presumably, $\zeta^E >> \zeta^S$, since $\zeta^S$ is measuring the effect on spinouts from one particular firm, and there are many firms in the economy. 

\subsection{Potential solution: estimating $\zeta^E / \zeta^S$}

One solution to the above problem is to have an estimate of $\zeta^E$ and $\zeta^S$. 

For example, suppose that $\log E, \log S$ would have the same $\zeta$ coefficient in the analogous regression. That means that $\zeta^E = (\bar{E}/\bar{S}) \zeta^S$, where $\bar{E},\bar{S}$ are relevant averages. If we have an estimate of $\zeta^E / \zeta^S = \chi$, we can use $S_{ijlt} - \chi^{-1} E_{jlt}$ as an outcome variable. Direct dependence on $\rho_{jt}$ is purged. 



\subsection{Remaining issues}
Not robust to the possibility that, around the dates of the shocks, there are systematically differing trends in non-spinout and spinout entry rates. Made more likely by the fact that most of the shocks are in the same direction, so a constant differing trend could introduce this bias.


\section{Diff-in-diff-in-diff: diffing by out-of-state spinout/non-spinout entry gap diff}

\subsection{Basic idea}
Look at the gap between the outcome in the previous idea, and the analogous outcome in a bordering state. This adds robustness to the criticism above. A good robustness check. 


\section{Identification based on whether spinout is related to R\&D}

\subsection{Basic idea}
Using patent data, we can measure the extent to which a spinout builds on the technology of its parent firm. Then we can simply assume that these spinouts are **caused** by having R\&D employees (who then know the knowledge of the parent firm). We could even build a measure of how much more spinouts cite incumbent firm patents than similar entrants (how to define ``similar''?), which would help confirm that this is something that appears to result from spinouts. 

\subsection{Problem}
This is much more indirect.

\section{Identifying product market rivalry between spinout and parent firm}

This is a different question - when do we classify these things as competing, and do we distinguish between technology class and product class? Presumably, I am interested in spinouts that are members of both. 

Will likely do it using industry codes - need to construct a matching between industry codes and NAICS. Another possibility would be to get the product data, although this is incomplete it would probably be useful to do robustness with this method...













\end{document}