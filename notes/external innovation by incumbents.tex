\documentclass[12pt,english]{article}
\usepackage{lmodern}
\usepackage[T1]{fontenc}
\usepackage[latin9]{inputenc}
\usepackage{geometry}
\usepackage{amsthm}
\usepackage{courier}
\usepackage{verbatim}
\geometry{verbose,tmargin=1in,bmargin=1in,lmargin=1in,rmargin=1in}
\usepackage{setspace}
%\usepackage{esint}
\onehalfspacing
\usepackage{babel}
\usepackage{amsmath}

\theoremstyle{remark}
\newtheorem*{remark}{Remark}
\begin{document}
	
\title{Addition of external innovation by incumbent firms}
\author{Nicolas Fernandez-Arias}
\maketitle

\section{Introduction}

It would be nice to be able to use the fraction of firms which are spinouts in my calibration. Currently, however, this is impossible, since in my model all innovation by non-incumbents and non-spinouts is classified as entry of a new firm. This would massively overshoot things. Hence, I need a model with reasonable calibrations for external innovation, etc. Is this necessary? What am I missing by not having this?

Well, 





\end{document}