\documentclass[12pt,english]{article}
\usepackage{lmodern}
\usepackage[T1]{fontenc}
\usepackage[latin9]{inputenc}
\usepackage{geometry}
\usepackage{amsthm}
\usepackage{courier}
\usepackage{verbatim}
\geometry{verbose,tmargin=1in,bmargin=1in,lmargin=1in,rmargin=1in}
\usepackage{setspace}
%\usepackage{esint}
\onehalfspacing
\usepackage{babel}
\usepackage{amsmath}

\theoremstyle{remark}
\newtheorem*{remark}{Remark}
\begin{document}
	
\title{Preliminary Calibration for ``Endogenous Growth with Creative Destruction by Employee Spinouts''}
\author{Nicolas Fernandez-Arias}
\maketitle

\section{Description}

There are 7 moments and 10 parameters. Of the 10 parameters, 3 are fixed without refrence to the moments. So the exercise below is effectively 1 moment per parameter. Not meant as a test of the model, but rather as a way to explore its workings at reasonable parameter values.

\subsection{Moments used}

Where applicable, I list as (target,model).

\begin{enumerate}
	\item Interest rate: 0.05 (fixes discount factor $\rho$ - model is risk-neutral so $\rho$ is excessive for now)
	\item Profit / sales ratio (fixes $\beta$)
	\item R\&D spending / sales: $(0.15,0.09)$
	\item Internal patent share: $(0.2,0.202113)$
	\item Spinout entry rate (loosely based on entry rate of 5\% in economy - target should probably be lower): $(0.05,0.08)$
	\item Share of creative destruction by spinouts (not based on anything, really): $(0.30,0.28)$
	\item Growth rate: $(0.015, 0.02)$
\end{enumerate}

\subsection{Parameters}
\begin{enumerate}
	\item $\rho = 0.05$
	\item $\beta = 0.106$
	\item $L = 1$ (normalization)
	\item $\chi_I = 3.2523$
	\item $\chi_S = 1.42543$
	\item $\chi_E = 1.2029$
	\item $\psi_I = \psi_E = 0.5$ (not calibrated)
	\item $\lambda = 1.0532733$
	\item $\nu = 0.020499$
	\item $\xi = 5.0$
\end{enumerate}




\end{document}