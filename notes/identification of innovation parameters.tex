\documentclass[12pt,english]{article}
%\usepackage{lmodern}
\linespread{1.05}
\usepackage{mathpazo}
%\usepackage{mathptmx}
%\usepackage{utopia}
\usepackage{microtype}
\usepackage[T1]{fontenc}
\usepackage[latin9]{inputenc}
\usepackage[dvipsnames]{xcolor}
\usepackage{geometry}
\usepackage{amsthm}
\usepackage{amsfonts}

\usepackage{courier}
\usepackage{verbatim}
\usepackage[round]{natbib}
\bibliographystyle{plainnat}


\definecolor{red1}{RGB}{128,0,0}
%\geometry{verbose,tmargin=1.25in,bmargin=1.25in,lmargin=1.25in,rmargin=1.25in}
\geometry{verbose,tmargin=1in,bmargin=1in,lmargin=1in,rmargin=1in}
\usepackage{setspace}

\usepackage[colorlinks=true, linkcolor={red!70!black}, citecolor={blue!50!black}, urlcolor={blue!80!black}]{hyperref}
%\usepackage{esint}
\onehalfspacing
\usepackage{babel}
\usepackage{amsmath}
\usepackage{graphicx}

\theoremstyle{remark}
\newtheorem{remark}{Remark}
\begin{document}
	

\title{Identifying innovation parameters}
\author{Nicolas Fernandez-Arias}
\maketitle

I need to identify $\chi_I, \chi_E, \chi_S, \lambda, \nu, \theta, \zeta, \kappa$. Here is my plan.

\begin{enumerate}
	\item Identify $\chi_E$ by matching the entry rate in the model to the entry rate in the data.
	\begin{itemize}
		\item Problem: entry in the data includes firms before they have become profitable. I am identifying the difference in $\chi_S / \chi_E$ based on the different likelihood of attaining profitability, which is maybe a bit of a stretch. But it allows me to disentangle $\chi_S$ and $\nu$. Given that, I should match entry in the model to entry into profitability. So, new profitable firms divided by total profitable firms. No idea how to calculate this. So, I will just proceed for now by assuming that it is the same, maybe ask someone.
	\end{itemize}
	\item Identify $\chi_I$ by matching the fraction of innovations (citation-weighted?) that are internal to the firm, as in AK 2017 (but compute the statistic yourself)
	\item Identify some function of $\nu,\chi_S$ by the fraction of entry which are spinouts *caused by R\&D according to the estimate*
	\begin{itemize}
		\item Calculate this based on product of (1) fraction of entry into race, and (2) relative likelihood of attaining profitability, two things I measured in the data
	\end{itemize}
	\item Separate $\nu$ from $\chi_S$ by using the difference in the rate at which spinouts versus ordinary entrants go from early stage to profitability / revenue
	\item Identify $\lambda$ via the growth rate
	\item Identify $\theta$ using fraction of spinouts in same NAICS 4 code
	\begin{itemize}
		\item Talk about robustness to this parameter
	\end{itemize}
	\item Identify $\kappa$ by some estimate based on marketing as a percentage of sales
	\begin{itemize}
		\item This is the "fixed" cost of firm formation, which in the current model actually is a percentage of the value of the firm, so it causes no selection and hence no predictions, really.
		\item Could instead replace it with something that scales with $q$ and not $m$, so that it significantly affects the free-entry condition.
		\item But it would really just affect the entry of spinouts, which I'm already using.
	\end{itemize}
\end{enumerate}

Other parameters: $\beta,\rho,\psi$ identified via external calibration. 

Remaining issues:
\begin{enumerate}
	\item Entry rate in model corresponds to entry into being acquired or public in the data...so not the "entry rate" as usually defined. That is, if I want to consider entry into the VC dataset as something that happens "before" entry in the data. So I should count the number of new public firms each year, I guess. That doesn't actually seem too hard. Not perfect, but at least coherent. Try this. 
	\begin{itemize}
		\item Is the answer to do something analogous to AK 2017, but with VC funding?
		\item Add some VC funding parameters, e.g., fraction of startups getting VC funding, fraction of 
	\end{itemize}
	\item Parameter $\kappa$ not well identified - robustness
	\item Model depends heavily on assumption that spinouts are generated by R\&D. This means (1) the possibility of spinouts heavily and directly disincentivizes R\&D, and (2) the R\&D wage is super low, since only R\&D employees can do spinouts. (2) is a genuine problem which I'm not sure how to solve, but (1) seems fine because isnt model equivalent to a model where having higher quality leads to more spinouts? At the end of the day, it's just a rate at which you increase the rate of spinouts in expectation due to more R\&D spending. R\&D spending is constant in my model, so I think there's literally no difference.
	\item Knowledge expiry: need to assume that firms are only innovative for one "generation", unless they become incumbents, in which case they have the incumbent R\&D technology. The interpretation is that being an incumbent producer opens up the firm to more learning opportunities, but being a spinout of an incumbent producer doesn't help beyond making that first increment, because you are inheriting KNOWLEDGE, not INNOVATIVENESS. (could write a model about inheriting innovativeness, but my model is about inheritance of knowledge). My model is about how any firm with knowledge spills it over. 
	\item 
\end{enumerate}







\end{document}