\documentclass[12pt,english]{article}
\usepackage{lmodern}
\linespread{1.05}
%\usepackage{mathpazo}
%\usepackage{mathptmx}
%\usepackage{utopia}
\usepackage{microtype}
\usepackage[T1]{fontenc}
\usepackage[latin9]{inputenc}
\usepackage[dvipsnames]{xcolor}
\usepackage{geometry}
\usepackage{amsthm}
\usepackage{amsfonts}

\usepackage{courier}
\usepackage{verbatim}
\usepackage[round]{natbib}
\bibliographystyle{plainnat}


\definecolor{red1}{RGB}{128,0,0}
%\geometry{verbose,tmargin=1.25in,bmargin=1.25in,lmargin=1.25in,rmargin=1.25in}
\geometry{verbose,tmargin=1in,bmargin=1in,lmargin=1in,rmargin=1in}
\usepackage{setspace}

\usepackage[colorlinks=true, linkcolor={red!70!black}, citecolor={blue!50!black}, urlcolor={blue!80!black}]{hyperref}
%\usepackage{esint}
\onehalfspacing
\usepackage{babel}
\usepackage{amsmath}
\usepackage{graphicx}

\theoremstyle{remark}
\newtheorem{remark}{Remark}
\begin{document}
	

\title{Non-competing spinouts}
\author{Nicolas Fernandez-Arias}
\maketitle

Suppose some fraction $\theta$ of spinouts generated do not compete with the parent firm but rather become high-type potential entrants in a random line $j$. Then the value of this is
\begin{align*}
	\mathcal{W} &= \int_0^{\infty} W(m) \gamma(m) \mu(m)dm
\end{align*}

so that the indifference condition becomes
\begin{align*}
	\overline{w} &= w^K(m) + \nu^K \Big( \theta \mathcal{W} + (1-\theta) W(m) \Big)
\end{align*}

In order to get this result we have to make a more complicated assumption in the full model. For a line $j$ in state $(m,q,Q)$, intensity per unit of effort of a potential spinout forming in line $j$ is $(1-\theta) (q/Q)^{-1} \nu^K$. But the intensity for a spinout forming in a different line $j$ is simply $\theta \nu^K$. This assumption is necessary so that the wage does not depend on $q/Q$, since $\mathcal{W} = \mathcal{W}(Q)$ while $W(m,q,Q)$ depends on $q/Q$. This means that we have to include $\gamma$ in there. 

The above adds a flow of high-type spinouts into all lines $j$, given by
\begin{align*}
	\overline{a} &= \theta \int_0^{\infty} \Big( \nu^I z^I(m) + \nu^S z^S(m) \Big) \gamma(m) \mu(m) dm 
\end{align*}

The drift in a line $j$ in state $m$ is
\begin{align*}
	\dot{m} &= \overline{a} + (1-\theta) \Big(\nu^I z^I + \nu^S z^S + \nu^E z^E \Big) 
\end{align*}

\section{Improved version}

Suppose there are decreasing returns to innovation in a particular line, $f(\ell)$. Then the maximization problem is
\begin{align*}
	\max_{\ell_j} \int_0^1 f(\ell_j) q_j W(m_j) dj 
\end{align*}

subject to
\begin{align*}
	\int_0^1 \ell_j (q_j/Q) dj \le 1
\end{align*}

Lagrangean:
\begin{align*}
	\mathcal{L} &= \int_0^1 f(\ell_j) q_j W(m_j) dj  + \lambda \Big( 1 - \int_0^1 \ell_j (q_j/Q) dj \Big)
\end{align*}

FOC:
\begin{align*}
	f'(\ell_j) q_j W(m_j) &= \lambda \frac{q_j}{Q}
\end{align*}

Because of the assumptions, the $q_j$ cancels out, leaving
\begin{align}
	f'(\ell_j) =  \frac{\lambda}{Q W(m_j)} \label{FOC}
\end{align}

Suppose $f(\ell) = \ell^{\alpha}$ for some $\alpha \in [0,1)$, where $\alpha = 0$ is understood to represent $f(\ell) = \log(\ell)$. Then $f'(\ell) = \alpha \ell^{\alpha-1}$. Dividing (\ref{FOC}) at different values of $j$ yields
\begin{align}
	\frac{\ell_j}{\ell_{j'}} &= \Bigg(\frac{W(m_{j})}{W(m_{j'})}\Bigg) ^{\frac{1}{1-\alpha}} \label{laborRatio}
\end{align}

Substituting into the budget constraint,
\begin{align*}
	1 &= \int_0^1 \ell_0 \frac{\ell_j}{\ell_0} \frac{q_j}{Q} dj \\
	  &= \int_0^1 \ell_0 \Bigg(\frac{W(m_{j})}{W(m_0)}\Bigg) ^{\frac{1}{1-\alpha}} \frac{q_j}{Q} dj 
\end{align*}

determines $\ell_0$. 

Practically, it is necessary to compute this integral in a different space, 
\begin{align*}
	1 = \int_0^{\infty} \ell(0) \Bigg(\frac{W(m)}{W(0)}\Bigg) ^{\frac{1}{1-\alpha}} \gamma(m) \mu(m) dm 
\end{align*}

where $\ell(0)$ is the allocation per unit of relative quality to each line in state $m = 0$. Given $\ell(0)$ one can compute $\ell(m)$ for all $m > 0$ using the spinout value $W(m)$. 

Finally, the rate of drift for a good in state $m$ is
\begin{align*}
	f(\ell(m)) \theta \nu \int_0^{\infty} z_I(m) \gamma(m) \mu(m) dm 
\end{align*} 

which can be calculated readily from everything available. 





\end{document}