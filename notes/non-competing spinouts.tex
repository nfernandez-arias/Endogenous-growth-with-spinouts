\documentclass[12pt,english]{article}
%\usepackage{lmodern}
\linespread{1.05}
\usepackage{mathpazo}
%\usepackage{mathptmx}
%\usepackage{utopia}
\usepackage{microtype}
\usepackage[T1]{fontenc}
\usepackage[latin9]{inputenc}
\usepackage[dvipsnames]{xcolor}
\usepackage{geometry}
\usepackage{amsthm}
\usepackage{amsfonts}

\usepackage{courier}
\usepackage{verbatim}
\usepackage[round]{natbib}
\bibliographystyle{plainnat}


\definecolor{red1}{RGB}{128,0,0}
%\geometry{verbose,tmargin=1.25in,bmargin=1.25in,lmargin=1.25in,rmargin=1.25in}
\geometry{verbose,tmargin=1in,bmargin=1in,lmargin=1in,rmargin=1in}
\usepackage{setspace}

\usepackage[colorlinks=true, linkcolor={red!70!black}, citecolor={blue!50!black}, urlcolor={blue!80!black}]{hyperref}
%\usepackage{esint}
\onehalfspacing
\usepackage{babel}
\usepackage{amsmath}
\usepackage{graphicx}

\theoremstyle{remark}
\newtheorem{remark}{Remark}
\begin{document}
	
	
	
\title{Non-competing spinouts}
\author{Nicolas Fernandez-Arias}
\maketitle

Suppose some fraction $\theta$ of spinouts generated do not compete with the parent firm but rather become high-type potential entrants in a random line $j$. Then the value of this is
\begin{align*}
	\mathcal{W} &= \int_0^{\infty} W(m)\mu(m)dm
\end{align*}

so that the indifference condition becomes
\begin{align*}
	\overline{w} &= w^K(m) + \nu^K \Big( \theta \mathcal{W} + (1-\theta) W(m) \Big)
\end{align*}

In order to get this result we have to make a more complicated assumption in the full model. For a line $j$ in state $(m,q,Q)$, intensity per unit of effort of a potential spinout forming in line $j$ is $(1-\theta) (q/Q)^{-1} \nu^K$. But the intensity for a spinout forming in a different line $j$ is simply $\theta \nu^K$. This assumption is necessary so that the wage does not depend on $q/Q$, since $\mathcal{W} = \mathcal{W}(Q)$ while $W(m,q,Q)$ depends on $q/Q$.

The above adds a flow of high-type spinouts into all lines $j$, given by
\begin{align*}
	\overline{a} &= \theta \int_0^{\infty} \Big( \nu^I z^I(m) + \nu^S z^S(m) \Big) \mu(m) dm 
\end{align*}

The drift in a line $j$ in state $m$ is
\begin{align*}
	\dot{m} &= \overline{a} + (1-\theta) \Big(\nu^I z^I + \nu^S z^S + \nu^E z^E \Big) 
\end{align*}










\end{document}