\documentclass[12pt,english]{article}
\usepackage{lmodern}
\usepackage[T1]{fontenc}
\usepackage[latin9]{inputenc}
\usepackage{geometry}
\usepackage{amsthm}
\usepackage{courier}
\usepackage{verbatim}
\geometry{verbose,tmargin=1in,bmargin=1in,lmargin=1in,rmargin=1in}
\usepackage{setspace}
%\usepackage{esint}
\onehalfspacing
\usepackage{babel}
\usepackage{amsmath}
\date{}
\theoremstyle{remark}
\newtheorem*{remark}{Remark}
\begin{document}
	
\title{To-Do List 6-23-2018}
\author{Nicolas Fernandez-Arias}
\maketitle

\begin{itemize}
	\item Look at \textbf{document you wrote} about predictions of my model. Need to be able to articulate how the predictions are different from the standard endogenous growth model, as Moll said.
	\item Code:
	\begin{itemize}
		\item Understand issue causing lack of stability in code. 
		\item Try to solve next part of code...maybe this can be estimated..
	\end{itemize}
	\item Estimation vs. calibration. Try to articulate exactly what I am going to do and ask Esteban what he thinks. Does ``estimation'' mean that the model is able to perfectly fit the data (i.e. I interpret all of the errors within the model)?
	\item Finally, think about what my model needs to capture. 
	\begin{itemize}
		\item E.g., in the current model, cannot identify firms in the model with firms in the data because the firms in the data only have one product...so I can only do things like startup rate, not firm performance. Moreover, in general, there is no "spinout performance" in my model, because they just do R\&D...this is kind of a big problem in my opinion. It almost seems that my model is more suited to identifying the advantages of spinouts by using an approach like in Akcigit \& Kerr -do spinout patents have more citations. Model would need some reworking - advantage of spinouts over other entrants would be in the step size of their ladder.
		\item If I want to be able to use the LBD data on the performance of spinouts, and in particular draw more direct inspiration from the LEHD-identification-based work about human capital of spinouts, screening of spinouts by non-competes, which is based on spinout performance as measured by employment etc., need a model where spinouts actually have some amount of employment depending on their productivity. 
		\item In any case, keep in mind that while in my model all spinouts compete, this is fine because I will only classify firms as spinouts if they compete with parents (and non-competes can only ever apply to these types of spinouts).
	\end{itemize} 
	
\end{itemize}








\end{document}
