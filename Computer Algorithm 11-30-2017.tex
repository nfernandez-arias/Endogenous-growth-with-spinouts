%% LyX 2.2.3 created this file.  For more info, see http://www.lyx.org/.
%% Do not edit unless you really know what you are doing.
\documentclass[english]{article}
\usepackage[T1]{fontenc}
\usepackage[latin9]{inputenc}
\usepackage{geometry}
\geometry{verbose,tmargin=1in,bmargin=1in,lmargin=1in,rmargin=1in}
\usepackage{babel}
\begin{document}

\title{Computer Algorithm: }
\maketitle

\subsection*{High-level overview and logic}

A ``good'' computer algorithm does two things. (1) It decompose
the larger problem into a nested (i.e. recursive) sequence of smaller
problems (e.g. in this case, fixed point problem for an operator on
a lower-dimensional vector space); and (2) each of these subproblems
is guaranteed to have a solution. The purpose of (1) is that it makes
it more clear how to update a guess. With a scalar problem this is
often as simple as checking one inequality. The purpose of (2) is
to make this procedure coherent. In general (1) is easy \textendash{}
even trivial \textendash{} to accomplish: simply split up a large
dimensional guess/verify/update into a series of guesses and verifications
and updates. But to accomplish (1) while accomplishing (2) as well
requires some economic logic, i.e. the inner problem is a partial
equilibrium model given a guess in an outer problem. 
\begin{enumerate}
\item Find $L^{F}$ that satisfies the resource constraint on labor. To
do this, need to guess $L^{F}$, then...
\begin{enumerate}
\item Compute growth rate given $L^{f}$. To do this, need to guess growth
rate $g$, then... 
\begin{enumerate}
\item Compute (partial)-equilibrium given $L^{f},g$. To do this, need to
guess $w(q,m,n)$, then...
\begin{enumerate}
\item Compute Nash equilibrium given $L^{f},g$ and $w(q,m,n)$.\footnote{Conjecture: pure strategy NE exists and is unique. Reasoning: }
\end{enumerate}
\item Then can check consistency: $w(q,m,n)+\nu W^{NC}(q,m,n)=\overline{w}$
\end{enumerate}
\item Then check consistency: $g=g^{*}$ where $g^{*}$ is computed by simulating
the model over time
\end{enumerate}
\item Then check consistency: $L^{F}+L^{I}+L^{RD}=1$.
\end{enumerate}

\subsection*{Details}

The details that are missing are the following: 
\begin{enumerate}
\item Initial guesses for (indicated by subscript $0$)
\begin{enumerate}
\item $L_{0}^{F}$
\item $g_{0}(L^{F})$
\item $w_{0}(q,m,n|g,L^{F})$
\end{enumerate}
\item Update rules (indicated by subscript $1$):
\begin{enumerate}
\item $ $
\item $w_{1}(q,m,n)$
\item $g_{1}$
\end{enumerate}
\item Guess $L^{f}$
\begin{enumerate}
\item Guess $g$
\begin{enumerate}
\item Guess $w(q,m,n)$
\begin{enumerate}
\item Compute Nash Equilibrium of innovation race game between incumbent
firms and entrant firms, assuming they can hire exactly what workers
they want at exogenous wage $w(q,m,n)$. 
\item Involves initial guesses of $z_{E}(q,m,n)$, $F^{*}(q,n)$ and/or
$z_{I}(q,m,n)$. 
\item Outputs policy functions $z_{E}(q,m,n)=\xi\min(m,F^{*}(q,n)$ and
$z_{I}(q,m,n)$, as well as value functions $A(q,m,n),W^{NC}(q,m,n),W^{F}(q,m,n)$.
\end{enumerate}
\end{enumerate}
\end{enumerate}
\end{enumerate}

\end{document}
