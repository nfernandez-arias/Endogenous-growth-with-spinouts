% use option [draft] for initial mission
%            [final] for the prepublication
\documentclass[ecta,nameyear,final]{econsocart}
%
%\usepackage{}
\RequirePackage[colorlinks,citecolor=blue,linkcolor=blue,urlcolor=blue,pagebackref]{hyperref}

\startlocaldefs

%%%%%%%%%%%%%%%%%%%%%%%%%%%%%%%%%%%%%%%%%%%%%%
%%                                          %%
%% Uncomment next line to change            %%
%% the type of equation numbering           %%
%%                                          %%
%%%%%%%%%%%%%%%%%%%%%%%%%%%%%%%%%%%%%%%%%%%%%%
%\numberwithin{equation}{section}
%%%%%%%%%%%%%%%%%%%%%%%%%%%%%%%%%%%%%%%%%%%%%%
%%                                          %%
%% For Assumption, Axiom, Claim, Corollary, %%
%% Lemma, Theorem, Proposition, Hypothezis, %%
%% Fact                                     %%
%% use \theoremstyle{plain}                 %%
%%                                          %%
%%%%%%%%%%%%%%%%%%%%%%%%%%%%%%%%%%%%%%%%%%%%%%
\theoremstyle{plain}
\newtheorem{axiom}{Axiom}
\newtheorem{claim}[axiom]{Claim}
\newtheorem{theorem}{Theorem}[section]
\newtheorem{lemma}[theorem]{Lemma}
\newtheorem*{fact}{Fact}
%%%%%%%%%%%%%%%%%%%%%%%%%%%%%%%%%%%%%%%%%%%%%%
%%                                          %%
%% For Definition, Example, ,         %%
%% Notation, Property                       %%
%% use \theoremstyle{}                %%
%%                                          %%
%%%%%%%%%%%%%%%%%%%%%%%%%%%%%%%%%%%%%%%%%%%%%%
\theoremstyle{remark}
\newtheorem{definition}[theorem]{Definition}
\newtheorem*{example}{Example}

%%%%%%%%%%%%%%%%%%%%%%%%%%%%%%%%%%%%%%%%%%%%%%
%% Please put your definitions here:        %%
%%%%%%%%%%%%%%%%%%%%%%%%%%%%%%%%%%%%%%%%%%%%%%


\endlocaldefs

\begin{document}

\begin{frontmatter}

\title{A Sample Article Title}
\runtitle{A Sample Running Head Title}

\begin{aug}
% use \particle for den|der|de|van|von (only lc!)
% [id=?,addressref=?,corref]{\fnms{}~\snm{}\ead[label=e?]{}\thanksref{}}
%
%% e-mail is mandatory for each author
%
%%% initials in fnms (if any) with spaces
%
\author[id=au1,addressref={add1,add11}]{\fnms{First}~\snm{Author}\ead[label=e1]{first@somewhere.com}}
\author[id=au2,addressref={add2}]{\fnms{Second}~\snm{Author}\ead[label=e2]{second@somewhere.com}}
\author[id=au3,addressref={add2}]{\fnms{Third}~\snm{Author}\ead[label=e3]{third@somewhere.com}}
%%%%%%%%%%%%%%%%%%%%%%%%%%%%%%%%%%%%%%%%%%%%%%
%% Addresses                                %%
%%%%%%%%%%%%%%%%%%%%%%%%%%%%%%%%%%%%%%%%%%%%%%
\address[id=add1]{%
\orgdiv{First Department of the First Author},
\orgname{University}}

\address[id=add11]{%
\orgdiv{Second Department of the First Author},
\orgname{University}}

\address[id=add2]{%
\orgdiv{Department of the Second and Third Authors},
\orgname{University}}
\end{aug}

%% Put support info here.  Reminder: do not thank the handling coeditor anonymously or by name
\support{We thank four anonymous referees. The first author gratefully acknowledges
financial support from the National Science Foundation through Grant XXX-0000000.}
%
\begin{abstract}
The abstract should summarize the contents of the paper. It should be clear,
descriptive, self-explanatory and not longer than 150 words. It should also be
suitable for publication in abstracting services. Please avoid using math formulas
as much as possible.
\end{abstract}

\begin{keyword}
\kwd{First keyword}
\kwd{second keyword}
\end{keyword}

\end{frontmatter}
%%%%%%%%%%%%%%%%%%%%%%%%%%%%%%%%%%%%%%%%%%%%%%%%%%%%%%%%%%%%%%%%%%%%%%%%%
%%%% Main text entry area:
%%%%%%%%%%%%%%%%%%%%%%%%%%%%%%%%%%%%%%%%%%%%%%%%%%%%%%%%%%%%%%%%%%%%%%%%%

\section{Introduction}

\caps{This template helps you} to create a properly formatted \LaTeXe\ manuscript. Note that the first few words in the first paragraph should be in small caps.
Prepare your paper in the same style as used in this sample .pdf file.
Try to avoid excessive use of italics and bold face.
Please do not use any \LaTeXe\ or \TeX\ commands that affect the layout
or formatting of your document (i.e., commands like \verb|\textheight|,
\verb|\textwidth|, etc.).

\section{Section Headings}
Here are some subsections:
\subsection{A Subsection}
Regular text.
\subsubsection{A Subsubsection}
Regular text.

\section{Text}

\subsection{Lists}

The following is an example of an \emph{itemized} list,
two levels deep.
\begin{itemize}
\item
This is the first item of an itemized list.  Each item
in the list is marked with a ``tick.''  The document
style determines what kind of tick mark is used.
\item
This is the second item of the list.  It contains another
list nested inside of it.
\begin{itemize}
\item This is the first item of an itemized list that
is nested within the itemized list.
\item This is the second item of the inner list.  \LaTeX\
allows you to nest lists deeper than you really should.
\end{itemize}
This is the rest of the second item of the outer list.
\item
This is the third item of the list.
\end{itemize}

The following is an example of an \emph{enumerated} list of one level.

\begin{enumerate}[(ii)]
\item[(i)] This is the first item of an enumerated list.
\item[(ii)] This is the second item of an enumerated list.
\end{enumerate}

The following is an example of an \emph{enumerated} list, two levels deep.
\begin{enumerate}
\item
This is the first item of an enumerated list.  Each item
in the list is marked with a ``tick.''  The document
style determines what kind of tick mark is used.
\item
This is the second item of the list.  It contains another
list nested inside of it.
\begin{enumerate}
\item
This is the first item of an enumerated list that
is nested within.
\item
This is the second item of the inner list.  \LaTeX\
allows you to nest lists deeper than you really should.
\end{enumerate}
This is the rest of the second item of the outer list.
\item
This is the third item of the list.
\end{enumerate}

\subsection{Punctuation}
Avoid unnecessary hyphenation; many hyphenated words can be treated as one or two words.
Dashes come in three sizes: a hyphen, an intra-word dash like ``$U$-statistics'' or ``the time-homogeneous model'';
a medium dash (also called an ``en-dash'') for number ranges or between two equal entities like ``1--2'' or ``Cauchy--Schwarz inequality'';
and a punctuation dash (also called an ``em-dash'') in place of a comma, semicolon,
colon or parentheses---like this.

Generating an ellipsis \ldots\ with the right spacing
around the periods requires a special command.

\subsection{Citation}
Only include in the reference list entries for which there are text citations,
and make sure all citations are included in the reference list.
Simple author and year cite: \cite{b1}.
Multiple bibliography items cite: \cite{b2,b3,b4}.
Author only cite: \citeauthor{b5}.
Year only cite: (\citeyear{b5}).

\section{Fonts}
Please use text fonts in text mode, e.g.:
\begin{itemize}
\item[]\textrm{Roman}
\item[]\textit{Italic}
\item[]\textbf{Bold}
\item[]\textsc{Small Caps}
\item[]\textsf{Sans serif}
\item[]\texttt{Typewriter}
\end{itemize}
Please use mathematical fonts in mathematical mode, e.g.:
\begin{itemize}
\item[] $\mathrm{ABCabc123}$
\item[] $\mathit{ABCabc123}$
\item[] $\mathbf{ABCabc123}$
\item[] $\boldsymbol{ABCabc123\alpha\beta\gamma}$
\item[] $\mathcal{ABC}$
\item[] $\mathbb{ABC}$
\item[] $\mathsf{ABCabc123}$
\item[] $\mathtt{ABCabc123}$
\item[] $\mathfrak{ABCabc123}$
\end{itemize}
Note that \verb|\mathcal, \mathbb| belongs to capital letters-only font typefaces.

\section{Notes}
Footnotes\footnote{This is an example of a footnote.}
pose no problem.\footnote{Note that footnote number is after punctuation.}

\section{Quotations}

Text is displayed by indenting it from the left margin. There are short quotations
\begin{quote}
This is a short quotation.  It consists of a
single paragraph of text.  There is no paragraph
indentation.
\end{quote}
and longer ones.
\begin{quotation}
This is a longer quotation.  It consists of two paragraphs
of text.  The beginning of each paragraph is indicated
by an extra indentation.

This is the second paragraph of the quotation.  It is just
as dull as the first paragraph.
\end{quotation}

\section{Environments}

\subsection{Examples for \emph{\texttt{plain}}-Style Environments}

\begin{axiom}\label{ax1}
This is the body of Axiom \ref{ax1}.
\end{axiom}


\begin{claim}\label{cl1}
This is the body of Claim \ref{cl1}. Claim \ref{cl1} is numbered after
Axiom \ref{ax1} because we used \verb|[axiom]| in \verb|\newtheorem|.
\end{claim}

\begin{theorem}\label{th1}
This is the body of Theorem \ref{th1}. Theorem \ref{th1} numbering is
dependent on section because we used \verb|[section]| after \verb|\newtheorem|.
\end{theorem}

\begin{proof}
This is the body of the proof of the theorem above.
\end{proof}

\begin{theorem}[Title of the Theorem]\label{th2}
This is the body of Theorem \ref{th2}. Theorem~\ref{th2} has additional title.
\end{theorem}

\begin{lemma}\label{le1}
This is the body of Lemma \ref{le1}. Lemma \ref{le1} is numbered after
Theorem \ref{th2} because we used \verb|[theorem]| in \verb|\newtheorem|.
\end{lemma}

\begin{fact}
This is the body of the fact. Fact is unnumbered because we used \verb|\newtheorem*|
instead of \verb|\newtheorem|.
\end{fact}

\begin{proof}[Proof of Theorem \ref{th2}]
This is the body of the proof of Theorem \ref{th2}.
\end{proof}


\subsection{Examples for \emph{\texttt{remark}}-Style Environments}
\begin{definition}\label{de1}
This is the body of Definition \ref{de1}. Definition \ref{de1} is numbered after
Lemma~\ref{le1} because we used \verb|[theorem]| in \verb|\newtheorem|.
\end{definition}

\begin{example}
This is the body of the example. Example is unnumbered because we used \verb|\newtheorem*|
instead of \verb|\newtheorem|.
\end{example}

\section{Equations and the Like}
Only number equations to which there is a subsequent reference.
See equations below (\ref{ccs})--(\ref{e7}).

Two equations:
\begin{equation}
    C_{s}  =  K_{M} \frac{\mu/\mu_{x}}{1-\mu/\mu_{x}} \label{ccs}
\end{equation}
and
\begin{equation}
    G = \frac{P_{\mathrm{opt}} - P_{\mathrm{ref}}}{P_{\mathrm{ref}}}  100(\%).
\end{equation}
Equation arrays:
\begin{eqnarray}
  \frac{dS}{dt} & = & - \sigma X + s_{F} F,\\
  \frac{dX}{dt} & = &   \mu    X,\\
  \frac{dP}{dt} & = &   \pi    X - k_{h} P,\\
  \frac{dV}{dt} & = &   F.
\end{eqnarray}
One long equation:
\begin{eqnarray}
 \mu_{\text{normal}} & = & \mu_{x} \frac{C_{s}}{K_{x}C_{x}+C_{s}}  \nonumber\\
                     & = & \mu_{\text{normal}} - Y_{x/s}\bigl(1-H(C_{s})\bigr)(m_{s}+\pi /Y_{p/s})\nonumber\\
                     & = & \mu_{\text{normal}}/Y_{x/s}+ H(C_{s}) (m_{s}+ \pi /Y_{p/s}).\label{e7}
\end{eqnarray}
Note that variables made of more than one letter should use command \verb|\mathit|,
e.g., $\mathit{sov}=550$, where $\mathit{sov}$ is sum of votes. Abbreviations used in subscripts or superscripts should use \verb|\mathrm|,
e.g., $t_{\mathrm{max}}-t_{\mathrm{min}} =10$. Operator names should use \verb|\operatorname|, e.g. $\operatorname{AR}(1)$. Also, note that $\emptyset$ symbol is preferred as opposed to $\varnothing$.

\section{Tables and Figures}
Cross-references to labeled tables: As you can see in Table~\ref{sphericcase}
and also in Table~\ref{parset}.

Sample of cross-reference to figure: Figure~\ref{penG} shows that it is not easy to get something on paper.

\begin{table*}
\caption{The Spherical Case ($I_1=0$, $I_2=0$)\tabnoteref[a]{tab1}}
\label{sphericcase}
\begin{tabular}{@{}lrrrrc@{}@{}}
\hline
Equil. points
& \multicolumn{1}{c}{$x$}
& \multicolumn{1}{c}{$y$}
& \multicolumn{1}{c}{$z$}
& \multicolumn{1}{c}{$C$}
& S \\
\hline
$L_1$    & $-$2.485252241 & 0.000000000    & 0.017100631    & 8.230711648    & U \\
$L_2$    & 0.000000000    & 0.000000000    & 3.068883732    & 0.000000000    & S \\
$L_3$    & 0.009869059    & 0.000000000    & 4.756386544    & $-$0.000057922 & U \\
$L_4$    & 0.210589855    & 0.000000000    & $-$0.007021459 & 9.440510897    & U \\
$L_5$    & 0.455926604    & 0.000000000    & $-$0.212446624 & 7.586126667    & U \\
$L_6$    & 0.667031314    & 0.000000000    & 0.529879957    & 3.497660052    & U \\
$L_7$    & 2.164386674    & 0.000000000    & $-$0.169308438 & 6.866562449    & U \\
$L_8$    & 0.560414471    & 0.421735658    & $-$0.093667445 & 9.241525367    & U \\
$L_9$    & 0.560414471    & $-$0.421735658 & $-$0.093667445 & 9.241525367    & U \\
$L_{10}$ & 1.472523232    & 1.393484549    & $-$0.083801333 & 6.733436505    & U \\
$L_{11}$ & 1.472523232    & $-$1.393484549 & $-$0.083801333 & 6.733436505    & U \\
\hline
\end{tabular}
\tabnotetext[a]{tab1}{This is how table note should be presented.
Please do not use asterisks or bold face to denote statistical significance.
We encourage authors to report standard errors and coverage sets or confidence intervals.}
\end{table*}

\begin{table}
\caption{Sample Posterior Estimates for Each Model}
\label{parset}
%
\begin{tabular}{@{}lcrcrrr@{}@{}}
\hline
& & & &\multicolumn{3}{c}{Quantile} \\
\cline{5-7}
Model
& Parameter
& \multicolumn{1}{c}{Mean}
& Std. dev.
& \multicolumn{1}{c}{2.5\%}
& \multicolumn{1}{c}{50\%}
& \multicolumn{1}{c@{}}{97.5\%} \\
\hline
{Model 0} & $\beta_0$ & $-$12.29 & 2.29 & $-$18.04 & $-$11.99 & $-$8.56 \\
          & $\beta_1$ & 0.10     & 0.07 & $-$0.05  & 0.10     & 0.26    \\
          & $\beta_2$ & 0.01     & 0.09 & $-$0.22  & 0.02     & 0.16    \\[6pt]
{Model 1} & $\beta_0$ & $-$4.58  & 3.04 & $-$11.00 & $-$4.44  & 1.06    \\
          & $\beta_1$ & 0.79     & 0.21 & 0.38     & 0.78     & 1.20    \\
          & $\beta_2$ & $-$0.28  & 0.10 & $-$0.48  & $-$0.28  & $-$0.07 \\[6pt]
{Model 2} & $\beta_0$ & $-$11.85 & 2.24 & $-$17.34 & $-$11.60 & $-$7.85 \\
          & $\beta_1$ & 0.73     & 0.21 & 0.32     & 0.73     & 1.16    \\
          & $\beta_2$ & $-$0.60  & 0.14 & $-$0.88  & $-$0.60  & $-$0.34 \\
          & $\beta_3$ & 0.22     & 0.17 & $-$0.10  & 0.22     & 0.55    \\
\hline
\end{tabular}
%
\end{table}

\begin{figure}
\includegraphics{figure1}
\caption{Pathway of the penicillin G biosynthesis.}
\label{penG}
\end{figure}


%%%%%%%%%%%%%%%%%%%%%%%%%%%%%%%%%%%%%%%%%%%%%%
%% Example with single Appendix:            %%
%%%%%%%%%%%%%%%%%%%%%%%%%%%%%%%%%%%%%%%%%%%%%%
\begin{appendix}
\section*{Title}\label{appn} %% if no title is needed, leave empty \section*{}.
Appendices should be provided in \verb|{appendix}| environment. If there is only one appendix,
then please refer to it in text as \ldots\ in the \hyperref[appn]{Appendix}.
\end{appendix}

%%%%%%%%%%%%%%%%%%%%%%%%%%%%%%%%%%%%%%%%%%%%%%
%% Example with multiple Appendixes:        %%
%%%%%%%%%%%%%%%%%%%%%%%%%%%%%%%%%%%%%%%%%%%%%%
\begin{appendix}
\section{Title of the First Appendix}\label{appA}
If there are more than one appendix, then please refer to it
as \ldots\ in Appendix \ref{appA}, Appendix \ref{appB}, etc.

\section{Title of the Second Appendix}\label{appB}
\subsection{First Subsection of Appendix \protect\ref{appB}}

Use the standard \LaTeX\ commands for headings in \verb|{appendix}|.
Headings and other objects will be numbered automatically.
\begin{equation}
\mathcal{P}=(j_{k,1},j_{k,2},\dots,j_{k,m(k)}). \label{path}
\end{equation}

Sample of cross-reference to formula (\ref{path}) in Appendix \ref{appB}.
\end{appendix}

\begin{thebibliography}{}
%
\bibitem[\protect\citeauthoryear{Aumann}{1987}]{b1}
\textsc{Aumann, R. J.} (1987):
``Correlated Equilibrium as an Expression of Bayesian Rationality,''
\textit{Econometrica}, 55, 1--18.
\endbibitem

\bibitem[\protect\citeauthoryear{Peck}{1994}]{b2}
\textsc{Peck, J.} (1994):
``Competition in Transactions Mechanisms: The Emergence of Competition,''
Unpublished Manuscript, Ohio State University.
\endbibitem

\bibitem[\protect\citeauthoryear{Enelow and Hinich}{1990}]{b3}
\textsc{Enelow, J., and M. Hinich}, eds. (1990):
\textit{Advances in the Spatial Theory of Voting}.
Cambridge, U.K.: Cambridge University Press.
\endbibitem

\bibitem[\protect\citeauthoryear{Wittman}{1990}]{b4}
\textsc{Wittman, D.} (1990):
``Spatial Strategies when Candidates Have Policy Preferences,''
in \textit{Advances in the Spatial Theory of Voting},
ed. by M. Hinich and J. Enelow.
Cambridge, U.K.: Cambridge University Press, 66--98.
\endbibitem

\bibitem[\protect\citeauthoryear{Cahuc, Postel-Vinay and Robin}{2006}]{b5}
\textsc{Cahuc, P., F. Postel-Vinay, and J.-M. Robin} (2006): 
``Supplement to `Wage Bargaining with On-the-Job Search: Theory and Evidence',''
\textit{Econometrica Supplementary Material}, 74.
\endbibitem
\end{thebibliography}

\end{document}
