\documentclass[12pt,english]{article}
\usepackage{lmodern}
\linespread{1.05}
%\usepackage{mathpazo}
%\usepackage{mathptmx}
%\usepackage{utopia}
\usepackage{microtype}
\usepackage[section]{placeins}
\usepackage[T1]{fontenc}
\usepackage[latin9]{inputenc}
\usepackage[dvipsnames]{xcolor}
\usepackage{geometry}
\usepackage{amsthm}
\usepackage{amsfonts}
\usepackage{svg}
\usepackage{booktabs}
\usepackage{caption}
\usepackage{blindtext}
%\renewcommand{\arraystretch}{1.2}
\usepackage{multirow}
\usepackage{float}
\usepackage{rotating}

\usepackage{chngcntr}

% TikZ stuff

\usepackage{tikz}
\usepackage{mathdots}
\usepackage{yhmath}
\usepackage{cancel}
\usepackage{color}
\usepackage{siunitx}
\usepackage{array}
\usepackage{amssymb}
\usepackage{gensymb}
\usepackage{tabularx}
\usetikzlibrary{fadings}
\usetikzlibrary{patterns}
\usetikzlibrary{shadows.blur}

\usepackage[font=small]{caption}
%\usepackage[printfigures]{figcaps}
%\usepackage[nomarkers]{endfloat}


%\usepackage{caption}
%\captionsetup{justification=raggedright,singlelinecheck=false}

\usepackage{courier}
\usepackage{verbatim}
\usepackage[round]{natbib}
\bibliographystyle{plainnat}

\definecolor{red1}{RGB}{128,0,0}
%\geometry{verbose,tmargin=1.25in,bmargin=1.25in,lmargin=1.25in,rmargin=1.25in}
\geometry{verbose,tmargin=1in,bmargin=1in,lmargin=1in,rmargin=1in}
\usepackage{setspace}

\usepackage[colorlinks=true, linkcolor={red!70!black}, citecolor={blue!50!black}, urlcolor={blue!80!black}]{hyperref}
%\usepackage{esint}
\onehalfspacing
\usepackage{babel}
\usepackage{amsmath}
\usepackage{graphicx}

\theoremstyle{remark}
\newtheorem{remark}{Remark}
\begin{document}
	
\title{SIMPLE MODEL for Endogenous Growth with Creative Destruction by Employee Spinouts}
\author{Nicolas Fernandez-Arias}
\date{\today}
\maketitle

\section{Introduction}

In this document I describe a simple expository version of my model.  It can be viewed as a modification of the model presented in 14.3 of Acemoglu's growth textbook ("Innovation by Incumbents and Entrants").

\section{Individual endowments and preferences}

The representative household has CRRA preferences over consumption, 
\begin{align}
U_0 &= \int_0^{\infty} e^{-\rho t} \frac{C(t)^{1-\theta} - 1}{1-\theta} dt \label{preferences}
\end{align}

The representative household supplies labor inelastically and work in final good production $(L_F)$, intermediate good production $(L_I)$, and R\&D $(L_{RD})$. The resource constraint on labor supply is given by
\begin{align}
L_F + L_I + L_{RD} &= 1 \label{labor_resource_constraint}
\end{align}


\section{Static production of final and intermediate goods}

Below I suppress the $t$ subscript where it is clear. The final good $Y$ is produced competitively using labor and a continuum of intermediate goods indexed by $j \in [0,1]$, with production technology
\begin{align}
Y = F(L,\{q_j\},\{k_j\}) &= \frac{L^{\beta}}{1-\beta} \int_0^1 q_j^{\beta} k_j^{1-\beta} dj \label{final_goods_production}
\end{align}

where $q_j,k_j$ are the quality and quantity of intermediate input $j$. 
The price of the final good is normalized to 1 in every period. 

The final good production function assumes that only one technology for producing each good is used. This is without loss of generality because different versions of the leading edge good are perfect substitutes, intermediate goods producers engage in a form of Bertrand competition, and intermediate goods production functions are constant returns to scale. 

Second, intermediate goods are aggregated in CES form with elasticity of substitution greater than 1, rather than the Cobb-Douglas form in e.g., \cite{grossman_quality_1991} and \cite{baslandze_spinout_2019}. This reduces the complexity of the firm problem. In those models, Cobb-Douglas guarantees that equilibrium expenditure on each intermediate good does not depend on its quality. This requires limit pricing to be explicitly modeled; otherwise increasing the price always increases profits and the firm problem is not well-defined. To model limit pricing, one must track the gap between leader and follow in each line $j$, adding a state variable to the firm problem and to the aggregation of the model. In the current setup, by contrast, expenditure is decreasing in the price of the intermediate good, so even if one abstract from limit pricing (by assuming a two-stage competition with an entry fee -- details below), intermediate goods firms have a constant optimal markup. I take advantage of this reduced complexity by introducing more complexity in the employee spinout and firm entry process.

There is no storage technology for consumption, so in equilibrium
\begin{align*}
Y = C = \int_0^1 c_i di
\end{align*}

For each intermediate good $j$, the production technology is given by
\begin{align*}
k_j = H(l_j;Q) &= Q l_j
\end{align*}
where $l_j$ is the labor input and $Q = \int_0^1 q_j dj$ is the average quality level in the economy. 

\section{Innovation and spinouts}

The innovation part of this model is significantly simplified from the full model so that it admits a closed form solution. 

For each product $j$, the incumbent and a mass of ordinary entrants attempt innovation by doing R\&D. In addition, innovation by spinouts occurs in proportion to R\&D spending by the incumbent. Whenever an innovation occurs, the party responsible becomes the incumbent with quality $\lambda q_j$. 

\subsection{Incumbents}

The incumbent in line $j$ can perform $z$ units of R\&D by hiring $\frac{q_j}{Q}z$ units of R\&D labor. In return, the incumbent receives a Poisson intensity of $\chi_I z$ of innovating on product $j$. In the full model, the incumbent's innovation technology has decreasing returns. Here, I make this assumption for tractability reasons: the CRS specification allows one to immediately derive a closed form for the incumbent's value function, which enables closed form solutions for the rest of the model.\footnote{Alternatively, one could set up the model so that it is entrants who have a CRS innovation technology.}

\subsection{Generation of spinouts}

When an incumbent conducts $z$ flow units of R\&D, he faces a certain Poisson intensity of being replaced by a spinout firm, given by 
\begin{align*}
	\tau_{sjt} &= (1-x_{jt}) \nu z_{jt}
\end{align*} 
where $x_{jt} = 1$ if and only if a non-compete is imposed. Imposing a non-compete on a good of quality $q$ implies a flow cost of $\kappa_{c} V(q)$ units of the final good, where $V(q)$ is the incumbent value.\footnote{The purpose of the linear scaling with $V$ is so that the optimal non-compete policy can be derived independent of the endogenous value of $V$, allowing a closed-form solution.}

\paragraph{Value of future spinout formation}

If a spinout is formed, it is owned by the representative household. The household takes this into account when deciding where to allocate its labor, accepting a lower wage for R\&D labor as a result. When assessing this value, the household \textit{does not} take into account the fact that this spinout steals the profits of the previous incumbent, also owned by the household. 

While somewhat unrealistic, this setup is analogous to the assumption that incumbent firms and ordinary entrants (described below) are owned by the household but maximize their individual profits. It is also similar in spirit to the assumption of competitive labor markets, where an individual agent does not take into account the effect of his or her behavior on prices and thereby behaves suboptimally when viewed strictly as an individual agent.

\subsection{Ordinary entrants}

For each $j$ there is a unit mass of entrants indexed by $e \in [0,1]$ who each perform $z_e$ units of R\&D by hiring $\frac{q_j}{Q} z_e$ units of R\&D labor. Each entrant receives a Poisson intensity of $z \chi_E z_E^{-\psi}$ of innovating on product $j$, where $z_E \equiv \int_0^1 z_e de$. If a successful innovation arrives, an entrant becomes the new incumbent of good $j$, with quality $\lambda q_j$. I assume without loss of generality that $z_e \equiv z_E$ for all $e \in [0,1]$. Note that entrants have constant returns to scale individually but decreasing returns to scale at the level of good $j$. 


\subsection{Entry cost}

In addition to the R\&D costs of innovation, ordinary entrants and spinout must pay an entry cost $\kappa_{e} V(q_{jt})$ when an innovation is discovered, in order to become the incumbent. The interpretation of this cost can be either (1) the additional costs of building a business and finding customers that an incumbent would not need to incur, or (2) as a reduced form for lower markups in the industry during the battle for incumbency (which the entrant ultimately wins, due to higher fundamental quality). Economically, this parameter is crucial as it reduces the bilateral efficiency of spinouts, creating a role non-compete agreements. 


\section{Solving the model}

I will solve for a BGP of the above model with constant innovation effort by incumbents, entrants, constant innnovation rates by incumbents, entrants and spinouts; a constant growth rate $g$ and constant fractions of labor allocated to R\&D and production, and all wages increasing at the constant rate of productivity growth. I guess and verify that the incumbent value function takes the form $V(q,t) = \tilde{V}q$. 

\subsection{Static equilibrium}

Final goods producer optimization implies the following inverse demand function for intermediate goods, 
\begin{align*}
p_j &= L_F^{\beta} q_j^{\beta} k_j^{-\beta}	
\end{align*}

\paragraph{Quality gaps and limit pricing} Let $\bar{q}_j(t)$ denote the highest quality level of good $j$ available in the economy at time $t$. As mentioned before, typically there is limit pricing, and the markup charged by the technology leader in line $j$ would depend on his gap relative to the next laggard, e.g. \cite{baslandze_spinout_2019} or \cite{aghion_competition_2005}, only equating to the monopolistic competition markup for large enough gaps. In this model, I abstract from limit pricing (feasible due to the CES specification), using an assumption borrowed from \cite{akcigit_growth_2018}. At each time $t$, intermediate goods firms play a two-stage Bertrand competition game. In the first stage, participants bear a cost of $\varepsilon > 0$ units of the final good in exchange for a right to compete in the product market. In the second stage, they engage in Bertrand competition. Limit pricing in the second stage Bertrand game implies that all producers not on the frontier will earn zero profits; therefore, they do not pay the entry cost. 

In this setup, intermediate goods producers maximize profits according to
\begin{align}
\pi(q_j) = \max_{k_j \ge 0} \Big\{ L_F^{\beta} q_j^{\beta} k_j^{1-\beta} - \frac{\overline{w}}{Q} k_j \Big\} \label{incumbent_profit}
\end{align}

where $\overline{w}$ is the equilibrium final goods / intermediate goods wage.
This yields optimal pricing, labor demand and production of intermediate goods,
\begin{align}
k_j &= \Big[ \frac{(1-\beta) Q}{\overline{w}} \Big]^{1/\beta}L_F q_j  \label{optimal_k}\\
l_j &= k_j / Q \label{optimal_l}\\
p_j &= \frac{\overline{w}}{(1-\beta) Q} \label{optimal_p}
\end{align}

Substituting (\ref{optimal_k}) into the first-order condition for final goods firm optimal labor demand yields a closed form expression for the equilibrium wage $\overline{w}$:
\begin{align}
\overline{w} &= \tilde{\beta} Q \label{wbar} \\
\tilde{\beta} &= \beta^{\beta} (1-\beta)^{1-2\beta} \label{def_cbeta}
\end{align}

Substituting (\ref{optimal_k}) and (\ref{wbar}) into the expression for profit in (\ref{incumbent_profit}) yields
\begin{align}
\pi_j &= (1-\beta) \tilde{\beta} L_F q_j \label{profits_eq}
\end{align}

Substituting (\ref{optimal_k}) into (\ref{optimal_l}) and integrating $L_I = \int_0^1 l_j dj$ yields aggregate labor allocated to intermediate goods production,
\begin{align}
L_I &= \Big( \frac{1-\beta}{\tilde{\beta}} \Big)^{1 / \beta} L_F \label{intermediate_goods_labor}
\end{align}

and substituting (\ref{intermediate_goods_labor}) into the labor resource constraint (\ref{labor_resource_constraint}) yields
\begin{align}
L_F &= \frac{1 - L_{RD}}{1 + \Big(\frac{1-\beta}{\tilde{\beta}}\Big)^{1/\beta}}
\end{align}

The value of $L_{RD}$ is determined endogenously by incumbents' and entrants' innovation decisions, described in the next subsection. Output can be computed by substituting (\ref{optimal_k}) into (\ref{final_goods_production}), 
\begin{align}
Y = \frac{(1-\beta)^{1-2\beta}}{\beta^{1-\beta}} Q L_F \label{flow_output}
\end{align}

\subsection{Household optimization and non-competes}

The household chooses to supply labor to final goods production $L_F$, intermediate goods production $L_I$, and intermediate goods R\&D $L_{RD}$. The latter two are themselves aggregates over labor supplied to each good $j$,
\begin{align}
	L_I &= \int_0^1 \ell_{I,j} dj \\
	L_{RD} &= \int_0^1 \ell_{RD,j} dj
\end{align}



This implies an indifference condition analogous to that in the full model. In turn, given the incumbent HJB (solved below) this yields a condition
\begin{align}
	x^* = \begin{cases}
		1 & \textrm{if } (1-(1-\kappa_{e})\lambda)\nu > \kappa_{c}\\
		0 & \textrm{o.w.}
	\end{cases} \label{eq_nca_policy}
\end{align}

\subsection{Equilibrium innovation}

Fix all parameters except $\kappa_{c}$, such that $z_I > 0$ for all $\kappa_{c} \ge 0$, and consider $\kappa_{c}$ such that $x^* = 1$. 

Given that $x^* = 1$, the incumbent's HJB is given by 
\begin{align}
(r + \tau_E) \tilde{V} &= \tilde{\pi} + \max_{z \ge 0} \Big\{z \big(\chi_I (\lambda - 1) \tilde{V} - \bar{w} - \kappa_{c} \tilde{V}\big) \Big\} \label{eq:hjb_incumbent}
\end{align}

In an interior solution, the term multiplying $z$ in (\ref{eq:hjb_incumbent}) must be equal zero. Solving for $\tilde{V}$ yields
\begin{align}
	\tilde{V} &= \frac{\bar{w}}{\chi_I(\lambda - 1) - \kappa_{c}} \label{eq:hjb_incumbent_foc}
\end{align}

where 
\begin{align}
\bar{w} &= \tilde{\beta} Q \label{eq:wage_prod}\\
\tilde{\beta} &= \beta^{\beta}(1-\beta)^{1-2\beta} \label{eq:Kbeta}
\end{align}

Given $\tilde{V}$, entrant innovation is determined by the free entry condition
\begin{align}
	z_E^{-\psi} \chi_E \lambda (1-\kappa_{e}) \tilde{V} &= \bar{w}  \label{eq:free_entry}\\
	\Rightarrow z_E &= \Big( \frac{\chi_E (1-\kappa_{e}) \lambda \tilde{V}}{\bar{w}} \Big)^{1/\psi} \label{eq:effort_entrant}
\end{align}

Then substituting the FOC (\ref{eq:hjb_incumbent_foc}) into the incumbent HJB (\ref{eq:hjb_incumbent}) implies
\begin{align}
	(r + \tau_E) \tilde{V} &= \tilde{\pi} \label{eq:interest_rate}
\end{align}

The representative household's Euler equation then determines equilibrium growth $g$, 
\begin{align}
	g &= \frac{1}{\theta} (r - \rho) \label{eq:euler}
\end{align}

Finally, $z_I$ is determined by the growth accounting equation,
\begin{align}
	g &= (\lambda - 1)(\tau_I + \tau_E) \label{eq:growth_accounting}
\end{align}

where
\begin{align}
	\tau_I &= \chi_I z_I \label{eq:arrival_incumbent}\\
	\tau_E &= \chi_E z_E^{1-\psi} \label{eq:arrival_entrant}
\end{align}

Finally note that
\begin{align}
	\tilde{\pi} &= (1-\beta)\tilde{\beta}L_F \label{eq:profit}\\
	L_{RD} &= z_I + z_E  \label{eq:agg_labor_rd}\\
	L_F &= \frac{1-L_{RD}}{1+\big(\frac{1-\beta}{K(\beta}\big)^{1/\beta}} \label{eq:agg_labor_final_good}
\end{align}

\subsection{Solving the system of equations}

Now we have a system of equations defining an equilibrium, consisting of 12 equations (\ref{eq:hjb_incumbent_foc}), (\ref{eq:wage_prod}),(\ref{eq:Kbeta}), (\ref{eq:effort_entrant}), (\ref{eq:interest_rate}), (\ref{eq:euler}), (\ref{eq:growth_accounting}),(\ref{eq:arrival_incumbent}), (\ref{eq:arrival_entrant}), (\ref{eq:profit}), (\ref{eq:agg_labor_rd}), and (\ref{eq:agg_labor_final_good}). The 12 unknowns consist of \linebreak $\tilde{V},z_I,z_E,\tau_I,\tau_E,\bar{w},\tilde{\beta},\tilde{\pi}, L_F, L_{RD}, r, g$.\footnote{In addition, we must have $\rho > (1-\theta)g$ in equilibrium so that household utility is finite, and we must have $r > g$ in equilibrium so that the transversality condition is satisfied. (Make sure this is right...)}

This system is non-trivial to solve because profits $\tilde{\pi}$ depend on the labor allocation to final goods and vice versa. The former results directly from the equation (\ref{eq:profit}). At the same time, the latter effect is present because higher profits implies a higher interest rate via (\ref{eq:interest_rate}) and hence higher growth and R\&D labor allocation via the Euler equation (\ref{eq:euler}) and growth accounting equation (\ref{eq:growth_accounting}). Through the labor resource constraint equation (\ref{eq:agg_labor_final_good}), this implies a lower allocation of labor to final goods production. The solution to the system is a level of $L_{RD} \in (0,1)$ that balances these two forces. The guess of an interior BGP is verified provided that the solution has $z_I > 0$. 

The above reasoning leads to an equation characterizing $z_I$, 
\begin{align}
	\frac{1}{\theta} \Bigg( \frac{(1-\beta)\tilde{\beta}(1- z_I - z_E)}{\tilde{V}\Big(1 + \big(\frac{1-\beta}{\tilde{\beta}}\big)^{1/\beta}\Big)} - \tau_E - \rho \Bigg) &= (\lambda - 1)\big(\chi_Iz_I + \tau_E \big)
\end{align}

All variables other than $z_I$ in the above equation can be written directly in terms of parameters of the model. Solving the above equation for $z_I$ yields
\begin{align}
	z_I &= \frac{\theta^{-1}\Bigg( \frac{(1-\beta)\tilde{\beta}(1- z_E)}{\tilde{V}\big(1 + \big(\frac{1-\beta}{\tilde{\beta}}\big)^{1/\beta}\big)} - \tau_E - \rho \Bigg) - (\lambda-1) \tau_E}{(\lambda - 1) \chi_I + \frac{(1-\beta)\tilde{\beta}}{\theta \tilde{V}\big(1 + \big(\frac{1-\beta}{\tilde{\beta}}\big)^{1/\beta}\big)}} \label{eq:effort_incumbent}
\end{align}

\subsection{Solving the model}

Given $z_I,z_E$, $L_{RD}$ can be computed from (\ref{eq:agg_labor_rd}) and then $L_F$ by (\ref{eq:agg_labor_final_good}). Final good output is then given by
\begin{align}
	Y(Q) &= \frac{(1-\beta)^{1-2\beta}}{\beta^{1-\beta}} Q L_F \label{eq:agg_final_good_output}
\end{align}

Finally, total utility is given by 
\begin{align}
	W(1) &= \frac{\big(Y(1) - \tau_E \kappa_{e} \lambda \tilde{V} - z_I \kappa_{c} \tilde{V}\big)^{1-\theta}}{(1-\theta)(\rho - g(1-\theta))} - \frac{1}{(1-\theta)\rho} \label{eq:agg_welfare}
\end{align}

For welfare comparisons to be meaningful, they must be converted into consumption-equivalent (CEV) terms. For $\theta < 1$, a $\frac{x}{1-\theta}\%$ increase in CEV welfare results from a $x\%$ increase in the absolute value of the first term in (\ref{eq:agg_welfare}). For $\theta > 1$, a $\frac{x}{\theta-1}\%$ increase in CEV welfare results from an $x\%$ decrease in the absolute value of the same term. The case $\theta = 1$ corresponds to log utility, in which case
\begin{align}
	W(1) &= \frac{\rho \log(C_0) + g}{\rho^2} \label{eq:agg_welfare_log}
\end{align}

In this case, there is no simple correspondence to obtain CEV welfare changes, but they are easy to compute directly. Under the null policy, initial consumption is $C_0$ and growth is $g$. Under the new policy, initial consumption is $C_0^+$ and growth is $g^+$. The CEV welfare change is $\frac{C_0^* - C_0}{C_0}$, where $C_0^*$ is defined by 
\begin{align}
	\frac{\rho\log(C_0^*) + g}{\rho^2} = \frac{\rho \log(C_0^+) + g^+}{\rho^2} \label{eq:agg_welfare_log_CEV}
\end{align}

\subsection{Allowing $x$ to vary}

Consider $\kappa_{e}, \lambda, \nu$ and define the threshold $\bar{\kappa}_c$ by 
\begin{align}
	(1-(1-\kappa_{e})\lambda)\nu = \bar{\kappa}_c \label{eq_nca_threshold}
\end{align}

For $0 \le \kappa_{c} < \bar{\kappa}_c$, incumbents use non-competes and the results in the previous section apply.

For $\kappa_{c} \ge \bar{\kappa}_c$, incumbents do not use non-competes. The equilibrium is independent of the particular value of $\kappa_{c} > \bar{\kappa}_c$. The wage paid by incumbents for R\&D is determined by the household's indifference condition,
\begin{align}
	w = \bar{w} - \nu (1-\kappa_{e}) \lambda \tilde{V} \label{eq:wage_rd}
\end{align}

The incumbent's HJB is now given by 
\begin{align}
	(r + \tau_E) \tilde{V} &= \tilde{\pi} + \tau_I (\lambda - 1) \tilde{V} - z^*_I(\bar{w} - \nu (1-\kappa_{e}) \lambda \tilde{V} + \nu \tilde{V} ) \label{eq:hjb_incumbent_noNCA}
\end{align}

This implies a modified expression for the incumbent's value,
\begin{align}
\tilde{V} &= \frac{\bar{w}}{\chi_I(\lambda - 1) - (1-(1-\kappa_{e})\lambda)\nu} \label{eq:hjb_incumbent_foc_noNCA}
\end{align}

Note that (\ref{eq:hjb_incumbent_foc_noNCA}) is equal to (\ref{eq:hjb_incumbent_foc}) at $\kappa_{c} = \bar{\kappa}_c$. There is also a modified growth accounting equation,
\begin{align}
g &= (\lambda - 1)(\tau_I + \tau_S + \tau_E) \label{eq:growth_accounting_noNCA}
\end{align}
where
\begin{align}
	\tau_S &= \nu z_I \label{eq:arrival_spinout_noNCA}
\end{align}

Putting this together yields a modified equation for equilibrium $z_I$,
\begin{align}
z_I &= \frac{\theta^{-1}\Bigg( \frac{(1-\beta)\tilde{\beta}(1- z_E)}{\tilde{V}\big(1 + \big(\frac{1-\beta}{\tilde{\beta}}\big)^{1/\beta}\big)} - \tau_E - \rho \Bigg) - (\lambda-1) \tau_E}{(\lambda - 1) (\chi_I + \nu) + \frac{(1-\beta)\tilde{\beta}}{\theta \tilde{V}\big(1 + \big(\frac{1-\beta}{\tilde{\beta}}\big)^{1/\beta}\big)}}  \label{eq:effort_incumbent_noNCA}
\end{align}

Finally, the welfare equation is also modified, now given by 
\begin{align}
W(1) &= \frac{\big(Y(1) - (\tau_E + \tau_S) \kappa_{e} \lambda \tilde{V}\big)^{1-\theta}}{(1-\theta)(\rho - g(1-\theta))} - \frac{1}{(1-\theta)\rho}  \label{eq:agg_welfare_noNCA}
\end{align}

As before, welfare changes are converted into CEV terms.

\section{Policy counterfactual}

\subsection{Effect of the cost of non-competes on welfare}

For $\kappa_{c} \le \bar{\kappa}_c$, social welfare is strictly decreasing in $\kappa_{c}$. As $\kappa_{c}$ increases, enforcing non-competes is more expensive. To satisfy the incumbent FOC, the value of incumbency in must increase as given in (\ref{eq:hjb_incumbent_foc}). Holding constant the labor allocation, the equilibrium interest rate falls to satisfy the incumbent's HJB (\ref{eq:hjb_incumbent}). By the Euler equation (\ref{eq:euler}), the equilibrium growth rate declines. The innovation rate by ordinary entrants necessarily increases due to the increased value of incumbency. The growth accounting equation (\ref{eq:growth_accounting}) then implies that incumbent innovation expenditures decline. Therefore, growth declines and a larger share of innovation expenditures is accounted by ordinary entrants, which are inefficient compared to incumbents. Furthermore, their innovation expenditures strictly increase. The net effect on the share of labor allocated to R\&D depends on parameters. If the share of labor allocated to R\&D falls as $\kappa_{c}$ rises, equilibrium profits increase and the effects above are attenuated. Otherwise, they are amplified. 

The effects of the above on welfare are unambiguously negative because growth declines, innovation expenditures by ordinary entrants increase and innovation expenditures by incumbents decrease. In this model, the decentralized equilibrium can exhibit too much growth. However, this is due to excessive innovation by entrants. Therefore, it is not possible to increase welfare by increase innovation by entrants and simultaneously reducing growth. \textbf{[Need to think about this argument...]}

When $\kappa_{c} > \bar{\kappa}_c$, the equilibrium does not depend on the particular value of $\kappa_{c}$. Upon crossing the threshold, welfare necessarily rises. The incumbent value and therefore innovation expenditures by ordinary entrants are constant for all $\kappa_{c} \ge \bar{\kappa}_c$. Holding constant the labor allocation to R\&D, the interest rate is unchanged. The Euler equation therefore implies the same growth rate. The growth accounting equation them implies a $z_I$ such that $\tau_I + \tau_S$ is the same as for $\kappa_{c} = \bar{\kappa}_c$. This requires lower $z_I$, which reduces the allocation of labor to R\&D. This feeds back into the incumbent HJB, requiring a higher interest rate. The Euler equation then implies a higher growth rate, which requires a higher incumbent R\&D expenditure. This increases the allocation of labor to R\&D, attenuating the effect. This logic has a fixed point at the new equilibrium. This new equilibrium has lower incumbent R\&D expenditure, the same entrant R\&D expenditure, and a higher growth rate. However, the above logic does not imply that welfare increases or decreases overall because I have no considered the net effect of (1) eliminating expenditures on enforcing non-competes and (2) increasing creative destruction expenditures.

The difference in welfare is
\begin{align}
	W(1;\bar{\kappa}_c + \epsilon) - W(1;\bar{\kappa}_c) &= \frac{\big(Y(1)^+ - (\tau_E^+ + \nu z_I^+) \kappa_{e} \lambda \tilde{V}^+\big)^{1-\theta}}{(1-\theta)(\rho - g^+(1-\theta))} - \frac{\big(Y(1) - \tau_E \kappa_{e} \lambda \tilde{V} - z_I \bar{\kappa}_c \tilde{V}\big)^{1-\theta}}{(1-\theta)(\rho - g(1-\theta))}
\end{align}

The arguments above amount to $Y(1)^+ > Y(1)$, $\tilde{V}^+ = \tilde{V}$, $z_I^+ < z_I$, $\tau_E^+ = \tau_E$, and $g^+ > g$. Given this, a sufficient condition for the above difference to be positive is 
\begin{align*}
\nu z_I^+ \kappa_e \lambda <  z_I \bar{\kappa}_c
\end{align*}

Since $z_I^+ < z_I$, a sufficient condition is
\begin{align}
	\nu \kappa_e \lambda \le \bar{\kappa}_c  \label{sufficient_condition_welfareBump_1}
\end{align}

We know that $\bar{\kappa}_c$ is defined by (\ref{eq_nca_threshold}), which after rearranging becomes 
\begin{align*}
	\bar{\kappa}_c &= (1 - (1-\kappa_e)\lambda) \nu 
\end{align*}

Substituting into (\ref{sufficient_condition_welfareBump_1}) yields
\begin{align}
	\kappa_e \lambda &\le 1 - (1-\kappa_e) \lambda \\
	                 &= 1 - \lambda + \kappa_e \lambda 
\end{align}

Thus, this inequality is impossible if $\lambda > 1$, which is assumed. It turns out that this is not a large 







\bibliography{references.bib}




\end{document}