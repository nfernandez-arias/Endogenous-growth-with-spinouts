\documentclass[english,usenames,dvipsnames]{beamer}
\usetheme{default}
\beamertemplatenavigationsymbolsempty
\setbeamertemplate{footline}[frame number]
\setbeamercolor{alerted text}{fg=blue1}
%\setbeamercolor{frametitle}{fg=blue2}
\usepackage[utf8]{inputenc}
\usepackage{caption}
\usepackage{booktabs}
\usepackage{appendixnumberbeamer}
\usepackage{babel}
\usepackage{amsmath}
\usepackage{hyperref}
\usepackage{geometry}
\usepackage{bbm}
\usepackage{amsthm}
\usepackage{verbatim}
%\usepackage{palatino}
\definecolor{red1}{RGB}{255,50,0}
\definecolor{blue1}{RGB}{80,80,255}
\definecolor{blue2}{rgb}{0.22,0.37,1}
\definecolor{green1}{RGB}{34,139,35}

\setbeamertemplate{itemize items}[default]


\title{Employee Spinouts, Creative Destruction and Endogenous Growth}
\author{Nicolas Fernandez-Arias}
%\date{March 7, 2019}

\begin{document}

\maketitle

\begin{frame}{Motivation}
\begin{itemize}
	\item Entry contributes significantly to productivity growth
	\begin{itemize}
		\item Over 10-year horizon, 25\% of labor productivity growth accounted by entry in manufacturing (Baily-Bartelsman-Haltiwanger 1996)
		\item ~25\% of aggregate productivity growth due to entrants (Akcigit-Kerr 2017)
	\end{itemize}
	\item Spinouts: entrants whose founders use human capital developed at previous employer
	\begin{itemize}
		\item All Spinouts: more patents / R\&D, sales growth, survival (Baslandze 2019) 
		\item Same-industry spinouts: 15-30\% of entrants; larger, grow faster, higher survival rates (Muendler et al. 2012, Brazilian data)
		\item Fairchild semiconductor / Silicon Valley (Saxenian 1994); Detroit automakers
		\item Modern high-profile examples: Tinder $\rightarrow$ Bumble; Ableton $\rightarrow$ Bitwig Studio
	\end{itemize}
\end{itemize}
\end{frame}

\begin{frame}
\begin{itemize}
	\item Non-competes policy debate
	\begin{itemize}
		\item Non-competes restrict spinouts
		\item Many states passing laws restricting enforcement of noncompetes (Hawaii in 2015, Massachussetts and Maryland in 2019, many others)
		\item Attempt to emulate Silicon Valley (very low enforcement)
	\end{itemize}
\end{itemize}
\end{frame}

\begin{frame}{Spinouts of Fairchild Semiconductor}
\begin{figure}
	\includegraphics[scale=0.34]{../figures/fairchildren_early.png}
	\caption{Source: Endeavor Insights}
\end{figure}
\end{frame}

\begin{frame}{Motivation - Theory}
\label{theory_big_picture}
\begin{itemize}
	\item Schumpeter 1942, Arrow 1962, Romer 1986, Grossman-Helpman 1991, etc.: \alert{underinvestment} in knowledge due to \alert{limited excludability}
	\item Patent literature: dynamic efficiency vs. static monopoly distortion tradeoff
	\item Creative destruction by spinouts similar tradeoff
	\item Should we encourage or discourage spinout formation?
	\item Existing frameworks (e.g., Franco-Filson 2006, Baslandze 2019) underemphasize disincentive for firm R\&D 
\end{itemize}
\end{frame}

\begin{frame}{Related literature}
\begin{itemize}
	\item Firm dynamics and endogenous growth
	\begin{itemize}
		\item Romer 1986, Grossman \& Helpman 1991, Aghion \& Howitt 1992, Klette \& Kortum 2004, Acmemoglu \& Akcigit 2012, Akcigit \& Kerr 2017
	\end{itemize}
	\item Models of employee spinouts
	\begin{itemize}
		\item Klepper 2002, Klepper \& Sleeper 2005, Franco \& Filson 2006, Franco \& Mitchell 2008, Rauch 2015, Rossi-Hansberg \& Chatterjee 2012
		\item Baslandze 2019
	\end{itemize}
	\item Empirics on employee mobility, spinouts
	\begin{itemize}
		\item Spawning of spinouts: Gompers et al. 2005, Garmaise 2011, Baslandze 2019
		\item Effect on parent firms: Campbell et. al 2012, Wezel et al. 2006
		\item Effect of non-compete enforcement: Garmaise 2009, Marx et al 2009, Samila-Sorenson 2011, Jeffers 2018, Shi 2018
	\end{itemize}
\end{itemize}
\end{frame}

\begin{frame}{Overview and results}
\begin{itemize}
	\item Empirics of spinouts and WSOs
	\begin{itemize}
		\item Spinouts, WSOs are more innovative than ordinary firms
		\item R\&D leads to formation of more spinouts, WSOs
	\end{itemize}
	\item Develop GE model of endogenous growth with creative destruction by employee spinouts
	\item Calibrate model to match empirics, aggregate data
	\item Policy exercise: allowing noncompetes increases PDV of consumption by 10\%
	\begin{itemize}
		\item 20\% higher R\&D spending 
		\item 27\% higher growth rate on BGP
		\item Little effect on entry rate
	\end{itemize}
\end{itemize}
\end{frame}

\begin{frame}{Data - Venture Source}
\begin{itemize}
	\item Venture Source
	\begin{itemize}
		\item Data on startups funded by VCs from 1986-2015: about 40,000 startups
		\item Also includes funding by PEs and IPOs and acquisitions, and business status (e.g. product development, earning revenue, profitable)
		\item Key feature: \alert{employment biographies} for founders / C-level / board members
		\item Firm data: employment, valuation, revenue, M\&A / IPO, bankruptcy / exit
	\end{itemize}
\end{itemize}
\end{frame}

\begin{frame}{Data - merging with Compustat and patent data}
\begin{itemize}
	\item Merge Venture Source with Compustat
	\begin{itemize}
		\item Consider President, CEO, Chairman, CTOs and flagged "Founders" at startups
		\item Parse previous employer from employment biography
		\item No company identifier: need to match to Compustat by company name
		\item Difficult due to misspellings, informal names (e.g. acronyms) and subsidiaries
		\item \alert{Solution:} regular expressions + merchant-mapper tool by AltDG (designed for linking credit card transaction data to firms)
		\item Merge with Compustat by ticker symbol
	\end{itemize}
	\item NBER-USPTO patent data
	\begin{itemize}
		\item Data on all USPTO-registered patents and their citations (also data on inventors, associated firms)
		\item Merge to Compustat using gvkey
	\end{itemize}
\end{itemize}
\end{frame}

\begin{frame}{Previous employers and positions}
	\begin{figure}
		\includegraphics[scale=0.25]{figures/presentation/titleCounts.png}
	\end{figure}
\end{frame}

\begin{frame}{Results of match to Compustat}
\begin{figure}
	\includegraphics[scale=0.3]{figures/presentation/table1_founder2.png}
\end{figure}
\end{frame}

\begin{frame}{$\mathrm{Pr}(\textrm{Child industry} | \textrm{Parent industry})$}
\begin{figure}
	\includegraphics[scale=0.39]{../empirics/figures/plots/industry_row_heatmap_naics2_founder2.pdf}
	\caption{\footnotesize Heatmap displaying the distribution of child 2-digit NAICS code (column), conditional on parent 2-digit NAICS code (row).}
\end{figure}
\end{frame}

\begin{frame}{$\mathrm{Pr}(\textrm{Parent industry} | \textrm{Child Industry})$}
\begin{figure}
	\includegraphics[scale=0.39]{../empirics/figures/plots/industry_column_heatmap_naics2_founder2.pdf}
	\caption{\footnotesize Heatmap displaying the distribution of parent 2-digit NAICS code (row), conditional on child 2-digit NAICS code (column).}
\end{figure}
\end{frame}

\begin{frame}{$\mathrm{Pr}(\textrm{Child state} | \textrm{Parent state})$}
\begin{figure}
	\includegraphics[scale=0.39]{../empirics/figures/plots/state_row_heatmap_founder2.pdf}
	\caption{\footnotesize Heatmap displaying the distribution of child state (column), conditional on parent state (row).}
\end{figure}
\end{frame}

\begin{frame}{$\mathrm{Pr}(\textrm{Parent state} | \textrm{Child state})$}
\begin{figure}
\includegraphics[scale=0.39]{../empirics/figures/plots/state_column_heatmap_founder2.pdf}
\caption{\footnotesize Heatmap displaying the distribution of parent state (row), conditional on child state (column).}
\end{figure}
\end{frame}

\begin{frame}{Spinouts, WSOs employment}
	\begin{figure}
		\centering
		\includegraphics[scale= 0.45]{../empirics/figures/plots/startupLifeCycle_founder2_avg_SvNoS_WSO4vNoWSO4_lEmployeeCount.pdf}
		\caption{Comparison of age-employment distribution: spinouts vs non-spinouts, WSO spinouts vs non-WSO spinouts.}
	\end{figure}
\end{frame}

\begin{frame}{Spinouts, WSOs have higher employment}
	\begin{table}
		\tiny
		\centering
		{
\def\sym#1{\ifmmode^{#1}\else\(^{#1}\)\fi}
\begin{tabular}{l*{4}{c}}
\hline\hline
                    &\multicolumn{1}{c}{(1)}         &\multicolumn{1}{c}{(2)}         &\multicolumn{1}{c}{(3)}         &\multicolumn{1}{c}{(4)}         \\
\hline
Spinout=1           &        0.31\sym{***}&        0.21\sym{***}&        0.22\sym{***}&        0.20\sym{***}\\
                    &     (0.038)         &     (0.011)         &     (0.017)         &     (0.013)         \\
[1em]
WSO4=1              &       0.067         &        0.13\sym{***}&        0.13\sym{***}&        0.13\sym{***}\\
                    &      (0.15)         &    (0.0056)         &   (0.00040)         &    (0.0072)         \\
[1em]
Constant            &        3.14\sym{***}&        3.09\sym{***}&        3.12\sym{***}&        3.10\sym{***}\\
                    &     (0.075)         &    (0.0039)         &    (0.0052)         &     (0.012)         \\
[1em]
State-NAICS4-Year FE&          No         &         Yes         &         Yes         &          No         \\
[1em]
State-NAICS4-Age FE &          No         &         Yes         &          No         &         Yes         \\
[1em]
State-NAICS4-Cohort FE&          No         &          No         &         Yes         &         Yes         \\
[1em]
No FE               &         Yes         &          No         &          No         &          No         \\
\hline
r2\_a                &       0.012         &        0.39         &        0.46         &        0.44         \\
r2\_a\_within         &       0.012         &       0.015         &       0.014         &       0.013         \\
N                   &       54424         &       43323         &       44124         &       43802         \\
\hline\hline
\multicolumn{5}{l}{\footnotesize Standard errors in parentheses}\\
\multicolumn{5}{l}{\footnotesize \sym{*} \(p<0.1\), \sym{**} \(p<0.05\), \sym{***} \(p<0.01\)}\\
\end{tabular}
}

		\caption{\footnotesize The regresssions above compare \textbf{employment} in WSO4 spinouts, non-WSO4 spinouts and non-spinouts. The first regression uses no controls. The following three regressions in addition control for year effects, age effects, and / or cohort effects, in each case allowing the relevant effect to differ by State-NAICS4 combination. Standard errors are multi-way clustered at the state, NAICS4 and year levels.}
	\end{table}
\end{frame}

\begin{frame}{Spinouts, WSOs valuations}
\begin{figure}
	\centering
	\includegraphics[scale= 0.45]{../empirics/figures/plots/startupLifeCycle_founder2_avg_SvNoS_WSO4vNoWSO4_lPostValUSD.pdf}
	\caption{Comparison of age-valuation distribution: spinouts vs non-spinouts, WSO spinouts vs non-WSO spinouts.}
\end{figure}
\end{frame}

\begin{frame}{Spinouts, WSOs have higher valuations}
	\begin{table}
		\tiny
		\centering
		{
\def\sym#1{\ifmmode^{#1}\else\(^{#1}\)\fi}
\begin{tabular}{l*{4}{c}}
\hline\hline
                    &\multicolumn{1}{c}{(1)}         &\multicolumn{1}{c}{(2)}         &\multicolumn{1}{c}{(3)}         &\multicolumn{1}{c}{(4)}         \\
\hline
Spinout=1           &        0.28\sym{***}&        0.20\sym{***}&        0.18\sym{***}&        0.18\sym{***}\\
                    &     (0.066)         &     (0.043)         &     (0.037)         &     (0.046)         \\
[1em]
WSO4=1              &        0.15\sym{***}&        0.15\sym{***}&        0.13\sym{***}&        0.10\sym{***}\\
                    &     (0.047)         &     (0.017)         &   (0.00065)         &     (0.012)         \\
[1em]
Constant            &        3.51\sym{***}&        3.55\sym{***}&        3.59\sym{***}&        3.56\sym{***}\\
                    &      (0.10)         &     (0.014)         &     (0.014)         &     (0.055)         \\
[1em]
State-NAICS4-Year FE&          No         &         Yes         &         Yes         &          No         \\
[1em]
State-NAICS4-Age FE &          No         &         Yes         &          No         &         Yes         \\
[1em]
State-NAICS4-Cohort FE&          No         &          No         &         Yes         &         Yes         \\
[1em]
No FE               &         Yes         &          No         &          No         &          No         \\
\hline
r2\_a                &       0.011         &        0.31         &        0.31         &        0.29         \\
r2\_a\_within         &       0.011         &       0.012         &      0.0085         &      0.0069         \\
N                   &       26504         &       19566         &       19845         &       20823         \\
\hline\hline
\multicolumn{5}{l}{\footnotesize Standard errors in parentheses}\\
\multicolumn{5}{l}{\footnotesize \sym{*} \(p<0.1\), \sym{**} \(p<0.05\), \sym{***} \(p<0.01\)}\\
\end{tabular}
}

		\caption{\footnotesize The regresssions above compare \textbf{valuation} in WSO4 spinouts, non-WSO4 spinouts and non-spinouts. The first regression uses no controls. The following three regressions in addition control for year effects, age effects, and / or cohort effects, in each case allowing the relevant effect to differ by State-NAICS4 combination. Standard errors are multi-way clustered at the state, NAICS4 and year levels.}
	\end{table}
\end{frame}

\begin{frame}{Spinouts, WSOs revenue}
\begin{figure}
	\centering
	\includegraphics[scale= 0.45]{../empirics/figures/plots/startupLifeCycle_founder2_avg_SvNoS_WSO4vNoWSO4_lrevenue.pdf}
	\caption{Comparison of age-revenue distribution: spinouts vs non-spinouts, WSO spinouts vs non-WSO spinouts.}
\end{figure}
\end{frame}

\begin{frame}{Spinouts, WSOs have higher revenue}
\begin{table}
\tiny
\centering
{
\def\sym#1{\ifmmode^{#1}\else\(^{#1}\)\fi}
\begin{tabular}{l*{4}{c}}
\hline\hline
                    &\multicolumn{1}{c}{(1)}         &\multicolumn{1}{c}{(2)}         &\multicolumn{1}{c}{(3)}         &\multicolumn{1}{c}{(4)}         \\
\hline
WSO4                &       -0.12         &        0.43\sym{***}&        0.42\sym{***}&        0.39\sym{***}\\
                    &      (0.16)         &      (0.14)         &      (0.13)         &      (0.14)         \\
[1em]
Constant            &        1.90\sym{***}&        1.80\sym{***}&        1.84\sym{***}&        1.83\sym{***}\\
                    &      (0.14)         &     (0.010)         &    (0.0096)         &     (0.010)         \\
[1em]
State FE            &          No         &         Yes         &         Yes         &         Yes         \\
[1em]
NAICS4-Year FE      &          No         &         Yes         &         Yes         &          No         \\
[1em]
NAICS4-Age FE       &          No         &         Yes         &          No         &         Yes         \\
[1em]
NAICS4-Cohort FE    &          No         &          No         &         Yes         &         Yes         \\
[1em]
No FE               &         Yes         &          No         &          No         &          No         \\
\hline
clustvar            &statecode naics1\_4 year         &statecode naics1\_4         &statecode naics1\_4         &statecode naics1\_4         \\
r2\_a                &     0.00011         &        0.31         &        0.36         &        0.36         \\
r2\_a\_within         &     0.00011         &      0.0028         &      0.0027         &      0.0022         \\
N                   &       17838         &       16891         &       16875         &       17134         \\
\hline\hline
\multicolumn{5}{l}{\footnotesize Standard errors in parentheses}\\
\multicolumn{5}{l}{\footnotesize \sym{*} \(p<0.1\), \sym{**} \(p<0.05\), \sym{***} \(p<0.01\)}\\
\end{tabular}
}

\caption{\footnotesize The regresssions above compare \textbf{revenue} in WSO4 spinouts, non-WSO4 spinouts and non-spinouts. The first regression uses no controls. The following three regressions in addition control for year effects, age effects, and / or cohort effects, in each case allowing the relevant effect to differ by State-NAICS4 combination. Standard errors are multi-way clustered at the state, NAICS4 and year levels.}
\end{table}
\end{frame}

\begin{frame}{Spinouts, WSOs M\&A / IPO hazard}
\begin{figure}
	\centering
	\includegraphics[scale= 0.45]{../empirics/figures/plots/startupLifeCycle_founder2_avg_SvNoS_WSO4vNoWSO4_successfullyExiting.pdf}
	\caption{Comparison of age-M\&A and IPO hazard rate distribution: spinouts vs non-spinouts, WSO spinouts vs non-WSO spinouts.}
\end{figure}
\end{frame}

\begin{frame}{Spinouts, WSOs have higher M\&A / IPO hazard}
\begin{table}
\tiny
\centering
{
\def\sym#1{\ifmmode^{#1}\else\(^{#1}\)\fi}
\begin{tabular}{l*{4}{c}}
\toprule
                    &\multicolumn{1}{c}{(1)}         &\multicolumn{1}{c}{(2)}         &\multicolumn{1}{c}{(3)}         &\multicolumn{1}{c}{(4)}         \\
\midrule
$\frac{\text{WSO4 founders}}{\text{Total founders}}$&        2.52\sym{***}&        2.19\sym{***}&        2.03\sym{***}&        2.01\sym{***}\\
                    &     (0.056)         &      (0.14)         &      (0.18)         &      (0.18)         \\
\addlinespace
State-Year FE       &          No         &         Yes         &         Yes         &          No         \\
\addlinespace
State-Age FE        &          No         &         Yes         &          No         &         Yes         \\
\addlinespace
State-Cohort FE     &          No         &          No         &         Yes         &         Yes         \\
\addlinespace
NAICS4-Year FE      &          No         &         Yes         &         Yes         &          No         \\
\addlinespace
NAICS4-Age FE       &          No         &         Yes         &          No         &         Yes         \\
\addlinespace
NAICS4-Cohort FE    &          No         &          No         &         Yes         &         Yes         \\
\midrule
Clustering          &State, Industry         &State, Industry         &State, Industry         &State, Industry         \\
R-squared (adj.)    &     0.00046         &       0.035         &       0.033         &       0.035         \\
R-squared (within, adj)&     0.00046         &     0.00034         &     0.00027         &     0.00027         \\
Observations        &      240155         &      239696         &      239788         &      239959         \\
\bottomrule
\multicolumn{5}{l}{\footnotesize Standard errors in parentheses}\\
\multicolumn{5}{l}{\footnotesize \sym{*} \(p<0.1\), \sym{**} \(p<0.05\), \sym{***} \(p<0.01\)}\\
\end{tabular}
}

\caption{\footnotesize The regresssions above compare the \textbf{M\&A and IPO hazard rate} in WSO4 spinouts, non-WSO4 spinouts and non-spinouts. The first regression uses no controls. The following three regressions in addition control for year effects, age effects, and / or cohort effects, in each case allowing the relevant effect to differ by State-NAICS4 combination. Standard errors are multi-way clustered at the state, NAICS4 and year levels.}
\end{table}
\end{frame}

\begin{frame}{Spinouts, WSOs failure hazard}
\begin{figure}
	\centering
	\includegraphics[scale= 0.45]{../empirics/figures/plots/startupLifeCycle_founder2_avg_SvNoS_WSO4vNoWSO4_goingOutOfBusiness.pdf}
	\caption{Comparison of age-failure hazard rate distribution: spinouts vs non-spinouts, WSO spinouts vs non-WSO spinouts.}
\end{figure}
\end{frame}

\begin{frame}{Spinouts, WSOs have lower failure hazard}
\begin{table}
\tiny
\centering
{
\def\sym#1{\ifmmode^{#1}\else\(^{#1}\)\fi}
\begin{tabular}{l*{4}{c}}
\toprule
                    &\multicolumn{1}{c}{(1)}         &\multicolumn{1}{c}{(2)}         &\multicolumn{1}{c}{(3)}         &\multicolumn{1}{c}{(4)}         \\
\midrule
$\frac{\text{WSO4 founders}}{\text{Total founders}}$&       -0.16         &       -0.57\sym{***}&       -0.54\sym{***}&       -0.56\sym{***}\\
                    &      (0.23)         &      (0.12)         &      (0.11)         &     (0.099)         \\
\addlinespace
Constant            &        1.59\sym{***}&        1.61\sym{***}&        1.61\sym{***}&        1.61\sym{***}\\
                    &      (0.35)         &    (0.0013)         &    (0.0023)         &    (0.0016)         \\
\addlinespace
State-Year FE       &          No         &         Yes         &         Yes         &          No         \\
\addlinespace
State-Age FE        &          No         &         Yes         &          No         &         Yes         \\
\addlinespace
State-Cohort FE     &          No         &          No         &         Yes         &         Yes         \\
\addlinespace
NAICS4-Year FE      &          No         &         Yes         &         Yes         &          No         \\
\addlinespace
NAICS4-Age FE       &          No         &         Yes         &          No         &         Yes         \\
\addlinespace
NAICS4-Cohort FE    &          No         &          No         &         Yes         &         Yes         \\
\addlinespace
No FE               &         Yes         &          No         &          No         &          No         \\
\midrule
Clustering          &statecode naics1\_4 year         &statecode naics1\_4         &statecode naics1\_4         &statecode naics1\_4         \\
R-squared (adj.)    &   0.0000015         &       0.030         &       0.032         &       0.017         \\
R-squared (within, adj)&   0.0000015         &    0.000065         &    0.000054         &    0.000057         \\
Observations        &      251910         &      251460         &      251552         &      251710         \\
\bottomrule
\multicolumn{5}{l}{\footnotesize Standard errors in parentheses}\\
\multicolumn{5}{l}{\footnotesize \sym{*} \(p<0.1\), \sym{**} \(p<0.05\), \sym{***} \(p<0.01\)}\\
\end{tabular}
}

\caption{\footnotesize The regresssions above compare the \textbf{failure rate} in WSO4 spinouts, non-WSO4 spinouts and non-spinouts. The first regression uses no controls. The following three regressions in addition control for year effects, age effects, and / or cohort effects, in each case allowing the relevant effect to differ by State-NAICS4 combination. Standard errors are multi-way clustered at the state, NAICS4 and year levels.}
\end{table}
\end{frame}


\begin{frame}{Corporate R\&D and spinout formation}
	\begin{figure}[!htb]
		\centering
		\includegraphics[scale= 0.4]{../empirics/figures/scatterPlot_RD-Founders_dIntersection.png}
		\caption{Scatterplot of average yearly founder counts in $t+1,t+2,t+3$ versus average yearly R\&D spending in $t,t-1,t-2$.}
	\end{figure}
\end{frame}

\begin{frame}{Corporate R\&D and WSO formation}

\begin{figure}[!htb]
	\centering
	\includegraphics[scale= 0.4]{../empirics/figures/scatterPlot_RD-FoundersWSO4_dIntersection.png}
	\caption{Scatterplot of average yearly WSO4 founder counts in $t+1,t+2,t+3$ versus average yearly R\&D spending in $t,t-1,t-2$.}
	\end{figure}
\end{frame}



\begin{frame}{Corporate R\&D and spinout formation regression: specification 1}
	\begin{table}
		\Tiny
		\centering
		{
\def\sym#1{\ifmmode^{#1}\else\(^{#1}\)\fi}
\begin{tabular}{l*{8}{c}}
\toprule
                    &\multicolumn{1}{c}{(1)}&\multicolumn{1}{c}{(2)}&\multicolumn{1}{c}{(3)}&\multicolumn{1}{c}{(4)}&\multicolumn{1}{c}{(5)}&\multicolumn{1}{c}{(6)}&\multicolumn{1}{c}{(7)}&\multicolumn{1}{c}{(8)}\\
                    &\multicolumn{1}{c}{Founders}&\multicolumn{1}{c}{Founders}&\multicolumn{1}{c}{Founders}&\multicolumn{1}{c}{Founders}&\multicolumn{1}{c}{WSO4}&\multicolumn{1}{c}{WSO4}&\multicolumn{1}{c}{WSO4}&\multicolumn{1}{c}{WSO4}\\
\midrule
R\&D                &        0.34\sym{**} &        0.73\sym{***}&        0.73\sym{***}&        0.63\sym{***}&        0.19\sym{***}&        0.32\sym{***}&        0.31\sym{***}&        0.28\sym{***}\\
                    &      (0.13)         &      (0.24)         &      (0.23)         &      (0.14)         &     (0.045)         &     (0.067)         &     (0.065)         &     (0.050)         \\
\addlinespace
No FE               &         Yes         &          No         &          No         &          No         &         Yes         &          No         &          No         &          No         \\
\addlinespace
Firm FE             &          No         &         Yes         &         Yes         &         Yes         &          No         &         Yes         &         Yes         &         Yes         \\
\addlinespace
Year FE             &          No         &         Yes         &          No         &          No         &          No         &         Yes         &          No         &          No         \\
\addlinespace
Age FE              &          No         &          No         &         Yes         &          No         &          No         &          No         &         Yes         &          No         \\
\addlinespace
Industry-Age FE     &          No         &          No         &          No         &         Yes         &          No         &          No         &          No         &         Yes         \\
\addlinespace
Industry-Year FE    &          No         &          No         &         Yes         &          No         &          No         &          No         &         Yes         &          No         \\
\addlinespace
State-Year FE       &          No         &          No         &         Yes         &          No         &          No         &          No         &         Yes         &          No         \\
\addlinespace
Industry-State-Year FE&          No         &          No         &          No         &         Yes         &          No         &          No         &          No         &         Yes         \\
\midrule
r2\_a                &        0.24         &        0.69         &        0.69         &        0.75         &        0.21         &        0.65         &        0.64         &        0.61         \\
r2\_a\_within         &        0.24         &        0.26         &        0.26         &        0.24         &        0.21         &        0.25         &        0.23         &        0.15         \\
N                   &       65009         &       63732         &       62211         &       37810         &       65009         &       63732         &       62211         &       37810         \\
\bottomrule
\multicolumn{9}{l}{\footnotesize Standard errors in parentheses}\\
\multicolumn{9}{l}{\footnotesize \sym{*} \(p<0.1\), \sym{**} \(p<0.05\), \sym{***} \(p<0.01\)}\\
\end{tabular}
}

		\caption{\tiny The dependent variable is average yearly number of founders joining startups in years $t+1,t+2,t+3$. The independent variables are averages over $t,t-1,t-2$. Firm controls are employment, assets, intangible assets, investment, net income, cumulative citation-weighted patents, and the product of Tobin's Q and Assets. Standard errors are clustered by firm.}
	\end{table}
\end{frame}

\begin{frame}{Corporate R\&D and spinout formation regression: specification 2}
\begin{table}
	\Tiny
	\centering
	{
\def\sym#1{\ifmmode^{#1}\else\(^{#1}\)\fi}
\begin{tabular}{l*{8}{c}}
\toprule
                    &\multicolumn{1}{c}{(1)}&\multicolumn{1}{c}{(2)}&\multicolumn{1}{c}{(3)}&\multicolumn{1}{c}{(4)}&\multicolumn{1}{c}{(5)}&\multicolumn{1}{c}{(6)}&\multicolumn{1}{c}{(7)}&\multicolumn{1}{c}{(8)}\\
                    &\multicolumn{1}{c}{$\frac{\textrm{Founders}}{\textrm{Assets}}$}&\multicolumn{1}{c}{$\frac{\textrm{Founders}}{\textrm{Assets}}$}&\multicolumn{1}{c}{$\frac{\textrm{Founders}}{\textrm{Assets}}$}&\multicolumn{1}{c}{$\frac{\textrm{Founders}}{\textrm{Assets}}$}&\multicolumn{1}{c}{$\frac{\textrm{WSO4}}{\textrm{Assets}}$}&\multicolumn{1}{c}{$\frac{\textrm{WSO4}}{\textrm{Assets}}$}&\multicolumn{1}{c}{$\frac{\textrm{WSO4}}{\textrm{Assets}}$}&\multicolumn{1}{c}{$\frac{\textrm{WSO4}}{\textrm{Assets}}$}\\
\midrule
$\frac{\textrm{R\&D}}{\textrm{Assets}}$&        1.71\sym{***}&        1.12         &        1.15         &        0.28         &        0.82\sym{***}&        0.62\sym{**} &        0.57         &        0.77         \\
                    &      (0.34)         &      (0.68)         &      (0.71)         &      (1.05)         &      (0.17)         &      (0.32)         &      (0.36)         &      (0.88)         \\
\addlinespace
NAICS4-State-Age-Year FE&          No         &          No         &          No         &         Yes         &          No         &          No         &          No         &         Yes         \\
\addlinespace
NAICS4-Year FE      &          No         &          No         &         Yes         &          No         &          No         &          No         &         Yes         &          No         \\
\addlinespace
State-Year FE       &          No         &          No         &         Yes         &          No         &          No         &          No         &         Yes         &          No         \\
\addlinespace
Firm FE             &          No         &         Yes         &         Yes         &         Yes         &          No         &         Yes         &         Yes         &         Yes         \\
\addlinespace
Age FE              &          No         &          No         &         Yes         &          No         &          No         &          No         &         Yes         &          No         \\
\addlinespace
Year FE             &          No         &         Yes         &          No         &          No         &          No         &         Yes         &          No         &          No         \\
\addlinespace
No FE               &         Yes         &          No         &          No         &          No         &         Yes         &          No         &          No         &          No         \\
\midrule
r2\_a                &       0.014         &        0.26         &        0.22         &        0.25         &      0.0088         &        0.27         &        0.21         &        0.40         \\
r2\_a\_within         &       0.014         &      0.0028         &      0.0025         &      0.0014         &      0.0088         &      0.0014         &      0.0012         &      0.0046         \\
N                   &       60687         &       59477         &       58201         &       23665         &       60687         &       59477         &       58201         &       23665         \\
\bottomrule
\multicolumn{9}{l}{\footnotesize Standard errors in parentheses}\\
\multicolumn{9}{l}{\footnotesize \sym{*} \(p<0.1\), \sym{**} \(p<0.05\), \sym{***} \(p<0.01\)}\\
\end{tabular}
}

	\caption{\tiny The dependent variable is the average yearly number of founders from the parent firm joining startups in years $t+1,t+2,t+3$, normalized by a trailing five-year moving average of assets. Independent variables are also normalized by assets. Standard errors are clustered at the firm level.}
	\label{table:RDandSpinoutFormation_at_founder2_l3f3}
\end{table}
\end{frame}

\begin{frame}{Corporate R\&D and spinout formation regression: IV specification}
\begin{table}
	\Tiny
	\centering
	{
\def\sym#1{\ifmmode^{#1}\else\(^{#1}\)\fi}
\begin{tabular}{l*{8}{c}}
\toprule
                    &\multicolumn{1}{c}{(1)}&\multicolumn{1}{c}{(2)}&\multicolumn{1}{c}{(3)}&\multicolumn{1}{c}{(4)}&\multicolumn{1}{c}{(5)}&\multicolumn{1}{c}{(6)}&\multicolumn{1}{c}{(7)}&\multicolumn{1}{c}{(8)}\\
                    &\multicolumn{1}{c}{Founders}&\multicolumn{1}{c}{Founders}&\multicolumn{1}{c}{Founders}&\multicolumn{1}{c}{Founders}&\multicolumn{1}{c}{WSO4}&\multicolumn{1}{c}{WSO4}&\multicolumn{1}{c}{WSO4}&\multicolumn{1}{c}{WSO4}\\
\midrule
log(R\&D)           &        0.19\sym{***}&        0.84\sym{+}  &       -0.94         &        0.38         &        0.18\sym{***}&       -0.92         &        1.08\sym{*}  &        0.14         \\
                    &     (0.022)         &      (0.53)         &      (8.21)         &      (0.58)         &     (0.026)         &      (4.68)         &      (0.61)         &      (0.76)         \\
\addlinespace
highNCC=0 $\times$ log(R\&D)&           0         &           0         &           0         &           0         &           0         &           0         &           0         &           0         \\
                    &         (.)         &         (.)         &         (.)         &         (.)         &         (.)         &         (.)         &         (.)         &         (.)         \\
\addlinespace
highNCC=1 $\times$ log(R\&D)&       0.033         &       -1.32         &        3.85         &       -0.21         &       0.018         &        3.16         &        0.20         &     -0.0060         \\
                    &     (0.053)         &      (1.45)         &      (18.3)         &      (0.67)         &     (0.069)         &      (12.3)         &      (0.73)         &      (0.53)         \\
\addlinespace
NAICS4-Year FE      &          No         &          No         &          No         &         Yes         &          No         &          No         &          No         &         Yes         \\
\addlinespace
Firm FE             &         Yes         &         Yes         &         Yes         &         Yes         &         Yes         &         Yes         &         Yes         &         Yes         \\
\addlinespace
Firm Age FE         &          No         &          No         &         Yes         &         Yes         &          No         &          No         &         Yes         &         Yes         \\
\addlinespace
Year FE             &          No         &          No         &         Yes         &          No         &          No         &          No         &         Yes         &          No         \\
\midrule
r2\_a                &        0.68         &       -0.58         &       -1.96         &       -0.30         &        0.60         &       -3.98         &       -0.12         &       -0.27         \\
r2\_a\_within         &       0.099         &                     &                     &                     &        0.12         &                     &                     &                     \\
N                   &        7111         &        4007         &        1206         &         909         &        2950         &        1675         &         467         &         389         \\
\bottomrule
\multicolumn{9}{l}{\footnotesize Standard errors in parentheses}\\
\multicolumn{9}{l}{\footnotesize \sym{++} \(p<0.3\), \sym{+} \(p<0.2\), \sym{*} \(p<0.1\), \sym{**} \(p<0.05\), \sym{***} \(p<0.01\)}\\
\end{tabular}
}

	\caption{\tiny The dependent variable is the log of the average yearly number of founders from the parent firm joining startups, over the years $t+1,t+2,t+3$. Independent variables are similarly averaged over $t,t-1,t-2$. Columns (1) and (5) are estimated by OLS. The remaining columns columns are estimated by instrumenting R\&D spending using firm-specific tax incentives for R\&D, from Bloom 2013. Standard errors are clustered at the firm level.}
	\label{table:RDandSpinoutFormation_iv_founder2_l3f3}
\end{table}
\end{frame}


\begin{frame}{Economic magnitude}
\begin{figure}[!htb]
	\includegraphics[scale=0.3]{../empirics/figures/founder2_founders_f3_Accounting.pdf}
	\caption{\tiny Economic magnitude of regression estimates. The first row of figures compares the predicted number of employee founders (dotted lines) to the observed number of employee founders (solid lines). The left figure considers all founders, the right figure only founders of firms in the same 4-digit NAICS industry as their previous employers. The bottom row shows the percentage explained in each year.}
\end{figure}
\end{frame}












\end{document}