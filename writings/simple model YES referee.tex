\documentclass[11pt,english]{article}
\usepackage{lmodern}
\linespread{1.05}
%\usepackage{mathpazo}
%\usepackage{mathptmx}
%\usepackage{utopia}
\usepackage{microtype}



\usepackage{chngcntr}

\usepackage[nocomma]{optidef}

\usepackage[section]{placeins}
\usepackage[T1]{fontenc}
\usepackage[latin9]{inputenc}
\usepackage[dvipsnames]{xcolor}
\usepackage{geometry}

\usepackage{babel}
\usepackage{amsmath}
\usepackage{graphicx}
\usepackage{amsthm}
\usepackage{amssymb}
\usepackage{bm}
\usepackage{amsfonts}

\usepackage{accents}
\newcommand\munderbar[1]{%
	\underaccent{\bar}{#1}}


\usepackage{svg}
\usepackage{booktabs}
\usepackage{caption}
\usepackage{blindtext}
%\renewcommand{\arraystretch}{1.2}
\usepackage{multirow}
\usepackage{float}
\usepackage{rotating}
\usepackage{mathtools}
\usepackage{chngcntr}

% TikZ stuff

\usepackage{tikz}
\usepackage{mathdots}
\usepackage{yhmath}
\usepackage{cancel}
\usepackage{color}
\usepackage{siunitx}
\usepackage{array}
\usepackage{gensymb}
\usepackage{tabularx}
\usetikzlibrary{fadings}
\usetikzlibrary{patterns}
\usetikzlibrary{shadows.blur}

\usepackage[font=small]{caption}
%\usepackage[printfigures]{figcaps}
%\usepackage[nomarkers]{endfloat}


%\usepackage{caption}
%\captionsetup{justification=raggedright,singlelinecheck=false}

\usepackage{courier}
\usepackage{verbatim}
\usepackage[round]{natbib}
\bibliographystyle{plainnat}

\definecolor{red1}{RGB}{128,0,0}
%\geometry{verbose,tmargin=1.25in,bmargin=1.25in,lmargin=1.25in,rmargin=1.25in}
\geometry{verbose,tmargin=1in,bmargin=1in,lmargin=1in,rmargin=1in}
\usepackage{setspace}

\usepackage[colorlinks=true, linkcolor={red!70!black}, citecolor={blue!50!black}, urlcolor={blue!80!black}]{hyperref}
%\usepackage{esint}
\onehalfspacing

%\theoremstyle{remark}
%\newtheorem{remark}{Remark}
%\newtheorem{theorem}{Theorem}[section]
\newtheorem{proposition}{Proposition}
\newtheorem{proposition_corollary}{Corollary}[proposition]
\newtheorem{lemma}{Lemma}
\newtheorem{lemma_corollary}{Corollary}[lemma]

\begin{document}
	
\title{Creating Creative Destruction: Endogenous Growth with Employee Spinouts and Non-compete Agreements}

\author{Nicolas Fernandez-Arias} 
\date{\today \\ \small
	\href{https://drive.google.com/file/d/1gu4CT1ft4LY4MsKKgluxb8Gu_YoP8DLD/view?usp=sharing}{Click for most recent version}}
\maketitle


%\setcounter{tocdepth}{2}
%\tableofcontents

\begin{abstract}
	I study the effect of non-compete agreements (NCAs) on aggregate productivity growth. I first develop an augmented quality ladders model of endogenous growth with NCAs and within-industry employee spinouts (WSOs). Next, I assemble a new dataset of VC-funded startups matched to the previous employers of their founding members. Using these data I find suggestive evidence of a causal relationship between corporate R\&D and employee spinout formation which quantitatively can account for roughly 50\% of WSOs in the data. I calibrate the model to match the micro estimates, aggregate moments and estimates from the literature. According to the calibrated model, fully enforcing NCAs can increase welfare by approximately 1.5\% in consumption equivalent terms. Blanket R\&D subsidies can reduce growth and welfare by misallocating R\&D to entrants engaging in creative destruction. The optimal policy is a combination of R\&D subsidies targeted at own-product innovation and a ban on the use of NCAs. I close with a discussion of the barriers to practically implementing the optimal policy.
\end{abstract}

\section{Introduction}

Knowledge spillovers and firm entry are both major contributors to aggregate productivity growth. Firm entry in turn is often the result of knowledge spillovers from existing firms. In particular, within-industry employee spinouts (WSOs) -- new firms founded by former employees of incumbent firms in the same industry -- often take advantage of knowledge gained at previous employers. Figure \ref{fairchild_spinouts} shows the many direct and indirect spinouts of Fairchild Semiconductor, one of the first leading semiconductor firms of Silicon Valley -- itself a spinout of Shockley Laboratories, another semiconductor firm. Although Fairchild was founded in the 1950s, its list of spinouts includes some of the most well-known modern firms in the industry, such as Intel and AMD. 

\begin{figure}	\phantomsection
	\center
	\includegraphics[scale = 0.77]{../figures/fairchildren_early.png}
	\caption{Direct and indirect spinouts of Fairchild Semiconductor}
	\label{fairchild_spinouts}
\end{figure}


To avoid the possibility of competition from such firms, incumbents may reduce their investment in R\&D and other forms of costly knowledge creation. Alternatively, they may take steps to prevent WSOs directly, mitigate the disincentive to R\&D restricting productivity-enhancing knowledge spillovers. The most salient example of this kind of effort is the non-compete agreement (NCA), an employment precluding the employees from founding a competing firm after ceasing his or her current employment until a prespecified amount of time has passed. Given the aforementioned tradeoff, it is not clear what the effect of NCAs is on aggregate productivity growth, nor is it clear how the answer to this question depends on structural parameters that may be different in different locations, industries or time periods. Further, from a normative perspective, it is natural to ask whether it is socially optimal to permit the free use of NCAs.

This paper is an attempt to provide quantitative answers to these questions. To do so, I first develop a tractable model of endogenous growth which augments the standard quality ladders framework in \cite{acemoglu_introduction_2009} to include WSOs and NCAs. I then construct and analyze a micro dataset of incumbent firms and startups, providing evidence for a causal relationship between parent firm R\&D and subsequent startup formation and find suggestive evidence of an economically meaningful causal relationship. The model is then calibrated using aggregate statistics and the microeconomic relationship between R\&D and employee entrepreneurship. Using the calibrated model, I study the effect of varying the barriers to enforcement of NCAs and describe the model-implied optimal policy. I find that eliminiation of all barriers to NCAs can increase welfare by approximately 1.5\% in consumption-equivalent terms. R\&D subsidies can have the counterintuitive effect of reducing growth by misallocating R\&D labor to entrants instead of incumbents.\footnote{This stems in part from an assumption of a fixed stock of R\&D labor. In a fuller model, this mechanism would simply dull the growth-enhancing effect of R\&D subsidies. I plant to extend the model in that direction.} R\&D subsidies targeted at own-product innovation can work well in tandem with a ban on NCAs, but are difficult if not impossible to implement in practice.

The model consists of a standard general equilibrium model of endogenous growth with creative destruction, augmented to include employee entrepreneurship and NCAs. The model takes as given that, absent NCAs, R\&D employees eventually gain the knowledge to form a competing WSO. Given the R\&D wage, this reduces the incentive for R\&D. However, in equilibrium, R\&D workers accept a lower wage, as they internalize the expected future profits WSOs they form. In this sense they "pay" ex-ante for the damage they will cause. Whether their payment is sufficient to fully mitigate the disincentive therefore depends on whether the value gained by the founder of the WSO is larger than the value lost by the incumbent. In fact, NCAs are used only if WSO formation does not maximize the ex-ante joint value of the employment relationship. 

In order for the model to generate a role for NCAs, there must be some other contracting friction or in the employer-employee relationship.  Otherwise, the incumbent could offer to buy any WSOs and shut them down. Ex-ante, the employee accepts a lower wage and the firm conducts R\&D as though it had imposed an NCA (and pays a higher wage). private information concerning the quality of the idea, disagreements between the employee and the employer concerning the idea's quality, and the lack of commitment power on the part of the employee (i.e., the employee cannot commit not to implement the idea even after notionally selling it to his employer). I leave the particular friction unmodeled, simply assuming that there is no market in which WSOs can be sold to the incumbent firm that generated them.\footnote{In future work, I plan to explore this area further.}

The result is a model in which WSOs expand the innovation possibilities frontier of the economy while having an ambiguous effect on equilibrium innovation and productivity growth. The freedom to use NCAs, which prevent WSOs, also may increase or decrease the equilibrium growth rate. To go further, some discipline needs to be imposed on the model parameters.

To do so, I assemble a dataset of parent firms and startups founded by their employees by combining Compustat data on publicly traded firms and Venture Source data on VC-funded startups and their founders. While the Venture Source data has limitations, it is the only dataset of startups with information on the most recent employer of the startup's key employees. Matching these data is somewhat challenging since there are no company identifiers across datasets: the match must be done by name only. This is non-trivial since companies go by different names. I solve this problem by using string matching techniques (e.g., regular expressions), Compsutat data on subsidiaries and finally the merchant-mapper tool by Alternative Data Group, a startup that links credit card transactions data to firms using machine learning (itself a spinout of 1010 Data). I define a startup as a spinout if its CEO, CTO, President, Chairman or Founder (1) was most recently employed at a firm in Compustat and (2) joined the startup in its first three years. Using this definition, I identify approximately 3,000 WSOs in the data. Finally, I match this dataset to data on all US patents taken from the NBER-USPTO patent database.

Figures \ref{figure:scatterPlot_RD-Founders_dIntersection} and \ref{figure:scatterPlot_RD-FoundersWSO4_dIntersection} illustrate the primary motivation for this paper: firm-level R\&D is associated with subsequent employee entrepreneurship. To be precise, the x-axis shows firm-level average R\&D spending over periods $t,t-1,t-2$ and the y-axis shows firm-level average yearly number of employee founders from that firm in $t+1,t+2,t+3$. Both of these variables are then purged sequentially of their firm- and state-industry-age-year means. Figure \ref{figure:scatterPlot_RD-FoundersWSO4_dIntersection} confirms this when restricting attention to employee entrepreneurship in the same 4-digit industry of the initial employer. 

\begin{figure}[]
	\centering
	\includegraphics[scale= 0.5]{../empirics/figures/scatterPlot_RD-Founders_dIntersection.pdf}
	\caption{Scatterplot of average yearly founder counts in $t+1,t+2,t+3$ versus average yearly R\&D spending in $t,t-1,t-2$.}
	\label{figure:scatterPlot_RD-Founders_dIntersection}
\end{figure}

\begin{figure}[]
	\centering
	\includegraphics[scale= 0.5]{../empirics/figures/scatterPlot_RD-FoundersWSO4_dIntersection.pdf}
	\caption{Scatterplot of average yearly founder counts (restricted to same 4-digit NAICS industry) in $t+1,t+2,t+3$ versus average yearly R\&D spending in $t,t-1,t-2$.}
	\label{figure:scatterPlot_RD-FoundersWSO4_dIntersection}
\end{figure}

However, a simple regression of number of employee spinouts on lagged R\&D spending suffers from omitted variable bias, as factors such as changes in demand or changes in technological investment opportunities likely affect both variables in the same direction. To control for this, I use firm, state-year, NAICS 4 digit industry-year (at 4-digit NAICS level), and firm age fixed effects, as well as firm-specific controls, such as employment, assets, Tobin's Q, and citation-weighted patents. The resulting estimates vary by specification obut are typically statistically and economically significant. According to these estimates, R\&D can account for roughly 50\% of WSOs in the data. 

I next calibrate the model using the estimates above as well as aggregate statistics and growth accounting estimates from \cite{garcia-macia_how_2019} and \cite{klenow_innovative_2020}. I also choose some parameters from the literature. I use the calibrated model to study the effect on productivity growth and welfare of varying the cost of using NCAs. I discuss how these results depend on the parameters and, via the calibration, on the value of the targeted moments. Finally, I discuss other policies that could improve welfare in this context.

\paragraph{Related literature}

Some work has attempted to answer this question directly using empirical methods. Papers in this literature have typically used either cross-sectional and/or longitudinal variation in the state-level enforcement of non-competes.\footnote{Sometimes this variation is argued to be exogenous, either due to legislative error as in \cite{marx_mobility_2009} and \cite{marx_regional_2015}, or due to unexpected judicial precedent as in \cite{jeffers_impact_2018}. Often there is a control industry that is believed to be unaffected by the variation in CNC enforcement policy (e.g. law firms are typically exempt from CNC restrictions).} The results are inconclusive and suggest an important tradeoff between entry of spinouts and investment by incumbent firms. \cite{stuart_liquidity_2003} find more local  entrepreneurship in response to local IPO (a "liquidity event") in regions not enforcing CNCs. \cite{marx_mobility_2009} finds that inventor mobility declines in response to an increase in non-compete enforcement. \cite{samila_venture_2010} finds that an increase in VC funding supply increases entrepreneurship more in states without non-compete restrictions, using an IV design. \cite{garmaise_ties_2011} finds that, in states where CNCs are more enforceable, managers are less mobile, have lower compensation, and invest less in their human capital, to the point of offsetting increased investments by the firm. On the other hand, \cite{conti_non-competition_2014} finds evidence that non-compete enforceability leads to incumbent firms pursuing riskier R\&D projects. \cite{colombo_does_2013} finds evidence that easier spinout formation -- proxied by access to finance -- leads to a reduction in incumbent firm knowledge investments.  Most recently, \cite{jeffers_impact_2018} uses data on influential state-level court precedents matched with LinkedIn data and finds that enforcement indeed reduces spinout formation while increasing capital investment by incumbent firms. Finally, \cite{marx_regional_2015} finds that CNC enforcement leads to inventor mobility out of the state, suggesting that differences in outcomes could be in part due to reallocation. 

Theoretical work has also explored this question. As mentioned previously, \cite{franco_spin-outs_2006} develops a model in which employees learn from their employers and use this knowledge to form spinouts. They emphasize the "paying for knowledge" effect, whereby employees implicitly pay for the knowledge they take from the parent firm through lower equilibrium wages. Importantly, they assume spinout firms do not steal business from their parents: the only effect of a spinout on the parent firm is a reduction in the price of the output good, which the parent firm is assumed not to take into account. This, combined with the "paying for knowledge" mechanism, ensures that the competitive equilibrium allocation is Pareto efficient, even without resorting to elaborate labor contracts.

\cite{franco_covenants_2008} studies a two-period, two-region model with employee spinouts in which the region which does not enforce CNCs initially lags but eventually overtakes the region in which CNCs are enforced. In the first period, entry is more valuable in the enforcing region. But in the second period, spinouts enter in the non-enforcing region, there is Cournot competition with parent firms in the product market, and output increases relative to the enforcing region. The analysis emphasizes how asymmetric information about whether an employee has learned leads some firms in the non-enforcing region to allow spinouts (assuming firms cannot commit to wage backloading). This can be taken as a rough microfoundation of my assumption that labor contracts are "simple" in  a non-enforcing region: just a wage, with no attempts at retention in the case of learning. Relative to this study, my analysis considers a fully dynamic model rather than two-period model. In addition, I emphasize the role of R\&D investment in spawning spinout firms.

\cite{shi_restrictions_2018} uses a rich model of contracting disciplined by data on executive non-compete contracts to study the effect of non-competes on executive mobility and firm investment. She finds that the optimal policy is to somewhat restrict the permitted duration of CNCs. Her approach allows her to study the optimal contracting problem in more detail than in mine. However, she is mainly interested in poaching, which involves an attempt to extract a payout from the poaching firm. Also, her calibration considers firm investment in capital expenditures, whereas I am interested in innovative investment in R\&D.

\cite{baslandze_spinout_2019}, the study closest to this paper, studies the effect of spinout entrepreneurship on entry and growth. She also uses a GE model of endogenous growth with employee spinouts, using Compustat and NBER-USPTO patent data to discipline the analysis. She finds the optimal policy is to ban NCAs. However she is focused in her framework on the harm from losing a valuable employee rather than the harm from competition with the parent firm. My paper focuses instead on creative destruction of the parent firm using knowledge rather than the loss of productivity from losing valuable employees.  The other key difference is that I model the use of NCAs while her analysis assumes that they are used when available. To my knowledge, mine is the first general equilibrium model of endogenous growth to have this kind of feature.

\cite{babina_entrepreneurial_2019} find evidence of a causal relationship from corporate R\&D spending to employee spinout formation. My empirical analysis confirms their findings on a subset of firms particularly connected with productivity growth, VC-funded startups. Together, they motivate the use of a model like the one developed in this paper.


\section{Model}\label{sec:model}

\subsection{Individual endowments and preferences}

The model is in continuous time, starting at $t = 0$. The representative household has CRRA preferences over consumption, given by\footnote{There is no expectation operator use there is no aggregate uncertainty in this setting (more on this in later sections).}
\begin{align}
U_t &= \int_0^{\infty} e^{-\rho s} \frac{C(t+s)^{1-\theta} - 1}{1-\theta} ds \label{preferences}
\end{align}

In each period $t \ge 0$, the household is endowed with $\bar{L}_{RD} \in (0,1)$ units of R\&D labor as well as $1 - \bar{L}_{RD}$ units of production labor which can be used to make the final good $(L_F)$ or the intermediate goods $(L_I)$. The household therefore chooses $L_{RD},L_F,L_I$ subject to the resource constraints
\begin{align}
L_{RD} &\le \bar{L}_{RD} \label{labor_resource_constraint2} \\
L_F + L_I &\le 1 - \bar{L}_{RD} \label{labor_resource_constraint} 
\end{align}

\subsection{Production of final and intermediate goods} \label{subsec:staticproduction}

The final good $Y$ is produced competitively using labor and a continuum of intermediate goods indexed by $j \in [0,1]$. In turn, at any time $t \ge 0$, the intermediate good $j$ is available in $n_{jt}$ different qualities $\{\lambda^0,\lambda^1,\ldots,\lambda^{n_{jt}}\}$, where $\lambda > 1$ is exogenous and $n_{jt}$ is endogenous and determined by R\&D investment, described in detail in Section \ref{subsec:innovation}. As a matter of notation, let $\bar{q}_{jt} = \lambda^{n_{jt}}$ denote the frontier quality of good $j$ at time $t$, and similarly let $\bar{k}_{jt}$ denote its quantity. Below I suppress the $t$ subscript where it is clear. 

The final goods production technology is given by \footnote{Intermediate goods are aggregated in a CES form with an elasticity of substitution greater than 1, rather than the Cobb-Douglas form in e.g., \cite{grossman_quality_1991} and \cite{baslandze_spinout_2019}. This reduces the complexity of the firm problem. In those models, Cobb-Douglas implies that expenditure on each intermediate good is constant in quality. This requires limit pricing to be explicitly modeled, otherwise increasing the price always increases profits and the firm problem is not well-defined. To model limit pricing, one must track the gap between leader and follow in each good $j$, adding a state variable to the firm problem and to the aggregation of the model. In the current setup, by contrast, expenditure is decreasing in the price of the intermediate good, so even if one abstract from limit pricing (microfoundation below), intermediate goods firms have a constant optimal markup. In the full model, I will take advantage of this reduced complexity by introducing more complexity in the employee spinout and firm entry process.}
\begin{align}
Y = F(L_F,\{n_j\},\{k_{ji}\}) &= \frac{L_F^{\beta}}{1-\beta} \int_0^1 \Big(\sum_{i = 0}^{n_{j}} (\lambda^{i})^{\frac{\beta}{1-\beta}} k_{ji} \Big)^{1-\beta} dj \label{final_goods_production}
\end{align}

where $k_{ji} \ge 0$ for $0 \le i \le n_j$ is the quantity used of intermediate good $j$ of quality $\lambda^i$. There is no storage technology for the final good and its price is normalized to 1 in every period. 

Intermediate goods $j$ of any quality $\lambda^i$ are produced using the technology given by 
\begin{align}
k_{ji} = H(\ell_{ji};Q) &= Q \ell_{ji} \label{intermediate_goods_production}
\end{align}
where $\ell_{ji} \ge 0$ is the labor input and $Q = \int_0^1 \bar{q}_{j} dj$ is the average frontier quality level in the economy.\footnote{The linear scaling with the aggregate economy $Q$ is to ensure a balanced growth path (BGP), given that the total quantity of labor stays the same over time. It is analogous to assuming a constant marginal cost in a model where the final good, rather than labor, is the input of intermediate goods production.} Each quality of good $j$ is produced by a firm which has a monopoly on that quality of good $j$. 

I refer to the producer of $j$ with access to the frontier technology $\bar{q}_{jt}$ as the \textit{incumbent} of good $j$ or as \textit{incumbent $j$}. Note that (\ref{final_goods_production}) implies that different qualities of good $j$ are perfect substitutes in final goods production and (\ref{intermediate_goods_production}) means intermediate goods production functions have constant returns to scale. Later I will specify that producers of a given intermediate good $j$ engage in Bertrand competition with each other. Together this implies that in equilibrium only incumbents produce positive amounts. This leads to the more familiar representation of final goods production used in  \cite{acemoglu_introduction_2009},
\begin{align}
	Y = F(L_F,\{\bar{q}_j\},\{\bar{k}_j\}) &= \frac{L_F^{\beta}}{1-\beta} \int_0^1 \bar{q}_j^{\beta} \bar{k}_j^{1-\beta} dj  \label{eq_final_goods_production}
\end{align}

\subsection{Innovation}\label{subsec:innovation}

There are three types of innovation in this economy. First, intermediate goods producers at the frontier can expend R\&D to improve on their own product (own innovation, or OI). Second, R\&D by incumbents leads to creative destruction (CD) by WSOs founded by R\&D employees. Third, entrants can expend R\&D in order to innovate on the frontier quality of each good $j$, engaging in CD innovation to replace the previous incumbent.

An innovation at time $t$ on the frontier quality of good $j$ yields a perpetual patent on the production of good $j$ of quality $\bar{q}_{jt} = \lambda \lim_{t' \uparrow t} \bar{q}_{jt'}$, where $\lambda > 1$ is the exogenous quality ladder step size. That is, the new frontier quality improves on the previous frontier quality by a factor $\lambda$. Importantly, the perpetual patent \textit{does not} prevent entrants or WSOs from displacing the incumbent with an even better quality product. In addition to the monopoly on production of the frontier quality, a successful innovator gains access to the R\&D technology for OI (described in Section \ref{subsubsec:OI}).

\paragraph{No catch-up innovation}

I assume that only incumbents can perform OI.\footnote{This is standard in quality ladders models with OI and is usually not made explicit. Without it, the incumbent problem would have an additional state variable (since falling away from the frontier is no longer an absorbing state) and an additional distribution would need to be tracked (the number of incumbents with the technology to produce each infra-frontier good $j$. The setting is so intractable that even papers focused on this mechanism, such as \cite{aghion_competition_2005}, make extreme simplifying assumptions analogous to mine: they assume that a producer cannot fall more than $1$ step behind so that good $j$ has a finite rather than a countable set of states.} When an incumbent is overtaken by an entrant or spinout, she loses access to the OI R\&D technology and therefore cannot use it to "catch up" to the frontier. One possible interpretation is that learning by doing means the current producer of a product has unique insights into how to improve on it. From a mathematical standpoint, the assumption dramatically simplifies the analysis.\footnote{For a producer $n$ steps behind the frontier, the assumption has bite if the expected discounted present value of the cost of $n + 1$ innovations using the OI innovation technology is lower than the expected cost of one innovation using the freely available entrant technology (described in Section \ref{subsubsec:entrants}). For certain parameter values, this inequality will hold for small $n$.}  

In the rest of Section \ref{sec:model}, I omit the dependence on $t$ of equilibrium variables. 

\subsubsection{Own-product innovation by incumbents} \label{subsubsec:OI}

Incumbent $j$ can perform $z_j$ units of R\&D effort by hiring $\frac{\bar{q}_{j}}{Q_t}z_j$ units of R\&D labor. In return, she receives a Poisson intensity of $\chi z_j$ of innovating on good $j$, where $\chi > 0$ is an exogenous parameter representing the incumbent's R\&D productivity. 

The Poisson arrival rate of OI by the incumbent is denoted
\begin{align}
	\tau_j &= \chi z_j
\end{align}

Note that incumbent R\&D does not exhibit any decreasing returns to scale. This is necessary to have analytical solutions, but the model can be easily extended and solved numerically.

\subsubsection{Generation of spinouts}\label{subsubsec:generation_of_spinouts}

When incumbent $j$ conducts $z_j$ units of R\&D effort, she faces a Poisson intensity of spawning a WSO, given by 
\begin{align}
	\tau^S_j &= (1-x_{j}) \nu z_j \label{def:tau_S}
\end{align} 
where $x_{j} = 1$ if and only if an NCA is used (determined instant by instant) and where $\nu \ge 0$ is an exogenous parameter representing the rate at which incumbent R\&D produces employee spinouts. Spinouts spawned by (frontier) incumbents of quality $q_j$ have the ability to produce good $j$ of quality $\lambda q_j$. Such a spinout immediately becomes the new frontier incumbent and, recalling the "no catch-up" assumption in Section \ref{subsec:innovation}, the previous incumbent's profits go to zero forever after. The parameter $\nu$ is a reduced form encoding the rate at which corporate R\&D increases the likelihood of replacement by a WSO.

In the context of this model, this specification amounts to assuming that the rate of spinout generation of a unit of R\&D labor is inversely proportional to the relative quality of the good to which it is applied, $\frac{\bar{q}_{jt}}{Q_t}$. Because R\&D labor demand is $\frac{\bar{q}_{jt}}{Q_t} z_j$, the factors cancel out and the rate of spinout formation is linear in $z_j$. In this way, this is the specification of spinout generation that is analogous to the specification of the R\&D technology for OI. Intuitively, it means that, in both OI and spinout generation, higher quality goods require proportionally more human capital to improve.

\paragraph{No idea stealing} WSO entry does not directly reduce the rate at which incumbent R\&D results in successful OI. Instead, WSOs happen when additional, independent Poisson process with rate $\nu z_j$ has an arrival. The interpretation is that WSOs in this model do not steal ideas that otherwise would have been implemented by the parent firm. Rather, when unbound by NCAs, R\&D labor generates \textit{additional} innovations which the household uses to displace the incumbent firm.\footnote{This assumption has important consequences for the private and social usefulness of NCAs. If spinouts steal ideas, they are less bilaterally valuable and NCAs are more useful privately. For the same reason, WSOs are less socially valuable and therefore so are NCAs. This is an interesting topic for further research.}

\paragraph{Direct cost of using NCAs}\label{paragraph:nca_cost}

When incumbent $j$ imposes an NCA on $z_j$ units of R\&D, she must pay a flow cost $\kappa_{c} \nu V(j,t|\bar{q}_j) z_j$ units of the final good, where $V(j,t|q)$ is the value of incumbent $j$ at time $t$ given quality $q$. Given (\ref{def:tau_S}), incumbent $j$ overall pays
\begin{align}
	\textrm{NCA cost}_{j,t,q} &= \tau^S_j \kappa_c V(j,t|q) \label{def:nca_cost}
\end{align}

The NCA enforcement cost reflects the direct cost using an NCA. Even if there are no technical restrictions on what kinds of NCAs are valid, determining competition between businesses may be expensive. Moreover, many jurisdictions do, in fact, impose such restrictions, and resources can be invested to prove that the conditions of those restrictions do not apply. Overall, it seems plausible that investing resources increases the likelihood of a successful enforcement of an NCA.  Note also that a value of $\kappa_c = \infty$ can be interpreted as ban on the use of NCAs.

The factor $V(j,t|q)$ means that the cost of enforcing NCAs increases in the equilibrium value of the incumbent firm. The economic justification is that valuable incumbency positions require more resources to protect via NCAs. In the context of the model, this specification means that the cost of enforcing an NCA on a given unit of R\&D labor is proportional to both the value of the WSOs that labor will generate in the absence of an NCA, and the expected loss of incumbent value from an absence of NCAs. In addition to economics, the assumption improves model tractability by simplifying the analysis of the optimal noncompete policy (see Section \ref{subsubsec:dynamic_equilibrium_original_solution}).\footnote{BGP only requires that\begin{align*}
	\textrm{NCA cost}_{j,t,q} &\propto \tau^S_j q
	\end{align*}}

\paragraph{Value of future spinout formation}

When R\&D induces the formation of a WSO, it is immediately sold by the representative household to a competitive financial intermediary which owns all firms in the economy, which is in turn owned by the household. When valuing the WSO, the intermediary does not take into account that its entry reduces the value of the incumbent, which it also owns. The intermediary remits all corporate profits to the representative household; the household also takes these profits as given.\footnote{This setup can be microfounded in several ways. One is to assume, instead of a representative household, a continuum of households each consisting of a continuum of agents who fully insure one other against idiosyncratic risk. What is essential is that (1) households behave as though risk-neutral, and that (2) households do not internalize the business-stealing effects of the WSOs they form.} 



\subsubsection{Entrants} \label{subsubsec:entrants}

For each good $j$ there is a unit mass (normalization) of entrants indexed by $e \in [0,1]$ who each perform $\hat{z}_{je}$ units of R\&D on the frontier quality of good $j$.\footnote{That is, by assumption entrants only innovate on the frontier good. This assumption could be relaxed without affecting the model, as entrants would never find it optimal to innovate on goods below the frontier.} This requires $\frac{q_j}{Q_t}\hat{z}_{je}$ units of R\&D labor per entrant. In return, an entrant receives a Poisson intensity of $\hat{z}_{je} \hat{\chi} \hat{z}_j^{-\psi}$ of innovating on good $j$, where $\hat{z}_j = \int_0^1 \hat{z}_{je} de$ denotes aggregate entrant effort on improving good $j$. 

Let $\tau_j^E$ denote the arrival rate of entrant innovations in good $j$, given by
\begin{align}\label{model:entrantsInnovationTechnology}
	\hat{\tau}_j &= \hat{\chi} \hat{z}_j^{1-\psi} = \hat{\chi} \int_0^1 \hat{z}_{je} \hat{\chi} \hat{z}_j^{-\psi} de 
\end{align}

The choice $\psi > 0$ introduces decreasing returns and represents a reduced form \textit{congestion} externality in the entrant innovation technology. This is similar to the congestion externality present in search and matching models. Intuitively, due to a lack coordination, entrants attempt similar approaches to solve the same problem. This duplication of effort reduces the overall returns to entrant R\&D when considered at the level of good $j$.

Upon successfully innovating and achieving the ability to produce quality $\bar{q}_{jt}$, an entrant sells his firm to the competitive financial intermediary. As with WSOs, the intermediary does not take into account the effect of entry of these new firms on the value of incumbents in the overall portfolio.

\subsubsection{Entry cost}

In addition to the R\&D costs of innovation, entrants and spinouts must pay an entry cost of $\kappa_{e} V(j,t|\lambda q)$ units of the final good in order to enter once they have successfully innovated, for exogenous $\kappa_e \in [0,1)$. This reduced form represents non-R\&D expenditures required by CD innovation but not by OI innovation. Examples of such expenditures could be firm set-up or marketing costs for a new product or brand.

The scaling with $V(j,t|\lambda q)$ parallels the scaling of the NCA cost in Section \ref{paragraph:nca_cost}. This is economically reasonable given the interpretation above. It is also important for matching the data, in particular the R\&D / GDP ratio and the entry rate.\footnote{This relates to \cite{comin_rd_2004}, which finds in a similar model that the model-implied expected payoff to aggregate R\&D is much higher than aggregate R\&D spending in the data. This is inconsistent with free entry. His interpretation is that R\&D is in fact responsible for only a small part of innovation, which reduces the model-implied payoff of aggregate R\&D spending. Instead, my interpretation, via this model, is that there are additional non-R\&D expenditures associated with innovation. The calibration sets these expenditures so that the return to innovative investment is equal to the market rate.} In addition, scaling with $V(j,t|\lambda)$ enables a simple analytical solution to the model.\footnote{As in Section \ref{paragraph:nca_cost}, BGP only requires that 
\begin{align*}
	\textrm{Entry cost}_{j,t,q} &\propto q
\end{align*}}

\subsection{Equilibrium}\label{subsec:decentralized_equilibrium}

I will solve for a BGP equlibrium with a constant growth rate of output ($g_t = g$) as well as constant innovation effort by incumbents ($z_{jt} = z$) and entrants ($\hat{z}_{jet} = \hat{z})$. I will refer to this class of BGP as a \textit{symmetric BGP}.\footnote{In principle I allow $x_{jt}$ to vary across $j$ and over time $t$; Propositions \ref{proposition:purstrategyeq:positiveOI} and \ref{proposition:purstrategyeq:zeroOI} below show that $x_{jt} = x$ in a symmetric BGP except on a knife-edge in the parameter space.}\footnote{One could relax the assumption that $z_{jt} = z$ as long as $\int_0^1 z_{jt} \frac{\bar{q}_{jt}}{Q_t}dj$ is constant on the BGP. This induces a continuum of BGPs which have the same aggregate variables (except higher moments of the quality distribution, which are irrelevant for the equilibrium), since this term appears in the growth accounting equation analogous to (\ref{eq:growth_accounting}) and the R\&D labor market clearing condition (\ref{eq:RD_labor_market_clearing}). As such, this kind of multiplicity does not affect aggregate growth or prices and is a technical artefact of the assumed CRS R\&D technology for incumbents. I therefore assume that $z_{jt} = z$ because it simplifies the algebra.} Symmetric BGPs are a natural type of equilibrium given the symmetric setup of the model. In this case, I am able to prove existence and uniqueness within the class of symmetric BGP, except on a knife-edge when there is a continuum of BGPs. I leave to future work the question of multiplicity of asymmetric BGPs in this setting.\footnote{There can in principle be equilibria where goods $j$ cycle between low $\hat{\tau}_{j}$ and high $\hat{\tau}_{j}$ states, either during the life of a given incumbent or across incumbents. This could be self-fulfilling because an expectation of low $\hat{\tau}_j$ in the future justifies high $\hat{\tau}_j$ today, and vice versa. That is, while at a given $t$ the good $j$ game played by entrants exhibits strategic substitutability (due to the congestion externality), once dynamics are considered the payoff from victory can vary over time, creating a new potential source of multiplicity. I have never seen a proof of nonexistence for this type of equilibrium, nor have I been able to write one; at the same time, I have not been able to construct such an equilibrium, nor have I seen an example of it being constructed in the literature.} 

To do so, first I characterize the static equilibrium given a profile of frontier qualities $\{ \bar{q}_{j}\}$. Next, using the assumption that $\hat{z}_{jet} = \hat{z}, z_{jt} = z$, I prove that there exists $\tilde{V} > 0$ such that $V(j,t|\bar{q}_{jt}) = \tilde{V} \bar{q}_{jt}$ is the value of incumbent $j$ of quality $\bar{q}_{jt}$ at time $t$. This in turn implies that R\&D wages scale linearly with aggregate productivity $Q_t$ as well. Given the linear scaling with $\bar{q}_{jt}$ and $Q_t$, respectively, all references to $q$ and $Q_t$ drop out of the equilibrium equations. The result is a simple system of equations that can be solved recursively for all equilibrium variables. 

\subsubsection{Static equilibrium}

In this section, I omit the dependence on $t$ of all equilibrium variables. In addition, since only the frontier quality is produced in equilibrium, I will drop the bar notation and refer to the frontier good's quality and quantity by $q_j$ and $k_j$, respectively.

Final goods producer optimization implies the following inverse demand function for intermediate goods, 
\begin{align*}
p_j &= L_F^{\beta} q_j^{\beta} k_j^{-\beta}	
\end{align*}

To continue computing the equilibrium of the model, the market structure for intermediate goods must be specified. 

\paragraph{Intermediate goods market structure} The following setup is drawn from \cite{akcigit_growth_2018}. Within each good $j$, intermediate goods producers play a two-stage Bertrand competition game. In the first stage, participants bear a cost of $\varepsilon > 0$ units of the final good in exchange for a right to compete in the second stage. Then, in the second stage, the engage in Bertrand competition. Optimal pricing under Bertrand competition in the second stage implies that all producers not on the frontier will earn zero profits. By backward induction, they do not pay the entry cost in equilibrium, and the leader has a second-stage monopoly over good $j$.\footnote{Without this assumption, there is limit pricing, and the markup charged by the technology leader in good $j$ would depend on his gap relative to the next laggard, e.g. \cite{baslandze_spinout_2019} or \cite{aghion_competition_2005}, only equating to the monopolistic competition markup for large enough gaps.} Different good $j$ incumbents compete against each other under monopolistic competition.

This market structure implies that the incumbent for each good $j$ can effectively ignore lower quality producers of good $j$. She maximizes profits according to
\begin{align}
\pi(q_j) = \max_{k_j \ge 0} \Big\{ L_F^{\beta} q_j^{\beta} k_j^{1-\beta} - \frac{\overline{w}}{Q} k_j \Big\} \label{incumbent_profit}
\end{align}

where $\overline{w}$ is the equilibrium final goods / intermediate goods wage.
This yields optimal pricing, labor demand and production of intermediate goods,
\begin{align}
k_j &= \Big[ \frac{(1-\beta) Q}{\overline{w}} \Big]^{1/\beta}L_F q_j  \label{optimal_k}\\
\ell_j &= k_j / Q \label{optimal_l}\\
p_j &= \frac{\overline{w}}{(1-\beta) Q} \label{optimal_p}
\end{align}

Substituting (\ref{optimal_k}) into the first-order condition for final goods firm optimal labor demand yields a closed form expression for the equilibrium wage $\overline{w}$:
\begin{align}
\overline{w} &= \tilde{\beta} Q \label{wbar} \\
\tilde{\beta} &= \beta^{\beta} (1-\beta)^{1-2\beta} \label{def_cbeta}
\end{align}

Substituting (\ref{optimal_k}) and (\ref{wbar}) into the expression for profit in (\ref{incumbent_profit}) yields
\begin{align}
\pi_j &= (1-\beta) \tilde{\beta} L_F q_j \label{profits_eq}
\end{align}

Substituting (\ref{optimal_k}) into (\ref{optimal_l}) and integrating $L_I = \int_0^1 l_j dj$ yields aggregate labor allocated to intermediate goods production,
\begin{align}
L_I &= \Big( \frac{1-\beta}{\tilde{\beta}} \Big)^{1 / \beta} L_F \label{intermediate_goods_labor}
\end{align}

and substituting (\ref{intermediate_goods_labor}) into the labor resource constraint (\ref{labor_resource_constraint}) yields
\begin{align}
L_F &= \frac{1 - \bar{L}_{RD}}{1 + \Big(\frac{1-\beta}{\tilde{\beta}}\Big)^{1/\beta}}
\end{align}

Output can be computed by substituting (\ref{optimal_k}) into (\ref{final_goods_production}), 
\begin{align}
Y = \frac{(1-\beta)^{1-2\beta}}{\beta^{1-\beta}} Q L_F \label{flow_output}
\end{align}

\subsubsection{Dynamic equilibrium}\label{subsubsec:dynamic_equilibrium_original_solution}

\paragraph{Household optimization}


Household optimization will yield two types equilibrium conditions: an indifference condition on R\&D wages resulting from optimal labor supply, and an Euler equation coming from optimal consumption smoothing.

The household household problem is\footnote{There are also non-negativity constraints on all control variables $\{C(t), \ell_{RD,j} (t), \hat{\ell}_{RD,j}(t), L(t) \}_{t\ge 0}$ which do not bind at the optmium and so are ommitted for clarity.}

\small
\begin{maxi*}[1]<b>
	{\substack{\{C(t) \}_{t \ge 0} \\ \{ L(t)  \}_{t \ge 0} \\ \{\ell_{RD,j}(t)\}_{j \in [0,1], t \ge 0} \\ \{\hat{\ell}_{RD,j}(t)\}_{j \in [0,1], t \ge 0}  }} {\mathbb{E} \int_0^{\infty} \frac{C(t)^{1-\theta}-1}{1-\theta} dt}{}{}
	\addConstraint{ C(t)}{ \le \Pi_t + \bar{w}_tL(t)} 
	\addConstraint{ }{+ \int_0^1 \big( w_{RD,jt} + (1-x_{jt})(\frac{\bar{q}_{jt}}{Q_t})^{-1} \nu (1-\kappa_e) V(j,t|\lambda \bar{q}_{jt}) \big) \ell_{RD,j}(t) dj}
	\addConstraint{ }{+ \int_0^1 \hat{w}_{RD,t} \hat{\ell}_{RD,j}(t) dj}
	\addConstraint{\int_0^1 (\ell_{RD,j}(t) + \hat{\ell}_{RD,j}(t))dj}{ \le \bar{L}_{RD}} 
	\addConstraint{L(t)}{\le 1 - \bar{L}_{RD}}
\end{maxi*}

\normalsize


where $L(t) = L_I(t) + L_F(t)$ denotes production labor, $\ell_{RD,j}(t)$ denotes R\&D labor supplied to incumbent $j$, and $\hat{\ell}_{RD,j}(t)$ denotes R\&D labor supplied to entrants attempting innovation on good $j$.\footnote{Because the household's effective "taste" for working at a given incumbent $j$ depends on the expected DPV of WSOs formed after working there, it is necessary to explicitly model the household's R\&D labor allocation across goods $j$. This is not necessary for R\&D labor supplied to entrants, since they are all identical. However, I present the problem symmetrically.} The household consumes out of profits remitted by the intermediary $\Pi_t$ (which it takes as given), wages earned from production labor supply $\bar{w}_t L(t)$, wages earned from R\&D labor supply $\int_0^1 \big(w_{RD,jt} \ell_{RD,j}(t) + \hat{w}_{RD.t} \hat{\ell}_{RD,j}(t) \big) dj$, and earnings from sales of WSOs to the financial intermediary $ \int_0^1 (1-x_{jt}) (\frac{\bar{q}_{jt}}{Q_t})^{-1} \nu (1-\kappa_e) V(j,t|\lambda \bar{q}_{jt}) \big)\ell_{RD,j}(t) dj$. The last term aggregates using the fact that the household sells a WSO in good $j$ with intensity $(1-x_{jt}) (\frac{\bar{q}_{jt}}{Q_t})^{-1} \nu \ell_{RD,j}(t)$ and has value $(1-\kappa_e) V(j,t|\lambda \bar{q}_{jt})$, given that the entry cost $\kappa_e V(j,t|\lambda \bar{q}_{jt})$ must be paid to enter.\footnote{Technically this assumes that $\kappa_e < 1$. When I discuss entry taxes in Section \ref{subsec:cd_tax} and $\kappa_e + T_e \ge 1$, the value will be $\max\{0,(1-\kappa_e - T_e) \lambda \tilde{V} \}$.} 

Note that, because the household problem involves allocating R\&D labor to different goods $j$ depending on the random processes $\{\bar{q}_{jt}\}_{t \ge 0}, \{w_{RD,jt}\}_{t\ge 0}, \{x_{jt}\}_{t\ge 0}, \{V(j,t|\bar{q}_{jt})\}_{t\ge 0}$ for $j \in [0,1]$, the problem is formulated as a stochastic optimal control problem. However, in a symmetric BGP, there will be no uncertainty in the household's consumption stream, because equilibrium compensation (including the value of WSOs generated) will be the same in all goods $j$ to satisfy market clearing. The following lemma makes this precise.

\begin{lemma}\label{lemma:RD_worker_indifference}
	In a symmetric BGP with $z > 0$, R\&D wages satisfy
	\begin{align}
	\hat{w}_{RD,t} &= w_{RD,jt} + (1-x_{jt}) (\frac{\bar{q}_{jt}}{Q_t})^{-1} \nu (1-\kappa_e) V(j,t|\bar{q}_{jt}) \label{eq:RD_worker_indifference}
	\end{align}
	for all $t \ge 0$ and $j \in [0,1]$.
\end{lemma}

\begin{proof}
	First note that $\hat{z} > 0$ in any equilibrium due to the Inada conditions on the entrant innovation technology given in  (\ref{model:entrantsInnovationTechnology}). If in addition $z > 0$, then $\ell_{RD,j}(t) > 0, \hat{\ell}_{RD,j}(t) > 0$ for all $j,t$. Optimality dictates that the household supplies R\&D labor only to jobs which provide the highest compensation. Therefore, in order to be consistent with household optimal labor supply, (\ref{eq:RD_worker_indifference}) must hold for all $t \ge 0$ and $j \in [0,1]$, 
\end{proof}

To derive the Euler equation, technically one needs to add a market for a instantaneous risk-free bond, which in equilibrium is in zero net supply. I have not done this explicitly to avoid cumbersome notation. Denote the interest rate for this bond by $r_t$, which in principle can be time-varying. The resulting household problem is standard and gives rise to the Euler equation at each time $t \ge 0$, 
\begin{align}
\frac{\dot{C}(t)}{C(t)} = \frac{1}{\theta} (r_t - \rho) \label{eq:euler0} 
\end{align}

Because the household takes as given its ownership of the financial intermediary, the only asset holdings it actively controls are its holdings of the risk-free bond, which is in zero net supply in equilibrium. That is, the only wealth which is controlled by the household is fixed at zero. This means the household's transversality condition holds trivially. However, there is a transversality condition stemming from the financial intermediary, which optimizes by choosing its holdings of firms to maximize cash flows given the interest rate $r_t$. A necessary condition for this to be optimal is the transversality condition
\begin{align}
	\lim_{t' \to \infty} \mathbb{E} \Big[ e^{-\int_t^{t'} r_s ds} \int_0^1 V(j,t'|\bar{q}_{jt'}) dj \Big] &= 0 \label{eq:tvc0}
\end{align}

because the time $t$ discounted present value of profits $\Pi_t$ is equal to $\int_0^1 V(j,t | \bar{q}_{jt})$ in equilibrium.

\paragraph{Equilibrium innovation}

Due to perfect competition, the incumbent faces an infinitely elastic supply curve for R\&D labor: given a choice of $x_{jt}$, she can hire as much as she likes provided she offers total compensation required by the indifference condition (\ref{eq:RD_worker_indifference}). The incumbent takes this into account when deciding whether to offer contracts with an NCA ($x = 1$) or without an NCA ($x = 0$). Let $w_{RD,jt}(x)$ denote the  wage given the incumbent's choice of $x$, given by (\ref{eq:RD_worker_indifference}).

The incumbent HJB is given by 
\begin{align}
(r_t + \hat{\tau}(j,t|q)) &V(j,t |q) - \dot{V}(j,t|q) = \pi(j,t|q) \nonumber \\_{}
&+ \max_{\substack{x \in \{0,1\} \\ z \ge 0}} \Bigg\{ z \Big[ \chi \big( V(j,t|\lambda q) - V(j,t|q)\big)  \nonumber \\
&- \big(\frac{q}{Q_t}\big) \Big( w_{RD,jt}(x) + \big(\frac{q}{Q_t}\big)^{-1} (1-x) \nu V(j,t|q) + \big(\frac{q}{Q_t}\big)^{-1}  x \kappa_c \nu V(j,t|q) \Big)  \Big] \Bigg\} \label{eq:hjb_incumbent_0}
\end{align}
where the discounting is at the risk-free rate $r_t$ because the financial intermediary diversifies across incumbents and there is no aggregate risk.

In order for a solution to the incumbent's HJB to reflect the value of the sequence problem, it must also satisfy a transversality condition.\footnote{Often when this type of model is presented, this transversality condition is not discussed because other considerations imply that there is a unique solution to the incumbent's HJB (\ref{eq:hjb_incumbent_0}) that is compatible with equilibrium. For example, if entrants have CRS R\&D technology, as in the seminal article \cite{grossman_quality_1991}, then $\hat{z} > 0$ implies $V(j,t|\bar{q}_{jt}) = \tilde{V} \bar{q}_{jt}$. Alternatively, if incumbents have CRS R\&D technology as in \cite{acemoglu_introduction_2009}, the additional assumption that $z > 0$ implies, via the incumbent's FOC, that $V(j,t|\bar{q}_{jt}) = \tilde{V} \bar{q}_{jt}$. Finally, \cite{acemoglu_innovation_2015} allows both entrants and incumbents to have DRS R\&D technology, resolving the complexity by restricting attention to BGPs where $V(j,t|\bar{q}_{jt}) = \tilde{V} \bar{q}_{jt}$, proving existence and uniqueness of BGPs of that restricted class. Importantly, while they state in a footnote that they can prove that all BGPs are linear in the case that incumbents have CRS R\&D technology (which would make the model like \cite{acemoglu_introduction_2009}), they do not provide the proof; an exercise in the latter claims to show the result, but the solutions manual also assumes $z_{jt} > 0$ for all $j \in [0,1]$ and $t \ge 0$, referring to an early draft of \cite{acemoglu_introduction_2009} for the proof of that assumption. This assumption ensures that the incumbent FOC holds with equality, immediately yielding the crucial linearity of the value function which implies the rest of the results. In all of these cases, equilibrium equations besides the incumbent TVC rule out explosive paths for the incumbent value function, so the incumbent TVC is not necessary. This would also be true in my model if I assumed $z > 0$ as in \cite{acemoglu_introduction_2009}. Thus the proposition really shows that there is still uniqueness when $z = 0$ provided that $\hat{z}_{jt} = \hat{z}$.} For all $j \in [0,1]$ and $t \ge 0$,
\begin{align}
\lim_{t' \to \infty} \mathbb{E} \Big[ e^{-\int_t^{t'} r_s ds} \mathbf{1}_{s(j,t) > t'} V(j,t'|\bar{q}_{jt'}) \Big] &= 0 \label{eq:tvc_incumbent}
\end{align} 

where the random process $s(j,t)$ denotes the supremum over future times $s \ge t$ during which the time $t$ incumbent is active; that is, $s(j,t) = \sup\{ s : s \ge t \text{ and the time-} t \text{ incumbent of line } j \text{ is still active}\}$. In this case the expression technically allows for aggregate uncertainty (i.e., uncertainty in the path of $r_t$); however, on a BGP it turns out that $r_s$ has a deterministic path (see below). 

\begin{proposition}\label{proposition:hjb_scaling}
	In a symmetric BGP, the value function of the incumbent is given by
	\begin{align*}
		V(j,t|q) &= \tilde{V} q
	\end{align*}
	for some $\tilde{V} > 0$. Further, the equilibrium R\&D wages are given by 
	\begin{align*}
		\hat{w}_{RD,t} &= \hat{w}_{RD} Q_t \\
		w_{RD,jt}(x) &= w_{RD}(x) Q_t \textrm{, if $z > 0$}
	\end{align*}
\end{proposition}

The result follows from two facts. First, in a BGP the interest rate is constant, by the Euler equation (\ref{eq:euler0}). Second, by definition $\hat{z}_{jt} = \hat{z}$ in a symmetric BGP. Together these imply that solutions to the incumbent HJB either satisfy $V(j,t|q) = \tilde{V} q$ or fail to satisfy necessary conditions for optimality. The technical details of the proof are contained in Appendix \ref{appendix:proofs:proposition:hjb_scaling}. 

The above implies the following corollary.

\begin{proposition_corollary}
	In a symmetric BGP, the equilibrium R\&D wages are given by 
	\begin{align*}
	\hat{w}_{RD,t} &= \hat{w}_{RD} Q_t \\
	w_{RD,jt}(x) &= w_{RD}(x) Q_t \textrm{, if $z > 0$}
	\end{align*}
\end{proposition_corollary}

\begin{proof}
	The entrant's free entry condition is
	\begin{align}
	\hat{\chi} \hat{z}^{-\psi} V(j,t|\lambda \bar{q}_{jt}) &= \frac{\bar{q}_{jt}}{Q_t} \hat{w}_{RD,t}
	\end{align}
	
	where $V(j,t|\lambda \bar{q}_{jt})$ is the value that the entrant will have as the new incumbent if he successfully innovates in the next instant. By the previous formula, $V(j,t | \lambda \bar{q}_{jt}) = \tilde{V} \lambda \bar{q}_{jt}$. Substituting yields
	\begin{align}
	\hat{\chi} \hat{z}^{-\psi} \tilde{V} \lambda &= \frac{\hat{w}_{RD,t}}{Q_t}
	\end{align}
	
	implying that $\frac{\hat{w}_{RD,t}}{Q_t}$ must be constant, i.e. $\hat{w}_{RD,t} = \hat{w}_{RD} Q_t$ for some $\hat{w}_{RD}$. Using this and $V(j,t | \bar{q}_{jt}) = \tilde{V}\bar{q}_{jt}$ in Lemma \ref{lemma:RD_worker_indifference} yields $w_{RD,jt}(x) = w_{RD}(x) Q_t$. 
\end{proof}

The next proposition characterizes the optimal NCA policy of the incumbent in a symmetric BGP with $z > 0$. 

\begin{proposition}\label{proposition:optimalNCApolicy}
	In a symmetric BGP with $z > 0$, the optimal NCA policy of the incumbent is given by
	\begin{align}
	x = \begin{cases}
	1 & \textrm{if } \kappa_{c} < \bar{\kappa}_c (\kappa_e, \lambda) \\
	0 & \textrm{if } \kappa_{c} > \bar{\kappa}_c (\kappa_e, \lambda)\\
	\{0,1\} & \textrm{if } \kappa_c = \bar{\kappa}_c (\kappa_e, \lambda) 
	\end{cases} \label{eq_nca_policy}
	\end{align}
	where $\bar{\kappa}_c (\kappa_e, \lambda) = 1 - (1-\kappa_e)\lambda$.

\end{proposition}

Note that on the knife-edge $\kappa_c = \bar{\kappa}_c$, the incumbent is indifferent between $x = 0$ and $x = 1$. The proof entails some algebraic manipulations using the previous two lemmas. 

\begin{proof}
	Using the representation $V(j,t|q) = \tilde{V}q$ derived in Lemma \ref{proposition:hjb_scaling} in the incumbent HJB (\ref{eq:hjb_incumbent_0}) and dividing both sides by $q$ yields
	\begin{align}
	(r + \hat{\tau}) &\tilde{V} = \tilde{\pi} + \max_{\substack{x \in \{0,1\} \\ z \ge 0}} \Bigg\{ z \Big( \chi (\lambda -1) \tilde{V}- w_{RD}(x) - (1-x) \nu \tilde{V} - x \kappa_c \nu \tilde{V} \Big)\Bigg\} \label{eq:hjb_incumbent_1}
	\end{align}
	
	Substituting in $w_{RD}(x)$ using the indifference condition (\ref{eq:RD_worker_indifference}) derived in Lemma \ref{lemma:RD_worker_indifference} yields
	\begin{align}
	(r + \hat{\tau}) \tilde{V} &= \tilde{\pi} + \max_{\substack{x \in \{0,1\} \\ z \ge 0}} \Big\{z \Big( \overbrace{\chi (\lambda - 1) \tilde{V}}^{\mathclap{\mathbb{E}[\textrm{Benefit from R\&D}]}}- \hat{w}_{RD} -  \underbrace{(1-x)(1 - (1-\kappa_{e})\lambda)\nu \tilde{V}}_{\mathclap{\text{Net cost from spinout formation}}} - \overbrace{x \kappa_{c} \nu \tilde{V}}^{\mathclap{\text{Direct cost of NCA}}}\Big) \Big\} \label{eq:hjb_incumbent_workerIndiff}
	\end{align}

	
	Let $\bar{\kappa}_c (\kappa_e, \lambda) = 1 - (1-\kappa_e)\lambda$. If $z > 0$, the incumbent maximizes her flow payoff by choosing $x \in \{0,1\}$ which maximizes the term multiplying $z$. Therefore, $x = 1$ is strictly preferred iff $1 - (1-\kappa_e) \lambda > \kappa_c$, which is equation (\ref{eq_nca_policy}).
\end{proof}

Equation (\ref{eq:hjb_incumbent_workerIndiff}) has an intuitive economic interpretation. The left-hand side is the equilibrium flow payoff on an asset with value $\tilde{V}$. The RHS is the flow payoff of incumbency. The term $\chi(\lambda -1) \tilde{V}$ is the expected benefit per unit of R\&D effort. Notice the factor $\lambda -1$, which takes into account the opportunity cost of no longer producing with the obsolete technology. The term $-\hat{w}_{RD}$ reflects the cost of R\&D effort due to the contribution from the prevailing R\&D wage. The term $-(1-x)(1 - (1-\kappa_e) \lambda) \nu \tilde{V}$ represents the expected net harm to the incumbent due to spinouts from the employee. Expanding this term, the term $-(1-x)\nu \tilde{V}$ reflects the direct harm from creative destruction by spinouts. The second term $(1-x)(1-\kappa_e)\lambda \nu \tilde{V}$ reflects the reduction in R\&D wage accepted by the R\&D employee in return for being free to open spinouts. Finally, the term $-x \kappa_c \nu \tilde{V}$ reflects the direct cost of enforcing NCAs.

\paragraph{Entry, aggregation and market clearing}

As before, suppose first that $z > 0$. The incumbent's FOC implies that, in equilibrium, the term multiplying $z$ in (\ref{eq:hjb_incumbent_workerIndiff}) must equal zero,
\begin{align*}
	0 &= \chi(\lambda-1)\tilde{V}- \hat{w}_{RD} - (1-x)(1 - (1-\kappa_e)\lambda) \nu \tilde{V} - x \kappa_c \nu \tilde{V}
\end{align*}

Solving for $\tilde{V}$ yields
\begin{align}
	\tilde{V} &= \frac{\hat{w}_{RD}}{\chi(\lambda - 1) - (1-x) (1- (1-\kappa_e)\lambda)\nu - x \kappa_{c} \nu} \label{eq:hjb_incumbent_foc}
\end{align}

Entrant innovation satisfies a free entry condition,\footnote{The original condition is 
	\begin{align*}
		\hat{\chi} \hat{z}^{-\psi} (1-\kappa_e)  V(j,t|\lambda q) = \frac{q}{Q_t} \hat{w}_{RD,t}
	\end{align*}
	Using $V(j,t|q) = \tilde{V} q$ and $\hat{w}_{RD,t} = \hat{w}_{RD} Q_t$ yields (\ref{eq:free_entry_condition}).}
\begin{align}
	\underbrace{\hat{\chi} \hat{z}^{-\psi}}_{\mathclap{\text{Marginal innovation rate}}} \overbrace{(1-\kappa_e) \lambda \tilde{V}}^{\mathclap{\text{Payoff from innovation}}} &= \underbrace{\hat{w}_{RD}}_{\mathclap{\text{MC of R\&D}}} \label{eq:free_entry_condition}
\end{align}

Substituting $\tilde{V}$ using (\ref{eq:hjb_incumbent_foc}) yields an expression for entrant R\&D effort, 
\begin{align}
	\hat{z} &= \Big( \frac{\hat{\chi} (1-\kappa_{e}) \lambda}{\chi(\lambda-1) - (1-x) (1- (1-\kappa_e)\lambda)\nu - x \kappa_{c} \nu} \Big)^{1/\psi} \label{eq:effort_entrant}
\end{align}

Market clearing for R\&D labor requires
\begin{align}
	\bar{L}_{RD} &= \int_0^1 \frac{q_j}{Q} (z + \hat{z}) dj = z + \hat{z} \label{eq:RD_labor_market_clearing}
\end{align}
 
which implies
\begin{align}
	z &= \bar{L}_{RD} - \hat{z} \label{eq:zI_asFuncZe}
\end{align}

Growth is determined by the growth accounting equation\footnote{To see this, let $\Delta > 0$ and let $J_0(\Delta)$ ($J_1(\Delta)$) denote the indices $j\in [0,1]$ on which innovation occurs zero (one) times between $t$ and $t+\Delta$. By the law of large numbers, the set $J_1(\Delta)$ has measure $\mu_1 \Delta = (\tau + \tau^S + \hat{\tau})\Delta + o(\Delta)$. The set $J_0(\Delta)$ has measure $1 - \mu_1 \Delta + o(\Delta)$. 
		\begin{align*}
			Q_{t+\Delta} = \int_0^1 \bar{q}_{j,t+\Delta} dj &= \int_{j \in J_0} \bar{q}_{jt} dj + \int_{j \in J_1} \lambda \bar{q}_{jt} dj + o(\Delta) \\
			&= (1 - \mu_1\Delta - o(\Delta)) Q_t + (\mu_1 \Delta + o(\Delta) ) \lambda Q_t + o(\Delta) \\
			&= (1 - \mu_1\Delta) Q_t + \mu_1\Delta \lambda Q_t + o(\Delta)
 	\end{align*}
 where I used the fact that $\mathbb{E}[\bar{q}_{jt} | j \in J_0, t]  = \mathbb{E}[\bar{q}_{jt} | j \in J_1, t] = Q_t$, since innovations happen at the same rate regardless of $\bar{q}_{jt}$. It follows that
\begin{align*}
	\frac{\dot{Q}_t}{Q_t} = \frac{\lim_{\Delta \to 0} \frac{Q_{t+\Delta} - Q_t}{\Delta}}{Q_t} &= (\lambda - 1)\mu_1 
	\end{align*}}
\begin{align}
g &= (\lambda - 1)(\tau + \tau^S + \hat{\tau}) \label{eq:growth_accounting}
\end{align}

The Euler equation determines the interest rate, 
\begin{align}
	g &= \frac{\dot{C}}{C} = \frac{1}{\theta} (r - \rho) \label{eq:euler} \\
	\therefore r &= \theta g + \rho \label{eq:interest_rate}
\end{align}

Substituting the incumbent's FOC into the incumbent's HJB, and using the expression for the interest rate, yields the incumbent's value $\tilde{V}$,
\begin{align}
	 \tilde{V} &= \frac{\tilde{\pi}}{r + \hat{\tau}} \label{eq:hjb_incumbent_gordon_formula}
\end{align}

Finally, the free entry condition (\ref{eq:free_entry_condition}) determines the equilibrium value of $\hat{w}_{RD}$. If (\ref{eq:effort_entrant}) implies that $\hat{z} > \bar{L}_{RD}$, then the assumption that $z > 0$ in a symmetric equilibrium leads to a contradiction, and instead equilibrium has $\hat{z} = \bar{L}_{RD}$ and $z = \tau = \tau^S = 0$. Then $g = (\lambda - 1) \hat{\tau}$ and the rest of the equilibrium allocation and prices can be computed in the same way as before. This implies the following lemma.

\begin{lemma}\label{model:lemma:zge0condition}
	If a symmetric BGP with $z > 0$ exists, it has $\Big( \frac{\hat{\chi} (1-\kappa_{e}) \lambda}{\chi(\lambda-1) - \nu \min\{ 1-(1-\kappa_e) \lambda, \kappa_c \}} \Big)^{1/\psi} < \bar{L}_{RD}$. Conversely, if $\Big( \frac{\hat{\chi} (1-\kappa_{e}) \lambda}{\chi(\lambda-1) - \nu \min\{ 1-(1-\kappa_e) \lambda, \kappa_c \}} \Big)^{1/\psi} < \bar{L}_{RD}$ and a symmetric BGP exists, it has $z > 0$.
\end{lemma}

\begin{proof}
	Proposition \ref{proposition:optimalNCApolicy} implies that entrant effort $\hat{z}$, given by (\ref{eq:effort_entrant}), is equal to $\Big( \frac{\hat{\chi} (1-\kappa_{e}) \lambda}{\chi(\lambda-1) - \nu \min\{ 1-(1-\kappa_e) \lambda, \kappa_c \}} \Big)^{1/\psi}$. 
	
	Using the resulting inequality in (\ref{eq:zI_asFuncZe}) yields $z > 0$. Conversely, using $z > 0$ in (\ref{eq:zI_asFuncZe}) implies $z_E < L_{RD}$ which, given the first part of the proof, implies $\Big( \frac{\hat{\chi} (1-\kappa_{e}) \lambda}{\chi(\lambda-1) - \nu \min\{ 1-(1-\kappa_e) \lambda, \kappa_c \}} \Big)^{1/\psi}$. By 
\end{proof}

The next result provides conditions under which the transversality condition of the financial intermediary holds.

\begin{lemma}
	The transversality condition (\ref{eq:tvc0}) holds provided $\rho > (1-\theta) g$. If $\theta \ge 1$, then it holds for all $\rho > 0$.
\end{lemma}

\begin{proof}
	Using $V(j,t|\bar{q}_{jt}) = \tilde{V} \bar{q}_{jt}$ and $\int_0^1 \bar{q}_{jt} dj = Q_t$, the transversality condition (\ref{eq:tvc0}) becomes
	\begin{align}
	\lim_{t \to \infty} e^{-rt} \tilde{V} Q_t = 0 \label{eq:tvc}
	\end{align}
	
	Because $Q_t = Q_0 e^{gt}$, (\ref{eq:tvc}) is satisfied as long as $r > g$. Given the Euler equation (\ref{eq:euler}), the condition holds as long as $\rho > (1-\theta)g$.  For $\theta \ge 1$, this holds for all $\rho > 0$.  For $\theta < 1$, the condition has bite as it requires $\rho > K(g,\theta)$ for some function $K(g,\theta) > 0$. If the condition does not hold given the values $\tilde{V},r$ and the growth rate $g$ derived above, it means that there does not exist a symmetric BGP for the chosen parameters. 
\end{proof}

Since $\theta \ge 1$ is the empirically relevant case, I do not explore the case $\theta < 1$ in detail. The next lemma shows that, in a symmetric BGP, the financial intermediary TVC is a sufficient condition for the incumbent TVC. Thus, once the incumbent TVC is used to prove $V(j,t|\bar{q}_{jt}) = \tilde{V} \bar{q}_{jt}$, it is sufficient to discard it and verify the equilibrium with the simpler financial intermediary BGP. 

\begin{lemma}
	In a symmetric BGP, the incumbent transversality condition (\ref{eq:tvc_incumbent}) holds whenever the financial intermediary transversality condition (\ref{eq:tvc0}) holds.
\end{lemma}

\begin{proof}
	In a symmetric BGP, Proposition \ref{proposition:hjb_scaling} implies that $V(j,t|\bar{q}_{jt}) = \tilde{V} \bar{q}_{jt}$. Using (\ref{eq:tvc0}), ones has 
	\begin{align}
		0 = \lim_{t \to \infty} e^{-rt} \tilde{V}Q_t &= \lim_{t \to \infty} e^{-rt} \mathbb{E}[ \tilde{V} \bar{q}_{jt} ] \\
												 &\ge \lim_{t \to \infty} e^{-rt} \mathbb{E} [\mathbf{1}_{s(j,t) > t'} \tilde{V} \bar{q}_{jt}] \\
												 &\ge 0
	\end{align}
	The first equality is (\ref{eq:tvc0}). The second equality follows from $V(j,t|\bar{q}_{jt}) = \tilde{V} \bar{q}_{jt}$ and $\int_0^1 \bar{q}_{jt} dji = Q_t$. The first inequality follows from $\tilde{V} \bar{q}_{jt} > 0$ and $\mathbf{1}_{s(j,t) > t'} \le 1$. The second inequality follows from $\tilde{V}\bar{q}_{jt} \ge 0$ and $\mathbf{1}_{s(j,t) > t'} \ge 0$. 
\end{proof}

I can now state some propositions concerning existence and uniqueness of the symmetric BGP. 

\begin{proposition}\label{proposition:purstrategyeq:positiveOI}
	If $\theta \ge 1$, $\kappa_c \ne \bar{\kappa}_c$, and $\Big( \frac{\hat{\chi} (1-\kappa_{e}) \lambda}{\chi(\lambda-1) - \nu \min\{ 1-(1-\kappa_e) \lambda, \kappa_c \}} \Big)^{1/\psi} < \bar{L}_{RD}$, then:
	\begin{enumerate}
		\item There exists a unique symmetric BGP.
		\item On the symmetric BGP, $z > 0$ and $x_{jt} = x$
	\end{enumerate}
\end{proposition}

\begin{proof}
	The derivation above and the hypothesis $\theta \ge 1$ proves that there exists a symmetric BGP. The hypothesis in Lemma \ref{model:lemma:zge0condition}, namely that $\Big( \frac{\hat{\chi} (1-\kappa_{e}) \lambda}{\chi(\lambda-1) - \nu \min\{ 1-(1-\kappa_e) \lambda, \kappa_c \}} \Big)^{1/\psi} < \bar{L}_{RD}$, holds,. This implies that $z > 0$ in a symmetric BGP. Finally, the condition that $\kappa_c \ne \bar{\kappa}_c$ determines $x_{jt} = x$ uniquely, by Proposition \ref{proposition:optimalNCApolicy}. The equilibrium variables are determined uniquely given the choice of $x$ and the representation $V(j,t|q) = \tilde{V}q$ and the scaling of wages $\hat{w}_{RD,t} = \hat{w}_{RD}Qt$ and $w_{RD,j}(x) = w_RD(x) Q_t$. By Lemma \ref{proposition:hjb_scaling}, any symmetric BGP with $z > 0$ admits this representation. Hence, a unique symmetric BGP exists and has $z > 0$ and $x_{jt} = x$. 
\end{proof}

\begin{proposition}\label{proposition:purestrategyeq:incumbents_indifferent}
	If $\theta \ge 1$, $\kappa_c = \bar{\kappa}_c$, and $\Big( \frac{\hat{\chi} (1-\kappa_{e}) \lambda}{\chi(\lambda-1) - \nu \min\{ 1-(1-\kappa_e) \lambda, \kappa_c \}} \Big)^{1/\psi} < \bar{L}_{RD}$, then:
	\begin{enumerate}
		\item There exist exactly two symmetric BGPs with $x_{jt} = x$: one with $x_{jt} = 0$ and one with $x_{jt} = 1$.
		\item Symmetric equilibria with $x_{jt} = x$ all have the same R\&D labor allocation $z, \hat{z}$
		\item The equilibrium with $x_{jt} = 0$ has a higher growth rate $g$ 
	\end{enumerate} 
\end{proposition}

\begin{proof}
	The proof of the first part is essentially the same as that of the previous proposition. The only difference is that either choice $x_{jt} = 1$ or $x_{jt} = 0$ is valid under Proposition \ref{proposition:optimalNCApolicy}. Given the representation $V(j,t|q) = \tilde{V}q$ and the scaling of wages $\hat{w}_{RD,t} = \hat{w}_{RD}Qt$ and $w_{RD,j}(x) = w_RD(x) Q_t$, the derivation above uniquely determines uniquely the rest of the equilibrium conditional on $x$. By Lemma \ref{proposition:hjb_scaling}, this representation applies in any symmetric BGP. 
	
	The second part follows from the fact that when $\kappa_c = \bar{\kappa}_c$, the expressions for equilibrium R\&D effort $\hat{z},z$ do not depend on $x$. The reason is that $x$ only affects $\hat{z},z$ through its effect on the incumbent's effective wage, but here is the incumbent is indifferent between $x = 1$ and $x = 0$ hence faces the same effective wage. Mathematically, (\ref{eq:effort_entrant}) has the expression $(1-x)(1-(1-\kappa_e)\lambda)\nu - x \kappa_c \nu = (1-x) \bar{\kappa}_c \nu + x \kappa_c \nu$ in the denominator. Since $\kappa_c = \bar{\kappa}_c$, $\hat{z}$ is unaffected by $x$, which in turn implies $z$ is also unaffected.
	
	The last statement follows from the fact that $z,\hat{z}$ are the same in both equilibria, but $\tau^S = 0$ when $x = 1$ and $\tau^S = \nu z^I > 0$ when $x = 0$. By the growth accounting equation (\ref{eq:growth_accounting}), this implies $g$ is higher when $x = 0$. 
\end{proof}

\begin{proposition}\label{proposition:purstrategyeq:zeroOI}
	If $\theta \ge 1$ and $\Big( \frac{\hat{\chi} (1-\kappa_{e}) \lambda}{\chi(\lambda-1) - \nu \min\{ 1-(1-\kappa_e) \lambda, \kappa_c \}} \Big)^{1/\psi} \ge \bar{L}_{RD}$, there is a unique symmetric BGP (modulo irrelevant incumbent choice of $x_{jt}$); and this BGP has $z = 0$ and $\hat{z} = \bar{L}_{RD}$.
\end{proposition}

\begin{proof}
	By Lemma \ref{model:lemma:zge0condition}, $z = 0$ in a symmetric BGP. Then $\hat{z} = \bar{L}_{RD}$ by R\&D labor market clearing (\ref{eq:zI_asFuncZe}). The rest of the equilibrium is pinned down given $x$ by the derivation given the representation $V(j,t|q) = \tilde{V}q$ and the scaling of wages $\hat{w}_{RD,t} = \hat{w}_{RD}Qt$ and $w_{RD,j}(x) = w_RD(x) Q_t$, which Lemma \ref{proposition:hjb_scaling} proves holds on any symmetric BGP. This shows that the equilibrium does not depend on the choice of $x$ by incumbents in a symmetric equilibrium with $z = 0$. Hence, the symmetric BGP is unique modulo an irrelevant choice by the incumbent of whether to use $x_{jt} = 0$ or $x_{jt} = 1$. 
\end{proof}

Finally, there is a technical possibility of "mixed strategy" equilibria on the knife-edge $\kappa_c = \bar{\kappa}_c$ where both choices of $x$ occur in equilibrium. This result is included for completeness but I will not study this case further in this paper.

\begin{proposition}\label{proposition:mixedstrategyeq}
	If $\theta \ge 1$, $\kappa_c = \bar{\kappa}_c$, and $\Big( \frac{\hat{\chi} (1-\kappa_{e}) \lambda}{\chi(\lambda-1) - \kappa_{c} \nu} \Big)^{1/\psi} < \bar{L}_{RD}$, then for all $f \in (0,1)$ there exists a symmetric BGP in which, at any given time $t$, a fraction $f$ of incumbents $j$ have $x_{jt} = 1$.  
\end{proposition}

\begin{proof}
	See Appendix \ref{appendix:model:proofs:proposition:mixedstrategyeq}.
\end{proof}


\section{Efficiency and theoretical policy analysis}\label{model:efficiency:efficiency}

Below I discuss theoretically the efficiency of the decentralized equilibrium and how this depends on parameters. 

\subsection{Preliminaries}

\subsubsection{Welfare}

Denote social welfare from time $t$ onwards by $W_t$. Since the model's only agent is a representative household, social welfare is simply the household's utility from consumption,
\begin{align}
	W_t = \int_t^{\infty} e^{-\rho s} \frac{C(t+s)^{1-\theta} - 1}{1-\theta} ds \label{eq:agg_welfare0}
\end{align}

Using $C(t+s) = C(t) e^{gs}$ on the BGP and integrating yields
\begin{align}
	W_t &= \frac{C(t)^{1-\theta} }{(1-\theta)(\rho - g(1-\theta))} + R(\rho,\theta)
\end{align}

for some $R(\rho,\theta)$ which is constant across equilibria and is therefore irrelevant to welfare comparisons. Since $C(t) = \tilde{C}e^{gt}$, one can define
\begin{align}
	\tilde{W} &= \frac{\tilde{C}^{1-\theta}}{(1-\theta)(\rho - g(1-\theta))} \label{eq:agg_welfare1}
\end{align}

so that $\tilde{W} e^{(1-\theta)gt} = W_t - R(\rho,\theta)$. This illustrates that social welfare can be decomposed into two channels: a \textit{growth} channel via $g$ and a \textit{steady-state consumption} channel via $\tilde{C}$. Higher values for either imply higher welfare. In turn, $\tilde{C}$ can be decomposed using 
\begin{align}
	\tilde{C} &= \tilde{Y} - \overbrace{(\hat{\tau} + \tau^S) \kappa_e \lambda \tilde{V}}^{\mathclap{\text{Creative destruction cost}}} - \underbrace{x z \kappa_c \nu \tilde{V}}_{\mathclap{\text{NCA enforcement cost}}} \label{eq:agg_consumption_decomposition}
\end{align}

so that steady-state consumption is flow output of the final good minus the final goods cost of creative destruction and of NCA enforcement.

\subsubsection{Efficiency of decentralized equilibrium}

A natural starting point for assessing efficiency is to solve the social planner's problem, which can serve as a benchmark against which to evaluate the efficiency of the decentralized equilibrium. However, in this model some of the primitives, in particular the cost of entry $\kappa_e  V(j,t|\lambda \bar{q}_{jt})$ and the cost of NCAs $\kappa_c \nu V(j,t|\bar{q}_{jt})$, are specified as a function of the equilibrium value of incumbency. These objects are only well-defined in a decentralized market equilibrium, so the first-best allocation is itself not well-defined in this setting.\footnote{As discussed previously, the model could be modified not to have this feature, but the solution would be analytically less convenient. In particular, inequality (\ref{cs:growth_misallocation_condition}) below would much more complicated.} 

Nevertheless, the decentralized equilibrium is inefficient in the sense that policy can be used to improve welfare. In the next two sections, I make some general observations about the two types of misallocation in the model, and in particular provide a key equation that can be used to quantify the extent of misallocation of R\&D labor. In Section \ref{model:efficiency:policy_analysis}, I apply these insights to analyze the effect of reducing the cost of NCA enforcement $\kappa_c$, as well as other substitute and complementary policies such as R\&D subsidies. After calibrating the model in Sections \ref{sec:empirics} and \ref{sec:calibration}, I numerically compute the welfare effect of varying these policies in Section \ref{sec:policy_analysis}.

\subsubsection{Types of misallocation}

The model has three factors of production: production labor, R\&D labor and intermediate goods, which are used in the production of the final good. Production of the final good is competitive and introduces no distortions on its own. Hence, the discussion around efficiency of use of productive factors focuses on the allocation of production labor and the allocation of R\&D labor. In addition, since the use of NCAs affects the effective rate of production of new spinouts, there is also scope for misallocation of NCAs. Below I discuss these factors in detail.

\subsection{Misallocation of production labor: monopoly distortion}

The allocation of production labor in the economy is distorted by the monopolistic competition in the intermediate goods market. Producers of intermediate goods charge a price higher than marginal cost and therefore produce less output than in a competitive market. In equilibrium this leads to an overallocation of production labor to final goods production. The effect is that steady-state consumption $\tilde{C}$ is lower than would be achievable by a social planner. From now I ignore this source of inefficiency as it is not the focus of this model.\footnote{In this setting with exogenous total supply of R\&D, a subsidy to intermediate goods production would correct this externality and have no effect on equilibrium growth.}

\subsection{Misallocation of R\&D labor}\label{model:efficiency:misallocationRD}

The decentralized allocation of R\&D labor is also, in general, not efficient. Because total R\&D spending is exogenous, any inefficiency must be due to a misallocation of R\&D \textit{between} OI by incumbents and CD by entrants. And in turn, any inefficiency manifests as either lower growth $g$ or lower steady-state consumption $\tilde{C}$.

\subsubsection{Effect on growth}

To isolate the determinants of the degree of equilibrium misallocation, first consider the equilibrium marginal effects on the innovation rate from more incumbent OI and entrant CD, respectively. If the marginal effect of entrant CD on innovation is lower, then equilibrium innovation, and therefore growth, would increase after a reallocation of R\&D labor to incumbent OI. The marginal effect of OI, including the induced innovation by WSOs, is equal to $\chi + (1-x) \nu$. The marginal effect of CD by entrants is
\begin{align}
\frac{d}{d\hat{z}} \hat{\tau} &= (1-\psi) \hat{\chi} \hat{z}^{-\psi} \label{eq:marginal_effect_effort_entrant}
\end{align}
%
Substituting the expression for $\hat{z}$ in (\ref{eq:effort_entrant}), dividing by $\chi + (1-x)\nu$, and rearranging yields 
\begin{align}
	\frac{\frac{d}{d\hat{z}} \hat{\tau}}{\chi + (1-x)\nu} &= \underbrace{\frac{1}{1-\kappa_{e}}}_{\mathclap{\text{Entry cost}}} \times \overbrace{\frac{\chi(\lambda-1) -(1-x) (1-(1-\kappa_e)\lambda)\nu - x \kappa_c \nu}{\chi(\lambda-1)}}^{\mathclap{\text{Entrants face different effective cost of R\&D}}} \times \underbrace{\frac{\chi}{\chi + (1-x)\nu}}_{\mathclap{\text{Effect of OI on WSO formation}}} \times \overbrace{\frac{\lambda-1}{\lambda}}^{\mathclap{\text{Business stealing}}} \times  \underbrace{(1-\psi)}_{\mathclap{\text{Congestion}}}  \label{cs:growth_misallocation_condition}
\end{align}

\paragraph{Effective cost of R\&D} 

The term $\frac{\chi(\lambda-1) -(1-x) (1-(1-\kappa_e)\lambda)\nu - x \kappa_c \nu}{\chi(\lambda-1)}$ reflects the fact that entrants pay a different effective cost of R\&D than the incumbent. Typically, incumbents pay a higher effective cost because they internalize the harm from future WSOs, or alternatively must pay a cost to enforce NCAs to prevent them. In equilibrium, this means entrants do more R\&D and hence they have a lower marginal effect on growth. If $(1-\kappa_e) \lambda > 1$, however, then incumbents actually benefit from WSOs \textit{ex ante} because they are bilaterally efficient, and as a consequence they pay a lower effective cost of R\&D. This has the opposite effect of increasing the equilibrium marginal effect on growth of entrant R\&D. 

\paragraph{Entry cost}

Similarly, the term $\frac{1}{1-\kappa_e} \ge 1$ reflects the additional entry cost paid by entrants upon innovating. It changes the effective cost of innovation for entrants relative to incumbents. This reduces $\hat{z}$ in equilibrium provided $\kappa_e > 0$. This tends to reduce the extent of misallocation, as it works against the net of the other terms on the RHS of equation (\ref{cs:growth_misallocation_condition}), although this does come at the cost of reduced steady-state consumption. I consider these effects in the next section. In the calibration, the decentralized R\&D labor allocation is sufficiently suboptimal for growth (i.e., (\ref{cs:growth_misallocation_condition}) is sufficiently less than unity) so growth effects dominate. 

\paragraph{Spawning of WSOs}

The term $\frac{\chi}{\chi + (1-x)\nu}$ reflects the contribution to the productivity of OI stemming from entry by WSOs. If $x = 0$ and $\nu > 0$, incumbents generate more growth through R\&D than simply their OI. Because this reflects higher effective productivity of OI in producing innovations, all else equal it increases the extent of misallocation. If $x = 1$ or $\nu = 0$ this term is equal to 1 and has no effect on the inequality, corresponding to $\tau^S = 0$.

\paragraph{Business stealing}

The next term, $\frac{\lambda - 1}{\lambda} < 1$, reflects the \textit{business stealing} externality: innovation by entrants imposes a negative externality on the profits of the incumbent. This means that entrants can earn the required (private) return on R\&D with a lower innovation rate per marginal cost than incumbents. In the calibration, $\lambda \approx 1.16$ so $\frac{\lambda-1}{\lambda} \approx 0.14$ -- quite a strong effect.\footnote{In models such as \cite{aghion_competition_2005}, this effect is attenuated by the fact that incumbents engage in neck-and-neck competition within each good $j$. This means R\&D by incumbents has a negative externality on other incumbents in the same good $j$, making the situation more symmetric between incumbents and entrants. I plan to explore this question further in later work.}

\paragraph{Congestion}

The term $1-\psi < 1$ reflects the \textit{congestion} externality: individual entrants impose a negative externality on the expected returns of other entrants by duplicating each other's efforts. As with business-stealing, the congestion externality also tends to overallocate R\&D to entrants. To give a sense of magnitude, in the calibration $\psi = 0.5$ so $1-\psi = 0.5$.

\subsubsection{Effect on steady-state consumption}

The allocation of R\&D labor also has an effect on steady-state consumption. Restating (\ref{eq:agg_consumption_decomposition}), steady-state consumption is given by
\begin{align} 
\tilde{C} = \tilde{Y} - \big(\hat{\tau} + (1-x) z \nu\big) \kappa_e \lambda \tilde{V}  - x z \nu \bar{\kappa}_c \tilde{V}    
\end{align} 

Consider how the above expression varies in $\hat{z}$. First, a higher $\hat{z}$ implies a higher entrant innovation rate $\hat{\tau}$ and thereby a higher flow cost of creative destruction $(\hat{\tau} + \tau^S)\kappa_e \lambda \tilde{V}$. Furthermore, the increase in $\hat{z}$ reduces $z$ by the R\&D labor market clearing condition (\ref{eq:RD_labor_market_clearing}). If NCAs are used ($x = 1$) then this implies a reduction in the flow cost of NCA enforcement, $x z \kappa_c \nu \tilde{V}$. If NCAs are not used ($x = 0$) then lower $z$ reduces the entry cost paid by WSOs, $(1-x) \nu  z \kappa_e \lambda \tilde{V}$. Finally, (\ref{eq:hjb_incumbent_gordon_formula}) implies that the increase in $\hat{\tau}$ can reduce $\tilde{V}$, although it may be partly or wholly offset by the reduction in the interest rate $r$ stemming from lower growth $g$, via the Euler equation (\ref{eq:interest_rate}). Overall, $\tilde{V}$ may increase or decrease. \textbf{[Put an expression adding up all of these effects showing that the sign is not determined]} 

To make things more clear, consider a model where $\hat{z}$ is set exogenously, i.e. the free entry condition (\ref{eq:free_entry_condition}) does not necessarily hold. The propositions on existence and uniqueness of symmetric BGPS in the previous chapter go through with the natural modifications. When $\psi = 0.5$ as will be the case in the calibration of Section \ref{sec:calibration}, the change of variables $\zeta = \hat{z}^{1/2}$ yields
\begin{align}
	\tilde{C} = \tilde{Y} - \Big(\big(\hat{\chi} \zeta \kappa_e \lambda + (1-x) (\bar{L}_{RD} - \zeta^2) \nu\big) \kappa_e \lambda   - x (\bar{L}_{RD} - \zeta^2) \nu \bar{\kappa}_c \Big)\tilde{V} 
\end{align}

where I used the R\&D labor market clearing condition (\ref{eq:RD_labor_market_clearing}) and the definition of $\hat{\tau}$. Differentiating this expression with respect to $\zeta$ yields
\begin{align}
	\frac{d\tilde{C}}{d\zeta} &= - \Lambda \frac{d\tilde{V}}{d\zeta}  + \Big(  \hat{\chi} \kappa_e \lambda \big( 1 - 2\frac{1-\kappa_e}{\chi(\lambda -1) - x \kappa_c \nu - (1-x) (1 - (1-\kappa_e) \lambda) \nu }\big) \Big) \tilde{V} \\
	\Lambda &= \hat{\chi} \hat{z}^{1/2} \kappa_e \lambda  + x (\bar{L}_{RD} - \hat{z}) \nu \kappa_c + (1-x)  (\bar{L}_{RD} - \hat{z}) \nu \kappa_e \lambda 
\end{align}

where I substituted for $\zeta$ in the first line using (\ref{eq:effort_entrant}) and in the second line using $\zeta = \hat{z}^{1/2}$. 

By (\ref{eq:hjb_incumbent_gordon_formula}),
\begin{align}
	\tilde{V} &= \frac{\tilde{\pi}}{\underbrace{\theta (\lambda -1) (\chi (\bar{L}_{RD} - \hat{z}) + \hat{\chi} \hat{z}^{1/2} + (1-x) \nu (\bar{L}_{RD} - \hat{z})) + \rho}_{\mathclap{= r\text{ by (\ref{eq:growth_accounting}), (\ref{eq:interest_rate}) }} }  + \hat{\chi} \hat{z}^{1/2}} 
\end{align}

The value of incumbency is increasing in the growth rate, via the interest rate, and decreasing in the rate of creative destruction by entrants $\hat{\chi} \hat{z}^{1/2}$. These two mechanisms work opposite each other so that in the calibration $\frac{d\tilde{V}}{d\hat{z}}$, and hence $\frac{d\tilde{V}}{d\zeta}$, is relatively small in magntiude. Whether steady-state consumption $\tilde{C}$ increases or decreases from an overallocation to $\hat{z}$ is therefore largely determined by the sign of 
\begin{align}
	\frac{d\Lambda}{d\zeta} &= \hat{\chi} \kappa_e \lambda \Big( 1 - 2\frac{1-\kappa_e}{\chi(\lambda -1) - x \kappa_c \nu - (1-x) (1 - (1-\kappa_e) \lambda) \nu }\Big)
\end{align} 

Heuristically, if $\frac{d\Lambda}{d\zeta} < 0$ and $\frac{d\tilde{V}}{d\zeta}$ is small in magnitude, then $\tilde{C}$ decreases when R\&D labor is exogenously reallocated from OI by incumbents to CD by entrants. In the calibration of Section \ref{sec:calibration}, this is the case.

\subsubsection{Effect on welfare}

Because the sign of the change in consumption can be positive or negative depending on parameters, in the decentralized equilibrium it is not possible to say in general whether R\&D is over or underallocated to CD by entrants or OI by incumbents. Numerically, in the calibration of Section \ref{sec:calibration}, there is overallocation of R\&D to CD by entrants. 

\subsection{Misallocation of NCAs}\label{model:efficiency:misallocationNCAs}

\subsubsection{Effect on growth}

Consider a BGP with incumbent innovation effort $z$. Switching from $x = 1$ to $x = 0$ while holding $z,\hat{z}$ constant increases the growth rate by $(\lambda -1) z \nu$. Therefore a choice of $x = 1$ \textit{always} constitutes a misallocation in that, in partial equilibrium, it reduces the aggregate growth rate. In general equilibrium, however, this misallocation reduces underallocation of OI R\&D, by increasing the incentives to perform it. As is often the case, introducing a second source of misallocation can improve the outcome in a decentralized equilibrium.

Still, in certain parts of the parameter space, there is a deeper sense of misallocation of NCAs in that an exogenous change to $x = 0$ increases growth and welfare even after taking into account the reduction in incumbents' R\&D incentives. That is, incumbents overuse NCAs from a social perspective even taking into account their interaction with the other externalities in the economy. This can be most clearly seen by considering a case where the cost of NCAs such that incumbents are indifferent between using and not using NCAs, i.e. $\kappa_c = \bar{\kappa}_c$; by continuity the argument can be extended to the case $\kappa_c = \bar{\kappa}_c - \varepsilon$ for $\varepsilon > 0$ sufficiently small. By Proposition \ref{proposition:purestrategyeq:incumbents_indifferent}, there are exactly two symmetric BGPs with $x_{jt} = x$: a low-growth BGP with $x = 1$ and a high-growth BGP with $x = 0$. Intuitively, when NCAs are not used ($x = 0$) incumbents have the same effective cost of R\&D; hence, even if (\ref{cs:growth_misallocation_condition}) is less than unity, there is no reallocation of R\&D and growth increases unequivocally due to the increased innovation by WSOs. 

\subsubsection{Effect on steady-state consumption}

Whether the increase in growth in the last paragraph actually increases welfare depends on its effect on steady-state consumption. The change in steady-state consumption from $x = 1$ to $x = 0$ is the net of the reduction in NCA enforcement costs and the reduction in entry costs,
\begin{align}
	\Delta \tilde{C} &= z_{x = 1} \nu \bar{\kappa}_c  \tilde{V}_{x = 1} - z_{x = 0} \nu \kappa_e \lambda \tilde{V}_{x = 0} + \hat{\tau}_{x = 1} \kappa_e \lambda \tilde{V}_{x=1} - \hat{\tau}_{x = 0} \kappa_e \lambda \tilde{V}_{x =0}
\end{align}

where the subscript $x = i$ for $i \in \{0,1\}$ denotes the BGP the variable is drawn from. The first two terms are the reductions in NCA costs (positive) and entry costs (negative) associated with WSOs. The last two terms constitute the reduction in entry costs associated entrants. As argued in the last paragraph, $z_{x = 0} = z_{x=1}$ and $\hat{\tau}_{x = 0} = \hat{\tau}_{x = 1}$. By contrast, $\tilde{V}_{x=0} < \tilde{V}_{x=1}$ due to the increased interest rate induced by the higher BGP growth rate (see equation (\ref{eq:interest_rate})). Using the definition of $\bar{\kappa}_c = 1 - (1-\kappa_e) \lambda$ and rearranging yields
\begin{align}
\Delta \tilde{C} &= \tilde{V}_{x = 1} \Bigg( z \nu \Big( 1 - (1-\kappa_e)\lambda - \frac{\tilde{V}_{x = 0}}{\tilde{V}_{x = 1}}\kappa_e \lambda \Big) + \hat{\tau} \kappa_e \lambda \Big( 1 - \frac{\tilde{V}_{x = 0}}{\tilde{V}_{x = 1}}\Big) \Bigg) \label{eq:misallocation_NCA_consumption}
\end{align}

The second term in parentheses is always positive. The first term in parentheses is positive if and only if\footnote{The ratio $\frac{V_{x = 0}}{V_{x = 1}}$ is simply equal to the ratio of the discount factors used in either equilibrium, $\frac{V_{x = 0}}{V_{x = 1}} = \frac{r_{x = 1} + \hat{\tau}}{r_{x = 0} + \hat{\tau}}$ which can be expressed as a closed form,
	\begin{align*}
	\frac{V_{x = 0}}{V_{x = 1}} &= \frac{\theta(\lambda - 1) (\tau+ \hat{\tau}) + \rho + \hat{\tau})}{\theta(\lambda - 1) (\tau+ \hat{\tau} + z\nu )  + \rho + \hat{\tau})}
	\end{align*} 
	What is left is to substitute the equilibrium expressions for $z$, $\tau$, and $\hat{\tau}$}
\begin{align}
	\frac{\lambda -1}{\lambda} < \big( 1 - \frac{V_{x = 0}}{V_{x = 1}} \big) \kappa_e  \label{cs:consumption_decreasing_condition}
\end{align}

The first term dominates when $z$ is large compared to $\hat{z}$. Everything is expressible in closed form the but the expressions are unwieldy enough to not provide further insight. In the calibration of Section \ref{sec:calibration}, (\ref{cs:consumption_decreasing_condition}) is violated as $1 - \frac{V_{x = 0}}{V_{x = 1}}$ is on the order of $0.01$ so the RHS of is close to zero, while the LHS is always strictly positive as $\lambda > 1$ so the first term is negative. Furthermore, incumbents do about twice as much R\&D as entrants ($z / \hat{z} \approx 2 $) so the first term in (\ref{eq:misallocation_NCA_consumption}) dominates and $\Delta \tilde{C} < 0$.

\subsubsection{Effect on welfare}

Given that $\Delta \tilde{C}$ may be positive or negative, the effect on welfare from the misallocation in NCA use can in principle be positive or negative. Intuitively, however, the increase in growth dominates for two reasons. First, the reduction in $\tilde{C}$ is the net of two forces of similar magnitude, so it is small. Second, the increase in growth has positive externalities because it improves the effective productivity of the rest of the economy, both in static terms (directly by increasing $\tilde{C}$ and indirectly by improving intermediate goods production technology (\ref{intermediate_goods_production})) and in dynamic terms (entrants work to improve a higher quality technology). In the calibration of Section \ref{sec:calibration}, $\Delta \tilde{C} < 0$ is sufficiently small in magnitude that, for $\varepsilon > 0$ small $\Delta \tilde{W} > 0$ upon a switch from $x = 1$ to $x = 0$. 

\subsection{Effect of NCA enforcement and other policies}\label{model:efficiency:policy_analysis}

To go further in studying the efficiency properties of the model, I conduct a sequence of theoretical second-best analyses assuming the planner can control one or more parameters and/or Pigouvian taxes. The key question is whether NCA enforcement is good or bad for growth and / or social welfare. In addition, I also study other policies which may substitute or complement NCA enforceability policy, such as R\&D subsidies. While such policies can technically be studied in a standard quality ladders model, their \textit{interaction} with the endogenous use of NCAs is novel and yields some new theoretical insights.

\paragraph{Policies considered} 

I study planners who can control:

\begin{enumerate}
	\item Cost of NCAs: $\kappa_c$ 
	\item R\&D subsidy (tax): $T_{RD}$
	\item Creative destruction tax (subsidy): $T_e$
	\item OI R\&D subsidy (tax): $T_{RD,I}$
	\item All of the above: $\{\kappa_c, T_{RD}, T_{RD,I}, T_e\}$
\end{enumerate}

\paragraph{Comparative statics}

All comparisons below are static comparisons between BGPs. I often use language like "as [a certain parameter or tax] increases..." or "as [parameter] crosses [a threshold], [equilibrium variable] jumps...". This does not refer to a transition path of the economy but rather a static comparing the BGPs for each value of the parameter or tax.\footnote{That being said, in this model it is the case that the economy immediately jumps to the unique new BGP following a parameter or tax change, provided it is assumed that the pre- and post-change equilibria are symmetric.}

\paragraph{Public finance} 

In cases of taxes (subsidies), I assume that they are rebated to (financed by) the representative household in a lump-sum payment. Because there is no labor-leisure choice, this does not create any additional distortions in the economy.\footnote{In general, however, this is a matter of first-order importance and an interesting avenue for further research. Policies should be evaluated in terms of "bang for buck."}

\subsubsection{NCA cost $\kappa_c$} 

As discussed in preceding section, the use of NCAs in the decentralized model may increase or decrease growth and welfare. Therefore the same is true of a change NCA enforcement. 

\textbf{[Maybe shorten this given I discuss a lot of it already]}The overall effect is the net of two countervailing forces. On the one hand, if NCAs can be used, then spinouts which are otherwise bilaterally inefficient can be prevented from entering. This mitigates the disincentive WSOs pose for OI by incumbents, increasing OI at the expense of CD and thereby increasing growth provided (\ref{cs:growth_misallocation_condition}) is less than unity (as it is in the quantitative analysis of Section \ref{sec:policy_analysis}). On the other hand, as discussed in the previous section, there are several positive externalities of innovation: it directly increases steady-state consumption, and in addition it improves the overall productivity of the intermediate goods sector (recall the production function (\ref{intermediate_goods_production})) and improves the innovation technology of entrants. Thus WSOs are typically socially valuable even when they are bilaterally inefficient. There is a cost to preventing their entry which may outweigh the increased incentive for incumbent OI. If the former effect dominates, inhibiting the use of NCAs decreases growth; if the latter effect dominates, the opposite is the case. In addition, as discussed previously, steady-state consumption is also relevant but it tends to be second order.

To analyze this question using the model, consider a planner who controls the parameter $\kappa_c$. I interpret this as a policymaker changing the extent of restrictions on NCAs so that they are more or less expensive to enforce. A natural assumption is to posit a minimum cost $\munderbar{\kappa}_c \ge 0$ such that $\kappa_c \ge \munderbar{\kappa}_c$. NCAs require costs for enforcement even if they are fully endorsed by the legal system: a contract must be written and, in the case of infringement, it must be established that the employee is, in fact, competing with their previous employer. For simplicity, in the analysis below I assume $\munderbar{\kappa}_c = 0$.

\paragraph{Effect on growth}

Suppose first that $x = 1$ and $z > 0$ at $\kappa_{c0}$ and consider a counterfactual in which the planner has set the cost of NCAs to $\kappa_{c1} > \kappa_{c0}$ instead. Equation (\ref{eq:effort_entrant}) immediately implies that $\hat{z}$ increases and Equation (\ref{eq:zI_asFuncZe}) in turn that $z$ decreases. Intuitively, the increase in $\kappa_c$ makes R\&D more expensive for incumbents, reducing $z$ to zero in partial equilibrium. To clear the labor market, $\hat{w}_{RD}$ must decline to induce more R\&D. Because incumbents pay for R\&D not just through wages but implicitly through future WSOs, in the new equilibrium, their effective cost of R\&D is higher relative to entrants, whose only R\&D cost is the R\&D wage. As a result, incumbents employs a smaller share of the R\&D labor in equilibrium.

This reallocation of R\&D labor decreases the BGP growth rate if and only if the marginal effect on growth of incumbent R\&D is higher than the marginal efect on growth of entrant R\&D. As discussed in Section \ref{model:efficiency:misallocationRD}, the former is equal to $\chi + (1-x)\nu$, while the latter is equal to 
\begin{align}
\frac{d}{d\hat{z}} \hat{\tau}(\kappa_c) &= (1-\psi) \hat{\chi} \hat{z} (\kappa_c)^{-\psi} \label{eq:marginal_effect_effort_entrant_kappaC}
\end{align}

where $\hat{\tau}(\kappa_c), \hat{z}(\kappa_c)$ denote the equilibrium value of $\hat{\tau}, \hat{z}$ for a given parameter value of $\kappa_c$. This is a restatement of (\ref{eq:marginal_effect_effort_entrant}) which explicitly shows the dependence of equilibrium objects on the parameter $\kappa_c$. Because $\hat{z}(\kappa_c)$ is monotonically increasing and $\psi > 0$, $\frac{d}{d\hat{z}} \hat{\tau} |_{\kappa_c}$ is monotonically decreasing in $\kappa_c$. Therefore, $\frac{d}{d\hat{z}}\hat{\tau} |_{\kappa_c} < \frac{d}{d\hat{z}}\hat{\tau}|_0$, which implies that if $\chi + (1-x)\nu > \frac{d}{d\hat{z}} \hat{\tau} |_0$, then $\chi + (1-x)\nu > \frac{d}{d\hat{z}} \hat{\tau} |_{\kappa_c}$ for $\kappa_c > 0$. That is, if entrant R\&D has a lower marginal effect on innovation than that of incumbents when $\kappa_c = 0$, then the same statement holds for $\kappa_c > 0$, as the marginal effect of R\&D by entrants is decreasing in $\kappa_c$.  Using (\ref{cs:growth_misallocation_condition}) at $\kappa_c = 0$ and $x = 1$ yields
\begin{align} \overbrace{\frac{\lambda-1}{\lambda}}^{\mathclap{\text{Business stealing}}} \times \underbrace{(1-\psi)}_{\mathclap{\text{Fishing out}}} \times  \overbrace{\frac{1}{1-\kappa_{e}}}^{\mathclap{\text{Entry cost}}}< 1 \label{cs:growth_decreasing_condition}
\end{align}

If (\ref{cs:growth_decreasing_condition}) holds, then increasing $\kappa_c$ with $x = 1$ reduces growth. In the calibration of Section \ref{sec:calibration}, the LHS of (\ref{cs:growth_decreasing_condition}) is equal to about 0.33 and $\kappa_c = 0$ implies $x = 1$ because $\bar{\kappa}_c > 0$. Therefore, increasing $\kappa_c$ on $[0,\bar{\kappa}_c)$ reduces the equilibrium growth rate.

Next, suppose $x = 0$ and $z > 0$ at $\kappa_{c0} < \bar{\kappa}_c$ and consider an increase to $\kappa_{c1}$. If $\kappa_{c1} < \bar{\kappa}_c$ then $x = 0$ and there is no change in the equilibrium because the NCA cost is not paid in either case. If $\kappa_{c1}$ is large enough then $x = 1$. Upon crossing the $\bar{\kappa}_c$ threshold, $\tau^S$ jumps from $0$ to $\nu z$ while $z,\hat{z}$ do not jump. As discussed in Section \ref{model:efficiency:misallocationNCAs}, this unequivocally increases grwoth. Specifically, by the growth accounting equation (\ref{eq:growth_accounting}), the growth rate jumps to $g_1 > g_0$. By the Euler equation (\ref{eq:euler}), the interest rate jumps from $r$ to $r_1>r_0$. Equation (\ref{eq:hjb_incumbent_gordon_formula}) then implies that $\tilde{V}$ declines. If $z > 0$ with $\kappa_{c1}$, the incumbent FOC (\ref{eq:hjb_incumbent_foc}) then implies that the R\&D wage declines to $w_{RD1} < w_{RD0}$. If $z = 0$ with $\kappa_{c1}$, then the above changes occur until the threshold $\kappa_{c}'$ such that $z = 0$, beyond which further increases in $\kappa_c$ have no effect on the equilibrium. 

Intuitively, the increase in $\tau^S$ at the jump from $x = 1$ to $x = 0$ has no direct effect on innovation incentives because at the threshold $\kappa_c = \bar{\kappa}_c$, incumbents' effective cost of R\&D is the same whether $x = 1$ or $x = 0$. In partial equilibrium, this means that the growth rate increases which, in general equilibrium, increases the interest rate. This reduces the value of incumbency, but as this affects incumbents' and entrants' payoffs to R\&D symmetrically and total R\&D spending is fixed, it has no effect on the allocation of R\&D between incumbents and entrants and therefore growth. Hence, growth increases in general equilibrium as well. The reduced value of incumbency is simply passed through to lower R\&D wages.

Finally, note that the above discussion implies that if $z = 0$ at $\kappa_{c0}$ then increasing $\kappa_{c0}$ has no effect, regardless of the value of $x$. 

\paragraph{Consumption}\label{cs:consumption1}

Consider again $\kappa_c \in [0, \bar{\kappa}_c)$. As $\tau^S = 0$ and using R\&D labor market clearing (\ref{eq:RD_labor_market_clearing}), consumption is given by 
\begin{align}
\tilde{C} &= \tilde{Y} - \Big( \hat{\chi} (\bar{L}_{RD} - z)^{1-\psi} \kappa_e \lambda + z \nu \kappa_c \Big) \tilde{V} \label{cs:consumption_eq}
\end{align}

As discussed in the preceding section, an increase in $\kappa_c$ reduces $z$ and increases $\hat{z}$. The logic in Section \ref{model:efficiency:misallocationRD} means that, on its own, this may increase or decrease $\tilde{C}$. In addition, in this case $\kappa_c$ is increasing, which adds a term $- z\nu \tilde{V}$ to the derivative of $\tilde{C}$ in $z$. 

As $\kappa_c$ crosses the $\bar{\kappa}_c$ threshold, incumbents no longer use NCAs ($x = 0$). The logic in Section \ref{model:efficiency:misallocationNCAs} applies in this case without modification, as the change in $\kappa_c$ while crossing the threshold is infinitesimal. Therefore the change in steady-state consumption $\tilde{C}$ can be positive or negative, depending on parameters. 

\paragraph{Welfare}

The previous two sections show that an increase in $\kappa_c$ may increase or decrease both growth and welfare, depending on parameters. I postpone discussion of welfare until the parameters have been disciplined by the calibration in Section \ref{sec:calibration}.

\subsubsection{RD subsidy (tax)}

Suppose that the planner subsidizes R\&D spending at rate $T_{RD}$ (tax if $T_{RD} < 0$). Such a policy is natural to study since R\&D subsidies are prevalent and large throughout the developed world. In particular, in the United States they are large at the federal and state levels (around 20\%).

In this case, in a symmetric BGP the incumbent's normalized HJB becomes
\begin{align}
(r + \hat{\tau}) \tilde{V} = \tilde{\pi} + \max_{\substack{x \in \{0,1\} \\ z \ge 0}} \Big\{z &\Big( \overbrace{\chi (\lambda - 1) \tilde{V}}^{\mathclap{\mathbb{E}[\textrm{Benefit from R\&D}]}}- (\underbrace{1-T_{RD}}_{\mathclap{\text{R\&D Subsidy}}}) \big( \overbrace{\hat{w}_{RD} - (1-x)(1-\kappa_e)\lambda \nu \tilde{V}}^{\mathclap{\text{R\&D wage}}}\big) \label{eq:hjb_incumbent_RDsubsidy} \nonumber \\ 
&-  \underbrace{(1-x) \nu \tilde{V}}_{\mathclap{\text{Net cost from spinout formation}}} - \overbrace{x \kappa_{c} \nu \tilde{V}}^{\mathclap{\text{Direct cost of NCA}}}\Big) \Big\} 
\end{align}

This can be rearranged to a form analogous to (\ref{eq:hjb_incumbent_workerIndiff}),
\begin{align}
(r + \hat{\tau}) \tilde{V} = \tilde{\pi} + \max_{\substack{x \in \{0,1\} \\ z \ge 0}} \Big\{z &\Big( \overbrace{\chi (\lambda - 1) \tilde{V}}^{\mathclap{\mathbb{E}[\textrm{Benefit from R\&D}]}}- (1-T_{RD}) \hat{w}_{RD} \\
&-  \underbrace{(1-x)(1 - (1-T_{RD})(1-\kappa_{e})\lambda)\nu \tilde{V}}_{\mathclap{\text{Net cost from spinout formation}}} - \overbrace{x \kappa_{c} \nu \tilde{V}}^{\mathclap{\text{Direct cost of NCA}}}\Big) \Big\} \label{eq:hjb_incumbent_RDsubsidy_2}
\end{align}

Define
\begin{align}
\tilde{\bar{\kappa}}_c(\kappa_e,\lambda;T_{RD}) = 1 - (1-T_{RD})(1-\kappa_e)\lambda
\end{align} 

If $z > 0$, the incumbent's optimal NCA policy is given by 
\begin{align}
x = \begin{cases}
1 & \textrm{if } \kappa_{c} < \tilde{\bar{\kappa}}_c (\kappa_e, \lambda;T_{RD}) \\
0 & \textrm{if } \kappa_{c} > \tilde{\bar{\kappa}}_c (\kappa_e, \lambda;T_{RD})\\
\{0,1\} & \textrm{if } \kappa_c = \tilde{\bar{\kappa}}_c (\kappa_e, \lambda;T_{RD})
\end{cases} \label{eq:nca_policy_RDsubsidy}
\end{align}

Since the argument is the same as in Section \ref{subsubsec:dynamic_equilibrium_original_solution}, I will not be as detailed in my proof. Assuming $z > 0$, by the same logic as before the incumbent's FOC can be rearranged to
\begin{align}
\tilde{V} &= \frac{(1-T_{RD})\hat{w}_{RD}}{\chi(\lambda -1) - \nu (x\kappa_c + (1-x)(1 - (1-T_{RD})(1-\kappa_e)\lambda)) } \label{eq:hjb_incumbent_foc_RDsubsidy}
\end{align}

The free entry condition is
\begin{align}
\underbrace{\hat{\chi} \hat{z}^{-\psi}}_{\mathclap{\text{Marginal innovation rate}}} \overbrace{(1-\kappa_e) \lambda \tilde{V}}^{\mathclap{\text{Payoff from innovation}}} &= \overbrace{(\underbrace{1-T_{RD}}_{\mathclap{\text{R\&D subsidy}}})\hat{w}_{RD}}^{\mathclap{\text{MC of R\&D}}} \label{eq:free_entry_condition_RDsubsidy}
\end{align}

Substituting (\ref{eq:hjb_incumbent_foc_RDsubsidy}) into (\ref{eq:free_entry_condition_RDsubsidy}) to eliminate $\tilde{V}$ yields an expression for $\hat{z}$, 
\begin{align}
\hat{z} &= \Bigg( \frac{\hat{\chi} (1-\kappa_{e}) \lambda}{\chi(\lambda -1) - \nu (x\kappa_c + (1-x)(1 - (1-T_{RD})(1-\kappa_e)\lambda)) } \Bigg)^{1/\psi} \label{eq:effort_entrant_RDsubsidy}
\end{align}

The rest of the equilibrium allocation and prices can be computed by using the following equations in sequence to compute the variable on the LHS:
\begin{align}
\hat{\tau} &= \hat{\chi} \hat{z}^{1-\psi} \\
z &= \bar{L}_{RD} - \hat{z} \label{eq:labor_resource_constraint_RDsubsidy}\\ 
\tau &= \chi z \\
\tau^S &= (1-x) \nu z \\
g &= (\lambda - 1) (\tau + \tau^S + \hat{\tau}) \\
r &= \theta g + \rho \\
\tilde{V} &= \frac{\tilde{\pi}}{r + \hat{\tau}} \\ 
\hat{w}_{RD} &= (1-T_{RD})^{-1}\hat{\chi} \hat{z}^{-\psi} (1-\kappa_e) \lambda \tilde{V} \label{eq:wage_rd_labor_RDsubsidy}
\end{align}

\paragraph{Effect on growth}

First suppose $x = 0$ and consider a small increase in $T_{RD}$ from $T_{RD}^0$ to $T_{RD}^1 > T_{RD}^0$. If $x = 0$ after the increase in $T_{RD}$, then by (\ref{eq:effort_entrant_RDsubsidy}), $\hat{z}$ increases; and by (\ref{eq:labor_resource_constraint_RDsubsidy}) $z$ decreases. If (\ref{cs:growth_decreasing_condition}) holds, this reduces growth. Intuitively, the increased R\&D subsidy reduces the wage expenses paid for R\&D by the same factor $1-\frac{1-T_{RD}^1}{1-T_{RD}^0}$ for incumbents and entrants. However, the incumbent's effective cost of R\&D also includes the shadow cost of more creative destruction by spinouts. Therefore, her effective cost of R\&D is reduced by a factor $\tilde{\tau}_{RD} < 1-\frac{1-T_{RD}^1}{1-T_{RD}^0}$. In equilibrium, R\&D labor is reallocated to entrants and growth falls.

If the increase in $T_{RD}$ is large enough, $x$ changes from $x = 0$ to $x = 1$ and therefore $\tau^S$ jumps to zero, reducing growth further. Intuitively, higher R\&D subsidies mean the incumbent prefers to pay for the R\&D with wages, which are tax-deductible, rather than implicitly through future spinouts, the implicit cost of which is not tax-deductible. Incumbents therefore opt to use NCAs, bringing spinout entry to zero and reducing growth by a discrete jump. In addition, there are no indirect effects on growth through changes in $\hat{z}$,$z$, as these variables do not jump: according to (\ref{eq:nca_policy_RDsubsidy}), the transition from $x= 0$ to $x =1$ occurs at the value of $T_{RD}$ such that $\kappa_c$ is equal to the term multiplying $(1-x)$, implying that $\hat{z}$, and therefore $z$, does not jump.

Finally, if $T_{RD}$ is increased even further, there is no change in the equilibrium allocation. The only change is the wage of R\&D labor, which by (\ref{eq:wage_rd_labor_RDsubsidy}) increases to equilibriate the R\&D labor market.

\paragraph{Effect on consumption}

As argued above, there can be two effects of an increase in $T_{RD}$: either it increases $\hat{\tau}$ at the expense of $\tau$, or it induces a switch to $x = 1$, or some combination of the two. In both cases, based on the discussion of Sections \ref{model:efficiency:misallocationRD} and \ref{model:efficiency:misallocationNCAs} imply that the sign of the effect on steady-state consumption depends on parameters.

\subsubsection{CD tax (subsidy)}

Suppose that the planner taxes entry at rate $T_e$ (subsidy if $T_e < 0$). Specifically, the planner taxes the entry fixed cost $\kappa_e \lambda \tilde{V} q$ at rate $T_e$ so that a firm entering with quality $\lambda q$ perceives a total cost of $(1+T_e) \kappa_e \lambda \tilde{V}q$ units of the final good. Economically, this can be interpreted as a tax on non-R\&D expenses related to the development of new versions of products currently not sold by the firm in question.\footnote{\textbf{[Put this footnote earlier in model exposition]} Because the tax is proportional to these expenses, rather than a fixed tax on entry, it does not induce any reallocation of R\&D towards higher quality goods. This property is not only analytically convenient -- it is necessary for a BGP to exist. In the baseline model, the expected growth rate of normalized frontier quality $\tilde{\bar{q}}_j = \frac{\bar{q}_j}{Q}$ is constant for all $j \in [0,1]$ and there is no exit of low quality firms (and subsequent injection of "average quality" firms). Running this stochastic process forward in time, the distribution of $\tilde{\bar{q}}_j$ spreads out, i.e. its variance and higher order measures of dispersion increase, which implies that there is no stationary distribution of $\tilde{\bar{q}}_j$. A BGP continues to exist, however, because only the mean of $\tilde{\bar{q}}_j$, $\mathbb{E}[\tilde{\bar{q}}_j] = 1$, is relevant for aggregate variables. This is why, e.g. the growth accounting equation (\ref{eq:growth_accounting}) can be written so simply. If, instead, growth is faster for higher $\tilde{\bar{q}}_j$, as is the case with a fixed entry fee, there is again no stationary distribution of $\tilde{\bar{q}}_j$, as before. However, in addition, there is no BGP, because aggregate variables such as the growth rate and the R\&D wage now depend on the entire distribution of $\tilde{\bar{q}}_j$, which is not stationary.}

In this setup, the R\&D labor supply indifference condition becomes
\begin{align}
\hat{w}_{RD} &= w_{RD,j} + (1-x_j) \nu (1-(1+T_e)\kappa_e) \lambda \tilde{V} \label{eq:RD_worker_indifference_entryTax}
\end{align}

The incumbent HJB is can be rearranged to
\begin{align}
(r + \hat{\tau}) \tilde{V} = \tilde{\pi} + \max_{\substack{x \in \{0,1\} \\ z \ge 0}} \Big\{z &\Big( \overbrace{\chi (\lambda - 1) \tilde{V}}^{\mathclap{\mathbb{E}[\textrm{Benefit from R\&D}]}}- \hat{w}_{RD} \\
&-  \underbrace{(1-x)(1 - (1-(1+T_e)\kappa_{e})\lambda)\nu \tilde{V}}_{\mathclap{\text{Net cost from spinout formation}}} - \overbrace{x \kappa_{c} \nu \tilde{V}}^{\mathclap{\text{Direct cost of NCA}}}\Big) \Big\} \label{eq:hjb_incumbent_entryTax_2}
\end{align}

Define
\begin{align}
\hat{\bar{\kappa}}_c(\kappa_e,\lambda;T_e) = 1 - (1-(1+T_e)\kappa_e)\lambda  \label{eq:barkappa_entryTax}
\end{align} 

If $z > 0$, the incumbent's optimal NCA policy is given by 
\begin{align}
x = \begin{cases}
1 & \textrm{if } \kappa_{c} < \hat{\bar{\kappa}}_c (\kappa_e, \lambda;T_{RD}) \\
0 & \textrm{if } \kappa_{c} > \hat{\bar{\kappa}}_c (\kappa_e, \lambda;T_{RD})\\
\{0,1\} & \textrm{if } \kappa_c = \hat{\bar{\kappa}}_c (\kappa_e, \lambda;T_{RD})
\end{cases} \label{eq:nca_policy_entryTax}
\end{align}

By the usual argument, $z > 0$ implies that the incumbent's FOC can be rearranged to
\begin{align}
\tilde{V} &= \frac{\hat{w}_{RD}}{\chi(\lambda -1) - \nu (x\kappa_c + (1-x)(1 - (1-(1+T_e)\kappa_e)\lambda)) } \label{eq:hjb_incumbent_foc_entryTax}
\end{align}

If $(1 + T_e) \kappa_e > 1$ then $\hat{z} = 0$ and $z = \bar{L}_{RD}$. Otherwise, the free entry condition is
\begin{align}
\underbrace{\hat{\chi} \hat{z}^{-\psi}}_{\mathclap{\text{Marginal innovation rate}}} \overbrace{(1-(1+T_e)\kappa_e) \lambda \tilde{V}}^{\mathclap{\text{Payoff from innovation}}} &= \underbrace{\hat{w}_{RD}}_{\mathclap{\text{MC of R\&D}}} \label{eq:free_entry_condition_entryTax}
\end{align}

Substituting (\ref{eq:hjb_incumbent_foc_entryTax}) into (\ref{eq:free_entry_condition_entryTax}) to eliminate $\tilde{V}$ yields an expression for $\hat{z}$, 
\begin{align}
\hat{z} &= \Bigg( \frac{\hat{\chi} (1-(1+T_e)\kappa_{e}) \lambda}{\chi(\lambda -1) - \nu (x\kappa_c + (1-x)(1 - (1-(1+T_e)\kappa_e)\lambda)) } \Bigg)^{1/\psi} \label{eq:effort_entrant_entryTax}
\end{align}

From here, the rest of the model (including the case where $(1-\kappa_e)\lambda < 1$ and $\hat{z} = 0$) can be solved in a similar way as before (details in Appendix \ref{appendix:model:efficiencyderivations:CDtax}). 

\paragraph{Effect on growth}

Suppose that $x = 1$ and the tax is increased from $T_e$ to $T_e' > T_e$. Then (\ref{eq:effort_entrant_entryTax}) implies that $\hat{z}$ falls, (\ref{eq:labor_resource_constraint_entryTax}) implies that $z$ increase to keep $L_{RD} = \bar{L}_{RD}$. Following the same logic as Section \ref{model:efficiency:misallocationRD}, if (\ref{cs:growth_decreasing_condition}) holds, then growth increases. Intuitively, when $x = 1$ the only effect of the entry tax is to reduce the misallocation of R\&D labor to entrants. 

However, if $x = 0$, the situation changes, for two reasons. First, as can be seen readily in (\ref{eq:effort_entrant_entryTax}), the effect of $T_e$ on $\hat{z}$ is ambiguous, since the denominator now decreases in $T_e$ as well as the numerator. Intuitively, an increase in $T_e$ reduces the value of future spinouts, requiring incumbents to compensate workers with higher wages in equilibrum. However, the expected harm to incumbents from WSOs per unit of $z$ is unchanged. This follows from the assumption that potential WSOs arise as a by-product of working in R\&D rather than as a result of intentional side projects by R\&D workers. The net effect is that incumbents' effective cost of R\&D increases and R\&D labor is reallocated to the entrant. The mechanism in the previous paragraph is still present, however; the logic here only serves to attenuate the increase in growth from an increase in $T_e$ when (\ref{cs:growth_decreasing_condition}) holds with $x = 0$.

Second, by (\ref{eq:nca_policy_entryTax}) and (\ref{eq:barkappa_entryTax}), a sufficiently large increase in $T_e$ induces a change from $x = 0$ to $x = 1$. Intuitively, as mentioned in the previous paragraph, a higher $T_e$ means it is relatively more expensive for incumbents to compensate their employees with future spinouts as they are less valuable but cause the same harm to the incumbent. For a high enough $T_e$, incumbents prefer to use NCAs and pay their employees with wages directly. Using the logic of Section \ref{model:efficiency:misallocationNCAs}, this switch implies a reduction in growth.

\paragraph{Effect on consumption}

A CD tax reallocates R\&D labor to OI and can also induce incumbents to use NCAs. The discussions in Sections (\ref{model:efficiency:misallocationRD}) and (\ref{model:efficiency:misallocationNCAs}) imply that the sign of the change in steady-state consumption depends on parameters.

\subsubsection{OI R\&D subsidy (tax)}

Suppose that the plannner can subsidize R\&D spent on improving a product while excluding R\&D aiming at creative destruction. In the model, this corresponds to a targeted subsidy to R\&D spending by incumbents, of magnitude $T_{RD,I}$ (tax if $T_{RD,I} < 0$). In practice, this policy may be difficult to implement for the same reason as the CD tax. Firms may not be expected to be truthful regarding the purpose of their R\&D or the effect of their R\&D on their competitors' profits. It may not even be possible to tell in advance whether R\&D will result in creative CD, OI, or even new varieties of products. Furthermore, innovation to improve existing products can be a form of creative destruction. Nevertheless, it is still useful as a theoretical benchmark.

In this case, the incumbent HJB can be rearranged to a form analogous to (\ref{eq:hjb_incumbent_workerIndiff}),
\begin{align}
(r + \hat{\tau}) \tilde{V} = \tilde{\pi} + \max_{\substack{x \in \{0,1\} \\ z \ge 0}} \Big\{z &\Big( \overbrace{\chi (\lambda - 1) \tilde{V}}^{\mathclap{\mathbb{E}[\textrm{Benefit from R\&D}]}}- (1-T_{RD,I}) \hat{w}_{RD} \\
&-  \underbrace{(1-x)(1 - (1-T_{RD,I})(1-\kappa_{e})\lambda)\nu \tilde{V}}_{\mathclap{\text{Net cost from spinout formation}}} - \overbrace{x \kappa_{c} \nu \tilde{V}}^{\mathclap{\text{Direct cost of NCA}}}\Big) \Big\} \label{eq:hjb_incumbent_RDsubsidyTargeted_2}
\end{align}

The non-compete policy is the same as with untargeted R\&D subsidies. That is, define
\begin{align}
\tilde{\bar{\kappa}}_c(\kappa_e,\lambda;T_{RD}) = 1 - (1-T_{RD,I})(1-\kappa_e)\lambda
\end{align} 

Then $z > 0$ implies that the incumbent's optimal NCA policy is given by 
\begin{align}
x = \begin{cases}
1 & \textrm{if } \kappa_{c} < \tilde{\bar{\kappa}}_c (\kappa_e, \lambda;T_{RD,I}) \\
0 & \textrm{if } \kappa_{c} > \tilde{\bar{\kappa}}_c (\kappa_e, \lambda;T_{RD,I})\\
\{0,1\} & \textrm{if } \kappa_c = \tilde{\bar{\kappa}}_c (\kappa_e, \lambda;T_{RD,I})
\end{cases} \label{eq:nca_policy_RDsubsidyTargeted}
\end{align}

Assuming $z > 0$, the FOC of the incumbent HJB implies
\begin{align}
\tilde{V} &= \frac{(1-T_{RD,I})\hat{w}_{RD}}{\chi(\lambda -1) - \nu (x\kappa_c + (1-x)(1 - (1-T_{RD,I})(1-\kappa_e)\lambda)) } \label{eq:hjb_incumbent_foc_RDsubsidyTargeted}
\end{align}

The free entry condition is
\begin{align}
\underbrace{\hat{\chi} \hat{z}^{-\psi}}_{\mathclap{\text{Marginal innovation rate}}} \overbrace{(1-\kappa_e) \lambda \tilde{V}}^{\mathclap{\text{Payoff from innovation}}} &= \underbrace{\hat{w}_{RD}}_{\mathclap{\text{MC of R\&D}}} \label{eq:free_entry_condition_RDsubsidyTargeted}
\end{align}

Substituting (\ref{eq:hjb_incumbent_foc_RDsubsidyTargeted}) into (\ref{eq:free_entry_condition_RDsubsidyTargeted}) to eliminate $\tilde{V}$ yields an expression for $\hat{z}$, 
\begin{align}
\hat{z} &= \Bigg( \frac{(1-T_{RD,I})\hat{\chi} (1-\kappa_{e}) \lambda}{\chi(\lambda -1) - \nu (x\kappa_c + (1-x)(1 - (1-T_{RD,I})(1-\kappa_e)\lambda)) } \Bigg)^{1/\psi} \label{eq:effort_entrant_RDsubsidyTargeted}
\end{align}

The rest of the equilibrium allocation and prices can be computed in a similar way as before (details in Appendix \ref{appendix:model:efficiencyderivations:OIRDtax}). 






\paragraph{Effect on growth}

If $x = 1$, increasing $T_{RD,I}$ reduces $\hat{z}$ by (\ref{eq:effort_entrant_RDsubsidyTargeted}) and will increase growth if the condition (\ref{cs:growth_decreasing_condition}) holds. Intuitively, a subsidy to incumbent R\&D causes the R\&D wage to increase, reducing R\&D by the entrant in equilibrium. 

If $x = 0$, increasing $T_{RD,I}$ has a more complicated effect on $\hat{z}$ because it reduces the denominator as well. This follows from the same reasoning as in the case of the untargeted R\&D subsidy: incumbents pay partially through future spinouts and so not all of their costs are subsidized at rate $T_{RD,I}$. From this economic interpretation, it follows immediately that the net effect is still to reduce incumbent R\&D expenses relative to those of the entrant and hence to lower $\hat{z}$ and increase $z$, and this is confirmed in the numerical analysis of the next subsection.

Finally, (\ref{eq:nca_policy_RDsubsidyTargeted}) implies that if the increase in $T_{RD,I}$ is sufficiently large, it will induce the use of NCAs by incumbents. As in the case of the untargeted R\&D subsidy, targeted R\&D subsidies do not reduce the harm to the incumbent's profits due to future employee spinouts. At a certain point, the incumbent prefers the higher but tax-deductible wages of an NCA contract. This switch unambiguously reduces growth.

The last observation implies that even targeted R\&D subsidies are unable to achieve the socially valuable outcome high spinout entry and high incumbent R\&D. In order to achieve this result, it is necessary to pair the targeted R\&D subsidy with an increase in legal barriers to NCAs $\kappa_c$ or an increase in the tax on NCA usage $T_{NCA}$. 

\paragraph{Effect on consumption}

As discussed above, targeted R\&D subusidy to OI reallocates R\&D labor to OI and can also induce incumbents to use NCAs. The discussions in Sections \ref{model:efficiency:misallocationRD} and \ref{model:efficiency:misallocationNCAs} imply that the sign of the change in steady-state consumption depends on parameters.

\subsubsection{All policies}

The BGP of the model when the planner can use all of the above policies is derived in Appendix \ref{append}. I discuss this case in the quantitative analysis of Section \ref{sec:policy_analysis}.


\section{Empirics}\label{sec:empirics}

In this section I describe the empirics which are used in the calibration and quantitative analysis of the following sections.

\subsection{Data}

\subsubsection{Sources}

\paragraph{VentureSource}

The data on startups comes from Venture Source (VS), a proprietary dataset containing information on venture capital (VC) firms and VC-funded startups.\footnote{When starting this project the data were owned by Dow Jones but they have since been sold to CB insights.} I use a subsample of the data for US-based startups founded between 1986 and 2008 which contain information on their founding year. The data cover 23,434 startups, 89,382 financing rounds, and 297,119 individual-firm pairs. For each financing round, the data contain information on valuation, amount raised, and status of the business at the time of the round -- employment, revenue, net income, burn rate -- albeit with substantial missing data. Most importantly for this analysis, the data contain employment biographies for each of the startup's founders and key employees (C-level, high-ranking executives and managers) and board members. In this regard, Venture Source is unique among VC investment databases. Some summary information about the dataset is contained in \autoref{table:VS_summaryTable}. The dataset is described in detail in \cite{kaplan_how_2002} and \cite{kaplan_venture_2016}. 

\paragraph{Compustat}

The data on R\&D spending comes from Compustat, a comprehensive database of fundamental financial and market information on publicly traded companies. I consider a subsample consisting of all firms headquartered in the United States in operation at any point between 1986 and 2006, consisting of 20,534 firms. In addition to data on R\&D spending, the Compustat data contain information on industrial classification and time-varying firm variables such as market value, tangible and intangible assets, employees, sales, etc.

\paragraph{NBER-USPTO}

The NBER-USPTO database contains comprehensive information on all patents granted in the United States from 1976 to 2006, and is linked to Compustat. I consider the subsample of patents assigned to US firms, consisting of 1,457,136 patents. 

\subsubsection{Construction of dataset}

\paragraph{Classifying founders}

The Venture Source data contain information on high level employees and board members. For the purposes of this study, however, not all of these employees should be considered founders of the startup in question. In particular, only those employees whose human capital is crucial to the value proposition of the startup should be considered founders. 

To this end, I first restrict attention to employees who join a startup in its first three years. When information on the individual's date of joining the startup is missing, I impute it as the founding date of the startup. I also conduct robustness exercises where I exclude these individual-startup observations. 

Next, I only consider employees whose job titles relate to the core operations of the firm. \autoref{table:VS_titlesSummaryTable} shows a breakdown of the 20 most frequent titles. Nearly 35\% of named employees are outside board members (e.g. VC investors). For the purpose of this study, I define a \textit{founder} as an employee with the title Founder, Chief, CEO, CTO or President.\footnote{I explore how the analysis changes with a different definition of founder in Section \textbf{XYZ}.}

% latex table generated in R 3.6.3 by xtable 1.8-4 package
% Tue Apr 28 17:18:36 2020
\begin{table}[!htb]
\centering
\begingroup\footnotesize
\begin{tabular}{rll}
  \toprule
Title & Individuals & Percentage \\ 
  \midrule
Board member (outsider) & 103367 & 34.6 \\ 
  Vice President & 59149 & 19.8 \\ 
  Chief Executive Officer & 24230 & 8.1 \\ 
  Chief Technology Officer & 13971 & 4.7 \\ 
  Chief Financial Officer & 11621 & 3.9 \\ 
  Director & 10988 & 3.7 \\ 
  Chief & 10846 & 3.6 \\ 
  President \& CEO & 9088 & 3.0 \\ 
  Senior Vice President & 8700 & 2.9 \\ 
  Founder & 8471 & 2.8 \\ 
  Chief Operating Officer & 6777 & 2.3 \\ 
  President & 5441 & 1.8 \\ 
  Chairman & 5029 & 1.7 \\ 
  Executive Vice President & 4920 & 1.6 \\ 
  Chairman \& CEO & 2755 & 0.9 \\ 
  Manager & 2357 & 0.8 \\ 
  Chief Scientific Officer & 1461 & 0.5 \\ 
  Controller & 1137 & 0.4 \\ 
  President \& COO & 1134 & 0.4 \\ 
  General Counsel & 1056 & 0.4 \\ 
   \bottomrule
\end{tabular}
\endgroup
\caption{Top 20 most frequent titles among founders in VS data.} 
\label{table:VS_titlesSummaryTable}
\end{table}


\paragraph{Extracting information on the most recent employer}

In order to relate the activities of employers to the entrepreneurship behavior of their employees, I link the Compustat data to the VS data using information in employee biographies. Because VS biographies are text fields, this requires matching entries by name to firm names in Compustat.  

The VS biographical data comes in a structured format, allowing parsing by regular expressions. Each prior job is represented in the format ``<position>, <employer>'' and different jobs are separated by ``;''. Job spells can be easily separated by splitting the string on the character ``;''. It is slightly more involved to separate positions from employer. It is not sufficient to simply separate on the right-most character ``,'' as <employer> can contain ``,''. However, in almost all cases, <employer> contains at most one ``,'' (e.g., in ``Microsoft, Inc.''), and in virtually all of these cases, the comma precedes one of a few strings (e.g. ``LLC'',``Inc'',``Corp''). Hence, I use a two-pass approach: first I split on the last ``,''; for employers that end up consisting only of corporate structure (e.g., ``LLC'', etc.), I split on the penultimate ``,'' instead. 

The above procedure yields a dataset containing, for each individual, all of his or her previous positions and employers. However, because an individual can take various jobs over the years at an individual startup, there are individuals whose most recent employer coincides with their startup. I exclude these cases by comparing the previous employer with the \texttt{EntityName} text field. Because these are both text fields with potentially different formatting, this entails two steps. First, I bring both fields to a common format, eliminating endings such as ``Inc.'', ``Corp.'' etc which may vary across them, and converting to lower case. I then exclude observations where the strings either exactly coincide or one contains the other. 

The results of this procedure are summarized in \autoref{table:VS_previousEmployersSummaryTable}. The top previous employers include several well-known technology firms such as Microsoft, IBM, Google, and Oracle. However, notice that many of the top previous employers are VC firms (e.g., New Enterprise Associates, Sequoia Capital, Kleiner Perkins), and the top employer is the unaffiliated category "Individual Investor." This is largely because many of the individuals affiliated with startups are investors turned board members: about 33\% of the individual-startup observations are outside board members, As my focus is on the flow of knowledge from previous employers to new startups, I will later restrict attention to individuals with knowledge-related and/or executive titles. Finally, notice that Stanford University is also listed in the top 20. Several other prominent research universities are also in the top 50. 

% latex table generated in R 3.6.3 by xtable 1.8-4 package
% Tue Jul  7 15:46:01 2020
\begin{table}[!htb]
\centering
\begingroup\footnotesize
\begin{tabular}{rlrll}
  \toprule
Employer & Count & Position & Count & Percentage \\ 
  \midrule
Individual Investor & 1115 & Board Member, Institutional Investor & 38527 & 14.2 \\ 
  Microsoft & 938 & Executive & 19617 & 7.2 \\ 
  Google & 780 & CEO & 9681 & 3.6 \\ 
  IBM & 773 & President \& CEO & 6871 & 2.5 \\ 
  Cisco Systems & 690 & CFO & 6567 & 2.4 \\ 
  Oracle & 602 & President & 5741 & 2.1 \\ 
  New Enterprise Associates & 565 & Founder & 4964 & 1.8 \\ 
  AT\&T & 466 & Cofounder & 3884 & 1.4 \\ 
  Verizon & 443 & Partner & 3757 & 1.4 \\ 
  Sun Microsystems & 402 & CTO & 3491 & 1.3 \\ 
  Intel & 389 & Managing Director & 3381 & 1.2 \\ 
  Hewlett-Packard & 386 & VP & 3365 & 1.2 \\ 
  Sequoia Capital & 384 & COO & 2445 & 0.9 \\ 
  Venrock & 342 & Board Member, Outsider & 2260 & 0.8 \\ 
  Kleiner Perkins Caufield \& Byers & 334 & Director & 2221 & 0.8 \\ 
  Mayfield Fund & 310 & Chairman \& CEO & 2147 & 0.8 \\ 
  McKinsey \& & 307 & Founder \& CEO & 2091 & 0.8 \\ 
  Bessemer Venture Partners & 305 & Chairman & 2086 & 0.8 \\ 
  Stanford University & 298 & Principal & 1836 & 0.7 \\ 
  Ernst \& Young & 275 & VP, Marketing & 1785 & 0.7 \\ 
   \bottomrule
\end{tabular}
\endgroup
\caption{Top 20 previous employers and previous positions for all founders in VS data.} 
\label{table:VS_previousEmployersSummaryTable}
\end{table}


% latex table generated in R 3.6.3 by xtable 1.8-4 package
% Sat Sep 26 15:56:53 2020
\begin{table}[]
\centering
\begingroup\normalsize
\begin{tabular}{rlrll}
  \toprule
Employer & Count & Position & Count & Percentage \\ 
  \midrule
IBM & 168 & Executive & 3503 & 9.5 \\ 
  Microsoft & 156 & CEO & 2479 & 6.7 \\ 
  Cisco Systems & 113 & President \& CEO & 2274 & 6.1 \\ 
  Oracle & 98 & President & 1389 & 3.8 \\ 
  Sun Microsystems & 92 & CTO & 1375 & 3.7 \\ 
  Verizon & 82 & Founder & 1236 & 3.3 \\ 
  AT\&T & 78 & Cofounder & 706 & 1.9 \\ 
  Google & 67 & Board Member, Institutional Investor & 565 & 1.5 \\ 
  Intel & 66 & VP & 512 & 1.4 \\ 
  Hewlett-Packard & 62 & Chairman \& CEO & 495 & 1.3 \\ 
  Stanford University & 51 & COO & 448 & 1.2 \\ 
  Lucent Technologies & 46 & Founder \& CEO & 411 & 1.1 \\ 
  Motorola & 42 & Chief Executive Officer & 348 & 0.9 \\ 
  Andersen Consulting & 41 & President \& COO & 297 & 0.8 \\ 
  Nortel Networks & 40 & Partner & 265 & 0.7 \\ 
  MIT & 39 & SVP & 261 & 0.7 \\ 
  Texas Instruments & 38 & Director & 239 & 0.6 \\ 
  McKinsey \& Company & 37 & Chairman & 236 & 0.6 \\ 
  Individual Investor & 35 & VP, Engineering & 234 & 0.6 \\ 
  Apple Computer & 35 & EVP & 230 & 0.6 \\ 
   \bottomrule
\end{tabular}
\endgroup
\caption{Top 20 previous employers and previous positions for founder2 founders in VS data.} 
\label{table:VS_previousEmployersSummaryTable}
\end{table}


\paragraph{Linking to Compustat}

The data on prior employers is matched to the variable \texttt{conml} in Compustat. To do this, first I standardize names as before, using regular expressions to trim e.g. ``Inc.", ``Corp.'' and variants thereof from each entry and converting to lower case. I look for exact matches to previous employers in the VS data. For previous employers in VS that do not match with any names in Compustat, I check against the business segment names, available from the Compustat Segments database. 

\paragraph{Defining WSOs}

This study emphasizes the importance of competition between spinouts and their parent firms. The best measure I have for the product market of publicly traded firms is their self reported NAICS code. While VS does not contain NAICS classifications for its startups, it does document their industry using a classification that, for the most part, coincides with NAICS 4 or 5 digit categories. I manually construct a crosswalk between the two classification schemes and use this to assign 4-digit NAICS codes to startups in VS.\footnote{An alternative would be to us VS's "Competition" variable, which documents directly the competitors of the startup observation. However, only 20\% of startups have this variable filled in: 30\% in the 90s, but dropping to around 10\% by the end of the sample.} Then, I classify a founder-startup observation as a WSO whenever the startup is in the same 4-digit NAICS category as its parent. 

\paragraph{Evaluating the match}

\autoref{table:GStable_founder2} documents the quality of this match. It corresponds roughly to Table 1 of \cite{gompers_entrepreneurial_2005}.\footnote{The numbers are different. I find a similar number of founders from public companies, but a substantially smaller fraction. I suspect this is due to startups being added to the data retroactively since the time of that article \textbf{[ask VS]}.} About 20\% of founders have a most recent previous employer that matches to a public firm in Compustat. In turn, about 35\% of these previous employers are in the same 4-digit NAICS industry as the startup.

\autoref{figure:industry_row_heatmap_naics2_founder2} and \autoref{figure:industry_column_heatmap_naics2_founder2} document the joint distribution of parent industry and child industry, defined by 2-digit NAICS codes. The raw joint distribution is too heavily concentrated to be easily visualized in this way, so instead I show the distribution of child industry (parent industry) conditional on parent industry (child industry), displayed in \autoref{figure:industry_row_heatmap_naics2_founder2} (\autoref{figure:industry_column_heatmap_naics2_founder2}). The dark diagonal lines in both figures reflects the prevalence of WSOs. In \autoref{figure:industry_row_heatmap_naics2_founder2}, the dark vertical line at column 51 (Information) indicates that parent firms of all industries tend to spawn spinouts in that industry. Similar dark regions appear at columns 54 (Professional, Scientific and Technical Services), and 32 and 33 (Manufacturing). In \autoref{figure:industry_column_heatmap_naics2_founder2}, the dark horizontal lines at 51 and to a lesser extend 32, 33, 52 and 54 indicate that child firms of all industries tend to have founders from those industries.

\subsection{Corporate R\&D and spinout formation}\label{subsec:empirics:corpRDandspinouts}

In this section, I consider the determinants of spinout formation. This is relevant to the general equilibrium consequences of spinouts. If employee spinout formation -- and, in particular, WSO4 formation -- is a consequence of parent firm decisions, then it could affect the parent firm decision making process, altering the general equilibrium consequences of facilitating WSO4 spinout formation by, e.g., prohibiting non-compete agreements. 

In particular, I focus on whether R\&D expenditures tend to produce employee spinouts. As discussed in the introduction, this is theoretically plausible because (1) employees undertaking innnovation must be trained, exposing and them to the firm's existing knowledge stock, and (2) such employees also may develop new ideas which may not be implementable within the firm.  

The purpose of this section is to provide discipline on the parametrization of the model used in the quantitative analysis. That model hypothesizes that R\&D by parent firms leads to a flow of employees into starting new spinout firms, and assumes that parent firms internalize this \textit{causal} relationship when making R\&D decisions. Therefore, the validity of my quantitative experiments depends crucially on whether the relationship established in this section is in fact causal. 

This presents a challenge, as many variables can be thought to simultaneously affect both corporate R\&D and employee entrepreneurship. For example, an innovative incumbent may have high R\&D expenses and also hire the most innovative employees who then start WSOs which they would have started regardless of being hired to do R\&D at the incumbent. Furthermore, there may be time-varying shocks to innovative investment opportunities, either at the aggregate level, or at the industry, state, or even industry-state level. 

I address this challenge in the standard way by using a regression analysis. I control for observable confounders and using multiple fixed effects to absorb contaminating variation from shocks which are not directly observable. Based on these regressions, I obtain suggestive evidence of a statistically and economically significant causal effect of corporate R\&D on WSO formation.

\subsubsection{Preliminaries}

\autoref{figure:scatterPlot_RD-Founders} shows a scatterplot illustrating the relationship between R\&D spending in years $t-2,t-1,t$ and employee entrepreneurship in years $t+1,t+2,t+3$. The dashed line shows the fit of a straight line through all of the points. The solid line shows the fit of a line only through firm-year observations with nonzero number of employee founders. The graph shows a positive relationship. \autoref{figure:scatterPlot_RD-Founders_dIntersection} shows the relationship between deviations from firm and State-industry-age-year means. The positive relationship remains. Finally, \autoref{figure:scatterPlot_RD-FoundersWSO4_dIntersection} shows that the same positive relationship holds when considering only WSO4 spinouts. 


\subsubsection{Regressions}

\autoref{table:RDandSpinoutFormation_absolute_founder2_l3f3} displays the results of a regression analysis relating employee entrepreneurship to parent firm R\&D spending. The dependent variable $Y_{it}$ is again the (annualized) number of founders previously employed at firm $i$ joining startups in years $t+1,t+2,t+3$. The independent variables $X_{it}$ are moving averages over years $t,t-1,t-2$. 

The first three columns consider the effect of R\&D on the number of founders leaving the parent firm. These regressions find a positive coefficient which is statistically significant at the 1\% level, even after including firm and year fixed effects. The magnitude of the coefficient indicates that in 2014, three billion dollars of R\&D over three years leads to on average 1.5 founders leaving to found new firms in the next three years.\footnote{The reason for the dependence on the year 2014 is that the specification assumes that the amount of R\&D that leads to a founder leaving grows at the rate of aggregate productivity growth.} 

The robustness of the result to the inclusion of age, industry-year, and State-year fixed effects is encouraging. However, in this context such fixed effects may not absorb much contaminating variation due to what amounts to a misspecification problem: shocks are likely to affect firm outcomes more in absolute terms for larger firms. Because firms vary in size, the fixed effect -- which must be constant in absolute terms for all firms -- leaves much firm-level variation unabsorbed. This is the same reason for the inclusion of Tobin's Q $\times$ Assets, rather than Tobin's Q. 

To address this, \autoref{table:RDandSpinoutFormation_at_founder2_l3f3} displays the results of a regression analysis where all variables are normalized by a trailing 5-year moving average of firm assets. Normalizing in this way improves the specification of the fixed effects, although it is not a panacea as it leaves unaddressed the problem of variation in firm R\&D / asset ratios. The magnitudes of the estimates of the coefficient on the measure of R\&D are strikingly similar. This is particularly true once the full battery of fixed effects is included. Normalizing by assets throws away any variation in absolute firm levels of R\&D, reducing power substantially. In spite of this, the most stringent estimate is significant at the 10\% level.

Finally \autoref{table:RDandSpinoutFormation_ppml_absolute_founder2_l3f3} shows the results of a Poisson pseudo-Maximum Likelihood estimation. This can be thought of as a log-linear regression which is more robust to zeros in the dependent variable.\footnote{Specifically, for any group (as defined by fixed effects), a group's observations are dropped iff the dependent variable equals zero for all observations in that group. In log-linear regression, all zero observations are dropped.} Robustness analysis for each of these specifications (i.e. varying controls, fixed effects, and clustering) is contained in the appendix, specifically robustness of the OLS regression with levels is analyzed in \ref{figure:speccheck2_levels_reghdfe} and  \ref{figure:speccheck2_levels_wso4_reghdfe}; of the asset-normalized OLS regressions in figures \ref{figure:speccheck2_at_reghdfe} and \ref{figure:speccheck2_at_wso4_reghdfe}; and of the PPML regressions in figures \ref{figure:speccheck2_levels_ppmlhdfe} and \ref{figure:speccheck2_levels_wso4_ppmlhdfe}.


\begin{table}[!htb]
	\scriptsize
	\centering
	{
\def\sym#1{\ifmmode^{#1}\else\(^{#1}\)\fi}
\begin{tabular}{l*{8}{c}}
\toprule
                    &\multicolumn{1}{c}{(1)}&\multicolumn{1}{c}{(2)}&\multicolumn{1}{c}{(3)}&\multicolumn{1}{c}{(4)}&\multicolumn{1}{c}{(5)}&\multicolumn{1}{c}{(6)}&\multicolumn{1}{c}{(7)}&\multicolumn{1}{c}{(8)}\\
                    &\multicolumn{1}{c}{Founders}&\multicolumn{1}{c}{Founders}&\multicolumn{1}{c}{Founders}&\multicolumn{1}{c}{Founders}&\multicolumn{1}{c}{WSO4}&\multicolumn{1}{c}{WSO4}&\multicolumn{1}{c}{WSO4}&\multicolumn{1}{c}{WSO4}\\
\midrule
R\&D                &        0.30\sym{**} &        0.62\sym{***}&        0.62\sym{***}&        0.34\sym{***}&        0.17\sym{***}&        0.28\sym{***}&        0.27\sym{***}&        0.22\sym{***}\\
                    &      (0.12)         &      (0.22)         &      (0.21)         &     (0.046)         &     (0.041)         &     (0.063)         &     (0.061)         &     (0.055)         \\
\addlinespace
NAICS4-State-Age-Year FE&          No         &          No         &          No         &         Yes         &          No         &          No         &          No         &         Yes         \\
\addlinespace
NAICS4-Year FE      &          No         &          No         &         Yes         &          No         &          No         &          No         &         Yes         &          No         \\
\addlinespace
State-Year FE       &          No         &          No         &         Yes         &          No         &          No         &          No         &         Yes         &          No         \\
\addlinespace
Firm FE             &          No         &         Yes         &         Yes         &         Yes         &          No         &         Yes         &         Yes         &         Yes         \\
\addlinespace
Age FE              &          No         &          No         &         Yes         &          No         &          No         &          No         &         Yes         &          No         \\
\addlinespace
Year FE             &          No         &         Yes         &          No         &          No         &          No         &         Yes         &          No         &          No         \\
\addlinespace
No FE               &         Yes         &          No         &          No         &          No         &         Yes         &          No         &          No         &          No         \\
\midrule
r2\_a                &        0.22         &        0.67         &        0.67         &        0.62         &        0.20         &        0.64         &        0.63         &        0.60         \\
r2\_a\_within         &        0.22         &        0.22         &        0.22         &        0.20         &        0.20         &        0.23         &        0.22         &        0.18         \\
N                   &       65009         &       63732         &       62488         &       25361         &       65009         &       63732         &       62488         &       25361         \\
\bottomrule
\multicolumn{9}{l}{\footnotesize Standard errors in parentheses}\\
\multicolumn{9}{l}{\footnotesize \sym{*} \(p<0.1\), \sym{**} \(p<0.05\), \sym{***} \(p<0.01\)}\\
\end{tabular}
}

	\caption{The regressions above relate corporate R\&D to the entrepreneurship decisions of employees. The dependent variable is average yearly number of founders joining startups in years $t+1,t+2,t+3$. The independent variables are averages over $t,t-1,t-2$. Firm controls are employment, assets, intangible assets, investment, net income, cumulative citation-weighted patents, and the product of Tobin's Q and Assets (i.e., firm market value). Standard errors are clustered by firm in columns (1)-(3) and (5)-(7). In columns (4) and (8), standard errors are multway clustered by State and 4-digit NAICS industry. Note that the regression does not include an interaction with an indicator for NAICS industries starting in the digit 9, since I focus on the private sector.}
	\label{table:RDandSpinoutFormation_absolute_founder2_l3f3}
\end{table}

\begin{table}[!htb]
	\scriptsize
	\centering
	{
\def\sym#1{\ifmmode^{#1}\else\(^{#1}\)\fi}
\begin{tabular}{l*{8}{c}}
\toprule
                    &\multicolumn{1}{c}{(1)}&\multicolumn{1}{c}{(2)}&\multicolumn{1}{c}{(3)}&\multicolumn{1}{c}{(4)}&\multicolumn{1}{c}{(5)}&\multicolumn{1}{c}{(6)}&\multicolumn{1}{c}{(7)}&\multicolumn{1}{c}{(8)}\\
                    &\multicolumn{1}{c}{$\frac{\textrm{Founders}}{\textrm{Assets}}$}&\multicolumn{1}{c}{$\frac{\textrm{Founders}}{\textrm{Assets}}$}&\multicolumn{1}{c}{$\frac{\textrm{Founders}}{\textrm{Assets}}$}&\multicolumn{1}{c}{$\frac{\textrm{Founders}}{\textrm{Assets}}$}&\multicolumn{1}{c}{$\frac{\textrm{WSO4}}{\textrm{Assets}}$}&\multicolumn{1}{c}{$\frac{\textrm{WSO4}}{\textrm{Assets}}$}&\multicolumn{1}{c}{$\frac{\textrm{WSO4}}{\textrm{Assets}}$}&\multicolumn{1}{c}{$\frac{\textrm{WSO4}}{\textrm{Assets}}$}\\
\midrule
$\frac{\textrm{R\&D}}{\textrm{Assets}}$&        1.87\sym{***}&        1.08\sym{+}  &        1.16\sym{+}  &        1.80\sym{*}  &        0.94\sym{***}&        0.52\sym{+}  &        0.49\sym{++} &        1.96\sym{**} \\
                    &      (0.35)         &      (0.71)         &      (0.74)         &      (1.00)         &      (0.17)         &      (0.32)         &      (0.36)         &      (0.94)         \\
\addlinespace
NAICS4-State-Age-Year FE&          No         &          No         &          No         &         Yes         &          No         &          No         &          No         &         Yes         \\
\addlinespace
NAICS4-Year FE      &          No         &          No         &         Yes         &          No         &          No         &          No         &         Yes         &          No         \\
\addlinespace
State-Year FE       &          No         &          No         &         Yes         &          No         &          No         &          No         &         Yes         &          No         \\
\addlinespace
Firm FE             &          No         &         Yes         &         Yes         &         Yes         &          No         &         Yes         &         Yes         &         Yes         \\
\addlinespace
Age FE              &          No         &          No         &         Yes         &          No         &          No         &          No         &         Yes         &          No         \\
\addlinespace
Year FE             &          No         &         Yes         &          No         &          No         &          No         &         Yes         &          No         &          No         \\
\addlinespace
No FE               &         Yes         &          No         &          No         &          No         &         Yes         &          No         &          No         &          No         \\
\midrule
r2\_a                &       0.014         &        0.26         &        0.21         &        0.55         &       0.010         &        0.27         &        0.22         &        0.58         \\
r2\_a\_within         &       0.014         &      0.0021         &      0.0019         &      0.0031         &       0.010         &     0.00088         &     0.00084         &      0.0088         \\
N                   &       60687         &       59477         &       57948         &       10842         &       60687         &       59477         &       57948         &       10842         \\
\bottomrule
\multicolumn{9}{l}{\footnotesize Standard errors in parentheses}\\
\multicolumn{9}{l}{\footnotesize \sym{++} \(p<0.2\), \sym{+} \(p<0.15\), \sym{*} \(p<0.1\), \sym{**} \(p<0.05\), \sym{***} \(p<0.01\)}\\
\end{tabular}
}

	\caption{The regressions above relate corporate R\&D to the entrepreneurship decisions of employees. The dependent variable is the average yearly number of founders from the parent firm joining startups in years $t+1,t+2,t+3$, normalized by a trailing five-year moving average of assets. Independent variables are also normalized by assets. Standard errors are clustered at the firm level.}
	\label{table:RDandSpinoutFormation_at_founder2_l3f3}
\end{table}

\begin{table}[!htb]
	\scriptsize
	\centering
	{
\def\sym#1{\ifmmode^{#1}\else\(^{#1}\)\fi}
\begin{tabular}{l*{8}{c}}
\toprule
                    &\multicolumn{1}{c}{(1)}&\multicolumn{1}{c}{(2)}&\multicolumn{1}{c}{(3)}&\multicolumn{1}{c}{(4)}&\multicolumn{1}{c}{(5)}&\multicolumn{1}{c}{(6)}&\multicolumn{1}{c}{(7)}&\multicolumn{1}{c}{(8)}\\
                    &\multicolumn{1}{c}{Founders}&\multicolumn{1}{c}{Founders}&\multicolumn{1}{c}{Founders}&\multicolumn{1}{c}{Founders}&\multicolumn{1}{c}{WSO4}&\multicolumn{1}{c}{WSO4}&\multicolumn{1}{c}{WSO4}&\multicolumn{1}{c}{WSO4}\\
\midrule
log(R\&D)           &        0.81\sym{***}&        0.49\sym{***}&        0.50\sym{**} &        0.50\sym{***}&        1.54\sym{***}&        0.55\sym{***}&        1.28\sym{***}&        1.28\sym{***}\\
                    &     (0.098)         &      (0.15)         &      (0.22)         &      (0.13)         &      (0.13)         &      (0.20)         &      (0.44)         &      (0.49)         \\
\addlinespace
No FE               &         Yes         &          No         &          No         &          No         &         Yes         &          No         &          No         &          No         \\
\addlinespace
Firm FE             &          No         &         Yes         &         Yes         &         Yes         &          No         &         Yes         &         Yes         &         Yes         \\
\addlinespace
Year FE             &          No         &         Yes         &          No         &          No         &          No         &         Yes         &          No         &          No         \\
\addlinespace
Age FE              &          No         &          No         &         Yes         &         Yes         &          No         &          No         &         Yes         &         Yes         \\
\addlinespace
Industry-Year FE    &          No         &          No         &         Yes         &         Yes         &          No         &          No         &         Yes         &         Yes         \\
\addlinespace
State-Year FE       &          No         &          No         &         Yes         &         Yes         &          No         &          No         &         Yes         &         Yes         \\
\midrule
Clustering          &       gvkey         &       gvkey         &       gvkey         &naics4 Statecode         &       gvkey         &       gvkey         &       gvkey         &naics4 Statecode         \\
pseudo R-squared    &        0.44         &        0.49         &        0.54         &        0.54         &        0.45         &        0.38         &        0.38         &        0.38         \\
Observations        &        7434         &        2416         &        1335         &        1335         &        7434         &         898         &         436         &         436         \\
\bottomrule
\multicolumn{9}{l}{\footnotesize Standard errors in parentheses}\\
\multicolumn{9}{l}{\footnotesize \sym{*} \(p<0.1\), \sym{**} \(p<0.05\), \sym{***} \(p<0.01\)}\\
\end{tabular}
}

	\caption{Poisson pseudo-Maximum Likelihood Regression. The dependent variable is average yearly number of founders joining startups in years $t+1,t+2,t+3$. The independent variables are in log terms and averages over $t,t-1,t-2$ Firm controls are employment, assets, intangible assets, investment, net income, cumulative citation-weighted patents, and the product of Tobin's Q and Assets (i.e., firm market value). Standard errors are clustered by firm in columns (1)-(3) and (5)-(7). In columns (4) and (8), standard errors are multi-way clustered by State and 4-digit NAICS industry.}
	\label{table:RDandSpinoutFormation_ppml_absolute_founder2_l3f3}
\end{table}

\subsubsection{Economic magnitude}

In each year $t$, I compute $\tilde{y}_{it}$, the expected number of founders per year starting firms in years $t+1,t+2,t+2$ by multiplying the R\&D in years $t,t-1,t-2$ by the relevant coefficient estimate. I then plot this against the realizations of $y_{it}$. \textbf{[I need to make explicit under what assumptions this kind of aggregation is valid. If taking my model seriously it is valid since R\&D linearly leads to spinouts at the firm level.]} \autoref{figure:founder2_founders_f3_Accounting} provides a visualization of the economic magnitude of the coefficient estimates. The left column is for all founders and the right column is for founders of firms in the same 4-digit NAICS industry as the parent. 

The regression estimates are economically significant. About half of all WSO4 spinouts are accounted for by parent firm corporate R\&D. The share explained is closer to 1/4 when considering all employee entrepreneurship. This is consistent with the emphasis on WSO4 spinouts in the model. 

\begin{figure}[!htb]
	\includegraphics[scale=0.5]{../empirics/figures/founder2_founders_f3_Accounting.pdf}
	\caption{Economic magnitude of regression estimates in Tables \ref{table:RDandSpinoutFormation_absolute_founder2_l3f3} and \ref{table:RDandSpinoutFormation_at_founder2_l3f3}. The first row of figures compares the predicted number of employee founders (dotted lines) to the observed number of employee founders (solid lines). The left figure considers all founders, the right figure only founders of firms in the same 4-digit NAICS industry as their previous employers. The bottom row shows the percentage explained in each year.}
	\label{figure:founder2_founders_f3_Accounting}
\end{figure}

\section{Calibration}\label{sec:calibration}

\subsection{Parameters}

The model has ten parameters given by $\{\rho, \theta, \beta, \psi, \lambda, \chi, \hat{\chi}, \kappa_e, \kappa_c, \nu, \bar{L}_{RD}\}$. The parameter $\bar{L}_{RD}$ is only a normalization determining the units in which R\&D human capital is measured. The remaining nine parameters are substantive. The elasticity parameters $\{\theta, \psi\}$ are chosen to match estimates from the literature. The remaining seven parameters $\{\rho, \lambda, \chi, \hat{\chi}, \kappa_e, \kappa_c, \nu\}$ pertain to preferences ($\rho$) and to the technology for innovation and NCA usage (all others), and are chosen to match six moments from the data. One parameter, $\kappa_c$, is partially identified as $\kappa_C > \bar{\kappa}_c$ by the observation that $\tau^S > 0$. The remaining six parameters are exactly identified and the model reproduces the target moments exactly. 

\subsection{Targets}

The targets of the calibration are displayed in \autoref{calibration_targets} and consist of the labor productivity growth rate, the R\&D / GDP ratio, the share of growth coming from OI, the real interest rate, the employment share of entering firms, and the employment share of R\&D-induced WSOs. Matching the productivity growth rate, R\&D / GDP ratio and growth share of OI helps calibrate the efficiency of R\&D in generating aggregate productivity growth through OI and CD. The interest rate, profit / GDP ratio and employment share of entering firms determines the discount factor and the reward to innovation, which is in turn determined by the flow payoff to innovation (profit / GDP) and the expected duration of an incumbency position (employment share of entering firms). Finally, matching the employment share of entering WSOs allows the model to capture the rate at which R\&D by incumbents increases their likelihood of being replaced by a WSO. This calibrates the magnitude of the disincentive to OI R\&D.

Below, I discuss issues pertaining to the measurement of the target moments. In particular, in this section I discuss how the results in Section \ref{subsec:empirics:corpRDandspinouts} are used to calibrate the model.

\paragraph{Growth rate}

The growth rate is calibrated to the growth in labor productivity due to CD and OI, as calculated in \cite{klenow_innovative_2020}. [\textbf{Insert description of how they identify}]

\paragraph{R\&D spending / GDP}

The data on R\&D spending is from the National Patterns of R\&D resources.\footnote{I take the average of business-funded R\&D business-performed R\&D.} In the data, about half of R\&D spending is wages for employees; in the model, the only input to R\&D is labor. I opt to match the model's aggregate R\&D intensity to that in the data, including costs other than labor. This means that the model captures the full cost of innovation. The computation of the corresponding model moment is described in \ref{appendix:calibration:rd/gdp}.

\paragraph{Growth share of own innovation}

The growth share of OI is calibrated to the growth share of OI as a fraction of OI and CD innovations, as estimated in \cite{klenow_innovative_2020} using a model similar to this one. They find that, from 1982 to 2013, roughly 70\% of CD + OI productivity growth was due to OI. The computation of the corresponding model moment is described in \ref{appendix:calibration:growthShareOI}.


\paragraph{Interest rate}

The short-term risk-free real interest rate averages about 5\% in the United States from 1986-2006. However, the real interest rate in the model actually corresponds to the discount factor used to price an unlevered firm. Since there is no systemic risk in the model, these are the same; however, since the data exhibits systemic risk, unlevered firms require a higher return than 5\% in the data. 

To adjust for this, I use a back of the envelope calculation to calculate the asset beta from the equity betas and leverage ratios, and hence compute the hypothetical risk-premium on an unlevered investment. First, the real return on the S\&P 500 in the time period 1986-2006 averaged about 7\%. The average debt-value ratio of the S\&P 500 in the US is about 40\% during this period. Assuming that this corporate debt does not earn a risk premium, the entire risk premium accrues to the equity. If there were no leverage, the risk premium would be smaller in percentage terms, since it is accruing to a larger value investment. Quantitatively, we need to multiply the excess return by $E / (D + E)$, which in this case is $1 - 40\% = 60\%$. I arrive at a calibration value of about 6\% for the real interest rate in the model.

\paragraph{Profits \% GDP} 

The data on aggregate profits as a percent of GDP comes from the BEA (computed as an average during the sample period of 1986-2008). In the model, this ratio is simple to calculate using the solution to the static equilibrium as $\tilde{\pi} / \tilde{Y}$.



\paragraph{Entry rate}

The entry rate target deserves some discussion. The purpose of including entry in the model is to capture the rate at which incumbent profits are destroyed due to creative destruction. As discussed in \cite{klenow_innovative_2020}, adjustment costs mean that, in the data, it can take several years for a new product to displace an old one. However, in the model, entrants that replace incumbents reach their mature size immediately upon entry. If the model matches the amount of employment in firms of age < 1, it might underestimate the true impact on employment reallocation of each new cohort of firms.\footnote{In the data, because firms grow to achieve their mature size over the first five years (and beyond), so that the employment of an entering cohort of firms does not decrease over time (i.e., including firm exit) very rapidly in the data. If the data were in continuous time, the employment of the cohort would increase at first, then decrease. In the model, firms enter at their mature size, so the employment of a cohort decreases over time.} Given this, I match the employment share of firms age <= 6 engaging in creative destruction, which is approximately 8.35\% during the sample period. The computation of the corresponding model moment is described in \ref{appendix:calibration:entryRate}.


 
\paragraph{R\&D-induced spinout share of employment}

Finally, matching the employment share of spinouts is of course crucial, in order for the model to properly estimate the burden such firms impose on the incumbents that spawn them. As will be discussed below, care must be taken to only match the employment share of spinouts which can be attributed to R\&D, since spinouts in the model correspond to those in the data which are "induced" by R\&D at the incumbent. I discipline this using micro data on spinouts. [\textbf{Section in progress}] The computation of the corresponding model moment is described in \ref{appendix:calibration:WSOempShare}.



\begin{table}[]
	\centering
	\captionof{table}{Calibration targets}\label{calibration_targets}
	\begin{tabular}{rll}
		\toprule \toprule
		& Target & Model \tabularnewline
		\midrule
		\multicolumn{1}{l}{\textbf{Analytically matched}} & & 
		\tabularnewline
		Profit (\% GDP) & 8.5\% & 8.5\% 
		\tabularnewline
		\tabularnewline
		\multicolumn{1}{l}{\textbf{Numerically matched}} & & 
		\tabularnewline
		Interest rate & 6\% & 6\% 
		\tabularnewline
		Growth rate (CD + OI) & 1.3\% & 1.3\%
		\tabularnewline		
		Growth share OI & 70\% & 70\%
		\tabularnewline
		Age $<$ 6 emp. share  & 8.35\% & 8.35\%
		\tabularnewline
		R\&D-induced spinout emp. share & 13.7\% & 13.7\%
		\tabularnewline
		R\&D spending (\% GDP) & 1.5\% & 1.5\%
		\tabularnewline
		\bottomrule
	\end{tabular}
\end{table}

\normalsize

\subsection{Identification}

\autoref{calibration_identificationSources} shows the elasticity of model moments to calibrated model parameters.\footnote{This is computed as the jacobian matrix of the mapping that takes log parameters to log model moments.} It suggests how identification occurs by showing which moments are sensitive to which parameters. All moments are influenced by all parameters. Therefore, there is no one-to-one correspondence between moments and parameters used to identify them. 

To get a better sense of how identification is actually occurring in the model, \autoref{calibration_sensitivity} shows the elasticity of calibrated parameters to moment targets.\footnote{This is calculated by inverting the matrix shown in the previous figure. This is feasible because the model is locally exactly identified by the target moments.} This provides a more complete picture of the identification, since it takes into account the interaction of the various sensitivities of moments to parameters when it inverts the matrix. And it shows that conclusions based on \autoref{calibration_identificationSources} can be misleading. For example, while an increase in $\lambda$ causes a large increase in Growth Share OI, increasing the Growth Share OI moment target decreases the estimated $\lambda$. Given all the moments that need to be matched, the calibration prefers to match the higher Growth Share OI with a much higher $\chi$ and slightly lower $\lambda$.

\autoref{calibration_identificationSources_full} augments \autoref{calibration_identificationSources} with non-calibrated parameters included as both parameters and target moments. As before, \autoref{calibration_sensitivity_full} inverts this matrix to obtain the elasticity of calibrated parameters to moment targets and non-calibrated parameters. This gives the complete picture of how the model parameters are inferred. Based on this picture, I draw the following conclusions:

\begin{enumerate}
	\item The discount rate $\rho$ is identified to simultaneously match the interest rate, growth rate and IES
	\item The parameters $\lambda, \chi_I, \chi_E$ are determined in a complicated way, as can be seen from the fact that their profiles of sensitivities have similar shapes (modulo flipping the vertical axis). Still, they are not exactly the same: the effect of E on $\chi_E$ is larger in proportion to the effect of OI on $\chi_E$ than it is for $\lambda, \chi_I$. In addition, there are other subtle differences. An increase in $r$ has the same effect on $\chi_I$ as an increase in OI or E, but a different effect on $\lambda, \chi_E$. An increase in $g$ has the same sign effect as OI or E on $\chi_E$, but a different sign effect for $\lambda, \chi_I$. Finally, it can be seen from the $\chi_E$ panel that the externally set parameters $\beta, \psi$ have a large effect on $\chi_E$.
	\item The parameter $\kappa_E$ is distinguished from $\chi_E$ by matching the R\&D / GDP ratio.
	\item The parameter $\nu$ is identified by matching employment share of WSOs. 
\end{enumerate}

\begin{table}[]
	\centering
	\captionof{table}{Baseline calibration}\label{calibration_parameters}
	\begin{tabular}{rlll}
		\toprule \toprule
		Parameter & Value & Description & Source \tabularnewline
		\midrule
		$\rho$ & 0.0339 & Discount rate  & Indirect inference \tabularnewline
		$\theta$ & 2 & $\theta^{-1} = $ IES & External calibration 
		\tabularnewline
		$\beta$ & 0.094 & $\beta^{-1} = $ EoS intermediate goods & Exactly identified \tabularnewline 
		$\psi$ & 0.5 & Entrant R\&D elasticity & External calibration \tabularnewline
		$\lambda$ & 1.166 & Quality ladder step size & Indirect inference 
		\tabularnewline
		$\chi$ & 1.86 & Incumbent R\&D productivity & Indirect inference 
		\tabularnewline
		$\hat{\chi}$ & 0.116 & Entrant R\&D productivity & Indirect inference \tabularnewline 
		$\kappa_e$ & 0.738 & Non-R\&D entry cost & Indirect inference \tabularnewline
		$\nu$ & 0.0488 & Spinout generation rate  & Indirect inference\tabularnewline
		$\bar{L}_{RD}$ & 0.05 & R\&D labor allocation  & Normalization \tabularnewline
		\bottomrule
	\end{tabular}
\end{table}

\begin{figure}[]
	\includegraphics[scale = 0.43]{../code/julia/figures/simpleModel/identificationSources.pdf}
	\caption{Plot showing the elasticity of moments to model parameters. This illustrates how the model's equilibrium is affected by the various choices of parameters. These elasticities are computed by taking the jacobian matrix of the mapping from log parameters to log model moments.}
	\label{calibration_identificationSources}
\end{figure}

\begin{figure}[]
	\includegraphics[scale = 0.43]{../code/julia/figures/simpleModel/calibrationSensitivityFull.pdf}
	\caption{Same as \autoref{calibration_sensitivity}, but now including non-calibrated parameters. As before, this calculated by inverting the jacobian displayed in \autoref{calibration_identificationSources_full}.}
	\label{calibration_sensitivity_full}
\end{figure}

\section{Welfare effect of NCA enforcement and other policies}\label{sec:policy_analysis}

\paragraph{Consumption-equivalent change in welfare} 

I will compare welfare across BGPs in consumption-equivalent terms. For the purposes of this discussion, an \textit{allocation} is a set $A = \{ \tilde{C}_A, g_A \}$. Let $\tilde{W}(A)$ denote the normalized welfare corresponding to a given allocation $A$ and consider an allocation $B$ such that $\tilde{W}(A) < \tilde{W}_B$. the CEV welfare improvement from allocation $A$ to allocation $B$ is the permanent increase in consumption in allocation $A$ that achieves the same welfare as allocation $B$. To make this precise, define the allocation $\alpha(A,B)$ as
\begin{align}
\alpha &= \{C_{\alpha}, g_A\} \\
\tilde{W} ( \alpha ) &= \tilde{W} ( B ) 
\end{align}

That is, allocation $\alpha$ has the same growth rate as allocation $A$ but a different consumption level $C_{\alpha}$ such that it provides the same welfare as allocation $B$. The consumption-equivalent percentage welfare improvement of allocation $B$ over $A$ can be calculated as
\begin{align}
100 \times \big(\frac{\hat{C}_A}{\tilde{C}_A} - 1 \big) 
\end{align}

For $\theta > 1$ (the case of interest in this paper), a $\frac{\xi}{\theta-1}\%$ CEV welfare improvement results from an $\xi\%$ decrease in the absolute value of $\tilde{W}$. For $\theta < 1$, a $\frac{\xi}{1-\theta}\%$ CEV welfare improvement results from a $\xi\%$ increase in $\tilde{W}$. \footnote{The case $\theta = 1$ corresponds to log utility, in which case
	\begin{align}
	\tilde{W} &= \frac{\rho \log(\tilde{C}) + g}{\rho^2} \label{eq:agg_welfare_log}
	\end{align}
	
	In this case, there is no simple correspondence to obtain CEV welfare changes, but they are easy to compute directly. Under the null policy, initial consumption is $\tilde{C}$ and growth is $g$. Under the new policy, initial consumption is $\tilde{C}^+$ and growth is $g^+$. The CEV welfare change is $\frac{\tilde{C}^* - \tilde{C}}{\tilde{C}}$, where $\tilde{C}^*$ is defined by 
	\begin{align}
	\frac{\rho\log(\tilde{C}^*) + g}{\rho^2} = \frac{\rho \log(\tilde{C}^+) + g^+}{\rho^2} \label{eq:agg_welfare_log_CEV}
	\end{align}}

\subsection{NCA cost $\kappa_c$}

\autoref{calibration_summaryPlot} shows how the equilibrium varies with $\kappa_c$. As $\kappa_c$ increases in $[0,\bar{\kappa}_c)$, welfare decreases (third row, third panel). This is driven by the changes in the growth rate (first row, third panel). The movements in the growth rate in turn drive movements in the interest rate, via the Euler equation (second row, second panel). Both R\&D wages paid by incumbents and entrants decline and then jump downwards at the $\bar{\kappa}_c$ threshold (second row, third panel). The growth rate is driven by the changes in the innovation rate (first row, second panel). The incumbent reduces innovation gradually, while the entrant increases gradually, but by less due to lower marginal returns to R\&D in equilibrium, as inequality (\ref{cs:growth_decreasing_condition}) holds (the LHS is equal to 0.33 in this parametrization). WSOs increase the growth rate by a discrete amount as the threshold $\bar{\kappa}_c$ is crossed. Finally, the incumbent value decreases continuously until the threshold, where it jumps downwards (second row, first panel). 

To validate this result, I study its robustness to variations in the target moments in \ref{appendix:policyanalysis:ncacost}. I conclude that the result is robust to up to a 10\% or so standard deviation in the target moments (with zero correlation between moment uncertainty). Also, I show that the welfare result is reversed when calibrating the model to a 4\%, rather than 8.35\%, employment share of entering firms. This occurs due to a much higher calibrated value of $\lambda$, reducing the slack with which (\ref{cs:growth_misallocation_condition}) holds.  

\begin{figure}[]
	\includegraphics[scale = 0.57]{../code/julia/figures/simpleModel/calibration_summaryPlot.pdf}
	\caption{Effect of varying $\kappa_c$ on equilibrium variables and welfare.}
	\label{calibration_summaryPlot}
\end{figure}

\subsection{R\&D subsidy (tax)}

As numerical exercises show welfare effects are driven by productivity growth, rather than consumption at a given level of productivity, typically the effect of increasing $T_{RD}$ will be to reduce welfare, provided of course that (\ref{cs:growth_decreasing_condition}) holds. \autoref{calibration_RDSubsidy_summaryPlot} shows how the equilibrium varies with the $T_{RD}$. For this exercise, I set $\kappa_c = 1.2 \tilde{\bar{\kappa}}_c(\kappa_e,\lambda;T_{RD} = 0)$. 

\begin{figure}[]
	\includegraphics[scale = 0.57]{../code/julia/figures/simpleModel/calibration_RDSubsidy_summaryPlot.pdf}
	\caption{Summary of equilibrium for baseline parameter values and various values of $T_{RD}$. This assumes that $\kappa_c = 1.2 \tilde{\bar{\kappa}}_c(\kappa_e,\lambda;T_{RD} = 0)$.}
	\label{calibration_RDSubsidy_summaryPlot}
\end{figure}

Notice that growth (first row, third column) and welfare (third row, third column) both fall with $T_{RD}$, and jump down when the increase in $T_{RD}$ increases the use of NCAs.


\subsection{CD tax (subsidy)}\label{subsec:cd_tax}


The plots in \autoref{calibration_entryTax_summaryPlot} show the impact of varying $T_e$ on equilibrium variables, growth and welfare, when $\kappa_c = 1.2 \hat{\tilde{\kappa}}_c(\kappa_e,\lambda;T_e = 0)$. As expected based on the discussion of the previous paragraphs, growth and welfare increase in the entry tax $T_e$ due to a reallocation of R\&D to the incumbent. When the tax $T_e$ is high enough (around 16\% below), incumbents begin to use noncompetes.  

\begin{figure}[]
	\includegraphics[scale = 0.57]{../code/julia/figures/simpleModel/calibration_EntryTax_summaryPlot.pdf}
	\caption{Summary of equilibrium for baseline parameter values and various values of $T_e$. This assumes that $\kappa_c = 1.2 \hat{\bar{\kappa}}_c(\kappa_e,\lambda;T_e = 0)$.}
	\label{calibration_entryTax_summaryPlot}
\end{figure}

\subsection{OI R\&D subsidy (tax)}\label{cs:oi_rd_subsidy}

In this calibraiton, an increase in $T_{RD,I}$ significantly increases growth and welfare, as shown in \autoref{calibration_RDSubsidyTargeted_summaryPlot}. It does so by increasing incumbent R\&D. For high values of $T_{RD,I}$, there is a switch from $x = 0$ to $x = 1$ and welfare jumps down slightly. Otherwise, it is monotonically increasing. 

\begin{figure}[]
	\includegraphics[scale = 0.57]{../code/julia/figures/simpleModel/calibration_RDSubsidyTargeted_summaryPlot.pdf}
	\caption{Summary of equilibrium for baseline parameter values and various values of $T_{RD,I}$. This assumes that $\kappa_c = 1.2 \tilde{\bar{\kappa}}_c(\kappa_e,\lambda;T_{RD,I} = 0)$.}
	\label{calibration_RDSubsidyTargeted_summaryPlot}
\end{figure}


\subsection{All policies}

Because $T_{NCA}$ dominates $\kappa_c$, we know that the planner will choose $\kappa_c = \munderbar{\kappa}_c$. Next, the planner wants to reallocate R\&D to the incumbent and ensure that $x = 0$ so that spinout potential is not wasted. He can accomplish this by raising both $T_{RD,I}$, which reallocates R\&D labor to the incumbent, and $T_{NCA}$, which prevents NCAs so that $x = 0$. \autoref{calibration_ALL_summaryPlot} confirms this intuition (second row, third column). The improvement in CEV welfare is more than 7\%, which is significantly more than any of the other improvements could achieve on their own. This is driven by the fact that $T_{NCA}$ raises the threshold at which $T_{RD,I}$ induces a switch to $x = 1$ (first row, third column). [\textbf{Would be better to look at the effect of a reasonable sized subsidy, rather than full effect, but I'd rather discuss before doing that.}] 

\begin{figure}[]
	\includegraphics[scale = 0.46]{../code/julia/figures/simpleModel/calibration_ALL_summaryPlot.pdf}
	\caption{Summary of equilibrium for various values of $T_{RD,I}$ and $T_{NCA}$. This assumes the planner chooses $\kappa_c = \underline{\kappa}_c = \frac{1}{2} \bar{\bar{\kappa}}_c(\kappa_e,\lambda;T_{RD},T_{RD,I},T_e))$.}
	\label{calibration_ALL_summaryPlot}
\end{figure}

\section{Conclusion}



\bibliography{references.bib}

\appendix

\counterwithin{proposition}{section}
\counterwithin{proposition_corollary}{section}
\counterwithin{lemma}{section}
\counterwithin{lemma_corollary}{section}

\section{Appendix of tables}

\setcounter{table}{0}
\renewcommand{\thetable}{\Alph{section}\arabic{table}}

% latex table generated in R 3.6.3 by xtable 1.8-4 package
% Tue Apr 28 17:19:03 2020
\begin{table}[!htb]
\centering
\begingroup\scriptsize
\begin{tabular}{p{1.75cm}p{1.75cm}p{1.75cm}p{1.75cm}p{1.75cm}p{1.75cm}p{1.75cm}p{1.75cm}}
  \toprule
Year & Number of founders & Number of start-ups & Number of founders from public companies & Fraction from public companies (\%) & Fraction from public companies when bio. info available (\%) & Fraction from public companies in same 4-digit NAICS (\%) & Fraction from public companies in same 4-digit NAICS when bio. info available (\%) \\ 
  \midrule
1986 & 269 & 260 & 45 & 16.7 & 22.8 & 5.2 & 7.1 \\ 
  1987 & 356 & 316 & 47 & 13.2 & 16.5 & 5.1 & 6.3 \\ 
  1988 & 372 & 324 & 61 & 16.4 & 21.0 & 5.1 & 6.5 \\ 
  1989 & 479 & 376 & 82 & 17.1 & 21.0 & 5.2 & 6.4 \\ 
  1990 & 484 & 365 & 91 & 18.8 & 22.3 & 7.2 & 8.6 \\ 
  1991 & 565 & 387 & 89 & 15.8 & 18.7 & 6.7 & 8.0 \\ 
  1992 & 711 & 490 & 113 & 15.9 & 19.1 & 4.1 & 4.9 \\ 
  1993 & 827 & 519 & 154 & 18.6 & 21.3 & 8.0 & 9.1 \\ 
  1994 & 1046 & 647 & 189 & 18.1 & 20.8 & 5.9 & 6.8 \\ 
  1995 & 1364 & 825 & 243 & 17.8 & 20.0 & 6.0 & 6.7 \\ 
  1996 & 2000 & 1143 & 356 & 17.8 & 19.5 & 5.9 & 6.5 \\ 
  1997 & 2096 & 1076 & 393 & 18.8 & 20.5 & 7.2 & 7.9 \\ 
  1998 & 3044 & 1443 & 601 & 19.7 & 20.7 & 6.3 & 6.6 \\ 
  1999 & 5376 & 2436 & 1046 & 19.5 & 20.3 & 5.5 & 5.7 \\ 
  2000 & 4343 & 1866 & 892 & 20.5 & 21.7 & 5.9 & 6.3 \\ 
  2001 & 2513 & 985 & 461 & 18.3 & 19.9 & 7.7 & 8.3 \\ 
  2002 & 2499 & 913 & 496 & 19.8 & 21.6 & 8.4 & 9.1 \\ 
  2003 & 2369 & 933 & 455 & 19.2 & 21.5 & 8.4 & 9.5 \\ 
  2004 & 2615 & 1024 & 512 & 19.6 & 22.0 & 8.8 & 9.8 \\ 
  2005 & 2739 & 1105 & 528 & 19.3 & 22.1 & 8.5 & 9.8 \\ 
  2006 & 2997 & 1246 & 595 & 19.9 & 22.9 & 7.7 & 8.9 \\ 
  2007 & 3235 & 1400 & 543 & 16.8 & 20.2 & 6.3 & 7.6 \\ 
  2008 & 3067 & 1337 & 544 & 17.7 & 21.3 & 6.7 & 8.1 \\ 
  2009 & 3129 & 1387 & 516 & 16.5 & 19.8 & 5.4 & 6.5 \\ 
  2010 & 3567 & 1631 & 542 & 15.2 & 18.5 & 5.2 & 6.3 \\ 
  2011 & 4673 & 2040 & 749 & 16.0 & 19.2 & 5.5 & 6.6 \\ 
  2012 & 5093 & 2204 & 826 & 16.2 & 18.8 & 5.1 & 5.9 \\ 
  2013 & 5400 & 2272 & 905 & 16.8 & 18.7 & 4.1 & 4.6 \\ 
  2014 & 5607 & 2282 & 943 & 16.8 & 18.6 & 4.5 & 4.9 \\ 
  2015 & 5312 & 2136 & 882 & 16.6 & 18.0 & 4.6 & 5.0 \\ 
   \bottomrule
\end{tabular}
\endgroup
\caption{Summary of founders. Here, "founder" includes all individuals employed at startups inthe VentureSource database who (1) joined the startup within 3 year(s) of its founding year; and (2) have the title of CEO, CTO, CCEO, PCEO, PRE, PCHM, PCOO, FDR, CHF.} 
\label{table:GStable_founder2}
\end{table}


% latex table generated in R 3.6.3 by xtable 1.8-4 package
% Wed Nov 25 13:59:18 2020
\begin{table}[!htb]
\centering
\begingroup\scriptsize
\begin{tabular}{p{4.5cm}llrllrll}
  \toprule
Industry & Startups & Individuals & State & Startups & Individuals & Year & Startups & Individuals \\ 
  \midrule
Business Applications Software & 1890 & 24572 & California & 8921 & 110208 & 1987 & 353 & 2680 \\ 
  Biotechnology Therapeutics & 1089 & 14381 & Massachussetts & 2279 & 30536 & 1988 & 356 & 2831 \\ 
  Communications Software & 1036 & 13387 & New York & 1644 & 16896 & 1989 & 403 & 3247 \\ 
  Advertising/Marketing & 962 & 12004 & Texas & 1372 & 15324 & 1990 & 396 & 3159 \\ 
  Network/Systems Management Software & 688 & 10866 & Pennsylvania & 927 & 8906 & 1991 & 422 & 3751 \\ 
  Vertical Market Applications Software & 561 & 7065 & Washington & 827 & 9647 & 1992 & 537 & 4854 \\ 
  Online Communities & 550 & 3984 & Colorado & 637 & 7592 & 1993 & 554 & 5294 \\ 
  Application-Specific Integrated Circuits & 464 & 6202 & Virginia & 618 & 7670 & 1994 & 689 & 6735 \\ 
  IT Consulting & 461 & 5482 & Georgia & 579 & 6420 & 1995 & 876 & 8910 \\ 
  Wired Communications Equipment & 454 & 6651 & Illinois & 573 & 5981 & 1996 & 1191 & 13102 \\ 
  Drug Development Technologies & 412 & 5038 & New Jersey & 567 & 6560 & 1997 & 1141 & 13426 \\ 
  Healthcare Administration Software & 403 & 4978 & Florida & 559 & 5277 & 1998 & 1513 & 19471 \\ 
  Therapeutic Devices (Minimally Invasive/Noninvasive) & 374 & 4668 & North Carolina & 466 & 5326 & 1999 & 2557 & 32463 \\ 
  Fiberoptic Equipment & 364 & 4911 & Maryland & 437 & 5230 & 2000 & 2003 & 24251 \\ 
  Database Software & 357 & 4500 & Minnesota & 378 & 4093 & 2001 & 1067 & 13268 \\ 
  Business Support Services: Other & 341 & 3889 & Ohio & 375 & 2911 & 2002 & 986 & 12928 \\ 
  Multimedia/Streaming Software & 337 & 4066 & Connecticut & 367 & 3727 & 2003 & 1037 & 11912 \\ 
  Entertainment & 335 & 2555 & Utah & 255 & 2570 & 2004 & 1110 & 13345 \\ 
  Procurement/Supply Chain & 327 & 4744 & Oregon & 224 & 2415 & 2005 & 1222 & 13305 \\ 
  Wireless Communications Equipment & 322 & 4506 & Tennessee & 222 & 2134 & 2006 & 1380 & 13823 \\ 
  Specialty Retailers & 308 & 2915 & Arizona & 211 & 2280 & 2007 & 1506 & 13045 \\ 
  Pharmaceuticals & 301 & 3629 & Michigan & 209 & 1707 & 2008 & 1416 & 10469 \\ 
  Data Management Services & 292 & 4034 & Wisonsin & 142 & 1110 & 2009 & 1494 & 9460 \\ 
   \bottomrule
\end{tabular}
\endgroup
\caption{Statistics on startups covered by VS sample. Industry information uses VS industrial classification. Startups are counted by founding year, individuals by year they joined the firm.} 
\label{table:VS_summaryTable}
\end{table}


% latex table generated in R 3.6.3 by xtable 1.8-4 package
% Sat Sep 26 15:56:53 2020
\begin{table}[]
\centering
\begingroup\normalsize
\begin{tabular}{rlrll}
  \toprule
Employer & Count & Position & Count & Percentage \\ 
  \midrule
IBM & 228 & Executive & 5491 & 9.0 \\ 
  Microsoft & 196 & CFO & 3986 & 6.5 \\ 
  Cisco Systems & 151 & CEO & 3257 & 5.3 \\ 
  Oracle & 143 & President \& CEO & 2847 & 4.6 \\ 
  AT\&T & 128 & President & 1919 & 3.1 \\ 
  Sun Microsystems & 126 & Founder & 1642 & 2.7 \\ 
  Verizon & 110 & CTO & 1536 & 2.5 \\ 
  Ernst \& Young & 96 & Board Member, Institutional Investor & 1179 & 1.9 \\ 
  Hewlett-Packard & 94 & Cofounder & 943 & 1.5 \\ 
  Intel & 91 & COO & 932 & 1.5 \\ 
  Google & 78 & VP & 819 & 1.3 \\ 
  Stanford University & 69 & Chairman \& CEO & 645 & 1.1 \\ 
  PricewaterhouseCoopers & 68 & Founder \& CEO & 553 & 0.9 \\ 
  Motorola & 64 & Chairman & 544 & 0.9 \\ 
  Lucent Technologies & 62 & Partner & 507 & 0.8 \\ 
  Individual Investor & 57 & Managing Director & 468 & 0.8 \\ 
  Andersen Consulting & 55 & President \& COO & 415 & 0.7 \\ 
  Nortel Networks & 54 & SVP & 414 & 0.7 \\ 
  McKinsey \& Company & 53 & Chief Executive Officer & 412 & 0.7 \\ 
  Arthur Andersen & 53 & EVP & 398 & 0.6 \\ 
   \bottomrule
\end{tabular}
\endgroup
\caption{Top 20 previous employers and previous positions for executive founders in VS data.} 
\label{table:VS_previousEmployersSummaryTable}
\end{table}


% latex table generated in R 3.6.3 by xtable 1.8-4 package
% Sat Sep 26 15:56:53 2020
\begin{table}[]
\centering
\begingroup\normalsize
\begin{tabular}{rlrll}
  \toprule
Employer & Count & Position & Count & Percentage \\ 
  \midrule
IBM & 79 & Executive & 1403 & 12.6 \\ 
  Microsoft & 66 & CTO & 1176 & 10.6 \\ 
  Oracle & 51 & Founder & 486 & 4.4 \\ 
  Cisco Systems & 45 & Cofounder & 308 & 2.8 \\ 
  Sun Microsystems & 40 & CEO & 178 & 1.6 \\ 
  Stanford University & 39 & VP, Engineering & 161 & 1.5 \\ 
  Google & 32 & President & 120 & 1.1 \\ 
  MIT & 32 & CIO & 109 & 1.0 \\ 
  AT\&T & 32 & VP & 106 & 1.0 \\ 
  Hewlett-Packard & 31 & Professor & 94 & 0.8 \\ 
  Verizon & 28 & Cofounder \& CTO & 93 & 0.8 \\ 
  Lucent Technologies & 28 & President \& CEO & 89 & 0.8 \\ 
  Intel & 23 & Founder \& CEO & 81 & 0.7 \\ 
  Silicon Graphics & 19 & Director & 75 & 0.7 \\ 
  Andersen Consulting & 17 & Consultant & 66 & 0.6 \\ 
  Apple & 16 & Engineer & 65 & 0.6 \\ 
  3Com & 16 & Chief Scientific Officer & 63 & 0.6 \\ 
  Carnegie Mellon University & 16 & Founder \& CTO & 63 & 0.6 \\ 
  Apple Computer & 16 & Chief Architect & 59 & 0.5 \\ 
  Massachusetts Institute of Technology & 15 & Chief Scientist & 59 & 0.5 \\ 
   \bottomrule
\end{tabular}
\endgroup
\caption{Top 20 previous employers and previous positions for technical founders in VS data.} 
\label{table:VS_previousEmployersSummaryTable}
\end{table}


\begin{table}[!htb]
	\scriptsize
	\centering
	{
\def\sym#1{\ifmmode^{#1}\else\(^{#1}\)\fi}
\begin{tabular}{l*{6}{c}}
\toprule
                    &\multicolumn{1}{c}{(1)}&\multicolumn{1}{c}{(2)}&\multicolumn{1}{c}{(3)}&\multicolumn{1}{c}{(4)}&\multicolumn{1}{c}{(5)}&\multicolumn{1}{c}{(6)}\\
                    &\multicolumn{1}{c}{Founders}&\multicolumn{1}{c}{Founders}&\multicolumn{1}{c}{Founders}&\multicolumn{1}{c}{WSO4}&\multicolumn{1}{c}{WSO4}&\multicolumn{1}{c}{WSO4}\\
\midrule
R\&D                &        0.43\sym{***}&        0.44\sym{***}&        0.44\sym{***}&        0.27\sym{***}&        0.26\sym{***}&        0.26\sym{***}\\
                    &      (0.13)         &      (0.13)         &      (0.14)         &     (0.051)         &     (0.053)         &     (0.040)         \\
\addlinespace
hNCA=0 $\times$ R\&D&           0         &           0         &           0         &           0         &           0         &           0         \\
                    &         (.)         &         (.)         &         (.)         &         (.)         &         (.)         &         (.)         \\
\addlinespace
hNCA=1 $\times$ R\&D&        0.56         &        0.56         &        0.56         &       0.080         &       0.078         &       0.078         \\
                    &      (0.44)         &      (0.41)         &      (0.40)         &      (0.13)         &      (0.13)         &      (0.12)         \\
\addlinespace
Firm FE             &         Yes         &         Yes         &         Yes         &         Yes         &         Yes         &         Yes         \\
\addlinespace
Year FE             &         Yes         &          No         &          No         &         Yes         &          No         &          No         \\
\addlinespace
Age FE              &          No         &         Yes         &         Yes         &          No         &         Yes         &         Yes         \\
\addlinespace
Industry-Year FE    &          No         &         Yes         &         Yes         &          No         &         Yes         &         Yes         \\
\addlinespace
State-Year FE       &          No         &         Yes         &         Yes         &          No         &         Yes         &         Yes         \\
\midrule
Clustering          &       gvkey         &       gvkey         &naics4 Statecode         &       gvkey         &       gvkey         &naics4 Statecode         \\
R-squared (adj.)    &        0.67         &        0.68         &        0.68         &        0.66         &        0.64         &        0.64         \\
R-squared (within, adj)&        0.29         &        0.30         &        0.30         &        0.26         &        0.25         &        0.25         \\
Observations        &       59485         &       57956         &       57956         &       59485         &       57956         &       57956         \\
\bottomrule
\multicolumn{7}{l}{\footnotesize Standard errors in parentheses}\\
\multicolumn{7}{l}{\footnotesize \sym{*} \(p<0.1\), \sym{**} \(p<0.05\), \sym{***} \(p<0.01\)}\\
\end{tabular}
}

	\caption{The regressions above relate corporate R\&D, and its interaction with 1-digit NAICS industry, to the  entrepreneurship decisions of employees. The dependent variable is average yearly number of founders joining startups in years $t+1,t+2,t+3$. The independent variables are averages over $t,t-1,t-2$. Firm controls are employment, assets, intangible assets, investment, net income, cumulative citation-weighted patents, and the product of Tobin's Q and Assets (i.e., firm market value). Standard errors are clustered by firm in columns (1)-(3) and (5)-(7). In columns (4) and (8), standard errors are multway clustered by State and 4-digit NAICS industry.}
	\label{table:RDandSpinoutFormation_absolute_founder2_hNCA_l3f3}
\end{table}


\begin{table}[!htb]
	\scriptsize
	\centering
	{
\def\sym#1{\ifmmode^{#1}\else\(^{#1}\)\fi}
\begin{tabular}{l*{6}{c}}
\toprule
                    &\multicolumn{1}{c}{(1)}&\multicolumn{1}{c}{(2)}&\multicolumn{1}{c}{(3)}&\multicolumn{1}{c}{(4)}&\multicolumn{1}{c}{(5)}&\multicolumn{1}{c}{(6)}\\
                    &\multicolumn{1}{c}{Founders}&\multicolumn{1}{c}{Founders}&\multicolumn{1}{c}{Founders}&\multicolumn{1}{c}{WSO4}&\multicolumn{1}{c}{WSO4}&\multicolumn{1}{c}{WSO4}\\
\midrule
naics1=1 $\times$ R\&D&        1.10\sym{***}&                     &                     &        0.17\sym{**} &                     &                     \\
                    &      (0.39)         &                     &                     &     (0.073)         &                     &                     \\
\addlinespace
naics1=2 $\times$ R\&D&      -0.061         &        0.12         &        0.12         &        0.11         &        0.13         &        0.13         \\
                    &      (0.16)         &      (0.22)         &      (0.17)         &     (0.076)         &         (.)         &     (0.098)         \\
\addlinespace
naics1=3 $\times$ R\&D&        0.44\sym{***}&        0.43\sym{***}&        0.43\sym{***}&        0.23\sym{***}&        0.22         &        0.22\sym{***}\\
                    &     (0.084)         &     (0.089)         &      (0.10)         &     (0.049)         &         (.)         &     (0.053)         \\
\addlinespace
naics1=4 $\times$ R\&D&        2.55\sym{***}&        1.61\sym{***}&        1.61\sym{***}&      0.0100\sym{*}  &       0.035         &       0.035\sym{**} \\
                    &      (0.38)         &      (0.59)         &      (0.54)         &    (0.0058)         &         (.)         &     (0.015)         \\
\addlinespace
naics1=5 $\times$ R\&D&        1.68\sym{***}&        1.80\sym{***}&        1.80\sym{***}&        0.69\sym{***}&        0.70         &        0.70\sym{***}\\
                    &      (0.25)         &      (0.21)         &      (0.17)         &     (0.064)         &         (.)         &     (0.056)         \\
\addlinespace
naics1=6 $\times$ R\&D&        2.42         &        6.12\sym{***}&        6.12\sym{***}&      -0.058         &        0.29         &        0.29         \\
                    &      (1.91)         &      (2.12)         &      (2.05)         &     (0.052)         &         (.)         &      (0.23)         \\
\addlinespace
naics1=7 $\times$ R\&D&        0.84         &        0.48         &        0.48         &        0.41\sym{*}  &       -0.33         &       -0.33         \\
                    &      (1.30)         &      (0.81)         &      (0.97)         &      (0.22)         &         (.)         &      (0.32)         \\
\addlinespace
naics1=8 $\times$ R\&D&        2.65\sym{***}&                     &                     &        0.35         &                     &                     \\
                    &      (1.00)         &                     &                     &      (0.25)         &                     &                     \\
\addlinespace
naics1=9 $\times$ R\&D&       -2.28\sym{***}&        0.51         &        0.51\sym{*}  &     -0.0013         &     6.9e-17         &     6.9e-17         \\
                    &      (0.77)         &      (0.52)         &      (0.27)         &    (0.0091)         &         (.)         &   (6.9e-09)         \\
\addlinespace
Firm FE             &         Yes         &         Yes         &         Yes         &         Yes         &         Yes         &         Yes         \\
\addlinespace
Year FE             &         Yes         &          No         &          No         &         Yes         &          No         &          No         \\
\addlinespace
NAICS1-Age FE       &          No         &         Yes         &         Yes         &          No         &         Yes         &         Yes         \\
\addlinespace
Industry-Year FE    &          No         &         Yes         &         Yes         &          No         &         Yes         &         Yes         \\
\addlinespace
NAICS1-State-Year FE&          No         &         Yes         &         Yes         &          No         &         Yes         &         Yes         \\
\midrule
Clustering          &       gvkey         &       gvkey         &naics4 Statecode         &       gvkey         &       gvkey         &naics4 Statecode         \\
R-squared (adj.)    &        0.74         &        0.73         &        0.73         &        0.69         &        0.66         &        0.65         \\
R-squared (within, adj)&        0.43         &        0.38         &        0.38         &        0.34         &        0.28         &        0.28         \\
Observations        &       59485         &       56448         &       56448         &       59485         &       56448         &       56448         \\
\bottomrule
\multicolumn{7}{l}{\footnotesize Standard errors in parentheses}\\
\multicolumn{7}{l}{\footnotesize \sym{*} \(p<0.1\), \sym{**} \(p<0.05\), \sym{***} \(p<0.01\)}\\
\end{tabular}
}

	\caption{The regressions above relate corporate R\&D, and its interaction with 1-digit NAICS industry as well as an indicator for NCA enforcement, to the  entrepreneurship decisions of employees. The dependent variable is average yearly number of founders joining startups in years $t+1,t+2,t+3$. The independent variables are averages over $t,t-1,t-2$. Firm controls are employment, assets, intangible assets, investment, net income, cumulative citation-weighted patents, and the product of Tobin's Q and Assets (i.e., firm market value). Standard errors are clustered by firm in columns (1)-(3) and (5)-(7). In columns (4) and (8), standard errors are multway clustered by State and 4-digit NAICS industry.}
	\label{table:RDandSpinoutFormation_absolute_founder2_naics1_l3f3}
\end{table}

\begin{table}[!htb]
	\scriptsize
	\centering
	{
\def\sym#1{\ifmmode^{#1}\else\(^{#1}\)\fi}
\begin{tabular}{l*{6}{c}}
\toprule
                    &\multicolumn{1}{c}{(1)}&\multicolumn{1}{c}{(2)}&\multicolumn{1}{c}{(3)}&\multicolumn{1}{c}{(4)}&\multicolumn{1}{c}{(5)}&\multicolumn{1}{c}{(6)}\\
                    &\multicolumn{1}{c}{Founders}&\multicolumn{1}{c}{Founders}&\multicolumn{1}{c}{Founders}&\multicolumn{1}{c}{WSO4}&\multicolumn{1}{c}{WSO4}&\multicolumn{1}{c}{WSO4}\\
\midrule
hNCA\_xrd\_l3\_ind\_1dig1&       -1.27         &           0         &           0         &       -0.88         &           0         &           0         \\
                    &      (3.61)         &         (.)         &         (.)         &      (0.56)         &         (.)         &         (.)         \\
\addlinespace
hNCA\_xrd\_l3\_ind\_1dig2&        2.91         &        3.48         &        3.48         &       -0.16         &        0.44         &        0.44         \\
                    &      (3.64)         &      (2.58)         &      (4.05)         &      (0.28)         &         (.)         &      (0.34)         \\
\addlinespace
hNCA\_xrd\_l3\_ind\_1dig3&      -0.047         &      -0.027         &      -0.027         &      -0.052         &      -0.055         &      -0.055         \\
                    &      (0.14)         &      (0.15)         &      (0.14)         &     (0.086)         &         (.)         &     (0.073)         \\
\addlinespace
hNCA\_xrd\_l3\_ind\_1dig4&       0.092         &        0.59         &        0.59         &       0.065         &       -0.14         &       -0.14         \\
                    &      (1.74)         &      (1.73)         &      (1.23)         &     (0.071)         &         (.)         &      (0.15)         \\
\addlinespace
hNCA\_xrd\_l3\_ind\_1dig5&       0.086         &        0.14         &        0.14         &       -0.39\sym{***}&       -0.43         &       -0.43\sym{**} \\
                    &      (0.39)         &      (0.43)         &      (0.52)         &      (0.13)         &         (.)         &      (0.17)         \\
\addlinespace
hNCA\_xrd\_l3\_ind\_1dig6&        2.94         &        2.12         &        2.12         &      -0.022         &       -0.34         &       -0.34         \\
                    &      (4.55)         &      (3.88)         &      (2.31)         &     (0.099)         &         (.)         &      (0.32)         \\
\addlinespace
hNCA\_xrd\_l3\_ind\_1dig7&        6.14         &        25.2\sym{*}  &        25.2\sym{**} &        0.60         &       -1.68         &       -1.68         \\
                    &      (5.95)         &      (13.6)         &      (12.0)         &      (0.53)         &         (.)         &      (1.94)         \\
\addlinespace
hNCA\_xrd\_l3\_ind\_1dig8&       -2.01\sym{*}  &           0         &           0         &       -0.48\sym{**} &           0         &           0         \\
                    &      (1.05)         &         (.)         &(0.00000084)         &      (0.20)         &         (.)         &(0.00000036)         \\
\addlinespace
hNCA\_xrd\_l3\_ind\_1dig9&       -2.71\sym{***}&       -1.80         &       -1.80\sym{***}&       0.025         &    -1.4e-15         &    -1.4e-15         \\
                    &      (0.99)         &      (1.43)         &      (0.61)         &     (0.036)         &         (.)         &   (3.5e-09)         \\
\addlinespace
naics1=1 $\times$ R\&D&        2.27         &                     &                     &        0.97\sym{*}  &                     &                     \\
                    &      (3.34)         &                     &                     &      (0.55)         &                     &                     \\
\addlinespace
naics1=2 $\times$ R\&D&      -0.064         &        0.12         &        0.12         &        0.11         &        0.13         &        0.13         \\
                    &      (0.16)         &      (0.22)         &      (0.19)         &     (0.076)         &         (.)         &     (0.096)         \\
\addlinespace
naics1=3 $\times$ R\&D&        0.46\sym{***}&        0.44\sym{***}&        0.44\sym{***}&        0.25\sym{***}&        0.25         &        0.25\sym{***}\\
                    &      (0.12)         &      (0.13)         &      (0.14)         &     (0.060)         &         (.)         &     (0.064)         \\
\addlinespace
naics1=4 $\times$ R\&D&        2.55\sym{***}&        1.59\sym{***}&        1.59\sym{**} &      0.0075         &       0.041         &       0.041\sym{**} \\
                    &      (0.39)         &      (0.61)         &      (0.60)         &    (0.0054)         &         (.)         &     (0.016)         \\
\addlinespace
naics1=5 $\times$ R\&D&        1.61\sym{***}&        1.69\sym{***}&        1.69\sym{***}&        1.01\sym{***}&        1.04         &        1.04\sym{***}\\
                    &      (0.39)         &      (0.43)         &      (0.31)         &      (0.12)         &         (.)         &      (0.13)         \\
\addlinespace
naics1=6 $\times$ R\&D&        1.69         &        4.98         &        4.98\sym{**} &      -0.054         &        0.47         &        0.47         \\
                    &      (1.39)         &      (3.81)         &      (2.25)         &     (0.056)         &         (.)         &      (0.29)         \\
\addlinespace
naics1=7 $\times$ R\&D&       -0.38         &       -0.13         &       -0.13         &        0.29\sym{***}&       -0.29         &       -0.29         \\
                    &      (0.58)         &      (0.76)         &      (1.38)         &     (0.099)         &         (.)         &      (0.26)         \\
\addlinespace
naics1=8 $\times$ R\&D&        3.34\sym{***}&                     &                     &        0.50\sym{***}&                     &                     \\
                    &      (0.82)         &                     &                     &      (0.19)         &                     &                     \\
\addlinespace
naics1=9 $\times$ R\&D&      -0.064         &        0.65         &        0.65\sym{*}  &      -0.022         &    -4.7e-17         &    -4.7e-17         \\
                    &      (0.51)         &      (0.58)         &      (0.35)         &     (0.029)         &         (.)         &   (2.5e-09)         \\
\addlinespace
Firm FE             &         Yes         &         Yes         &         Yes         &         Yes         &         Yes         &         Yes         \\
\addlinespace
Year FE             &         Yes         &          No         &          No         &         Yes         &          No         &          No         \\
\addlinespace
NAICS1-Age FE       &          No         &         Yes         &         Yes         &          No         &         Yes         &         Yes         \\
\addlinespace
Industry-Year FE    &          No         &         Yes         &         Yes         &          No         &         Yes         &         Yes         \\
\addlinespace
NAICS1-State-Year FE&          No         &         Yes         &         Yes         &          No         &         Yes         &         Yes         \\
\midrule
Clustering          &       gvkey         &       gvkey         &naics4 Statecode         &       gvkey         &       gvkey         &naics4 Statecode         \\
R-squared (adj.)    &        0.74         &        0.73         &        0.73         &        0.69         &        0.66         &        0.66         \\
R-squared (within, adj)&        0.43         &        0.38         &        0.38         &        0.34         &        0.28         &        0.28         \\
Observations        &       59485         &       56448         &       56448         &       59485         &       56448         &       56448         \\
\bottomrule
\multicolumn{7}{l}{\footnotesize Standard errors in parentheses}\\
\multicolumn{7}{l}{\footnotesize \sym{*} \(p<0.1\), \sym{**} \(p<0.05\), \sym{***} \(p<0.01\)}\\
\end{tabular}
}

	\caption{The regressions above relate corporate R\&D, and its interaction with 1-digit NAICS industry as well as an indicator for NCA enforcement, to the  entrepreneurship decisions of employees. The dependent variable is average yearly number of founders joining startups in years $t+1,t+2,t+3$. The independent variables are averages over $t,t-1,t-2$. Firm controls are employment, assets, intangible assets, investment, net income, cumulative citation-weighted patents, and the product of Tobin's Q and Assets (i.e., firm market value). Standard errors are clustered by firm in columns (1)-(3) and (5)-(7). In columns (4) and (8), standard errors are multway clustered by State and 4-digit NAICS industry.}
	\label{table:RDandSpinoutFormation_absolute_founder2_naics1_hNCA_l3f3}
\end{table}

\begin{table}[!htb]
	\scriptsize
	\centering
	{
\def\sym#1{\ifmmode^{#1}\else\(^{#1}\)\fi}
\begin{tabular}{l*{6}{c}}
\toprule
                    &\multicolumn{1}{c}{(1)}&\multicolumn{1}{c}{(2)}&\multicolumn{1}{c}{(3)}&\multicolumn{1}{c}{(4)}&\multicolumn{1}{c}{(5)}&\multicolumn{1}{c}{(6)}\\
                    &\multicolumn{1}{c}{Founders}&\multicolumn{1}{c}{Founders}&\multicolumn{1}{c}{Founders}&\multicolumn{1}{c}{WSO4}&\multicolumn{1}{c}{WSO4}&\multicolumn{1}{c}{WSO4}\\
\midrule
hNCA\_xrd\_l3\_ind\_2dig31&        0.11         &        0.21         &        0.21         &     -0.0097         &     -0.0023         &     -0.0023         \\
                    &      (0.34)         &      (0.29)         &      (0.39)         &    (0.0068)         &         (.)         &    (0.0073)         \\
\addlinespace
hNCA\_xrd\_l3\_ind\_2dig32&       0.090         &       0.086         &       0.086\sym{*}  &       -0.11         &       -0.13         &       -0.13\sym{***}\\
                    &     (0.060)         &     (0.061)         &     (0.046)         &     (0.085)         &         (.)         &     (0.036)         \\
\addlinespace
hNCA\_xrd\_l3\_ind\_2dig33&      -0.060         &      -0.061         &      -0.061         &       0.078         &       0.097         &       0.097         \\
                    &      (0.30)         &      (0.31)         &      (0.23)         &      (0.13)         &         (.)         &      (0.10)         \\
\addlinespace
hNCA\_xrd\_l3\_ind\_2dig51&      -0.010         &       -2.12\sym{*}  &       -2.12\sym{*}  &       -0.45\sym{***}&       -2.17         &       -2.17\sym{***}\\
                    &      (0.42)         &      (1.15)         &      (1.25)         &      (0.14)         &         (.)         &      (0.55)         \\
\addlinespace
hNCA\_xrd\_l3\_ind\_2dig52&        3.84\sym{*}  &        2.12         &        2.12         &        0.56         &        0.79         &        0.79         \\
                    &      (2.03)         &      (2.21)         &      (2.61)         &      (0.39)         &         (.)         &      (0.65)         \\
\addlinespace
hNCA\_xrd\_l3\_ind\_2dig53&        9.94\sym{***}&        0.56         &        0.56         &      -0.056         &     1.3e-16         &     1.3e-16         \\
                    &      (1.89)         &      (3.36)         &      (4.11)         &      (0.12)         &         (.)         &(0.00000028)         \\
\addlinespace
hNCA\_xrd\_l3\_ind\_2dig54&       -0.94         &       -1.66         &       -1.66         &       -0.25         &       -0.55         &       -0.55         \\
                    &      (0.85)         &      (1.22)         &      (1.35)         &      (0.51)         &         (.)         &      (0.42)         \\
\addlinespace
hNCA\_xrd\_l3\_ind\_2dig55&           0         &           0         &           0         &           0         &           0         &           0         \\
                    &         (.)         &         (.)         &(0.000000053)         &         (.)         &         (.)         &(0.00000027)         \\
\addlinespace
hNCA\_xrd\_l3\_ind\_2dig56&        4.82\sym{***}&        1.82         &        1.82         &      -0.060         &       -0.29         &       -0.29         \\
                    &      (1.77)         &      (4.24)         &      (4.46)         &      (0.11)         &         (.)         &      (0.77)         \\
\addlinespace
hNCA\_xrd\_l3\_notIn3or5&       -3.64\sym{***}&       -3.07\sym{***}&       -3.07\sym{***}&      -0.017         &      -0.018         &      -0.018         \\
                    &      (0.70)         &      (0.44)         &      (0.44)         &     (0.026)         &         (.)         &     (0.026)         \\
\addlinespace
naics2\_selected=0 $\times$ R\&D&        0.72         &        0.23         &        0.23         &      0.0055         &       0.015         &       0.015         \\
                    &      (0.52)         &      (0.23)         &      (0.39)         &     (0.022)         &         (.)         &     (0.024)         \\
\addlinespace
naics2\_selected=31 $\times$ R\&D&       -0.17         &       -0.20         &       -0.20         &      0.0036         &      0.0034         &      0.0034         \\
                    &      (0.29)         &      (0.26)         &      (0.32)         &    (0.0041)         &         (.)         &    (0.0051)         \\
\addlinespace
naics2\_selected=32 $\times$ R\&D&        0.39\sym{***}&        0.39\sym{***}&        0.39\sym{***}&        0.39\sym{***}&        0.39         &        0.39\sym{***}\\
                    &     (0.056)         &     (0.065)         &     (0.055)         &     (0.050)         &         (.)         &     (0.044)         \\
\addlinespace
naics2\_selected=33 $\times$ R\&D&        0.57\sym{**} &        0.53\sym{*}  &        0.53\sym{**} &       0.050         &       0.031         &       0.031         \\
                    &      (0.26)         &      (0.28)         &      (0.22)         &     (0.088)         &         (.)         &     (0.029)         \\
\addlinespace
naics2\_selected=51 $\times$ R\&D&        1.30\sym{***}&        1.42\sym{***}&        1.42\sym{***}&        0.99\sym{***}&        1.03         &        1.03\sym{***}\\
                    &      (0.37)         &      (0.34)         &      (0.40)         &      (0.13)         &         (.)         &      (0.13)         \\
\addlinespace
naics2\_selected=52 $\times$ R\&D&       -0.73\sym{*}  &       -0.38         &       -0.38         &       -0.67\sym{*}  &       -0.84         &       -0.84         \\
                    &      (0.44)         &      (0.70)         &      (0.85)         &      (0.39)         &         (.)         &      (0.59)         \\
\addlinespace
naics2\_selected=53 $\times$ R\&D&       -3.79\sym{**} &       -4.41\sym{***}&       -4.41\sym{**} &      -0.012         &     3.0e-17         &     3.0e-17         \\
                    &      (1.89)         &      (1.52)         &      (2.06)         &      (0.12)         &         (.)         &(0.00000013)         \\
\addlinespace
naics2\_selected=54 $\times$ R\&D&        0.48\sym{**} &        0.13         &        0.13         &        0.92\sym{**} &        0.93         &        0.93\sym{***}\\
                    &      (0.20)         &      (0.29)         &      (0.48)         &      (0.37)         &         (.)         &      (0.28)         \\
\addlinespace
naics2\_selected=56 $\times$ R\&D&        1.45         &        7.37\sym{**} &        7.37\sym{**} &       0.049         &       -0.29         &       -0.29         \\
                    &      (1.76)         &      (3.33)         &      (3.63)         &      (0.11)         &         (.)         &      (0.41)         \\
\addlinespace
Firm FE             &         Yes         &         Yes         &         Yes         &         Yes         &         Yes         &         Yes         \\
\addlinespace
Year FE             &         Yes         &          No         &          No         &         Yes         &          No         &          No         \\
\addlinespace
NAICS2*-Age FE      &          No         &         Yes         &         Yes         &          No         &         Yes         &         Yes         \\
\addlinespace
Industry-Year FE    &          No         &         Yes         &         Yes         &          No         &         Yes         &         Yes         \\
\addlinespace
NAICS2*-State-Year FE&          No         &         Yes         &         Yes         &          No         &         Yes         &         Yes         \\
\midrule
Clustering          &       gvkey         &       gvkey         &naics4 Statecode         &       gvkey         &       gvkey         &naics4 Statecode         \\
R-squared (adj.)    &        0.75         &        0.76         &        0.76         &        0.70         &        0.69         &        0.69         \\
R-squared (within, adj)&        0.47         &        0.26         &        0.26         &        0.37         &        0.22         &        0.22         \\
Observations        &       59485         &       56202         &       56202         &       59485         &       56202         &       56202         \\
\bottomrule
\multicolumn{7}{l}{\footnotesize Standard errors in parentheses}\\
\multicolumn{7}{l}{\footnotesize \sym{*} \(p<0.1\), \sym{**} \(p<0.05\), \sym{***} \(p<0.01\)}\\
\end{tabular}
}

	\caption{The regressions above relate corporate R\&D, and its interaction with some 2-digit NAICS industries as well as an indicator for NCA enforcement, to the  entrepreneurship decisions of employees. The dependent variable is average yearly number of founders joining startups in years $t+1,t+2,t+3$. The independent variables are averages over $t,t-1,t-2$. Firm controls are employment, assets, intangible assets, investment, net income, cumulative citation-weighted patents, and the product of Tobin's Q and Assets (i.e., firm market value). Standard errors are clustered by firm in columns (1)-(3) and (5)-(7). In columns (4) and (8), standard errors are multway clustered by State and 4-digit NAICS industry.}
	\label{table:RDandSpinoutFormation_absolute_founder2_naics2_hNCA_l3f3}
\end{table}


\begin{table}[!htb]
	\scriptsize
	\centering
	{
\def\sym#1{\ifmmode^{#1}\else\(^{#1}\)\fi}
\begin{tabular}{l*{6}{c}}
\toprule
                    &\multicolumn{1}{c}{(1)}&\multicolumn{1}{c}{(2)}&\multicolumn{1}{c}{(3)}&\multicolumn{1}{c}{(4)}&\multicolumn{1}{c}{(5)}&\multicolumn{1}{c}{(6)}\\
                    &\multicolumn{1}{c}{Founders}&\multicolumn{1}{c}{Founders}&\multicolumn{1}{c}{Founders}&\multicolumn{1}{c}{WSO4}&\multicolumn{1}{c}{WSO4}&\multicolumn{1}{c}{WSO4}\\
\midrule
hNCA\_xrd\_l3\_ind\_3dig325&       0.077         &       0.078         &       0.078         &       -0.11         &       -0.15\sym{**} &       -0.15\sym{***}\\
                    &     (0.055)         &     (0.056)         &     (0.055)         &     (0.079)         &     (0.076)         &     (0.039)         \\
\addlinespace
hNCA\_xrd\_l3\_ind\_3dig333&       -0.67         &       -0.65         &       -0.65         &       -0.36\sym{***}&       -0.41\sym{***}&       -0.41\sym{***}\\
                    &      (0.54)         &      (0.49)         &      (0.40)         &      (0.12)         &      (0.14)         &      (0.14)         \\
\addlinespace
hNCA\_xrd\_l3\_ind\_3dig334&        0.81\sym{**} &        0.85\sym{**} &        0.85\sym{***}&        0.39\sym{**} &        0.40\sym{**} &        0.40\sym{*}  \\
                    &      (0.34)         &      (0.40)         &      (0.11)         &      (0.19)         &      (0.18)         &      (0.20)         \\
\addlinespace
hNCA\_xrd\_l3\_ind\_3dig336&       0.016         &       0.062         &       0.062         &      -0.063\sym{*}  &      -0.073\sym{*}  &      -0.073         \\
                    &      (0.31)         &      (0.32)         &      (0.33)         &     (0.035)         &     (0.041)         &     (0.055)         \\
\addlinespace
hNCA\_xrd\_l3\_ind\_3dig511&       -2.65\sym{***}&       -4.21\sym{***}&       -4.21\sym{***}&       -1.97\sym{***}&       -2.67\sym{***}&       -2.67\sym{***}\\
                    &      (0.75)         &      (1.28)         &      (0.72)         &      (0.45)         &      (0.85)         &      (0.40)         \\
\addlinespace
hNCA\_xrd\_l3\_ind\_3dig519&        11.4\sym{***}&        11.1\sym{***}&        11.1\sym{***}&        1.42\sym{***}&        0.76         &        0.76         \\
                    &      (2.71)         &      (2.75)         &      (0.87)         &      (0.52)         &      (0.69)         &      (1.14)         \\
\addlinespace
naics3\_selected=0 $\times$ R\&D&        0.12         &       0.097         &       0.097         &       0.056         &       0.069\sym{*}  &       0.069         \\
                    &      (0.28)         &      (0.30)         &      (0.32)         &     (0.034)         &     (0.040)         &     (0.053)         \\
\addlinespace
naics3\_selected=325 $\times$ R\&D&        0.41\sym{***}&        0.43\sym{***}&        0.43\sym{***}&        0.38\sym{***}&        0.38\sym{***}&        0.38\sym{***}\\
                    &     (0.072)         &     (0.097)         &     (0.093)         &     (0.062)         &     (0.083)         &     (0.074)         \\
\addlinespace
naics3\_selected=333 $\times$ R\&D&        0.76\sym{**} &        0.83\sym{***}&        0.83\sym{**} &        0.29\sym{***}&        0.30\sym{***}&        0.30\sym{***}\\
                    &      (0.30)         &      (0.30)         &      (0.37)         &     (0.052)         &     (0.070)         &     (0.072)         \\
\addlinespace
naics3\_selected=334 $\times$ R\&D&        0.35         &        0.35         &        0.35\sym{**} &      -0.042         &      -0.049         &      -0.049         \\
                    &      (0.36)         &      (0.38)         &      (0.17)         &      (0.13)         &      (0.15)         &     (0.052)         \\
\addlinespace
naics3\_selected=511 $\times$ R\&D&        1.08\sym{***}&        1.26\sym{***}&        1.26\sym{***}&        0.83\sym{***}&        0.96\sym{***}&        0.96\sym{***}\\
                    &      (0.30)         &      (0.38)         &      (0.15)         &      (0.22)         &      (0.24)         &     (0.087)         \\
\addlinespace
naics3\_selected=519 $\times$ R\&D&       -3.56         &       -6.36\sym{***}&       -6.36\sym{***}&       -1.79\sym{***}&       -2.25\sym{***}&       -2.25\sym{***}\\
                    &      (2.51)         &      (1.07)         &      (0.46)         &      (0.51)         &      (0.30)         &      (0.28)         \\
\addlinespace
Firm FE             &         Yes         &         Yes         &         Yes         &         Yes         &         Yes         &         Yes         \\
\addlinespace
Year FE             &         Yes         &          No         &          No         &         Yes         &          No         &          No         \\
\addlinespace
NAICS3*-Age FE      &          No         &         Yes         &         Yes         &          No         &         Yes         &         Yes         \\
\addlinespace
Industry-Year FE    &          No         &         Yes         &         Yes         &          No         &         Yes         &         Yes         \\
\addlinespace
NAICS3*-State-Year FE&          No         &         Yes         &         Yes         &          No         &         Yes         &         Yes         \\
\midrule
Clustering          &       gvkey         &       gvkey         &naics4 Statecode         &       gvkey         &       gvkey         &naics4 Statecode         \\
R-squared (adj.)    &        0.75         &        0.76         &        0.75         &        0.72         &        0.72         &        0.72         \\
R-squared (within, adj)&        0.45         &        0.24         &        0.24         &        0.41         &        0.24         &        0.24         \\
Observations        &       59485         &       57149         &       57149         &       59485         &       57149         &       57149         \\
\bottomrule
\multicolumn{7}{l}{\footnotesize Standard errors in parentheses}\\
\multicolumn{7}{l}{\footnotesize \sym{*} \(p<0.1\), \sym{**} \(p<0.05\), \sym{***} \(p<0.01\)}\\
\end{tabular}
}

	\caption{The regressions above relate corporate R\&D, and its interaction with some 3-digit NAICS industries as well as an indicator for NCA enforcement, to the  entrepreneurship decisions of employees. The dependent variable is average yearly number of founders joining startups in years $t+1,t+2,t+3$. The independent variables are averages over $t,t-1,t-2$. Firm controls are employment, assets, intangible assets, investment, net income, cumulative citation-weighted patents, and the product of Tobin's Q and Assets (i.e., firm market value). Standard errors are clustered by firm in columns (1)-(3) and (5)-(7). In columns (4) and (8), standard errors are multway clustered by State and 4-digit NAICS industry.}
	\label{table:RDandSpinoutFormation_absolute_founder2_naics3_hNCA_l3f3}
\end{table}

\begin{table}[!htb]
	\scriptsize
	\centering
	{
\def\sym#1{\ifmmode^{#1}\else\(^{#1}\)\fi}
\begin{tabular}{l*{6}{c}}
\toprule
                    &\multicolumn{1}{c}{(1)}&\multicolumn{1}{c}{(2)}&\multicolumn{1}{c}{(3)}&\multicolumn{1}{c}{(4)}&\multicolumn{1}{c}{(5)}&\multicolumn{1}{c}{(6)}\\
                    &\multicolumn{1}{c}{$\frac{\textrm{Founders}}{\textrm{Assets}}$}&\multicolumn{1}{c}{$\frac{\textrm{Founders}}{\textrm{Assets}}$}&\multicolumn{1}{c}{$\frac{\textrm{Founders}}{\textrm{Assets}}$}&\multicolumn{1}{c}{$\frac{\textrm{WSO4}}{\textrm{Assets}}$}&\multicolumn{1}{c}{$\frac{\textrm{WSO4}}{\textrm{Assets}}$}&\multicolumn{1}{c}{$\frac{\textrm{WSO4}}{\textrm{Assets}}$}\\
\midrule
$\frac{\textrm{R\&D}}{\textrm{Assets}}$&        1.34         &        1.41         &        1.41         &        0.66\sym{*}  &        0.61\sym{+}  &        0.61\sym{**} \\
                    &      (1.12)         &      (1.14)         &      (1.19)         &      (0.40)         &      (0.40)         &      (0.30)         \\
\addlinespace
hNCA=0 $\times$ $\frac{\textrm{R\&D}}{\textrm{Assets}}$&           0         &           0         &           0         &           0         &           0         &           0         \\
                    &         (.)         &         (.)         &         (.)         &         (.)         &         (.)         &         (.)         \\
\addlinespace
hNCA=1 $\times$ $\frac{\textrm{R\&D}}{\textrm{Assets}}$&       -0.56         &       -0.54         &       -0.54         &       -0.30         &       -0.27         &       -0.27         \\
                    &      (1.39)         &      (1.42)         &      (1.36)         &      (0.76)         &      (0.76)         &      (0.47)         \\
\addlinespace
Firm FE             &         Yes         &         Yes         &         Yes         &         Yes         &         Yes         &         Yes         \\
\addlinespace
Year FE             &         Yes         &          No         &          No         &         Yes         &          No         &          No         \\
\addlinespace
Age FE              &          No         &         Yes         &         Yes         &          No         &         Yes         &         Yes         \\
\addlinespace
Industry-Year FE    &          No         &         Yes         &         Yes         &          No         &         Yes         &         Yes         \\
\addlinespace
State-Year FE       &          No         &         Yes         &         Yes         &          No         &         Yes         &         Yes         \\
\midrule
Clustering          &       gvkey         &       gvkey         &naics4 Statecode         &       gvkey         &       gvkey         &naics4 Statecode         \\
R-squared (adj.)    &        0.26         &        0.21         &        0.21         &        0.27         &        0.22         &        0.22         \\
R-squared (within, adj)&      0.0021         &      0.0020         &      0.0020         &     0.00093         &     0.00086         &     0.00086         \\
Observations        &       59477         &       57948         &       57948         &       59477         &       57948         &       57948         \\
\bottomrule
\multicolumn{7}{l}{\footnotesize Standard errors in parentheses}\\
\multicolumn{7}{l}{\footnotesize \sym{++} \(p<0.2\), \sym{+} \(p<0.15\), \sym{*} \(p<0.1\), \sym{**} \(p<0.05\), \sym{***} \(p<0.01\)}\\
\end{tabular}
}

	\caption{The regressions above relate corporate R\&D, and its interaction with 1-digit NAICS industry, to the  entrepreneurship decisions of employees. The dependent variable is average yearly number of founders joining startups in years $t+1,t+2,t+3$. The independent variables are averages over $t,t-1,t-2$. All LHS and RHS variable (except Tobin's Q) are normalized by a trailing 5 year moving average of assets. Firm controls are employment, assets, intangible assets, investment, net income, cumulative citation-weighted patents, and Tobin's Q. Standard errors are clustered by firm in columns (1)-(3) and (5)-(7). In columns (4) and (8), standard errors are multway clustered by State and 4-digit NAICS industry.}
	\label{table:RDandSpinoutFormation_at_founder2_hNCA_l3f3}
\end{table}

\begin{table}[!htb]
	\scriptsize
	\centering
	{
\def\sym#1{\ifmmode^{#1}\else\(^{#1}\)\fi}
\begin{tabular}{l*{6}{c}}
\toprule
                    &\multicolumn{1}{c}{(1)}&\multicolumn{1}{c}{(2)}&\multicolumn{1}{c}{(3)}&\multicolumn{1}{c}{(4)}&\multicolumn{1}{c}{(5)}&\multicolumn{1}{c}{(6)}\\
                    &\multicolumn{1}{c}{$\frac{\textrm{Founders}}{\textrm{Assets}}$}&\multicolumn{1}{c}{$\frac{\textrm{Founders}}{\textrm{Assets}}$}&\multicolumn{1}{c}{$\frac{\textrm{Founders}}{\textrm{Assets}}$}&\multicolumn{1}{c}{$\frac{\textrm{WSO4}}{\textrm{Assets}}$}&\multicolumn{1}{c}{$\frac{\textrm{WSO4}}{\textrm{Assets}}$}&\multicolumn{1}{c}{$\frac{\textrm{WSO4}}{\textrm{Assets}}$}\\
\midrule
xrd\_at\_l3\_industry\_1dig1&        0.13         &        0.57         &        0.57         &        0.17         &        0.39         &        0.39         \\
                    &      (0.36)         &      (1.05)         &      (0.70)         &      (0.21)         &      (0.86)         &      (0.39)         \\
\addlinespace
xrd\_at\_l3\_industry\_1dig2&       -0.14         &       -0.24         &       -0.24         &       0.013         &       0.046         &       0.046         \\
                    &      (0.14)         &      (0.19)         &      (0.17)         &     (0.073)         &      (0.12)         &     (0.066)         \\
\addlinespace
xrd\_at\_l3\_industry\_1dig3&        1.35         &        1.44         &        1.44         &        0.60         &        0.62         &        0.62\sym{***}\\
                    &      (0.87)         &      (0.89)         &      (0.94)         &      (0.38)         &      (0.42)         &      (0.20)         \\
\addlinespace
xrd\_at\_l3\_industry\_1dig4&       -0.33         &        0.38         &        0.38         &       -0.13         &        0.19         &        0.19         \\
                    &      (1.07)         &      (1.10)         &      (1.14)         &      (0.16)         &      (0.38)         &      (0.19)         \\
\addlinespace
xrd\_at\_l3\_industry\_1dig5&       0.034         &      -0.090         &      -0.090         &        0.33         &      -0.099         &      -0.099         \\
                    &      (1.02)         &      (1.07)         &      (1.15)         &      (0.75)         &      (0.90)         &      (0.85)         \\
\addlinespace
xrd\_at\_l3\_industry\_1dig6&        0.22         &        0.95         &        0.95         &       -0.12         &       -0.14         &       -0.14         \\
                    &      (0.57)         &      (0.94)         &      (1.11)         &      (0.16)         &      (0.21)         &      (0.21)         \\
\addlinespace
xrd\_at\_l3\_industry\_1dig7&       -13.8\sym{***}&       -15.8\sym{***}&       -15.8\sym{***}&       -0.31         &       -0.47         &       -0.47         \\
                    &      (5.20)         &      (4.09)         &      (3.44)         &      (0.71)         &      (0.95)         &      (0.86)         \\
\addlinespace
xrd\_at\_l3\_industry\_1dig8&        0.53\sym{*}  &        0.54         &        0.54         &      -0.060         &       -0.43         &       -0.43         \\
                    &      (0.28)         &      (0.79)         &      (0.69)         &      (0.17)         &      (0.33)         &      (0.33)         \\
\addlinespace
xrd\_at\_l3\_industry\_1dig9&       -0.30         &       -0.46         &       -0.46         &      -0.063         &      -0.026         &      -0.026         \\
                    &      (0.38)         &      (0.58)         &      (0.37)         &     (0.075)         &      (0.13)         &     (0.091)         \\
\addlinespace
Firm FE             &         Yes         &         Yes         &         Yes         &         Yes         &         Yes         &         Yes         \\
\addlinespace
Year FE             &         Yes         &          No         &          No         &         Yes         &          No         &          No         \\
\addlinespace
Age FE              &          No         &         Yes         &         Yes         &          No         &         Yes         &         Yes         \\
\addlinespace
Industry-Year FE    &          No         &         Yes         &         Yes         &          No         &         Yes         &         Yes         \\
\addlinespace
State-Year FE       &          No         &         Yes         &         Yes         &          No         &         Yes         &         Yes         \\
\midrule
Clustering          &       gvkey         &       gvkey         &naics4 Statecode         &       gvkey         &       gvkey         &naics4 Statecode         \\
R-squared (adj.)    &        0.26         &        0.21         &        0.21         &        0.27         &        0.22         &        0.22         \\
R-squared (within, adj)&      0.0023         &      0.0022         &      0.0022         &     0.00079         &     0.00083         &     0.00083         \\
Observations        &       59477         &       57948         &       57948         &       59477         &       57948         &       57948         \\
\bottomrule
\multicolumn{7}{l}{\footnotesize Standard errors in parentheses}\\
\multicolumn{7}{l}{\footnotesize \sym{*} \(p<0.1\), \sym{**} \(p<0.05\), \sym{***} \(p<0.01\)}\\
\end{tabular}
}

	\caption{The regressions above relate corporate R\&D, and its interaction with 1-digit NAICS industry as well as an indicator for NCA enforcement, to the  entrepreneurship decisions of employees. The dependent variable is average yearly number of founders joining startups in years $t+1,t+2,t+3$. The independent variables are averages over $t,t-1,t-2$. The independent variables are averages over $t,t-1,t-2$. All LHS and RHS variable (except Tobin's Q) are normalized by a trailing 5 year moving average of assets. Firm controls are employment, assets, intangible assets, investment, net income, cumulative citation-weighted patents, and the product of Tobin's Q. Standard errors are clustered by firm in columns (1)-(3) and (5)-(7). In columns (4) and (8), standard errors are multway clustered by State and 4-digit NAICS industry.}
	\label{table:RDandSpinoutFormation_at_founder2_naics1_l3f3}
\end{table}

\begin{table}[!htb]
	\scriptsize
	\centering
	{
\def\sym#1{\ifmmode^{#1}\else\(^{#1}\)\fi}
\begin{tabular}{l*{6}{c}}
\toprule
                    &\multicolumn{1}{c}{(1)}&\multicolumn{1}{c}{(2)}&\multicolumn{1}{c}{(3)}&\multicolumn{1}{c}{(4)}&\multicolumn{1}{c}{(5)}&\multicolumn{1}{c}{(6)}\\
                    &\multicolumn{1}{c}{$\frac{\textrm{Founders}}{\textrm{Assets}}$}&\multicolumn{1}{c}{$\frac{\textrm{Founders}}{\textrm{Assets}}$}&\multicolumn{1}{c}{$\frac{\textrm{Founders}}{\textrm{Assets}}$}&\multicolumn{1}{c}{$\frac{\textrm{WSO4}}{\textrm{Assets}}$}&\multicolumn{1}{c}{$\frac{\textrm{WSO4}}{\textrm{Assets}}$}&\multicolumn{1}{c}{$\frac{\textrm{WSO4}}{\textrm{Assets}}$}\\
\midrule
xrd\_at\_l3\_industry\_1dig1&        0.91         &        6.79         &        6.79         &        0.14         &        4.66         &        4.66         \\
                    &      (1.39)         &      (13.4)         &      (7.06)         &      (0.78)         &      (12.9)         &      (4.35)         \\
\addlinespace
xrd\_at\_l3\_industry\_1dig2&       -0.36         &       0.035         &       0.035         &       -0.21         &      -0.057         &      -0.057         \\
                    &      (0.97)         &      (1.33)         &      (0.99)         &      (0.51)         &      (0.76)         &      (0.49)         \\
\addlinespace
xrd\_at\_l3\_industry\_1dig3&        1.78         &        1.84         &        1.84         &        0.68         &        0.69         &        0.69\sym{**} \\
                    &      (1.36)         &      (1.36)         &      (1.54)         &      (0.47)         &      (0.45)         &      (0.32)         \\
\addlinespace
xrd\_at\_l3\_industry\_1dig4&      -0.040         &        0.53         &        0.53         &       -0.14         &       0.024         &       0.024         \\
                    &      (0.72)         &      (0.96)         &      (1.22)         &      (0.21)         &      (0.43)         &      (0.26)         \\
\addlinespace
xrd\_at\_l3\_industry\_1dig5&       -0.78         &       -0.95         &       -0.95         &        0.79         &        0.27         &        0.27         \\
                    &      (1.05)         &      (1.11)         &      (1.23)         &      (0.70)         &      (0.90)         &      (0.58)         \\
\addlinespace
xrd\_at\_l3\_industry\_1dig6&      -0.029         &        0.11         &        0.11         &      -0.096         &       0.063         &       0.063         \\
                    &      (0.43)         &      (0.57)         &      (0.58)         &      (0.17)         &      (0.20)         &      (0.22)         \\
\addlinespace
xrd\_at\_l3\_industry\_1dig7&       -17.4\sym{***}&       -18.4\sym{***}&       -18.4\sym{***}&       -0.31         &      -0.017         &      -0.017         \\
                    &      (3.52)         &      (2.94)         &      (2.17)         &      (0.86)         &      (0.95)         &      (0.91)         \\
\addlinespace
xrd\_at\_l3\_industry\_1dig8&        0.52\sym{*}  &        0.33         &        0.33         &      -0.057         &       -0.53         &       -0.53         \\
                    &      (0.28)         &      (0.72)         &      (0.80)         &      (0.17)         &      (0.37)         &      (0.44)         \\
\addlinespace
xrd\_at\_l3\_industry\_1dig9&       -0.54         &       -0.80         &       -0.80         &      -0.058         &       0.042         &       0.042         \\
                    &      (0.63)         &      (0.93)         &      (0.79)         &     (0.095)         &      (0.15)         &      (0.10)         \\
\addlinespace
hNCA\_xrd\_at\_l3\_industry\_1dig1&       -0.91         &       -6.60         &       -6.60         &       0.040         &       -4.54         &       -4.54         \\
                    &      (1.40)         &      (13.5)         &      (7.25)         &      (0.80)         &      (12.9)         &      (4.33)         \\
\addlinespace
hNCA\_xrd\_at\_l3\_industry\_1dig2&        0.24         &       -0.29         &       -0.29         &        0.24         &        0.11         &        0.11         \\
                    &      (0.97)         &      (1.36)         &      (0.90)         &      (0.51)         &      (0.75)         &      (0.46)         \\
\addlinespace
hNCA\_xrd\_at\_l3\_industry\_1dig3&       -0.95         &       -0.89         &       -0.89         &       -0.19         &       -0.17         &       -0.17         \\
                    &      (1.68)         &      (1.69)         &      (1.73)         &      (0.90)         &      (0.89)         &      (0.65)         \\
\addlinespace
hNCA\_xrd\_at\_l3\_industry\_1dig4&       -0.81         &       -0.44         &       -0.44         &       0.027         &        0.45         &        0.45         \\
                    &      (2.77)         &      (2.99)         &      (2.83)         &      (0.19)         &      (0.90)         &      (0.77)         \\
\addlinespace
hNCA\_xrd\_at\_l3\_industry\_1dig5&        1.62         &        1.65         &        1.65         &       -0.91         &       -0.70         &       -0.70         \\
                    &      (2.06)         &      (2.08)         &      (2.25)         &      (1.53)         &      (1.68)         &      (1.15)         \\
\addlinespace
hNCA\_xrd\_at\_l3\_industry\_1dig6&        0.58         &        2.25         &        2.25         &      -0.037         &       -0.53         &       -0.53         \\
                    &      (0.97)         &      (2.00)         &      (2.15)         &     (0.083)         &      (0.32)         &      (0.47)         \\
\addlinespace
hNCA\_xrd\_at\_l3\_industry\_1dig7&        18.0\sym{***}&        18.0\sym{***}&        18.0\sym{***}&     -0.0055         &       -2.82         &       -2.82\sym{**} \\
                    &      (3.80)         &      (4.34)         &      (1.61)         &      (1.01)         &      (2.42)         &      (1.28)         \\
\addlinespace
hNCA\_xrd\_at\_l3\_industry\_1dig8&       0.092         &        23.8         &        23.8         &     -0.0032         &        15.4         &        15.4         \\
                    &      (0.80)         &      (39.6)         &      (22.4)         &      (0.51)         &      (18.4)         &      (12.6)         \\
\addlinespace
hNCA\_xrd\_at\_l3\_industry\_1dig9&        0.53         &        0.76         &        0.76         &     -0.0084         &       -0.14         &       -0.14         \\
                    &      (0.62)         &      (0.94)         &      (0.71)         &     (0.058)         &      (0.15)         &      (0.13)         \\
\addlinespace
Firm FE             &         Yes         &         Yes         &         Yes         &         Yes         &         Yes         &         Yes         \\
\addlinespace
Year FE             &         Yes         &          No         &          No         &         Yes         &          No         &          No         \\
\addlinespace
Age FE              &          No         &         Yes         &         Yes         &          No         &         Yes         &         Yes         \\
\addlinespace
Industry-Year FE    &          No         &         Yes         &         Yes         &          No         &         Yes         &         Yes         \\
\addlinespace
State-Year FE       &          No         &         Yes         &         Yes         &          No         &         Yes         &         Yes         \\
\midrule
Clustering          &       gvkey         &       gvkey         &naics4 Statecode         &       gvkey         &       gvkey         &naics4 Statecode         \\
R-squared (adj.)    &        0.26         &        0.21         &        0.21         &        0.27         &        0.22         &        0.22         \\
R-squared (within, adj)&      0.0025         &      0.0023         &      0.0023         &     0.00072         &     0.00070         &     0.00070         \\
Observations        &       59477         &       57948         &       57948         &       59477         &       57948         &       57948         \\
\bottomrule
\multicolumn{7}{l}{\footnotesize Standard errors in parentheses}\\
\multicolumn{7}{l}{\footnotesize \sym{*} \(p<0.1\), \sym{**} \(p<0.05\), \sym{***} \(p<0.01\)}\\
\end{tabular}
}

	\caption{The regressions above relate corporate R\&D, and its interaction with 1-digit NAICS industry as well as an indicator for NCA enforcement, to the  entrepreneurship decisions of employees. The dependent variable is average yearly number of founders joining startups in years $t+1,t+2,t+3$. The independent variables are averages over $t,t-1,t-2$. The independent variables are averages over $t,t-1,t-2$. All LHS and RHS variable (except Tobin's Q) are normalized by a trailing 5 year moving average of assets. Firm controls are employment, assets, intangible assets, investment, net income, cumulative citation-weighted patents, and the product of Tobin's Q. Standard errors are clustered by firm in columns (1)-(3) and (5)-(7). In columns (4) and (8), standard errors are multway clustered by State and 4-digit NAICS industry.}
	\label{table:RDandSpinoutFormation_at_founder2_naics1_hNCA_l3f3}
\end{table}

\begin{table}[!htb]
	\scriptsize
	\centering
	{
\def\sym#1{\ifmmode^{#1}\else\(^{#1}\)\fi}
\begin{tabular}{l*{6}{c}}
\toprule
                    &\multicolumn{1}{c}{(1)}&\multicolumn{1}{c}{(2)}&\multicolumn{1}{c}{(3)}&\multicolumn{1}{c}{(4)}&\multicolumn{1}{c}{(5)}&\multicolumn{1}{c}{(6)}\\
                    &\multicolumn{1}{c}{$\frac{\textrm{Founders}}{\textrm{Assets}}$}&\multicolumn{1}{c}{$\frac{\textrm{Founders}}{\textrm{Assets}}$}&\multicolumn{1}{c}{$\frac{\textrm{Founders}}{\textrm{Assets}}$}&\multicolumn{1}{c}{$\frac{\textrm{WSO4}}{\textrm{Assets}}$}&\multicolumn{1}{c}{$\frac{\textrm{WSO4}}{\textrm{Assets}}$}&\multicolumn{1}{c}{$\frac{\textrm{WSO4}}{\textrm{Assets}}$}\\
\midrule
xrd\_at\_l3\_notIn3or5 &       -0.48         &       -0.53         &       -0.53         &      -0.080         &       0.042         &       0.042         \\
                    &      (0.49)         &      (0.62)         &      (0.50)         &      (0.14)         &      (0.15)         &      (0.15)         \\
\addlinespace
hNCA\_xrd\_at\_l3\_notIn3or5&        0.52         &        0.87         &        0.87         &       0.035         &       -0.13         &       -0.13         \\
                    &      (0.51)         &      (0.73)         &      (0.59)         &     (0.099)         &      (0.15)         &      (0.19)         \\
\addlinespace
xrd\_at\_l3\_industry\_2dig31&       -0.10         &        1.40         &        1.40\sym{**} &      -0.059         &       -0.34         &       -0.34         \\
                    &      (0.47)         &      (1.07)         &      (0.61)         &      (0.15)         &      (0.43)         &      (0.37)         \\
\addlinespace
xrd\_at\_l3\_industry\_2dig32&        0.78         &        0.89         &        0.89\sym{*}  &        1.02         &        1.03         &        1.03         \\
                    &      (0.79)         &      (0.79)         &      (0.46)         &      (0.72)         &      (0.70)         &      (0.62)         \\
\addlinespace
xrd\_at\_l3\_industry\_2dig33&        3.76         &        3.89         &        3.89         &       0.018         &     -0.0035         &     -0.0035         \\
                    &      (3.56)         &      (3.71)         &      (4.00)         &      (0.12)         &      (0.14)         &      (0.26)         \\
\addlinespace
hNCA\_xrd\_at\_l3\_industry\_2dig31&       -0.16         &       -1.78         &       -1.78         &        0.22         &       0.056         &       0.056         \\
                    &      (0.64)         &      (1.82)         &      (1.82)         &      (0.30)         &      (1.03)         &      (1.04)         \\
\addlinespace
hNCA\_xrd\_at\_l3\_industry\_2dig32&       -0.78         &       -0.67         &       -0.67\sym{*}  &       -0.86         &       -0.78         &       -0.78\sym{*}  \\
                    &      (1.42)         &      (1.44)         &      (0.35)         &      (1.26)         &      (1.23)         &      (0.46)         \\
\addlinespace
hNCA\_xrd\_at\_l3\_industry\_2dig33&       -1.37         &       -1.46         &       -1.46         &        1.13         &        1.07         &        1.07         \\
                    &      (4.00)         &      (4.09)         &      (4.87)         &      (1.10)         &      (1.07)         &      (0.99)         \\
\addlinespace
xrd\_at\_l3\_industry\_2dig51&       -1.42         &       -1.56         &       -1.56         &        0.60         &       -0.23         &       -0.23         \\
                    &      (1.38)         &      (1.37)         &      (1.60)         &      (0.55)         &      (0.93)         &      (0.27)         \\
\addlinespace
xrd\_at\_l3\_industry\_2dig52&       -2.27         &       -2.91         &       -2.91\sym{***}&       -0.29         &       -0.25         &       -0.25\sym{*}  \\
                    &      (2.72)         &      (3.56)         &      (1.03)         &      (0.31)         &      (0.28)         &      (0.13)         \\
\addlinespace
xrd\_at\_l3\_industry\_2dig53&     -0.0084         &        0.69         &        0.69         &      -0.074         &       0.067         &       0.067         \\
                    &      (0.34)         &      (0.67)         &      (0.54)         &      (0.15)         &      (0.30)         &      (0.34)         \\
\addlinespace
xrd\_at\_l3\_industry\_2dig54&        0.59         &       -0.16         &       -0.16         &        1.73         &        1.25         &        1.25         \\
                    &      (2.54)         &      (2.88)         &      (2.06)         &      (2.46)         &      (2.75)         &      (1.25)         \\
\addlinespace
xrd\_at\_l3\_industry\_2dig55&           0         &           0         &           0         &           0         &           0         &           0         \\
                    &         (.)         &         (.)         &   (3.4e-09)         &         (.)         &         (.)         &   (1.8e-09)         \\
\addlinespace
xrd\_at\_l3\_industry\_2dig56&       0.058         &        0.45         &        0.45         &      0.0045         &        0.29         &        0.29         \\
                    &      (0.29)         &      (0.42)         &      (0.45)         &      (0.13)         &      (0.30)         &      (0.30)         \\
\addlinespace
hNCA\_xrd\_at\_l3\_industry\_2dig51&        0.27         &        0.21         &        0.21         &       -2.13         &       -1.58         &       -1.58         \\
                    &      (2.44)         &      (2.50)         &      (2.42)         &      (2.00)         &      (2.17)         &      (1.00)         \\
\addlinespace
hNCA\_xrd\_at\_l3\_industry\_2dig52&        1.78         &        2.03         &        2.03\sym{*}  &        0.17         &       0.030         &       0.030         \\
                    &      (2.72)         &      (3.08)         &      (1.08)         &      (0.30)         &      (0.77)         &      (0.86)         \\
\addlinespace
hNCA\_xrd\_at\_l3\_industry\_2dig53&       0.066         &       -0.39         &       -0.39         &       0.053         &        0.29         &        0.29         \\
                    &      (0.29)         &      (0.69)         &      (0.66)         &      (0.11)         &      (0.36)         &      (0.32)         \\
\addlinespace
hNCA\_xrd\_at\_l3\_industry\_2dig54&        8.07         &        8.79         &        8.79\sym{***}&        2.76         &        2.47         &        2.47\sym{*}  \\
                    &      (6.33)         &      (6.01)         &      (3.00)         &      (3.74)         &      (3.91)         &      (1.43)         \\
\addlinespace
hNCA\_xrd\_at\_l3\_industry\_2dig55&           0         &           0         &           0         &           0         &           0         &           0         \\
                    &         (.)         &         (.)         &   (1.4e-10)         &         (.)         &         (.)         &   (2.6e-10)         \\
\addlinespace
hNCA\_xrd\_at\_l3\_industry\_2dig56&       -1.60         &       -2.41         &       -2.41         &       0.061         &        0.51         &        0.51         \\
                    &      (3.28)         &      (3.95)         &      (4.62)         &      (0.13)         &      (0.84)         &      (0.64)         \\
\addlinespace
Firm FE             &         Yes         &         Yes         &         Yes         &         Yes         &         Yes         &         Yes         \\
\addlinespace
Year FE             &         Yes         &          No         &          No         &         Yes         &          No         &          No         \\
\addlinespace
Age FE              &          No         &         Yes         &         Yes         &          No         &         Yes         &         Yes         \\
\addlinespace
Industry-Year FE    &          No         &         Yes         &         Yes         &          No         &         Yes         &         Yes         \\
\addlinespace
State-Year FE       &          No         &         Yes         &         Yes         &          No         &         Yes         &         Yes         \\
\midrule
Clustering          &       gvkey         &       gvkey         &naics4 Statecode         &       gvkey         &       gvkey         &naics4 Statecode         \\
R-squared (adj.)    &        0.27         &        0.22         &        0.21         &        0.27         &        0.22         &        0.22         \\
R-squared (within, adj)&      0.0051         &      0.0047         &      0.0047         &      0.0024         &      0.0021         &      0.0021         \\
Observations        &       59477         &       57948         &       57948         &       59477         &       57948         &       57948         \\
\bottomrule
\multicolumn{7}{l}{\footnotesize Standard errors in parentheses}\\
\multicolumn{7}{l}{\footnotesize \sym{*} \(p<0.1\), \sym{**} \(p<0.05\), \sym{***} \(p<0.01\)}\\
\end{tabular}
}

	\caption{The regressions above relate corporate R\&D, and its interaction with some 2-digit NAICS industries as well as an indicator for NCA enforcement, to the  entrepreneurship decisions of employees. The dependent variable is average yearly number of founders joining startups in years $t+1,t+2,t+3$. The independent variables are averages over $t,t-1,t-2$. The independent variables are averages over $t,t-1,t-2$. All LHS and RHS variable (except Tobin's Q) are normalized by a trailing 5 year moving average of assets. Firm controls are employment, assets, intangible assets, investment, net income, cumulative citation-weighted patents, and the product of Tobin's Q. Standard errors are clustered by firm in columns (1)-(3) and (5)-(7). In columns (4) and (8), standard errors are multway clustered by State and 4-digit NAICS industry.}
	\label{table:RDandSpinoutFormation_at_founder2_naics2_hNCA_l3f3}
\end{table}


\begin{table}[!htb]
	\scriptsize
	\centering
	{
\def\sym#1{\ifmmode^{#1}\else\(^{#1}\)\fi}
\begin{tabular}{l*{6}{c}}
\toprule
                    &\multicolumn{1}{c}{(1)}&\multicolumn{1}{c}{(2)}&\multicolumn{1}{c}{(3)}&\multicolumn{1}{c}{(4)}&\multicolumn{1}{c}{(5)}&\multicolumn{1}{c}{(6)}\\
                    &\multicolumn{1}{c}{Founders}&\multicolumn{1}{c}{Founders}&\multicolumn{1}{c}{Founders}&\multicolumn{1}{c}{WSO4}&\multicolumn{1}{c}{WSO4}&\multicolumn{1}{c}{WSO4}\\
\midrule
hNCA\_xrd\_at\_l3\_ind\_3dig325&       -69.2         &       -65.1         &       -65.1         &       -52.4         &       -64.2         &       -64.2         \\
                    &      (43.4)         &      (50.9)         &      (54.8)         &      (36.9)         &      (40.7)         &      (43.4)         \\
\addlinespace
hNCA\_xrd\_at\_l3\_ind\_3dig333&       582.7         &       789.0\sym{**} &       789.0\sym{***}&       141.9\sym{**} &       183.9\sym{***}&       183.9\sym{***}\\
                    &     (425.5)         &     (306.2)         &     (227.3)         &      (64.0)         &      (69.7)         &      (46.4)         \\
\addlinespace
hNCA\_xrd\_at\_l3\_ind\_3dig334&        52.6         &        97.6         &        97.6         &        14.3         &        15.2         &        15.2         \\
                    &     (181.9)         &     (203.9)         &     (155.1)         &      (54.5)         &      (61.8)         &      (36.1)         \\
\addlinespace
hNCA\_xrd\_at\_l3\_ind\_3dig336&      -378.7         &      -329.0         &      -329.0         &        7.24         &       -4.98         &       -4.98         \\
                    &     (393.9)         &     (350.0)         &     (331.1)         &     (180.1)         &     (185.2)         &      (68.5)         \\
\addlinespace
hNCA\_xrd\_at\_l3\_ind\_3dig511&     -1161.2         &       -18.5         &       -18.5         &      -404.0         &       -65.4         &       -65.4         \\
                    &    (1435.7)         &     (545.9)         &     (233.9)         &     (501.8)         &     (336.4)         &      (88.5)         \\
\addlinespace
hNCA\_xrd\_at\_l3\_ind\_3dig519&      1735.1         &      2426.6\sym{**} &      2426.6\sym{***}&        81.4         &       411.2         &       411.2\sym{**} \\
                    &    (1095.5)         &    (1119.6)         &     (413.7)         &     (190.0)         &     (562.4)         &     (165.8)         \\
\addlinespace
naics3\_selected=0 $\times$ $\frac{\textrm{R\&D}}{\textrm{Assets}}$&        16.7         &       -11.1         &       -11.1         &        29.3\sym{***}&        18.9\sym{*}  &        18.9         \\
                    &      (26.9)         &      (30.0)         &      (35.7)         &      (10.7)         &      (10.3)         &      (17.1)         \\
\addlinespace
naics3\_selected=325 $\times$ $\frac{\textrm{R\&D}}{\textrm{Assets}}$&        29.3         &        41.2         &        41.2         &        43.0\sym{*}  &        68.1\sym{*}  &        68.1\sym{**} \\
                    &      (28.9)         &      (43.2)         &      (38.5)         &      (24.5)         &      (36.8)         &      (30.7)         \\
\addlinespace
naics3\_selected=333 $\times$ $\frac{\textrm{R\&D}}{\textrm{Assets}}$&      -533.8         &      -634.9\sym{**} &      -634.9\sym{***}&      -148.3\sym{**} &      -170.0\sym{**} &      -170.0\sym{***}\\
                    &     (439.0)         &     (295.7)         &     (142.0)         &      (71.4)         &      (67.8)         &      (37.9)         \\
\addlinespace
naics3\_selected=334 $\times$ $\frac{\textrm{R\&D}}{\textrm{Assets}}$&        47.9         &        56.4         &        56.4         &        54.6         &        41.4         &        41.4         \\
                    &     (100.4)         &     (101.8)         &     (127.9)         &      (37.2)         &      (43.8)         &      (40.0)         \\
\addlinespace
naics3\_selected=511 $\times$ $\frac{\textrm{R\&D}}{\textrm{Assets}}$&        15.6         &       -55.6         &       -55.6         &        71.6         &        59.1         &        59.1         \\
                    &     (550.1)         &     (558.4)         &     (328.0)         &     (317.7)         &     (345.1)         &     (138.6)         \\
\addlinespace
naics3\_selected=519 $\times$ $\frac{\textrm{R\&D}}{\textrm{Assets}}$&       -11.4         &     -1422.0         &     -1422.0\sym{***}&      -119.3         &       -95.1         &       -95.1\sym{*}  \\
                    &     (621.9)         &    (1005.3)         &     (140.7)         &     (122.4)         &     (285.2)         &      (50.6)         \\
\addlinespace
Firm FE             &         Yes         &         Yes         &         Yes         &         Yes         &         Yes         &         Yes         \\
\addlinespace
Year FE             &         Yes         &          No         &          No         &         Yes         &          No         &          No         \\
\addlinespace
NAICS3*-Age FE      &          No         &         Yes         &         Yes         &          No         &         Yes         &         Yes         \\
\addlinespace
Industry-Year FE    &          No         &         Yes         &         Yes         &          No         &         Yes         &         Yes         \\
\addlinespace
NAICS3*-State-Year FE&          No         &         Yes         &         Yes         &          No         &         Yes         &         Yes         \\
\midrule
Clustering          &       gvkey         &       gvkey         &naics4 Statecode         &       gvkey         &       gvkey         &naics4 Statecode         \\
R-squared (adj.)    &        0.55         &        0.68         &        0.68         &        0.54         &        0.63         &        0.63         \\
R-squared (within, adj)&       0.017         &      0.0067         &      0.0067         &       0.017         &      0.0091         &      0.0091         \\
Observations        &       59481         &       57145         &       57145         &       59481         &       57145         &       57145         \\
\bottomrule
\multicolumn{7}{l}{\footnotesize Standard errors in parentheses}\\
\multicolumn{7}{l}{\footnotesize \sym{*} \(p<0.1\), \sym{**} \(p<0.05\), \sym{***} \(p<0.01\)}\\
\end{tabular}
}

	\caption{The regressions above relate corporate R\&D, and its interaction with some 3-digit NAICS industries as well as an indicator for NCA enforcement, to the  entrepreneurship decisions of employees. The dependent variable is average yearly number of founders joining startups in years $t+1,t+2,t+3$. The independent variables are averages over $t,t-1,t-2$. The independent variables are averages over $t,t-1,t-2$. All LHS and RHS variable (except Tobin's Q) are normalized by a trailing 5 year moving average of assets. Firm controls are employment, assets, intangible assets, investment, net income, cumulative citation-weighted patents, and the product of Tobin's Q. Standard errors are clustered by firm in columns (1)-(3) and (5)-(7). In columns (4) and (8), standard errors are multway clustered by State and 4-digit NAICS industry.}
	\label{table:RDandSpinoutFormation_at_founder2_naics3_hNCA_l3f3}
\end{table}

\begin{table}[]
	\centering
	\captionof{table}{2-digit NAICS codes summary}\label{}
	\begin{tabular}{rl}
		\toprule \toprule
		2-digit Code & Description \tabularnewline
		\midrule
		11  & Agriculture, Forestry, Fishing and Hunting \tabularnewline
		21  & Mining, Quarrying, and Oil and Gas Extraction\tabularnewline
		22  & Utilities\tabularnewline
		23  & Construction \tabularnewline
		31-33 & Manufacturing \tabularnewline
		42 & Wholesale trade \tabularnewline
		44-45 & Retail trade \tabularnewline
		48-49 & Transportation and warehousing \tabularnewline
		51 & Information \tabularnewline
		52 & Finance and insurance \tabularnewline
		53 & Real estate and Rental and Leasing \tabularnewline
		54 & Professional, Scientific, and Technical Services \tabularnewline
		55 & Management of Companies and Enterprises \tabularnewline
		56 & Administrative, Support, Waste Management, Remediation Service \tabularnewline
		61 & Educational services \tabularnewline
		62 & Health Care and Social Assistance \tabularnewline
		71 & Arts, Entertainment, Recreation \tabularnewline
		72 & Accomodation and Food Services \tabularnewline
		81 & Other Services (ecept public Admin.) \tabularnewline
		92 & Public Administration\tabularnewline
		\bottomrule
	\end{tabular}
\end{table}

\begin{table}[]
	\centering
	\captionof{table}{3-digit breakdown of NAICS 31-33}\label{}
	\begin{tabular}{rl}
		\toprule \toprule
		3-digit code & Description \tabularnewline
		\midrule
		311 & Food \tabularnewline 
		312 & Beverage and tobacco  \tabularnewline 
		313-316 & Textiles, apparel and leather  \tabularnewline
		321-323 & Wood, paper manufacturing and printing \tabularnewline  
		324 & Petroleum and coal products \tabularnewline
		325 & Chemical \tabularnewline
		326 & Plastics and rubber \tabularnewline
		327 & Nonmetallic mineral products \tabularnewline 
		331-332 & Primary and fabricated metal  \tabularnewline
		333 & Machinery \tabularnewline
		334 & Computer and electronic product \tabularnewline
		335 & Electrical equipment \tabularnewline
		336 & Transportation equipment \tabularnewline
		337 & Furniture \tabularnewline
		339 & Misc \tabularnewline
		\bottomrule
	\end{tabular}
\end{table}

\begin{table}[]
	\centering
	\captionof{table}{\textbf{[in progress]} 3-digit breakdown of NAICS 51-56}\label{}
	\begin{tabular}{rl}
		\toprule \toprule
		3-digit code & Description \tabularnewline
		\midrule
		511 & Publishing (non-internet) \tabularnewline 
		512 & Motion picture and sound recording \tabularnewline 
		515 & Broadcasting (non-internet) \tabularnewline
		517 & Telecommunications \tabularnewline  
		518 & Data processing, hosting and related services \tabularnewline
		519 & Other information services \tabularnewline
		 & Plastics and rubber \tabularnewline
		327 & Nonmetallic mineral products \tabularnewline 
		331-332 & Primary and fabricated metal  \tabularnewline
		333 & Machinery \tabularnewline
		334 & Computer and electronic product \tabularnewline
		335 & Electrical equipment \tabularnewline
		336 & Transportation equipment \tabularnewline
		337 & Furniture \tabularnewline
		339 & Misc \tabularnewline
		\bottomrule
	\end{tabular}
\end{table}





\begin{table}[!htb]
	\centering
	\captionof{table}{Alternative calibration}\label{calibration_2_parameters}
	\begin{tabular}{rlll}
		\toprule \toprule
		Parameter & Value & Description & Source \tabularnewline
		\midrule
		$\rho$ & 0.0339 & Discount rate  & Indirect inference \tabularnewline
		$\theta$ & 2 & $\theta^{-1} = $ IES & External calibration 
		\tabularnewline
		$\beta$ & 0.094 & $\beta^{-1} = $ EoS intermediate goods & Exactly identified \tabularnewline 
		$\psi$ & 0.5 & Entrant R\&D elasticity & External calibration \tabularnewline
		$\lambda$ & 1.170 & Quality ladder step size & Indirect inference 
		\tabularnewline
		$\chi$ & 1.80 & Incumbent R\&D productivity & Indirect inference 
		\tabularnewline
		$\hat{\chi}$ & 0.115 & Entrant R\&D productivity & Indirect inference \tabularnewline 
		$\kappa_e$ & 0.737 & Non-R\&D entry cost & Indirect inference \tabularnewline
		$\nu$ & 0.04766 & Spinout generation rate  & Indirect inference\tabularnewline
		$\bar{L}_{RD}$ & 0.05 & R\&D labor allocation  & Normalization \tabularnewline
		\bottomrule
	\end{tabular}
\end{table}



\newpage
\section{Appendix of figures}

\setcounter{figure}{0}
\renewcommand{\thefigure}{\Alph{section}\arabic{figure}}

\begin{figure}[!htb]
	\centering
	\includegraphics[scale=0.95]{../empirics/figures/plots/industry_row_heatmap_naics2_founder2.pdf}
	\caption{Heatmap displaying the distribution of child 2-digit NAICS code (column), conditional on parent NAICS code (row). Darker hues indicate a higher density.}
	\label{figure:industry_row_heatmap_naics2_founder2}
\end{figure}

\begin{figure}[!htb]
	\centering
	\includegraphics[scale=0.95]{../empirics/figures/plots/industry_column_heatmap_naics2_founder2.pdf}
	\caption{Heatmap displaying the distribution of parent 2-digit NAICS code (row), conditional on child NAICS code (column). Darker hues indicate a higher density.}
	\label{figure:industry_column_heatmap_naics2_founder2}
\end{figure}

\begin{figure}[!htb]
	\centering
	\includegraphics[scale=.97]{../empirics/figures/plots/industry_row_heatmap_naics3_founder2.pdf}
	\caption{Heatmap displaying the distribution of child 3-digit NAICS code (column), conditional on parent NAICS code (row). Darker hues indicate a higher density.}
	\label{figure:industry_row_heatmap_naics3_founder2}
\end{figure}

\begin{figure}[!htb]
	\centering
	\includegraphics[scale=.97]{../empirics/figures/plots/industry_column_heatmap_naics3_founder2.pdf}
	\caption{Heatmap displaying the distribution of parent 3-digit NAICS code (row), conditional on child NAICS code (column). Darker hues indicate a higher density.}
	\label{figure:industry_column_heatmap_naics3_founder2}
\end{figure}

\begin{figure}[!htb]
	\centering
	\includegraphics[scale=1]{../empirics/figures/plots/industry_row_heatmap_naics4_founder2.pdf}
	\caption{Heatmap displaying the distribution of child 4-digit NAICS code (column), conditional on parent NAICS code (row). Darker hues indicate a higher density.}
	\label{figure:industry_row_heatmap_naics4_founder2}
\end{figure}

\begin{figure}[!htb]
	\centering
	\includegraphics[scale=1]{../empirics/figures/plots/industry_column_heatmap_naics4_founder2.pdf}
	\caption{Heatmap displaying the distribution of parent 4-digit NAICS code (row), conditional on child NAICS code (column). Darker hues indicate a higher density.}
	\label{figure:industry_column_heatmap_naics4_founder2}
\end{figure}


\begin{figure}[!htb]
	\centering
	\includegraphics[scale= 0.7]{../empirics/figures/scatterPlot_RD-Founders.pdf}
	\caption{Scatterplot of average yearly founder counts in $t+1,t+2,t+3$ versus average yearly R\&D spending in $t,t-1,t-2$.}
	\label{figure:scatterPlot_RD-Founders}
\end{figure}

\begin{figure}[!htb]
	\centering
	\includegraphics[scale= 1.2]{../empirics/figures/plots/RDandSpinoutFormation_speccheck2_levels_reghdfe.pdf}
	\caption{Robustness of founders regression result to different sets of controls, fixed effects, and clustering assumptions. The regressions always control for employment and cumulative patent citations, always include firm and year fixed effects, and always cluster by firm. The above plots show the robustness to controlling for assets, intangible assets, capital expenditures, net income, market value, }
	\label{figure:speccheck2_levels_reghdfe}
\end{figure}

\begin{figure}[!htb]
	\centering
	\includegraphics[scale= 1.2]{../empirics/figures/plots/RDandSpinoutFormation_speccheck2_levels_wso4_reghdfe.pdf}
	\caption{Robustness of WSO4 founders regression result to different sets of controls, fixed effects, and clustering assumptions. The regressions always control for employment and cumulative patent citations, always include firm and year fixed effects, and always cluster by firm. The above plots show the robustness to controlling for assets, intangible assets, capital expenditures, net income, market value, }
	\label{figure:speccheck2_levels_wso4_reghdfe}
\end{figure}

\begin{figure}[!htb]
	\centering
	\includegraphics[scale= 1.2]{../empirics/figures/plots/RDandSpinoutFormation_speccheck2_at_reghdfe.pdf}
	\caption{Robustness of asset-normalized founders regression result to different sets of controls, fixed effects, and clustering assumptions. The regressions always control for employment and cumulative patent citations, always include firm and year fixed effects, and always cluster by firm. The above plots show the robustness to controlling for assets, intangible assets, capital expenditures, net income, market value, }
	\label{figure:speccheck2_at_reghdfe}
\end{figure}

\begin{figure}[!htb]
	\centering
	\includegraphics[scale= 1.2]{../empirics/figures/plots/RDandSpinoutFormation_speccheck2_at_wso4_reghdfe.pdf}
	\caption{Robustness of asset-normalized WSO4 founders regression result to different sets of controls, fixed effects, and clustering assumptions. The regressions always control for employment and cumulative patent citations, always include firm and year fixed effects, and always cluster by firm. The above plots show the robustness to controlling for assets, intangible assets, capital expenditures, net income, market value, }
	\label{figure:speccheck2_at_wso4_reghdfe}
\end{figure}

\begin{figure}[!htb]
	\centering
	\includegraphics[scale= 1.2]{../empirics/figures/plots/RDandSpinoutFormation_speccheck2_levels_ppmlhdfe.pdf}
	\caption{Robustness of founders Poisson pseudo-Maximum Likelihood regression result to different sets of controls, fixed effects, and clustering assumptions. The regressions always control for employment and cumulative patent citations, always include firm and year fixed effects, and always cluster by firm. The above plots show the robustness to controlling for assets, intangible assets, capital expenditures, net income, market value, }
	\label{figure:speccheck2_levels_ppmlhdfe}
\end{figure}

\begin{figure}[!htb]
	\centering
	\includegraphics[scale= 1.2]{../empirics/figures/plots/RDandSpinoutFormation_speccheck2_levels_wso4_ppmlhdfe.pdf}
	\caption{Robustness of WSO4 founders Poisson pseudo-Maximum Likelihood regression result to different sets of controls, fixed effects, and clustering assumptions. The regressions always control for log employment and log cumulative patent citations, always include firm and year fixed effects, and always cluster by firm. The above plots show the robustness to controlling for log assets, log intangible assets, log capital expenditures, log net income, log market value, }
	\label{figure:speccheck2_levels_wso4_ppmlhdfe}
\end{figure}

\begin{figure}[]
	\includegraphics[scale = 0.43]{../code/julia/figures/simpleModel/calibrationSensitivity.pdf}
	\caption{Plot showing the elasticity of parameters to moments. It is computed by inverting the jacobian matrix of the mapping from log parameters to log model moments (whose entries comprise the previous figure). These elasticities, along with estimates of the noisiness of the moments used in the calibration, can be used to estimate confidence intervals for the parameters in the model, and thereby for the welfare comparison in question.}
	\label{calibration_sensitivity}
\end{figure}

\begin{figure}[]
	\includegraphics[scale = 0.43]{../code/julia/figures/simpleModel/identificationSourcesFull.pdf}
	\caption{Plot showing the elasticity of moments to model parameters, including parameters taken from the literature $\theta , \beta, \psi$. These non-calibrated parameters are added in as effective moments to be matched, allowing the sensitivity of calibrated parameters $\rho, \lambda, \chi, \hat{\chi}, \kappa_E, \nu$ to these parameters to be computed by simply inverting this matrix, as before.}
	\label{calibration_identificationSources_full}
\end{figure}

\begin{figure}[!htb]
	\includegraphics[scale = 0.36]{../code/julia/figures/simpleModel/levelsWelfareComparisonSensitivityFull.pdf}
	\caption{Sensitivity of welfare comparison to moments. This is $(J^{-1})^T \nabla_p W$, where $W(p)$ maps log parameters to the percentage change in BGP consumption which is equivalent to the change in welfare from changing $\kappa_c$ from $\infty$ to $0$ (i.e. going from banning to frictionlessly enforcing NCAs). In contrast with the elasticity of the previous figure, this is a semi-elasticity. In particular it can allow for the change in welfare to be negative. The way to read this is the following. Looking at the column labeled \textit{E}, the chart says that a 1\% increase in the targeted employment share of young firms, which corresponds to a log change of about $0.01$, leads to an increase in the \% welfare improvement of approximately $6 \times 0.01 = 0.06$ percentage points.}
	\label{levelsWelfareComparisonSensitivityFull}
\end{figure}

\begin{figure}[]
	\includegraphics[scale = 0.36]{../code/julia/figures/simpleModel/welfareComparisonParameterSensitivityFull.pdf}
	\caption{Sensitivity of welfare comparison to moments. This is $\nabla_p W$, wahere $W(p)$ maps log parameters to the log of the percentage change in BGP consumption which is equivalent to the change in welfare from changing $\kappa_c$ from $\infty$ to $0$ (i.e. going from banning to frictionlessly enforcing NCAs).}
	\label{welfareComparisonParameterSensitivityFull}
\end{figure}

\begin{figure}[]
	\includegraphics[scale = 0.36]{../code/julia/figures/simpleModel/levelsWelfareComparisonParameterSensitivityFull.pdf}
	\caption{Sensitivity of welfare comparison to moments. This is $\nabla_p W$, wahere $W(p)$ maps log parameters to the percentage change in BGP consumption which is equivalent to the change in welfare from changing $\kappa_c$ from $\infty$ to $0$ (i.e. going from banning to frictionlessly enforcing NCAs).}
	\label{levelsWelfareComparisonParameterSensitivityFull}
\end{figure}

\section{Model}\label{appendix:model}

\subsection{Proofs of propositions}

\subsubsection{Proof of Proposition \ref{proposition:hjb_scaling}}\label{appendix:proofs:proposition:hjb_scaling}

\begin{proof}
	 Using the fact that $\hat{\tau}(j,t|q) = \hat{\tau}$ in a symmetric BGP, the incumbent HJB can be written as
	\begin{align}
	(r_t + \hat{\tau}) V(j,t|\bar{q}_{jt}) = \dot{V}(j,t|\bar{q}_{jt}) + \pi(j,t|\bar{q}_{jt}) \label{eq:HJB_solved0}
	\end{align}
	
	This follows immediately when $z = 0$; when $z > 0$, it follows because the term multiplying $z$ in the HJB must be equal to zero, since the FOC holds with equality,
	\begin{align}
	0 &= \chi \big( V(j,t|\lambda q) - V(j,t|q)\big)  \nonumber \\
	&- \big(\frac{q}{Q_t}\big) \Big( w_{RD,jt}(x) + \big(\frac{q}{Q_t}\big)^{-1} (1-x) \nu V(j,t|q) + \big(\frac{q}{Q_t}\big)^{-1}  x \kappa_c \nu V(j,t|q) \Big)
	\end{align} 
	
	Since there is no storage technology, flow consumption $C(t)$ is equal to output minus entry and NCA costs. That is, $C(t) = \tilde{Y}Q_t - (\tau^S + \hat{\tau}) \kappa_e \lambda \tilde{V} Q_t - x \kappa_c \tilde{V}$. This implies that $\frac{\dot{C}}{C} = \frac{\dot{Q}}{Q} = \frac{\dot{Y}}{Y} = g$ on the BGP. Then, the Euler equation (\ref{eq:euler0}) implies that $r_t = r$. Also, the static equilibrium implies that $\pi(j,t|q) = \tilde{\pi} q$. Using these observations in (\ref{eq:HJB_solved0}) yields
	\begin{align}
	(r + \hat{\tau}) V(j,t|\bar{q}_{jt}) &= \dot{V}(j,t|\bar{q}_{jt}) + \tilde{\pi}\bar{q}_{jt} \label{eq:HJB_solved1}
	\end{align}
	
	The ODE (\ref{eq:HJB_solved1}) has both constant and nonconstant solutions. The constant solution is 
	\begin{align}
	V &= \frac{\tilde{\pi} \bar{q}_{jt}}{r + \hat{\tau}} \label{eq:HJB_constant_solution}
	\end{align}
	
	The decreasing solution has 
	\begin{align}
	\dot{V} = (r + \hat{\tau})V - \tilde{\pi}\bar{q}_{jt}
	\end{align}
	
	so that with positive probability $V$ becomes negative. This violates the fact that $\tilde{\pi} q$ is always positive and the incumbent is always free to choose $z = 0$. If $\dot{V} > 0$, then
	\begin{align}
	\frac{\dot{V}}{V} = r + \hat{\tau} - \frac{\tilde{\pi}\bar{q}_{jt}}{V}
	\end{align}
	
	which implies that, absent any innovations, asymptotically $V(j,t|\bar{q}_{jt})$ grows at the exponential rate $r + \hat{\tau}$. Expected OI adds an expected growth rate of $\frac{\tau}{\tau + \hat{\tau} +_\tau^S} g$ and expected creative destruction by entrants subtracts $\hat{\tau}$ from the growth rate. Substituting into (\ref{eq:tvc_incumbent}) and using the fact that $r_t = r$ on the BGP yields
	\begin{align}
	\lim_{t' \to \infty} e^{-r (t' -t)} \mathbb{E}[ \mathbf{1}_{s(j,t) > t'} V(j,t'|\bar{q}_{jt'}) ] = 0 \label{appendix:eq:incumbentTvc0}
	\end{align}
	
	Asymptotically one has $\mathbb{E}[ \mathbf{1}_{s(j,t) > t'} V(j,t'|\bar{q}_{jt'}) ] \ge \mathcal{K} e^{ \big(r + \frac{\tau}{\tau + \hat{\tau} + \tau^S} g \big) (t' - t)}$ for some constant $\mathcal{K}$. Substituting into (\ref{appendix:eq:incumbentTvc0}) yields
	\begin{align}
	\lim_{t' \to \infty} \mathcal{K} \underbrace{e^{-r (t' -t)} e^{ (r + \frac{\tau}{\tau + \hat{\tau} + \tau^S} g ) (t' - t)}}_{\mathclap{e^{\frac{\tau}{\tau + \hat{\tau} + \tau^S}g(t'-t)}}} &= 0 \label{appendix:eq:incumbentTvc1}
	\end{align}
	
	which contradicts
	\begin{align}
		\lim_{t' \to \infty} e^{\frac{\tau}{\tau + \hat{\tau} + \tau^S}g(t'-t)} &= \infty
	\end{align} 
	
	Therefore, the only solution to the incumbent's HJB that is consistent with a symmetric equilibrium is given by (\ref{eq:HJB_constant_solution}).
	
\end{proof}

\subsection{Derivations for efficiency and theoretical policy analysis}

\subsubsection{CD tax (subsidy)}\label{appendix:model:efficiencyderivations:CDtax}

The incumbent HJB is given by 
\begin{align}
(r + \hat{\tau}) \tilde{V} = \tilde{\pi} + \max_{\substack{x \in \{0,1\} \\ z \ge 0}} \Big\{z &\Big( \overbrace{\chi (\lambda - 1) \tilde{V}}^{\mathclap{\mathbb{E}[\textrm{Benefit from R\&D}]}}-  \big( \overbrace{\hat{w}_{RD} - (1-x)(1-(1+T_e)\kappa_e)\lambda \nu \tilde{V}}^{\mathclap{\text{R\&D wage}}}\big) \label{eq:hjb_incumbent_entryTax} \nonumber \\ 
&-  \underbrace{(1-x) \nu \tilde{V}}_{\mathclap{\text{Net cost from spinout formation}}} - \overbrace{x \kappa_{c} \nu \tilde{V}}^{\mathclap{\text{Direct cost of NCA}}}\Big) \Big\} 
\end{align}

The equilibrium conditions not shown in the main text are
\begin{align}
\hat{\tau} &= \hat{\chi} \hat{z}^{1-\psi} \\
z &= \bar{L}_{RD} - \hat{z} \label{eq:labor_resource_constraint_entryTax}\\ 
\tau &= \chi z \\
\tau^S &= (1-x) \nu z \\
g &= (\lambda - 1) (\tau + \tau^S + \hat{\tau}) \\
r &= \theta g + \rho \\
\tilde{V} &= \frac{\tilde{\pi}}{r + \hat{\tau}} \\ 
\hat{w}_{RD} &= \begin{cases}
\hat{\chi} \hat{z}^{-\psi} (1-(1+T_e)\kappa_e) \lambda \tilde{V} &\textrm{, if } \hat{z} > 0\\
\Big( \chi(\lambda -1) - \nu (x\kappa_c + (1-x)\hat{\bar{\kappa}}_c(\kappa_e,\lambda;T_e))\Big) \tilde{V} &\textrm{, o.w.}
\end{cases} \label{eq:wage_rd_labor_entryTax}
\end{align}


\subsubsection{OI R\&D subsidy (tax)}\label{appendix:model:efficiencyderivations:OIRDtax}

The incumbent's HJB is given by
\begin{align}
(r + \hat{\tau}) \tilde{V} = \tilde{\pi} + \max_{\substack{x \in \{0,1\} \\ z \ge 0}} \Big\{z &\Big( \overbrace{\chi (\lambda - 1) \tilde{V}}^{\mathclap{\mathbb{E}[\textrm{Benefit from R\&D}]}}- (\underbrace{1-T_{RD,I}}_{\mathclap{\text{R\&D Subsidy}}}) \big( \overbrace{\hat{w}_{RD} - (1-x)(1-\kappa_e)\lambda \nu \tilde{V}}^{\mathclap{\text{R\&D wage}}}\big) \label{eq:hjb_incumbent_RDsubsidyTargeted} \nonumber \\ 
&-  \underbrace{(1-x) \nu \tilde{V}}_{\mathclap{\text{Net cost from spinout formation}}} - \overbrace{x \kappa_{c} \nu \tilde{V}}^{\mathclap{\text{Direct cost of NCA}}}\Big) \Big\} 
\end{align}

The equilibrium conditions not shown in the main text are
\begin{align}
\hat{\tau} &= \hat{\chi} \hat{z}^{1-\psi} \\
z &= \bar{L}_{RD} - \hat{z} \label{eq:labor_resource_constraint_RDsubsidyTargeted}\\ 
\tau &= \chi z \\
\tau^S &= (1-x) \nu z \\
g &= (\lambda - 1) (\tau + \tau^S + \hat{\tau}) \\
r &= \theta g + \rho \\
\tilde{V} &= \frac{\tilde{\pi}}{r + \hat{\tau}} \\ 
\hat{w}_{RD} &= \hat{\chi} \hat{z}^{-\psi} (1-\kappa_e) \lambda \tilde{V} \label{eq:wage_rd_labor_RDsubsidyTargeted}
\end{align}



\subsubsection{All policies}\label{appendix:model:efficiencyderivations:allPolicies}

The R\&D labor supply indifference condition is
\begin{align}
\hat{w}_{RD} &= w_{RD,j} + (1-x_j) \nu (1-(\underbrace{1+T_e}_{\mathclap{\text{Entry tax}}})\kappa_e) \lambda \tilde{V} \label{eq:RD_worker_indifference_all}
\end{align}

The incumbent HJB is
\begin{align}
(r + \hat{\tau}) \tilde{V} = \tilde{\pi} + \max_{\substack{x \in \{0,1\} \\ z \ge 0}} \Big\{z &\Big( \overbrace{\chi (\lambda - 1) \tilde{V}}^{\mathclap{\mathbb{E}[\textrm{Benefit from R\&D}]}}-  (\underbrace{1 - T_{RD} - T_{RD,I}}_{\mathclap{\text{R\&D subsidies}}})\big( \overbrace{\hat{w}_{RD} - (1-x)(1-(1+T_e)\kappa_e)\lambda \nu \tilde{V}}^{\mathclap{\text{R\&D wage}}}\big) \label{eq:hjb_incumbent_all} \nonumber \\ 
&-  \underbrace{(1-x) \nu \tilde{V}}_{\mathclap{\text{Net cost from spinout formation}}} - \overbrace{x \kappa_{c} \nu \tilde{V}}^{\mathclap{\text{Direct cost of NCA}}}\Big) \Big\} 
\end{align}

which can be rearranged to
\begin{align}
(r + \hat{\tau}) \tilde{V} = \tilde{\pi} + \max_{\substack{x \in \{0,1\} \\ z \ge 0}} \Big\{z &\Big( \overbrace{\chi (\lambda - 1) \tilde{V}}^{\mathclap{\mathbb{E}[\textrm{Benefit from R\&D}]}}- (1-T_{RD}- T_{RD,I})\hat{w}_{RD} \\
&-  \underbrace{(1-x)(1 - (1-T_{RD} - T_{RD,I})(1-(1+T_e)\kappa_{e})\lambda)\nu \tilde{V}}_{\mathclap{\text{Net cost from spinout formation}}} - \overbrace{x \kappa_{c}\nu \tilde{V}}^{\mathclap{\text{Direct cost of NCA}}}\Big) \Big\} \label{eq:hjb_incumbent_all_2}
\end{align}

Define
\begin{align}
\bar{\bar{\kappa}}_c(\kappa_e,\lambda;T_{RD},T_{RD,I},T_e) = 1 - (1-T_{RD} - T_{RD,I})(1-(1+T_e)\kappa_e)\lambda  \label{eq:barkappa_all}
\end{align} 

If $z > 0$, the incumbent's optimal NCA policy is given by 
\begin{align}
x = \begin{cases}
1 & \textrm{if } \kappa_c < \bar{\bar{\kappa}}_c (\kappa_e, \lambda;T_{RD},T_{RD,I},T_e)\\
0 & \textrm{if } \kappa_c > \bar{\bar{\kappa}}_c (\kappa_e, \lambda;T_{RD},T_{RD,I},T_e)\\
\{0,1\} & \textrm{if } \kappa_c = \bar{\bar{\kappa}}_c (\kappa_e, \lambda;T_{RD},T_{RD,I},T_e)
\end{cases} \label{eq:nca_policy_all}
\end{align}


By the usual argument, $z > 0$ implies that the incumbent's FOC can be rearranged to
\begin{align}
\tilde{V} &= \frac{(1-T_{RD} - T_{RD,I})\hat{w}_{RD}}{\chi(\lambda -1) - \nu (x\kappa_c + (1-x)(1 - (1-T_{RD} - T_{RD,I})(1-(1+T_e)\kappa_e)\lambda)) } \label{eq:hjb_incumbent_foc_all}
\end{align}

The free entry condition is
\begin{align}
\underbrace{\hat{\chi} \hat{z}^{-\psi}}_{\mathclap{\text{Marginal innovation rate}}} \overbrace{(1-(1+T_e)\kappa_e) \lambda \tilde{V}}^{\mathclap{\text{Payoff from innovation}}} &= (1-T_{RD})\underbrace{\hat{w}_{RD}}_{\mathclap{\text{MC of R\&D}}} \label{eq:free_entry_condition_all}
\end{align}

Substituting (\ref{eq:hjb_incumbent_foc_all}) into (\ref{eq:free_entry_condition_all}) to eliminate $\tilde{V}$ yields an expression for $\hat{z}$, 
\begin{align}
\hat{z} &= \Bigg( \frac{\Big(\frac{1-T_{RD} -T_{RD,I}}{1-T_{RD}} \Big)\hat{\chi} (1-(1+T_e)\kappa_{e}) \lambda}{\chi(\lambda -1) - \nu (x\kappa_c  + (1-x)(1 - (1-T_{RD} - T_{RD,I})(1-(1+T_e)\kappa_e)\lambda)) } \Bigg)^{1/\psi} \label{eq:effort_entrant_all}
\end{align}

From here, the rest of the model can be solved using
\begin{align}
\hat{\tau} &= \hat{\chi} \hat{z}^{1-\psi} \\
z &= \bar{L}_{RD} - \hat{z} \label{eq:labor_resource_constraint_all}\\ 
\tau &= \chi z \\
\tau^S &= (1-x) \nu z \\
g &= (\lambda - 1) (\tau + \tau^S + \hat{\tau}) \\
r &= \theta g + \rho \\
\tilde{V} &= \frac{\tilde{\pi}}{r + \hat{\tau}} \\ 
\hat{w}_{RD} &= \begin{cases}
(1-T_{RD})^{-1}\hat{\chi} \hat{z}^{-\psi} (1-(1+T_e)\kappa_e) \lambda \tilde{V} &\textrm{, if } \hat{z} > 0\\
\Big( \chi(\lambda -1) - \nu (x\kappa_c + (1-x)\bar{\bar{\kappa}}_c)\Big) \tilde{V} &\textrm{, o.w.}
\end{cases} \label{eq:wage_rd_labor_all}
\end{align}



\subsection{DRS incumbent innovation technology}

In this section I show why it is analytically convenient to have CRS innovation on the incumbent. Without it, the model must be solved numerically.

\begin{proposition}
	Suppose $z$ units of R\&D yields a Poisson rate
	\begin{align}
	\chi z^{1-\psi}  
	\end{align}
	for the incumbents and $\hat{z}$ units of R\&D yields a Poisson rate 
	\begin{align}
	\hat{\chi}\hat{z}^{1-\hat{\psi}}
	\end{align}
	for entrants.\footnote{Note that I have switched the notation so that $\psi$ with no hat refers to incumbents, so that it is consistent with the convention that hats go on variables related to entrants.}
	
	Consider $\psi \in [0,1)$. If $\psi = 0$, we have the baseline model, which admits a closed form solution. If $\psi = 0.5$, then $\tilde{V}$ has a closed form solution given parameters and $\hat{w}_{RD}$, but the model itself does not have a closed-form solution. For all other $\psi \in [0,1)$, there is no closed form solution for $\tilde{V}$ or the equilibrium given $\tilde{V}$.  
\end{proposition}

\begin{proof}
	The normalized incumbent HJB is now
	\begin{align}
	(r + \hat{\tau}) \tilde{V} &= \tilde{\pi} + \max_{\substack{x \in \{0,1\} \\ z \ge 0}} \Big\{  z \Big( z^{-\psi} \chi (\lambda - 1) \tilde{V} - \hat{w}_{RD} - (1-x)(1 - (1-\kappa_e) \lambda)\nu \tilde{V} - x \kappa_c \nu \tilde{V}  \Big)   \Big\} \label{appendix:model:drsincumbent:hjb}
	\end{align} 
	
	The first order condition is
	\begin{align}
	(1-\psi) z^{-\psi} \chi (\lambda -1)\tilde{V} &= \hat{w}_{RD} + (1-x) (1-(1-\kappa_e)\lambda)\nu \tilde{V} + x \kappa_c \nu \tilde{V}
	\end{align}
	
	which implies
	\begin{align}
	z^{1-\psi} &= \Big( \frac{\hat{w}_{RD} + \zeta_1\tilde{V}}{C_2\tilde{V}} \Big)^{\frac{\psi -1}{\psi}} \\
	\zeta_1 &= (1-x)(1-(1-\kappa_e)\lambda)\nu + x\kappa_c\nu \\
	\zeta_2 &= (1-\psi) \chi (\lambda-1)
	\end{align}
	
	Substituting into (\ref{appendix:model:drsincumbent:hjb}) yields
	\begin{align}
	(r + \hat{\tau}) \tilde{V} &= \tilde{\pi} + \Big( \frac{\hat{w}_{RD} + \zeta_1\tilde{V}}{C_2\tilde{V}} \Big)^{\frac{\psi -1}{\psi}} \zeta_2 \tilde{V} - \hat{w}_{RD} - \zeta_1 \tilde{V} 
	\end{align}
	
	This equation has no closed form expression for $\tilde{V}$ except in the quadratic cost case of $\psi = 0.5$, when $\frac{\psi - 1}{\psi} = -1$ and the above becomes
	\begin{align}
	(r + \hat{\tau}) \tilde{V} &= \tilde{\pi} +  \frac{1}{\hat{w}_{RD} + \zeta_1\tilde{V}} - \hat{w}_{RD} - \zeta_1 \tilde{V} 
	\end{align}
	
	Multiplying both sides by $\hat{w}_{RD} + \zeta_1 \tilde{V}$ yields a quadratic equation for $\tilde{V}$, which has solution
	\begin{align}
		\tilde{V} &= \frac{-b \pm \sqrt{b^2 - 4ac}}{2a}
	\end{align}
	
	However, the dependence of $\tilde{V}$ on $\hat{w}_{RD}$, given model parameters, is no longer linear. This means that $\hat{z}$ and $\hat{w}_{RD}$ must be solved in a simultaneous system which does not have a 
	
\end{proof}

\subsection{Symmetric equilibria without $x_{jt} = x$}\label{appendix:model:proofs:proposition:mixedstrategyeq}

Such a BGP is significantly more complicated to construct than the case with $x_{jt} = x$. The reason is that BGP requires that the endogenous variable $\mathbb{E}[\tilde{q}_j | x_j = 1]$ remain constant in equilibrium in order for the mapping from individual policies to the aggregate growth rate to be constant -- which is a requirement for BGP given that individually optimal policies are constant. 

In the construction of this proposition, I achieve this by positing that a constant fraction $f \in (0,1)$ of entering firms choose $x_j = 1$ and keep this choice $x_j$ throughout the entire life of the firm. The resulting problem is significantly simplified by the fact that $z_j$ does not depend on $x_j$. Still, there is some additional complexity coming from two main sources. Comparing BGPs, if $f$ increases, so that entering firms are more likely to choose $x_j = 1$, then $\mathbb{E}[\tilde{q}_j | x_j = 1]$ increases as well because new firms are of higher average quality than incumbents. And, for a given $f$, $x_j = 1$ firms tend to get replaced less often and hence $\mathbb{E}[\tilde{q}_j | x_j = 1]$ is lower as it puts more weight on older, hence lower quality, incumbents. That is, if $f = 1/2$ then $\mathbb{E}[\tilde{q}_j | x_j = 1]$ < $\mathbb{E}[\tilde{q}_j | x_j = 0]$ \textbf{[prove]}. \footnote{An alternative which sidesteps this issue is to use a Cobb-Douglas aggregator. This requires tracking limit pricing, but has the advantage that the relevant measure of aggregate quality is average of log quality, which means proportional improvements have a given effect on the aggregate regardless of the quality on which they improve.}

With the preceding discussion in mind, below I give a proof of Proposition \ref{proposition:mixedstrategyeq}.

\begin{proof}
	The proof makes concrete the argument in the preceding paragraph. Relative to the baseline model, the only substantial modification is that one needs to derive an expression for the evolution over time of $\mathbb{E}[\tilde{q}_j | x_j = \mathbf{x}]$ for $\mathbf{x} \in \{0,1\}$, and set it equal to zero. This expression will involve the fraction $f \in (0,1)$. 
	
	It will be more convenient to work with the object $\mathbb{E}_t[\bar{q}_{jt} | x_j = \mathbf{x}]$. To relieve cumbersome notation, denote $\Gamma^{\mathbf{x}} = \mathbb{E}[\bar{q}_j | x_j = \mathbf{x}]$ and let $m^{\mathbf{x}}$ denote the mass of goods $j$ which have $x_j = \mathbf{x}$. In general, $m^0,m^1$ satisfy
	\begin{align}
		\dot{m}^0 &= -(\hat{\tau} + \nu z) m^0 + (\tau + \hat{\tau} + m^0 \nu z) (1-f)\\
		\dot{m}^1 &= -\hat{\tau} m^1 + (\tau + \hat{\tau} + m^0 \nu z) f
	\end{align}
	
	On the BGP, $\dot{m}^{\mathbf{x}} = 0$. This implies
	\begin{align}
		m^0 &= \label{appendix:model:mixedstrategyeq:m0}\\
		m^1 &= \label{appendix:model:mixedstrategyeq:m1}
	\end{align}
	
	Next, $m^0\Gamma^0,m^1\Gamma^1$ have their own evolution equations,
	\begin{align}
		m_{t+\Delta}^0\Gamma^0_{t+\Delta} &= (1 - (\hat{\tau} + \nu z) \Delta ) m^0_t  \Gamma_t^0 + \tau \Delta (\lambda - 1)m_t^0 \Gamma_t^0 + (1-f) \lambda \Big( (\hat{\tau} + \nu z) \Delta m_t^0  \Gamma_t^0 +  \hat{\tau} \Delta m_t^1  \Gamma_t^1 \Big) + o(\Delta)\\
		m_{t+\Delta}^1 \Gamma^1_{t+\Delta} &= (1 - \hat{\tau} \Delta ) m^1_t  \Gamma_t^1  + \tau \Delta (\lambda -1 ) m_t^1 \Gamma_t^1  + f \lambda \Big( (\hat{\tau} + \nu z) \Delta m_t^0  \Gamma_t^0 + \hat{\tau} \Delta m_t^1  \Gamma_t^1 \Big) + o(\Delta) 
	\end{align}
	
	Subtracting $m_t^{\mathbf{x}} \Gamma_t^{\mathbf{x}}$, dividing by $\Delta$ and taking the limit as $\Delta \to 0$, and finally using the fact that $m_t^{\mathbf{x}} = m_{t+\Delta}^{\mathbf{x}} = m^{\mathbf{x}}$, yields
	\begin{align}
		m^0 \dot{\Gamma}_t^0 &= -(\hat{\tau} + \nu z) m^0 \Gamma_t^0 + \tau (\lambda - 1) m^0 \Gamma_t^0 + (1-f)\lambda \Big( (\hat{\tau} + \nu z) m^0 \Gamma_t^0 + \hat{\tau} m^1 \Gamma_t^1 \Big) \\
		m^1 \dot{\Gamma}_t^1 &= -\hat{\tau} m^1 \Gamma_t^1 + \tau (\lambda - 1) m^1 \Gamma_t^1 + f\lambda \Big( (\hat{\tau} + \nu z) m^0 \Gamma_t^0 + \hat{\tau} m^1 \Gamma_t^1 \Big)
	\end{align}
	
	Dividing by $m^x \Gamma_t^x$ yields
	\begin{align}
	\frac{\dot{\Gamma}_t^0}{\Gamma_t^0} &= -( \hat{\tau} + \nu z) + \tau (\lambda - 1) + (1-f)\lambda \Big( (\hat{\tau} + \nu z) + \hat{\tau} \frac{m^1 \Gamma_t^1 }{m^0 \Gamma_t^0}\Big) \\
	\frac{\dot{\Gamma}_t^1}{\Gamma_t^1} &= -\hat{\tau}  + \tau (\lambda - 1) + f\lambda \Big( (\hat{\tau} + \nu z) \big(\frac{m^1 \Gamma_t^1}{m^0 \Gamma_t^0}\big)^{-1} + \hat{\tau}  \Big)
	\end{align}
	
	Setting $\frac{\dot{\Gamma}_t^0}{\Gamma_t^0} = \frac{\dot{\Gamma}_t^1}{\Gamma_t^1}$ and multiplying both sides by $\frac{m^1 \Gamma_t^1}{m^0 \Gamma_t^0}$ yields a quadratic equation in $\frac{m^1 \Gamma_t^1}{m^0 \Gamma_t^0}$,
	\begin{align}
		content...
	\end{align}
	
	[\textbf{Prove that the equation has only one positive solution}] The BGP value of $\frac{\gamma_t^1}{\gamma_t^0}$ is determined by the unique positive solution to this quadratic equation. 
	
	The rest of the equilibrium is computed as before except with a modification to the aggregation. The same argument as before show that equation (\ref{eq:effort_entrant}) holds for $\hat{z}$. Market clearing then implies $z = \bar{L}_{RD} - \hat{z}$. The growth accounting equation is different, given by
	\begin{align}
		g &= (\lambda - 1) \Big( (\tau + \hat{\tau} + \nu z) \gamma^0 m^0 + (\tau + \hat{\tau}) \gamma^1 m^1) 
	\end{align}
	where $\gamma^{\mathbf{x}} = \mathbb{E}[\tilde{q}_{jt} | x_{jt} = \mathbf{x}]$.
	
	The rest of the equilibrium is computed as before. The growth accounting equation is derived from
	\begin{align}
		content...
	\end{align}
\end{proof}

\subsubsection{Application to model with static heterogeneity}

The above construction and derivation can be adapted to a more complicated model where there is heterogeneity in $\{\kappa_e, \kappa_c, \nu\}$ across goods $j$, inducing heterogeneity in chosen $\{z_j,x_j\}$. A tractable BGP in this setup only requires two things: that the state variable be constant throughout the life of the incumbent and that the Markov process by which goods $j$ move between states satisfy a monotone mixing condition. The former ensures that the incumbent's HJB has no additional state variables. The latter condition essentially requires that there be no "absorbing subset" of states. This is similar to the standard necessary conditions for the existence of a stationary equilibrium in models with heterogeneous agents.

The density $\mu(x)$ can be derived using the Kolmogorov forward equation,
\begin{align}
0 &= 
\end{align}
The system of difference equations for $m^{\textbf{x}} \Gamma_t^{\textbf{x}}$ are replaced by a functional difference equation, 
\begin{align}
\mu(x) \Gamma_{t+\Delta}^x &= (1- \text{CD}(x) \Delta) \mu(x) \Gamma_t^x + \text{OI}(x) \Delta (\lambda -1) \mu(x) \Gamma_t^x +  j^x \Delta  \lambda \int_{x' \in \mathbf{X}} \text{CD}(x') \Gamma_t^{x'} \mu(x') dx'
\end{align}
where $j^x$ is the injection rate into state $x$ out of new incumbents. This can then be used to derive a functional differential equation,
\begin{align}
\frac{\dot{\Gamma}_{t}^x}{\Gamma_t^x} &= OI(x) (\lambda -1) - CD(x) + j^x \lambda (\mu(x) \Gamma_t^x)^{-1} \int_{x' \in \mathbf{X}} \text{CD}(x') \Gamma_t^{x'} \mu(x') dx'
\end{align}

Imposing the condition $\frac{\dot{\Gamma}_{t}^x}{\Gamma_t^x} = g$ for an unknown constant $g$ pins down the ratio $\frac{\int_{x' \in \mathbf{X}} \text{CD}(x') \Gamma_t^{x'} \mu(x') dx'}{\mu(x) \Gamma_t^x}$ for each $x$, determining the shape of the distribution $\Gamma_t^x$ (since $\mu(x)$ is already determined by the KF equation). If the relevant functions are differentiable, the condition can also be derived by differentiating the expression for $\frac{\dot{\Gamma}_t^x}{\Gamma_t^x}$ with respect to $x$ and setting it equal to zero. This yields
\begin{align}
0 = \text{OI}'(x) (\lambda -1) - \text{CD}'(x) + \lambda \int_{x' \in \mathbf{X}} \text{CD}(x') \Gamma_t^{x'} \mu(x') dx' \Big(\frac{d}{dx} j^x \mu(x) \Gamma_t^x \Big)^{-1}
\end{align} 

where one would need to expand the last derivative using the product rule (recalling that all three terms depend on $x$). 

The scale of the distribution at time $t$ is determined by 
\begin{align}
\int_{x' \in \mathbf{X}} \Gamma_t^{x'} \mu(x') dx' = Q_t
\end{align}




\section{Calibration}\label{appendix:calibration}

\subsection{Computing model moments}

\subsubsection{R\&D / GDP}\label{appendix:calibration:rd/gdp}

In the model, the R\&D share is the ratio of the wage paid to R\&D workers to GDP. This is
\begin{align*}
\frac{\textrm{R\&D wage bill}}{\textrm{GDP}} &= \frac{w_{RD} z + \hat{w}_{RD} \hat{z}}{\tilde{Y}} \\ 
&= \frac{\hat{w}_{RD} (z + \hat{z}) + (w_{RD} - \hat{w}_{RD})z}{\tilde{Y}} \\
&= \frac{\hat{w}_{RD} (z + \hat{z}) - (1-\kappa_e) \lambda \tilde{V} \tau^S}{\tilde{Y}}
\end{align*}

where I used $w_{RD} - \hat{w}_{RD} = -(1-x)(1-\kappa_e) \lambda \tilde{V} \nu$ and $\tau^S = (1-x)\nu z$. 

\subsubsection{Growth share OI}\label{appendix:calibration:growthShareOI}

The model moment that corresponds here is the share of growth due to own innovation by incumbents of age >= 6. In the model, the fraction of OI growth due to incumbents in a given age group is exactly their fraction of employment: innovations arrive at the same rate for each incumbent, and their impact on aggregate growth is proportional to the incumbent's relative quality, which is proportional to employment. Hence old incumbents' share of growth due to own innovation is simply one minus the employment share calculated in the previous paragraph, $e^{((\hat{\tau}_I -1)g - (\hat{\tau} + (1-x)z \nu))\cdot 6}$. Finally, the fraction of aggregate growth due to OI is $\hat{\tau}_i$, defined above. The fraction of growth due to incumbents of age at least 6 is the product of the two, 
\begin{align*}
\textrm{Age >= 6 share of OI} &= \hat{\tau}_I \frac{\ell(6)}{\ell(0)} \\
&= \hat{\tau}_I e^{((\hat{\tau}_I -1)g - (\hat{\tau} + (1-x)z \nu))\cdot 6} 
\end{align*}


\subsubsection{Entry rate}\label{appendix:calibration:entryRate}

Let $\ell(a)$ denote the density of incumbent employment at age $a$ incumbents. Then $\ell(a)$ is characterized by 
\begin{align*}
\ell(a) &= \ell(0)e^{((\hat{\tau}_I -1)g - (\hat{\tau} + (1-x)z \nu))a}  \\
1 + \bar{L}_{RD} - \hat{z} &= \int_0^{\infty} \ell(a) da
\end{align*}

where $\hat{\tau}_I = \frac{\tau}{\tau + \hat{\tau} + \tau^S}$ is the fraction of innovations that are incumbents' own innovations. 

The intuition for this characterization of $\ell(a)$ has two parts. First, because all shocks are \textit{iid} across firms in equilibrium, the law of large numbers applied to each cohort of firms implies that we can consider directly the evolution of the cohort as a whole instead of explicitly analyzing the dynamics each individual firm in the cohort.  Second, the employment of a firm is proportional to its relative quality, $l_j \propto \tilde{q}_j = q_j / Q$, as long as it is the leader. When it is no longer the leader, its employment is zero forever. Putting these two together, $\ell(a)$ must decline at exponential rate $g$ due to the increase in $Q_t$ (obsolescence), increase at rate $\hat{\tau}_I g$ due to incumbents own innovations, and decline at rate $\hat{\tau} + \tau^S$ due to creative destruction.\footnote{The second equation imposes consistency with aggregate employment; it implies $\ell(0) = -((\hat{\tau}_I -1)g - (\hat{\tau} + \tau^S))(1 + \bar{L}_{RD})$. The calibration does not require this explicit calculation since it is based only on employment shares.} Note that the employment density is strictly decreasing in $a$. This is because there are no adjustment costs: firms achieve their optimal scale immediately upon entry, and subsequently become obsolete (on average) or lose the innovation race to an entrant. Finally, due to the constant exponential decay of $\ell(a)$, the share of incumbent employment in incumbents of strictly less than 6 years of age is given by 
\begin{align*}
\Xi_{[0,6)} &=  1 - \frac{\ell(6)}{\ell(0)} \\
&= 1 - e^{((\hat{\tau}_I -1)g - (\hat{\tau} + \tau^S))\cdot 6}
\end{align*}  


The share of overall employment in incumbents of age < 6, including R\&D performed by non-producing entrants, is equal to the previously calculated $\Xi_{[0,6)}$, multiplied by the share of total labor in incumbents, $1 - \hat{z}$, added to the R\&D labor used by entrants $\hat{z}$, 
\begin{align*}
\textrm{Age < 6 share of employment} &= \frac{2}{3}(\Xi_{[0,6)} (1-\hat{z}) + \hat{z})
\end{align*}

The factor 2/3 follows from interpreting entrants in the model as either new firms or incumbents engaging in creative destruction. According to KH 2020, creative destruction by incumbents is responsible for half as much growth as creative destruction by entrants. In this interpretation of the model, both types of creative destruction use the same technology. Therefore, 2/3 of employment in young firms in the model represents employment in young firms in the data.\footnote{Also, note that this formula extrapolates the employment-age distribution to the entire economy and calculated shares in that way. If I did not do this, the formula would be
	\begin{align*}
	\textrm{Age < 6 share of employment} &= \frac{2}{3} \frac{(\Xi_{[0,6)} (1 - L_F -\hat{z}) + \hat{z})}{1-L_F}
	\end{align*}
	
	This has only minor effects on the inferred parameters. They are listed in \autoref{calibration_2_parameters}.}

\subsubsection{Employment share of WSOs}\label{appendix:calibration:WSOempShare}

Because entering spinouts and entering ordinary firms have identical life-cycles post entry in expectation, the BGP share of employment in firms started as spinouts is their share of new incumbents $\frac{\tau^S}{\tau^S+ (\frac{2}{3})\hat{\tau}}$, multiplied by the employment share of incumbents $1- (\frac{2}{3})\hat{z}$, 
\begin{align*}
\textrm{Spinout employment share} &= \frac{\tau^S}{\tau^S + \frac{2}{3}\hat{\tau}} (1 - \frac{2}{3}\hat{z}) 
\end{align*}

Again, the factor 2/3 is because this is the fraction of entrants in the model which the calibration maps to new firms in the data.


\section{Policy analysis}

\subsection{NCA cost $\kappa_c$}\label{appendix:policyanalysis:ncacost}

\paragraph{Robustness of welfare gain from NCA enforcement}

\autoref{welfareComparisonSensitivityFull} shows the sensitivity of the welfare comparison the moments targeted, including the externally calibrated parameters as pseudo-moments as before. It is computed as $\nabla_m \tilde{W}|_m = (J^{-1})^T \nabla_p W|_p$, where $J$ is the Jacobian of the mapping from log parameters to moments (so that $J^{-1}$ is the Jacobian of the inverse mapping), and $W$ is the mapping from parameters the log \% change (or raw \% change, in \autoref{levelsWelfareComparisonSensitivityFull})) in CEV welfare from reducing $\kappa_c$ from $\infty$ to $0$. That is, it is the gradient of the change in welfare to the change in target moments or uncalibrated parameters, taking as given the change in parameters required to continue matching the target moments. For reference, $\nabla_p W|_p$  for each definition of $W$ can be found in \autoref{welfareComparisonParameterSensitivityFull} and \autoref{levelsWelfareComparisonParameterSensitivityFull}.

Suppose that the log of each moment is assumed to have a standard deviation of $\sigma = 0.05$, and that this uncertainty is statistically independent across moments. The uncertainty propagates such that the standard deviation of the CEV welfare change is the square root of $(\nabla_m \tilde{W}|_m)^T \Sigma_m \nabla_m \tilde{W}|_m$, where $\Sigma_m = \sigma^2 I_{9\times 9}$. In this examples this yields 0.31 log points (0.45 percentage points). Also, in both cases the result is linear in $\sigma$. Hence with $\sigma = 0.1$, the result is 0.62 log points (0.90 percentage points), etc. 

The estimated welfare improvement is about 1.4\%. Taking the uncertainty into account, the $2\sigma$ uncertainty region excludes zero for $\sigma \le .112$. This suggests the result is quite sensitive to the moments and non-calibrated parameters used in the calibration. 


\begin{figure}[]
	\includegraphics[scale = 0.36]{../code/julia/figures/simpleModel/welfareComparisonSensitivityFull.pdf}
	\caption{Sensitivity of welfare comparison to moments. This is $(J^{-1})^T \nabla_p W$, where $W(p)$ maps log parameters to the log of the percentage change in BGP consumption which is equivalent to the change in welfare from changing $\kappa_c$ from $\infty$ to $0$ (i.e. going from banning to frictionlessly enforcing NCAs). The way to read this is the following. Looking at the column labeled \textit{E}, the chart says that a 1\% increase in the targeted employment share of young firms, which corresponds to a log change of about $0.01$, leads to a 4\% increase in the percentage CEV percentage welfare change. In this calibration it is about 1.42\%, so this is about $0.057$ percentage points.}
	\label{welfareComparisonSensitivityFull}
\end{figure}


\paragraph{When are NCAs bad for welfare?}

The sensitivity of the welfare improvement to the entry rate shown in (\ref{welfareComparisonSensitivityFull}) suggests that a calibration targeting a lower rate of creative destruction could have the opposite result. \autoref{calibration_lowEntry_summaryPlot} shows the analogue of \autoref{calibration_summaryPlot} if entry rate targeted is 4\% instead of 8.35\%. The model is again able to match the moments exactly; inferred parameter values are shown in \autoref{calibration_lowEntry_parameters}.

As expected, growth and welfare fall when $\kappa_C$ is reduced so that $x = 1$. Mathematically, this results from the much higher inferred value of $\lambda$: it is 1.65 in this calibration instead of 1.17 in the original calibration. Intuitively, the lower rate of entry means that each entry must have a higher effect on growth in order for the model to match the growth rate. Furthermore, as shown in \autoref{welfareComparisonParameterSensitivityFull}, the increase in $\lambda$ significantly reduces the welfare gain from reducing $\kappa_C$. A higher value of $\lambda$ weakens inequality (\ref{cs:growth_decreasing_condition}), reducing the effect of the incumbent disincentive on growth. While the inequality still holds, the extent of misallocation is weaker. Therefore, inducing further misallocation via an increase in $\kappa_C$ has a smaller negative effect on growth. The present exercise shows that these local relationships are to some extent global and in fact strong enough to switch the sign of the welfare comparison.

\begin{figure}[]
	\includegraphics[scale = 0.57]{../code/julia/figures/simpleModel/calibration_lowEntry_summaryPlot.pdf}
	\caption{Summary of equilibrium for baseline parameter values and various values of $\kappa_c$.}
	\label{calibration_lowEntry_summaryPlot}
\end{figure}

\begin{table}[]
	\centering
	\captionof{table}{Low entry rate calibration}\label{calibration_lowEntry_parameters}
	\begin{tabular}{rlll}
		\toprule \toprule
		Parameter & Value & Description & Source \tabularnewline
		\midrule
		$\rho$ & 0.0339 & Discount rate  & Indirect inference \tabularnewline
		$\theta$ & 2 & $\theta^{-1} = $ IES & External calibration 
		\tabularnewline
		$\beta$ & 0.094 & $\beta^{-1} = $ EoS intermediate goods & Exactly identified \tabularnewline 
		$\psi$ & 0.5 & Entrant R\&D elasticity & External calibration \tabularnewline
		$\lambda$ & 1.65 & Quality ladder step size & Indirect inference 
		\tabularnewline
		$\chi$ & 0.366 & Incumbent R\&D productivity & Indirect inference 
		\tabularnewline
		$\hat{\chi}$ & 0.0474 & Entrant R\&D productivity & Indirect inference \tabularnewline 
		$\kappa_e$ & 0.703 & Non-R\&D entry cost & Indirect inference \tabularnewline
		$\nu$ & 0.0126 & Spinout generation rate  & Indirect inference\tabularnewline
		$\bar{L}_{RD}$ & 0.05 & R\&D labor allocation  & Normalization \tabularnewline
		\bottomrule
	\end{tabular}
\end{table}









\end{document}