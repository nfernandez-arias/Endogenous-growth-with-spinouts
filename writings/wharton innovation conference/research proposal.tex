\documentclass[11pt,english]{article}
%\usepackage{lmodern}
\linespread{1.05}
\usepackage{mathpazo}
%\usepackage{mathptmx}
%\usepackage{utopia}
\usepackage{microtype}
\usepackage[T1]{fontenc}
\usepackage[latin9]{inputenc}
\usepackage[dvipsnames]{xcolor}
\usepackage{geometry}
\usepackage{amsthm}
\usepackage{amsfonts}

\usepackage{courier}
\usepackage{verbatim}
\usepackage[round]{natbib}
\bibliographystyle{plainnat}


\definecolor{red1}{RGB}{128,0,0}
%\geometry{verbose,tmargin=1.25in,bmargin=1.25in,lmargin=1.25in,rmargin=1.25in}
\geometry{verbose,tmargin=1in,bmargin=1in,lmargin=1in,rmargin=1in}
\usepackage{setspace}

\usepackage[colorlinks=true, linkcolor={red!70!black}, citecolor={blue!50!black}, urlcolor={blue!80!black}]{hyperref}
%\usepackage{esint}
%\onehalfspacing
\usepackage{babel}
\usepackage{amsmath}
\usepackage{graphicx}

\theoremstyle{remark}
\newtheorem{remark}{Remark}
\begin{document}
	
	
	
\title{Research Proposal for "Innovation and Growth with Creative Destruction by Employee Spinouts"}
%\author{Nicolas Fernandez-Arias}
\maketitle

\section{Introduction}

Silicon Valley (SV) is often considered a paragon of economic dynamism and innovation. And not without reason: labor productivity growth in the MSA containing SV averaged 2.72\% from 1978 to 2015, compared to an average of about 2\% for the whole country, implying a 30\% difference in cumulative growth. The leading theory holds that \textit{employee spinouts} -- firms founded by former employees --  played a crucial role.\footnote{See \cite{saxenian_regional_1994}, \cite{gilson_legal_1999}, \cite{fallick_job-hopping_2006}, \cite{franco_covenants_2008}.} An illustrative example is Fairchild Semiconductor, one of the first semiconductor firms of SV, and itself a spinout of Shockley Laboratories, another semiconductor firm. Founded in the 1950s, Fairchild's spinouts include many well-known modern firms in SV, e.g. Intel and AMD. 

The literature has, somewhat plausibly, argued that SV's high employee mobility and entrepreneurship is due in large part to California's radical prohibition of \textit{covenants not to compete } (CNCs). Such contracts prevent departing employees from joining or founding competing firms for a span of time (usually 6 months to 2 years) and often in a certain geographical region. At the same time, theory is ambiguous about the relationship between CNC enforcement and innovation. The limited excludability of knowledge means the returns to investment cannot be fully appropriated, reducing investments in knowledge. If CNC enforcement reduces the formation of spinouts, incumbent firms can better appropriate the returns of knowledge (or capital) investments, increasing incentives to innovate. Viewed this way, CNCs are similar to patents, which also create incentives while reducing competition and knowledge spillovers.

Of course, the specific cause of SV's rise is difficult to ascertain. Nevertheless, the preceding question motivates the general question of what effect spinout entrepreneurship -- and in particular enforceability of non-competes -- has on innovation and labor productivity growth. This paper attempts to take a step towards answering this question.

In order to do so, I employ the following methodology. I develop a quality ladders model of endogenous growth with creative destruction, in which employees learn on the job how to form spinout firms. The model has regimes: one in which workers have commitment (i.e. CNCs can be enforced) and another in which they do not. According to the model, two key parameters determine the productivity growth / welfare comparison between the two regimes: (1) the rate of employee spinout idea discovery, and (2) the degree to which spinout profits come at the expense of the profit of the parent firm. Typically, in the non-enforcement regime, the effect on growth and welfare of increasing the rate of employee learning -- which expands the production possibilities frontier -- is non-monotonic. This implies  I provide some empirical discipline on these parameters by assembling a dataset from micro data on VC-funded startups and their founders (VentureSource), information on R\&D and patents of public firms in Compustat/CRSP, patent data for public firms from the NBER-USPTO database, , and state- and federal-level R\&D subsidies in Bloom et al. 2013. I discipline the other parameters of the model using indirect inference.\footnote{There are two ways to do this: calibrate the no-commitment model with data from California, or calibrate the commitment model with data from states which do enforce.} I validate the model by comparing its predictions across regimes to the data on state-level changes in CNC enforcement. Finally, I use the model to study the growth effects of policies that discourage or encourage spinout formation, such as non-compete enforcement but also R\&D subsidies for entering firms.

\section{Related literature}

Several strands of the literature point to the importance of answering this question. A large literature that has argued for the importance of business entry in labor productivity growth.\footnote{Using an accounting decomposition, \cite{foster_aggregate_2001} estimates that entry accounts for 25\% of aggregate productivity growth in the United States. See \cite{brandt_creative_2012} for evidence from China. See \cite{asturias_firm_2019} for evidence from Chile and Korea. \cite{akcigit_growth_2018} arrives at a similar estimate using a structural approach disciplined by patent citations data.} Another empirical literature has documented that employee spinouts are more productive, grow faster, and survive longer than other entrants.\footnote{See \cite{baslandze_spinout_2019} for evidence from US patent data and Compustat. See \cite{muendler_employee_2012} for evidence from Brazilian employer-employee matched data.} A related literature has argued that spinouts inherit knowledge from their parents, and that parent firms tend to be productive and knowledge intensive.\footnote{See \cite{klepper_entry_2005}, \cite{gompers_entrepreneurial_2005}, and \cite{baslandze_spinout_2019}.} 


The overall results of empirical work on the effect of CNC enforcement on productivity growth are inconclusive and, importantly, point to a tradeoff between entry of spinouts and investment by incumbent firms. \cite{stuart_liquidity_2003} find more local  entrepreneurship in response to local IPO (a "liquidity event") in regions not enforcing CNCs. \cite{marx_mobility_2009} finds that inventor mobility declines in response to an increase in non-compete enforcement. \cite{samila_venture_2010} finds that an increase in VC funding supply increases entrepreneurship more in states without non-compete restrictions, using an IV design. \cite{garmaise_ties_2011} finds that, in states where CNCs are more enforceable, managers are less mobile, have lower compensation, and invest less in their human capital, to the point of offsetting increased investments by the firm. On the other hand, \cite{conti_non-competition_2014} finds evidence that non-compete enforceability leads to incumbent firms pursuing riskier R\&D projects. \cite{colombo_does_2013} finds evidence that easier spinout formation -- proxied by access to finance -- leads to a reduction in incumbent firm knowledge investments.  Most recently, \cite{jeffers_impact_2018} uses data on influential state-level court precedents matched with LinkedIn data and finds that enforcement indeed reduces spinout formation while increasing capital investment by incumbent firms. Finally, \cite{marx_regional_2015} finds that CNC enforcement leads to inventor mobility out of the state, suggesting that differences in outcomes could be in part due to reallocation. 

Theoretical work has also explored this question. \cite{franco_spin-outs:_2006} studies a setting in which employees learn from their employers and use this knowledge to form spinouts. Employees implicitly pay for "stolen" knowledge through lower equilibrium wages. The equilibrium is Pareto efficient due to the assumption of a competitive output market.\footnote{Profits are generated from a decreasing returns to scale production function. Firms do not internalize the effect of their innovation on the price; coupled with their implicit sale of knowledge to their employee, this implies the private return is equal to the social return, ensuring Pareto optimality.}  

\cite{franco_covenants_2008} studies a two-period, two-region model with employee spinouts in which the region which does not enforce CNCs initially lags but eventually overtakes the region in which CNCs are enforced. In the first period, CNC enforcement means entry is more valuable in the enforcing region, hence more entry occurs. In the second period, spinouts enter in the non-enforcing region, competing with parent firms in the product market, and output increases relative to the enforcing region due to more competition.\footnote{The model is useful as an illustration of the mechanism, but it is a two period model and therefore leaves out an important part of the story. In particular, if non-enforcement reduces innovative investment, in later periods productivity could be lower in spite of higher competition. To assess this, a fully dynamic model is needed.}

\cite{shi_restrictions_2018} uses a rich model of contracting disciplined by Compustat data on firm investment and Execucomp data on executive non-compete contracts to study the effect of non-competes on executive mobility and firm investment. Her main finding is that the optimal policy is to somewhat restrict the permitted duration of CNCs. The key differences from my study are that (1) she focuses on executive mobility rather than employee entrepreneurship, and (2) she studies physical capital investment (\texttt{CAPEX} in Compustat) rather than R\&D investment (\texttt{XRD} in Compustat).\footnote{The firm's vulnerability to CNC makes the firm particularly vulnerable to knowledge leakage, so it seems reasonable that knowledge investment is particularly affected by CNC enforcement, relative to physical capital investment (as suggested by the previously cited empirical studies). The determinants of knowledge investment are particularly relevant to a study of long-run growth.} 

\cite{baslandze_spinout_2019} is the work closest to this one. Baslandze studies effect of spinout entrepreneurship on entry and growth by using patent data to identify employee spinouts, statistically describing them and their spawning firms, and using these statistics to calibrate a model of growth. Her conclusion is that the optimal enforcement of non-competes is zero. However, while tractable and useful in various ways, her model has two main shortcomings. First, it is not focused on creative destruction of parent firms by spinouts: the harm to the parent firm results from the additional assumption that the parent firm's technology gap -- relative to potential entrants in its own product-line -- drops to zero upon spinout formation. This is modeled as other firms "catching up" technologically to the incumbent, although it is interpreted as the loss of match-specific productivity upon employee spinout formation. In other words, the knowledge of the firm is fully embodied in its R\&D manager. By contrast my paper focuses on knowledge embodied in the firm, where the harm to the parent is due to reduced monopoly power. Second, higher non-compete enforcement is modeled as a higher cost of spinout formation rather than as a reduction in the contracting space. This means that even bilaterally efficient spinouts -- e.g., those which do not compete with the parent firm -- will not be allowed even in CNC enforcing regions, potentially biasing the results of the analysis in the direction of her conclusion. 

Finally, there is the empirical study of \cite{gompers_entrepreneurial_2005} which examines the differences between firms that spawn spinouts and those that do not, also using Compustat and Venture Source data. 

\section{Theory}

The model in this paper is based most closely on the model in \cite{akcigit_growth_2018}, which in turn builds on \cite{grossman_quality_1991}, with modifications to allow for spinout formation by employees. As is standard in the literature, I study a balanced growth equilibrium: comparative statics in my analysis are comparisons across different balanced growth paths. Due to space constraints, the exposition below only contains the model's essential ingredients.

Individuals are risk neutral and value the present discounted value of consumption (i.e. not leisure). They borrow and save using an instantaneous risk-free bond denominated in units of the final good. Individuals work in three capacities: final good production ($l_F^i$), intermediate good production ($l_I^i$), and R\&D ($l_{RD}^i$). Managing a spinout or firm does not require labor. Individuals consume a unique final good $Y(t)$ (price normalized to 1 in every period), which is produced competitively by labor and a continuum of intermediate goods $j \in [0,1]$. \footnote{Final goods production:
\begin{align*}
Y(t) &= \frac{L_F^{\beta}(t)}{1-\beta} \int_0^1 q_j^{\beta}(t) k_j^{1-\beta} (t) dj
\end{align*}
where $k_j(t)$ is the quantity of intermediate good $j$ and $q_j(t)$ is its quality.
}

At any (continuous) time $t \ge 0$, to each intermediate good $j$ there corresponds a firm $j$ with a monopoly on its production.\footnote{It is best to interpret these firms as products in the data; otherwise the model interprets all entry as new firm entry, rather than some entry being due to creative destruction by existing firms.} Firm $j$ produces its intermediate good of quality $q_j$ with a linear technology $k_j = Q l_j$, where $l_j$ is the labor input and $Q = \int_0^1 q_j dj$.\footnote{The dependence on aggregate quality $Q$ introduces a positive externality to R\&D.}

\subsection{R\&D, entry and spinouts}

Incumbents, spinouts and a unit mass of ordinary entrants hire R\&D labor in order to achieve a monopoly on an improved version of intermediate good $j$ with quality $\lambda q_j$, for exogenous $\lambda > 1$. Hiring R\&D labor yields a Poisson intensity of winning the race for the next rung on the quality ladder. This intensity is a function of \textit{effective R\&D effort} $z = \tilde{q}^{-1} l^{RD}$, where $\tilde{q}_j = q_j/Q$. This scaling assumption ensures that all $j$ have the same equilibrium innovation arrival rate, greatly simplifying the model.

The current producer has technology $\tau^I(z) = \chi^I z \phi^I(z)$. Let $z^S_j = \int_0^{m_j} z^S_{j,s} ds$ and $z^E_j = \int_0^1 z^E_{j,e} de$ denote total good $j$ spinout and ordinary entrant innovation effort. Then ordinary entrants have technology $\tau^E(z) = \chi^E z \phi^{SE} (z^S_j + z^E_j)$ and spinouts have technology $\tau^S(z) = \chi^S z \phi^{SE} (z^S_j + z^E_j)$. Finally, in order for the mass $m_j$ of spinouts to matter, spinouts must have individually decreasing returns to $R\&D$. In the baseline, I achieve this in the simplest way possible by imposing a capacity constraint $z^S_{j,s} \le \xi$ for exogenous $\xi > 0$.

\paragraph{Formation of potential spinouts}

Potential spinouts are formed as a result of R\&D labor hired by incumbents or other potential spinouts. The mass $m_j$ of such potential spinouts follows the law of motion 
\begin{align*}
	\dot{m}_j &= \tilde{q}_j^{-1} \nu \big(z^I_j + f\Big(\frac{\chi^S}{\chi^I}\Big) z^S_j \big) 
\end{align*}

where $\nu \ge 0 $ is an exogenous parameter and $f \ge 0$ is an exogenous weakly increasing function with $f(1) = 1$ (e.g., constant or the identity). When an innovation occurs, $m_j$ jumps to $0$ and $\tilde{q}_j$ jumps to $\lambda \tilde{q}_j$, and either the incumbent's monopoly continues or she is replaced by a spinout or entrant.

\paragraph{Non-competes}

Non-competes are the result of optimal contracting. Due to risk-neutrality, utility is transferrable hence the firm and employee use the contract which maximizes their joint surplus. In this setting, this either has no restrictions on entrepreneurship (when $m$ is low) or complete restrictions on entrepreneurship (when $m$ is high). 

\section{Data}

I study a new micro-level dataset which combines publicly available and private data. 

\paragraph{Constituent datasets}

The key dataset I employ is a private dataset on VC-funded startups in the United States dating back to 1986 (negotiated for purchase for academic license). The dataset is described in \cite{kaplan_how_2002}. 

For public firms, I observe firm-year level accounting data Compustat, and firm-day level stock prices from CRSP. I also observe all patents applied for and granted, as well as their field and cited / citing patents, from the publicly available NBER-USPTO patent database. An instrument for R\&D spending at the firm level is provided by \cite{bloom_identifying_2013}, publicly available and updated to 2017 on the author's website.

\cite{starr_noncompetes_2019} provides data on CNC enforcement across the United States. \cite{marx_mobility_2009}, and \cite{jeffers_impact_2018} provide longitudinal variation in state-level CNC enforcement. 

\section{Preliminary results}

\subsection{Theory}

\paragraph{Inverse U-shape effect of spinout formation rate on innovation \& welfare}

\subsection{Empirics}

\paragraph{Identifying spinouts}

I use the biographical information on startup founders in VentureSource to identify their previous employers and link this to Compustat, as in \cite{gompers_entrepreneurial_2005}. However, the database has grown significantly since 1999, containing more than 40,000 unique parent firms. Automating this poses a challenge since this matching must be done only using information on the name of the parent firm. I use a multi-pronged approached. For minor inconsistencies, I can use regular expressions. For more difficult problems (such as matching "AMD" to "Advanced Micro Devices") I scrape and parse Google search results and use a "merchant mapper" tool provided by the data science firm AltDG (in progress). Using this approach I have identified approximately 15,000 VC-funded spinouts since 1986, but I expect this number to increase significantly after I incorporate the tool from AltDG.

\paragraph{Effect of R\&D on spinout formation}

I have implemented some preliminary analyses relating R\&D expenditures, patents and firm characteristics to spinout formation. 

\footnotesize
\centering
{
\def\sym#1{\ifmmode^{#1}\else\(^{#1}\)\fi}
\begin{tabular}{l*{2}{c}}
\hline\hline
            &\multicolumn{1}{c}{(1)}&\multicolumn{1}{c}{(2)}\\
            &\multicolumn{1}{c}{Spinouts}&\multicolumn{1}{c}{SpinoutsDEV}\\
\hline
emp         &    -0.00172         &       0.156         \\
            &     (-1.22)         &      (0.62)         \\
[1em]
xrd         &    0.000427\sym{**} &      0.0158         \\
            &      (2.92)         &      (1.61)         \\
[1em]
patentapplicationcount\_cw&   -0.000107\sym{**} &     0.00189         \\
            &     (-2.91)         &      (1.24)         \\
[1em]
patentcount\_cw\_ma3&    0.000200\sym{***}&    -0.00313         \\
            &      (4.14)         &     (-0.93)         \\
[1em]
\_cons      &      0.0814\sym{**} &       2.052         \\
            &      (2.97)         &      (0.54)         \\
[1em]
State-Year FE&         Yes         &         Yes         \\
[1em]
NAICS4-Year FE&         Yes         &         Yes         \\
\hline
clustvar    &gvkey naics4#year stateCode#year         &gvkey naics4#year stateCode#year         \\
r2          &       0.258         &      0.0417         \\
r2\_a\_within &       0.162         &     0.00293         \\
N           &       11455         &       11455         \\
\hline\hline
\multicolumn{3}{l}{\footnotesize \textit{t} statistics in parentheses}\\
\multicolumn{3}{l}{\footnotesize \sym{*} \(p<0.05\), \sym{**} \(p<0.01\), \sym{***} \(p<0.001\)}\\
\end{tabular}
}




\section{Conclusion}

\small

\bibliography{references.bib}







\end{document}