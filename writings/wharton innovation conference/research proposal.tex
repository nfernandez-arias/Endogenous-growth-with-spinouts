\documentclass[11pt,english]{article}
%\usepackage{lmodern}
\linespread{1.05}
\usepackage{mathpazo}
%\usepackage{mathptmx}
%\usepackage{utopia}
\usepackage{microtype}
\usepackage[T1]{fontenc}
\usepackage[latin9]{inputenc}
\usepackage[dvipsnames]{xcolor}
\usepackage{geometry}
\usepackage{amsthm}
\usepackage{amsfonts}

\usepackage{courier}
\usepackage{verbatim}
\usepackage[round]{natbib}
\bibliographystyle{plainnat}


\definecolor{red1}{RGB}{128,0,0}
%\geometry{verbose,tmargin=1.25in,bmargin=1.25in,lmargin=1.25in,rmargin=1.25in}
\geometry{verbose,tmargin=1in,bmargin=1in,lmargin=1in,rmargin=1in}
\usepackage{setspace}

\usepackage[colorlinks=true, linkcolor={red!70!black}, citecolor={blue!50!black}, urlcolor={blue!80!black}]{hyperref}
%\usepackage{esint}
%\onehalfspacing
\usepackage{babel}
\usepackage{amsmath}
\usepackage{graphicx}

\theoremstyle{remark}
\newtheorem{remark}{Remark}
\begin{document}

\subsection*{Introduction}

Silicon Valley (SV) is typically considered an example of economic dynamism and innovation. Indeed, labor productivity growth in the MSA containing SV averaged 2.72\% from 1978 to 2015, compared to an average of about 2\% for the whole country, implying a 30\% difference in cumulative growth. \footnote{\cite{parilla_understanding_2017}.} A popular account holds that \textit{employee spinouts} -- firms founded by former employees --  have played a crucial role in generating this innovation.\footnote{See e.g. \cite{saxenian_regional_1994}, \cite{franco_covenants_2008}.} An illustrative example is Fairchild Semiconductor, one of the first semiconductor firms of SV, and itself a spinout of Shockley Laboratories, another semiconductor firm. Founded in the 1950s, Fairchild's spinouts include many well-known modern firms in SV, e.g. Intel and AMD.\footnote{Bitwig Studio from Ableton Live (digital audio workstations) and Bumble from Tinder (online dating) are two more modern examples.}

The literature has argued that SV's high employee mobility and entrepreneurship is due in large part to California's radical prohibition of \textit{covenants not to compete } (CNCs). Such contracts prevent departing employees from joining or founding competing firms for a span of time (usually six months to two years) and often in a certain geographical region. However, theory is ambiguous about the relationship between CNC enforcement and innovation. The limited excludability of knowledge means the returns to investment cannot be fully appropriated, reducing investments in knowledge. If CNC enforcement reduces the formation of spinouts, incumbent firms can better appropriate the returns of knowledge (or capital) investments, increasing incentives to innovate. Viewed this way, CNCs are similar to patents, which also create incentives while reducing competition and knowledge spillovers.

Of course, the specific cause of SV's rise is difficult to ascertain. Nevertheless, the preceding question motivates the general question of what effect spinout entrepreneurship -- and in particular enforceability of non-competes -- has on innovation and labor productivity growth. This paper attempts to take a step towards answering this question.

To do so, I first develop a quality ladders model of endogenous growth with creative destruction, in which employees learn on the job how to form spinout firms. The model has two regimes: one in which workers have commitment (i.e. CNCs can be enforced) and another in which they do not. Easier spinout formation can have a non-monotonic effect on growth and welfare. To discipline the model, I analyze a new dataset which I construct by merging five micro-level datasets: (1)  data on VC-funded startups and their founders (VentureSource), (2) data on R\&D and (3) stock prices of public firms in Compustat and CRSP, (4) patent data for public firms from the NBER-USPTO database, and (5) firm-specific instruments for R\&D expenditure based on subsidies (from  \cite{bloom_identifying_2013}). I use the resulting dataset to calibrate the theoretical model's parameters. The model's predictions across CNC enforcement regimes can be compared to those in the data. Finally, I use the model to study the growth and welfare effects of CNC enforcement and R\&D subsidies.

\subsection*{Related literature}

A large literature that has argued for the importance of business entry in labor productivity growth.\footnote{Using an accounting decomposition, \cite{foster_aggregate_2001} estimates that entry accounts for 25\% of aggregate productivity growth in the United States. \cite{akcigit_growth_2018} arrives at a similar estimate using a structural approach disciplined by patent citations data.} Another empirical literature has documented that employee spinouts are more productive, grow faster, and survive longer than other entrants.\footnote{See \cite{baslandze_spinout_2019} for evidence from US patent data and Compustat.} A related literature has argued that spinouts inherit knowledge from their parents, and that parent firms tend to be productive and knowledge intensive.\footnote{See e.g., \cite{gompers_entrepreneurial_2005}, and \cite{baslandze_spinout_2019}.} 


The overall results of empirical work on the effect of CNC enforcement on productivity growth are inconclusive and, importantly, point to a tradeoff between entry of spinouts and investment by incumbent firms. Well-known contributions finding a positive effect include \cite{marx_mobility_2009}, which finds that inventor mobility declines in response to an increase in non-compete enforcement;  \cite{garmaise_ties_2011}, which finds that, in states where CNCs are more enforceable, managers are less mobile, have lower compensation, and invest less in their human capital, to the point of offsetting increased investments by the firm. On the other hand, \cite{conti_non-competition_2014} finds evidence that non-compete enforceability leads to incumbent firms pursuing riskier R\&D projects. \cite{colombo_does_2013} finds evidence that easier spinout formation -- proxied by access to finance -- leads to a reduction in incumbent firm knowledge investments.  Most recently, \cite{jeffers_impact_2018} uses data on influential state-level court precedents matched with LinkedIn data and finds that enforcement indeed reduces spinout formation while increasing capital investment by incumbent firms. 

Theoretical work has also explored this question. \cite{franco_spin-outs:_2006} studies a setting in which employees learn from their employers and use this knowledge to form spinouts. Employees implicitly pay for "stolen" knowledge through lower equilibrium wages. The equilibrium is Pareto efficient due to the assumption of a competitive output market.

\cite{franco_covenants_2008} studies a two-period, two-region model with employee spinouts in which the region which does not enforce CNCs initially lags but eventually overtakes the region in which CNCs are enforced. In the first period, CNC enforcement means entry is more valuable in the enforcing region, hence more entry occurs. In the second period, spinouts enter in the non-enforcing region, competing with parent firms in the product market, and output increases relative to the enforcing region due to more competition.\footnote{The model is useful as an illustration of the mechanism, but it is a two period model and therefore leaves out an important part of the story. In particular, if non-enforcement reduces innovative investment, in later periods productivity could be lower in spite of higher competition. To assess this, a fully dynamic model is needed.}

\cite{shi_restrictions_2018} uses a rich model of contracting disciplined by Compustat data on firm investment and Execucomp data on executive non-compete contracts to study the effect of non-competes on executive mobility and firm investment. Her main finding is that the optimal policy is to somewhat restrict the permitted duration of CNCs. The key differences from my study are that (1) she focuses on executive mobility rather than employee entrepreneurship, and (2) she studies physical capital investment (\texttt{CAPEX} in Compustat) rather than R\&D investment (\texttt{XRD} in Compustat).\footnote{The firm's vulnerability to CNC makes the firm particularly vulnerable to knowledge leakage, so it seems reasonable that knowledge investment is particularly affected by CNC enforcement, relative to physical capital investment (as suggested by the previously cited empirical studies). The determinants of knowledge investment are particularly relevant to a study of long-run growth.} 

\cite{baslandze_spinout_2019} is the work closest to mine, studying a similar question using different data and model. She identifies spinouts using inventor data from the NBER-USPTO patent database, calculates some statistics describing the relationships between firm characteristics and spinout formation, and uses these statistics to calibrate a model of endogenous growth with employee spinouts and non-compete agreements. She concludes that optimal enforcement of non-competes is zero. Relative to hers, my empirical analysis focuses on VC-funded spinouts and I have information on their exit value, giving me a different, potentially better measure impactful spinout formation. My theoretical analysis also focuses on knowledge embodied in the firm, whereas hers is really about knowledge embodied in the employee. Perhaps most importantly, higher CNC enforcement is modeled as a higher fixed cost of spinout formation rather than as a restriction of the contracting space. This biases the results in favor of prohibiting CNCs. 

Finally, there is the empirical study of \cite{gompers_entrepreneurial_2005} which examines the differences between firms that spawn spinouts and those that do not, also using Compustat and Venture Source data. 

\subsection*{Theory}

The model in this paper is based most closely on the model in \cite{akcigit_growth_2018}, which in turn builds on \cite{grossman_quality_1991}, with modifications to allow for spinout formation by employees. 

Individuals are risk neutral and value the present discounted value of consumption (i.e. not leisure). They borrow and save using an instantaneous risk-free bond denominated in units of the final good. Individuals work in three capacities: final good production ($l_F^i$), intermediate good production ($l_I^i$), and R\&D ($l_{RD}^i$). Managing a spinout or firm does not require labor. Individuals consume a unique final good $Y(t)$ (price normalized to 1 in every period), which is produced competitively by labor and a continuum of intermediate goods $j \in [0,1]$. \footnote{Final goods production:
\begin{align*}
Y(t) &= \frac{L_F^{\beta}(t)}{1-\beta} \int_0^1 q_j^{\beta}(t) k_j^{1-\beta} (t) dj
\end{align*}
where $k_j(t)$ is the quantity of intermediate good $j$ and $q_j(t)$ is its quality.
}

At any (continuous) time $t \ge 0$, to each intermediate good $j$ there corresponds a firm $j$ with a monopoly on its production. Firm $j$ produces its intermediate good of quality $q_j$ with a linear technology $k_j = Q l_j$, where $l_j$ is the labor input and $Q = \int_0^1 q_j dj$.

\paragraph{Innovation}Incumbents, spinouts and a unit mass of ordinary entrants hire R\&D labor in order to achieve a monopoly on an improved version of intermediate good $j$ with quality $\lambda q_j$, for exogenous $\lambda > 1$. Hiring R\&D labor yields a Poisson intensity of winning the race for the next rung on the quality ladder. This intensity is a function of \textit{effective R\&D effort} $z = \tilde{q}^{-1} l^{RD}$, where $\tilde{q}_j = q_j/Q$. This scaling assumption ensures that all $j$ have the same equilibrium innovation arrival rate, greatly simplifying the model.

The current producer has technology $\tau^I(z) = \chi^I z \phi^I(z)$. Let $z^S_j = \int_0^{m_j} z^S_{j,s} ds$ and $z^E_j = \int_0^1 z^E_{j,e} de$ denote total good $j$ spinout and ordinary entrant innovation effort. Then ordinary entrants have technology $\tau^E(z) = \chi^E z \phi^{SE} (z^S_j + z^E_j)$ and spinouts have technology $\tau^S(z) = \chi^S z \phi^{SE} (z^S_j + z^E_j)$. Finally, in order for the mass $m_j$ of spinouts to matter, spinouts must have individually decreasing returns to $R\&D$. In the baseline, I achieve this in the simplest way possible by imposing a capacity constraint $z^S_{j,s} \le \xi$ for exogenous $\xi > 0$.

\paragraph{Formation of potential spinouts}

Potential spinouts are formed as a result of R\&D labor hired by incumbents or other potential spinouts. The mass $m_j$ of such potential spinouts follows the law of motion 
\begin{align*}
	\dot{m}_j &= \tilde{q}_j^{-1} \nu \big(z^I_j + z^S_j \big) 
\end{align*}

where $\nu \ge 0 $ is an exogenous parameter. When an innovation occurs, $m_j$ jumps to $0$ and $\tilde{q}_j$ jumps to $\lambda \tilde{q}_j$, and either the incumbent's monopoly continues or she is replaced by a spinout or entrant.

\paragraph{Non-competes}

Non-competes are the result of optimal contracting. Due to risk-neutrality, utility is transferrable hence the firm and employee use the contract which maximizes their joint surplus. In this setting, this either has no restrictions on entrepreneurship (when $m$ is low) or complete restrictions on entrepreneurship (when $m$ is high).

\paragraph{Equilibrium and solution}

Th equilibrium studied is a balanced growth path with constant growth rate and a constant cross-sectional distribution of product lines in $m$-space. I use the method in \cite{achdou_income_2017} as a foundation for my numerical solution algorithm. Due to space constraints I cannot offer a more detailed description of this part of the paper in this research proposal. 

\subsection*{Data}

The key dataset I employ is a private dataset on VC-funded startups in the United States dating back to 1986 (private data). The dataset is described in \cite{kaplan_how_2002}. I match spinouts in this dataset to public firms Compustat by parsing founder employment biographies for firm names. A firm is classified as a spinout if one of its founders most recent employers is in my dataset (working on developing industry code cross-walk). 

For public firms, I observe firm-year level accounting data Compustat, and firm-day level stock prices from CRSP. I also observe all patents applied for and granted, as well as their field and cited / citing patents, from the publicly available NBER-USPTO patent database. An instrument for R\&D spending at the firm level is provided by \cite{bloom_identifying_2013}, publicly available and updated to 2017 on the author's website.

\cite{starr_noncompetes_2019} provides data on CNC enforcement across the United States. \cite{marx_mobility_2009}, and \cite{jeffers_impact_2018} provide longitudinal variation in state-level CNC enforcement. 

\subsection*{Preliminary results}

\paragraph{Identifying spinouts}

I use the biographical information on startup founders in VentureSource to identify their previous employers and link this to Compustat, as in \cite{gompers_entrepreneurial_2005}. The database contains more than 45,000 startups. Automating this process poses a challenge since this matching must be done only using information on the name of the parent firm. I handle minor discrepancies using RegEx. For difficult cases -- e.g., matching "AMD" to "Advanced Micro Devices" -- I scrape and parse search results. I will also use a "merchant mapper" tool provided by the data science firm AltDG (meant for matching credit card receipts to firms). Using this approach so far I have identified approximately 7,000 VC-funded spinouts of public firms (8,000 parent-spinout pairs) since 1986. By comparison, \cite{gompers_entrepreneurial_2005} find about 3,000 entrepreneurial teams from 1986-1999. When restricting my methodology to spinouts founded between 1986-1999, I identify 2,000 entrepreneurial teams. This only covers public companies, though, and \cite{gompers_entrepreneurial_2005} find that roughly half of entrepreneurs are from public companies in their time frame. Therefore, my methodology broadly replicates theirs. I expect it to improve significantly after incorporating the tool from AltDG. 

\paragraph{Firm R\&D and spinout formation}

I have implemented some preliminary analyses relating R\&D expenditures, patents and firm characteristics to spinout formation. The basic equation for the regression shown in Table \ref{preliminary_results} is $Y_{ijst} = \alpha + \sigma \textrm{RD}_{it} + \beta X_{it} + \gamma_{jt} + \eta_{st} + \epsilon_{ijst}$, where $Y_{ijst}$ is the outcome variable for parent firm $i$ in industry $j$ and state $s$ at time $t$ and $X_{it}$ is a vector of firm-year controls. The first column finds a significant relationship, but the second column just misses due to the additional noise in the dependent variable (though p-value of one-sided test is approximately 5\%, which is more applicable here). It may be possible to improve this estimate by improving my identification of spinouts. In spite of the industry-year and state-year fixed effects, there is potential for omitted variable bias from unobserved time-varying firm-level characteristics. I hope to improve on this by using an IV approach based on \cite{bloom_identifying_2013}.

\begin{table}[h] \phantomsection
	\centering
	\footnotesize
	{
\def\sym#1{\ifmmode^{#1}\else\(^{#1}\)\fi}
\begin{tabular}{l*{2}{c}}
\hline\hline
            &\multicolumn{1}{c}{(1)}&\multicolumn{1}{c}{(2)}\\
            &\multicolumn{1}{c}{Spinouts}&\multicolumn{1}{c}{SpinoutsDEV}\\
\hline
emp         &    -0.00172         &       0.156         \\
            &     (-1.22)         &      (0.62)         \\
[1em]
xrd         &    0.000427\sym{**} &      0.0158         \\
            &      (2.92)         &      (1.61)         \\
[1em]
patentapplicationcount\_cw&   -0.000107\sym{**} &     0.00189         \\
            &     (-2.91)         &      (1.24)         \\
[1em]
patentcount\_cw\_ma3&    0.000200\sym{***}&    -0.00313         \\
            &      (4.14)         &     (-0.93)         \\
[1em]
\_cons      &      0.0814\sym{**} &       2.052         \\
            &      (2.97)         &      (0.54)         \\
[1em]
State-Year FE&         Yes         &         Yes         \\
[1em]
NAICS4-Year FE&         Yes         &         Yes         \\
\hline
clustvar    &gvkey naics4#year stateCode#year         &gvkey naics4#year stateCode#year         \\
r2          &       0.258         &      0.0417         \\
r2\_a\_within &       0.162         &     0.00293         \\
N           &       11455         &       11455         \\
\hline\hline
\multicolumn{3}{l}{\footnotesize \textit{t} statistics in parentheses}\\
\multicolumn{3}{l}{\footnotesize \sym{*} \(p<0.05\), \sym{**} \(p<0.01\), \sym{***} \(p<0.001\)}\\
\end{tabular}
}

	\footnotesize \caption{\footnotesize The dependent variable in the first column is the number of founder-spinouts to spawn from firm $i$ in year $t$ (i.e. spinouts with more than one founder receive higher weight, as a way to measure the "size" of the spinout). In the second column, I proxy the size of a spinout instead by the present discounted value (yearly discount 5\%) of its eventual exit (IPO or M\&A). Then, in order to avoid double counting, firm $i$'s contribution to the value of spinout $s$ is the fraction of founders of $s$ coming from firm $i$. Both regressions include firm $i$ employment and citation weighted patent applications at time $t$ as well as a 3-year moving average of citation-weighted patents granted. Both regressions use state-year and naics4-year level fixed effects. Clustering is at the firm, naics4-year and state-year level. The sample is restricted to parent firms in which at least one spinout event is observed.} 
	\label{preliminary_results}
\end{table}

\paragraph{Effect of spinout formation on growth}

I conduct a preliminary calibration based on statistics mostly taken from \cite{akcigit_growth_2018} (not shown). Figure \ref{welfare_plots} shows that welfare -- calculated as present discounted aggregate consumption -- is non-monotonic in the effective rate of spinout formation $\nu \xi$. This is in spite of the fact that this expands the production possibilities frontier. The decrease in welfare is driven by a reduction in the productivity growth rate. Peak decentralized welfare is increasing in $\chi^S$.

\begin{figure}[h] \phantomsection
	\centering
	\includegraphics[scale=0.33]{nuxi_chiS_welfare_plot.png}
	\caption{Welfare implications of varying effective rate of spinout formation $\nu \xi$.}
	\label{welfare_plots}
\end{figure}


\subsection*{Next steps}

The highest priority is refining and completing my dataset. This includes improving my method for identifying spinouts using the AltDG merchant mapper tool, completing my cross-walk from the Venture Source industrial classification to NAICS4, developing a methodology for assessing the effect of spinout funding events on parent firm stock prices and future earnings, and constructing firm-level instruments for R\&D spending. After that I can implement my full empirical analysis and calibration. I am also working on theoretical extensions, e.g. allowing spinouts which do not compete with parents (hence not disallowed by CNCs). 

\footnotesize

\bibliography{references.bib}







\end{document}