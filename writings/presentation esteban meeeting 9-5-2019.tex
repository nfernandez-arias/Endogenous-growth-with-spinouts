\documentclass[english,usenames,dvipsnames]{beamer}
\usetheme{default}
\beamertemplatenavigationsymbolsempty
\setbeamertemplate{footline}[frame number]
\setbeamercolor{alerted text}{fg=blue1}
%\setbeamercolor{frametitle}{fg=blue2}
\usepackage[utf8]{inputenc}
\usepackage{caption}
\usepackage{booktabs}
\usepackage{appendixnumberbeamer}
\usepackage{babel}
\usepackage{amsmath}
\usepackage{hyperref}
\usepackage{geometry}
\usepackage{bbm}
\usepackage{amsthm}
\usepackage{verbatim}
%\usepackage{palatino}
\definecolor{red1}{RGB}{255,50,0}
\definecolor{blue1}{RGB}{80,80,255}
\definecolor{blue2}{rgb}{0.22,0.37,1}
\definecolor{green1}{RGB}{34,139,35}

\setbeamertemplate{itemize items}[default]


\title{Employee Spinouts, Creative Destruction and Endogenous Growth}
\author{Nicolas Fernandez-Arias}
%\date{March 7, 2019}

\begin{document}

\maketitle

\begin{frame}{Motivation}
\begin{itemize}
	\item Question: should we encourage or discourage employee spinouts? 
	\begin{itemize}
		\item Spinouts = firms founded by ex-employees
	\end{itemize}
	\item Entry contributes significantly to productivity growth
	\begin{itemize}
		\item Over 10-year horizon, 25\% of labor productivity growth accounted by entry in manufacturing (Baily-Bartelsman-Haltiwanger 1996)
		\item ~25\% of aggregate productivity growth due to entrants (Akcigit-Kerr 2017)
	\end{itemize}
	\item Spinouts are most innovative entrants
	\begin{itemize}
		\item Fairchild semiconductor / Silicon Valley (Saxenian 1994); Detroit automakers
		\item Spinouts: more patents / R\&D, sales growth, survival (Baslandze 2019) 
		\item Spinouts 15-30\% of entrants; larger, grow faster, more survival (Muendler et al. 2012, Brazilian data)
	\end{itemize}
\end{itemize}
\end{frame}

\begin{frame}{Spinouts of Fairchild Semiconductor}
\begin{figure}
	\includegraphics[scale=0.32]{../figures/fairchildren_early.png}
	\caption{Source: Endeavor Insights}
\end{figure}
\end{frame}

\begin{frame}{Motivation - Theory}
\label{theory_big_picture}
\begin{itemize}
	\item Schumpeter 1942, Arrow 1962, Romer 1986, Grossman-Helpman 1991, etc.: \alert{underinvestment} in knowledge due to \alert{limited excludability}
	\item Patent literature: dynamic efficiency vs. static monopoly distortion tradeoff
	\item Creative destruction by spinouts similar tradeoff
	\item Should we encourage or discourage spinout formation?
	\item Existing frameworks (e.g., Franco-Filson 2006, Baslandze 2019) underemphasize disincentive for firm R\&D 
\end{itemize}
\end{frame}

\begin{frame}{Related literature}
\begin{itemize}
	\item Firm dynamics and endogenous growth
	\begin{itemize}
		\item Romer 1986, Grossman \& Helpman 1991, Aghion \& Howitt 1992, Klette \& Kortum 2004, Acmemoglu \& Akcigit 2012, Akcigit \& Kerr 2017
	\end{itemize}
	\item Models of employee spinouts
	\begin{itemize}
		\item Klepper 2002, Klepper \& Sleeper 2005, Franco \& Filson 2006, Franco \& Mitchell 2008, Rauch 2015, Rossi-Hansberg \& Chatterjee 2012
		\item Baslandze 2019 closest to this paper
	\end{itemize}
	\item Empirics on employee mobility, spinouts
	\begin{itemize}
		\item Spawning of spinouts: Gompers et al. 2005, Garmaise 2011, Baslandze 2019
		\item Effect on parent firms: Campbell et. al 2012, Wezel et al. 2006
		\item Effect of non-compete enforcement: Garmaise 2009, Marx et al 2009, Samila-Sorenson 2011, Jeffers 2018
	\end{itemize}
\end{itemize}
\end{frame}

\begin{frame}{Overview and results}
\begin{itemize}
	\item Develop model of endogenous growth with creative destruction
	\item Calibrate model with help from new microdata on employee spinouts (VentureSource + Compustat + NBER USPTO)
	\begin{itemize}
		\item Information on founders of VC-funded startups
		\item Information on valuations / amounts raised at funding rounds / exits
	\end{itemize}
	\item Key results so far:
	\begin{itemize}
		\item Enforcing non-competes increases welfare by 18\% in current calibration
		\item Incumbents do almost twice as much R\&D (for a roughly 50\% increase in innovation arrival rates)
		\item and spinout entry only goes down slightly: within-industry spinouts are replaced by out-of-industry spinouts.
	\end{itemize}
\end{itemize}
\end{frame}

\begin{frame}{Model}
\begin{itemize}	
	\item Quality ladder model of endogenous growth through creative destruction 
	\begin{itemize}
		\item builds on Akcigit-Kerr 2017, "Growth through heterogeneous innovations"
		\item Grossman-Helpman 1991
	\end{itemize}
	\item \textbf{New feature:} R\&D workers \alert{learn on the job} how to form \alert{competing spinouts}
	\begin{itemize}
		\item \alert{Endogenous knowledge spillovers} \ldots
		\item \ldots Can \alert{disincentivize incumbent R\&D}
		\item \alert{Escape competition effect} incentivizes incumbent innovation 
	\end{itemize} 
\end{itemize}
\end{frame}

\begin{frame}{Model - Overview}
\begin{itemize}
	\item Households consume and supply labor to production of intermediate and final goods and R\&D on intermediate goods
	\item Intermediate goods firms do R\&D to improve production technology, driving aggregate TFP growth
	\item R\&D by incumbents transfers knowledge to employees, who become high-type potential entrants, creating more R\&D competition
	\item Non-compete contracts can allow incumbents to protect their monopoly on their knowledge, but may reduce entry
\end{itemize}
\end{frame}

\begin{frame}{Model - Households}
\begin{itemize}
	\item Unit continuum of households are risk-neutral, discount rate $\rho > 0$, objective
	\begin{align*}
	\mathbb{E}_t \int_0^{\infty} e^{-\rho s} c_i(t+s) ds
	\end{align*}
	where $c_i(t)$ is household $i$ consumption of final good at time $t$
	\item Unit endowment of labor, supplied inelastically to final goods production, intermediate goods production or R\&D
\end{itemize}
\end{frame}



\begin{frame}{Model - Final good}
\begin{itemize}
	\item Final good output $Y$ produced using production function
	\begin{align*}
	F(L,\{q_j\},\{k_j\}) &= \frac{L^{\beta}(t)}{1-\beta} \int_0^1 q_j^{\beta} k_j^{1-\beta}  dj 
	\end{align*}
	\item Labor $L_F(t)$
	\item Intermediate goods $\{k_j\}$, qualities $\{q_j\}$, $j \in [0,1]$
\end{itemize}
\end{frame}

\begin{frame}{Model - Intermediate goods}
\begin{itemize}
	\item Production technology
	\begin{align*}
	k_j = Q l_j
	\end{align*}
	where $Q = \int_0^1 q_j dj$ is average quality
	\item Scaling to ensure balanced growth
	\item Based on setup in AK 2017
\end{itemize}
\end{frame}

\begin{frame}{Model - Innovation}
\begin{itemize}
	\item Innovation in good $j$ leads to quality jumps to $\lambda q_j$, for $\lambda > 1$
	\item Temporary monopoly in good $j$
	\begin{itemize}
		\item no limit pricing due to two-stage Bertrand assumption from AK 2017
	\end{itemize}
\end{itemize}
\end{frame}

\begin{frame}{Model - Innovation}
\begin{itemize}
	\item Each good line $j$,
	\begin{itemize}
		\item Mass $1$ of ordinary \alert{potential entrants}
		\item Mass $m_j$ of \alert{potential spinouts} 
	\end{itemize}
	\item $m_j$ grows over time; upon innovation, jumps to $0$
	\item Effort of potential entrant $e \in [0,1]$ and potential entrant $s \in [0,m_j]$: $z_E^e, z_S^s$ 
	\item Aggregate $z_E = \int_0^1 z_E^e de$ and $z_S = \int_0^{m_j} z_S^s ds$ 
\end{itemize}
\end{frame}

\begin{frame}{Model - Innovation}
\begin{itemize}
	\item Scaling assumption: one unit of R\&D on quality $q$ requires $(q/Q)$ units of labor in R\&D
	\item Incumbent ($I$) discovers next innovation with Poisson intensity
	\begin{align*}
	\tau^I(z) &= \chi_I z^{1-\psi}
	\end{align*}
	where $z$ is units of R\&D, $\chi_I$ is R\&D productivity, and $\psi \in (0,1)$ measures decreasing returns to R\&D 
	\item Ordinary entrants ($E$) and spinouts ($S$) 
	\begin{align*}
	\tau^S(z,\overbrace{z_E+z_S}^{\textrm{aggregate}}) &= \chi_{S} z (z_E + z_S)^{-\psi} \\
	\tau^E(z,z_E+z_S) &= \chi_{E} z (z_E + z_S)^{-\psi}
	\end{align*}
	\item Aggregate DRS: \alert{fishing out} externalities
	\item Spinouts have \alert{capacity constraint} $z_S^s \le \xi$ (can replace with DRS)
\end{itemize}
\end{frame}



\begin{frame}{Model - Formation of potential spinouts}
\begin{itemize}
	\item At rate $(1-\theta)(Q/q_j) \nu$ per year of labor, R\&D worker gets idea for a potential spinout in own line $j$
	\item Idea generated in random line $j' \ne j$ with Poisson probability $\theta \nu$ 
	\item Immediately begins hiring R\&D to try to implement
	\begin{itemize}
		\item Continue to work - no labor / entrepreneurship choice
	\end{itemize}
	\item Law of motion for mass of potential spinouts: $\dot{m}_j = \dot{m}_j^{\textrm{from $j$}} + \dot{m}_j^{\textrm{from $j' \ne j$}}$ where
	\begin{align*}
	\dot{m}_j^{\textrm{from $j$}}&= (1-\theta) \nu z_{I,j}\\
	\dot{m}_j^{\textrm{from $j' \ne j$}} &= \theta \nu \int_0^1 \frac{q_j}{Q} z_{I,j} dj
	\end{align*}
\end{itemize}
\end{frame}

\begin{frame}{Model - Aggregation}
\begin{itemize}
	\item $\tau_j(t)$ denotes the Poisson intensity of an innovation arriving at $j$ at time $t$
	\item Growth rate of aggreagte TFP $Q(t)$
	\begin{align*}
	g_t &= (\lambda -1) \int_0^1 \Big(q_{j}(t)/Q(t)\Big) \tau_j(t) dj 
	\end{align*}
\end{itemize}
\end{frame}

\begin{frame}{Equilibrium - Static equilibrium conditions}
\begin{itemize}
	\item Given aggregate R\&D labor allocation $L^{RD}$ calculate static equilibrium variables
	\item Final goods wage $\bar{w}(t) = \Big(\frac{\beta}{1-\beta} (1-\beta)^{\frac{1-\beta}{\beta}} \Big)^{\beta} Q_t = C(\beta) Q_t$
	\item Intermediate goods price $p = (1-\beta)^{-1} \frac{w_F}{Q} = (1-\beta)^{-1} C(\beta)$
	\item Final goods production labor allocation $L_F = \frac{1 - L_{RD}}{1 + \frac{1-\beta}{C(\beta)}}$
	\item Intermediate goods firm profits $\beta q_j L_F$
\end{itemize}
\end{frame}

\begin{frame}{Equilibrium - Dynamic equilibrium conditions}
\begin{itemize}
	\item Look for BGP equilibrium with $g_t \equiv g, r_t \equiv r, L_{RD}(t) = L_{RD}$ and so on  
	\item Incumbent, potential spinout HJBs
	\item Potential entrants zero profit condition
	\item Worker indifference conditions
\end{itemize}
\end{frame}

\begin{frame}{Equilibrium - Scaling of value functions}
\begin{itemize}
	\item Incumbent and spinout value functions $V(q,m,t),W(q,m,t)$
	\item (Recast as proposition) Guess and verify: BGP equilibrium with $V(q,m,t) = qV(m); W(q,m,t) = qW(m)$ and all innovation arrival rates depend on $m$ only
	\item Unfortunately, \alert{risk aversion breaks this}
	\begin{itemize}
		\item Relies on the fact that the value of a $(Q/q) \nu (1-\theta)$ intensity of discovering an idea of value $W(m)q$ be \alert{constant in $q$}
	\end{itemize}
\end{itemize}
\end{frame}

\begin{frame}{Incmbent - HJB equations}
\begin{itemize}
	\item Incumbent HJB
	\small
	\begin{align*}
		(r + \tau_S(m) + &\tau_E(m)) V(m) = \Pi + \bar{\sigma}V'(m)  \\
		                                &+ \max_{z \ge 0} z \Bigg\{ \chi_I z^{-\psi} \Big( \lambda V(0) - V(m) \Big) - (w(m) - \theta \nu V'(m))   \Bigg\}
	\end{align*}
	\normalsize
	\item Potential spinout HJB
	\small
	\begin{align*}
		(r + \tau_S(m) + \tau_E(m)) W(m) &= \Big( \bar{\sigma} + \sigma(m) \Big)W'(m)  \\
										&+ \max_{0 \le z \le \xi} z \Bigg\{ \chi_S z^{-\psi} (1-\kappa) \lambda  V(0) - w(m)  \Bigg\}
	\end{align*}
	\normalsize
	where $\kappa \lambda V(0)$ is the cost of creative destruction 
\end{itemize}
\end{frame}

\begin{frame}{Equilibrium - Free entry}
\begin{itemize}
	\item Free entry implies zero expected profits for ordinary entrants if R\&D effort is positive:
	\begin{align*}
		\chi_E \Big(z_E(m) + z_S(m) \Big)^{-\psi} (1-\kappa) \lambda V(0) \le \bar{w}
	\end{align*} 
	with equality whenever $z_E(m) > 0$
	\item $\chi_S > \chi_E$ $\Rightarrow$ potential spinouts perform R\&D whenever $z_E(m) > 0$
	\item Potential spinout free entry: $M > 0$ such that
	\begin{align*}
		\chi_S \big( \xi M \big)^{-\psi} \lambda V(0) = \bar{w}
	\end{align*}
\end{itemize}
\end{frame}

\begin{frame}{Equilibrium - Entrant and spinout policies}
\begin{itemize}
	\item (aggregated) R\&D policy functions
	\begin{align*}
	z_S(m) &= \min \Big\{\xi m, \xi M \Big\} \\
	z_E(m) &= \max\Bigg\{0, \Bigg(\frac{\bar{w}}{\chi_E\lambda V(0)}\Bigg)^{-1/\psi} - z_S(m) \Bigg\}
	\end{align*}
	\item $\chi_S \le \chi_E$: nests model with no spinouts, no state variable
\end{itemize}
\end{frame}

\begin{frame}{Equilibrium - Worker indifference conditions}
\begin{itemize}
	\item Value of working in R\&D for incumbent equals value of working in other employment
	\item $W(m)$ is value of potential spinout in line $j$
	\item $\mathcal{W} \equiv \int_0^\infty W(m) \mu(m) dm$  
	\item Indifference condition
	\small
	\begin{align*}
	w(m) + \nu (1-\zeta) \Big(\theta \mathcal{W} + (1-\theta) W(m) \Big) &= \bar{w}
	\end{align*}
	where $\zeta \in (0,1)$ is the proportional cost of spinning.
\end{itemize}
\end{frame}

\begin{frame}{Equilibrum - aggregation}
\begin{itemize}
	\item Stationary distribution of lines $j$ in $m$-space satisfies Kolmogorov Forward equation
	\begin{align*}
		0 = -\frac{d}{dm} \Big( \big(\bar{\sigma} + \sigma(m) \big)\mu(m) \Big) - \tau(m) \mu(m) 
	\end{align*}
	\item Also,  $\mathbb{E}_{q}[q/Q_t|m,t] = \gamma(m)$, exists and satisfies
	\begin{align*}
		\gamma(m) &= C_{\gamma} e^{-g s(m)}
	\end{align*}
	where $s(m)$ is the equilibrium number of years to reach $m$ (conditional on no innovations)
	\item Law of iterated expectations determines $C_{\gamma}$: 
	\begin{align*}
		1 = E_{q}[q/Q,t] = E_m[ E_{q}[q/Q|m,t] ,t] = \int_0^{\infty} \gamma(m) \mu(m) dm
	\end{align*}
\end{itemize}
\end{frame}

\begin{frame}{Non-competes}
\begin{itemize}
	\item Incumbent HJB can impose a permanent noncompete
	\small
	\begin{align*}
	(r + \tau_{SE}(m)) V(m) &= \Pi + \bar{\sigma}V'(m)  \\
	+ &\max_{z \ge 0, x \in \{0,1\} } z \Bigg\{ \chi_I z^{-\psi} \Big( \lambda V(0) - V(m) \Big) \\
	- &(1-x) \Big(w(m) - \theta \nu V'(m) \Big) - x w^{NC} \Bigg\}
	\end{align*}
	\normalsize
	where $w^{NC}$ satisfies another worker indifference condition
	\begin{align*}
		\bar{w} &= w^{NC} + \nu (1-\zeta) (1-\theta) \mathcal{W}
	\end{align*}
	since spinouts are still allowed in other lines $j' \ne j$
	\item Optimal policy is $x = 1$ iff $w^{NC} < w(m) - \theta \nu V'(m)$ 
\end{itemize}
\end{frame}


\begin{frame}{Solving the model}
\begin{enumerate}
	\small
	\item Guess $\{g, L_{RD}, w(m), w^{NC}, M, \bar{\sigma} \}$
	\begin{enumerate}
		\small
		\item Static equilibrium conditions $\Rightarrow L_I,L_F,\pi$
		\begin{enumerate}
			\small
			\item Free entry, etc. $\Rightarrow z_S(m), z_E(m)$
			\item Incumbent HJB $\Rightarrow  V(m),z_I(m)$
			\item Update $M$ and check for convergence
		\end{enumerate}
		\item Spinout HJB $\Rightarrow$ $W(m)$
		\item Aggregation $\Rightarrow$ $\mu(m),\gamma(m),g,L_{RD},\bar{\sigma},\mathcal{W}$
		\item Worker indifference $\Rightarrow$ $w(m),w^{NC}$
		\item Check convergence of of $\{g, L_{RD}, w(m), w^{NC}, M, \bar{\sigma} \}$
	\end{enumerate}
\end{enumerate}
\end{frame}


\begin{frame}{Calibration - Targets}
\begin{table}[H]
	\scriptsize
	\centering{}%
	\begin{tabular}{lll}
		Table 1 &  &  \tabularnewline
		Calibration Targets &  &  \tabularnewline
		\hline 
		Moment  & Target & Model \tabularnewline
		&   & \tabularnewline
		\hline 
		Growth rate $g$ & 0.015 & 0.02 \tabularnewline
		Interest rate $r$ & 0.05 & 0.05 \tabularnewline
		Profit / sales ratio & .109 & .109 \tabularnewline
		R\&D spending / sales ratio & \alert{0.04} & \alert{0.002} \tabularnewline
		Internal patent share & 0.5 & 0.55 \tabularnewline
		Spinout entry rate & 0.03 & 0.025 \tabularnewline
		Share of entry by spinouts & 0.30 & 0.21 \tabularnewline
		(*) Spinout / entrant development $\rightarrow$ revenue (exit?) transition rates & x & x \tabularnewline
		(*) Pre-money valuation at first funding generated by dollar of R\&D & x & x \tabularnewline
		R\&D share of effective employment & \alert{0.08} & \alert{0.031} \tabularnewline
		R\&D wages to production wages (incumbent) & \alert{0.7} & \alert{0.42} \tabularnewline
		R\&D wages to production wages (all) & 0.9 & 0.95
	\end{tabular}
\end{table}
\end{frame}

\begin{frame}{Calibration - Parameters}
\begin{table}[h]
	\scriptsize
	\centering{}%
	\begin{tabular}{lll}
		Table 2 &  &  \tabularnewline
		Calibration Parameters &  &  \tabularnewline
		\hline 
		Parameter & Explanation & Value\tabularnewline
		\tabularnewline
		\hline 
		Normalized & & \tabularnewline
		$\xi$ & Spinout R\&D capacity & 20 \tabularnewline
		&  & \tabularnewline
		Chosen outside model & & \tabularnewline
		$\psi_I$ & Incumbent R\&D curvature & 0.5\tabularnewline
		$\psi_E$ & non-Incumbent R\&D curvature & 0.5\tabularnewline
		&  & \tabularnewline
		Chosen to match data & & \tabularnewline
		$\rho$ & Discount rate & 0.05\tabularnewline
		$\beta$ & $\beta^{-1}$ is EoS between intermediate goods & 0.106\tabularnewline
		$\lambda$ & Innovation step size & 1.073\tabularnewline
		$\chi_I$ & Incumbent R\&D productivity & 2.891\tabularnewline
		$\chi_E$ & Entrant R\&D productivity & 0.656\tabularnewline
		$\chi_S$ & Spinout R\&D productivity & 1.1667\tabularnewline
		$\nu$ & Learning rate & 0.025 \tabularnewline
		$\theta$ & Fraction of spinouts in $j' \ne j$ & 0.453 \tabularnewline
		$\zeta$ & Worker discounting of knowledge value & 0.5147 \tabularnewline
		$\kappa$ & Cost of creative destruction & 0.389
	\end{tabular}
\end{table}
\end{frame}


\begin{frame}{Identification}
\begin{itemize}
	\item Some parameters are calibrated directly to statistics in the data
	\begin{itemize}
		\item $\rho$: interest rate
		\item $\beta$: profit-sales ratio
	\end{itemize}
	\footnotesize
	\item The rest calibrated using simulated method of moments
	\item Growth rate informs all innovation parameters
	\item Incumbent R\&D spending / incumbent sales ratio informs $\chi_I$
	\item Internal patent share informs $\chi_I / \chi_S, \chi_I / \chi_E, \kappa, \lambda, \nu$
	\item The entry rate informs $\chi_S,\chi_E,\kappa,\nu$
	\item The ratio of spinout entry to non-spinout entry informs $\chi_S / \chi_E, \nu$
	\item (*) Spinout / entrant transition to revenue rate informs $\chi_S / \chi_E$
	\item (*) Pre-money valuation at first funding generated by dollar R\&D informs $\nu$ and $\zeta$ 
	\item R\&D share of effective employment informs innovation parameters
	\item Ratio of wages of R\&D workers at incumbents to wages of R\&D workers at spinouts informs $\theta$
	\item Ratio of wages of R\&D workers to production workers informs $\zeta , \nu$ 
	\item Ratio of spinout formation rate in enforcing vs. non-enforcing states informs $\theta,\zeta,\nu$ 
\end{itemize}
\end{frame}

\begin{frame}{Discussion}
\begin{itemize}
	\item $\theta$ can act as a reduced form for a "baseline" level of non-compete enforcement
	\item Model misses wages
	\begin{itemize}
		\footnotesize
		\item How to measure wage of R\&D worker compared to production worker, holding constant human capital?
	\end{itemize}
	\item Model misses R\&D spending badly
	\begin{itemize}
		\footnotesize
		\item First, in data, not all R\&D spending is wages
		\item Second, Firms in model are kind of like products in the data, assuming firms start with one product. 
		\item Model about R\&D spending on own products, not entry into other products
		\item Entrants in model should include existing firms destroying other incumbents' products (as in AK 2017)
	\end{itemize}
	\normalsize
	\item Model misseses R\&D labor allocation
	\begin{itemize}
		\footnotesize
		\item How to measure effective employment, not nominal employment, in R\&D? Average human capital of R\&D workers
	\end{itemize}
\end{itemize}
\end{frame}

\begin{frame}{Studying the model}
\begin{itemize}
	\item Compute welfare by starting model at $t=0$ with $Q_0 = 1$ and computing PDV of consumption
	\begin{align*}
		\textrm{Welfare} &= \frac{\overbrace{Y_0}^{\textrm{Output}} - \overbrace{\kappa \lambda V(0) \int_0^{\infty} \tau_{SE}(m) \gamma(m) \mu(m) dm}^{\textrm{Cost of creative destr.}}}{\rho - g} 
	\end{align*}
	\item How to interprete worker discount of value of spinning out, $\zeta$? 
\end{itemize}
\end{frame}

\begin{frame}{Comparative static: adding non-competes}
\begin{itemize}
	\item With non-competes enforced, welfare increases by $18.9$\%
	\item This is due to an increase in $g$ of $0.5$ percentage points to $2.5\%$
	\item What key weakly identified parameters does this rely on?
	\begin{itemize}
		\item $\psi$: if $\psi$ is higher, DRS is more severe, hence the response to changing incentives is smaller
		\item $\zeta$: if $\zeta$ is lower, R\&D wages are lower, hence non-competes reduce by less the effective wage paid by incumbents
		\item $\theta$: higher $\theta$ means fewer spinouts ruled out by noncompetes, but also smaller benefit to incentives. So not clear. 
		\item The assumption that R\&D generates spinouts. If spinout entry is matched, R\&D has to generate a lot of competition. If some spinouts are generated regardless of R\&D then this could give a different answer, then I try to match what I find in the VentureSource data.
	\end{itemize}
\end{itemize}
\end{frame}

\begin{frame}{Comparing moments}
\begin{table}[H]
	\scriptsize
	\centering{}%
	\begin{tabular}{lll}
		Table 3 &  &  \tabularnewline
		Comparing regimes &  &  \tabularnewline
		\hline 
		Moment  & Non-competes & No non-competes\tabularnewline
		&   & \tabularnewline
		\hline 
		Growth rate $g$ & 0.0248 & 0.02 \tabularnewline
		Interest rate $r$ & 0.05 & 0.05 \tabularnewline
		Profit / sales ratio & .109 & .109 \tabularnewline
		R\&D spending / sales ratio & 0.007 & 0.002 \tabularnewline
		Internal patent share & 0.64 & 0.55 \tabularnewline
		Spinout entry rate & 0.022 & 0.025 \tabularnewline
		Share of entry by spinouts & 0.18 & 0.21 \tabularnewline
		R\&D share of effective employment & 0.033 & 0.031 \tabularnewline
		R\&D wages to production wages (incumbent) & 0.77 & 0.42 \tabularnewline
		R\&D wages to production wages (all) & 0.96 & 0.95
	\end{tabular}
\end{table}
\end{frame}

\begin{frame}{Comparing innovation rates}
\begin{figure}
	\includegraphics[scale=0.25]{figures/compStatNC/noNC/innovation_rates_t.png}
	\includegraphics[scale=0.25]{figures/compStatNC/NC/innovation_rates_t.png}
	\caption{\footnotesize Incumbent innovates more, but with less steep slope. Spinouts grow at a similar rate: there is twice as much R\&D by incumbents, so twice as many non-competing spinouts, and no effect on total spinouts.}
\end{figure}
\end{frame}

\begin{frame}{Comparing innovation rates}
\begin{figure}
	\includegraphics[scale=0.25]{figures/compStatNC/noNC/innovation_rates_m.png}
	\includegraphics[scale=0.25]{figures/compStatNC/NC/innovation_rates_m.png}
	\caption{\footnotesize The innovation rate of the incumbent given $m$ starts higher and is concave in $m$ (not a general result). It finishes slightly lower as well, but the model spend almost no time in that part of the state space.}
\end{figure}
\end{frame}

\begin{frame}{Drift in $m$-space}
\begin{figure}
	\includegraphics[scale=0.25]{figures/compStatNC/noNC/drift_sources_t.png}
	\includegraphics[scale=0.25]{figures/compStatNC/NC/drift_sources_t.png}
	\caption{\footnotesize Drift starts out slightly higher in the enforcement regime, but increases in the non-enforcement between innovations, since incumbents increase their R\&D.}
\end{figure}
\end{frame}

\begin{frame}{Comparing stationary distributions $\mu$}
\begin{figure}
	\includegraphics[scale=0.25]{figures/compStatNC/noNC/mu_vs_t_plots.png}
	\includegraphics[scale=0.25]{figures/compStatNC/NC/mu_vs_t_plots.png}
	\caption{\footnotesize The mass in $\mu$ shifts closer to 0 as overall innovation increases.}
\end{figure}
\end{frame}


\appendix

\begin{frame}{Microfoundation of external spinouts}
\begin{itemize}
	\item Worker has time 1 for external spinout formation, which can be targeted, but costs $q/Q$ units when targeted at $q$
	\item CRS would imply all target $m = 0$, DRS implies non-degenerate distribution
	\item End up with 
\end{itemize}
\end{frame}

























\end{document}