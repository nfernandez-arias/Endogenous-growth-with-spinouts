\documentclass[11pt,english]{article}
\usepackage{lmodern}
\linespread{1.05}
%\usepackage{mathpazo}
%\usepackage{mathptmx}
%\usepackage{utopia}
\usepackage{microtype}





\usepackage[section]{placeins}
\usepackage[T1]{fontenc}
\usepackage[latin9]{inputenc}
\usepackage[dvipsnames]{xcolor}
\usepackage{geometry}

\usepackage{babel}
\usepackage{amsmath}
\usepackage{graphicx}
\usepackage{amsthm}
\usepackage{amssymb}
\usepackage{bm}
\usepackage{amsfonts}

\usepackage{accents}
\newcommand\munderbar[1]{%
	\underaccent{\bar}{#1}}


\usepackage{svg}
\usepackage{booktabs}
\usepackage{caption}
\usepackage{blindtext}
%\renewcommand{\arraystretch}{1.2}
\usepackage{multirow}
\usepackage{float}
\usepackage{rotating}
\usepackage{mathtools}
\usepackage{chngcntr}

% TikZ stuff

\usepackage{tikz}
\usepackage{mathdots}
\usepackage{yhmath}
\usepackage{cancel}
\usepackage{color}
\usepackage{siunitx}
\usepackage{array}
\usepackage{gensymb}
\usepackage{tabularx}
\usetikzlibrary{fadings}
\usetikzlibrary{patterns}
\usetikzlibrary{shadows.blur}

\usepackage[font=small]{caption}
%\usepackage[printfigures]{figcaps}
%\usepackage[nomarkers]{endfloat}


%\usepackage{caption}
%\captionsetup{justification=raggedright,singlelinecheck=false}

\usepackage{courier}
\usepackage{verbatim}
\usepackage[round]{natbib}
\bibliographystyle{plainnat}

\definecolor{red1}{RGB}{128,0,0}
%\geometry{verbose,tmargin=1.25in,bmargin=1.25in,lmargin=1.25in,rmargin=1.25in}
\geometry{verbose,tmargin=1in,bmargin=1in,lmargin=1in,rmargin=1in}
\usepackage{setspace}

\usepackage[colorlinks=true, linkcolor={red!70!black}, citecolor={blue!50!black}, urlcolor={blue!80!black}]{hyperref}
%\usepackage{esint}
\onehalfspacing

\theoremstyle{remark}
\newtheorem{remark}{Remark}
\begin{document}
	
\title{Creating Creative Destruction: Endogenous Growth with Employee Spinouts and Non-compete Agreements}

\author{Nicolas Fernandez-Arias} 
\date{\today \\ \small
	\href{https://drive.google.com/file/d/11n9nHuGGPFOg_Hb_tr51DqG0h6vpaZZv/view?usp=sharing}{Click for most recent version}}
\maketitle


%\setcounter{tocdepth}{2}
%\tableofcontents

\begin{abstract}
	I study the effect of non-compete agreements (NCAs) on aggregate productivity growth. I first develop an augmented quality ladders model of endogenous growth with NCAs and within-industry spinouts (WSOs). Next, I assemble a new dataset of VC-funded startups matched to the previous employers of their founding members. Using these data I find suggestive evidence of a causal relationship between corporate R\&D and employee spinout formation which quantitatively can account for roughly 50\% of WSOs in the data. I calibrate the model to match the micro estimates, aggregate moments and estimates from the literature. According to the calibrated model, fully enforcing NCAs can increase welfare by approximately 1.5\% in consumption equivalent terms. R\&D subsidies can reduce growth and welfare by misallocating R\&D to entrants engaging in creative destruction. The optimal policy is a combination of R\&D subsidies targeted at own-product innovation and a ban on the use of NCAs. I close with a discussion of the barriers to practically implementing the optimal policy.
\end{abstract}

\section{Introduction}

Knowledge spillovers and firm entry are both major contributors to aggregate productivity growth. Firm entry in turn is often the result of knowledge spillovers from existing firms. In particular, within-industry employee spinouts (WSOs) -- new firms founded by former employees of incumbent firms in the same industry -- often take advantage of knowledge gained at previous employers. Figure \ref{fairchild_spinouts} shows the many direct and indirect spinouts of Fairchild Semiconductor, one of the first leading semiconductor firms of Silicon Valley -- itself a spinout of Shockley Laboratories, another semiconductor firm. Although Fairchild was founded in the 1950s, its list of spinouts includes some of the most well-known modern firms in the industry, such as Intel and AMD. 

\begin{figure}	\phantomsection
	\center
	\includegraphics[scale = 0.77]{../figures/fairchildren_early.png}
	\caption{Direct and indirect spinouts of Fairchild Semiconductor}
	\label{fairchild_spinouts}
\end{figure}


To avoid the possibility of competition from such firms, incumbents may reduce their investment in R\&D and other forms of costly knowledge creation. Alternatively, they may take steps to prevent WSOs directly, mitigate the disincentive to R\&D restricting productivity-enhancing knowledge spillovers. The most salient example of this kind of effort is the non-compete agreement (NCA), an employment precluding the employees from founding a competing firm after ceasing his or her current employment until a prespecified amount of time has passed. Given the aforementioned tradeoff, it is not clear what the effect of NCAs is on aggregate productivity growth, nor is it clear how the answer to this question depends on structural parameters that may be different in different locations, industries or time periods. Further, from a normative perspective, it is natural to ask whether it is socially optimal to permit the free use of NCAs.

This paper is an attempt to provide quantitative answers to these questions. To do so, I first develop a tractable model of endogenous growth which augments the standard quality ladders framework in \cite{acemoglu_introduction_2009} to include WSOs and NCAs. I then construct and analyze a micro dataset of incumbent firms and startups, providing evidence for a causal relationship between parent firm R\&D and subsequent startup formation and find suggestive evidence of an economically meaningful causal relationship. The model is then calibrated using aggregate statistics and the microeconomic relationship between R\&D and employee entrepreneurship. Using the calibrated model, I study the effect of varying the barriers to enforcement of NCAs and describe the model-implied optimal policy. I find that eliminiation of all barriers to NCAs can increase welfare by approximately 1.5\% in consumption-equivalent terms. R\&D subsidies can have the counterintuitive effect of reducing growth by misallocating R\&D labor to entrants instead of incumbents.\footnote{This stems in part from an assumption of a fixed stock of R\&D labor. In a fuller model, this mechanism would simply dull the growth-enhancing effect of R\&D subsidies. I plant to extend the model in that direction.} R\&D subsidies targeted at own-product innovation can work well in tandem with a ban on NCAs, but are difficult if not impossible to implement in practice.

The model consists of a standard general equilibrium model of endogenous growth with creative destruction, augmented to include employee entrepreneurship and NCAs. The model takes as given that, absent NCAs, R\&D employees eventually gain the knowledge to form a competing WSO. Given the R\&D wage, this reduces the incentive for R\&D. However, in equilibrium, R\&D workers accept a lower wage, as they internalize the expected future profits WSOs they form. In this sense they "pay" ex-ante for the damage they will cause. Whether their payment is sufficient to fully mitigate the disincentive therefore depends on whether the value gained by the founder of the WSO is larger than the value lost by the incumbent. In fact, NCAs are used only if WSO formation does not maximize the ex-ante joint value of the employment relationship. 

In order for the model to generate a role for NCAs, there must be some other contracting friction or in the employer-employee relationship.  Otherwise, the incumbent could offer to buy any WSOs and shut them down. Ex-ante, the employee accepts a lower wage and the firm conducts R\&D as though it had imposed an NCA (and pays a higher wage). private information concerning the quality of the idea, disagreements between the employee and the employer concerning the idea's quality, and the lack of commitment power on the part of the employee (i.e., the employee cannot commit not to implement the idea even after notionally selling it to his employer). I leave the particular friction unmodeled, simply assuming that there is no market in which WSOs can be sold to the incumbent firm that generated them.\footnote{In future work, I plan to explore this area further.}

The result is a model in which WSOs expand the innovation possibilities frontier of the economy while having an ambiguous effect on equilibrium innovation and productivity growth. The freedom to use NCAs, which prevent WSOs, also may increase or decrease the equilibrium growth rate. To go further, some discipline needs to be imposed on the model parameters.

To do so, I assemble a dataset of parent firms and startups founded by their employees by combining Compustat data on publicly traded firms and Venture Source data on VC-funded startups and their founders. While the Venture Source data has limitations, it is the only dataset of startups with information on the most recent employer of the startup's key employees. Matching these data is somewhat challenging since there are no company identifiers across datasets: the match must be done by name only. This is non-trivial since companies go by different names. I solve this problem by using string matching techniques (e.g., regular expressions), Compsutat data on subsidiaries and finally the merchant-mapper tool by Alternative Data Group, a startup that links credit card transactions data to firms using machine learning (itself a spinout of 1010 Data). I define a startup as a spinout if its CEO, CTO, President, Chairman or Founder (1) was most recently employed at a firm in Compustat and (2) joined the startup in its first three years. Using this definition, I identify approximately 3,000 WSOs in the data. Finally, I match this dataset to data on all US patents taken from the NBER-USPTO patent database.

Figures \ref{figure:scatterPlot_RD-Founders_dIntersection} and \ref{figure:scatterPlot_RD-FoundersWSO4_dIntersection} illustrate the primary motivation for this paper: firm-level R\&D is associated with subsequent employee entrepreneurship. To be precise, the x-axis shows firm-level average R\&D spending over periods $t,t-1,t-2$ and the y-axis shows firm-level average yearly number of employee founders from that firm in $t+1,t+2,t+3$. Both of these variables are then purged sequentially of their firm- and state-industry-age-year means. Figure \ref{figure:scatterPlot_RD-FoundersWSO4_dIntersection} confirms this when restricting attention to employee entrepreneurship in the same 4-digit industry of the initial employer. 

\begin{figure}[]
	\centering
	\includegraphics[scale= 0.5]{../empirics/figures/scatterPlot_RD-Founders_dIntersection.png}
	\caption{Scatterplot of average yearly founder counts in $t+1,t+2,t+3$ versus average yearly R\&D spending in $t,t-1,t-2$.}
	\label{figure:scatterPlot_RD-Founders_dIntersection}
\end{figure}

\begin{figure}[]
	\centering
	\includegraphics[scale= 0.5]{../empirics/figures/scatterPlot_RD-FoundersWSO4_dIntersection.png}
	\caption{Scatterplot of average yearly founder counts (restricted to same 4-digit NAICS industry) in $t+1,t+2,t+3$ versus average yearly R\&D spending in $t,t-1,t-2$.}
	\label{figure:scatterPlot_RD-FoundersWSO4_dIntersection}
\end{figure}

However, a simple regression of number of employee spinouts on lagged R\&D spending suffers from omitted variable bias, as factors such as changes in demand or changes in technological investment opportunities likely affect both variables in the same direction. To control for this, I use firm, state-year, NAICS 4 digit industry-year (at 4-digit NAICS level), and firm age fixed effects, as well as firm-specific controls, such as employment, assets, Tobin's Q, and citation-weighted patents. The resulting estimates vary by specification obut are typically statistically and economically significant. According to these estimates, R\&D can account for roughly 50\% of WSOs in the data. 

I next calibrate the model using the estimates above as well as aggregate statistics and growth accounting estimates from \cite{garcia-macia_how_2019} and \cite{klenow_innovative_2020}. I also choose some parameters from the literature. I use the calibrated model to study the effect on productivity growth and welfare of varying the cost of using NCAs. I discuss how these results depend on the parameters and, via the calibration, on the value of the targeted moments. Finally, I discuss other policies that could improve welfare in this context.

\paragraph{Related literature}

Some work has attempted to answer this question directly using empirical methods. Papers in this literature have typically used either cross-sectional and/or longitudinal variation in the state-level enforcement of non-competes.\footnote{Sometimes this variation is argued to be exogenous, either due to legislative error as in \cite{marx_mobility_2009} and \cite{marx_regional_2015}, or due to unexpected judicial precedent as in \cite{jeffers_impact_2018}. Often there is a control industry that is believed to be unaffected by the variation in CNC enforcement policy (e.g. law firms are typically exempt from CNC restrictions).} The results are inconclusive and suggest an important tradeoff between entry of spinouts and investment by incumbent firms. \cite{stuart_liquidity_2003} find more local  entrepreneurship in response to local IPO (a "liquidity event") in regions not enforcing CNCs. \cite{marx_mobility_2009} finds that inventor mobility declines in response to an increase in non-compete enforcement. \cite{samila_venture_2010} finds that an increase in VC funding supply increases entrepreneurship more in states without non-compete restrictions, using an IV design. \cite{garmaise_ties_2011} finds that, in states where CNCs are more enforceable, managers are less mobile, have lower compensation, and invest less in their human capital, to the point of offsetting increased investments by the firm. On the other hand, \cite{conti_non-competition_2014} finds evidence that non-compete enforceability leads to incumbent firms pursuing riskier R\&D projects. \cite{colombo_does_2013} finds evidence that easier spinout formation -- proxied by access to finance -- leads to a reduction in incumbent firm knowledge investments.  Most recently, \cite{jeffers_impact_2018} uses data on influential state-level court precedents matched with LinkedIn data and finds that enforcement indeed reduces spinout formation while increasing capital investment by incumbent firms. Finally, \cite{marx_regional_2015} finds that CNC enforcement leads to inventor mobility out of the state, suggesting that differences in outcomes could be in part due to reallocation. 

Theoretical work has also explored this question. As mentioned previously, \cite{franco_spin-outs:_2006} develops a model in which employees learn from their employers and use this knowledge to form spinouts. They emphasize the "paying for knowledge" effect, whereby employees implicitly pay for the knowledge they take from the parent firm through lower equilibrium wages. Importantly, they assume spinout firms do not steal business from their parents: the only effect of a spinout on the parent firm is a reduction in the price of the output good, which the parent firm is assumed not to take into account. This, combined with the "paying for knowledge" mechanism, ensures that the competitive equilibrium allocation is Pareto efficient, even without resorting to elaborate labor contracts.

\cite{franco_covenants_2008} studies a two-period, two-region model with employee spinouts in which the region which does not enforce CNCs initially lags but eventually overtakes the region in which CNCs are enforced. In the first period, entry is more valuable in the enforcing region. But in the second period, spinouts enter in the non-enforcing region, there is Cournot competition with parent firms in the product market, and output increases relative to the enforcing region. The analysis emphasizes how asymmetric information about whether an employee has learned leads some firms in the non-enforcing region to allow spinouts (assuming firms cannot commit to wage backloading). This can be taken as a rough microfoundation of my assumption that labor contracts are "simple" in  a non-enforcing region: just a wage, with no attempts at retention in the case of learning. Relative to this study, my analysis considers a fully dynamic model rather than two-period model. In addition, I emphasize the role of R\&D investment in spawning spinout firms.

\cite{shi_restrictions_2018} uses a rich model of contracting disciplined by data on executive non-compete contracts to study the effect of non-competes on executive mobility and firm investment. She finds that the optimal policy is to somewhat restrict the permitted duration of CNCs. Her approach allows her to study the optimal contracting problem in more detail than in mine. However, she is mainly interested in poaching, which involves an attempt to extract a payout from the poaching firm. Also, her calibration considers firm investment in capital expenditures, whereas I am interested in innovative investment in R\&D.

\cite{baslandze_spinout_2019}, the study closest to this paper, studies the effect of spinout entrepreneurship on entry and growth. She also uses a GE model of endogenous growth with employee spinouts, using Compustat and NBER-USPTO patent data to discipline the analysis. She finds the optimal policy is to ban NCAs. However she is focused in her framework on the harm from losing a valuable employee rather than the harm from competition with the parent firm. My paper focuses instead on creative destruction of the parent firm using knowledge rather than the loss of productivity from losing valuable employees.  The other key difference is that I model the use of NCAs while her analysis assumes that they are used when available. To my knowledge, mine is the first general equilibrium model of endogenous growth to have this kind of feature.

\cite{babina_entrepreneurial_2019} find evidence of a causal relationship from corporate R\&D spending to employee spinout formation. My empirical analysis confirms their findings on a subset of firms particularly connected with productivity growth, VC-funded startups. Together, they motivate the use of a model like the one developed in this paper.


\section{Model}

\subsection{Individual endowments and preferences}

The model is in continuous time, starting at $t = 0$. The representative household has CRRA preferences over consumption, given by\footnote{There is no expectation operator use there is no aggregate uncertainty in this setting (more on this in later sections).}
\begin{align}
U_t &= \int_0^{\infty} e^{-\rho s} \frac{C(t+s)^{1-\theta} - 1}{1-\theta} ds \label{preferences}
\end{align}

In each period $t \ge 0$, the household is endowed with $\bar{L}_{RD} \in (0,1)$ units of R\&D labor as well as $1 - \bar{L}_{RD}$ units of production labor which can be used to make the final good $(L_F)$ or the intermediate goods $(L_I)$. The household therefore chooses $L_{RD},L_F,L_I$ subject to the resource constraints
\begin{align}
L_{RD} &\le \bar{L}_{RD} \label{labor_resource_constraint2} \\
L_F + L_I &\le 1 - \bar{L}_{RD} \label{labor_resource_constraint} 
\end{align}

\subsection{Production of final and intermediate goods} \label{subsec:staticproduction}

The final good $Y$ is produced competitively using labor and a continuum of intermediate goods indexed by $j \in [0,1]$. In turn, at any time $t \ge 0$, the intermediate good $j$ is available in $n_{jt}$ different qualities $\{\lambda^0,\lambda^1,\ldots,\lambda^{n_{jt}}\}$, where $\lambda > 1$ is exogenous and $n_{jt}$ is endogenous and determined by R\&D investment, described in detail in Section \ref{subsec:innovation}. As a matter of notation, let $\bar{q}_{jt} = \lambda^{n_{jt}}$ denote the frontier quality of good $j$ at time $t$, and similarly let $\bar{k}_{jt}$ denote its quantity. Below I suppress the $t$ subscript where it is clear. 

The final goods production technology is given by \footnote{Intermediate goods are aggregated in a CES form with an elasticity of substitution greater than 1, rather than the Cobb-Douglas form in e.g., \cite{grossman_quality_1991} and \cite{baslandze_spinout_2019}. This reduces the complexity of the firm problem. In those models, Cobb-Douglas implies that expenditure on each intermediate good is constant in quality. This requires limit pricing to be explicitly modeled, otherwise increasing the price always increases profits and the firm problem is not well-defined. To model limit pricing, one must track the gap between leader and follow in each good $j$, adding a state variable to the firm problem and to the aggregation of the model. In the current setup, by contrast, expenditure is decreasing in the price of the intermediate good, so even if one abstract from limit pricing (microfoundation below), intermediate goods firms have a constant optimal markup. In the full model, I will take advantage of this reduced complexity by introducing more complexity in the employee spinout and firm entry process.}
\begin{align}
Y = F(L_F,\{n_j\},\{k_{ji}\}) &= \frac{L_F^{\beta}}{1-\beta} \int_0^1 \Big(\sum_{i = 0}^{n_{j}} (\lambda^{i})^{\frac{\beta}{1-\beta}} k_{ji} \Big)^{1-\beta} dj \label{final_goods_production}
\end{align}

where $k_{ji} \ge 0$ for $0 \le i \le n_j$ is the quantity used of intermediate good $j$ of quality $\lambda^i$. There is no storage technology for the final good and its price is normalized to 1 in every period. 

Intermediate goods $j$ of any quality $\lambda^i$ are produced using the technology is given by
\begin{align}
k_{ji} = H(\ell_{ji};Q) &= Q \ell_{ji} \label{intermediate_goods_production}
\end{align}
where $\ell_{ji} \ge 0$ is the labor input and $Q = \int_0^1 \bar{q}_{j} dj$ is the average frontier quality level in the economy.\footnote{The linear scaling with the aggregate economy $Q$ is to ensure a BGP, given that the total quantity of labor stays the same over time. It is analogous to assuming a constant marginal cost in a model where the final good, rather than labor, is the input of intermediate goods production.} Each quality of good $j$ is produced by a firm which has a monopoly on that quality of good $j$. 

Note that (\ref{final_goods_production}) implies that different qualities of good $j$ are perfect substitutes in final goods production and (\ref{intermediate_goods_production}) means intermediate goods production functions have constant returns to scale. Later I will specify that producers of a given intermediate good $j$ engage in Bertrand competition with each other. Together this implies that, in equilibrium, only the frontier quality $\bar{q}_{j}$ is used in final goods production. I refer to the producer of $j$ with access to the frontier technology $\bar{q}_{jt}$ as the \textit{incumbent} of good $j$. 

The above discussion leads to the more familiar representation of final goods production used in  \cite{acemoglu_introduction_2009},
\begin{align}
	Y = F(L_F,\{\bar{q}_j\},\{\bar{k}_j\}) &= \frac{L_F^{\beta}}{1-\beta} \int_0^1 \bar{q}_j^{\beta} \bar{k}_j^{1-\beta} dj  \label{eq_final_goods_production}
\end{align}

\subsection{Innovation}\label{subsec:innovation}

There are three types of innovation in this economy. First, intermediate goods producers at the frontier (those with the technology to produce with quality $\bar{q}_{jt}$ at time $t$, hereafter known as incumbents) can expend R\&D to improve on their own product (own innovation, or OI). Second, R\&D by incumbents leads to creative destruction (CD) by spinouts (founded by their R\&D employees). Third, entrants can expend R\&D to do their own innovation which, if successful, allows them to replace an incumbent via creative destruction.

An innovation on good $j$ of frontier quality $\bar{q}_{jt}$ yields for the innovator a perpetual patent on the production of good $j$ of quality $\lambda q_{jt}$, where $\lambda > 1$ is the exogenous quality ladder step size. Crucially, this perpetual patent does not prevent spinouts or entrants from leap-frogging the incumbent with an even better innovation. Further, in addition to receiving a perpetual patent on the production of the good at the new frontier quality, a successful innovator gains access to the R\&D technology for OI (see \ref{subsubsec:OI}).
%
I assume that an incumbent who is overtaken by an entrant or spinout is unable to "catch up" to the frontier by incrementally improving her own obsolete version of good $j$. Once a producer stops producing, she stops producing forever. The assumption of no catch-up growth is standard -- and usually not made explicit -- in quality ladders models with innovation by incumbents, like the one described in Acemoglu section 14.3. One possible interpretation is that access to the own-innovation technology requires current production of the good due to learning by doing. In any case, the assumption dramatically simplifies the analysis.\footnote{Without it, the incumbent problem would have an additional state variable (since falling away from the frontier is no longer an absorbing state) and an additional distribution would need to be tracked (the number of incumbents with the technology to produce each infra-frontier good $j$). The setting is so intractable that even papers focused on this mechanism, such as \cite{aghion_competition_2005}, make extreme simplifying assumptions analogous to mine.} Entrants can be allowed to innovate on any quality good, as in equilibrium they will choose to innovate only on the frontier goods.

 
Below, I omit the dependence on $t$ of equilibrium variables. 

\subsubsection{Own-product innovation by incumbents} \label{subsubsec:OI}

The incumbent in good $j$ of quality $\bar{q}_{j}$ can perform $z_{I,j}$ units of R\&D effort by hiring $\frac{\bar{q}_{j}}{Q_t}z_{I,j}$ units of R\&D labor. In return, she receives a Poisson intensity of $\chi_I z_{I,j}$ of innovating on good $j$, where $\chi_I > 0$ is an exogenous parameter representing the incumbent's R\&D productivity. The choice of a CRS technology here is for the purpose of tractability.\footnote{Either incumbents or entrants must have CRS R\&D efficiency in order to maintain a simple model (i.e., otherwise an iterative procedure must be used to numerically compute optimal R\&S policies). Given that entrants are in competition with each other and therefore do not coordinate their R\&D, it is economically reasonable to assume that entrants, as a group, have more rapidly decreasing returns than the incumbent.}

Define
\begin{align}
	\tau_{I,j} &= \chi_I z_{I,j}
\end{align}

as the Poisson arrival rate of incumbent innovations.

\subsubsection{Generation of spinouts}

When an incumbent conducts $z$ flow units of R\&D, she faces a certain Poisson intensity of spawning a spinout, given by 
\begin{align*}
	\tau_{S,j} &= (1-x_{j}) \nu z_{I,j}
\end{align*} 
where $x_{j} = 1$ if and only if an NCA is used and where $\nu \ge 0$ is an exogenous parameter representing the rate at which incumbent R\&D produces employee spinouts. Spinouts spawned by (frontier) incumbents of quality $q_j$ have the ability to produce good $j$ of quality $\lambda q_j$. Such a spinout immediately becomes the new frontier incumbent and, recalling the "no catch-up" assumption in Section \ref{subsec:innovation}, the previous incumbent's profits go to zero forever after.

The parameter $\nu$ encodes the rate at which R\&D induces creative destruction of the incumbent by its employees' spinouts. It is meant to reflect both the rate at which employees accumulate the requisite human capital to form spinouts and the rate at which this human capital translates into successful creative destruction of the incumbent.

\paragraph{No idea stealing} Notice that I have assumed that the possibility of spinouts does not directly reduce the rate at which incumbent R\&D results in innovations for the incumbent. Instead, it simply adds an additional, independent Poisson process by which the incumbent can be replaced by an innnovating spinout. The interpretation is that R\&D employees forming spinouts in this model do not steal ideas that otherwise would have been implemented by the parent firm. Rather, when unbound by NCAs, R\&D employees generate \textit{additional} ideas which they can then use to steal the incumbent's profits. Economically, this assumption is reasonable because innovations which, under an NCA, would have been implemented by the incumbent are liable to, absent an NCA, be acquired from the R\&D employee and implemented by incumbent. If this is the case, the NCA is no longer necessary to prevent a bilaterally inefficient outcome -- its only effect in equilibrium is to determine \textit{when} the firm makes payments to the employee -- and NCAs are irrelevant to such spinouts.\footnote{Note that this assumption has implications for the efficiency consequences of spinouts and therefore of NCAs. If spinouts add less to the overall R\&D efficiency of the economy, then preventing their entry via NCAs has much more limited scope for improving welfare.}

\paragraph{Direct cost of using NCAs} 

If incumbent $j$ would like to use an NCA, she must pay a flow cost $\kappa_{c} \nu V_{j} z_{I,j}$ units of the final good, where $V_{j}$ is incumbent $j$'s value at time $t$. Taking into account that $x_j = 1$ iff incumbent $j$ uses an NCA, incumbent $j$ overall pays NCA enforcement flow cost
\begin{align*}
	\textrm{NCA cost}_{j} &= \tau_{S,j} \kappa_c V_{j}
\end{align*}

The NCA enforcement cost reflects the direct technological constraints of implementing NCAs. Even if NCAs were fully enforced by courts, the question of whether the employee is in fact competing with his former employer is something that must be determined through a potentially expensive legal process. In reality, state-level precedent and statues are such that court enforcement of non-competes depends on the non-competes meeting certain conditions as well, exacerbating this. In general, it seems likely to be the case that investing resources (e.g., highly paid lawyers, frequent and long lawsuits) increases their likelihood of a successful NCA; $\kappa_c$ reflects, in a reduce form way, this kind of firm investment. Finally, note that a value of $\kappa_c = \infty$ can be interpreted as statutory ban on the use of NCAs, such as in California (except in certain very limited cases outside the scope of this model). 

The specification in terms of the endogenous variable $V_j$ ensures that the optimal NCA policy can be computed without knowing the value of $V_j$ in equilibrium, simplifying the analysis of the model significantly.

\paragraph{Value of future spinout formation}

If a spinout is formed, it is owned by the representative household. The household takes this into account when deciding where to allocate its labor, accepting a lower wage in equilibrium for R\&D labor supplied to incumbents. However, when assessing this value, the household \textit{does not} take into account the fact that this spinout steals the profits of the previous incumbent, which is also owned by the household.\footnote{A possible microfoundation is to assume, instead of a representative household, a continuum of households each of which consists of a continuum of agents who fully insure each other against idiosyncratic risk (i.e. the risk of being / not being the one to open the spinout). The aggregate equilibrium variables of such a model are the same as those of the one presented here.} 



\subsubsection{Entrants} \label{subsubsec:entrants}

For each good $j$ there is a unit mass (normalization) of entrants indexed by $e \in [0,1]$ who each perform $z_{E,ej}(q_j)$ units of R\&D on good $j$ of quality $q_j \le \bar{q}_{j}$, which implies hiring $\frac{q_j}{Q_t}z_{E,ej}$ units of R\&D per entrant. Thus the scaling with the relative quality $\frac{q_j}{Q_t}$ is just as with frontier incumbent OI. In return for the R\&D expenditure $\frac{q_j}{Q_t} z_{E,ej}$, an entrant receives a Poisson intensity of $z_{E,ej} \chi_E z_{E,j}^{-\psi}$ of innovating on good $j$, where $z_{E,j} = \int_0^1 z_{E,ej} de$ denotes aggregate entrant effort on improving good $j$. Note that entrants have constant returns to scale at the individual level, but decreasing returns at the level of good $j$. Restricting attention to equilibria with no mass points, it is without loss of generality to assume that $z_{E,ej} \equiv z_{E,j}$ for all $e,j \in [0,1]$.\footnote{Alternatively, one can assume that entrants have individually decreasing returns. It follows that there are no mass points in any equilibria; symmetry then implies that all entrants choose the same effort.}

Finally, entrants only conduct frontier innovation in equilibrium. While innovating on a good $j$ below the technological frontier is in fact cheaper than frontier innovation, the "no catch-up" assumption made in Section \ref{subsec:innovation} implies that infra-frontier innovation yields nothing of value to the entrant. As such, it is without further loss of generality to assume that entrants can only attempt entry on the frontier quality $\bar{q}_j$, and I proceed under this assumption for expositional simplicity (that is, instead of solving the full model and finding that $z_{Ej}(q_j) = 0$ for $q_j < \bar{q}_j$).\footnote{This is analogous to the use of the simplified final goods production function that assumes only frontier goods $j$ are used.}

Define
\begin{align*}
	\tau_{E,j} &= \chi_E z_{E,j}^{1-\psi}
\end{align*}

as the arrival rate of innovations by entrants.


\subsubsection{Entry cost}

In addition to the R\&D costs of innovation, entrants and spinout must pay an entry cost $\kappa_{e} \lambda V_{j}$ when an innovation is discovered on good $j$, in order to become the incumbent, where $V_{j}$ is the value of incumbent $j$ before their innovation. Creative destruction requires additional non-R\&D expenditures relative to own innovation. These can be interpreted as the costs of setting up a new firm or of acquiring customers. \footnote{This feature of the model allows it to fit the rate of creative destruction job reallocation and the rate of R\&D spending simultaneously. Without it, the argument in Comin 2004 "R\&D: a small contribution to productivity growth", which applies to a degree in this setting, means that measured R\&D spending in the data is significantly lower than the model-implied value of discovering a new innovation. Because the model assumes free entry into R\&D, these must be equal in the estimated model, and therefore the model estimates either too high R\&D spending (in my case) or too low growth, as in that paper.}


\subsection{Decentralized equilibrium}

\subsubsection{Static equilibrium}

In this section, I omit the dependence on $t$ of all equilibrium variables. In addition, since only the frontier quality is relevant, I will drop the bar notation and refer to the frontier good by $q_j, k_j$.

Final goods producer optimization implies the following inverse demand function for intermediate goods, 
\begin{align*}
p_j &= L_F^{\beta} q_j^{\beta} k_j^{-\beta}	
\end{align*}

To continue computing the equilibrium of the model, the market structure for intermediate goods must be specified. 

\paragraph{Intermediate goods market structure} Different good $j$ incumbents compete against each other under monopolistic competition. Within each good $j$, intermediate goods producers play a two-stage Bertrand competition game. In the first stage, participants bear a cost of $\varepsilon > 0$ units of the final good in exchange for a right to compete in the second stage. Then, in the second stage, the engage in Bertrand competition. Optimal pricing under Bertrand competition in the second stage implies that all producers not on the frontier will earn zero profits. By backward induction, they do not pay the entry cost in equilibrium, and the leader has a second-stage monopoly over good $j$.\footnote{Without this assumption, there is limit pricing, and the markup charged by the technology leader in good $j$ would depend on his gap relative to the next laggard, e.g. \cite{baslandze_spinout_2019} or \cite{aghion_competition_2005}, only equating to the monopolistic competition markup for large enough gaps.}

The assumed market structure implies that the incumbents for each good $j$ face a standard monopolistic competition market structure and can effectively ignore other producers of their own good. They maximize profits according to
\begin{align}
\pi(q_j) = \max_{k_j \ge 0} \Big\{ L_F^{\beta} q_j^{\beta} k_j^{1-\beta} - \frac{\overline{w}}{Q} k_j \Big\} \label{incumbent_profit}
\end{align}

where $\overline{w}$ is the equilibrium final goods / intermediate goods wage.
This yields optimal pricing, labor demand and production of intermediate goods,
\begin{align}
k_j &= \Big[ \frac{(1-\beta) Q}{\overline{w}} \Big]^{1/\beta}L_F q_j  \label{optimal_k}\\
\ell_j &= k_j / Q \label{optimal_l}\\
p_j &= \frac{\overline{w}}{(1-\beta) Q} \label{optimal_p}
\end{align}

Substituting (\ref{optimal_k}) into the first-order condition for final goods firm optimal labor demand yields a closed form expression for the equilibrium wage $\overline{w}$:
\begin{align}
\overline{w} &= \tilde{\beta} Q \label{wbar} \\
\tilde{\beta} &= \beta^{\beta} (1-\beta)^{1-2\beta} \label{def_cbeta}
\end{align}

Substituting (\ref{optimal_k}) and (\ref{wbar}) into the expression for profit in (\ref{incumbent_profit}) yields
\begin{align}
\pi_j &= (1-\beta) \tilde{\beta} L_F q_j \label{profits_eq}
\end{align}

Substituting (\ref{optimal_k}) into (\ref{optimal_l}) and integrating $L_I = \int_0^1 l_j dj$ yields aggregate labor allocated to intermediate goods production,
\begin{align}
L_I &= \Big( \frac{1-\beta}{\tilde{\beta}} \Big)^{1 / \beta} L_F \label{intermediate_goods_labor}
\end{align}

and substituting (\ref{intermediate_goods_labor}) into the labor resource constraint (\ref{labor_resource_constraint}) yields
\begin{align}
L_F &= \frac{1 - \bar{L}_{RD}}{1 + \Big(\frac{1-\beta}{\tilde{\beta}}\Big)^{1/\beta}}
\end{align}

Output can be computed by substituting (\ref{optimal_k}) into (\ref{final_goods_production}), 
\begin{align}
Y = \frac{(1-\beta)^{1-2\beta}}{\beta^{1-\beta}} Q L_F \label{flow_output}
\end{align}

\subsubsection{Dynamic equilibrium}\label{subsubsec:dynamic_equilibrium_original_solution}

I will solve for a BGP of the above model with constant innovation effort by incumbents ($z_{I,jt} = z_I$) and entrants ($z_{E,ejt} = z_{E})$, constant NCA policy by incumbents ($x_{jt} = x$), constant innnovation rates by incumbents ($\tau_{I,jt} = \tau_I$), entrants ($\tau_{E,ejt} = \tau_E$) and spinouts ($\tau_{S,jt} = \tau_S$), a constant growth rate of output, consumption and average intermediate goods quality ($g_t = g$) and constant interest rate ($r_t = r$), and wages increasing at exponential rate $g$ ($\bar{w}_t = \bar{w} e^{gt}$, $w_{RD,Et} = w_{RD}e^{gt}$, and $w_{RD,jt} = w_{RD,I}e^{gt}$). One can verify that along such a BGP there exists $\tilde{V} > 0$ such that the value of an incumbent firm of quality $q$ at time $t$ is $V(q,t) = \tilde{V}q$. Using a guess and verify approach, I start with this functional form and solve for $\{z_I,z_E,\tau_I,\tau_E,\tau_S,g,r,\bar{w},w_{RD,E},w_{RD,I},\tilde{V}\}$. 

\paragraph{Household optimization and non-competes}

For each $j$ the household chooses $\ell_{RD,j}$ such that
\begin{align}
\int_0^1 \ell_{RD,Ij} dj + L_{RD,E} = \bar{L}_{RD}
\end{align}
where $L_{RD,E}$ is R\&D labor supplied to entrants (aggregated over all $j \in [0,1]$).

In any equilibrium where entrants and incumbents both perform R\&D, the household must be indifferent between supplying R\&D labor to different firms. Inada conditions on entrants' innovation technology guarantee that $z_E > 0$ in equilibrium and hence that the flow expected value to the household from R\&D employment at an incumbent must also be equal to $w_{RD,E}$, the wage paid by entrants for R\&D. Given $V(q_j,t) = \tilde{V}q_j$ and the rate $\frac{Q}{q_j}$ of spinout formation implies an indifference condition, 
\begin{align}
	w_{RD,E} &= w_{RD,j} + (1-x_j) \nu (1-\kappa_e) \lambda \tilde{V} \label{eq:RD_worker_indifference}
\end{align}

\paragraph{Equilibrium innovation}

As in any market with perfect competition, the incumbent faces a perfectly elastic supply curve for R\&D labor, and can acquire as much as she likes provided she offers total compensation (wage + value of future spinouts) required by the indifference condition (\ref{eq:RD_worker_indifference}). The incumbent takes this into account when deciding whether to offer contracts with an NCA ($x = 1$) or without an NCA ($x = 0$). Her HJB is therefore given by 
\begin{align}
	(r + \tau_E) \tilde{V} &= \tilde{\pi} + \max_{\substack{x \in \{0,1\} \\ z \ge 0}} \Big\{z \Big( \overbrace{\chi_I (\lambda - 1) \tilde{V}}^{\mathclap{\mathbb{E}[\textrm{Benefit from R\&D}]}}- w_{RD,E} -  \underbrace{(1-x)(1 - (1-\kappa_{e})\lambda)\nu \tilde{V}}_{\mathclap{\text{Net cost from spinout formation}}} - \overbrace{x \kappa_{c} \nu \tilde{V}}^{\mathclap{\text{Direct cost of NCA}}}\Big) \Big\} \label{eq:hjb_incumbent_workerIndiff}
\end{align}

The term $\chi_I(\lambda -1) \tilde{V}$ is the expected benefit per unit of R\&D effort. Notice the factor $\lambda -1$, which takes into account the opportunity cost of no longer producing with the obsolete technology. The term $-w_{RD,E}$ reflects the cost of R\&D effort due to the contribution from the prevailing R\&D wage. The term $-(1-x)(1 - (1-\kappa_e) \lambda) \nu \tilde{V}$ represents the expected net harm to the incumbent due to spinouts from the employee. Expanding this term, the term $-(1-x)\nu \tilde{V}$ reflects the direct harm from creative destruction by spinouts. The second term $(1-x)(1-\kappa_e)\lambda \nu \tilde{V}$ reflects the reduction in R\&D wage accepted by the R\&D employee in return for being free to open spinouts. Finally, the term $-x \kappa_c \nu \tilde{V}$ reflects the direct cost of enforcing NCAs.

Let $\bar{\kappa}_c (\kappa_e, \lambda) = 1 - (1-\kappa_e)\lambda$. If $z > 0$, the above implies the incumbent's optimal noncompete policy is given by\footnote{If $z_I = 0$, the incumbent is indifferent between $x \in\{0,1\}$, but this choice is irrelevant for the equilibrium allocation and prices.} 
\begin{align}
x = \begin{cases}
1 & \textrm{if } \kappa_{c} < \bar{\kappa}_c (\kappa_e, \lambda) \\
0 & \textrm{if } \kappa_{c} > \bar{\kappa}_c (\kappa_e, \lambda)\\
\{0,1\} & \textrm{if } \kappa_c = \bar{\kappa}_c (\kappa_e, \lambda) 
\end{cases} \label{eq_nca_policy}
\end{align}

Note that if $\kappa_c = \bar{\kappa}_c$, the incumbent is indifferent between $x = 0$ and $x = 1$.

\paragraph{Plan for solving the model}

I solve the rest of the model in three cases, according to whether we are in the $x = 1$, $x = 0$ or $x \in \{0,1\}$ regimes. In each case, first I assume that $z_I > 0$ in equilibrium and compute the candidate equilibrium prices and allocation that result. If this computed allocation has $z_I > 0$, then it is an equilibrium. Otherwise, we know that $z_I = 0$ in equilibrium, which gives $z_E = \bar{L}_{RD}$, and the remainder of the equilibrium can be computed in the same way, with the R\&D wage $w_{RD,E}$ falling to ensure equilibrium in the R\&D labor market. 

\paragraph{Case 1: $\kappa_c < \bar{\kappa}_c(\kappa_e,\lambda)$ and $x = 1$}

Equation (\ref{eq:RD_worker_indifference}) implies $w_{RD,j} = w_{RD,E}$ and the incumbent's HJB is given by 
\begin{align}
(r + \tau_E) \tilde{V} &= \tilde{\pi} + \max_{z \ge 0} \Big\{z \big(\chi_I (\lambda - 1) \tilde{V} - w_{RD,E} - \kappa_{c} \nu \tilde{V}\big) \Big\} \label{eq:hjb_incumbent}
\end{align}

Assuming an interior solution $z_I > 0$, the incumbent's FOC implies that, in equilibrium, the term multiplying $z$ in (\ref{eq:hjb_incumbent}) must equal zero,
\begin{align*}
	0 &= \chi_I(\lambda-1)\tilde{V}- w_{RD,E} - \kappa_c \nu \tilde{V}
\end{align*}

Solving for $\tilde{V}$ yields
\begin{align}
	\tilde{V} &= \frac{w_{RD,E}}{\chi_I(\lambda - 1) - \kappa_{c} \nu} \label{eq:hjb_incumbent_foc}
\end{align}

Entrant innovation satisfies a free entry condition,\footnote{The original condition is (with some dependence on $j$ and $t$ omitted for clarity), 
	\begin{align*}
		\chi_E z_E^{-\psi} (1-\kappa_e) \lambda V(q,t) = \tilde{q}_t w_{RD,E,t}
	\end{align*}
	where $\tilde{q}_t = \frac{q}{Q_t}$. Using $V(q,t) = q \tilde{V}$ and $w_{RD,E,t} = w_{RD,E} Q_t$ yields (\ref{eq:free_entry_condition}).}
\begin{align}
	\underbrace{\chi_E z_E^{-\psi}}_{\mathclap{\text{Marginal innovation rate}}} \overbrace{(1-\kappa_e) \lambda \tilde{V}}^{\mathclap{\text{Payoff from innovation}}} &= \underbrace{w_{RD,E}}_{\mathclap{\text{MC of R\&D}}} \label{eq:free_entry_condition}
\end{align}

Substituting $\tilde{V}$ using (\ref{eq:hjb_incumbent_foc}) yields an expression for entrant R\&D effort, 
\begin{align}
	z_E &= \Big( \frac{\chi_E (1-\kappa_{e}) \lambda}{\chi_I(\lambda-1) - \kappa_c \nu } \Big)^{1/\psi} \label{eq:effort_entrant}
\end{align}

Market clearing for R\&D labor requires
\begin{align}
	\bar{L}_{RD} &= \int_0^1 \frac{q_j}{Q} (z_{I} + z_{E}) dj = z_I + z_E
\end{align}
 
which implies
\begin{align}
	z_I &= \bar{L}_{RD} - z_E \label{eq:zI_asFuncZe}
\end{align}

Growth is determined by the growth accounting equation\footnote{To see this, compute...}
\begin{align}
g &= (\lambda - 1)(\tau_I + \tau_S + \tau_E) \label{eq:growth_accounting}
\end{align}

The Euler equation and $g = \frac{\dot{C}}{C}$ determine the interest rate, 
\begin{align}
	g &= \frac{\dot{C}}{C} = \frac{1}{\theta} (r - \rho) \label{eq:euler} \\
	\Rightarrow r &= \theta g + \rho \nonumber
\end{align}

Substituting the incumbent's FOC into the incumbent's HJB, and using the expression for the interest rate, yields the incumbent's value $\tilde{V}$,
\begin{align}
	 \tilde{V} &= \frac{\tilde{\pi}}{r + \tau_E}
\end{align}

Finally, the free entry condition (\ref{eq:free_entry_condition}) determines the equilibrium value of $w_{RD,E}$.

If (\ref{eq:effort_entrant}) implies that $z_E > \bar{L}_{RD}$, then in equilibrium $z_E = \bar{L}_{RD}$ and $z_I = \tau_I = \tau_S = 0$.\footnote{In fact, (\ref{eq:effort_entrant}) characterizes when this occurs as a function of the model parameters.} Then $g = (\lambda - 1) \tau_E$ and the rest of the equilibrium allocation and prices can be computed in the same way as before. 

\paragraph{Case 2: $\kappa_c > \bar{\kappa}_c(\kappa_e,\lambda)$ and $x = 0$}
Proceeding as before, the incumbent's HJB is now given by 
\begin{align}
	(r + \tau_E) \tilde{V} &= \tilde{\pi} + \max_{z \ge 0 } \Big\{ z \Big( \chi_I (\lambda - 1) \tilde{V} - w_{RD,E} - (1 - (1-\kappa_e)\lambda) \nu \tilde{V} \Big)  \Big\}\label{eq:hjb_incumbent_noNCA}
\end{align}

In a manner analogous to Case 1, (\ref{eq:hjb_incumbent_noNCA}) and $z_I > 0$ imply, via the incumbent FOC, that
\begin{align}
\tilde{V} &= \frac{w_{RD,E}}{\chi_I(\lambda - 1) - (1-(1-\kappa_{e})\lambda)\nu} \label{eq:hjb_incumbent_foc_noNCA}
\end{align}

As before, free entry and the incumbent FOC yield entrant R\&D effort, 
\begin{align}
	z_E &= \Big( \frac{\chi_E (1-\kappa_{e}) \lambda}{\chi_I(\lambda-1) - (1-(1-\kappa_e)\lambda)\nu } \Big)^{1/\psi} \label{eq:effort_entrant_case2}
\end{align}

Growth $g$ is computed as before,
\begin{align}
g &= (\lambda - 1)(\tau_I + \tau_S + \tau_E) \label{eq:growth_accounting_noNCA}
\end{align}
where now $\tau_S > 0$ because $x = 0$. Similarly, the Euler equation again returns the equilibrium interest rate 
\begin{align*}
	r &= \theta g + \rho
\end{align*}

and the incumbent HJB and free entry condition yield the value $\tilde{V}$,
\begin{align}
	\tilde{V} &= 
\end{align} 

and the wage $w_{RD,E}$, respectively. 

If the above equations imply $z_E > \bar{L}_{RD}$ then, as before, in equilibrium we have $z_E = \bar{L}_{RD}$ and $z_I = \tau_I = \tau_S = 0$. Then $g,r,\tilde{V},w_{RD,E}$ can be computed in a manner analogous to that in Case 1.  

\paragraph{Case 3: $\kappa_c = \bar{\kappa}_c(\kappa_e,\lambda)$ and $x \in \{0,1\}$}

In this case, the incumbent is indifferent between choosing $x = 0$ or $x = 1$. There exist BGPs where either $x_{jt} = 0$ for all $j\in[0,1],t \ge 0$ or $x_{jt} = 1$ for all $j\in[0,1],t \ge 0$.

In addition, for any fraction $f \in (0,1)$ there exists a BGP where at all $t \ge 0$, a fraction $f$ of intermediate goods $j$ have $x_j = 1$. However, such a BGP is a more complicated construction than the "pure strategy" case because goods $j$ that use $x_j = 1$ are innovated upon less frequently and therefore tend to fall behind the rest of the economy. BGP requires that the endogenous variable $\mathbb{E}[\tilde{q}_j | x_j = 1]$ remain constant in equilibrium in order for the mapping from individual policies to the aggregate growth to remain constant. One relatively simple way to ensure this is to assume that a constant fraction of entering firms choose $x_j = 1$ and all firms use the same choice $x_j$ throughout their entire lives. Then while surviving goods with $x_j = 1$ tend to lag behind the rest of the economy, they receive a larger boost in proportional terms from the injection of recently innovated upon products whose incumbents set $x_j = 1$. This type of setup would also ensure a BGP in an alternative (I think promising) model where there is heterogeneity in $\kappa_e$. 

\paragraph{Transversality condition}

Household wealth is equal to the value of corporate assets. This is given by the value of incumbents and the value of the potential for spinout formation. The aggregate value of incumbents is $\tilde{V}Q_t$. The aggregate capitalized private value of spinouts is $\frac{\tau_S \lambda \tilde{V} Q_t}{r-g}$. The transversality condition for the household is therefore given by 
\begin{align}
	\lim_{t \to \infty} e^{-rt} \big(1 + \frac{\tau_S \lambda }{r-g}\big)\tilde{V} Q_t = 0 \label{eq:tvc}
\end{align}

Because $Q_t = Q_0 e^{gt}$, (\ref{eq:tvc}) is satisfied provided that $r > g$. Given the Euler equation (\ref{eq:euler}), for $\theta \ge 1$ the condition holds for all $\rho > 0$.  

\section{Empirics}

\subsection{Data}

\subsubsection{Sources}

\paragraph{VentureSource}

The data on startups comes from Venture Source (VS), a proprietary dataset containing information on venture capital (VC) firms and VC-funded startups.\footnote{When starting this project the data were owned by Dow Jones but they have since been sold to CB insights.} I use a subsample of the data for US-based startups founded between 1986 and 2008 which contain information on their founding year. The data cover 23,434 startups, 89,382 financing rounds, and 297,119 individual-firm pairs. For each financing round, the data contain information on valuation, amount raised, and status of the business at the time of the round -- employment, revenue, net income, burn rate -- albeit with substantial missing data. Most importantly for this analysis, the data contain employment biographies for each of the startup's founders and key employees (C-level, high-ranking executives and managers) and board members. In this regard, Venture Source is unique among VC investment databases. Some summary information about the dataset is contained in \autoref{table:VS_summaryTable}. The dataset is described in detail in \cite{kaplan_how_2002} and \cite{kaplan_venture_2016}. 

\paragraph{Compustat}

The data on R\&D spending comes from Compustat, a comprehensive database of fundamental financial and market information on publicly traded companies. I consider a subsample consisting of all firms headquartered in the United States in operation at any point between 1986 and 2006, consisting of 20,534 firms. In addition to data on R\&D spending, the Compustat data contain information on industrial classification and time-varying firm variables such as market value, tangible and intangible assets, employees, sales, etc.

\paragraph{NBER-USPTO}

The NBER-USPTO database contains comprehensive information on all patents granted in the United States from 1976 to 2006, and is linked to Compustat. I consider the subsample of patents assigned to US firms, consisting of 1,457,136 patents. 

\subsubsection{Construction of dataset}

\paragraph{Classifying founders}

The Venture Source data contain information on high level employees and board members. For the purposes of this study, however, not all of these employees should be considered founders of the startup in question. In particular, only those employees whose human capital is crucial to the value proposition of the startup should be considered founders. 

To this end, I first restrict attention to employees who join a startup in its first three years. When information on the individual's date of joining the startup is missing, I impute it as the founding date of the startup. I also conduct robustness exercises where I exclude these individual-startup observations. 

Next, I only consider employees whose job titles relate to the core operations of the firm. \autoref{table:VS_titlesSummaryTable} shows a breakdown of the 20 most frequent titles. Nearly 35\% of named employees are outside board members (e.g. VC investors). For the purpose of this study, I define a \textit{founder} as an employee with the title Founder, Chief, CEO, CTO or President. 

% latex table generated in R 3.6.3 by xtable 1.8-4 package
% Tue Jul  7 15:40:11 2020
\begin{table}[!htb]
\centering
\begingroup\footnotesize
\begin{tabular}{rll}
  \toprule
Title & Individuals & Percentage \\ 
  \midrule
Board member (outsider) & 103367 & 34.6 \\ 
  Vice President & 59149 & 19.8 \\ 
  Chief Executive Officer & 24230 & 8.1 \\ 
  Chief Technology Officer & 13971 & 4.7 \\ 
  Chief Financial Officer & 11621 & 3.9 \\ 
  Director & 10988 & 3.7 \\ 
  Chief & 10846 & 3.6 \\ 
  President \& CEO & 9088 & 3.0 \\ 
  Senior Vice President & 8700 & 2.9 \\ 
  Founder & 8471 & 2.8 \\ 
  Chief Operating Officer & 6777 & 2.3 \\ 
  President & 5441 & 1.8 \\ 
  Chairman & 5029 & 1.7 \\ 
  Executive Vice President & 4920 & 1.6 \\ 
  Chairman \& CEO & 2755 & 0.9 \\ 
  Manager & 2357 & 0.8 \\ 
  Chief Scientific Officer & 1461 & 0.5 \\ 
  Controller & 1137 & 0.4 \\ 
  President \& COO & 1134 & 0.4 \\ 
  General Counsel & 1056 & 0.4 \\ 
   \bottomrule
\end{tabular}
\endgroup
\caption{Top 20 most frequent titles among founders in VS data.} 
\label{table:VS_titlesSummaryTable}
\end{table}


\paragraph{Extracting information on the most recent employer}

In order to relate the activities of employers to the entrepreneurship behavior of their employees, I link the Compustat data to the VS data using information in employee biographies. Because VS biographies are text fields, this requires matching entries by name to firm names in Compustat.  

The VS biographical data comes in a structured format, allowing parsing by regular expressions. Each prior job is represented in the format ``<position>, <employer>'' and different jobs are separated by ``;''. Job spells can be easily separated by splitting the string on the character ``;''. It is slightly more involved to separate positions from employer. It is not sufficient to simply separate on the right-most character ``,'' as <employer> can contain ``,''. However, in almost all cases, <employer> contains at most one ``,'' (e.g., in ``Microsoft, Inc.''), and in virtually all of these cases, the comma precedes one of a few strings (e.g. ``LLC'',``Inc'',``Corp''). Hence, I use a two-pass approach: first I split on the last ``,''; for employers that end up consisting only of corporate structure (e.g., ``LLC'', etc.), I split on the penultimate ``,'' instead. 

The above procedure yields a dataset containing, for each individual, all of his or her previous positions and employers. However, because an individual can take various jobs over the years at an individual startup, there are individuals whose most recent employer coincides with their startup. I exclude these cases by comparing the previous employer with the \texttt{EntityName} text field. Because these are both text fields with potentially different formatting, this entails two steps. First, I bring both fields to a common format, eliminating endings such as ``Inc.'', ``Corp.'' etc which may vary across them, and converting to lower case. I then exclude observations where the strings either exactly coincide or one contains the other. 

The results of this procedure are summarized in \autoref{table:VS_previousEmployersSummaryTable}. The top previous employers include several well-known technology firms such as Microsoft, IBM, Google, and Oracle. However, notice that many of the top previous employers are VC firms (e.g., New Enterprise Associates, Sequoia Capital, Kleiner Perkins), and the top employer is the unaffiliated category "Individual Investor." This is largely because many of the individuals affiliated with startups are investors turned board members: about 33\% of the individual-startup observations are outside board members, As my focus is on the flow of knowledge from previous employers to new startups, I will later restrict attention to individuals with knowledge-related and/or executive titles. Finally, notice that Stanford University is also listed in the top 20. Several other prominent research universities are also in the top 50. 

% latex table generated in R 3.4.4 by xtable 1.8-4 package
% Thu Feb  6 14:38:23 2020
\begin{table}[!htb]
\centering
\begingroup\footnotesize
\begin{tabular}{rlrll}
  \toprule
Employer & Count & Position & Count & Percentage \\ 
  \midrule
Individual Investor & 1115 & Board Member, Institutional Investor & 38527 & 14.2 \\ 
  Microsoft & 938 & Executive & 19617 & 7.2 \\ 
  Google & 780 & CEO & 9681 & 3.6 \\ 
  IBM & 773 & President \& CEO & 6871 & 2.5 \\ 
  Cisco Systems & 690 & CFO & 6567 & 2.4 \\ 
  Oracle & 602 & President & 5741 & 2.1 \\ 
  New Enterprise Associates & 565 & Founder & 4964 & 1.8 \\ 
  AT\&T & 466 & Cofounder & 3884 & 1.4 \\ 
  Verizon & 443 & Partner & 3757 & 1.4 \\ 
  Sun Microsystems & 402 & CTO & 3491 & 1.3 \\ 
  Intel & 389 & Managing Director & 3381 & 1.2 \\ 
  Hewlett-Packard & 386 & VP & 3365 & 1.2 \\ 
  Sequoia Capital & 384 & COO & 2445 & 0.9 \\ 
  Venrock & 342 & Board Member, Outsider & 2260 & 0.8 \\ 
  Kleiner Perkins Caufield \& Byers & 334 & Director & 2221 & 0.8 \\ 
  Mayfield Fund & 310 & Chairman \& CEO & 2147 & 0.8 \\ 
  McKinsey \& & 307 & Founder \& CEO & 2091 & 0.8 \\ 
  Bessemer Venture Partners & 305 & Chairman & 2086 & 0.8 \\ 
  Stanford University & 298 & Principal & 1836 & 0.7 \\ 
  Ernst \& Young & 275 & VP, Marketing & 1785 & 0.7 \\ 
   \bottomrule
\end{tabular}
\endgroup
\caption{Top 20 previous employers and previous positions for founders in VS data.} 
\label{table:VS_previousEmployersSummaryTable}
\end{table}


\paragraph{Linking to Compustat}

The data on prior employers is matched to the variable \texttt{conml} in Compustat. To do this, first I standardize names as before, using regular expressions to trim e.g. ``Inc.", ``Corp.'' and variants thereof from each entry and converting to lower case. I look for exact matches to previous employers in the VS data. For previous employers in VS that do not match with any names in Compustat, I check against the business segment names, available from the Compustat Segments database. 

\paragraph{Defining WSOs}

This study emphasizes the importance of competition between spinouts and their parent firms. The best measure I have for the product market of publicly traded firms is their self reported NAICS code. While VS does not contain NAICS classifications for its startups, it does document their industry using a classification that, for the most part, coincides with NAICS 4 or 5 digit categories. I manually construct a crosswalk between the two classification schemes and use this to assign 4-digit NAICS codes to startups in VS.\footnote{An alternative would be to us VS's "Competition" variable, which documents directly the competitors of the startup observation. However, only 20\% of startups have this variable filled in: 30\% in the 90s, but dropping to around 10\% by the end of the sample.} Then, I classify a founder-startup observation as a \textbf{$i$-digit within-industry spinout (WSOi)} whenever the startup is in the same $i$-digit NAICS category as its parent. 

\paragraph{Evaluating the match}

\autoref{table:GStable_founder2} documents the quality of this match. It corresponds roughly to Table 1 of \cite{gompers_entrepreneurial_2005}.\footnote{The numbers are different. I find a similar number of founders from public companies, but a substantially smaller fraction. I suspect this is due to startups being added to the data retroactively since the time of that article \textbf{[ask VS]}.} About 20\% of key founders have previous employers that match to a public firm in Compustat. In turn, about 35\% of these come from the same 4-digit NAICS codes. The numbers for different definitions of founders are similar. 

\autoref{figure:industry_row_heatmap_naics2_founder2} and \autoref{figure:industry_column_heatmap_naics2_founder2} document the joint distribution of parent industry and child industry, defined by 2-digit NAICS codes. The raw joint distribution is too heavily concentrated to be easily visualized in this way, so instead I show the distribution of child industry (parent industry) conditional on parent industry (child industry), displayed in \autoref{figure:industry_row_heatmap_naics2_founder2} (\autoref{figure:industry_column_heatmap_naics2_founder2}). The dark diagonal lines in both figures reflects the prevalence of WSOs. In \autoref{figure:industry_row_heatmap_naics2_founder2}, the dark vertical line at column 51 (Information) indicates that parent firms of all industries tend to spawn spinouts in that industry. Similar dark regions appear at columns 54 (Professional, Scientific and Technical Services), and 32 and 33 (Manufacturing). In \autoref{figure:industry_column_heatmap_naics2_founder2}, the dark horizontal lines at 51 and to a lesser extend 32, 33, 52 and 54 indicate that child firms of all industries tend to have founders from those industries.

\subsection{Corporate R\&D and spinout formation}

In this section, I consider the determinants of spinout formation. This is relevant to the general equilibrium consequences of spinouts. If employee spinout formation -- and, in particular, WSO4 formation -- is a consequence of parent firm decisions, then it could affect the parent firm decision making process, altering the general equilibrium consequences of facilitating WSO4 spinout formation by, e.g., prohibiting non-compete agreements. 

In particular, I focus on whether R\&D expenditures tend to produce employee spinouts. As discussed in the introduction, this is theoretically plausible because (1) employees undertaking innnovation must be trained, exposing and them to the firm's existing knowledge stock, and (2) such employees also may develop new ideas which may not be implementable within the firm.  

The purpose of this section is to provide discipline on the parametrization of the model used in the quantitative analysis. That model hypothesizes that R\&D by parent firms leads to a flow of employees into starting new spinout firms, and assumes that parent firms internalize this \textit{causal} relationship when making R\&D decisions. Therefore, the validity of my quantitative experiments depends crucially on whether the relationship established in this section is in fact causal. This presents a challenge, as many variables can be thought to simultaneously affect both corporate R\&D and employee entrepreneurship.

I address this challenge using two regression analyses. The first addresses the problem of omitted variable bias by including parent firm controls and fixed effects. Second, I use an IV approach, using the instruments developed in \cite{bloom_identifying_2013}. The latter is similar to the simultaneous work by \cite{babina_entrepreneurial_2018}. Using both approaches, I find evidence that firm-level R\&D leads to employee spinouts. 

\paragraph{Preliminaries}

\autoref{figure:scatterPlot_RD-Founders} shows a scatterplot illustrating the relationship between R\&D spending in years $t-2,t-1,t$ and employee entrepreneurship in years $t+1,t+2,t+3$. The dashed line shows the fit of a straight line through all of the points. The solid line shows the fit of a line only through firm-year observations with nonzero number of employee founders. The graph shows a positive relationship. \autoref{figure:scatterPlot_RD-Founders_dIntersection} shows the relationship between deviations from firm and State-industry-age-year means. The positive relationship remains. Finally, \autoref{figure:scatterPlot_RD-FoundersWSO4_dIntersection} shows that the same positive relationship holds when considering only WSO4 spinouts. 


\paragraph{Regressions}

\autoref{table:RDandSpinoutFormation_absolute_founder2_l3f3} displays the results of a regression analysis relating employee entrepreneurship to parent firm R\&D spending. The dependent variable $Y_{it}$ is again the (annualized) number of founders previously employed at firm $i$ joining startups in years $t+1,t+2,t+3$. The independent variables $X_{it}$ are moving averages over years $t,t-1,t-2$. 

The first three columns consider the effect of R\&D on the number of founders leaving the parent firm. These regressions find a positive coefficient which is statistically significant at the 1\% level, even after including firm and year fixed effects. The magnitude of the coefficient indicates that in 2014, three billion dollars of R\&D over three years leads to on average 1.5 founders leaving to found new firms in the next three years.\footnote{The reason for the dependence on the year 2014 is that the specification assumes that the amount of R\&D that leads to a founder leaving grows at the rate of aggregate productivity growth.}

The robustness of the result to the inclusion of age, industry-year, and State-year fixed effects is encouraging. However, in this context such fixed effects may not absorb much contaminating variation due to what amounts to a misspecification problem: shocks are likely to affect firm outcomes more in absolute terms for larger firms. Because firms vary in size, the fixed effect -- which must be constant in absolute terms for all firms -- leaves much firm-level variation unabsorbed. This is the same reason for the inclusion of Tobin's Q $\times$ Assets, rather than Tobin's Q. 

To address this, \autoref{table:RDandSpinoutFormation_at_founder2_l3f3} displays the results of a regression analysis where all variables are normalized by a trailing 5-year moving average of firm assets. Normalizing in this way improves the specification of the fixed effects, although it is not a panacea as it leaves unaddressed the problem of variation in firm R\&D / asset ratios. The magnitudes of the estimates of the coefficient on the measure of R\&D are strikingly similar. This is particularly true once the full battery of fixed effects is included. Normalizing by assets throws away any variation in absolute firm levels of R\&D, reducing power substantially. In spite of this, the most stringent estimate is significant at the 10\% level.

\begin{table}[!htb]
	\scriptsize
	\centering
	{
\def\sym#1{\ifmmode^{#1}\else\(^{#1}\)\fi}
\begin{tabular}{l*{8}{c}}
\toprule
                    &\multicolumn{1}{c}{(1)}&\multicolumn{1}{c}{(2)}&\multicolumn{1}{c}{(3)}&\multicolumn{1}{c}{(4)}&\multicolumn{1}{c}{(5)}&\multicolumn{1}{c}{(6)}&\multicolumn{1}{c}{(7)}&\multicolumn{1}{c}{(8)}\\
                    &\multicolumn{1}{c}{Founders}&\multicolumn{1}{c}{Founders}&\multicolumn{1}{c}{Founders}&\multicolumn{1}{c}{Founders}&\multicolumn{1}{c}{WSO4}&\multicolumn{1}{c}{WSO4}&\multicolumn{1}{c}{WSO4}&\multicolumn{1}{c}{WSO4}\\
\midrule
R\&D                &        0.34\sym{**} &        0.73\sym{***}&        0.73\sym{***}&        0.63\sym{***}&        0.19\sym{***}&        0.32\sym{***}&        0.31\sym{***}&        0.28\sym{***}\\
                    &      (0.13)         &      (0.24)         &      (0.23)         &      (0.14)         &     (0.045)         &     (0.067)         &     (0.065)         &     (0.050)         \\
\addlinespace
No FE               &         Yes         &          No         &          No         &          No         &         Yes         &          No         &          No         &          No         \\
\addlinespace
Firm FE             &          No         &         Yes         &         Yes         &         Yes         &          No         &         Yes         &         Yes         &         Yes         \\
\addlinespace
Year FE             &          No         &         Yes         &          No         &          No         &          No         &         Yes         &          No         &          No         \\
\addlinespace
Age FE              &          No         &          No         &         Yes         &          No         &          No         &          No         &         Yes         &          No         \\
\addlinespace
Industry-Age FE     &          No         &          No         &          No         &         Yes         &          No         &          No         &          No         &         Yes         \\
\addlinespace
Industry-Year FE    &          No         &          No         &         Yes         &          No         &          No         &          No         &         Yes         &          No         \\
\addlinespace
State-Year FE       &          No         &          No         &         Yes         &          No         &          No         &          No         &         Yes         &          No         \\
\addlinespace
Industry-State-Year FE&          No         &          No         &          No         &         Yes         &          No         &          No         &          No         &         Yes         \\
\midrule
r2\_a                &        0.24         &        0.69         &        0.69         &        0.75         &        0.21         &        0.65         &        0.64         &        0.61         \\
r2\_a\_within         &        0.24         &        0.26         &        0.26         &        0.24         &        0.21         &        0.25         &        0.23         &        0.15         \\
N                   &       65009         &       63732         &       62211         &       37810         &       65009         &       63732         &       62211         &       37810         \\
\bottomrule
\multicolumn{9}{l}{\footnotesize Standard errors in parentheses}\\
\multicolumn{9}{l}{\footnotesize \sym{*} \(p<0.1\), \sym{**} \(p<0.05\), \sym{***} \(p<0.01\)}\\
\end{tabular}
}

	\caption{The regressions above relate corporate R\&D to the entrepreneurship decisions of employees. The dependent variable is average yearly number of founders joining startups in years $t+1,t+2,t+3$. The independent variables are averages over $t,t-1,t-2$. Firm controls are employment, assets, intangible assets, investment, net income, cumulative citation-weighted patents, and the product of Tobin's Q and Assets. Standard errors are clustered by firm.}
	\label{table:RDandSpinoutFormation_absolute_founder2_l3f3}
\end{table}

\begin{table}[!htb]
	\scriptsize
	\centering
	{
\def\sym#1{\ifmmode^{#1}\else\(^{#1}\)\fi}
\begin{tabular}{l*{8}{c}}
\toprule
                    &\multicolumn{1}{c}{(1)}&\multicolumn{1}{c}{(2)}&\multicolumn{1}{c}{(3)}&\multicolumn{1}{c}{(4)}&\multicolumn{1}{c}{(5)}&\multicolumn{1}{c}{(6)}&\multicolumn{1}{c}{(7)}&\multicolumn{1}{c}{(8)}\\
                    &\multicolumn{1}{c}{$\frac{\textrm{Founders}}{\textrm{Assets}}$}&\multicolumn{1}{c}{$\frac{\textrm{Founders}}{\textrm{Assets}}$}&\multicolumn{1}{c}{$\frac{\textrm{Founders}}{\textrm{Assets}}$}&\multicolumn{1}{c}{$\frac{\textrm{Founders}}{\textrm{Assets}}$}&\multicolumn{1}{c}{$\frac{\textrm{WSO4}}{\textrm{Assets}}$}&\multicolumn{1}{c}{$\frac{\textrm{WSO4}}{\textrm{Assets}}$}&\multicolumn{1}{c}{$\frac{\textrm{WSO4}}{\textrm{Assets}}$}&\multicolumn{1}{c}{$\frac{\textrm{WSO4}}{\textrm{Assets}}$}\\
\midrule
$\frac{\textrm{R\&D}}{\textrm{Assets}}$&        1.71\sym{***}&        1.12         &        1.15         &        0.28         &        0.82\sym{***}&        0.62\sym{**} &        0.57         &        0.77         \\
                    &      (0.34)         &      (0.68)         &      (0.71)         &      (1.05)         &      (0.17)         &      (0.32)         &      (0.36)         &      (0.88)         \\
\addlinespace
NAICS4-State-Age-Year FE&          No         &          No         &          No         &         Yes         &          No         &          No         &          No         &         Yes         \\
\addlinespace
NAICS4-Year FE      &          No         &          No         &         Yes         &          No         &          No         &          No         &         Yes         &          No         \\
\addlinespace
State-Year FE       &          No         &          No         &         Yes         &          No         &          No         &          No         &         Yes         &          No         \\
\addlinespace
Firm FE             &          No         &         Yes         &         Yes         &         Yes         &          No         &         Yes         &         Yes         &         Yes         \\
\addlinespace
Age FE              &          No         &          No         &         Yes         &          No         &          No         &          No         &         Yes         &          No         \\
\addlinespace
Year FE             &          No         &         Yes         &          No         &          No         &          No         &         Yes         &          No         &          No         \\
\addlinespace
No FE               &         Yes         &          No         &          No         &          No         &         Yes         &          No         &          No         &          No         \\
\midrule
r2\_a                &       0.014         &        0.26         &        0.22         &        0.25         &      0.0088         &        0.27         &        0.21         &        0.40         \\
r2\_a\_within         &       0.014         &      0.0028         &      0.0025         &      0.0014         &      0.0088         &      0.0014         &      0.0012         &      0.0046         \\
N                   &       60687         &       59477         &       58201         &       23665         &       60687         &       59477         &       58201         &       23665         \\
\bottomrule
\multicolumn{9}{l}{\footnotesize Standard errors in parentheses}\\
\multicolumn{9}{l}{\footnotesize \sym{*} \(p<0.1\), \sym{**} \(p<0.05\), \sym{***} \(p<0.01\)}\\
\end{tabular}
}

	\caption{The regressions above relate corporate R\&D to the entrepreneurship decisions of employees. The dependent variable is the average yearly number of founders from the parent firm joining startups in years $t+1,t+2,t+3$, normalized by a trailing five-year moving average of assets. Independent variables are also normalized by assets. Standard errors are clustered at the firm level.}
	\label{table:RDandSpinoutFormation_at_founder2_l3f3}
\end{table}

\paragraph{Economic magnitude}

In each year $t$, I compute $\tilde{y}_{it}$, the expected number of founders per year starting firms in years $t+1,t+2,t+2$ by multiplying the R\&D in years $t,t-1,t-2$ by the relevant coefficient estimate. I then plot this against the realizations of $y_{it}$. \autoref{figure:founder2_founders_f3_Accounting} provides a visualization of the economic magnitude of the coefficient estimates. The left column is for all founders and the right column is for founders of firms in the same 4-digit NAICS industry as the parent. 

The regression estimates are economically significant. About half of all WSO4 spinouts are accounted for by parent firm corporate R\&D. The share explained is closer to 1/4 when considering all employee entrepreneurship. This is consistent with the emphasis on WSO4 spinouts in the model. 

\begin{figure}[!htb]
	\includegraphics[scale=0.5]{../empirics/figures/founder2_founders_f3_Accounting.pdf}
	\caption{Economic magnitude of regression estimates in Tables \ref{table:RDandSpinoutFormation_absolute_founder2_l3f3} and \ref{table:RDandSpinoutFormation_at_founder2_l3f3}. The first row of figures compares the predicted number of employee founders (dotted lines) to the observed number of employee founders (solid lines). The left figure considers all founders, the right figure only founders of firms in the same 4-digit NAICS industry as their previous employers. The bottom row shows the percentage explained in each year.}
	\label{figure:founder2_founders_f3_Accounting}
\end{figure}


\subsection{Characteristics of spinouts and other firms}

\section{Calibration}

\subsection{Targets}

In the baseline calibration I target the interest rate, the labor productivity growth rate, the profit share of GDP, the R\&D share of GDP, the growth share of OI, the employment share of firms up to six years of age (exclusive), and the employment share of spinouts. \autoref{calibration_targets} shows the calibration targets. 

In the sections below, for each moment I briefly explain the reason for targeting it and describe how it is calculated from the relevant data sources.

\subsubsection{Interest rate}

Matching the interest rate is important as the model is about the decision by firms to invest in the future, which involves the discount factor and hence the interest rate $r$. The interest rate is calibrated to a return on equity of 6\%. 

\subsubsection{Growth rate}

Matching the growth rate is essential a model that pretends to explain the forces determining the growth rate. To be specific, it means that the correctly quantifies the importance of the described mechanisms for the determination of growth. The growth rate is calibrated to the growth in labor productivity due to creative destruction and own innovation, as calculated in Klenow \& Li 2020. [\textbf{Insert description of how they identify}]

\subsubsection{Profits \% GDP} 

Matching the profit share of GDP is important because intermediate goods firms in the economy innovate in search of profits. If the model's measure of profits is inaccurate, it will incorrectly infer the other parameters determining the decision to innovate. Such parameters must be accurately inferred if the model is to be useful for assessing the effect of policy on growth and welfare. The data on aggregate profits as a percent of GDP comes from the BEA (computed as an average during the sample period of 1986-2008). 

\subsubsection{R\&D spending / GDP}

Matching the R\&D share of GDP is important in order to properly assess the efficiency by which firms produce innovations. The data on R\&D spending is from the National Patterns of R\&D resources.\footnote{I take the average of business-funded R\&D business-performed R\&D.} In the data, about half of R\&D spending is wages for employees; in the model, the only input to R\&D is labor. I opt to match the model's aggregate R\&D intensity to that in the data, including costs other than labor. This means that the model captures the full cost of innovation. 

\subsubsection{Growth share of own innovation}

It is important to match this moment to properly assess the relative importance for innovation of OI. If OI is not a large contributor to growth, then creating a disincentive to OI can be worth it if it creates more spinouts. 

This moment cannot be directly measured. Instead, it is inferred using a model similar to the one presented here and data on employment growth at the establishment level. \cite{klenow_innovative_2020} 2020 finds that, from 1982 to 2013, roughly 70\% of CD + OI productivity growth was due to OI. 

\subsubsection{Entry rate}

Matching the employment share of entering firms is important to assess the overall level of turnover in the economy. For a given rate of productivity growth, more turnover implies that innovation occurs via smaller, more frequent improvements in quality. As I will show later, this drastically influences the welfare consequences of allowing NCAs.

The entry rate target deserves some discussion. The purpose of including entry in the model is to capture the rate at which incumbent profits are destroyed due to creative destruction. As discussed in \cite{klenow_innovative_2020}, adjustment costs mean that, in the data, it can take several years for a new product to displace an old one. However, in the model, entrants that replace incumbents reach their mature size immediately upon entry. If the model matches the amount of employment in firms of age < 1, it might underestimate the true impact on employment reallocation of each new cohort of firms.\footnote{In the data, because firms grow to achieve their mature size over the first five years (and beyond), so that the employment of an entering cohort of firms does not decrease over time (i.e., including firm exit) very rapidly in the data. If the data were in continuous time, the employment of the cohort would increase at first, then decrease. In the model, firms enter at their mature size, so the employment of a cohort decreases over time.} Given this, I match the employment share of firms age <= 6 engaging in creative destruction, which is approximately 8.35\% during the sample period.
 
\subsubsection{R\&D-induced spinout share of employment}

Finally, matching the employment share of spinouts is of course crucial, in order for the model to properly estimate the burden such firms impose on the incumbents that spawn them. As will be discussed below, care must be taken to only match the employment share of spinouts which can be attributed to R\&D, since spinouts in the model correspond to those in the data which are "induced" by R\&D at the incumbent. I discipline this using micro data on spinouts. [\textbf{Section in progress}]

\subsection{Computing the corresponding model moments}

\subsubsection{R\&D / GDP} 

In the model, the R\&D share is the ratio of the wage paid to R\&D workers to GDP. This is
\begin{align*}
\frac{\textrm{R\&D wage bill}}{\textrm{GDP}} &= \frac{\overline{w}_{RD,I} z_I + \overline{w}_{RD,E} z_E}{\tilde{Y}} \\ 
&= \frac{\overline{w}_{RD,E} (z_I + z_E) + (\overline{w}_{RD,I} - \overline{w}_{RD,E})z_I}{\tilde{Y}} \\
&= \frac{\overline{w}_{RD,E} (z_I + z_E) - (1-\kappa_e) \lambda \tilde{V} \tau_S}{\tilde{Y}}
\end{align*}

where I used $\overline{w}_{RD,I} - \overline{w}_{RD,E} = -(1-x)(1-\kappa_e) \lambda \tilde{V} \nu$ and $\tau_S = (1-x)\nu z_I$. 

\subsubsection{Entry rate}

This has a counterpart in the model which can be calculated in closed form. Let $\ell(a)$ denote the density of incumbent employment at age $a$ incumbents. Then $\ell(a)$ is characterized by 
\begin{align*}
\ell(a) &= \ell(0)e^{((\hat{\tau}_I -1)g - (\tau_E + (1-x)z_I \nu))a}  \\
1 + \bar{L}_{RD} - z_E &= \int_0^{\infty} \ell(a) da
\end{align*}

where $\hat{\tau}_I = \frac{\tau_I}{\tau_I + \tau_E + \tau_S}$ is the fraction of innovations that are incumbents' own innovations. 

The intuition for this characterization of $\ell(a)$ has two parts. First, because all shocks are \textit{iid} across firms in equilibrium, the law of large numbers applied to each cohort of firms implies that we can consider directly the evolution of the cohort as a whole instead of explicitly analyzing the dynamics each individual firm in the cohort.  Second, the employment of a firm is proportional to its relative quality, $l_j \propto \tilde{q}_j = q_j / Q$, as long as it is the leader. When it is no longer the leader, its employment is zero forever. Putting these two together, $\ell(a)$ must decline at exponential rate $g$ due to the increase in $Q_t$ (obsolescence), increase at rate $\hat{\tau}_I g$ due to incumbents own innovations, and decline at rate $\tau_E + \tau_S$ due to creative destruction.\footnote{The second equation imposes consistency with aggregate employment; it implies $\ell(0) = -((\hat{\tau}_I -1)g - (\tau_E + \tau_S))(1 + \bar{L}_{RD})$. The calibration does not require this explicit calculation since it is based only on employment shares.} Note that the employment density is strictly decreasing in $a$. This is because there are no adjustment costs: firms achieve their optimal scale immediately upon entry, and subsequently become obsolete (on average) or lose the innovation race to an entrant. Finally, due to the constant exponential decay of $\ell(a)$, the share of incumbent employment in incumbents of strictly less than 6 years of age is given by 
\begin{align*}
\Xi_{[0,6)} &=  1 - \frac{\ell(6)}{\ell(0)} \\
&= 1 - e^{((\hat{\tau}_I -1)g - (\tau_E + \tau_S))\cdot 6}
\end{align*}  


The share of overall employment in incumbents of age < 6, including R\&D performed by non-producing entrants, is equal to the previously calculated $\Xi_{[0,6)}$, multiplied by the share of total labor in incumbents, $1 - z_E$, added to the R\&D labor used by entrants $z_E$, 
\begin{align*}
\textrm{Age < 6 share of employment} &= \frac{2}{3}(\Xi_{[0,6)} (1-z_E) + z_E)
\end{align*}

The factor 2/3 follows from interpreting entrants in the model as either new firms or incumbents engaging in creative destruction. According to KH 2020, creative destruction by incumbents is responsible for half as much growth as creative destruction by entrants. In this interpretation of the model, both types of creative destruction use the same technology. Therefore, 2/3 of employment in young firms in the model represents employment in young firms in the data.\footnote{Also, note that this formula extrapolates the employment-age distribution to the entire economy and calculated shares in that way. If I did not do this, the formula would be
	\begin{align*}
	\textrm{Age < 6 share of employment} &= \frac{2}{3} \frac{(\Xi_{[0,6)} (1 - L_F -z_E) + z_E)}{1-L_F}
	\end{align*}
	
	This has only minor effects on the inferred parameters. They are listed in \autoref{calibration_2_parameters}.}

\subsubsection{Growth share of own innovation}

The model moment that corresponds here is the share of growth due to own innovation by incumbents of age >= 6. In the model, the fraction of OI growth due to incumbents in a given age group is exactly their fraction of employment: innovations arrive at the same rate for each incumbent, and their impact on aggregate growth is proportional to the incumbent's relative quality, which is proportional to employment. Hence old incumbents' share of growth due to own innovation is simply one minus the employment share calculated in the previous paragraph, $e^{((\hat{\tau}_I -1)g - (\tau_E + (1-x)z_I \nu))\cdot 6}$. Finally, the fraction of aggregate growth due to OI is $\hat{\tau}_i$, defined above. The fraction of growth due to incumbents of age at least 6 is the product of the two, 
\begin{align*}
\textrm{Age >= 6 share of OI} &= \hat{\tau}_I \frac{\ell(6)}{\ell(0)} \\
 &= \hat{\tau}_I e^{((\hat{\tau}_I -1)g - (\tau_E + (1-x)z_I \nu))\cdot 6} 
\end{align*}

\subsubsection{R\&D-induced spinout employment share}

Because entering spinouts and entering ordinary firms have identical life-cycles post entry in expectation, the BGP share of employment in firms started as spinouts is their share of new incumbents $\frac{\tau_S}{\tau_S+ (\frac{2}{3})\tau_E}$, multiplied by the employment share of incumbents $1- (\frac{2}{3})z_E$, 
\begin{align*}
	\textrm{Spinout employment share} &= \frac{\tau_S}{\tau_S + \frac{2}{3}\tau_E} (1 - \frac{2}{3}z_E) 
\end{align*}

Again, the factor 2/3 is because this is the fraction of entrants in the model which the calibration maps to new firms in the data.


\begin{table}[]
	\centering
	\captionof{table}{Calibration targets}\label{calibration_targets}
	\begin{tabular}{rll}
		\toprule \toprule
		& Target & Model \tabularnewline
		\midrule
		\multicolumn{1}{l}{\textbf{Analytically matched}} & & 
		\tabularnewline
		Profit (\% GDP) & 8.5\% & 8.5\% 
		\tabularnewline
		\tabularnewline
		\multicolumn{1}{l}{\textbf{Numerically matched}} & & 
		\tabularnewline
		Interest rate & 6\% & 6\% 
		\tabularnewline
		Growth rate (CD + OI) & 1.3\% & 1.3\%
		\tabularnewline		
		Growth share OI & 70\% & 70\%
		\tabularnewline
		Age $<$ 6 emp. share  & 8.35\% & 8.35\%
		\tabularnewline
		R\&D-induced spinout emp. share & 13.7\% & 13.7\%
		\tabularnewline
		R\&D spending (\% GDP) & 1.5\% & 1.5\%
		\tabularnewline
		\bottomrule
	\end{tabular}
\end{table}

\normalsize

\subsection{Identification}

\autoref{calibration_identificationSources} shows the elasticity of model moments to calibrated model parameters. This is the jacobian matrix of the mapping that takes log parameters to log model moments. On its own it suggests how identification occurs by showing which moments are sensitive to which parameters. 

\autoref{calibration_sensitivity} inverts this matrix to obtain the elasticity of calibrated parameters to moment targets. This is feasible because the model is locally exactly identified by the target moments. This provides a more complete picture of the identification, since it takes into account the interaction of the various sensitivities of moments to parameters when it inverts the matrix. For example, while an increase in $\lambda$ causes a large increase in Growth Share OI, increasing the Growth Share OI moment target decreases the estimated $\lambda$. The calibration prefers to match the higher Growth Share OI with a much higher $\chi_I$ and slightly lower $\lambda$ to compensate for the distortions it causes. This kind of effect is difficult to infer from \autoref{calibration_identificationSources}. 

Next, \autoref{calibration_identificationSources_full} augments \autoref{calibration_identificationSources} with non-calibrated parameters included as both parameters and target moments. As before, \autoref{calibration_sensitivity_full} inverts this matrix to obtain the elasticity of calibrated parameters to moment targets and non-calibrated parameters. 

\begin{table}[]
	\centering
	\captionof{table}{Baseline calibration}\label{calibration_parameters}
	\begin{tabular}{rlll}
		\toprule \toprule
		Parameter & Value & Description & Source \tabularnewline
		\midrule
		$\rho$ & 0.0339 & Discount rate  & Indirect inference \tabularnewline
		$\theta$ & 2 & $\theta^{-1} = $ IES & External calibration 
		\tabularnewline
		$\beta$ & 0.094 & $\beta^{-1} = $ EoS intermediate goods & Exactly identified \tabularnewline 
		$\psi$ & 0.5 & Entrant R\&D elasticity & External calibration \tabularnewline
		$\lambda$ & 1.166 & Quality ladder step size & Indirect inference 
		\tabularnewline
		$\chi_I$ & 1.86 & Incumbent R\&D productivity & Indirect inference 
		\tabularnewline
		$\chi_E$ & 0.116 & Entrant R\&D productivity & Indirect inference \tabularnewline 
		$\kappa_e$ & 0.738 & Non-R\&D entry cost & Indirect inference \tabularnewline
		$\nu$ & 0.0488 & Spinout generation rate  & Indirect inference\tabularnewline
		$\bar{L}_{RD}$ & 0.05 & R\&D labor allocation  & Normalization \tabularnewline
		\bottomrule
	\end{tabular}
\end{table}

\begin{figure}[]
	\includegraphics[scale = 0.43]{../code/julia/figures/simpleModel/identificationSources.pdf}
	\caption{Plot showing the elasticity of moments to model parameters. This illustrates how the model's equilibrium is affected by the various choices of parameters. These elasticities are computed by taking the jacobian matrix of the mapping from log parameters to log model moments.}
	\label{calibration_identificationSources}
\end{figure}

\begin{figure}[]
	\includegraphics[scale = 0.43]{../code/julia/figures/simpleModel/calibrationSensitivity.pdf}
	\caption{Plot showing the elasticity of parameters to moments. It is computed by inverting the jacobian matrix of the mapping from log parameters to log model moments (whose entries comprise the previous figure). These elasticities, along with estimates of the noisiness of the moments used in the calibration, can be used to estimate confidence intervals for the parameters in the model, and thereby for the welfare comparison in question.}
	\label{calibration_sensitivity}
\end{figure}

\begin{figure}[]
	\includegraphics[scale = 0.43]{../code/julia/figures/simpleModel/identificationSourcesFull.pdf}
	\caption{Plot showing the elasticity of moments to model parameters, including parameters taken from the literature $\theta , \beta, \psi$. These non-calibrated parameters are added in as effective moments to be matched, allowing the sensitivity of calibrated parameters $\rho, \lambda, \chi_I, \chi_E, \kappa_E, \nu$ to these parameters to be computed by simply inverting this matrix, as before.}
	\label{calibration_identificationSources_full}
\end{figure}

\begin{figure}[]
	\includegraphics[scale = 0.43]{../code/julia/figures/simpleModel/calibrationSensitivityFull.pdf}
	\caption{Same as \autoref{calibration_sensitivity}, but now including non-calibrated parameters. As before, this calculated by inverting the jacobian displayed in \autoref{calibration_identificationSources_full}.}
	\label{calibration_sensitivity_full}
\end{figure}

\section{Policy analysis}

The key tradeoff in this economy is between the fact that innovation and creative destruction by spinouts expands the innovation possibilities frontier while reducing incentives for incumbents to use their existing innovation possibilities. The net effect on the growth rate from, say, an increase in $\nu$, is ambiguous. Because welfare is increasing in and very sensitive to the BGP growth rate, the net effect on welfare is also ambiguous.

In some cases, the availability of NCAs can, in theory, help mitigate this friction. NCAs ensure that if spinouts are ex-ante bilaterally inefficient -- that is, if the pair would be better off setting $\nu = 0$ and splitting the proceeds -- then spinouts in fact will not occur. This can mitigate the disincentive effect and thereby could improve growth and welfare. 

On the other hand, NCA usage can also reduce aggregate growth. First, bilateral inefficiency does not imply social inefficiency. Innovation by spinouts has positive externalities: it increases the consumer surplus, improves the overall productivity of the economy (recall the production function (\ref{intermediate_goods_production})), and improves the innovation technology of entrants. Second, in this economy the aggregate amount of R\&D labor does not depend on incentives. A reduction in R\&D spending by incumbent is met with an equivalent increase in R\&D by entrants. This attenuates its negative effect on growth. 

Typically, the most fundamental social planner problem to study is that of characterizing the first-best allocation. However, in this model some of the primitives, in particular the cost of entry $\kappa_e \lambda V(q,t)$ and the cost of NCAs $\kappa_c \nu V(q,t)$, are specified as a function of value functions. As these are only well-defined in a decentralized market equilibrium, the first-best allocation is itself not well-defined.

To circumvent this difficulty, I conduct a sequence of second-best analyses assuming the planner can control one or more parameters or Pigouvian taxes. Specifically, I consider planners who can control the following: 

\begin{enumerate}
	\item Cost of NCAs: $\kappa_c$ 
	\item NCA tax (subsidy): $T_{NCA}$
	\item R\&D subsidy (tax): $T_{RD}$
	\item Creative destruction tax (subsidy): $T_e$
	\item OI R\&D subsidy (tax): $T_{RD,I}$
	\item All policies: $\{\kappa_c, T_{RD}, T_{RD,I}, T_{NCA}, T_e\}$
\end{enumerate}

\paragraph{Public finance} 

In cases of taxes (subsidies), I assume that they are rebated to (financed by) the representative household in a lump-sum payment. Because there is no labor-leisure choice, this does not create any additional distortions in the economy. 

\paragraph{Welfare}

Over the course of a given equilibrium, there exist $\tilde{Y},\tilde{C},\tilde{W}$ such that output, consumption and welfare at time $t$ are given by $Y_t = \tilde{Y} Q_t, C_t = \tilde{C} Q_t, W_t = \tilde{W} Q_t^{1-\theta}$. Normalized welfare $\tilde{W}$ is given by\footnote{I have added the constant $\frac{1}{(1-\theta)\rho}$ to welfare, since it does not affect the implications for consumption-equivalent measures of welfare.}
\begin{align}
\tilde{W} &= \frac{\big(\overbrace{\tilde{Y} - (\tau_E + \tau_S) \kappa_{e} \lambda \tilde{V} - x z_I \kappa_c \nu \tilde{V}}^{\tilde{C}}\big)^{1-\theta}}{(1-\theta)(\rho - g(1-\theta))} \label{eq:agg_welfare}
\end{align}


For welfare comparisons to be meaningful, they must be converted into consumption-equivalent (CEV) terms. For $\theta < 1$, a $\frac{\xi}{1-\theta}\%$ increase in CEV welfare results from a $\xi\%$ increase in the first term in (\ref{eq:agg_welfare}). For $\theta > 1$, a $\frac{\xi}{\theta-1}\%$ increase in CEV welfare results from an $\xi\%$ decrease in the absolute value of the same term.\footnote{The case $\theta = 1$ corresponds to log utility, in which case
	\begin{align}
	\tilde{W} &= \frac{\rho \log(\tilde{C}) + g}{\rho^2} \label{eq:agg_welfare_log}
	\end{align}
	
	In this case, there is no simple correspondence to obtain CEV welfare changes, but they are easy to compute directly. Under the null policy, initial consumption is $\tilde{C}$ and growth is $g$. Under the new policy, initial consumption is $\tilde{C}^+$ and growth is $g^+$. The CEV welfare change is $\frac{\tilde{C}^* - \tilde{C}}{\tilde{C}}$, where $\tilde{C}^*$ is defined by 
	\begin{align}
	\frac{\rho\log(\tilde{C}^*) + g}{\rho^2} = \frac{\rho \log(\tilde{C}^+) + g^+}{\rho^2} \label{eq:agg_welfare_log_CEV}
	\end{align}}

\subsection{NCA cost $\kappa_c$} 

Consider a planner who controls the parameter $\kappa_c$. I interpret this as a policymaker changing the extent of restrictions on NCAs so that they are more or less expensive to enforce. 

A natural assumption is to posit a minimum cost $\munderbar{\kappa}_c \ge 0$ such that $\kappa_c \ge \munderbar{\kappa}_C$. NCAs require costs for enforcement even if they are fully endorsed by the legal system: a contract must be written and, in the case of infringement, it must be established that the employee is, in fact, competing with their previous employer. For simplicity, in the analysis below I assume $\munderbar{\kappa}_c = 0$.

\subsubsection{Effect on growth}

Suppose that $x = 1$ and $z_I > 0$ at $\kappa_{c0}$ and the planner increases the cost of NCAs to $\kappa_{c1} > \kappa_{c0}$. Equation (\ref{eq:effort_entrant}) immediately implies that $z_E$ increases and Equation (\ref{eq:zI_asFuncZe}) in turn that $z_I$ decreases. Intuitively, the increase in $\kappa_c$ makes R\&D more expensive for the incumbent, reducing $z_I$ to zero in partial equilibrium. To clear the labor market, $w_{RD,E}$ must decline to induce more R\&D. In the new equilibrium, the premium the incumbent must pay in effective cost of R\&D, relative to the entrant, is larger, and as a result she employs a smaller share of the R\&D labor.

This reallocation of R\&D labor decreases the BGP growth rate if and only if the marginal effect on growth of entrant R\&D is less than the marginal efect on growth of incumbent R\&D. The latter is equal to the constant exogenous parameter $\chi_I$. The former is equal to 
\begin{align}
\frac{d}{dz_E} \tau_E(\kappa_c) &= (1-\psi) \chi_E z_E (\kappa_c)^{-\psi} \label{eq:marginal_effect_effort_entrant}
\end{align}

where $\tau_E(\kappa_c), z_E(\kappa_c)$ denote the equilibrium value of $\tau_E, z_E$ for a given parameter value of $\kappa_c$. Because $z_E(\kappa_c)$ is monotonically increasing and $\psi > 0$, $\frac{d}{dz_E} \tau_E |_{\kappa_c}$ is monotonically decreasing in $\kappa_c$. Therefore, $\frac{d}{dz_E}\tau_E |_{\kappa_c} < \frac{d}{dz_E}\tau_E|_0$. Substituting the expression for $z_E$ in (\ref{eq:effort_entrant}) into the inequality $\frac{d}{dz_E}\tau_E |_{0}< \chi_I$ and rearranging yields 
\begin{align}
\overbrace{\frac{\lambda-1}{\lambda}}^{\mathclap{\text{Business stealing}}} \times \underbrace{(1-\psi)}_{\mathclap{\text{Fishing out}}} \times  \overbrace{\frac{1}{1-\kappa_{e}}}^{\mathclap{\text{Entry cost}}}< 1 \label{cs:growth_decreasing_condition}
\end{align}

The term $\frac{\lambda - 1}{\lambda} < 1$ reflects the \textit{business stealing} effect: innovation by entrants imposes a negative externality on the profits of the incumbent. The term $1-\psi < 1$ reflects the \textit{fishing out} effect: individual entrants impose a negative externality on the expected returns of other entrants by reducing their rate of winning the innovation race per unit of R\&D. Both of these terms reflect additional incentives for innovation by entrants than exist for incumbents, pushing equilibrium $z_E$ to a level such that its marginal effect on growth is lower than that of $z_I$. Finally, the term $\frac{1}{1-\kappa_e} > 1$ reflects the additional entry cost paid by entrants upon innovating. As this reduces $z_E$ in equilibrium, it leads to an increase in the marginal effect on growth of innovation by entrants.\footnote{Of course, it also entails a reduction in $\tilde{C}$ of $\kappa_e \lambda \tilde{V}$, which tends to reduce welfare.}

Next, suppose $x = 0$ and $z_I > 0$ at $\kappa_{c0} < \bar{\kappa}_c$ and consider an increase to $\kappa_{c1}$. If $\kappa_{c1} < \bar{\kappa}_c$ then $x = 0$ and there is no change in the equilibrium. If $\kappa_{c1}$ is large enough then $x = 1$. Upon crossing the $\bar{\kappa}_c$ threshold, $\tau_S$ jumps from $0$ to $\nu z_I$ while $z_I,z_E$ do not jump. By the growth accounting equation (\ref{eq:growth_accounting}), the growth rate jumps to $g_1 > g_0$. By the Euler equation (\ref{eq:euler}), the interest rate jumps from $r$ to $r_1>r_0$. To preserve the incumbent's HJB (\ref{eq:hjb_incumbent_noNCA}), the R\&D wage declines to $w_{RD1} < w_{RD0}$.

The countervailing economic forces described in the introduction of this section are clearly laid out in the above comparisons. If the LHS of (\ref{cs:growth_decreasing_condition}) is much smaller than the RHS, the disincentive to the incumbent is large compared to the mitigation by increased R\&D from entrants and the boost to the economy's overall innovation efficiency from free spinout entry. More concretely, the discussion implies that, small increases in $\kappa_c$ are negative for growth but large increases can be positive or negative.

Finally, note that the above discussion implies that if $z_I = 0$ at $\kappa_{c0}$ then increasing $\kappa_{c0}$ has no effect, regardless of the value of $x$. 

\subsubsection{Consumption}\label{cs:consumption1}

It is more difficult to establish the effect of NCAs on normalized consumption $\tilde{C}$. Consider again $\kappa_c \in [0, \bar{\kappa}_c)$. As $\tau_S = 0$, consumption is given by 
\begin{align}
\tilde{C} &= \tilde{Y} - \Big( \tau_E  \kappa_e \lambda + z_I \nu \kappa_c \Big) \tilde{V} \\
&= \tilde{Y} - \Big( \chi_E (\bar{L}_{RD} - z_I)^{1-\psi} \kappa_e \lambda + z_I \nu \kappa_c \Big) \tilde{V} \label{cs:consumption_eq}
\end{align}

The presence of the exponent $1-\psi$ complicates the analysis here. 

However, if we assume $\psi = 0.5$ as in the baseline calibration, it can be shown that when $z_I / z_E$ is high, the elasticity of $z_I$ with respect to $\kappa_c$ is greater than $-1$. In such a case, consumption falls as $\kappa_c$ increases. 

\subsubsection{Effect on welfare}

\autoref{calibration_summaryPlot} shows how the equilibrium varies with $\kappa_c$. As $\kappa_c$ increases in $[0,\bar{\kappa}_c)$, welfare decreases (third row, third panel). This is driven by the changes in the growth rate (first row, third panel). The movements in the growth rate in turn drive movements in the interest rate, via the Euler equation (second row, second panel). Both R\&D wages paid by incumbents and entrants decline and then jump downwards at the $\bar{\kappa}_c$ threshold (second row, third panel). The growth rate is driven by the changes in the innovation rate (first row, second panel). The incumbent reduces innovation gradually, while the entrant increases gradually, but by less due to lower marginal returns to R\&D in equilibrium, as inequality (\ref{cs:growth_decreasing_condition}) holds (the LHS is equal to 0.33 in this parametrization). The spinout increases innovation discretely as the threshold increases. Finally, the incumbent value decreases continuously until the threshold, where it jumps downwards (second row, first panel).

\begin{figure}[]
	\includegraphics[scale = 0.57]{../code/julia/figures/simpleModel/calibration_summaryPlot.pdf}
	\caption{Effect of varying $\kappa_c$ on equilibrium variables and welfare.}
	\label{calibration_summaryPlot}
\end{figure}




\paragraph{Robustness of welfare gain from NCA enforcement}

\autoref{welfareComparisonSensitivityFull} shows the sensitivity of the welfare comparison the moments targeted, including the externally calibrated parameters as pseudo-moments as before. It is computed as $\nabla_m \tilde{W}|_m = (J^{-1})^T \nabla_p W|_p$, where $J$ is the Jacobian of the mapping from log parameters to moments (so that $J^{-1}$ is the Jacobian of the inverse mapping), and $W$ is the mapping from parameters the log \% change (or raw \% change, in \autoref{levelsWelfareComparisonSensitivityFull})) in CEV welfare from reducing $\kappa_c$ from $\infty$ to $0$. That is, it is the gradient of the change in welfare to the change in target moments or uncalibrated parameters, taking as given the change in parameters required to continue matching the target moments. For reference, $\nabla_p W|_p$  for each definition of $W$ can be found in \autoref{welfareComparisonParameterSensitivityFull} and \autoref{levelsWelfareComparisonParameterSensitivityFull}.

Suppose that the log of each moment is assumed to have a standard deviation of $\sigma = 0.05$, and that this uncertainty is statistically independent across moments. The uncertainty propagates such that the standard deviation of the CEV welfare change is the square root of $(\nabla_m \tilde{W}|_m)^T \Sigma_m \nabla_m \tilde{W}|_m$, where $\Sigma_m = \sigma^2 I_{9\times 9}$. In this examples this yields 0.31 log points (0.45 percentage points). Also, in both cases the result is linear in $\sigma$. Hence with $\sigma = 0.1$, the result is 0.62 log points (0.90 percentage points), etc. 

The estimated welfare improvement is about 1.4\%. Taking the uncertainty into account, the $2\sigma$ uncertainty region excludes zero for $\sigma \le .112$. This suggests the result is quite sensitive to the moments and non-calibrated parameters used in the calibration. 


\begin{figure}[]
	\includegraphics[scale = 0.36]{../code/julia/figures/simpleModel/welfareComparisonSensitivityFull.pdf}
	\caption{Sensitivity of welfare comparison to moments. This is $(J^{-1})^T \nabla_p W$, where $W(p)$ maps log parameters to the log of the percentage change in BGP consumption which is equivalent to the change in welfare from changing $\kappa_c$ from $\infty$ to $0$ (i.e. going from banning to frictionlessly enforcing NCAs). The way to read this is the following. Looking at the column labeled \textit{E}, the chart says that a 1\% increase in the targeted employment share of young firms, which corresponds to a log change of about $0.01$, leads to a 4\% increase in the percentage CEV percentage welfare change. In this calibration it is about 1.42\%, so this is about $0.057$ percentage points.}
	\label{welfareComparisonSensitivityFull}
\end{figure}


\paragraph{When are NCAs bad for welfare?}

The sensitivity of the welfare improvement to the entry rate shown in (\ref{welfareComparisonSensitivityFull}) suggests that a calibration targeting a lower rate of creative destruction could have the opposite result. \autoref{calibration_lowEntry_summaryPlot} shows the analogue of \autoref{calibration_summaryPlot} if entry rate targeted is 4\% instead of 8.35\%. The model is again able to match the moments exactly; inferred parameter values are shown in \autoref{calibration_lowEntry_parameters}.

As expected, growth and welfare fall when $\kappa_C$ is reduced so that $x = 1$. Mathematically, this results from the much higher inferred value of $\lambda$: it is 1.65 in this calibration instead of 1.17 in the original calibration. Intuitively, the lower rate of entry means that each entry must have a higher effect on growth in order for the model to match the growth rate. Furthermore, as shown in \autoref{welfareComparisonParameterSensitivityFull}, the increase in $\lambda$ significantly reduces the welfare gain from reducing $\kappa_C$. A higher value of $\lambda$ weakens inequality (\ref{cs:growth_decreasing_condition}), reducing the effect of the incumbent disincentive on growth. While the inequality still holds, the extent of misallocation is weaker. Therefore, inducing further misallocation via an increase in $\kappa_C$ has a smaller negative effect on growth. The present exercise shows that these local relationships are to some extent global and in fact strong enough to switch the sign of the welfare comparison.

\begin{figure}[]
	\includegraphics[scale = 0.57]{../code/julia/figures/simpleModel/calibration_lowEntry_summaryPlot.pdf}
	\caption{Summary of equilibrium for baseline parameter values and various values of $\kappa_c$.}
	\label{calibration_lowEntry_summaryPlot}
\end{figure}

\begin{table}[]
	\centering
	\captionof{table}{Low entry rate calibration}\label{calibration_lowEntry_parameters}
	\begin{tabular}{rlll}
		\toprule \toprule
		Parameter & Value & Description & Source \tabularnewline
		\midrule
		$\rho$ & 0.0339 & Discount rate  & Indirect inference \tabularnewline
		$\theta$ & 2 & $\theta^{-1} = $ IES & External calibration 
		\tabularnewline
		$\beta$ & 0.094 & $\beta^{-1} = $ EoS intermediate goods & Exactly identified \tabularnewline 
		$\psi$ & 0.5 & Entrant R\&D elasticity & External calibration \tabularnewline
		$\lambda$ & 1.65 & Quality ladder step size & Indirect inference 
		\tabularnewline
		$\chi_I$ & 0.366 & Incumbent R\&D productivity & Indirect inference 
		\tabularnewline
		$\chi_E$ & 0.0474 & Entrant R\&D productivity & Indirect inference \tabularnewline 
		$\kappa_e$ & 0.703 & Non-R\&D entry cost & Indirect inference \tabularnewline
		$\nu$ & 0.0126 & Spinout generation rate  & Indirect inference\tabularnewline
		$\bar{L}_{RD}$ & 0.05 & R\&D labor allocation  & Normalization \tabularnewline
		\bottomrule
	\end{tabular}
\end{table}

\subsection{NCA tax (subsidy)}

Consider a planner that can subsidize or tax the use of NCAs. Specifically, suppose that the flow cost of using an NCA is with quality $q$ is $(\kappa_c + T_{NCA})\nu  V(q,t)$. If $T_{NCA} < 0$, the planner subsidizes the use of NCAs. This specification restricts the tax or subsidy to be proportional to $\nu V(q,t)$.\footnote{The scaling with $\nu$ is only a normalization since $\nu$ is assumed to be constant.} This scaling, or at least linear scaling with $q$, are essential for a BGP to exist. I also rule out $T_{NCA} < -\kappa_c$ so that NCAs are only used by incumbents and R\&D employees.

NCA taxes strictly dominate costly legal barriers to NCA enforcement. To be precise, for any equilibrium with $\kappa_{c0} > \munderbar{\kappa}_c$, there corresponds an equilibrium with $\kappa_{c1} = \munderbar{\kappa}_c$ and $T_NCA > \kappa_{c0} - \munderbar{\kappa}_c$ which has the same innovation rates and growth but weakly higher consumption and welfare. This results from the fact that $T_{NCA}$ is isomorphic to $\kappa_c$ except for the fact that it is a transfer, not a static social cost. 

In the baseline calibration, however, there is no social improvement to the availability of NCA taxes. Welfare is optimized either at $\kappa_c = 0$ or $\kappa_c > \bar{\kappa}_c$. In neither case is a cost paid to enforce NCAs in equilibrium, since $x = 0$ when $\kappa_c > \bar{\kappa}_c$. If a planner who can only control $\kappa_c$ a high value of $\kappa_c = \kappa_c^*$, the planner who can choose $T_{NCA}$ as well will also find this equilibrium optimal (that is, with $T_{NCA} = 0$); and will be indifferent between it and an identical equilibrium with $\kappa_c = 0$ and $T_{NCA} = \kappa_c^*$. 

If $\kappa_c = 0$ in the baseline social optimum, then it turns out that a subsidy to the use of NCAs could in principle improve welfare for two reasons. First, if $x = 1$ in the social optimum, then a subsidy to the use of NCAs becomes equivalent to a targeted subsidy to incumbent R\&D. If (\ref{cs:growth_decreasing_condition}) holds, the planner would like to subsidize incumbent R\&D specifically in order to reallocate R\&D labor to incumbents instead of entrants. I will take up this case in \autoref{cs:oi_rd_subsidy}.

If $\kappa_c = x = 0$ in the baseline social optimum, then subsidizing the use of NCAs so that $x = 1$ could potentially improve welfare if NCAs are underutilized from a social perspective even when they are free to enforce. However, I suspect this will typically not be the case, since the vast majority of any social harm from the spinout is borne by the incumbent who generates it. More likely, inducing $x = 1$ could be worthwhile because further subsidies to NCAs act as a targeted subsidy to incumbent R\&D.

\subsection{R\&D subsidy (tax)}

Suppose that the planner subsidizes R\&D spending at rate $T_{RD}$ (tax if $T_{RD} < 0$). Such a policy is natural to study since R\&D subsidies are prevalent and large throughout the developed world. In particular, in the United States they are large at the federal and state levels (something like 20\%).

In this case, the incumbent's HJB becomes
\begin{align}
	(r + \tau_E) \tilde{V} = \tilde{\pi} + \max_{\substack{x \in \{0,1\} \\ z \ge 0}} \Big\{z &\Big( \overbrace{\chi_I (\lambda - 1) \tilde{V}}^{\mathclap{\mathbb{E}[\textrm{Benefit from R\&D}]}}- (\underbrace{1-T_{RD}}_{\mathclap{\text{R\&D Subsidy}}}) \big( \overbrace{w_{RD,E} - (1-x)(1-\kappa_e)\lambda \nu \tilde{V}}^{\mathclap{\text{R\&D wage}}}\big) \label{eq:hjb_incumbent_RDsubsidy} \nonumber \\ 
	&-  \underbrace{(1-x) \nu \tilde{V}}_{\mathclap{\text{Net cost from spinout formation}}} - \overbrace{x \kappa_{c} \nu \tilde{V}}^{\mathclap{\text{Direct cost of NCA}}}\Big) \Big\} 
\end{align}

This can be rearranged to a form analogous to (\ref{eq:hjb_incumbent_workerIndiff}),
\begin{align}
	(r + \tau_E) \tilde{V} = \tilde{\pi} + \max_{\substack{x \in \{0,1\} \\ z \ge 0}} \Big\{z &\Big( \overbrace{\chi_I (\lambda - 1) \tilde{V}}^{\mathclap{\mathbb{E}[\textrm{Benefit from R\&D}]}}- (1-T_{RD}) w_{RD,E} \\
	&-  \underbrace{(1-x)(1 - (1-T_{RD})(1-\kappa_{e})\lambda)\nu \tilde{V}}_{\mathclap{\text{Net cost from spinout formation}}} - \overbrace{x \kappa_{c} \nu \tilde{V}}^{\mathclap{\text{Direct cost of NCA}}}\Big) \Big\} \label{eq:hjb_incumbent_RDsubsidy_2}
\end{align}

Define
\begin{align}
	\tilde{\bar{\kappa}}_c(\kappa_e,\lambda;T_{RD}) = 1 - (1-T_{RD})(1-\kappa_e)\lambda
\end{align} 

If $z_I > 0$, the incumbent's optimal NCA policy is given by 
\begin{align}
x = \begin{cases}
1 & \textrm{if } \kappa_{c} < \tilde{\bar{\kappa}}_c (\kappa_e, \lambda;T_{RD}) \\
0 & \textrm{if } \kappa_{c} > \tilde{\bar{\kappa}}_c (\kappa_e, \lambda;T_{RD})\\
\{0,1\} & \textrm{if } \kappa_c = \tilde{\bar{\kappa}}_c (\kappa_e, \lambda;T_{RD})
\end{cases} \label{eq:nca_policy_RDsubsidy}
\end{align}

Since the argument is the same as in Section \ref{subsubsec:dynamic_equilibrium_original_solution}, I will not be as detailed in my proof. Assuming $z_I > 0$, by the same logic as before the incumbent's FOC can be rearranged to
\begin{align}
	\tilde{V} &= \frac{(1-T_{RD})w_{RD,E}}{\chi_I(\lambda -1) - \nu (x\kappa_c + (1-x)(1 - (1-T_{RD})(1-\kappa_e)\lambda)) } \label{eq:hjb_incumbent_foc_RDsubsidy}
\end{align}

The free entry condition is
\begin{align}
\underbrace{\chi_E z_E^{-\psi}}_{\mathclap{\text{Marginal innovation rate}}} \overbrace{(1-\kappa_e) \lambda \tilde{V}}^{\mathclap{\text{Payoff from innovation}}} &= \overbrace{(\underbrace{1-T_{RD}}_{\mathclap{\text{R\&D subsidy}}})w_{RD,E}}^{\mathclap{\text{MC of R\&D}}} \label{eq:free_entry_condition_RDsubsidy}
\end{align}

Substituting (\ref{eq:hjb_incumbent_foc_RDsubsidy}) into (\ref{eq:free_entry_condition_RDsubsidy}) to eliminate $\tilde{V}$ yields an expression for $z_E$, 
\begin{align}
z_E &= \Bigg( \frac{\chi_E (1-\kappa_{e}) \lambda}{\chi_I(\lambda -1) - \nu (x\kappa_c + (1-x)(1 - (1-T_{RD})(1-\kappa_e)\lambda)) } \Bigg)^{1/\psi} \label{eq:effort_entrant_RDsubsidy}
\end{align}

The rest of the equilibrium allocation and prices can be computed by using the following equations in sequence to compute the variable on the LHS:
\begin{align}
	\tau_E &= \chi_E z_E^{1-\psi} \\
	z_I &= \bar{L}_{RD} - z_E \label{eq:labor_resource_constraint_RDsubsidy}\\ 
	\tau_I &= \chi_I z_I \\
	\tau_S &= (1-x) \nu z_I \\
	g &= (\lambda - 1) (\tau_I + \tau_S + \tau_E) \\
	r &= \theta g + \rho \\
	\tilde{V} &= \frac{\tilde{\pi}}{r + \tau_E} \\ 
	w_{RD,E} &= (1-T_{RD})^{-1}\chi_E z_E^{-\psi} (1-\kappa_e) \lambda \tilde{V} \label{eq:wage_rd_labor_RDsubsidy}
\end{align}

\subsubsection{Effect on growth}

First suppose $x = 0$ and consider a small increase in $T_{RD}$ from $T_{RD}^0$ to $T_{RD}^1 > T_{RD}^0$. If $x = 0$ after the increase in $T_{RD}$, then by (\ref{eq:effort_entrant_RDsubsidy}), $z_E$ increases; and by (\ref{eq:labor_resource_constraint_RDsubsidy}) $z_I$ decreases. If (\ref{cs:growth_decreasing_condition}) holds, this reduces growth. Intuitively, the increased R\&D subsidy reduces the wage expenses paid for R\&D by the same factor $1-\frac{1-T_{RD}^1}{1-T_{RD}^0}$ for incumbents and entrants. However, the incumbent's effective cost of R\&D also includes the shadow cost of more creative destruction by spinouts. Therefore, her effective cost of R\&D is reduced by a factor $\tilde{\tau}_{RD} < 1-\frac{1-T_{RD}^1}{1-T_{RD}^0}$. In equilibrium, R\&D labor is reallocated to entrants and growth falls.

If the increase in $T_{RD}$ is large enough, $x$ changes from $x = 0$ to $x = 1$ and therefore $\tau_S$ jumps to zero, reducing growth further. Intuitively, higher R\&D subsidies mean the incumbent prefers to pay for the R\&D with wages, which are tax-deductible, rather than implicitly through future spinouts, the implicit cost of which is not tax-deductible. Incumbents therefore opt to use NCAs, bringing spinout entry to zero and reducing growth by a discrete jump. In addition, there are no indirect effects on growth through changes in $z_E$,$z_I$, as these variables do not jump: according to (\ref{eq:nca_policy_RDsubsidy}), the transition from $x= 0$ to $x =1$ occurs at the value of $T_{RD}$ such that $\kappa_c$ is equal to the term multiplying $(1-x)$, implying that $z_E$, and therefore $z_I$, does not jump.

Finally, if $T_{RD}$ is increased even further, there is no change in the equilibrium allocation. The only change is the wage of R\&D labor, which by (\ref{eq:wage_rd_labor_RDsubsidy}) increases to equilibriate the R\&D labor market.

\subsubsection{Effect on consumption}

Steady state consumption is given by
\begin{align}
\tilde{C} &= \tilde{Y} - \Big( (\tau_E  + \tau_S)\kappa_e \lambda + x z_I \nu \kappa_c \Big) \tilde{V} \\
&= \tilde{Y} - \Big( \big( \chi_E (\bar{L}_{RD} - z_I)^{1-\psi} + (1-x) \nu z_I \big) \kappa_e \lambda + x z_I \nu \kappa_c \Big) \tilde{V}  \label{cs:scen2:consumption_eq}
\end{align}

As argued above $\tau_E$ is increasing in $T_{RD}$; $\tau_S$ is constant or decreasing in $T_{RD}$; and $z_I$ is decreasing in $T_{RD}$. The overall effect on steady state consumption depends on parameters. 

\subsubsection{Effect on welfare}

As numerical exercises show welfare effects are driven by productivity growth, rather than consumption at a given level of productivity, typically the effect of increasing $T_{RD}$ will be to reduce welfare, provided of course that (\ref{cs:growth_decreasing_condition}) holds. \autoref{calibration_RDSubsidy_summaryPlot} shows how the equilibrium varies with the $T_{RD}$. For this exercise, I set $\kappa_c = 1.2 \tilde{\bar{\kappa}}_c(\kappa_e,\lambda;T_{RD} = 0)$. 

\begin{figure}[]
	\includegraphics[scale = 0.57]{../code/julia/figures/simpleModel/calibration_RDSubsidy_summaryPlot.pdf}
	\caption{Summary of equilibrium for baseline parameter values and various values of $T_{RD}$. This assumes that $\kappa_c = 1.2 \tilde{\bar{\kappa}}_c(\kappa_e,\lambda;T_{RD} = 0)$.}
	\label{calibration_RDSubsidy_summaryPlot}
\end{figure}

Notice that growth (first row, third column) and welfare (third row, third column) both fall with $T_{RD}$, and jump down when the increase in $T_{RD}$ increases the use of NCAs.


\subsection{CD tax (subsidy)}

Suppose that the planner taxes entry at rate $T_e$ (subsidy if $T_e < 0$). Specifically, the planner taxes the entry fixed cost $\kappa_e \lambda \tilde{V} q$ at rate $T_e$ so that a firm entering with quality $\lambda q$ perceives a total cost of $(1+T_e) \kappa_e \lambda \tilde{V}q$ units of the final good. Economically, this can be interpreted as a tax on non-R\&D expenses related to the development of new versions of products currently not sold by the firm in question.\footnote{Because the tax is proportional to these expenses, rather than a fixed tax on entry, it does not induce any reallocation of R\&D towards higher quality goods. This property is not only analytically convenient -- it is necessary for a BGP to exist. In the baseline model, the expected growth rate of normalized frontier quality $\tilde{\bar{q}}_j = \frac{\bar{q}_j}{Q}$ is constant for all $j \in [0,1]$ and there is no exit of low quality firms (and subsequent injection of "average quality" firms). Running this stochastic process forward in time, the distribution of $\tilde{\bar{q}}_j$ spreads out, i.e. its variance and higher order measures of dispersion increase, which implies that there is no stationary distribution of $\tilde{\bar{q}}_j$. A BGP continues to exist, however, because only the mean of $\tilde{\bar{q}}_j$, $\mathbb{E}[\tilde{\bar{q}}_j] = 1$, is relevant for aggregate variables. This is why, e.g. the growth accounting equation (\ref{eq:growth_accounting}) can be written so simply. If, instead, growth is faster for higher $\tilde{\bar{q}}_j$, as is the case with a fixed entry fee, there is again no stationary distribution of $\tilde{\bar{q}}_j$, as before. However, in addition, there is no BGP, because aggregate variables such as the growth rate and the R\&D wage now depend on the entire distribution of $\tilde{\bar{q}}_j$, which is not stationary.}

In this setup, the R\&D labor supply indifference condition becomes
\begin{align}
	w_{RD,E} &= w_{RD,j} + (1-x_j) \nu (1-(1+T_e)\kappa_e) \lambda \tilde{V} \label{eq:RD_worker_indifference_entryTax}
\end{align}

The incumbent HJB is
\begin{align}
(r + \tau_E) \tilde{V} = \tilde{\pi} + \max_{\substack{x \in \{0,1\} \\ z \ge 0}} \Big\{z &\Big( \overbrace{\chi_I (\lambda - 1) \tilde{V}}^{\mathclap{\mathbb{E}[\textrm{Benefit from R\&D}]}}-  \big( \overbrace{w_{RD,E} - (1-x)(1-(1+T_e)\kappa_e)\lambda \nu \tilde{V}}^{\mathclap{\text{R\&D wage}}}\big) \label{eq:hjb_incumbent_entryTax} \nonumber \\ 
&-  \underbrace{(1-x) \nu \tilde{V}}_{\mathclap{\text{Net cost from spinout formation}}} - \overbrace{x \kappa_{c} \nu \tilde{V}}^{\mathclap{\text{Direct cost of NCA}}}\Big) \Big\} 
\end{align}

which can be rearranged to
\begin{align}
(r + \tau_E) \tilde{V} = \tilde{\pi} + \max_{\substack{x \in \{0,1\} \\ z \ge 0}} \Big\{z &\Big( \overbrace{\chi_I (\lambda - 1) \tilde{V}}^{\mathclap{\mathbb{E}[\textrm{Benefit from R\&D}]}}- w_{RD,E} \\
&-  \underbrace{(1-x)(1 - (1-(1+T_e)\kappa_{e})\lambda)\nu \tilde{V}}_{\mathclap{\text{Net cost from spinout formation}}} - \overbrace{x \kappa_{c} \nu \tilde{V}}^{\mathclap{\text{Direct cost of NCA}}}\Big) \Big\} \label{eq:hjb_incumbent_entryTax_2}
\end{align}

Define
\begin{align}
\hat{\bar{\kappa}}_c(\kappa_e,\lambda;T_e) = 1 - (1-(1+T_e)\kappa_e)\lambda  \label{eq:barkappa_entryTax}
\end{align} 

If $z_I > 0$, the incumbent's optimal NCA policy is given by 
\begin{align}
x = \begin{cases}
1 & \textrm{if } \kappa_{c} < \hat{\bar{\kappa}}_c (\kappa_e, \lambda;T_{RD}) \\
0 & \textrm{if } \kappa_{c} > \hat{\bar{\kappa}}_c (\kappa_e, \lambda;T_{RD})\\
\{0,1\} & \textrm{if } \kappa_c = \hat{\bar{\kappa}}_c (\kappa_e, \lambda;T_{RD})
\end{cases} \label{eq:nca_policy_entryTax}
\end{align}

By the usual argument, $z_I > 0$ implies that the incumbent's FOC can be rearranged to
\begin{align}
\tilde{V} &= \frac{w_{RD,E}}{\chi_I(\lambda -1) - \nu (x\kappa_c + (1-x)(1 - (1-(1+T_e)\kappa_e)\lambda)) } \label{eq:hjb_incumbent_foc_entryTax}
\end{align}

If $(1 + T_e) \kappa_e > 1$ then $z_E = 0$ and $z_I = \bar{L}_{RD}$. Otherwise, the free entry condition is
\begin{align}
\underbrace{\chi_E z_E^{-\psi}}_{\mathclap{\text{Marginal innovation rate}}} \overbrace{(1-(1+T_e)\kappa_e) \lambda \tilde{V}}^{\mathclap{\text{Payoff from innovation}}} &= \underbrace{w_{RD,E}}_{\mathclap{\text{MC of R\&D}}} \label{eq:free_entry_condition_entryTax}
\end{align}

Substituting (\ref{eq:hjb_incumbent_foc_entryTax}) into (\ref{eq:free_entry_condition_entryTax}) to eliminate $\tilde{V}$ yields an expression for $z_E$, 
\begin{align}
z_E &= \Bigg( \frac{\chi_E (1-(1+T_e)\kappa_{e}) \lambda}{\chi_I(\lambda -1) - \nu (x\kappa_c + (1-x)(1 - (1-(1+T_e)\kappa_e)\lambda)) } \Bigg)^{1/\psi} \label{eq:effort_entrant_entryTax}
\end{align}

From here, the rest of the model (including the case where $(1-\kappa_e)\lambda < 1$ and $z_E = 0$) can be solved using
\begin{align}
\tau_E &= \chi_E z_E^{1-\psi} \\
z_I &= \bar{L}_{RD} - z_E \label{eq:labor_resource_constraint_entryTax}\\ 
\tau_I &= \chi_I z_I \\
\tau_S &= (1-x) \nu z_I \\
g &= (\lambda - 1) (\tau_I + \tau_S + \tau_E) \\
r &= \theta g + \rho \\
\tilde{V} &= \frac{\tilde{\pi}}{r + \tau_E} \\ 
w_{RD,E} &= \begin{cases}
				\chi_E z_E^{-\psi} (1-(1+T_e)\kappa_e) \lambda \tilde{V} &\textrm{, if } z_E > 0\\
				\Big( \chi_I(\lambda -1) - \nu (x\kappa_c + (1-x)\hat{\bar{\kappa}}_c(\kappa_e,\lambda;T_e))\Big) \tilde{V} &\textrm{, o.w.}
			\end{cases} \label{eq:wage_rd_labor_entryTax}
\end{align}

\subsubsection{Effect on growth}

Suppose that $x = 1$ and the tax is increased from $T_e$ to $T_e' > T_e$. Then (\ref{eq:effort_entrant_entryTax}) implies that $z_E$ falls, (\ref{eq:labor_resource_constraint_entryTax}) implies that $z_I$ increase to keep $L_{RD} = \bar{L}_{RD}$, and then by (\ref{cs:growth_decreasing_condition}) growth increases. Intuitively, when $x = 1$ the only effect of the entry tax is to reduce the misallocation of R\&D labor to entrants. Because the laissez-faire equilibrium overallocates R\&D to entrants, due to the business-stealing effect, equilibrium growth increases.

However, if $x = 0$, the situation changes, for two reasons. First, as can be seen readily in (\ref{eq:effort_entrant_entryTax}), the effect of $T_e$ on $z_E$ is ambiguous, since the denominator now decreases in $T_e$ as well as the numerator. Intuitively, an increase in $T_e$ reduces the value of future spinouts, requiring incumbents to compensate workers with higher wages. Because the expected harm from future spinouts remains constant, this implies a net disincentive to incumbent R\&D. In equilibrium, labor is reallocated to the entrant. However, the mechanism in the first paragraph is still present and the net effect of the two could be positive or negative depending on parameters.

Second, by (\ref{eq:nca_policy_entryTax}) and (\ref{eq:barkappa_entryTax}), a sufficiently large increase in $T_e$ induces a change from $x = 0$ to $x = 1$. Intuitively, a higher $T_e$ means it is relatively more expensive for incumbents to compensate their employees with future spinouts. For a high enough $T_e$, incumbents prefer to use NCAs and pay their employees with wages directly.

\subsubsection{Effect on consumption}

As before, steady state consumption is given by
\begin{align}
\tilde{C} &= \tilde{Y} - \Big( (\tau_E  + \tau_S)\kappa_e \lambda + x z_I \nu \kappa_c \Big) \tilde{V} \\
&= \tilde{Y} - \Big( \big( \chi_E (\bar{L}_{RD} - z_I)^{1-\psi} + (1-x) \nu z_I \big) \kappa_e \lambda + x z_I \nu \kappa_c \Big) \tilde{V}  \label{cs:scen3:consumption_eq}
\end{align}

Even when the effect on $z_I,\tau_E,\tau_S$ could be computed analytically, the effect on consumption is ambiguous and depends on parameters. In this case, the effect of $T_e$ on $z_I,\tau_E,\tau_S$ in addition depends on parameters, so the situation is even less transparent. 

\subsubsection{Effect on welfare}

The plots in \autoref{calibration_entryTax_summaryPlot} show the impact of varying $T_e$ on equilibrium variables, growth and welfare, when $\kappa_c = 1.2 \hat{\tilde{\kappa}}_c(\kappa_e,\lambda;T_e = 0)$. As expected based on the discussion of the previous paragraphs, growth and welfare increase in the entry tax $T_e$ due to a reallocation of R\&D to the incumbent. When the tax $T_e$ is high enough (around 16\% below), incumbents begin to use noncompetes.  

\begin{figure}[]
	\includegraphics[scale = 0.57]{../code/julia/figures/simpleModel/calibration_EntryTax_summaryPlot.pdf}
	\caption{Summary of equilibrium for baseline parameter values and various values of $T_e$. This assumes that $\kappa_c = 1.2 \hat{\bar{\kappa}}_c(\kappa_e,\lambda;T_e = 0)$.}
	\label{calibration_entryTax_summaryPlot}
\end{figure}

\subsection{OI R\&D subsidy (tax)}\label{cs:oi_rd_subsidy}

Suppose that the plannner can subsidize R\&D spent on improving a product while excluding R\&D aiming at creative destruction. In the model, this corresponds to a targeted subsidy to R\&D spending by incumbents, of magnitude $T_{RD,I}$ (tax if $T_{RD,I} < 0$). In practice, this policy may be difficult to implement for the same reason as the CD tax. Firms may not be expected to be truthful regarding the purpose of their R\&D or the effect of their R\&D on their competitors' profits. It may not even be possible to tell in advance whether R\&D will result in creative CD, OI, or even new varieties of products. Furthermore, innovation to improve existing products can be a form of creative destruction. Nevertheless, it is still useful as a theoretical benchmark.

In this case, the incumbent's HJB becomes
\begin{align}
(r + \tau_E) \tilde{V} = \tilde{\pi} + \max_{\substack{x \in \{0,1\} \\ z \ge 0}} \Big\{z &\Big( \overbrace{\chi_I (\lambda - 1) \tilde{V}}^{\mathclap{\mathbb{E}[\textrm{Benefit from R\&D}]}}- (\underbrace{1-T_{RD,I}}_{\mathclap{\text{R\&D Subsidy}}}) \big( \overbrace{w_{RD,E} - (1-x)(1-\kappa_e)\lambda \nu \tilde{V}}^{\mathclap{\text{R\&D wage}}}\big) \label{eq:hjb_incumbent_RDsubsidyTargeted} \nonumber \\ 
&-  \underbrace{(1-x) \nu \tilde{V}}_{\mathclap{\text{Net cost from spinout formation}}} - \overbrace{x \kappa_{c} \nu \tilde{V}}^{\mathclap{\text{Direct cost of NCA}}}\Big) \Big\} 
\end{align}

This can be rearranged to a form analogous to (\ref{eq:hjb_incumbent_workerIndiff}),
\begin{align}
(r + \tau_E) \tilde{V} = \tilde{\pi} + \max_{\substack{x \in \{0,1\} \\ z \ge 0}} \Big\{z &\Big( \overbrace{\chi_I (\lambda - 1) \tilde{V}}^{\mathclap{\mathbb{E}[\textrm{Benefit from R\&D}]}}- (1-T_{RD,I}) w_{RD,E} \\
&-  \underbrace{(1-x)(1 - (1-T_{RD,I})(1-\kappa_{e})\lambda)\nu \tilde{V}}_{\mathclap{\text{Net cost from spinout formation}}} - \overbrace{x \kappa_{c} \nu \tilde{V}}^{\mathclap{\text{Direct cost of NCA}}}\Big) \Big\} \label{eq:hjb_incumbent_RDsubsidyTargeted_2}
\end{align}

The non-compete policy is the same as with untargeted R\&D subsidies. That is, define
\begin{align}
\tilde{\bar{\kappa}}_c(\kappa_e,\lambda;T_{RD}) = 1 - (1-T_{RD,I})(1-\kappa_e)\lambda
\end{align} 

Then $z_I > 0$ implies that the incumbent's optimal NCA policy is given by 
\begin{align}
x = \begin{cases}
1 & \textrm{if } \kappa_{c} < \tilde{\bar{\kappa}}_c (\kappa_e, \lambda;T_{RD,I}) \\
0 & \textrm{if } \kappa_{c} > \tilde{\bar{\kappa}}_c (\kappa_e, \lambda;T_{RD,I})\\
\{0,1\} & \textrm{if } \kappa_c = \tilde{\bar{\kappa}}_c (\kappa_e, \lambda;T_{RD,I})
\end{cases} \label{eq:nca_policy_RDsubsidyTargeted}
\end{align}

Assuming $z_I > 0$, the FOC of the incumbent HJB implies
\begin{align}
\tilde{V} &= \frac{(1-T_{RD,I})w_{RD,E}}{\chi_I(\lambda -1) - \nu (x\kappa_c + (1-x)(1 - (1-T_{RD,I})(1-\kappa_e)\lambda)) } \label{eq:hjb_incumbent_foc_RDsubsidyTargeted}
\end{align}

The free entry condition is
\begin{align}
\underbrace{\chi_E z_E^{-\psi}}_{\mathclap{\text{Marginal innovation rate}}} \overbrace{(1-\kappa_e) \lambda \tilde{V}}^{\mathclap{\text{Payoff from innovation}}} &= \underbrace{w_{RD,E}}_{\mathclap{\text{MC of R\&D}}} \label{eq:free_entry_condition_RDsubsidyTargeted}
\end{align}

Substituting (\ref{eq:hjb_incumbent_foc_RDsubsidyTargeted}) into (\ref{eq:free_entry_condition_RDsubsidyTargeted}) to eliminate $\tilde{V}$ yields an expression for $z_E$, 
\begin{align}
z_E &= \Bigg( \frac{(1-T_{RD,I})\chi_E (1-\kappa_{e}) \lambda}{\chi_I(\lambda -1) - \nu (x\kappa_c + (1-x)(1 - (1-T_{RD,I})(1-\kappa_e)\lambda)) } \Bigg)^{1/\psi} \label{eq:effort_entrant_RDsubsidyTargeted}
\end{align}

The rest of the equilibrium allocation and prices can be computed by using the following equations in sequence to compute the variable on the LHS:
\begin{align}
\tau_E &= \chi_E z_E^{1-\psi} \\
z_I &= \bar{L}_{RD} - z_E \label{eq:labor_resource_constraint_RDsubsidyTargeted}\\ 
\tau_I &= \chi_I z_I \\
\tau_S &= (1-x) \nu z_I \\
g &= (\lambda - 1) (\tau_I + \tau_S + \tau_E) \\
r &= \theta g + \rho \\
\tilde{V} &= \frac{\tilde{\pi}}{r + \tau_E} \\ 
w_{RD,E} &= \chi_E z_E^{-\psi} (1-\kappa_e) \lambda \tilde{V} \label{eq:wage_rd_labor_RDsubsidyTargeted}
\end{align}

\subsubsection{Effect on growth}

If $x = 1$, increasing $T_{RD,I}$ reduces $z_E$ by (\ref{eq:effort_entrant_RDsubsidyTargeted}) and will increase growth if the condition (\ref{cs:growth_decreasing_condition}) holds. Intuitively, a subsidy to incumbent R\&D causes the R\&D wage to increase, reducing R\&D by the entrant in equilibrium. 

If $x = 0$, increasing $T_{RD,I}$ has a more complicated effect on $z_E$ because it reduces the denominator as well. This follows from the same reasoning as in the case of the untargeted R\&D subsidy: incumbents pay partially through future spinouts and so not all of their costs are subsidized at rate $T_{RD,I}$. From this economic interpretation, it follows immediately that the net effect is still to reduce incumbent R\&D expenses relative to those of the entrant and hence to lower $z_E$ and increase $z_I$, and this is confirmed in the numerical analysis of the next subsection.

Finally, (\ref{eq:nca_policy_RDsubsidyTargeted}) implies that if the increase in $T_{RD,I}$ is sufficiently large, it will induce the use of NCAs by incumbents. As in the case of the untargeted R\&D subsidy, targeted R\&D subsidies do not reduce the harm to the incumbent's profits due to future employee spinouts. At a certain point, the incumbent prefers the higher but tax-deductible wages of an NCA contract. This switch unambiguously reduces growth.

The last observation implies that even targeted R\&D subsidies are unable to achieve the socially valuable outcome high spinout entry and high incumbent R\&D. In order to achieve this result, it is necessary to pair the targeted R\&D subsidy with an increase in legal barriers to NCAs $\kappa_c$ or an increase in the tax on NCA usage $T_{NCA}$. 

\subsubsection{Effect on consumption}

As in the previous cases, the effect on consumption is not possible to determine in general due to the interplay between entry costs and NCA costs. 

\subsubsection{Effect on welfare}

In this calibraiton, an increase in $T_{RD,I}$ significantly increases growth and welfare, as shown in \autoref{calibration_RDSubsidyTargeted_summaryPlot}. It does so by increasing incumbent R\&D. For high values of $T_{RD,I}$, there is a switch from $x = 0$ to $x = 1$ and welfare jumps down slightly. Otherwise, it is monotonically increasing. 

\begin{figure}[]
	\includegraphics[scale = 0.57]{../code/julia/figures/simpleModel/calibration_RDSubsidyTargeted_summaryPlot.pdf}
	\caption{Summary of equilibrium for baseline parameter values and various values of $T_{RD,I}$. This assumes that $\kappa_c = 1.2 \tilde{\bar{\kappa}}_c(\kappa_e,\lambda;T_{RD,I} = 0)$.}
	\label{calibration_RDSubsidyTargeted_summaryPlot}
\end{figure}


\subsection{All policies}

Now consider a planner who can use either of the instruments described above. In this setup, the R\&D labor supply indifference condition becomes
\begin{align}
w_{RD,E} &= w_{RD,j} + (1-x_j) \nu (1-(\underbrace{1+T_e}_{\mathclap{\text{Entry tax}}})\kappa_e) \lambda \tilde{V} \label{eq:RD_worker_indifference_all}
\end{align}

The incumbent HJB is
\begin{align}
(r + \tau_E) \tilde{V} = \tilde{\pi} + \max_{\substack{x \in \{0,1\} \\ z \ge 0}} \Big\{z &\Big( \overbrace{\chi_I (\lambda - 1) \tilde{V}}^{\mathclap{\mathbb{E}[\textrm{Benefit from R\&D}]}}-  (\underbrace{1 - T_{RD} - T_{RD,I}}_{\mathclap{\text{R\&D subsidies}}})\big( \overbrace{w_{RD,E} - (1-x)(1-(1+T_e)\kappa_e)\lambda \nu \tilde{V}}^{\mathclap{\text{R\&D wage}}}\big) \label{eq:hjb_incumbent_all} \nonumber \\ 
&-  \underbrace{(1-x) \nu \tilde{V}}_{\mathclap{\text{Net cost from spinout formation}}} - \overbrace{x (\kappa_{c} + T_{NCA}) \nu \tilde{V}}^{\mathclap{\text{Direct cost of NCA}}}\Big) \Big\} 
\end{align}

which can be rearranged to
\begin{align}
(r + \tau_E) \tilde{V} = \tilde{\pi} + \max_{\substack{x \in \{0,1\} \\ z \ge 0}} \Big\{z &\Big( \overbrace{\chi_I (\lambda - 1) \tilde{V}}^{\mathclap{\mathbb{E}[\textrm{Benefit from R\&D}]}}- (1-T_{RD}- T_{RD,I})w_{RD,E} \\
&-  \underbrace{(1-x)(1 - (1-T_{RD} - T_{RD,I})(1-(1+T_e)\kappa_{e})\lambda)\nu \tilde{V}}_{\mathclap{\text{Net cost from spinout formation}}} - \overbrace{x (\kappa_{c} + T_{NCA}) \nu \tilde{V}}^{\mathclap{\text{Direct cost of NCA}}}\Big) \Big\} \label{eq:hjb_incumbent_all_2}
\end{align}

Define
\begin{align}
\bar{\bar{\kappa}}_c(\kappa_e,\lambda;T_{RD},T_{RD,I},T_e) = 1 - (1-T_{RD} - T_{RD,I})(1-(1+T_e)\kappa_e)\lambda  \label{eq:barkappa_all}
\end{align} 

If $z_I > 0$, the incumbent's optimal NCA policy is given by 
\begin{align}
x = \begin{cases}
1 & \textrm{if } \kappa_{c} + T_{NCA} < \bar{\bar{\kappa}}_c (\kappa_e, \lambda;T_{RD},T_{RD,I},T_e)\\
0 & \textrm{if } \kappa_{c} + T_{NCA} > \bar{\bar{\kappa}}_c (\kappa_e, \lambda;T_{RD},T_{RD,I},T_e)\\
\{0,1\} & \textrm{if } \kappa_c + T_{NCA} = \bar{\bar{\kappa}}_c (\kappa_e, \lambda;T_{RD},T_{RD,I},T_e)
\end{cases} \label{eq:nca_policy_all}
\end{align}


By the usual argument, $z_I > 0$ implies that the incumbent's FOC can be rearranged to
\begin{align}
\tilde{V} &= \frac{(1-T_{RD} - T_{RD,I})w_{RD,E}}{\chi_I(\lambda -1) - \nu (x(\kappa_c + T_{NCA}) + (1-x)(1 - (1-T_{RD} - T_{RD,I})(1-(1+T_e)\kappa_e)\lambda)) } \label{eq:hjb_incumbent_foc_all}
\end{align}

The free entry condition is
\begin{align}
\underbrace{\chi_E z_E^{-\psi}}_{\mathclap{\text{Marginal innovation rate}}} \overbrace{(1-(1+T_e)\kappa_e) \lambda \tilde{V}}^{\mathclap{\text{Payoff from innovation}}} &= (1-T_{RD})\underbrace{w_{RD,E}}_{\mathclap{\text{MC of R\&D}}} \label{eq:free_entry_condition_all}
\end{align}

Substituting (\ref{eq:hjb_incumbent_foc_all}) into (\ref{eq:free_entry_condition_all}) to eliminate $\tilde{V}$ yields an expression for $z_E$, 
\begin{align}
z_E &= \Bigg( \frac{\Big(\frac{1-T_{RD} -T_{RD,I}}{1-T_{RD}} \Big)\chi_E (1-(1+T_e)\kappa_{e}) \lambda}{\chi_I(\lambda -1) - \nu (x(\kappa_c + T_{NCA}) + (1-x)(1 - (1-T_{RD} - T_{RD,I})(1-(1+T_e)\kappa_e)\lambda)) } \Bigg)^{1/\psi} \label{eq:effort_entrant_all}
\end{align}

From here, the rest of the model can be solved using
\begin{align}
\tau_E &= \chi_E z_E^{1-\psi} \\
z_I &= \bar{L}_{RD} - z_E \label{eq:labor_resource_constraint_all}\\ 
\tau_I &= \chi_I z_I \\
\tau_S &= (1-x) \nu z_I \\
g &= (\lambda - 1) (\tau_I + \tau_S + \tau_E) \\
r &= \theta g + \rho \\
\tilde{V} &= \frac{\tilde{\pi}}{r + \tau_E} \\ 
w_{RD,E} &= \begin{cases}
(1-T_{RD})^{-1}\chi_E z_E^{-\psi} (1-(1+T_e)\kappa_e) \lambda \tilde{V} &\textrm{, if } z_E > 0\\
\Big( \chi_I(\lambda -1) - \nu (x(\kappa_c + T_{NCA}) + (1-x)\bar{\bar{\kappa}}_c)\Big) \tilde{V} &\textrm{, o.w.}
\end{cases} \label{eq:wage_rd_labor_all}
\end{align}

\subsubsection{Optimal policy}

Because $T_{NCA}$ dominates $\kappa_c$, we know that the planner will choose $\kappa_c = \munderbar{\kappa}_c$. Next, the planner wants to reallocate R\&D to the incumbent and ensure that $x = 0$ so that spinout potential is not wasted. He can accomplish this by raising both $T_{RD,I}$, which reallocates R\&D labor to the incumbent, and $T_{NCA}$, which prevents NCAs so that $x = 0$. \autoref{calibration_ALL_summaryPlot} confirms this intuition (second row, third column). The improvement in CEV welfare is more than 7\%, which is significantly more than any of the other improvements could achieve on their own. This is driven by the fact that $T_{NCA}$ raises the threshold at which $T_{RD,I}$ induces a switch to $x = 1$ (first row, third column). [\textbf{Would be better to look at the effect of a reasonable sized subsidy, rather than full effect, but I'd rather discuss before doing that.}] 

\begin{figure}[]
	\includegraphics[scale = 0.46]{../code/julia/figures/simpleModel/calibration_ALL_summaryPlot.png}
	\caption{Summary of equilibrium for various values of $T_{RD,I}$ and $T_{NCA}$. This assumes the planner chooses $\kappa_c = \underline{\kappa}_c = \frac{1}{2} \bar{\bar{\kappa}}_c(\kappa_e,\lambda;T_{RD},T_{RD,I},T_e))$.}
	\label{calibration_ALL_summaryPlot}
\end{figure}

\bibliography{references.bib}

\appendix

\section{Appendix of tables}

% latex table generated in R 3.6.3 by xtable 1.8-4 package
% Sat Sep 26 15:59:48 2020
\begin{sidewaystable}[!htb]
\centering
\begingroup\tiny
\begin{tabular}{p{1.75cm}p{1.75cm}p{1.75cm}p{1.75cm}p{1.75cm}p{1.75cm}p{1.75cm}p{1.75cm}}
  \toprule
Year & Number of founders & Number of start-ups & Number of founders from public companies & Fraction from public companies (\%) & Fraction from public companies when bio. info available (\%) & Fraction from public companies in same 4-digit NAICS (\%) & Fraction from public companies in same 4-digit NAICS when bio. info available (\%) \\ 
  \midrule
1986 & 269 & 216 & 45 & 16.7 & 22.8 & 5.2 & 7.1 \\ 
  1987 & 356 & 280 & 43 & 12.1 & 15.1 & 3.9 & 4.9 \\ 
  1988 & 372 & 281 & 58 & 15.6 & 19.9 & 4.6 & 5.8 \\ 
  1989 & 479 & 341 & 75 & 15.7 & 19.2 & 4.2 & 5.1 \\ 
  1990 & 478 & 329 & 85 & 17.8 & 21.1 & 6.3 & 7.5 \\ 
  1991 & 540 & 356 & 81 & 15.0 & 17.9 & 6.3 & 7.5 \\ 
  1992 & 674 & 450 & 100 & 14.8 & 17.9 & 3.3 & 3.9 \\ 
  1993 & 778 & 490 & 137 & 17.6 & 20.3 & 6.7 & 7.7 \\ 
  1994 & 999 & 611 & 167 & 16.7 & 19.3 & 4.9 & 5.7 \\ 
  1995 & 1326 & 772 & 224 & 16.9 & 19.0 & 5.2 & 5.8 \\ 
  1996 & 1926 & 1077 & 319 & 16.6 & 18.1 & 4.9 & 5.3 \\ 
  1997 & 1986 & 1036 & 345 & 17.4 & 19.0 & 5.9 & 6.5 \\ 
  1998 & 2895 & 1390 & 541 & 18.7 & 19.6 & 5.2 & 5.5 \\ 
  1999 & 5189 & 2388 & 975 & 18.8 & 19.6 & 5.0 & 5.2 \\ 
  2000 & 4084 & 1832 & 786 & 19.2 & 20.4 & 5.2 & 5.5 \\ 
  2001 & 2245 & 948 & 384 & 17.1 & 18.7 & 6.3 & 6.9 \\ 
  2002 & 2113 & 884 & 385 & 18.2 & 20.1 & 7.3 & 8.0 \\ 
  2003 & 1979 & 903 & 344 & 17.4 & 19.8 & 7.5 & 8.5 \\ 
  2004 & 2098 & 988 & 365 & 17.4 & 20.1 & 6.8 & 7.9 \\ 
  2005 & 2278 & 1068 & 400 & 17.6 & 20.7 & 6.5 & 7.7 \\ 
  2006 & 2492 & 1212 & 432 & 17.3 & 20.5 & 6.3 & 7.5 \\ 
  2007 & 2817 & 1366 & 388 & 13.8 & 17.0 & 4.9 & 6.1 \\ 
  2008 & 2710 & 1307 & 422 & 15.6 & 19.1 & 5.4 & 6.6 \\ 
   \bottomrule
\end{tabular}
\endgroup
\caption{\scriptsize Summary of founders. Here, "founder" includes all individuals employed at startups inthe VentureSource database who (1) joined the startup within 3 year(s) of its founding year; and (2) have the title of CEO, CTO, CCEO, PCEO, PRE, PCHM, PCOO, FDR, CHF.} 
\label{table:GStable_founder2}
\end{sidewaystable}


% latex table generated in R 3.4.4 by xtable 1.8-4 package
% Thu Feb  6 14:38:22 2020
\begin{table}[!htb]
\centering
\begingroup\scriptsize
\begin{tabular}{p{4.5cm}llrllrll}
  \toprule
Industry & Startups & Individuals & State & Startups & Individuals & Year & Startups & Individuals \\ 
  \midrule
Business Applications Software & 1790 & 31218 & California & 8433 & 140958 & 1986 & 293 & 2103 \\ 
  Biotechnology Therapeutics & 1037 & 19264 & Massachussetts & 2217 & 37185 & 1987 & 353 & 2732 \\ 
  Communications Software & 996 & 14859 & New York & 1490 & 26450 & 1988 & 356 & 2877 \\ 
  Advertising/Marketing & 880 & 15211 & Texas & 1299 & 18452 & 1989 & 403 & 3293 \\ 
  Network/Systems Management Software & 671 & 13907 & Pennsylvania & 883 & 10759 & 1990 & 396 & 3222 \\ 
  Vertical Market Applications Software & 536 & 8401 & Washington & 784 & 12187 & 1991 & 422 & 3801 \\ 
  Online Communities & 467 & 6460 & Virginia & 606 & 8964 & 1992 & 537 & 4896 \\ 
  Application-Specific Integrated Circuits & 463 & 6475 & Colorado & 605 & 9337 & 1993 & 554 & 5322 \\ 
  Wired Communications Equipment & 458 & 6808 & Georgia & 562 & 7426 & 1994 & 689 & 6771 \\ 
  IT Consulting & 451 & 6378 & New Jersey & 557 & 7309 & 1995 & 876 & 8946 \\ 
  Drug Development Technologies & 400 & 5725 & Florida & 533 & 6524 & 1996 & 1191 & 13134 \\ 
  Healthcare Administration Software & 378 & 6500 & Illinois & 525 & 8054 & 1997 & 1141 & 13468 \\ 
  Fiberoptic Equipment & 362 & 4981 & North Carolina & 455 & 6333 & 1998 & 1513 & 19512 \\ 
  Therapeutic Devices (Minimally Invasive/Noninvasive) & 358 & 5635 & Maryland & 430 & 6223 & 1999 & 2557 & 32495 \\ 
  Business Support Services: Other & 341 & 4087 & Minnesota & 373 & 4661 & 2000 & 2003 & 24276 \\ 
  Procurement/Supply Chain & 325 & 4941 & Connecticut & 355 & 4614 & 2001 & 1067 & 13295 \\ 
  Multimedia/Streaming Software & 322 & 4460 & Ohio & 346 & 3876 & 2002 & 986 & 12946 \\ 
  Wireless Communications Equipment & 319 & 5045 & Utah & 249 & 3407 & 2003 & 1037 & 11922 \\ 
  Database Software & 318 & 6701 & Tennessee & 217 & 2828 & 2004 & 1110 & 13363 \\ 
  Specialty Retailers & 309 & 3354 & Oregon & 209 & 3071 & 2005 & 1222 & 13318 \\ 
  Entertainment & 295 & 3676 & Arizona & 207 & 2770 & 2006 & 1380 & 13829 \\ 
  Pharmaceuticals & 289 & 4282 & Michigan & 191 & 2460 & 2007 & 1506 & 13058 \\ 
  Therapeutic Devices (Invasive) & 285 & 3808 & Wisonsin & 140 & 1508 & 2008 & 1416 & 10504 \\ 
   \bottomrule
\end{tabular}
\endgroup
\caption{Statistics on startups covered by VS sample. Industry information uses VS industrial classification. Startups are counted by founding year, individuals by year they joined the firm.} 
\label{table:VS_summaryTable}
\end{table}


\begin{table}[!htb]
	\centering
	\captionof{table}{Alternative calibration}\label{calibration_2_parameters}
	\begin{tabular}{rlll}
		\toprule \toprule
		Parameter & Value & Description & Source \tabularnewline
		\midrule
		$\rho$ & 0.0339 & Discount rate  & Indirect inference \tabularnewline
		$\theta$ & 2 & $\theta^{-1} = $ IES & External calibration 
		\tabularnewline
		$\beta$ & 0.094 & $\beta^{-1} = $ EoS intermediate goods & Exactly identified \tabularnewline 
		$\psi$ & 0.5 & Entrant R\&D elasticity & External calibration \tabularnewline
		$\lambda$ & 1.170 & Quality ladder step size & Indirect inference 
		\tabularnewline
		$\chi_I$ & 1.80 & Incumbent R\&D productivity & Indirect inference 
		\tabularnewline
		$\chi_E$ & 0.115 & Entrant R\&D productivity & Indirect inference \tabularnewline 
		$\kappa_e$ & 0.737 & Non-R\&D entry cost & Indirect inference \tabularnewline
		$\nu$ & 0.04766 & Spinout generation rate  & Indirect inference\tabularnewline
		$\bar{L}_{RD}$ & 0.05 & R\&D labor allocation  & Normalization \tabularnewline
		\bottomrule
	\end{tabular}
\end{table}

\newpage
\section{Appendix of figures}

\begin{figure}[!htb]
	\centering
	\includegraphics[scale=0.85]{../empirics/figures/plots/industry_row_heatmap_naics2_founder2.pdf}
	\caption{Heatmap displaying the distribution of child 2-digit NAICS code (column), conditional on parent NAICS code (row). Darker hues indicate a higher density.}
	\label{figure:industry_row_heatmap_naics2_founder2}
\end{figure}

\begin{figure}[!htb]
	\centering
	\includegraphics[scale=0.85]{../empirics/figures/plots/industry_column_heatmap_naics2_founder2.pdf}
	\caption{Heatmap displaying the distribution of parent 2-digit NAICS code (row), conditional on child NAICS code (column). Darker hues indicate a higher density.}
	\label{figure:industry_column_heatmap_naics2_founder2}
\end{figure}

\begin{figure}[!htb]
	\centering
	\includegraphics[scale= 0.7]{../empirics/figures/scatterPlot_RD-Founders.png}
	\caption{Scatterplot of average yearly founder counts in $t+1,t+2,t+3$ versus average yearly R\&D spending in $t,t-1,t-2$.}
	\label{figure:scatterPlot_RD-Founders}
\end{figure}

\begin{figure}[!htb]
	\includegraphics[scale = 0.36]{../code/julia/figures/simpleModel/levelsWelfareComparisonSensitivityFull.png}
	\caption{Sensitivity of welfare comparison to moments. This is $(J^{-1})^T \nabla_p W$, where $W(p)$ maps log parameters to the percentage change in BGP consumption which is equivalent to the change in welfare from changing $\kappa_c$ from $\infty$ to $0$ (i.e. going from banning to frictionlessly enforcing NCAs). In contrast with the elasticity of the previous figure, this is a semi-elasticity. In particular it can allow for the change in welfare to be negative. The way to read this is the following. Looking at the column labeled \textit{E}, the chart says that a 1\% increase in the targeted employment share of young firms, which corresponds to a log change of about $0.01$, leads to an increase in the \% welfare improvement of approximately $6 \times 0.01 = 0.06$ percentage points.}
	\label{levelsWelfareComparisonSensitivityFull}
\end{figure}

\begin{figure}[]
	\includegraphics[scale = 0.36]{../code/julia/figures/simpleModel/welfareComparisonParameterSensitivityFull.png}
	\caption{Sensitivity of welfare comparison to moments. This is $\nabla_p W$, wahere $W(p)$ maps log parameters to the log of the percentage change in BGP consumption which is equivalent to the change in welfare from changing $\kappa_c$ from $\infty$ to $0$ (i.e. going from banning to frictionlessly enforcing NCAs).}
	\label{welfareComparisonParameterSensitivityFull}
\end{figure}

\begin{figure}[]
	\includegraphics[scale = 0.36]{../code/julia/figures/simpleModel/levelsWelfareComparisonParameterSensitivityFull.png}
	\caption{Sensitivity of welfare comparison to moments. This is $\nabla_p W$, wahere $W(p)$ maps log parameters to the percentage change in BGP consumption which is equivalent to the change in welfare from changing $\kappa_c$ from $\infty$ to $0$ (i.e. going from banning to frictionlessly enforcing NCAs).}
	\label{levelsWelfareComparisonParameterSensitivityFull}
\end{figure}





\end{document}