\documentclass[11pt,english]{article}
\usepackage{lmodern}
\linespread{1.05}
%\usepackage{mathpazo}
%\usepackage{mathptmx}
%\usepackage{utopia}
\usepackage{microtype}



\usepackage{chngcntr}
\usepackage[nocomma]{optidef}

\usepackage[section]{placeins}
\usepackage[T1]{fontenc}
\usepackage[latin9]{inputenc}
\usepackage[dvipsnames]{xcolor}
\usepackage{geometry}

\usepackage{babel}
\usepackage{amsmath}
\usepackage{graphicx}
\usepackage{amsthm}
\usepackage{amssymb}
\usepackage{bm}
\usepackage{bbm}
\usepackage{amsfonts}

\usepackage{accents}
\newcommand\munderbar[1]{%
	\underaccent{\bar}{#1}}


\usepackage{svg}
\usepackage{booktabs}
\usepackage{caption}
\usepackage{blindtext}
%\renewcommand{\arraystretch}{1.2}
\usepackage{multirow}
\usepackage{float}
\usepackage{rotating}
\usepackage{mathtools}
\usepackage{chngcntr}

% TikZ stuff

\usepackage{tikz}
\usepackage{mathdots}
\usepackage{yhmath}
\usepackage{cancel}
\usepackage{color}
\usepackage{siunitx}
\usepackage{array}
\usepackage{gensymb}
\usepackage{tabularx}
\usetikzlibrary{fadings}
\usetikzlibrary{patterns}
\usetikzlibrary{shadows.blur}

\usepackage[font=small]{caption}
%\usepackage[printfigures]{figcaps}
%\usepackage[nomarkers]{endfloat}


%\usepackage{caption}
%\captionsetup{justification=raggedright,singlelinecheck=false}

\usepackage{courier}
\usepackage{verbatim}
\usepackage[round]{natbib}

\bibliographystyle{plainnat}

\definecolor{red1}{RGB}{128,0,0}
%\geometry{verbose,tmargin=1.25in,bmargin=1.25in,lmargin=1.25in,rmargin=1.25in}
\geometry{verbose,tmargin=1in,bmargin=1in,lmargin=1in,rmargin=1in}
\usepackage{setspace}

\usepackage[colorlinks=true, linkcolor={red!70!black}, citecolor={blue!50!black}, urlcolor={blue!80!black}]{hyperref}

\let\oldFootnote\footnote
\newcommand\nextToken\relax

\renewcommand\footnote[1]{%
	\oldFootnote{#1}\futurelet\nextToken\isFootnote}

\newcommand\isFootnote{%
	\ifx\footnote\nextToken\textsuperscript{,}\fi}

%\usepackage{esint}
\onehalfspacing

%\theoremstyle{remark}
%\newtheorem{remark}{Remark}
%\newtheorem{theorem}{Theorem}[section]
\newtheorem{assumption}{Assumption}
\newtheorem{proposition}{Proposition}
\newtheorem{proposition_corollary}{Corollary}[proposition]
\newtheorem{lemma}{Lemma}
\newtheorem{lemma_corollary}{Corollary}[lemma]

\begin{document}
	
\title{Creating Creative Destruction: Endogenous Growth with Employee Spinouts and Non-compete Agreements}

\author{Nicolas Fernandez-Arias} 
\date{\today \\ \small
	\href{https://drive.google.com/file/d/17bZL7-AUJKllRb78r9fIkZscnNdJwo1G/view?usp=sharing}{Click for most recent version}}

%\date{\today}

\maketitle


%\setcounter{tocdepth}{2}
%\tableofcontents

\begin{abstract}
	I study the effect of non-compete agreements (NCAs) on aggregate productivity growth. I first develop a quality ladders model of endogenous growth augmented with NCAs and within-industry employee spinouts (WSOs). I then assemble a new dataset of VC-funded startups matched to the previous employers of their founding members. I find a statistically and economically significant relationship between corporate R\&D and employee spinout formation which quantitatively can account for 75\% of employee departures to WSOs in the data. I calibrate the model to match the micro estimates, aggregate moments and estimates from the literature. According to the calibrated model, reducing all barriers to the use of NCAs increases welfare by 3\% in consumption equivalent terms by improving the allocation of R\&D spending. Blanket R\&D subsidies can reduce growth and welfare by overallocating R\&D to creative destruction. The optimal policy is a combination of large R\&D subsidies targeted at own-product innovation and a ban on the use of NCAs.
\end{abstract}

\section{Introduction}

Knowledge spillovers and firm entry are major contributors to aggregate productivity growth. In fact, these two phenomena are not distinct, as firm entry is frequently the result of knowledge spillovers from existing firms. In particular, within-industry employee spinouts (WSOs) -- new firms founded by former employees of incumbent firms in the same industry -- often take advantage of knowledge gained at previous employers. Figure \ref{fairchild_spinouts} shows the many direct and indirect spinouts of Fairchild Semiconductor, one of the first leading semiconductor firms of Silicon Valley -- itself a spinout of Shockley Laboratories, another semiconductor firm. Although Fairchild was founded in the 1950s, its list of spinouts includes some of the most well-known modern firms in the industry, such as Intel and AMD. 

\begin{figure}	\phantomsection
	\center
	\includegraphics[scale = 0.77]{../figures/fairchildren_early.png}
	\caption{Direct and indirect spinouts of Fairchild Semiconductor}
	\label{fairchild_spinouts}
\end{figure}

To avoid the possibility of competition from such firms, incumbents may reduce their investment in R\&D and other forms of costly knowledge creation, which necessarily involve training potential future rivals. Alternatively, they may take steps to prevent WSOs directly, mitigate the disincentive to their own innovative efforts but simultaneously preventing productivity-enhancing knowledge spillovers. The most salient example of this kind of effort is the non-compete agreement (NCA), an employment contract which precludes the employee from founding a competing firm after ceasing his or her current employment until a prespecified amount of time has passed. Given the aforementioned tradeoff, it is not clear what the effect of NCAs is on aggregate productivity growth, nor is it clear how the answer to this question depends on structural parameters that may be different in different locations, industries or time periods. Further, from a normative perspective, it is natural to ask whether it is socially optimal to permit the free use of NCAs.

This paper is an attempt to provide quantitative answers to these questions. To this end, I first develop a tractable model of endogenous growth which augments a standard quality ladders model (e.g., as described in \cite{acemoglu_introduction_2009}) to include WSOs and NCAs. I then construct and analyze a micro dataset of incumbent firms and startups and find a statistically significant and economically meaningful relationship between parent firm R\&D and subsequent employee startup formation. The model is then calibrated using aggregate statistics and the microeconomic relationship between R\&D and employee entrepreneurship. Using the calibrated model, I study the effect of varying the barriers to enforcement of NCAs and describe the model-implied optimal policy. I find that eliminiation of all barriers to NCAs can increase welfare by approximately 3\% in consumption-equivalent terms. R\&D subsidies can have the counterintuitive effect of reducing growth by misallocating R\&D labor to entrants instead of incumbents or by inducing incumbents to use noncompetes when they otherwise would not have.\footnote{This stems in part from an assumption of a fixed stock of R\&D labor. In a fuller model, this mechanism would simply dull the growth-enhancing effect of R\&D subsidies. I plan to extend the model in that direction.} R\&D subsidies targeted at own-product innovation can work well in tandem with a ban on NCAs, but may be difficult to implement in practice.

The model consists of a standard general equilibrium model of endogenous growth with creative destruction augmented to include employee entrepreneurship and NCAs. It assumes that, via a learning-by-doing type assumption, R\&D employees eventually gain the knowledge to form a competing spinout. This reduces the incentive for R\&D spending by the employers which fear being replaced. In equilibrium, NCAs are used exactly when they maximize the joint value of employment. This bilateral optimization occurs through the wage, as the employer is able to hire the employee at a lower wage if he does not require an NCA. In that case, R\&D employees effectively pay ex-ante for the damage they will cause. When this ex-ante implicit payment exceeds the employer's expected loss of profits due to future spinout formation, 

In order for the model to generate a role for NCAs, then, some friction in the employment relationship needs to be present so that, absent NCAs, bilaterally inefficient outcomes will occur. Specifically, it cannot be possible to synthesize a noncompete ex-post by, for example, buying an employee spinout only to shut it down (i.e., preventing the bilaterally inefficient outcome).\footnote{On the empirical side, the assumption is justified by the fact that , \cite{babina_entrepreneurial_2019} find that only 2-5\% of employee spinouts are bought out by their former employers.}  If this did occur, then in anticipation the employee would accept a lower wage and, on net, the firm would conduct R\&D as though it had imposed an NCA. I leave the particular friction unmodeled, simply assuming that there is no market in which WSOs can be sold to the incumbent firm that generated them. In reality, such a friction could relate to asymmetric information concerning the quality of the idea, disagreements between the employee and the employer concerning the idea's quality, or a simple lack of commitment power on the part of the employee (i.e., the employee cannot commit not to implement the idea even after selling it to his employer). In addition, antitrust law could prevent this type of ex-post buyout.

The result is a model in which WSOs expand the innovation possibilities frontier of the economy while having an ambiguous effect on equilibrium innovation and productivity growth. This is the case even though R\&D requires inelastically supplied labor because the threat of WSOs can worsen the allocation of R\&D across uses. Specifically, WSOs reduce own-product innovation, shifting R\&D labor into creative destruction, where it typically has a lower social return on the margin due to the business-stealing externality.\footnote{In the model, this relies on the fact that creative destruction involves creating a new product and hence is a different "innovation race" than own-product innovation.} Depending on the strength of this mechanism relative to the growth impact of WSOs, the freedom to use NCAs can increase or decrease the equilibrium growth rate. In turn, this depends on the model parameters. 

To discipline the model, I use microeconomic data on R\&D and spinout formation as well as aggregate data on productivity growth and the macroeconomy. First, I assemble a new dataset of parent firms and startups founded by their employees by combining Compustat data on publicly traded firms and private Venture Source data on VC-funded startups and their founders. Venture Source is the only dataset on startups with broad coverage of information on the most recent employer of the startup's key employees. Still, matching these datasets is somewhat challenging as there are is no common company identifier so it must be done by name only. This is non-trivial since companies go by different names. I solve this problem by using string matching techniques (e.g., regular expressions), Compsutat data on firm subsidiaries, and the merchant-mapper tool by Alternative Data Group, a startup that links credit card transactions data to firms using machine learning (itself a spinout of 1010 Data).\footnote{This component is in progress.} I define a startup as a spinout if its CEO, CTO, President, Chairman or Founder (1) was most recently employed at a firm in Compustat and (2) joined the startup in its first three years. Using this definition, I identify approximately 3,000 WSOs in the data. Finally, I match this dataset to the NBER-USPTO database, which contains information on all US patents.

Figure \ref{figure:scatterPlot_RD-FoundersWSO4_dIntersection} illustrates heuristically the primary motivation for the analysis used in this paper: firm-level R\&D is associated with subsequent employee entrepreneurship in the same 4-digit NAICS industry. To be precise, the x-axis shows firm-level average R\&D spending over periods $t,t-1,t-2$ and the y-axis shows firm-level average yearly number of employee founders from that firm founding startups in $t+1,t+2,t+3$.  Both of these variables are purged of their firm and state-industry-age-year means. 

\begin{figure}[]
	\centering
	\includegraphics[scale= 0.5]{../empirics/figures/scatterPlot_RD-FoundersWSO4_dIntersection.pdf}
	\caption{Scatterplot of average yearly founder counts (restricted to same 4-digit NAICS industry) in $t+1,t+2,t+3$ versus average yearly R\&D spending in $t,t-1,t-2$.}
	\label{figure:scatterPlot_RD-FoundersWSO4_dIntersection}
\end{figure}

However, R\&D spending is of course an endogenous variable: firms perform R\&D as a profit maximizing decision based on firm, industry, and aggregate economic conditions. A simple correlation of number of employee spinouts on lagged R\&D spending can therefore suffer from omitted variable bias unless such factors are controlled for. To control for this, I use firm, state-year, NAICS 4 digit industry-year (at 4-digit NAICS level), and firm age fixed effects, as well as firm-specific controls, such as employment, assets, Tobin's Q, and citation-weighted patents. Firm fixed effects control for unobservable firm-level factors that are time-invariant; firm age fixed effects control for the effect of the typical firm life cycle; and state-year and industry-year fixed effects attempt to control time-varying factors, such as shocks to investment opportunities or overall industry or state conditions. The resulting estimates vary little by specification and are typically statistically and economically significant. According to these estimates, R\&D can account for roughly 75\% of employee departures to WSOs in the data. I also consider the robustness of the results to this moment (and all other moments).  

I next calibrate the model using the estimates above as well as aggregate statistics and growth accounting estimates from \cite{garcia-macia_how_2019} and \cite{klenow_innovative_2020}. I also choose some parameters from the literature. The calibrated model can then be used to study the effect on productivity growth and welfare of reducing the cost of using NCAs to zero. As stated previously, I find that welfare rises 3\% in consumption-equivalent terms. I discuss how these results depend on the parameters and, via the calibration, on the value of the targeted moments. I also exhibit an alternative calibration with a smaller share of employment in young firms which returns the opposite conclusion and discuss why this results.

Finally, I study alternative policies that could improve welfare in this context. I consider R\&D subsidies, both overall and targeted specifically at own innovation by incumbent firms; a tax on creative destruction; and finally the combination of all studied policies. Two interesting findings emerge. First, untargeted R\&D subsidies (that is, that apply to creative destruction as well as own-product innovation) can have the unintended effect of shifting R\&D to entrants, and potentially even inducing the use of NCAs. The latter occurs because incumbents prefer to pay R\&D employees through subsidized wages rather than implicitly through future spinout formation (whose cost to the incumbent is not subsidized). The former is due to a similar reason. Second, targeted R\&D subsidies avoid this problem and, in combination with a ban on the use of NCAs, can achieve a first-best where the incumbent does enough R\&D while still allowing for spinouts to enter.\footnote{Not technically a first best, but as close as this model can come. I discuss this in detail in the body of the paper.} This might be a difficult policy to implement, and I discuss some of the potential barriers. I close with suggestions for future work. 

\paragraph{Related literature}

Some work has attempted to answer this question directly using empirical methods. Papers in this literature have typically used either cross-sectional and/or longitudinal variation in the state-level enforcement of non-competes.\footnote{Sometimes this variation is argued to be exogenous, either due to legislative error as in \cite{marx_mobility_2009} and \cite{marx_regional_2015}, or due to unexpected judicial precedent as in \cite{jeffers_impact_2018}. Often there is a control industry that is believed to be unaffected by the variation in CNC enforcement policy (e.g. CNCs typically are more difficult or impossible to enforce in the legal industry).} 

The results point to an important tradeoff between innovation by spinouts and investment by incumbent firms.  For example, \cite{stuart_liquidity_2003} find more local  entrepreneurship in response to local IPO (a "liquidity event") in regions not enforcing CNCs. \cite{marx_mobility_2009} finds that inventor mobility declines in response to an exogenous increase in non-compete enforcement. \cite{samila_venture_2010} finds that an increase in VC funding supply increases entrepreneurship more in states without non-compete restrictions, using an IV design. \cite{garmaise_ties_2011} finds that, in states where CNCs are more enforceable, managers are less mobile, have lower compensation, and invest less in their human capital, to the point of offsetting increased investments by the firm. 

On the other hand, \cite{conti_non-competition_2014} finds evidence that non-compete enforceability leads to incumbent firms pursuing riskier R\&D projects. \cite{colombo_does_2013} finds evidence that easier spinout formation -- proxied by access to finance -- leads to a reduction in incumbent firm knowledge investments.  Most recently, \cite{jeffers_impact_2018} uses data on influential state-level court precedents matched with LinkedIn data and finds that enforcement indeed reduces spinout formation while increasing capital investment by incumbent firms. Finally, \cite{marx_regional_2015} finds that CNC enforcement leads to inventor mobility out of the state, suggesting that cross-sectional differences in outcomes could be in part due to reallocation across states of the human capital inputs to entrepreneurship and innovation. 

The most directly related empirical paper is \cite{babina_entrepreneurial_2019}. They find evidence of a causal relationship from corporate R\&D spending to employee spinout formation.\footnote{To my knowledge, the papers are contemporaneous} My empirical analysis confirms their findings on a subset of firms particularly connected with productivity growth, VC-funded startups.  Together, they motivate the use of a model like the one developed in this paper.

This project also builds on a body of theoretical work that has considered this question in different frameworks. \cite{franco_spin-outs_2006} develops a model in which employees learn from their employers and use this knowledge to form spinouts. They emphasize the "paying for knowledge" effect, whereby employees implicitly pay for the knowledge they take from the parent firm through lower equilibrium wages. Importantly, they assume spinout firms do not steal business directly from their parents. The only effect of a spinout on the parent firm is a reduction in the price of the output good, which the parent firm is assumed not to take into account. Moreover, there is one good and it is produced competitively (individually decreasing returns to scale ensures that one producer does not dominate). This, combined with the "paying for knowledge" mechanism, ensures that the equilibrium allocation is Pareto efficient, even without resorting to elaborate labor contracts. In my case, by contrast, the equilibrium is generally not efficient due to creative destruction by entrants and spinouts as well as positive externalities of innovation.\footnote{This includes consumer surplus effects due to market power as well as innovation externalities in the vein of "standing on the shoulders of giants", which are standard in endogenous growth models}

\cite{franco_covenants_2008} studies a two-period, two-region model with employee spinouts in which the region which does not enforce CNCs initially lags but eventually overtakes the region in which CNCs are enforced. In the first period, entry is more valuable in the enforcing region. But in the second period, spinouts enter in the non-enforcing region, there is Cournot competition with parent firms in the product market, and output increases relative to the enforcing region. The analysis emphasizes how asymmetric information about whether an employee has learned leads some firms in the non-enforcing region to allow spinouts (assuming firms cannot commit to wage backloading). This can be taken as a rough microfoundation of my assumption that labor contracts are "simple" in  a non-enforcing region: just a wage, with no attempts at retention in the case of learning. Relative to this study, my analysis considers a fully dynamic model rather than two-period model, an important consideration as today's spinouts are tomorrow's incumbents. In addition, I emphasize the role of R\&D investment in spawning spinout firms and the potential additional disincentives to innovation that this can entail. My concession is that I study a model with only one enforcement regime.\footnote{I plan to pursue this idea in future work.} 

\cite{shi_restrictions_2018} uses a rich model of contracting disciplined by data on executive non-compete contracts to study the effect of non-competes on executive mobility and firm investment. She finds that the optimal policy is to somewhat restrict the permitted duration of CNCs. Her approach allows her to study the optimal contracting problem in more detail than in mine. However, she is mainly interested in an environment where the firm's productivity is embodied in the worker and where the concern is poaching, not spinout formation. Also, her calibration considers firm investment in capital expenditures, whereas I am interested in innovative investment in R\&D. Finally she performs her analysis in a partial equilibrium framework while I consider a fully specified general equilibrium model.

\cite{baslandze_spinout_2019}, the study closest to this paper, studies the effect of spinout entrepreneurship on entry and growth. She also uses a GE model of endogenous growth with employee spinouts, using Compustat and NBER-USPTO patent data to discipline the analysis. She finds the optimal policy is to ban NCAs. However she is focused in her framework on the harm from losing a valuable employee rather than the harm from competition with the parent firm. My paper focuses instead on creative destruction of the parent firm using knowledge rather than the loss of productivity from losing valuable employees.  The other key difference is that I model the use of NCAs while her analysis assumes that they are used when available. To my knowledge, mine is the first general equilibrium model of endogenous growth to have this kind of feature.

\section{Model}\label{sec:model}

\subsection{Individual endowments and preferences}

The model is in continuous time, starting at $t = 0$. The representative household has CRRA preferences over consumption, given by\footnote{There is no expectation operator use there is no aggregate uncertainty in this setting (more on this in later sections).}
\begin{align}
U_t &= \int_0^{\infty} e^{-\rho s} \frac{C(t+s)^{1-\theta} - 1}{1-\theta} ds \label{preferences}
\end{align}

In each period $t \ge 0$, the household is endowed with $\bar{L}_{RD} \in (0,1)$ units of R\&D labor as well as $1 - \bar{L}_{RD}$ units of production labor which is used in the production of intermediate and final goods. The labor resource constraints are 
\begin{align}
L_{RD} &\le \bar{L}_{RD} \label{labor_resource_constraint2} \\
L_{\text{production}} &\le 1 - \bar{L}_{RD} \label{labor_resource_constraint} 
\end{align}

Note that this means that the economy produces a fixed and exogenous total amount of R\&D. Growth in the model is therefore determined not by the amount of R\&D but by its allocation between different uses (see Section \ref{subsec:innovation}).\footnote{This biases my model in favor of NCAs, since any reduction in R\&D by incumbents due to the threat of spinout competition lower the R\&D wage and in general equilibrium are offset by more R\&D spending by entering firms. However, the net effect on growth will still be negative provided that R\&D labor is overallocated to creative destruction, which it is in my calibrated model. See Section \ref{model:efficiency:efficiency} for a theoretical discussion}.

\subsection{Production of intermediate and final goods} \label{subsec:staticproduction}

\paragraph{Intermediate goods} There is a continuum of intermediate goods $j\in [0,1]$ which at any given time $t$ exist in a finite set of qualities $\{q_{jti}\}_{0 \le i \le I_{jt}}$. Define $\bar{q}_{jt} = \max_{0 \le i \le I_{jt}} \{q_{jti}\}$ as the \emph{frontier} quality of good $j$, and refer to the producer of this good as \emph{incumbent} $j$. Intermediate goods $j$ of any quality are produced according to the production function
\begin{align}
k_{jti} = H(\ell_{jti};Q) &= Q \ell_{jti} \label{intermediate_goods_production}
\end{align}
where $\ell_{jti} \ge 0$ is the labor input and $Q_t = \int_0^1 \bar{q}_{jt} dj$ is the average frontier quality level in the economy.\footnote{The linear scaling with the aggregate economy $Q$ is to ensure a balanced growth path (BGP), given that the total quantity of labor stays the same over time. It is analogous to assuming a constant marginal cost in a model where the final good, rather than labor, is the input of intermediate goods production. \textbf{[Note about alternative formulation with no scaling in $Q_t$ here -- see Acemoglu textbook for equivalence results.]}} Each quality of good $j$ is produced by a firm which has a monopoly on the production of that quality of good $j$. There is no storage of intermediate goods.

\paragraph{Final good}

The final good $Y$ is produced competitively using production labor and intermediate goods. Its production technology is given by \footnote{Intermediate goods are aggregated in a CES form with an elasticity of substitution greater than 1, rather than the Cobb-Douglas form in e.g., \cite{grossman_quality_1991} and \cite{baslandze_spinout_2019}. This reduces the complexity of the firm problem. In those models, Cobb-Douglas implies that expenditure on each intermediate good is constant in quality. This requires limit pricing to be explicitly modeled, otherwise increasing the price always increases profits and the firm problem is not well-defined. To model limit pricing, one must track the gap between leader and follow in each good $j$, adding a state variable to the firm problem and to the aggregation of the model. In the current setup, by contrast, expenditure is decreasing in the price of the intermediate good, so even if one can abstract from limit pricing (as is the case in this model, see next footnote), intermediate goods firms have a constant optimal markup. Reducing the complexity in this way allows me to add more features to the model while maintaining a transparent analysis.}
\begin{align}
Y_t = F(L_{Ft},\{q_{jti}\},\{k_{jti}\}) &= \frac{L_{Ft}^{\beta}}{1-\beta} \int_0^1 \Big(\sum_{i = 0}^{I_{jt}} q_{jti}^{\frac{\beta}{1-\beta}} k_{jti} \Big)^{1-\beta} dj \label{final_goods_production}
\end{align}

where $k_{jti} \ge 0$ is the quantity used of intermediate good $j$ of quality $q_{jti}$. This specification assumes that different qualities of good $j$ are perfect substitutes in final goods production, and the intermediate goods production technology (\ref{intermediate_goods_production}) has constant returns to scale. Bertrand competition in the intermediate goods market then implies that in equilibrium only the highest quality producer of good $j$ will produce a positive amount.\footnote{Usually, this implies that the leader will use limit pricing unless each quality improvement is sufficiently large. To avoid this complication, I use an assumption that ensures monopolistic competition pricing regardless of the step size, borrowed from \cite{akcigit_growth_2018}.} This allows for a simpler representation,
\begin{align}
Y = F(L_F,\{\bar{q}_j\},\{\bar{k}_j\}) &= \frac{L_F^{\beta}}{1-\beta} \int_0^1 \bar{q}_j^{\beta} \bar{k}_j^{1-\beta} dj  \label{eq_final_goods_production}
\end{align}

Finally, there is no storage technology for the final good and its price is normalized to 1 in every period. 

\subsection{Innovation}\label{subsec:innovation}

There are three types of innovation in this economy. Incumbents can expend R\&D to improve on their own product. I refer to this as \textit{own-product innovation} or OI, following \cite{garcia-macia_how_2019} and \cite{klenow_innovative_2020}. The other two types of innovation involve \textit{creative destruction}, or CD. R\&D by incumbents can lead to the formation of employee spinouts, which in the model are simply new higher-quality incumbents producing the same good. In addition, entrants can perform targeted R\&D on each intermediate good. 

An innovator on the frontier quality of good $j$ becomes the new incumbent of good $j$ and holds a perpetual patent on the production of good $j$ of quality $\bar{q}_{jt} = \lambda \lim_{t' \uparrow t} \bar{q}_{jt'}$, where $\lambda > 1$ is the exogenous quality ladder step size. That is, the new frontier quality improves on the previous frontier quality by a factor $\lambda$. Importantly, the perpetual patent \textit{does not} prevent entrants or WSOs from leapfrogging the incumbent with an even better quality product.

\paragraph{No catch-up innovation}

I assume that only incumbents can perform OI. When an incumbent is overtaken by an entrant or spinout, she loses access to the OI R\&D technology and therefore cannot use it to "catch up" to the frontier. One possible interpretation is that learning by doing means the current producer of a product has unique insights into how to improve on it. From a mathematical standpoint, the assumption dramatically simplifies the analysis.\footnote{This is standard in quality ladders models with OI and is usually not made explicit. Without it, the incumbent problem would have an additional state variable (since falling away from the frontier is no longer an absorbing state) and an additional distribution would need to be tracked (the number of incumbents with the technology to produce each infra-frontier good $j$. The setting is so intractable that many papers which focus on this mechanism, such as \cite{aghion_competition_2005}, also make simplifying assumptions analogous to mine. For a producer $n$ steps behind the frontier, the assumption has bite if the expected discounted present value of the cost of $n + 1$ innovations using the OI innovation technology is lower than the expected cost of one innovation using the freely available entrant technology (described in Section \ref{subsubsec:entrants}). For certain parameter values, this inequality will hold for small $n$.}  

\paragraph{Scaling of cost of R\&D}

A given amount of R\&D applied to improving a product leads to a given Poisson intensity of innovating on a product, regardless of the quality of the product. However, when directed at a higher quality product, a given amount of R\&D requires more R\&D labor input. Specifically, one unit of R\&D costs $\frac{\bar{q}_{jt}}{Q_t}$ units of R\&D labor when directed at a product with frontier quality $\bar{q}_{jt}$. This scaling assumption is economically natural, as higher quality products require more human capital to improve. It is also necessary for the existence of a balanced growth path in this model.\footnote{The cost of R\&D could scale up faster than $\frac{\bar{q}_{jt}}{Q_t}$, which would imply that higher quality products grow slower. In the current setup this would violate BGP since there is no stationary distribution of product quality. Adding a fixed cost, however, would induce such a stationary distribution, because it creates a lower exit barrier. For more discussion of this type of question, see \cite{gabaix_power_2009} or \cite{acemoglu_innovation_2015}.}

\subsubsection{Own-product innovation by incumbents} \label{subsubsec:OI}

In return for $z_{jt}$ units of R\&D, the incumbent receives a Poisson intensity of $\chi z_{jt}$ of innovating on good $j$, where $\chi > 0$ is an exogenous parameter representing the incumbent's R\&D productivity. This implies an arrival rate of incumbent innovations of 
\begin{align}
	\tau_{jt} &= \chi z_{jt}
\end{align}

Note that incumbent R\&D exhibits constant returns to scale. This is necessary to have analytical solutions, but the model can be easily extended to the more realistic case of decreasing returns and solved numerically.


\subsubsection{Creative destruction by entrants} \label{subsubsec:entrants}

For each good $j$ there is a unit mass (normalization) of entrants indexed by $e \in [0,1]$.\footnote{For expositional simplicity I assume this, but it is actually an easily provable equilibrium result.} In return for $\hat{z}_{jet}$ units of R\&D, an entrant receives a Poisson intensity of $\hat{z}_{jet} \hat{\chi} \hat{z}_{jt}^{-\psi}$ of innovating on good $j$, where $\hat{z}_j = \int_0^1 \hat{z}_{je} de$ denotes aggregate entrant effort on improving good $j$. Aggregating over $e$, this implies an arrival rate of entrant innovations to good $j$ of 
\begin{align}\label{model:entrantsInnovationTechnology}
\hat{\tau}_{jt} &= \hat{\chi} \hat{z}_{jt}^{1-\psi}
\end{align}

The choice $\psi > 0$ introduces decreasing returns at the level of good $j$. It represents a \textit{congestion} externality in the entrant innovation technology. This is similar to the congestion externality present in search and matching models. Intuitively, due to a lack coordination, entrants attempt similar approaches to solve the same problem. This duplication of effort reduces the overall returns to entrant R\&D when considered at the level of good $j$.

\subsubsection{Creative destruction by employee spinouts}\label{subsubsec:generation_of_spinouts}

R\&D by incumbents leads to knowledge spillovers in the form of employee spinouts from R\&D labor. Incumbents can prevent spinouts by imposing a noncompete agreement on their R\&D labor, a decision which can be made instant-by-instant. The representative household takes this into account whether a noncompete is imposed when allocating its R\&D labor (also instant-by-instant, as it is a perfectly competitive market) and therefore demands a higher wage when a noncompete is imposed. I will return to this in the next section.

To make this precise, when incumbent $j$ conducts $z_{jt}$ units of R\&D effort, she faces a Poisson intensity of spawning a WSO, given by 
\begin{align}
	\tau^S_{jt} &= (1-\mathbbm{1}^{NCA}_{jt}) \nu z_{jt} \label{def:tau_S}
\end{align} 
where $\mathbbm{1}^{NCA}_{jt} = 1$ if and only if an NCA is used in that instant. 

Spinouts from incumbents of quality $q$ have the ability to produce the same good with quality $\lambda q$. They immediately become the new incumbent and, recalling the "no catch-up" assumption in Section \ref{subsec:innovation}, the previous incumbent's profits go to zero forever after. The exogenous parameter $\nu \ge 0$ is a reduced form encoding the rate at which corporate R\&D increases the likelihood of replacement by a WSO.

In the context of this model, this specification amounts to assuming that the rate of spinout generation of a unit of R\&D labor is inversely proportional to the relative quality of the good to which it is applied, $\frac{\bar{q}_{jt}}{Q_t}$. Because R\&D labor demand is $\frac{\bar{q}_{jt}}{Q_t} z_{jt}$, the factors cancel out and the rate of spinout formation is linear in $z_{jt}$. In this way, this is the specification of spinout generation that is analogous to the specification of the cost of R\&D.

\paragraph{No idea stealing} WSO entry does not directly reduce the rate at which incumbent R\&D results in successful OI. Instead, WSOs happen when additional, independent Poisson process with rate $\nu z_j$ has an arrival. The interpretation is that WSOs in this model do not steal ideas that otherwise would have been implemented by the parent firm. Rather, when unbound by NCAs, R\&D labor generates \textit{additional} innovations which the household uses to displace the incumbent firm.\footnote{This assumption has important consequences for the private and social usefulness of NCAs. If spinouts steal ideas, they are less bilaterally valuable and NCAs are more useful privately. For the same reason, WSOs are less socially valuable and therefore so are NCAs. This is an interesting topic for further research.}

\paragraph{Direct cost of using NCAs}\label{paragraph:nca_cost}

When incumbent $j$ imposes an NCA on $z_j$ units of R\&D, she must pay a flow cost $\kappa_{c} \nu V(j,t|\bar{q}_j) z_j$ units of the final good, where $V(j,t|q)$ is the value of incumbent $j$ at time $t$ given quality $q$. Given (\ref{def:tau_S}), incumbent $j$ overall pays
\begin{align}
	\textrm{NCA cost}_{jt} &= \tau^S_{jt} \kappa_c V(j,t|q) \label{def:nca_cost}
\end{align}

The NCA enforcement cost reflects the direct cost using an NCA. Even if there are no technical restrictions on what kinds of NCAs are valid, determining competition between businesses may be expensive. Moreover, many jurisdictions do, in fact, impose such restrictions, and resources can be invested to prove that the conditions of those restrictions do not apply. Overall, it seems plausible that investing resources increases the likelihood of a successful enforcement of an NCA.  Note also that a value of $\kappa_c = \infty$ can be interpreted as ban on the use of NCAs.

The factor $V(j,t|q)$ means that the cost of enforcing NCAs increases in the equilibrium value of the incumbent firm. The economic justification is that valuable incumbency positions require more resources to protect via NCAs. In the context of the model, this specification means that the cost of enforcing an NCA on a given unit of R\&D labor is proportional to both the value of the WSOs that labor will generate in the absence of an NCA, and the expected loss of incumbent value from an absence of NCAs. In addition to economics, the assumption improves model tractability by simplifying the analysis of the optimal noncompete policy (see Section \ref{subsubsec:dynamic_equilibrium_original_solution}).\footnote{BGP only requires that\begin{align*}
	\textrm{NCA cost}_{j,t,q} &\propto \tau^S_{jt} q
	\end{align*}}

\subsubsection{Entry cost}

In addition to the R\&D costs of innovation, entrants and spinouts must pay an entry cost of $\kappa_{e} V(j,t|\lambda q)$ units of the final good in order to enter once they have successfully innovated, for exogenous $\kappa_e \in [0,1)$. This reduced form represents non-R\&D expenditures required by CD innovation but not by OI innovation. Examples of such expenditures could be firm set-up or marketing costs for a new product or brand.

The scaling with $V(j,t|\lambda q)$ parallels the scaling of the NCA cost in Section \ref{paragraph:nca_cost}. This is economically reasonable given the interpretation above. It is also important for matching the data, in particular the R\&D / GDP ratio and the entry rate.\footnote{This relates to \cite{comin_rd_2004}, which finds in a similar model that the model-implied expected payoff to aggregate R\&D is much higher than aggregate R\&D spending in the data. This is inconsistent with free entry. His interpretation is that R\&D is in fact responsible for only a small part of innovation, which reduces the model-implied payoff of aggregate R\&D spending. Instead, my interpretation, via this model, is that there are additional non-R\&D expenditures associated with innovation. The calibration sets these expenditures so that the return to innovative investment is equal to the market rate.} In addition, as before, scaling with $V(j,t|\lambda q)$ enables a simple analytical solution to the model, for the same reason as the NCA cost assumption. It could easily be extended in the same way.\footnote{As in Section \ref{paragraph:nca_cost}, BGP only requires that 
\begin{align*}
	\textrm{Entry cost}_{j,t,q} &\propto q
\end{align*}}

\subsubsection{Competitive financial intermediary}

The representative household owns a competitive financial intermediary which in turn owns all firms in the economy and remits their profits back to the representative household. Individual firms in the economy maximize profits subject to the household's risk-free discount rate (there is no collusion due to common ownership). When the representative household receives a shock in good $j$ that allows it to form a spinout, it sells the spinout to the financial intermediary at full private value (i.e. discounting the spinouts profits at the same risk-free discount rate). The financial intermediary takes the entry of the spinout as given, and therefore is willing to pay this value even though the entry of the spinout reduces the value of an existing incumbent.\footnote{The purpose of this construction is to avoid having to assume that the representative household does not take into account the loss of value of the incumbents it owns when spinouts enter.}\footnote{An alternative setup instead of a representative household is a continuum of households each consisting of a continuum of agents who fully insure one other against idiosyncratic risk. What is essential is that households be fully insured (to avoid having to deal with modeling financial frictions) and that they not internalize the business-stealing effects of the spinouts.} 

\subsection{Equilibrium}\label{subsec:decentralized_equilibrium}

I will solve for a BGP equlibrium with a constant growth rate of output ($g_t = g$) as well as constant innovation effort by incumbents ($z_{jt} = z$) and entrants ($\hat{z}_{jet} = \hat{z})$. I will refer to this class of BGP as a \textit{symmetric BGP}.\footnote{In principle I allow $\mathbbm{1}^{NCA}_{jt}$ to vary across $j$ and over time $t$; Propositions \ref{proposition:purstrategyeq:positiveOI} and \ref{proposition:purstrategyeq:zeroOI} below show that $\mathbbm{1}^{NCA}_{jt} = x$ in a symmetric BGP except on a knife-edge in the parameter space.}\footnote{One could relax the assumption that $z_{jt} = z$ as long as $\int_0^1 z_{jt} \frac{\bar{q}_{jt}}{Q_t}dj$ is constant on the BGP. This induces a continuum of BGPs which have the same aggregate variables (except higher moments of the quality distribution, which are irrelevant for the equilibrium), since this term appears in the growth accounting equation analogous to (\ref{eq:growth_accounting}) and the R\&D labor market clearing condition (\ref{eq:RD_labor_market_clearing}). As such, this kind of multiplicity does not affect aggregate growth or prices and is a technical artefact of the assumed CRS R\&D technology for incumbents. I therefore assume that $z_{jt} = z$ because it simplifies the algebra.} Symmetric BGPs are a natural type of equilibrium given the symmetric setup of the model. In this case, there always exists a unique symmetric BGP, except on a knife-edge in which case there is a continuum of symmetric BGPs.

To do so, first I characterize the static equilibrium given a profile of frontier qualities $\{ \bar{q}_{j}\}$. Next, using the assumption that $\hat{z}_{jet} = \hat{z}, z_{jt} = z$, I prove that there exists $\tilde{V} > 0$ such that $V(j,t|\bar{q}_{jt}) = \tilde{V} \bar{q}_{jt}$ is the value of incumbent $j$ of quality $\bar{q}_{jt}$ at time $t$. This in turn implies that R\&D wages scale linearly with aggregate productivity $Q_t$ as well. Given the linear scaling with $\bar{q}_{jt}$ and $Q_t$, respectively, factors $\bar{q}_{jt}$ and $Q_t$ drop out of the equilibrium equations. The resulting system is straightforward to solve recursively.

\subsubsection{Static equilibrium}

In this section, I omit the dependence on $t$ of all equilibrium variables. In addition, since only the frontier quality is produced in equilibrium, I will drop the bar notation and refer to the frontier good's quality and quantity by $q_j$ and $k_j$, respectively.

Final goods producer optimization implies the following inverse demand function for intermediate goods, 
\begin{align*}
p_j &= L_F^{\beta} q_j^{\beta} k_j^{-\beta}	
\end{align*}

To continue computing the equilibrium of the model, the market structure for intermediate goods must be specified. 

\paragraph{Intermediate goods market structure} The following setup is drawn from \cite{akcigit_growth_2018}. Within each good $j$, intermediate goods producers play a two-stage Bertrand competition game. In the first stage, participants bear a cost of $\varepsilon > 0$ units of the final good in exchange for a right to compete in the second stage. Then, in the second stage, the engage in Bertrand competition. Optimal pricing under Bertrand competition in the second stage implies that all producers not on the frontier will earn zero profits. By backward induction, they do not pay the entry cost in equilibrium, and the leader has a second-stage monopoly over good $j$.\footnote{Without this assumption, there is limit pricing, and the markup charged by the technology leader in good $j$ would depend on his gap relative to the next laggard, e.g. \cite{baslandze_spinout_2019} or \cite{aghion_competition_2005}, only equating to the monopolistic competition markup for large enough gaps.} Different good $j$ incumbents compete against each other under monopolistic competition.

This market structure implies that the incumbent for each good $j$ can effectively ignore lower quality producers of good $j$. She maximizes profits according to
\begin{align}
\pi(q_j) = \max_{k_j \ge 0} \Big\{ L_F^{\beta} q_j^{\beta} k_j^{1-\beta} - \frac{\overline{w}}{Q} k_j \Big\} \label{incumbent_profit}
\end{align}

where $\overline{w}$ is the equilibrium final goods / intermediate goods wage.
This yields optimal pricing, labor demand and production of intermediate goods,
\begin{align}
k_j &= \Big[ \frac{(1-\beta) Q}{\overline{w}} \Big]^{1/\beta}L_F q_j  \label{optimal_k}\\
\ell_j &= k_j / Q \label{optimal_l}\\
p_j &= \frac{\overline{w}}{(1-\beta) Q} \label{optimal_p}
\end{align}

Substituting (\ref{optimal_k}) into the first-order condition for final goods firm optimal labor demand yields a closed form expression for the equilibrium wage $\overline{w}$:
\begin{align}
\overline{w} &= \tilde{\beta} Q \label{wbar} \\
\tilde{\beta} &= \beta^{\beta} (1-\beta)^{1-2\beta} \label{def_cbeta}
\end{align}

Substituting (\ref{optimal_k}) and (\ref{wbar}) into the expression for profit in (\ref{incumbent_profit}) yields
\begin{align}
\pi_j &= \overbrace{(1-\beta) \tilde{\beta} L_F}^{\mathclap{\tilde{\pi}}} q_j \label{profits_eq}
\end{align}

Substituting (\ref{optimal_k}) into (\ref{optimal_l}) and integrating $L_I = \int_0^1 l_j dj$ yields aggregate labor allocated to intermediate goods production,
\begin{align}
L_I &= \Big( \frac{1-\beta}{\tilde{\beta}} \Big)^{1 / \beta} L_F \label{intermediate_goods_labor}
\end{align}

and substituting (\ref{intermediate_goods_labor}) into the labor resource constraint (\ref{labor_resource_constraint}) yields
\begin{align}
L_F &= \frac{1 - \bar{L}_{RD}}{1 + \Big(\frac{1-\beta}{\tilde{\beta}}\Big)^{1/\beta}}
\end{align}

Output can be computed by substituting (\ref{optimal_k}) into (\ref{final_goods_production}), 
\begin{align}
Y = \frac{(1-\beta)^{1-2\beta}}{\beta^{1-\beta}} Q L_F \label{flow_output}
\end{align}

\subsubsection{Dynamic equilibrium}\label{subsubsec:dynamic_equilibrium_original_solution}

\paragraph{Household optimization}

The household takes as given for all $t \ge 0$ incumbent R\&D wages and NCA policies $\{w_{RD,jt}, \mathbbm{1}^{NCA}_{jt}\}$, entrant R\&D wages $\{\hat{w}_{RD,t}\}$, the production wage $\{\bar{w}_t\}$, interest rates $\{r_t\}$, and profits from the financial intermediary $\{\Pi_t\}$. 

The household problem is\footnote{There are also non-negativity constraints on all control variables $\{C(t), \ell_{RD,j} (t), \hat{\ell}_{RD,j}(t), L(t) \}_{t\ge 0}$, ommitted for clarity.}\footnote{The problem is formulated as a stochastic optimal control problem because it involves allocating R\&D labor to different goods $j$ depending on the random processes $\{\bar{q}_{jt}\}_{t \ge 0}, \{w_{RD,jt}\}_{t\ge 0}, \{\mathbbm{1}^{NCA}_{jt}\}_{t\ge 0}, \{V(j,t|\bar{q}_{jt})\}_{t\ge 0}$ for $j \in [0,1]$, However, in a symmetric BGP, there will be no uncertainty in the household's consumption stream because equilibrium compensation (including the value of WSOs generated) will be the same in all goods $j$ to satisfy market clearing. See Lemma \ref{lemma:RD_worker_indifference}.}

\begin{maxi*}[1]<b>
	{\substack{\{C(t) \}_{t \ge 0} \\ \{ L(t)  \}_{t \ge 0} \\ \{\ell_{RD,j}(t)\}_{j \in [0,1], t \ge 0} \\ \{\hat{\ell}_{RD,j}(t)\}_{j \in [0,1], t \ge 0}  }} {\mathbb{E} \int_0^{\infty} e^{-\rho t} \frac{C(t)^{1-\theta}-1}{1-\theta} dt}{}{}
	\addConstraint{ C(t)}{ \le \Pi_t + \bar{w}_tL(t)} 
	\addConstraint{ }{+ \int_0^1 \big( w_{RD,jt} + (1-\mathbbm{1}^{NCA}_{jt})(\frac{\bar{q}_{jt}}{Q_t})^{-1} \nu (1-\kappa_e) V(j,t|\lambda \bar{q}_{jt}) \big) \ell_{RD,j}(t) dj}
	\addConstraint{ }{+ \int_0^1 \hat{w}_{RD,t} \hat{\ell}_{RD,j}(t) dj}
	\addConstraint{\int_0^1 (\ell_{RD,j}(t) + \hat{\ell}_{RD,j}(t))dj}{ \le \bar{L}_{RD}} 
	\addConstraint{L(t)}{\le 1 - \bar{L}_{RD}}
\end{maxi*}

\normalsize


where $L(t) = L_I(t) + L_F(t)$ denotes production labor, $\ell_{RD,j}(t)$ denotes R\&D labor supplied to incumbent $j$, and $\hat{\ell}_{RD,j}(t)$ denotes R\&D labor supplied to entrants attempting innovation on good $j$.\footnote{Because the household's effective "taste" for working at a given incumbent $j$ depends on the expected DPV of WSOs formed after working there, it is necessary to explicitly model the household's R\&D labor allocation across goods $j$. This is not necessary for R\&D labor supplied to entrants, since they are all identical. However, I present the problem symmetrically.} The household consumes out of profits remitted by the intermediary $\Pi_t$ (which it takes as given), wages earned from production labor supply $\bar{w}_t L(t)$, wages earned from R\&D labor supply $\int_0^1 \big(w_{RD,jt} \ell_{RD,j}(t) + \hat{w}_{RD.t} \hat{\ell}_{RD,j}(t) \big) dj$, and earnings from sales of WSOs to the financial intermediary $ \int_0^1 (1-\mathbbm{1}^{NCA}_{jt}) (\frac{\bar{q}_{jt}}{Q_t})^{-1} \nu (1-\kappa_e) V(j,t|\lambda \bar{q}_{jt}) \big)\ell_{RD,j}(t) dj$.\footnote{To be more rigorous one would define a Poisson process for each $j$ denoted $N_{jt}$ which keeps track of the cumulative number of spinouts in line $j$. Then the budget constraint would have $\int_0^1 w_{RD,jt}\ell_{RD,j}(t) dj$ and a separate term $\int_0^1 (1-\kappa_e) V(j,t|\lambda \bar{q}_{jt}) dN_{jt} dj$, where arrival rate of $N_{jt}$ is $(\frac{\bar{q}_{jt}}{Q_t})^{-1} \nu \ell_{RD,j}(t)$. By the law of large numbers, this would yield the same expression as given above.} The last term is expected value of spinouts formed in good $j$, which occurs with intensity $(1-\mathbbm{1}^{NCA}_{jt}) (\frac{\bar{q}_{jt}}{Q_t})^{-1} \nu \ell_{RD,j}(t)$ and has value $(1-\kappa_e) V(j,t|\lambda \bar{q}_{jt})$, given that the entry cost $\kappa_e V(j,t|\lambda \bar{q}_{jt})$ must be paid to enter.\footnote{Technically this assumes that $\kappa_e < 1$. When I discuss entry taxes in Section \ref{subsec:cd_tax} and $\kappa_e + T_e \ge 1$, the value will be $\max\{0,(1-\kappa_e - T_e) \lambda \tilde{V} \}$.} 

Household optimization yields two types equilibrium conditions: an indifference condition on R\&D wages and an Euler equation. The former amounts to the condition that, if the worker supplies R\&D labor to all incumbents at all times, the expected compensation received from each incumbent is equal to the wage earned when supplying R\&D labor to entrants. The former is equal to the wage paid by the incumbent plus, if $\mathbbm{1}^{NCA} = 0$ (NCA not used), the expected future value of spinouts formed by the employee as a result of their employment at the incumbent. 


\begin{lemma}\label{lemma:RD_worker_indifference}
	In a symmetric BGP with $z > 0$, R\&D wages satisfy
	\begin{align}
	\hat{w}_{RD,t} &= w_{RD,jt} + (1-\mathbbm{1}^{NCA}_{jt}) (\frac{\bar{q}_{jt}}{Q_t})^{-1} \nu (1-\kappa_e) V(j,t|\lambda \bar{q}_{jt}) \label{eq:RD_worker_indifference}
	\end{align}
	for all $t \ge 0$ and $j \in [0,1]$.
\end{lemma}

\begin{proof}
	\textbf{[Probably just say it's proved in text.]}
	First note that $\hat{z} > 0$ in any symmetric BGP due to the Inada conditions on the entrant innovation technology given in  (\ref{model:entrantsInnovationTechnology}). If in addition $z > 0$, then $\ell_{RD,j}(t) > 0, \hat{\ell}_{RD,j}(t) > 0$ for all $j,t$. Optimality dictates that the household supplies R\&D labor only to jobs which provide the highest compensation. Therefore, in order to be consistent with household optimal labor supply, (\ref{eq:RD_worker_indifference}) must hold for all $t \ge 0$ and $j \in [0,1]$, 
\end{proof}

To derive the second condition -- the Euler equation --  technically one needs to add a market for a instantaneous risk-free bond, which in equilibrium is in zero net supply. I have not done this explicitly to avoid complicating the household's stated problem. Denote the interest rate for this bond by $r_t$, which in principle can be time-varying. The resulting household problem is standard\footnote{Because the household takes as given its ownership of the financial intermediary and this is the only asset that exists in positive net supply in the economy, there is no transversality condition associated with household optimization. However, this choice of modeling device does not actually expand the set of equilibria because any equilibrium which would have violated the household's transversality condition in the standard setup will violate finiteness of household utility in this set up.}  and gives rise to the Euler equation at each time $t \ge 0$, 
\begin{align}
\frac{\dot{C}(t)}{C(t)} = \frac{1}{\theta} (r_t - \rho) \label{eq:euler0} 
\end{align}



\paragraph{Incumbent optimization}

For all $t \ge 0$, incumbent $j$ takes as given the R\&D wage it must pay conditional on each choice of NCA policy, $w_{RD,jt}(\mathbbm{1}^{NCA})$, as well as the interest rate $r_t$ which it uses to discount future profits and the rate of creative destruction by entrants $\hat{\tau}(j,t|\bar{q}_{jt})$. I have introduced the conditional incumbent R\&D wages $w_{RD,jt}(\mathbbm{1}^{NCA})$ because at this point it is necessary to specify what wage the incumbent would have to pay given an off-equilibrium choice of $\mathbbm{1}^{NCA}_{jt}$, so that the equilibrium choice can be shown to be the optimal policy. To be consistent with the microfoundations of the perfect competition model, I assume that the incumbent expects to be able to hire as much R\&D labor as necessary at the going market compensation for R\&D labor, and none if a lower compensation is offered. This implies a result analogous to Lemma \ref{lemma:RD_worker_indifference} to $w_{RD,jt}(\mathbbm{1}^{NCA})$.

\begin{lemma}\label{lemma:RD_worker_indifference1}
	The incumbent takes as given the R\&D wage it must pay conditional on $\mathbbm{1}^{NCA}_{jt} = \mathbbm{1}^{NCA}$,
	\begin{align*}
		w_{RD,jt}(\mathbbm{1}^{NCA}) &= \hat{w}_{RD,t} - (1-\mathbbm{1}^{NCA}) (\frac{\bar{q}_{jt}}{Q_t})^{-1} \nu (1-\kappa_e) V(j,t|\lambda \bar{q}_{jt})
	\end{align*}
\end{lemma}

\begin{proof}
	Follows from the discussion.
\end{proof}

The value of an incumbent must satisfy a Hamilton-Jacobi-Bellman equation,
\begin{align}
(r_t + \overbrace{\hat{\tau}}^{\mathclap{\text{Creative destruction}}}) &V(j,t |q) - \dot{V}(j,t|q) = \overbrace{\tilde{\pi} q }^{\mathclap{\text{Flow profits}}}\nonumber \\_{}
&+ \max_{\substack{\mathbbm{1}^{NCA} \in \{0,1\} \\ z \ge 0}} \Bigg\{ z \Big[  \overbrace{\chi \big( V(j,t|\lambda q) - V(j,t|q)\big)}^{\mathclap{\mathbb{E}[\text{Payoff from own-innovation}]}}  \nonumber \\
&- \underbrace{\big(\frac{q}{Q_t}\big)}_{\mathclap{\text{scaling of R\&D cost}}} \Big( \overbrace{w_{RD,jt}(\mathbbm{1}^{NCA})}^{\mathclap{\text{R\&D wage depends on NCA}}} + \underbrace{\big(\frac{q}{Q_t}\big)^{-1}}_{\mathclap{\text{scaling of spinout formation rate}}} \overbrace{(1-\mathbbm{1}^{NCA}) \nu V(j,t|q)}^{\mathclap{\mathbb{E}[\text{Loss from spinout CD}]}} + \underbrace{\big(\frac{q}{Q_t}\big)^{-1}}_{\mathclap{\text{scaling of NCA cost}}}  \overbrace{\mathbbm{1}^{NCA} \kappa_c \nu V(j,t|q) }^{\mathclap{\text{NCA cost}}}\Big)  \Big] \Bigg\} \label{eq:hjb_incumbent_0}
\end{align}
where $\tilde{\pi}$ is defined in (\ref{profits_eq}). The discounting is at the risk-free rate $r_t$ because the financial intermediary diversifies across incumbents and there is no aggregate risk.

The first proposition shows that, in a symmetric BGP, the value function must have a linear form.

\begin{proposition}\label{proposition:hjb_scaling}
	In a symmetric BGP, the value function of the incumbent is given by
	\begin{align*}
		V(j,t|q) &= \tilde{V} q
	\end{align*}
	for some $\tilde{V} > 0$.
\end{proposition}

The result follows from two facts. First, in a BGP the interest rate is constant, by the Euler equation (\ref{eq:euler0}). Second, by definition $\hat{z}_{jt} = \hat{z}$ in a symmetric BGP. Together these imply that solutions to the incumbent HJB either satisfy $V(j,t|q) = \tilde{V} q$. The technical details of the proof are contained in Appendix \ref{appendix:proofs:proposition:hjb_scaling}.

The above implies the following corollary.

\begin{proposition_corollary}
	In a symmetric BGP, the equilibrium R\&D wages are given by 
	\begin{align*}
	\hat{w}_{RD,t} &= \hat{w}_{RD} Q_t \\
	w_{RD,jt}(\mathbbm{1}^{NCA}) &= w_{RD}(\mathbbm{1}^{NCA}_{jt}) Q_t \textrm{, if $z > 0$}
	\end{align*}
\end{proposition_corollary}

\begin{proof}
	The entrant's free entry condition is
	\begin{align}
	\hat{\chi} \hat{z}^{-\psi} V(j,t|\lambda \bar{q}_{jt}) &= \frac{\bar{q}_{jt}}{Q_t} \hat{w}_{RD,t}
	\end{align}
	
	where $V(j,t|\lambda \bar{q}_{jt})$ is the value that the entrant will have as the new incumbent if he successfully innovates in the next instant. By the previous formula, $V(j,t | \lambda \bar{q}_{jt}) = \tilde{V} \lambda \bar{q}_{jt}$. Substituting yields
	\begin{align}
	\hat{\chi} \hat{z}^{-\psi} \tilde{V} \lambda &= \frac{\hat{w}_{RD,t}}{Q_t}
	\end{align}
	
	implying that $\frac{\hat{w}_{RD,t}}{Q_t}$ must be constant, i.e. $\hat{w}_{RD,t} = \hat{w}_{RD} Q_t$ for some $\hat{w}_{RD}$. Using this and $V(j,t | \bar{q}_{jt}) = \tilde{V}\bar{q}_{jt}$ in Lemma \ref{lemma:RD_worker_indifference1} yields $w_{RD,jt}(\mathbbm{1}^{NCA}) = w_{RD}(\mathbbm{1}^{NCA}) Q_t$. 
\end{proof}

The next proposition characterizes the equilibrium NCA policy of all incumbents in a symmetric BGP with $z > 0$. 

\begin{proposition}\label{proposition:optimalNCApolicy}
	In a symmetric BGP with $z > 0$, the equilibrium NCA policy of all incumbents is given by 
	\begin{align}
	\mathbbm{1}^{NCA}_{jt} = \mathbbm{1}^{NCA} = \begin{cases}
	1 & \textrm{if } \kappa_{c} < \bar{\kappa}_c (\kappa_e, \lambda) \\
	0 & \textrm{if } \kappa_{c} > \bar{\kappa}_c (\kappa_e, \lambda)\\
	\{0,1\} & \textrm{if } \kappa_c = \bar{\kappa}_c (\kappa_e, \lambda) 
	\end{cases} \label{eq_nca_policy}
	\end{align}
	where $\bar{\kappa}_c (\kappa_e, \lambda) = 1 - (1-\kappa_e)\lambda$.

\end{proposition}

Note that on the knife-edge $\kappa_c = \bar{\kappa}_c$, the incumbent is indifferent between $\mathbbm{1}^{NCA} = 0$ and $\mathbbm{1}^{NCA} = 1$. The proof entails some algebraic manipulations using the previous two lemmas. 

\begin{proof}
	Using the representation $V(j,t|q) = \tilde{V}q$ derived in Proposition \ref{proposition:hjb_scaling} in the incumbent HJB (\ref{eq:hjb_incumbent_0}) and dividing both sides by $q$ yields
	\begin{align}
	(r + \hat{\tau}) &\tilde{V} = \tilde{\pi} + \max_{\substack{\mathbbm{1}^{NCA} \in \{0,1\} \\ z \ge 0}} \Bigg\{ z \Big( \chi (\lambda -1) \tilde{V}- w_{RD}(\mathbbm{1}^{NCA}) - (1-\mathbbm{1}^{NCA}) \nu \tilde{V} - \mathbbm{1}^{NCA} \kappa_c \nu \tilde{V} \Big)\Bigg\} \label{eq:hjb_incumbent_1}
	\end{align}
	
	In any symmetric BGP with $z > 0$, Lemma \ref{lemma:RD_worker_indifference} determines the relationship between $w_{RD,jt}(\mathbbm{1}^{NCA}_{jt})$ and $\hat{w}_{RD,t}$.  Substituting in $w_{RD}(\mathbbm{1}^{NCA})$ using the indifference condition (\ref{eq:RD_worker_indifference}) derived in Lemma \ref{lemma:RD_worker_indifference} yields
	\begin{align}
	(r + \hat{\tau}) \tilde{V} &= \tilde{\pi} + \max_{\substack{\mathbbm{1}^{NCA} \in \{0,1\} \\ z \ge 0}} \Big\{z \Big( \overbrace{\chi (\lambda - 1) \tilde{V}}^{\mathclap{\mathbb{E}[\textrm{Benefit from R\&D}]}}- \hat{w}_{RD} -  \underbrace{(1-\mathbbm{1}^{NCA})(1 - (1-\kappa_{e})\lambda)\nu \tilde{V}}_{\mathclap{\text{Net cost from spinout formation}}} - \overbrace{\mathbbm{1}^{NCA} \kappa_{c} \nu \tilde{V}}^{\mathclap{\text{Direct cost of NCA}}}\Big) \Big\} \label{eq:hjb_incumbent_workerIndiff}
	\end{align}

	
	Let $\bar{\kappa}_c (\kappa_e, \lambda) = 1 - (1-\kappa_e)\lambda$. If $z > 0$, the incumbent maximizes her flow payoff by choosing $\mathbbm{1}^{NCA} \in \{0,1\}$ which maximizes the term multiplying $z$. Therefore, $\mathbbm{1}^{NCA} = 1$ is strictly preferred iff $1 - (1-\kappa_e) \lambda > \kappa_c$, which is equation (\ref{eq_nca_policy}).
\end{proof}

Equation (\ref{eq:hjb_incumbent_workerIndiff}) has an intuitive economic interpretation. The left-hand side is the equilibrium flow payoff on an asset with value $\tilde{V}$ plus the expected loss upon creative destruction by an entrant. The RHS is the flow payoff of incumbency, absent creative destruction by entrants. The term $\chi(\lambda -1) \tilde{V}$ is the expected benefit per unit of R\&D effort. Notice the factor $\lambda -1$, which takes into account the opportunity cost of no longer producing with the obsolete technology. The term $-\hat{w}_{RD}$ reflects the cost of R\&D effort due to the contribution from the prevailing R\&D wage. The term $-(1-\mathbbm{1}^{NCA})(1 - (1-\kappa_e) \lambda) \nu \tilde{V}$ represents the expected net harm to the incumbent due to spinouts from the employee. Expanding this term, the term $-(1-\mathbbm{1}^{NCA})\nu \tilde{V}$ reflects the direct harm from creative destruction by spinouts. The second term $(1-\mathbbm{1}^{NCA})(1-\kappa_e)\lambda \nu \tilde{V}$ reflects the reduction in R\&D wage accepted by the R\&D employee in return for being free to open spinouts. Finally, the term $-\mathbbm{1}^{NCA} \kappa_c \nu \tilde{V}$ reflects the direct cost of enforcing NCAs.

\paragraph{Entry, aggregation and market clearing}

As before, suppose first that $z > 0$. The incumbent's FOC implies that, in equilibrium, the term multiplying $z$ in (\ref{eq:hjb_incumbent_workerIndiff}) must equal zero,
\begin{align*}
	0 &= \chi(\lambda-1)\tilde{V}- \hat{w}_{RD} - (1-\mathbbm{1}^{NCA})(1 - (1-\kappa_e)\lambda) \nu \tilde{V} - \mathbbm{1}^{NCA} \kappa_c \nu \tilde{V}
\end{align*}

Solving for $\tilde{V}$ yields
\begin{align}
	\tilde{V} &= \frac{\hat{w}_{RD}}{\chi(\lambda - 1) - (1-\mathbbm{1}^{NCA}) (1- (1-\kappa_e)\lambda)\nu - \mathbbm{1}^{NCA} \kappa_{c} \nu} \label{eq:hjb_incumbent_foc}
\end{align}

If the denominator is negative then $\tilde{V} < 0$ for positive wage $\hat{w}_{RD}$, which contradicts optimality as $z = 0$ yields a positive value. Suppose therefore that the denominator is positive. Entrant innovation satisfies a free entry condition,\footnote{The original condition is 
	\begin{align*}
		\hat{\chi} \hat{z}^{-\psi} (1-\kappa_e)  V(j,t|\lambda q) = \frac{q}{Q_t} \hat{w}_{RD,t}
	\end{align*}
	Using $V(j,t|q) = \tilde{V} q$ and $\hat{w}_{RD,t} = \hat{w}_{RD} Q_t$ yields (\ref{eq:free_entry_condition}).}
\begin{align}
	\underbrace{\hat{\chi} \hat{z}^{-\psi}}_{\mathclap{\text{Marginal innovation rate}}} \overbrace{(1-\kappa_e) \lambda \tilde{V}}^{\mathclap{\text{Payoff from innovation}}} &= \underbrace{\hat{w}_{RD}}_{\mathclap{\text{MC of R\&D}}} \label{eq:free_entry_condition}
\end{align}

Substituting $\tilde{V}$ using (\ref{eq:hjb_incumbent_foc}) yields an expression for entrant R\&D effort, 
\begin{align}
	\hat{z} &= \Big( \frac{\hat{\chi} (1-\kappa_{e}) \lambda}{\chi(\lambda-1) - (1-\mathbbm{1}^{NCA}) (1- (1-\kappa_e)\lambda)\nu - \mathbbm{1}^{NCA} \kappa_{c} \nu} \Big)^{1/\psi} \label{eq:effort_entrant}
\end{align}

Market clearing for R\&D labor requires
\begin{align}
	\bar{L}_{RD} &= \int_0^1 \frac{q_j}{Q} (z + \hat{z}) dj = z + \hat{z} \label{eq:RD_labor_market_clearing}
\end{align}
 
which implies
\begin{align}
	z &= \bar{L}_{RD} - \hat{z} \label{eq:zI_asFuncZe}
\end{align}

Growth is determined by the growth accounting equation\footnote{To see this, let $\Delta > 0$ and let $J_0(\Delta)$ ($J_1(\Delta)$) denote the indices $j\in [0,1]$ on which innovation occurs zero (one) times between $t$ and $t+\Delta$. By the law of large numbers, the set $J_1(\Delta)$ has measure $\mu_1 \Delta = (\tau + \tau^S + \hat{\tau})\Delta + o(\Delta)$. The set $J_0(\Delta)$ has measure $1 - \mu_1 \Delta + o(\Delta)$. 
		\begin{align*}
			Q_{t+\Delta} = \int_0^1 \bar{q}_{j,t+\Delta} dj &= \int_{j \in J_0} \bar{q}_{jt} dj + \int_{j \in J_1} \lambda \bar{q}_{jt} dj + o(\Delta) \\
			&= (1 - \mu_1\Delta - o(\Delta)) Q_t + (\mu_1 \Delta + o(\Delta) ) \lambda Q_t + o(\Delta) \\
			&= (1 - \mu_1\Delta) Q_t + \mu_1\Delta \lambda Q_t + o(\Delta)
 	\end{align*}
 where I used the fact that $\mathbb{E}[\bar{q}_{jt} | j \in J_0, t]  = \mathbb{E}[\bar{q}_{jt} | j \in J_1, t] = Q_t$, since innovations happen at the same rate regardless of $\bar{q}_{jt}$. It follows that
\begin{align*}
	\frac{\dot{Q}_t}{Q_t} = \frac{\lim_{\Delta \to 0} \frac{Q_{t+\Delta} - Q_t}{\Delta}}{Q_t} &= (\lambda - 1)\mu_1 
	\end{align*}}
\begin{align}
g &= (\lambda - 1)(\tau + \tau^S + \hat{\tau}) \label{eq:growth_accounting}
\end{align}

The Euler equation determines the interest rate, 
\begin{align}
	g &= \frac{\dot{C}}{C} = \frac{1}{\theta} (r - \rho) \label{eq:euler} \\
	\therefore r &= \theta g + \rho \label{eq:interest_rate}
\end{align}

Substituting the incumbent's FOC into the incumbent's HJB, and using the expression for the interest rate, yields the incumbent's value $\tilde{V}$,
\begin{align}
	 \tilde{V} &= \frac{\tilde{\pi}}{r + \hat{\tau}} \label{eq:hjb_incumbent_gordon_formula}
\end{align}

Finally, the free entry condition (\ref{eq:free_entry_condition}) determines the equilibrium value of $\hat{w}_{RD}$. If $\hat{w}_{RD}$ is negative or if (\ref{eq:effort_entrant}) implies that $\hat{z} > \bar{L}_{RD}$ then the assumption that $z > 0$ in a symmetric equilibrium leads to a contradiction, and instead equilibrium has $\hat{z} = \bar{L}_{RD}$ and $z = \tau = \tau^S = 0$. Then $g = (\lambda - 1) \hat{\tau}$. The interest rate is derived via the Euler equation (\ref{eq:interest_rate}), the incumbent value from (\ref{eq:hjb_incumbent_gordon_formula}), and the wage from (\ref{eq:free_entry_condition}). This derivation implies the following.

\begin{assumption}\label{model:assumption:zPositive0}
	$\chi(\lambda -1) > \nu \min \{ 1 - (1-\kappa_e) \lambda, \kappa_c \}$ 
\end{assumption}

\begin{assumption}\label{model:assumption:zPositive}
	$\Big( \frac{\hat{\chi} (1-\kappa_{e}) \lambda}{\chi(\lambda-1) - \nu \min\{ 1-(1-\kappa_e) \lambda, \kappa_c \}} \Big)^{1/\psi} < \bar{L}_{RD}$
\end{assumption}


\begin{lemma}\label{model:lemma:zge0condition}
	If a symmetric BGP exists, then Assumptions \ref{model:assumption:zPositive0} and \ref{model:assumption:zPositive} hold if and only if $z > 0$. 
\end{lemma}

\begin{proof}
	Follows from the discussion.
\end{proof}

The next assumption provides conditions under which household utility in the BGP is finite. This is necessary for the equilibrium to be well-defined.\footnote{In a standard setup, Assumption \ref{model:assumption:boundedUtility1} would be a consequence of the household's transversality condition.}

\begin{assumption}\label{model:assumption:boundedUtility1}
	$\rho > (1-\theta) g$
\end{assumption} 

In Assumption \ref{model:assumption:boundedUtility1}, $g$ stands for the closed form expression for $g$. Namely, if Assumptions \ref{model:assumption:zPositive0} and \ref{model:assumption:zPositive} hold then 
\begin{align}
g &= (\lambda - 1) \Big(  \big( \frac{\hat{\chi} (1-\kappa_e \lambda}{\chi(\lambda-1) - \nu \min \{1 - (1-\kappa_e) \lambda, \kappa_c \}} \big)^{(1-\psi)/\psi)} \\
&+ \big(\chi + (1- \mathbbm{1}^{NCA}_{\kappa_c < \bar{\kappa}_c(\kappa_e,\lambda)})\nu \big) \big( \bar{L}_{RD} -  \frac{\hat{\chi} (1-\kappa_e \lambda}{\chi(\lambda-1) - \nu \min \{1 - (1-\kappa_e) \lambda, \kappa_c \}} \big)^{1/\psi} \big) \Big) 
\end{align}

and otherwise
\begin{align}
g &= (\lambda -1) \hat{\chi} \bar{L}_{RD}^{1-\psi}
\end{align}

\begin{lemma}
	Under Assumption \ref{model:assumption:boundedUtility1}, the household's utility is finite on a symmetric BGP with growth rate $g$.
\end{lemma}

\begin{proof}
	Using $C(t) = C(0)e^{gt}$ on the BGP, the household's utility is
	\begin{align}
		U = \mathcal{K} \int_0^{\infty} e^{-\rho t} e^{(1-\theta)gt} dt + \text{Constant}
	\end{align}
	
	for some constant $\mathcal{K} > 0$. The integral $\int_0^{\infty} e^{-\rho t} e^{(1-\theta)gt} dt$ converges if and only if $\rho > (1-\theta)g$. 
\end{proof}

Note that $\theta \ge 1$ implies Assumption \ref{model:assumption:boundedUtility1}. This is the empirically relevant case which I consider in the calibration. I next state some propositions concerning existence and uniqueness of the symmetric BGP. 

\begin{proposition}\label{proposition:BGPexistence}
	Under Assumption \ref{model:assumption:boundedUtility1}, there exists a symmetric BGP.
\end{proposition}

\begin{proof}
	To see this, one does not even need Proposition \ref{proposition:hjb_scaling}. Simply guess and verify that $V(j,t|q) = \tilde{V} q$ by solving for equilibrium variables as described in the last section. Assumption \ref{model:assumption:boundedUtility1} guarantees that household utility is finite and therefore the obtained allocation is actually an equilibrium.
\end{proof}

\begin{proposition}\label{proposition:purstrategyeq:positiveOI}
	If Assumptions \ref{model:assumption:zPositive0}, \ref{model:assumption:zPositive} and \ref{model:assumption:boundedUtility1} hold and $\kappa_c \ne \bar{\kappa}_c$ (as defined in Proposition \ref{proposition:optimalNCApolicy}), then:
	\begin{enumerate}
		\item There exists a unique symmetric BGP.
		\item On the symmetric BGP, $z > 0$ and $\mathbbm{1}^{NCA}_{jt} = \mathbbm{1}^{NCA}$
	\end{enumerate}
\end{proposition}

\begin{proof}
	Existence follows from Proposition \ref{proposition:BGPexistence}. To show uniqueness, first notice that in any symmetric BGP, one has $V(j,t|\bar{q}_{jt}) = \tilde{V}\bar{q}_{jt}$ by Proposition \ref{proposition:hjb_scaling} and its corollary. Given this representation and $\kappa_c \ne \bar{\kappa}_c$, Proposition \ref{proposition:optimalNCApolicy} implies that all symmetric BGPs have $\mathbbm{1}^{NCA}_{jt} = \mathbbm{1}^{NCA}$ for the same value for $\mathbbm{1}^{NCA}$. Together, these facts uniquely pin down the solution to the system of equations described in the derivation of the model. Hence there is a unique symmetric BGP. Finally, by Lemma \ref{model:lemma:zge0condition}, the symmetric BGP has $z > 0$. 
\end{proof}

\begin{proposition}\label{proposition:purestrategyeq:incumbents_indifferent}
	If Assumptions \ref{model:assumption:zPositive0}, \ref{model:assumption:zPositive} and \ref{model:assumption:boundedUtility1} hold and $\kappa_c = \bar{\kappa}_c$ (as defined in Proposition \ref{proposition:optimalNCApolicy}), then:
	\begin{enumerate}
		\item There exist exactly two symmetric BGPs with $\mathbbm{1}^{NCA}_{jt} = \mathbbm{1}^{NCA}$: one with $\mathbbm{1}^{NCA}_{jt} = 0$ and one with $\mathbbm{1}^{NCA}_{jt} = 1$.
		\item Both such equilibria have the same R\&D labor allocations $z, \hat{z}$
		\item The equilibrium with $\mathbbm{1}^{NCA}_{jt} = 0$ has a higher growth rate $g$ 
	\end{enumerate} 
\end{proposition}

\begin{proof}
	The proof of the first part is essentially the same as that of the previous proposition. The only difference is that either choice $\mathbbm{1}^{NCA}_{jt} = 1$ or $\mathbbm{1}^{NCA}_{jt} = 0$ is valid under Proposition \ref{proposition:optimalNCApolicy}. Given the representation $V(j,t|q) = \tilde{V}q$ and the scaling of wages $\hat{w}_{RD,t} = \hat{w}_{RD}Qt$ and $w_{RD,j}(\mathbbm{1}^{NCA}) = w_RD(\mathbbm{1}^{NCA}) Q_t$, the derivation above uniquely determines uniquely the rest of the equilibrium conditional on $x$. This equilibrium has finite household utility as long as Assumption \ref{model:assumption:boundedUtility1} holds. 
	
	The second part follows from the fact that when $\kappa_c = \bar{\kappa}_c$, the expressions for equilibrium R\&D effort $\hat{z},z$ do not depend on $\mathbbm{1}^{NCA}$. The reason is that $\mathbbm{1}^{NCA}$ only affects $\hat{z},z$ through its effect on the incumbent's effective wage, but here is the incumbent is indifferent between $\mathbbm{1}^{NCA} = 1$ and $\mathbbm{1}^{NCA} = 0$ hence faces the same effective wage. Mathematically, (\ref{eq:effort_entrant}) has the expression $(1-\mathbbm{1}^{NCA})(1-(1-\kappa_e)\lambda)\nu - \mathbbm{1}^{NCA} \kappa_c \nu = (1-\mathbbm{1}^{NCA}) \bar{\kappa}_c \nu + \mathbbm{1}^{NCA} \kappa_c \nu$ in the denominator. Since $\kappa_c = \bar{\kappa}_c$, $\hat{z}$ is unaffected by $\mathbbm{1}^{NCA}$, which in turn implies $z$ is also unaffected.
	
	The last statement follows from the fact that $z,\hat{z}$ are the same in both equilibria, but $\tau^S = 0$ when $\mathbbm{1}^{NCA} = 1$ and $\tau^S = \nu z^I > 0$ when $\mathbbm{1}^{NCA} = 0$. By the growth accounting equation (\ref{eq:growth_accounting}), this implies $g$ is higher when $\mathbbm{1}^{NCA} = 0$. 
\end{proof}

\begin{proposition}\label{proposition:purstrategyeq:zeroOI}
	If Assumption \ref{model:assumption:boundedUtility1} holds but either of Assumptions \ref{model:assumption:zPositive0} or \ref{model:assumption:zPositive} do not, then:
	\begin{enumerate}
		\item There is a unique symmetric BGP (modulo irrelevant incumbent choice of $\mathbbm{1}^{NCA}_{jt}$)
		\item This BGP has $z = 0$ and $\hat{z} = \bar{L}_{RD}$.
	\end{enumerate} 
\end{proposition}

\begin{proof}
	By Lemma \ref{model:lemma:zge0condition}, $z = 0$ in a symmetric BGP. Then $\hat{z} = \bar{L}_{RD}$ by R\&D labor market clearing (\ref{eq:zI_asFuncZe}). The rest of the equilibrium is pinned down given $x$ by the derivation given the representation $V(j,t|q) = \tilde{V}q$ and the scaling of wages $\hat{w}_{RD,t} = \hat{w}_{RD}Qt$ and $w_{RD,j}(\mathbbm{1}^{NCA}) = w_RD(\mathbbm{1}^{NCA}) Q_t$, which Lemma \ref{proposition:hjb_scaling} proves holds on any symmetric BGP. This shows that the equilibrium does not depend on the choice of $x$ by incumbents in a symmetric equilibrium with $z = 0$. Hence, the symmetric BGP is unique modulo an irrelevant choice by the incumbent of whether to use $\mathbbm{1}^{NCA}_{jt} = 0$ or $\mathbbm{1}^{NCA}_{jt} = 1$. 
\end{proof}

Finally, there is a technical possibility of "mixed strategy" equilibria on the knife-edge $\kappa_c = \bar{\kappa}_c$ where both choices of $x$ occur in equilibrium. This result is included for completeness but I will not study this case further in this paper.

\begin{proposition}\label{proposition:mixedstrategyeq}
	If $\theta \ge 1$, $\kappa_c = \bar{\kappa}_c$, and $\Big( \frac{\hat{\chi} (1-\kappa_{e}) \lambda}{\chi(\lambda-1) - \kappa_{c} \nu} \Big)^{1/\psi} < \bar{L}_{RD}$, then for all $f \in (0,1)$ there exists a symmetric BGP in which, at any given time $t$, a fraction $f$ of incumbents $j$ have $\mathbbm{1}^{NCA}_{jt} = 1$.  
\end{proposition}

\begin{proof}
	See Appendix \ref{appendix:model:proofs:proposition:mixedstrategyeq}.
\end{proof}


\section{Efficiency and theoretical policy analysis}\label{model:efficiency:efficiency}


%\setcounter{secnumdepth}{3}

The goal of this section is to motivate the the empirics in Section \ref{sec:empirics} and the calibration that follows in Section \ref{sec:calibration} as well as to provide a theoretical foundation for the quantitative policy analysis of Section \ref{sec:policy_analysis}. I first discuss the three potential sources of misallocation in the decentralized equilibrium, corresponding to the three allocative margins in a symmetric BGP: production labor, R\&D labor, and NCAs. I argue that all three are in general inefficient but that, in certain cases, reducing barriers to the enforcement of NCAs can help mitigate the misallocation of R\&D labor. I discuss how this depends on the parameters of the model, motivating the empirical and quantitative analysis that follows. I then turn to other policies: R\&D subsidies, R\&D subsidies targeted at own-product innovation, a tax on creative destruction. I offer an analogous discussion of each policy and obtain some novel theoretical predictions about the effects of untargeted R\&D subsidies. I close by noting that OI-targeted R\&D subsidies combined with a ban on NCAs is optimal in this setting because it allows spinouts to form while correcting the misallocation of R\&D.\footnote{I defer detailed discussion of this policy to the numerical analysis of Section \ref{sec:policy_analysis}.}

\subsection{Preliminaries}

\subsubsection{Welfare}

Social welfare is simply the representative household's lifetime utility,\footnote{Technically this should be written in terms of $W_t$, the welfare at time $t$. I ignore this detail in the interest of expositional simplicity and without loss of generality since the model grows at constant rate so $W_t = e^{(1-\theta)gt}\tilde{W}$.} 
\begin{align}
	\tilde{W} = \int_0^{\infty} e^{-\rho t} \frac{C(t)^{1-\theta} - 1}{1-\theta} ds \label{eq:agg_welfare0}
\end{align}

Using $C(t) = \tilde{C} e^{gt}$ on the BGP and integrating yields
\begin{align}
	\tilde{W} &= \frac{\tilde{C}^{1-\theta} }{(1-\theta)(\rho - g(1-\theta))} + \text{Constant}
\end{align}

Social welfare can thus be decomposed into a \textit{growth} channel ($g$) and an \textit{initial consumption} channel ($\tilde{C}$). Higher values for either imply higher welfare. In turn, $\tilde{C}$ can be decomposed using 
\begin{align}
	\tilde{C} &= \tilde{Y} - \overbrace{(\hat{\tau} + \tau^S) \kappa_e \lambda \tilde{V}}^{\mathclap{\text{Creative destruction cost}}} - \underbrace{x z \kappa_c \nu \tilde{V}}_{\mathclap{\text{NCA enforcement cost}}} \label{eq:agg_consumption_decomposition}
\end{align}

so that steady-state consumption is flow output of the final good minus the final goods cost of creative destruction and of NCA enforcement.

\subsection{Efficiency of decentralized equilibrium}

As mentioned previously, the decentralized equilibrium is inefficient. The model has three margins: production labor, R\&D labor, and noncompetes. In this section I show that these margins are all affected by externalities in this model and hence in general are not optimal. While the social planer's first-best allocation is not well-defined in this setting\footnote{As discussed previously, the model could be modified in a straightforward way to not have this feature, but the solution would no longer be in closed form.} due to the presence of the equilibrium value $V(j,t|q)$ in the specification of certain technologies, one can still make precise statements regarding the efficiency of the allocation of production labor and the effect of the allocation of R\&D and NCAs on the growth rate. This will provide some theoretical background for the quantitative policy analysis of Section \ref{sec:policy_analysis}.

\subsubsection{Misallocation of production labor: monopoly distortion}

The allocation of production labor in the economy is distorted by the monopoly power of producers in the intermediate goods market. As is standard, this monopoly power induces pricing higher than marginal cost and hence to an underallocation of production labor to intermediate goods production. This reduces $\tilde{C}$. I mention this for completeness only; from now on, I will ignore this source of inefficiency as it is not the focus of this analysis.\footnote{In this setting with exogenous total supply of R\&D, a subsidy to intermediate goods production would correct this externality and have no effect on equilibrium growth.}\footnote{In a model with limit pricing, average markups would depend on the distance between leaders and followers in good $j$. In that case, this distortion interacts with the distortion to R\&D and would need to be considered.}

\subsubsection{Misallocation of R\&D labor}\label{model:efficiency:misallocationRD}

The decentralized allocation of R\&D labor is also, in general, not efficient. Because total R\&D spending is exogenous, any inefficiency must be due to a misallocation of R\&D \textit{between} OI by incumbents and CD by entrants.

To isolate the determinants of the degree of equilibrium misallocation, first consider the equilibrium marginal effects on the innovation rate from more incumbent OI and entrant CD, respectively. If the marginal effect of entrant CD on innovation is lower, then equilibrium innovation, and therefore growth, would increase after a reallocation of R\&D labor to incumbent OI. The marginal effect of OI, including the induced innovation by spinouts, is equal to $\chi + (1-\mathbbm{1}^{NCA}) \nu$. The marginal effect of CD by entrants is
\begin{align}
\frac{d}{d\hat{z}} \hat{\tau} &= (1-\psi) \hat{\chi} \hat{z}^{-\psi} \label{eq:marginal_effect_effort_entrant}
\end{align}
%
Substituting the expression for $\hat{z}$ in (\ref{eq:effort_entrant}), dividing by $\chi + (1-\mathbbm{1}^{NCA})\nu$, and rearranging yields 
\begin{align}
	\frac{\frac{d}{d\hat{z}} \hat{\tau}}{\chi + (1-\mathbbm{1}^{NCA})\nu} &= \overbrace{\frac{\lambda-1}{\lambda}}^{\mathclap{\text{Business stealing}}} \times \underbrace{(1-\psi)}_{\mathclap{\text{Congestion}}}  \times \overbrace{\frac{\chi(\lambda-1) -(1-\mathbbm{1}^{NCA}) (1-(1-\kappa_e)\lambda)\nu - \mathbbm{1}^{NCA} \kappa_c \nu}{\chi(\lambda-1)}}^{\mathclap{\text{Effective cost of R\&D}}} \nonumber \\
	&\times \underbrace{\frac{\chi}{\chi + (1-\mathbbm{1}^{NCA})\nu}}_{\mathclap{\text{Spinout formation}}} \times  \overbrace{\frac{1}{1-\kappa_{e}}}^{\mathclap{\text{Entry cost}}}  \label{cs:growth_misallocation_condition}
\end{align}

If the RHS is less than 1, then reallocating R\&D labor from entrants to incumbents increases the BGP growth rate. This depends on the value of the five factors on the RHS, which I discuss below. 

\subparagraph{Business stealing}

The term $\frac{\lambda - 1}{\lambda} < 1$, reflects the \textit{business stealing} externality.\footnote{This is sometimes referred to as \emph{Arrow's replacement effect}, which emphasizes the fact that incumbents, unlike entrants, take into account the fact that they replace their monopoly. They are two sides of the same coin.} Innovation by entrants imposes a negative externality on the profits of the incumbent. This means that entrants can earn the required (private) return on R\&D with a lower innovation rate per marginal cost than incumbents. In the calibration, $\lambda \approx 1.1$ so $\frac{\lambda-1}{\lambda} \approx 0.09$, so this effect can be strong in magnitude.\footnote{In models such as \cite{aghion_competition_2005}, this effect is attenuated by the fact that incumbents engage in neck-and-neck competition within each good $j$. This means R\&D by incumbents has a negative externality on other incumbents in the same good $j$, making the situation more symmetric between incumbents and entrants. I plan to explore this question further in later work.}


\paragraph{Congestion}

The term $1-\psi < 1$ reflects the \textit{congestion} externality. Individual entrants impose a negative externality on the expected returns of other entrants. As with business-stealing, the congestion externality also tends to overallocate R\&D to entrants. To give a sense of magnitude, in the calibration $\psi = 0.5$ so $1-\psi = 0.5$.

\paragraph{Effective cost of R\&D} 

The term $\frac{\chi(\lambda-1) -(1-\mathbbm{1}^{NCA}) (1-(1-\kappa_e)\lambda)\nu - \mathbbm{1}^{NCA} \kappa_c \nu}{\chi(\lambda-1)}$ reflects the fact that entrants pay a different effective cost of R\&D than the incumbent. As long as $(1 - (1-\kappa_e) \lambda > 0$, incumbents pay a higher effective cost because they either internalize the harm from future WSOs or must pay a cost to enforce NCAs to prevent them. All else equal, this means entrants have a higher private return to R\&D than incumbents. In equilibrium these private returns must equate; therefore, there is more more entry and it has a lower marginal effect on growth. Alternatively, if $1 - (1-\kappa_e) \lambda < 0$, incumbents benefit from spinouts \textit{ex ante} because they are bilaterally efficient, and as a consequence they pay a lower effective cost of R\&D. This has the opposite effect of increasing the equilibrium marginal effect on growth of entrant R\&D.


\paragraph{Spinout formation}

The term $\frac{\chi}{\chi + (1-\mathbbm{1}^{NCA})\nu} \le 1$ reflects the contribution to the productivity of OI stemming from entry by WSOs. If $\mathbbm{1}^{NCA} = 0$ and $\nu > 0$, the term is strictly less than 1. OI by incumbents has a positive growth externality (through spinout entry) hence, in equilibrium it generates a higher marginal effect on growth from OI. If $\mathbbm{1}^{NCA} = 1$ or $\nu = 0$ this term is equal to 1 and has no effect on the inequality, corresponding to $\tau^S = 0$.


\paragraph{Entry cost}

Finally, the term $\frac{1}{1-\kappa_e} \ge 1$ reflects the additional entry cost paid by entrants upon innovating. All else equal, this implies entrants have a lower private return from R\&D spending. In equilibrium, the returns to R\&D spending must be the same therefore $\hat{z}$ declines. This tends to reduce the extent of misallocation, as it works against the net of the other terms on the RHS of equation (\ref{cs:growth_misallocation_condition}). Of course, this comes at the cost of reduced initial consumption $\tilde{C}$. But the net effect of increasing $\kappa_e$ is typically to improve the allocation of R\&D labor overall. Intuitively, while a lack of arbitrage opportunities implies that the private return of innovation by entrants must be equal to that of incumbents, the ratio of their social returns is higher when the entrant's cost is in terms of final goods. The quantitative exercises in Section \ref{sec:policy_analysis} confirm this intuition.

\subsubsection{Misallocation of NCAs}\label{model:efficiency:misallocationNCAs}

Lastly, the allocation of NCAs is also technically inefficient on its own, but can increase inefficiency overall if it helps correct the misallocation of R\&D. First, note that using NCAs is generally suboptimal for growth in a partial equilibrium sense, since it throws away socially productive innovations.\footnote{This only applies when spinouts are socially valuable, which is the case of interest in this paper (and is the case in the calibrated model).} The easiest way to see this is to consider an exogenous shift from $\mathbbm{1}^{NCA} = 1$ to $\mathbbm{1}^{NCA} = 0$ while holding $z,\hat{z}$ constant. This increases the growth rate by $(\lambda -1) z \nu$ because spinout entrepreneurship is no longer prevented and because we have assumed that spinouts represent truly new innovations, rather than ideas stolen from the incumbent.

In general equilibrium, too, there is a force that leads to overallocation of NCAs, as incumbent and their employees do not internalize the positive growth externalities that spinouts have on the rest of the economy. Namely, the increase in steady state consumption for the representative household, the increase in the productivity of all other intermediate goods producers, and the fact that entrants are now able to innovate on a good of higher quality. In this model, this manifests as an expansion of the region in parameter space where NCAs are used but, were they not used, aggregate growth would be higher. In fact there is always such a region, provided spinouts are socially valuable. This can be most clearly seen by considering a case where the cost of NCAs such that incumbents are nearly indifferent between using and not using NCAs, i.e. $\kappa_c = \bar{\kappa}_c - \varepsilon$ for $\varepsilon > 0$ small enough. By Proposition \ref{proposition:purstrategyeq:positiveOI}, there are is a unique symmetric BGP with $\mathbbm{1}^{NCA} = 1$. As $\varepsilon \to 0$, incumbent R\&D effort with and without an NCA converges to the same value $z$. Hence, even if (\ref{cs:growth_misallocation_condition}) is less than 1, imposing $\mathbbm{1}^{NCA} = 0$ implies barely any growth-enhancing reallocation of R\&D, while increasing innnovation by spinouts by a discrete amount. 

However, incumbents also do not internalize the full social value of their own product innovation, for the same reasons as above. And the ability to use NCAs induces them to perform more of it. This means that if (\ref{cs:growth_misallocation_condition}) is less than one, so R\&D labor is not efficiently allocated, both using and not using NCAs has positive externalities, and it is therefore not determined whether imposing barriers to their use has a positive effect on growth and welfare. For example, in the previous paragraph, while it is beneficial to ban NCAs, it might be optimal to further reduce the barriers to the usage of NCAs. I analyze this question in more detail in Section \ref{subsubsec:ncacost}. 

\subsection{Effect of NCA enforcement and other policies}\label{model:efficiency:policy_analysis}

To make more concrete the ideas developed in the previous section, in this section I conduct a sequence of theoretical second-best analyses assuming the planner can control one or more parameters and/or Pigouvian taxes. The central question is whether reducing barriers to the use of NCAs increases or decreasing growth.\footnote{I defer the discussion of overall welfare to the quantitative analysis in Section \ref{sec:policy_analysis}, as the effect on $\tilde{C}$ of the policies I study depend in very complicated ways on parameters and as such the theoretical analysis generates more heat than light.} In addition, I also study other policies which may substitute or complement NCA enforceability policy, such as R\&D subsidies. While such policies can technically be studied in a standard quality ladders model, their \textit{interaction} with the endogenous use of NCAs is novel and yields some new theoretical insights. 

Overall, I find that the effect of NCA policy and adjacent policies on the growth rate (not to mention welfare) depends significantly on the parameters of the model. For this reason, in the following two sections I use data to calibrate the model.\footnote{In the empirical section, I obtain empirical discipline on $\nu$, which determines the magnitude of the effect of a reduction of NCA costs. In the calibration section, I use aggregate data, as well as some growth attribution estimates and standard parameters from the literature, to discipline the other parameters of the model, determining the sign.}

\paragraph{Policies considered} 

I study planners who can control:

\begin{enumerate}
	\item Cost of NCAs: $\kappa_c$ 
	\item R\&D subsidy (tax): $T_{RD}$
	\item Creative destruction tax (subsidy): $T_e$
	\item OI R\&D subsidy (tax): $T_{RD,I}$
	\item All of the above: $\{\kappa_c, T_{RD}, T_{RD,I}, T_e\}$
\end{enumerate}

\paragraph{Comparative statics}

All comparisons below are static comparisons between BGPs. I often use language like "as [a certain parameter or tax] increases..." or "as [parameter] crosses [a threshold], [equilibrium variable] jumps...". This does not refer to a transition path of the economy but to as static comparison of the BGPs for each value of the parameter or tax.\footnote{That being said, in this model it is the case that the economy immediately jumps to the unique new BGP following a parameter or tax change, provided it is assumed that the pre- and post-change equilibria are symmetric.}

\paragraph{Public finance} 

In cases of taxes (subsidies), I assume that they are rebated to (financed by) the representative household in a lump-sum payment. Because there is no labor-leisure choice, this does not create any additional distortions in the economy.\footnote{In general, however, this is a matter of first-order importance and an interesting avenue for further research. Policies should be evaluated in terms of "bang for buck."}

\subsubsection{NCA cost $\kappa_c$}\label{subsubsec:ncacost}

To analyze this question using the model, consider a planner who controls the parameter $\kappa_c$. I will consider the effect of a reduction in $\kappa_c$ on growth, starting from a value $\kappa_c > \bar{\kappa}_c$. I interpret this as a policymaker changing the extent of restrictions on NCAs so that they are more less difficult to use, or simply allowing their free use in cases where they are not currently allowed. Of course, NCAs require costs for enforcement even if they are fully endorsed by the legal system: a contract must be written and, in the case of infringement, it must be established that the employee is, in fact, competing with their previous employer. Therefore, it is reasonable to suppose a minimum NCA cost $\munderbar{\kappa}_c \ge 0$ such that $\kappa_c \ge \munderbar{\kappa}_c$. For simplicity, in the analysis below I assume $\munderbar{\kappa}_c = 0$.

Suppose first that $\mathbbm{1}^{NCA} = 0$ and $z > 0$, with $\kappa_c > \bar{\kappa}_c$. Initially, a marginal reduction in $\kappa_c$ has no effect on the equilibrium as NCAs $\mathbbm{1}^{NCA} = 0$ so the cost is not being paid by any agents. Upon crossing the threshold $\bar{\kappa}_c$, there is a shift to $\mathbbm{1}^{NCA} = 1$ and growth decreases by a discrete amount through a reduction in employee spinout formation. The allocation of R\&D is unchanged at this point because incumbents are indifferent between using and not using NCAs. The reduction in the growth rate induces via general equilibrium a desire to save, which lowers the equilibrium interest rate. This increases the value of the incumbent, so to keep the labor market in equilibrium, the R\&D wage rises discretely.

A further reduction in $\kappa_c$ reallocates R\&D labor from the entrant to the incumbent, using equations (\ref{eq:effort_entrant}) and (\ref{eq:zI_asFuncZe}). This reallocation of R\&D labor decreases the BGP growth rate if and only if the marginal effect on growth of incumbent R\&D is higher than the marginal efect on growth of entrant R\&D. As discussed in Section \ref{model:efficiency:misallocationRD}, this occurs whenever (\ref{cs:growth_misallocation_condition}) is sufficiently less than 1. 

The reallocation of R\&D occurs through a change in the ratio $\hat{w}_{RD} / \tilde{V}$. When $\kappa_c$ decreases, the ratio increases so that the incumbent's FOC continues to hold. This ratio then feeds into the reduction in entrant R\&D with sensitivity given by the return-elasticity of entrant R\&D spending. Intuitively, the increase in $\kappa_c$ makes R\&D more expensive for incumbents, reducing $z$ to zero in partial equilibrium. To clear the labor market, $\hat{w}_{RD}$ must decline to induce more R\&D. Because incumbents pay for R\&D not just through wages but implicitly through future WSOs, in the new equilibrium, their effective cost of R\&D is higher relative to entrants, whose only R\&D cost is the R\&D wage. As a result, incumbents employs a smaller share of the R\&D labor in equilibrium.

Finally note that if $z = 0$ initially, then decreasing $\kappa_{c}$ to values below $\bar{\kappa}_c$ has no effect, as the incumbent is on a corner solution. However, it is possible that for low enough $\kappa_c$, one has $z > 0$ in equilibrium. Thereafter, the effect of a reduction in $\kappa_c$ is the same as described above.

\paragraph{Does eliminating barriers to NCAs increase growth?}

The answer to this question depends on four main factors. The first three factors determine the sign of a maximal reduction in barriers to the use of NCAs. The fourth factor determines the magnitude.

\subparagraph{Extent of R\&D misallocation} The first factor is whether (\ref{cs:growth_misallocation_condition}) is less than 1, and by how much. The smaller is the RHS compared to 1, the more severe is the decentralized misallocation of R\&D spending and therefore the greater the growth increase from a marginal reallocation of R\&D. If the inequality does not hold, the reallocation of R\&D reduces growth and a reduction in $\kappa_c$ is unequivocally bad for welfare. 

\subparagraph{Elasticity of R\&D spending}

The second factor is how much (general equilibrium) reallocation will actually result from a given (partial equilibrium) change in the ratio of private returns to R\&D of incumbents and entrants. The latter is the mechanism by which a reduction in $\kappa_c$ affects the reallocation of R\&D. This depends on the price-elasticity of R\&D spending for both parties. The higher are these elasticities, the more reallocation there is in response to a given change in $\kappa_c$. In this model, the incumbent's elasticity is infinite, as she has constant returns to scale, while the entrant's elasticity is finite due to decreasing returns to scale. This would be more realistic if the incumbent's elasticity were finite, so I consider the case of decreasing returns to incumbent R\&D in a robustness check \textbf{[not yet in this draft]}. Experiments with this model using a very low entrant elasticity suggest that this will not change the results qualitatively, though it may reduce their magnitude somewhat.\footnote{If the incumbent had decreasing returns to R\&D, the increase in incumbent R\&D spending would help to bring the FOC back into alignment and there would be less required reallocation.} 

\subparagraph{Scope for reductions in $\kappa_c$}

Finally, the third factor is simply how large of a reduction in $\kappa_c$ is reasonably under the control of the policy maker. This has two components. If the model is to generate spinouts on the BGP (which is one of the calibrating assumptions), then it must set identify $\kappa_c \ge \bar{\kappa}_c = 1 - (1-\kappa_e) \lambda$. Therefore, the smaller is $(1-\kappa_e) \lambda$, the larger is the model-implied direct cost of using NCAs. In addition, it depends on the value of $\munderbar{\kappa}_c$, as discussed above. Essentially, this is the question of what fraction of the cost $\kappa_c$ should be interpreted as due to restrictions on contracting and how much is due to the direct costs of implementing an employment contract.\footnote{One might imagine subsidizing the use of NCAs (i.e. setting $\kappa_c < 0$), but as they are simply pieces of paper that can be produced even without actually doing R\&D, this would not be incentive compatible in reality. Alternatively, one could imagine directly subsidizing the R\&D spending of the incumbents. In the model when $\mathbbm{1}^{NCA} = 1$ these are equivalent. Later I consider this type of policy and find that it is actually part of the optimal policy.} 

\subparagraph{Rate of spinout formation}

The final factor is the rate of spinout formation $\nu$. A higher value amplifies the effect of reductions in $\kappa_c$. Mathematically, this occurs because $\kappa_c$ is the cost of NCAs per attempted spinout so it appears in the model in the form $\kappa_c \nu$. Hence a higher $\nu$ means that the cost of R\&D is more sensitive to changes in $\kappa_c$. Also, when $\kappa_c$ crosses the $\bar{\kappa}_c$ threshold, the effect on spinout entry is also amplified in the same way. Intuitively, a higher spinout attempted formation rate means that spinouts are more significant both to overall growth and to the incumbents who wish to avoid being replaced by them and hence their incentives for R\&D spending. Whether reducing barriers to NCAs is good or bad for growth, a higher rate of spinout formation amplifies the effect. 

Since all of these factors depend on parameters, I return to this question after I have calibrated the model (see Section \ref{sec:policy_analysis}). 
 

\subsubsection{RD subsidy (tax)}

The first alternative policy I consider is a subsidy to R\&D spending. This is a natural class of policy to study due to its significant magnitude the United States, where all told the Federal government funds about 15\% of private R\&D spending.\footnote{In the United States, the marginal R\&D subsidy rate is between 15 and 20\%, which is claimed via  deduction on corporate income taxes. The deduction can be carried forward twenty years. These R\&D subsidies are applied only to R\&D spending above a firm-specific base which is defined in reference to past levels of R\&D and firm sales. Taking this into account direct R\&D subsidies offer about a 5\% effective subsidy.  In addition, federal and local governments directly fund about 10\% of private business-performed R\&D.} In this context one might expect R\&D subsidies to have no effect at all on the equilibrium allocation, manifesting entirely in a higher equilibrium wage for R\&D labor. However, I will find instead that R\&D subsidies can affect the allocation of R\&D labor. Intuitively, they reallocate R\&D spending towards those agents for whom the costs of R\&D are more heavily composed of wages. As we have seen previously, the effective cost of R\&D for the incumbent is only partly wages, as it also consists of an expected future cost of replacement by a spinout. Thus, R\&D subsidies reallocate R\&D to entrants. This reduces growth if (\ref{cs:growth_misallocation_condition}) is less than one. 

To make this precise, suppose that the planner subsidizes R\&D spending at rate $T_{RD}$ (tax if $T_{RD} < 0$). In this case, in a symmetric BGP the incumbent's normalized HJB becomes
\begin{align}
(r + \hat{\tau}) \tilde{V} = \tilde{\pi} + \max_{\substack{\mathbbm{1}^{NCA} \in \{0,1\} \\ z \ge 0}} \Big\{z &\Big( \overbrace{\chi (\lambda - 1) \tilde{V}}^{\mathclap{\mathbb{E}[\textrm{Benefit from R\&D}]}}- (\underbrace{1-T_{RD}}_{\mathclap{\text{R\&D Subsidy}}}) \big( \overbrace{\hat{w}_{RD} - (1-\mathbbm{1}^{NCA})(1-\kappa_e)\lambda \nu \tilde{V}}^{\mathclap{\text{Incumbent R\&D wage, by Lemma \ref{lemma:RD_worker_indifference1} }}}\big) \label{eq:hjb_incumbent_RDsubsidy} \nonumber \\ 
&-  \underbrace{(1-\mathbbm{1}^{NCA}) \nu \tilde{V}}_{\mathclap{\text{Net cost from spinout formation}}} - \overbrace{\mathbbm{1}^{NCA} \kappa_{c} \nu \tilde{V}}^{\mathclap{\text{Direct cost of NCA}}}\Big) \Big\} 
\end{align}

Then if $z > 0$, the incumbent's optimal NCA policy is given by 
\begin{align}
x = \begin{cases}
1 & \textrm{if } \kappa_{c} < \tilde{\bar{\kappa}}_c  \\
0 & \textrm{if } \kappa_{c} > \tilde{\bar{\kappa}}_c \\
\{0,1\} & \textrm{if } \kappa_c = \tilde{\bar{\kappa}}_c 
\end{cases} \label{eq:nca_policy_RDsubsidy}
\end{align}

where $\tilde{\bar{\kappa}}_c = \tilde{\bar{\kappa}}_c(\kappa_e,\lambda;T_{RD}) = 1 - (1-T_{RD})(1-\kappa_e)\lambda$. Since the argument is the same as in Section \ref{subsubsec:dynamic_equilibrium_original_solution}, I omit the details. Assuming $z > 0$, by the same logic as before one can obtain an expression for equilibrium $\hat{z}$, 
\begin{align}
\hat{z} &= \Bigg( \frac{\hat{\chi} (1-\kappa_{e}) \lambda}{\chi(\lambda -1) - \nu (\mathbbm{1}^{NCA}\kappa_c + (1-\mathbbm{1}^{NCA})(1 - (1-T_{RD})(1-\kappa_e)\lambda)) } \Bigg)^{1/\psi} \label{eq:effort_entrant_RDsubsidy}
\end{align}

The rest of the equilibrium allocation and prices can be computed in the same way as before (including how to account for the possibility of $z = 0$), with the one exception being that the equilibrium R\&D wage is now given by 
\begin{align}
\hat{w}_{RD} &= (1-T_{RD})^{-1}\hat{\chi} \hat{z}^{-\psi} (1-\kappa_e) \lambda \tilde{V} \label{eq:wage_rd_labor_RDsubsidy}
\end{align}

\paragraph{Effect on growth}

First suppose $\mathbbm{1}^{NCA} = 0$ and consider a small increase in $T_{RD}$ from $T_{RD}^0$ to $T_{RD}^1 > T_{RD}^0$. If $\mathbbm{1}^{NCA} = 0$ after the increase in $T_{RD}$, then by (\ref{eq:effort_entrant_RDsubsidy}), $\hat{z}$ increases; and by the labor resource constraint $z$ decreases. If (\ref{cs:growth_misallocation_condition}) is less than 1, this reduces growth. Intuitively, the increased R\&D subsidy reduces the wage expenses paid for R\&D by the same factor $1-\frac{1-T_{RD}^1}{1-T_{RD}^0}$ for both incumbents and entrants. However, the incumbent's effective cost of R\&D also includes the increased likelihood of creative destruction by an employee spinout. Therefore, her effective cost of R\&D is reduced by a factor $\tilde{\tau}_{RD} < 1-\frac{1-T_{RD}^1}{1-T_{RD}^0}$. In general equilibrium, R\&D labor is reallocated to entrants and growth declines.

If the increase in $T_{RD}$ is large enough, $\mathbbm{1}^{NCA}$ changes from $\mathbbm{1}^{NCA} = 0$ to $\mathbbm{1}^{NCA} = 1$ and therefore $\tau^S$ jumps to zero, reducing growth further. Intuitively, higher R\&D subsidies mean the incumbent prefers to pay for the R\&D with wages, which receive a subsidy, rather than implicitly through future spinouts, the cost of which is not subsidized. Incumbents therefore opt to use NCAs, bringing spinout entry to zero and reducing growth by a discrete jump. In addition, there are no indirect effects on growth through changes in $\hat{z}$,$z$, as these variables do not jump: according to (\ref{eq:nca_policy_RDsubsidy}), the transition from $\mathbbm{1}^{NCA} = 0$ to $\mathbbm{1}^{NCA} =1$ occurs at the value of $T_{RD}$ such that $\kappa_c$ is equal to the term multiplying $(1-\mathbbm{1}^{NCA})$, implying that $\hat{z}$, and therefore $z$, does not jump. If $T_{RD}$ is increased even further beyond this point, there is no change in the equilibrium allocation. The only change is the wage of R\&D labor, which by (\ref{eq:wage_rd_labor_RDsubsidy}) increases to equilibriate the R\&D labor market.

Finally, note that all of these effects rely on the assumption that R\&D subsidies are manifested in the wage of R\&D labor rather than the total amount of R\&D labor supply. This is the case when the aggregate supply of R\&D labor is relatively price-inelastic, for example in the medium to short term. Still, in the price-elastic supply case, the present mechanism would mitigate the benefits of R\&D subsidies. It implies that they have less "bang-for-buck" than R\&D subsidies targeted at own-product innovation (see Section \ref{subsubsec:oiRDsubsidy}). 

\subsubsection{Creative destruction tax (subsidy)}

Creative destruction taxes are a natural policy to consider in this framework as the core misallocation in the economy is typically that there is too much creative destruction. Similar policies are often studied in quality ladder models, e.g. \cite{acemoglu_innovation_2015}. In this case, I will find that entry taxes can have a beneficial effect on growth and welfare. 

Suppose that the planner taxes entry at rate $T_e$ (subsidy if $T_e < 0$). Specifically, the planner taxes the entry fixed cost $\kappa_e \lambda \tilde{V} q$ at rate $T_e$ so that a firm entering with quality $\lambda q$ perceives a total cost of $(1+T_e) \kappa_e \lambda \tilde{V}q$ units of the final good. Economically, this can be interpreted as a tax on non-R\&D expenses related to the development of new versions of products currently not sold by the firm in question.\footnote{Because the tax is proportional to these expenses, rather than a fixed tax on entry, it does not induce any reallocation of R\&D towards higher quality goods.}

In this setup, the R\&D labor supply indifference condition becomes
\begin{align}
\hat{w}_{RD} &= w_{RD}(\mathbbm{1}^{NCA}) + (1-\mathbbm{1}^{NCA}) \nu (1-(1+T_e)\kappa_e) \lambda \tilde{V} \label{eq:RD_worker_indifference_entryTax}
\end{align}

Substituting this into the incumbent's HJB and using the same argument as before, this implies that if $z > 0$, the allocation of NCAs is 
\begin{align}
\mathbbm{1}^{NCA} = \begin{cases}
1 & \textrm{if } \kappa_{c} < \hat{\bar{\kappa}}_c  \\
0 & \textrm{if } \kappa_{c} > \hat{\bar{\kappa}}_c \\
\{0,1\} & \textrm{if } \kappa_c = \hat{\bar{\kappa}}_c 
\end{cases} \label{eq:nca_policy_entryTax}
\end{align}

where
\begin{align}
\hat{\bar{\kappa}}_c = \hat{\bar{\kappa}}_c(\kappa_e,\lambda;T_e) = 1 - (1-(1+T_e)\kappa_e)\lambda  \label{eq:barkappa_entryTax}
\end{align}

If $(1 + T_e) \kappa_e > 1$ then $\hat{z} = 0$ and $z = \bar{L}_{RD}$. Otherwise, the free entry condition is now
\begin{align}
\underbrace{\hat{\chi} \hat{z}^{-\psi}}_{\mathclap{\text{Marginal innovation rate}}} \overbrace{(1-(1+T_e)\kappa_e) \lambda \tilde{V}}^{\mathclap{\text{Payoff from innovation}}} &= \underbrace{\hat{w}_{RD}}_{\mathclap{\text{MC of R\&D}}} \label{eq:free_entry_condition_entryTax}
\end{align}

Substituting the incumbent FOC into (\ref{eq:free_entry_condition_entryTax}) to eliminate $\tilde{V}$ yields an expression for $\hat{z}$, 
\begin{align}
\hat{z} &= \Bigg( \frac{\hat{\chi} (1-(1+T_e)\kappa_{e}) \lambda}{\chi(\lambda -1) - \nu (\mathbbm{1}^{NCA}\kappa_c + (1-\mathbbm{1}^{NCA})(1 - (1-(1+T_e)\kappa_e)\lambda)) } \Bigg)^{1/\psi} \label{eq:effort_entrant_entryTax}
\end{align}

From here, the rest of the model (including the case where $(1-\kappa_e)\lambda < 1$ and $\hat{z} = 0$) can be solved in a similar way as before (details in Appendix \ref{appendix:model:efficiencyderivations:CDtax}). 

\paragraph{Effect on growth}

Suppose that $\mathbbm{1}^{NCA} = 1$ and the tax is increased from $T_e$ to $T_e' > T_e$. Then (\ref{eq:effort_entrant_entryTax}) implies that $\hat{z}$ falls, (\ref{eq:labor_resource_constraint_entryTax}) implies that $z$ increase to keep $L_{RD} = \bar{L}_{RD}$. Following the same logic as Section \ref{model:efficiency:misallocationRD}, if (\ref{cs:growth_misallocation_condition}) is less than 1, then growth increases. Intuitively, when $\mathbbm{1}^{NCA} = 1$ the only effect of the entry tax is to reduce the misallocation of R\&D labor to entrants. 

However, if $\mathbbm{1}^{NCA} = 0$, the situation changes, for two reasons. First, as can be seen readily in (\ref{eq:effort_entrant_entryTax}), the effect of $T_e$ on $\hat{z}$ is ambiguous, since the denominator now decreases in $T_e$ as well as the numerator. Intuitively, an increase in $T_e$ reduces the value of future spinouts, requiring incumbents to compensate workers with higher wages in equilibrum. However, the expected harm to incumbents from WSOs per unit of $z$ is unchanged. This follows from the assumption that potential WSOs arise as a by-product of working in R\&D rather than as a result of intentional side projects by R\&D workers. The net effect is that incumbents' effective cost of R\&D increases and R\&D labor is reallocated to the entrant. The mechanism in the previous paragraph is still present, however; the logic here only serves to attenuate the increase in growth from an increase in $T_e$ when (\ref{cs:growth_misallocation_condition}) is less than 1 with $\mathbbm{1}^{NCA} = 0$.

Second, by (\ref{eq:nca_policy_entryTax}) and (\ref{eq:barkappa_entryTax}), a sufficiently large increase in $T_e$ induces a change from $\mathbbm{1}^{NCA} = 0$ to $\mathbbm{1}^{NCA} = 1$. Intuitively, as mentioned in the previous paragraph, a higher $T_e$ means it is relatively more expensive for incumbents to compensate their employees with future spinouts as they are less valuable but cause the same harm to the incumbent. For a high enough $T_e$, incumbents prefer to use NCAs and pay their employees with wages directly. Using the logic of Section \ref{model:efficiency:misallocationNCAs}, this switch implies a reduction in growth.

Overall, entry taxes can be a useful tool to mitigate the overallocation of R\&D to creative destruction. However, they are not a particularly efficient way to do so as they also make employee spinouts (artifically) more bilaterally suboptimal, making them a stronger drag on incumbent R\&D spending when NCAs are too expensive to use, as incumbents receive a smaller wage discount when the tax on entry is higher. For this reason, next I consider R\&D subsidies targeted at incumbent R\&D, which do not have this property.  

\subsubsection{OI R\&D subsidy (tax)}\label{subsubsec:oiRDsubsidy}

Suppose that the plannner can subsidize R\&D spent on improving a product while excluding R\&D aiming at creative destruction. In the model, this corresponds to a targeted subsidy to R\&D spending by incumbents, of magnitude $T_{RD,I}$ (tax if $T_{RD,I} < 0$). In practice, this policy may be difficult to implement for the same reason as the CD tax. Firms may not be expected to be truthful regarding the purpose of their R\&D or the effect of their R\&D on their competitors' profits. It may not even be possible to tell in advance whether R\&D will result in creative CD, OI, or even new varieties of products. Furthermore, innovation to improve existing products can be a form of creative destruction when incumbents compete against each other.\footnote{I return to this when I make suggestions for future work in the conclusion.} Nevertheless, it is still useful as a theoretical benchmark.

In this case, the incumbent HJB can be rearranged to a form analogous to (\ref{eq:hjb_incumbent_workerIndiff}),
\begin{align}
(r + \hat{\tau}) \tilde{V} = \tilde{\pi} + \max_{\substack{\mathbbm{1}^{NCA} \in \{0,1\} \\ z \ge 0}} \Big\{z &\Big( \overbrace{\chi (\lambda - 1) \tilde{V}}^{\mathclap{\mathbb{E}[\textrm{Benefit from R\&D}]}}- (1-T_{RD,I}) \hat{w}_{RD} \\
&-  \underbrace{(1-\mathbbm{1}^{NCA})(1 - (1-T_{RD,I})(1-\kappa_{e})\lambda)\nu \tilde{V}}_{\mathclap{\text{Net cost from spinout formation}}} - \overbrace{\mathbbm{1}^{NCA} \kappa_{c} \nu \tilde{V}}^{\mathclap{\text{Direct cost of NCA}}}\Big) \Big\} \label{eq:hjb_incumbent_RDsubsidyTargeted_2}
\end{align}

The non-compete policy is the same as with untargeted R\&D subsidies. That is, define
\begin{align}
\tilde{\bar{\kappa}}_c = \tilde{\bar{\kappa}}_c(\kappa_e,\lambda;T_{RD}) = 1 - (1-T_{RD,I})(1-\kappa_e)\lambda
\end{align} 

Then $z > 0$ implies that the incumbent's optimal NCA policy is given by 
\begin{align}
\mathbbm{1}^{NCA} = \begin{cases}
1 & \textrm{if } \kappa_{c} < \tilde{\bar{\kappa}}_c  \\
0 & \textrm{if } \kappa_{c} > \tilde{\bar{\kappa}}_c \\
\{0,1\} & \textrm{if } \kappa_c = \tilde{\bar{\kappa}}_c 
\end{cases} \label{eq:nca_policy_RDsubsidyTargeted}
\end{align}

Using the same approach as before one obtains an expression for $\hat{z}$, 
\begin{align}
\hat{z} &= \Bigg( \frac{(1-T_{RD,I})\hat{\chi} (1-\kappa_{e}) \lambda}{\chi(\lambda -1) - \nu (\mathbbm{1}^{NCA} \kappa_c + (1-\mathbbm{1}^{NCA})(1 - (1-T_{RD,I})(1-\kappa_e)\lambda)) } \Bigg)^{1/\psi} \label{eq:effort_entrant_RDsubsidyTargeted}
\end{align}

The rest of the equilibrium allocation and prices can be computed in a similar way as before (details in Appendix \ref{appendix:model:efficiencyderivations:OIRDtax}). 






\paragraph{Effect on growth}

If $\mathbbm{1}^{NCA} = 1$, increasing $T_{RD,I}$ reduces $\hat{z}$ by (\ref{eq:effort_entrant_RDsubsidyTargeted}) and will increase growth if the (\ref{cs:growth_misallocation_condition}) is less than 1. Intuitively, a subsidy to incumbent R\&D causes the R\&D wage to increase, reducing R\&D by the entrant in equilibrium. 

If $\mathbbm{1}^{NCA} = 0$, increasing $T_{RD,I}$ has a more complicated effect on $\hat{z}$ because it reduces the denominator as well. This follows from the same reasoning as in the case of the untargeted R\&D subsidy: incumbents pay partially through future spinouts and so not all of their costs are subsidized at rate $T_{RD,I}$. From this economic interpretation, it follows immediately that the net effect is still to reduce incumbent R\&D expenses relative to those of the entrant and hence to lower $\hat{z}$ and increase $z$, and this is confirmed in the numerical analysis of the next subsection.

Finally, (\ref{eq:nca_policy_RDsubsidyTargeted}) implies that if the increase in $T_{RD,I}$ is sufficiently large, it will induce the use of NCAs by incumbents. As in the case of the untargeted R\&D subsidy, targeted R\&D subsidies do not reduce the harm to the incumbent's profits due to future employee spinouts. At a certain point, the incumbent prefers the higher but tax-deductible wages of an NCA contract. This switch unambiguously reduces growth.

The last observation implies that even targeted R\&D subsidies are unable to achieve the socially valuable outcome high spinout entry and high incumbent R\&D. In order to achieve this result, it is necessary to pair the targeted R\&D subsidy with an increase in legal barriers to NCAs $\kappa_c$ or an increase in the tax on NCA usage $T_{NCA}$. 

\subsubsection{All policies}

The BGP of the model when the planner can use all of the above policies simultaneously is derived in Appendix \ref{appendix:model:efficiencyderivations:allPolicies}. I discuss the growth and welfare implications of this case in detail in the quantitative analysis of Section \ref{sec:policy_analysis}. As a prelude, notice that with a combination of OI R\&D subsidies and an increase in $\kappa_c$, the social planner can achieve something like a "first best"\footnote{As noted previously, there is no well-defined first-best here. In fact, there is no well-defined second-best either because there is no well-defined equilibrium R\&D wage in a model where the social planner makes all R\&D decisions, and hence there is no well-defined value of the incumbent, which is necessary for computing consumption. Again, the model can easily be modified to allow this type of analysis, but the current one is more transparent.} where incumbents do the socially optimal amount of R\&D while still allowing for growth-enhancing spinouts to innovate.


\section{Empirics}\label{sec:empirics}

In this section I perform some empirical analyses that will form part of the calibration in the Section \ref{sec:calibration}. I analyze the microeconomic relationship between firm-level R\&D spending and subsequent employee spinout formation using a dataset I construct out of Venture Source, Compustat, and the NBER-USPTO patent database.\footnote{The same kind of matching (Venture Source and Compustat) was previously done by \cite{gompers_entrepreneurial_2005} using data through 1999.} Using a regression analysis, I find a statistically significant and economically large relationship between R\&D spending and employee spinouts in the next few years. This holds even after controlling for firm, industry-year, and state-year fixed effects, as well as time-varying firm controls, discussed in detail below. I then use these results to quantify the economic magnitude of R\&D-induced spinout formation. In the calibration, this disciplines the parameter $\nu$, which in turn determines the magnitude of the effect on growth of a reduction in $\kappa_c$. 

\subsection{Data}

\subsubsection{Sources}

\paragraph{VentureSource}

The data on startups comes from Venture Source (VS), a proprietary dataset containing information on venture capital (VC) firms and VC-funded startups.\footnote{When starting this project the data were owned by Dow Jones but they have since been sold to CB insights.} I use a subsample of the data for US-based startups founded between 1986 and 2008 which contain information on their founding year. The data cover 23,434 startups, 89,382 financing rounds, and 297,119 individual-firm pairs. For each financing round, the data contain information on valuation, amount raised, and status of the business at the time of the round -- employment, revenue, net income, burn rate -- albeit with substantial missing data. Most importantly for this analysis, the data contain employment biographies for each of the startup's founders and key employees (C-level, high-ranking executives and managers) and board members. In this regard, Venture Source is unique among VC investment databases. Some summary information about the dataset is contained in \autoref{table:VS_summaryTable}. The dataset is described in detail in \cite{kaplan_how_2002} and \cite{kaplan_venture_2016}. 

\paragraph{Compustat}

The data on R\&D spending comes from Compustat, a comprehensive database of fundamental financial and market information on publicly traded companies. I consider a subsample consisting of all firms headquartered in the United States in operation at any point between 1986 and 2006, consisting of 20,534 firms. In addition to data on R\&D spending, the Compustat data contain information on industrial classification and time-varying firm characteristics and balance sheet information.

\paragraph{NBER-USPTO}

The NBER-USPTO database contains comprehensive information on all patents granted in the United States from 1976 to 2006, and is linked to Compustat. I consider the subsample of patents assigned to US firms, consisting of 1,457,136 patents. 

\subsubsection{Construction of dataset}

\paragraph{Classifying founders}

The Venture Source data contain information on high level employees and board members. For the purposes of this study, however, not all of these employees should be considered founders of the startup in question. In particular, only those employees whose human capital is crucial to the value proposition of the startup should be considered founders. To this end, I restrict attention to employees who (1) have a job title related to the core operations of the firm (Founder, Chief, CEO, CTO or President) and (2) join the startup in its first three years since its founding date. When information on the individual's date of joining the startup is missing, I impute it as the founding date of the startup. \autoref{table:VS_founder2_titlesSummaryTable} shows a breakdown of the most frequent titles. 

% latex table generated in R 3.6.3 by xtable 1.8-4 package
% Wed Sep  2 15:53:15 2020
\begin{table}[]
\centering
\begingroup\normalsize
\begin{tabular}{rll}
  \toprule
Title & Individuals & Percentage \\ 
  \midrule
Chief Executive Officer & 10306 & 24.8 \\ 
  Chief Technology Officer & 8036 & 19.3 \\ 
  President \& CEO & 7806 & 18.8 \\ 
  Chief & 5400 & 13.0 \\ 
  President & 3969 & 9.5 \\ 
  Founder & 2634 & 6.3 \\ 
  Chairman \& CEO & 2385 & 5.7 \\ 
  President \& COO & 961 & 2.3 \\ 
  President \& Chairman & 104 & 0.2 \\ 
   \bottomrule
\end{tabular}
\endgroup
\caption{Most frequent titles among key founders in VS data.} 
\label{table:VS_founder2_titlesSummaryTable}
\end{table}


\paragraph{Finding previous employers}

In order to relate the activities of employers to the entrepreneurship behavior of their employees, I extract information on previous employment using VS biographical data.\footnote{Because VS biographies are text fields, this requires some cleaning. The VS biographical data comes in a structured format, allowing parsing by regular expressions. Each prior job is represented in the format ``<position>, <employer>'' and different jobs are separated by ``;''. Job spells can be easily separated by splitting the string on the character ``;''. It is slightly more involved to separate positions from employer. It is not sufficient to simply separate on the right-most character ``,'' as <employer> can contain ``,''. However, in almost all cases, <employer> contains at most one ``,'' (e.g., in ``Microsoft, Inc.''), and in virtually all of these cases, the comma precedes one of a few strings (e.g. ``LLC'',``Inc'',``Corp''). Hence, I use a two-pass approach: first I split on the last ``,''; for employers that end up consisting only of corporate structure (e.g., ``LLC'', etc.), I split on the penultimate ``,'' instead. } This yields a dataset containing, for each individual, all of his or her previous positions and employers.\footnote{However, because an individual can take various jobs over the years at an individual startup, there are individuals whose most recent employer coincides with their startup. I exclude these cases by comparing the previous employer with the \texttt{EntityName} text field. Because these are both text fields with potentially different formatting, this entails two steps. First, I bring both fields to a common format, eliminating endings such as ``Inc.'', ``Corp.'' etc which may vary across them, and converting to lower case. I then exclude observations where the strings either exactly coincide or one contains the other.} The results of this procedure are summarized in \autoref{table:VS_previousEmployersNoPositionsSummaryTable}. The top twenty previous employers include several well-known technology firms such as Google, Microsoft, and IBM.\footnote{I also have data on past positions, but displaying them in this way is misleading as certain positions are more common simply due to there being fewer different strings associated with that position. For instance, "Executive" is the most common position, even though there are many positions that \textit{involve} the word "engineering" they barely appear in the top 20 because they all are slightly different. Need to cluster.}

% latex table generated in R 3.6.3 by xtable 1.8-4 package
% Wed Nov 25 14:00:10 2020
\begin{table}[]
\centering
\begingroup\normalsize
\caption{Top 20 previous employers for founders in VS data.} 
\label{table:VS_previousEmployersNoPositionsSummaryTable}
\begin{tabular}{rlrl}
  \toprule
Employer & Count & Employer & Count \\ 
  \midrule
IBM & 174 & Stanford University & 56 \\ 
  Microsoft & 168 & Lucent Technologies & 46 \\ 
  Cisco Systems & 122 & AOL & 44 \\ 
  Oracle & 109 & Motorola & 42 \\ 
  Verizon & 101 & Andersen Consulting & 40 \\ 
  Sun Microsystems & 94 & Nortel Networks & 40 \\ 
  Google & 79 & MIT & 40 \\ 
  AT\&T & 76 & McKinsey \& Company & 39 \\ 
  Intel & 70 & Texas Instruments & 38 \\ 
  Hewlett-Packard & 64 & Apple & 37 \\ 
   \bottomrule
\end{tabular}
\endgroup

\end{table}


\paragraph{Linking to Compustat}

The data on each founder's previous employer is matched to the firm name variable \texttt{conml} in Compustat. As this is string matching, it requires standardizing names as before, using regular expressions to trim e.g. ``Inc.", ``Corp.'' and variants thereof from each entry and converting to lower case. I look for exact matches to previous employers in the VS data. For previous employers in VS that do not match with any names in Compustat, I check against the business segment names, available from the Compustat Segments database.

\paragraph{Defining WSOs}

This study emphasizes the importance of competition between spinouts and their parent firms. The best measure I have for the product market of publicly traded firms is their self reported NAICS code. While VS does not contain NAICS classifications for its startups, it does document their industry using a classification that, for the most part, coincides (at least in name) with NAICS 4 or 5 digit categories. I manually construct a crosswalk between the two classification schemes and use this to assign 4-digit NAICS codes to startups in VS.\footnote{An alternative would be to us VS's "Competition" variable, which documents directly the competitors of the startup observation. However, only 20\% of startups have this variable filled in: 30\% in the 90s, but dropping to around 10\% by the end of the sample.} Then, I classify a founder-startup observation as a WSO whenever the startup is in the same 4-digit NAICS category as its parent. 

\paragraph{Evaluating the match}

\autoref{table:GStable_founder2} documents the quality of this match. It corresponds roughly to Table 1 of \cite{gompers_entrepreneurial_2005} \footnote{Working on a replication of their table. Generally my algorithm gets fewer matches early on and more matches later in their sample period; but always a smaller fraction.}. About 20\% of startup founders have a most recent previous employer that matches to a public firm in Compustat. Nearly half of those came from an employer in the same four digit NAICS industry as the startup. The remainder were either most recently employed at a firm that is not publicly traded, and so is excluded from Compustat, or they are missed by the matching algorithm. This is likely due to (1) significantly different spellings or naming conventions between VS and Compustat, or (2) they worked at a subsidiary of a publicly traded firm, but the subsidiary is not named in the Compustat Segments database.\footnote{Further improving the quality of the match would likely require manually matching and is a potentially fruitful area for further work.} 

% latex table generated in R 3.6.3 by xtable 1.8-4 package
% Sat Sep 26 15:59:48 2020
\begin{sidewaystable}[!htb]
\centering
\begingroup\tiny
\begin{tabular}{p{1.75cm}p{1.75cm}p{1.75cm}p{1.75cm}p{1.75cm}p{1.75cm}p{1.75cm}p{1.75cm}}
  \toprule
Year & Number of founders & Number of start-ups & Number of founders from public companies & Fraction from public companies (\%) & Fraction from public companies when bio. info available (\%) & Fraction from public companies in same 4-digit NAICS (\%) & Fraction from public companies in same 4-digit NAICS when bio. info available (\%) \\ 
  \midrule
1986 & 269 & 216 & 45 & 16.7 & 22.8 & 5.2 & 7.1 \\ 
  1987 & 356 & 280 & 43 & 12.1 & 15.1 & 3.9 & 4.9 \\ 
  1988 & 372 & 281 & 58 & 15.6 & 19.9 & 4.6 & 5.8 \\ 
  1989 & 479 & 341 & 75 & 15.7 & 19.2 & 4.2 & 5.1 \\ 
  1990 & 478 & 329 & 85 & 17.8 & 21.1 & 6.3 & 7.5 \\ 
  1991 & 540 & 356 & 81 & 15.0 & 17.9 & 6.3 & 7.5 \\ 
  1992 & 674 & 450 & 100 & 14.8 & 17.9 & 3.3 & 3.9 \\ 
  1993 & 778 & 490 & 137 & 17.6 & 20.3 & 6.7 & 7.7 \\ 
  1994 & 999 & 611 & 167 & 16.7 & 19.3 & 4.9 & 5.7 \\ 
  1995 & 1326 & 772 & 224 & 16.9 & 19.0 & 5.2 & 5.8 \\ 
  1996 & 1926 & 1077 & 319 & 16.6 & 18.1 & 4.9 & 5.3 \\ 
  1997 & 1986 & 1036 & 345 & 17.4 & 19.0 & 5.9 & 6.5 \\ 
  1998 & 2895 & 1390 & 541 & 18.7 & 19.6 & 5.2 & 5.5 \\ 
  1999 & 5189 & 2388 & 975 & 18.8 & 19.6 & 5.0 & 5.2 \\ 
  2000 & 4084 & 1832 & 786 & 19.2 & 20.4 & 5.2 & 5.5 \\ 
  2001 & 2245 & 948 & 384 & 17.1 & 18.7 & 6.3 & 6.9 \\ 
  2002 & 2113 & 884 & 385 & 18.2 & 20.1 & 7.3 & 8.0 \\ 
  2003 & 1979 & 903 & 344 & 17.4 & 19.8 & 7.5 & 8.5 \\ 
  2004 & 2098 & 988 & 365 & 17.4 & 20.1 & 6.8 & 7.9 \\ 
  2005 & 2278 & 1068 & 400 & 17.6 & 20.7 & 6.5 & 7.7 \\ 
  2006 & 2492 & 1212 & 432 & 17.3 & 20.5 & 6.3 & 7.5 \\ 
  2007 & 2817 & 1366 & 388 & 13.8 & 17.0 & 4.9 & 6.1 \\ 
  2008 & 2710 & 1307 & 422 & 15.6 & 19.1 & 5.4 & 6.6 \\ 
   \bottomrule
\end{tabular}
\endgroup
\caption{\scriptsize Summary of founders. Here, "founder" includes all individuals employed at startups inthe VentureSource database who (1) joined the startup within 3 year(s) of its founding year; and (2) have the title of CEO, CTO, CCEO, PCEO, PRE, PCHM, PCOO, FDR, CHF.} 
\label{table:GStable_founder2}
\end{sidewaystable}


\paragraph{Prevalence of WSOs} 

To get a sense of the importance of within-industry spinouts in the data, \autoref{figure:industry_row_heatmap_naics2_founder2} documents the joint distribution of parent industry and child industry, defined by 2-digit NAICS codes for easier visualization even though I use 4-digit NAICS industries to define WSOs. The raw joint distribution is too heavily concentrated to be easily visualized in this way, so instead I show the distribution of child industry (column) conditional on parent industry (row). The dark diagonal line reflects the prevalence of WSOs.\footnote{\autoref{figure:industry_column_heatmap_naics2_founder2} shows the joint distribution the other way around, with the probability of parent industry conditional on child industry.}\footnote{In \autoref{figure:industry_row_heatmap_naics2_founder2}, the dark vertical line at column 51 (Information) indicates that parent firms of all industries tend to spawn spinouts in that industry. Similar dark regions appear at columns 54 (Professional, Scientific and Technical Services), and 32 and 33 (Manufacturing). In \autoref{figure:industry_column_heatmap_naics2_founder2}, the dark horizontal lines at 51 and to a lesser extend 32, 33, 52 and 54 indicate that child firms of all industries tend to have founders from those industries.}

\begin{figure}[]
	\centering
	\includegraphics[scale=0.65]{../empirics/figures/plots/industry_row_heatmap_naics2_founder2.pdf}
	\caption{Heatmap displaying the distribution of child 2-digit NAICS code (column), conditional on parent NAICS code (row). Darker hues indicate a higher density.}
	\label{figure:industry_row_heatmap_naics2_founder2}
\end{figure}

\subsection{Corporate R\&D and spinout formation}\label{subsec:empirics:corpRDandspinouts}

\subsubsection{Preliminaries}

\begin{figure}[]
	\centering
	\includegraphics[scale= 0.5]{../empirics/figures/scatterPlot_RD-FoundersWSO4_dIntersection.pdf}
	\caption{Scatterplot of average yearly founder counts (restricted to same 4-digit NAICS industry) in $t+1,t+2,t+3$ versus average yearly R\&D spending in $t,t-1,t-2$.}
	\label{figure:scatterPlot_RD-FoundersWSO4_dIntersection2}
\end{figure}

\autoref{figure:scatterPlot_RD-FoundersWSO4_dIntersection2} reproduces the scatterplot from the introduction. It shows the relationship between R\&D spending in years $t-2,t-1,t$ and employee entrepreneurship in years $t+1,t+2,t+3$.\footnote{Nominal R\&D spending is first deflated using the R\&D investment deflator, and then further deflated by productivity growth (all using 2014 as a reference year, without loss of generality). The reason for the latter is that the model specifies that, as per-capita output increases, proportionally more real units of R\&D are required for each innovation and to generate each spinout (hence the constant ratio of real units of R\&D to real units of consumption)}. Both variables are sequentially purged of their firm-level and then state-industry-age-year means. 

The solid line shows the best fit of a straight line through all of the resulting points. It indicates the presence of a noisy but positive relationship between R\&D deviations and founder departure deviations. However, this relationship could suffer from omitted variable bias. Time-varying firm-level observables that are not constant at the state-industry-age-year level could be associated with both R\&D spending and spinout formation in both positive and negative ways. For example, investment opportunities specific to the firm's technological niche within its state-industry-age-year would induce R\&D spending and potentially within-industry spinout formation by its employees pursuing related ideas. On the other hand, poor management or other tribulations at the parent firm could reduce R\&D spending and simultaneously induce employee departures. To control for these factors, as well as to properly assess statistical significance, I conduct three regression analyses in the next section.

\subsubsection{Regressions}

In this section I describe three regression specifications of employee spinout formation on R\&D. All three regressions include time-varying firm characteristics as controls (employment, cumulative patents, assets, intangible assets, net income, capital expenditures, and Tobin's Q). In addition, I include varying fixed effects at the firm, state-year, industry-year, and / or age of the firm,\footnote{Age of the firm is the time since the firm started reporting data in Compustat.} and cluster standard errors by firm or, in the most stringent regression, by state and industry simultaneously. The latter computes standard errors which are robust to an arbitrary correlation structure of regression errors within each state or 4-digit industry.\footnote{Interestingly, the standard errors with the more conservative clustering assumptions are often smaller. In those cases it means that there is actually negative correlation within each cluster.} Based on these regressions, I find evidence of an economically meaningful relationship between corporate R\&D and subsequent employee spinout formation.\footnote{I do not pursue an IV identification strategy because there are problems with the usual instrument for firm-level R\&D spending. Broadly speaking, it is difficult to alter economic conditions in a way that affects incumbent firm behavior without potentially affecting startups, hence spinouts, in a similar way. This means exogenous shifters of R\&D spending typically fail to satisfy the exclusion restriction. The resulting estimated relationship could be biased. The reason is that the standard instrument for firm-level R\&D spending is based on tax incentives for R\&D, as developed in \cite{bloom_identifying_2013}. Most of the variation in this instrument is based on state-level variables, which would also simultaneously affect the incentive for spinouts (most of which are in the same state as the parent firm) to conduct R\&D, either positively because they have access to R\&D subsidies, or negatively because the firm at which they work has the ability to pay them a higher wage for the same project due to increased R\&D subsidies. This would violate the exclusion restriction. In a working paper, \cite{babina_entrepreneurial_2019} argue that the exclusion restriction may be satisfied -- because R\&D tax incentives only reduce corporate taxes on profits, which startups don't have yet -- and use an IV approach based on these instruments. They find that the magnitude of the relationship between R\&D and spinout formation is five times larger in their IV estimation than when using OLS estimation. They conclude that the IV captures a local average treatment effect (LATE) that differs substantially from the ATE, which is what I am interested in here; and that they ultimately find their OLS estimates more reliable. Two other interpretations are that (1) the exclusion restriction is in fact violated (e.g., startups can carry forward their tax breaks) and (2) the omitted variable bias in the OLS regression leads to a downward, rather than an upward bias. All three interpretations suggest that IV estimation may not appropriate in this case.}

\paragraph{OLS in levels}

\autoref{table:RDandSpinoutFormation_absolute_founder2_l3f3} displays the results of a regression analysis relating employee entrepreneurship to parent firm R\&D spending. The dependent variable $Y_{it}$ is again the (annualized) number of founders previously employed at firm $i$ joining startups in years $t+1,t+2,t+3$. The independent variables $X_{it}$ are moving averages over years $t,t-1,t-2$. Moving from left to right adds more stringent fixed effects and clustering assumptions. These regressions find a positive coefficient which is statistically significant at the 1\% level, even after including firm and year fixed effects. The magnitude of the coefficient means that three billion 2014 dollars of R\&D over three years leads to an average of one employee leaving to found a startup in the same industry over the following three years.

The robustness of the result to the inclusion of age, industry-year, and State-year fixed effects is encouraging. However, in this context such fixed effects may not absorb many time-varying shocks due to what amounts to a misspecification problem. Precisely, while shocks (at the industry-year, state-year, or age level) are likely to affect firm outcomes more in absolute terms for larger firms, the fixed effects are restricted to be of the same absolute magnitude for a given industry-year, state-year or firm age. Because firms actually vary substantially in size -- even within industry-year, state-year, or age group -- this inevitably leaves much firm-level variation. This could induce a spurious coefficient estimate on R\&D spending.\footnote{This is the reason I control for market value in the levels regression rather than Tobin's Q}. 

The standard check in this case is to consider either (1) normalizing firm variables by a "firm size" variable, such as assets; or (2) performing a log-linear or Poisson regression.\footnote{This is roughly equivalent to a log-linear model except it can handle zeros in the dependent variable.} I consider these specifications in the next two sections.

\paragraph{OLS in intensities}

\autoref{table:RDandSpinoutFormation_at_founder2_l3f3} displays the results of a regression analysis where all variables are normalized by a trailing 5-year moving average of total firm assets. Under the specification of the model, the coefficients should be the same in either case. Broadly speaking this appears to be the case. Given the standard errors in both cases, it is unlikely that the estimates are statistically significantly different in the two specifications. Note that normalizing by assets throws away any variation in absolute firm levels of R\&D, obtaining identification only from variation in the ratio of R\&D to assets.\footnote{However, if assets play a role directly in generating spinouts -- say, those which are not used in R\&D -- which } And with the most conservative clustering assumptions (column 4), the estimate is significant at the 5\% level.

\paragraph{Poisson pseudo-Maximum Likelihood}

Finally \autoref{table:RDandSpinoutFormation_ppml_absolute_founder2_l3f3} shows the results of a Poisson pseudo-Maximum Likelihood (PPML) estimation. This can be thought of as a log-linear regression which can handle zeros in the dependent variable. It is also the specification in the model, as the rate of employee departures is assumed to depend on R\&D spending via a Poisson process.\footnote{However, even if the conditional distribution of spinout counts is actually not Poisson, a PPML regression consistently estimates the conditional mean} In this case, the coefficient on R\&D spending corresponds to an elasticity. The regression estimate is both statistically insignificant and statistically indistinguishable from 1. This is encouraging as a unit elasticity is the assumption in the model and in the previous two OLS specifications. \footnote{In the regression without fixed effects, the sample size is much smaller due to the fact that the specification now must take logarithms of RHS variables (this is the correct specification in the model). When fixed effects are added, the sample is additionally restricted because many of the induced categories have only one observation with non-missing data (due to the log specification).} 

\paragraph{Conclusion from regression analysis}

Based on the regression analysis above, I conclude that the data are consistent with an effect of corporate R\&D spending on employee spinout formation in the same 4-digit NAICS industry. While the absence of truly exogenous variation in R\&D spending prevents any final judgments, the preceding analysis justifies considering the implications of interpreting the previous results causally, which is the implicit assumption of the calibration. In the next section, I consider the economic magnitude of the OLS estimates just calculated.


\begin{table}[]
	\centering
	{
\def\sym#1{\ifmmode^{#1}\else\(^{#1}\)\fi}
\begin{tabular}{l*{4}{c}}
\toprule
                    &\multicolumn{1}{c}{(1)}&\multicolumn{1}{c}{(2)}&\multicolumn{1}{c}{(3)}&\multicolumn{1}{c}{(4)}\\
                    &\multicolumn{1}{c}{WSO4}&\multicolumn{1}{c}{WSO4}&\multicolumn{1}{c}{WSO4}&\multicolumn{1}{c}{WSO4}\\
\midrule
R\&D                &        0.18\sym{***}&        0.31\sym{***}&        0.30\sym{***}&        0.30\sym{***}\\
                    &     (0.040)         &     (0.063)         &     (0.061)         &     (0.070)         \\
\addlinespace
No FE               &         Yes         &          No         &          No         &          No         \\
\addlinespace
Firm FE             &          No         &         Yes         &         Yes         &         Yes         \\
\addlinespace
Year FE             &          No         &         Yes         &          No         &          No         \\
\addlinespace
Age FE              &          No         &          No         &         Yes         &         Yes         \\
\addlinespace
Industry-Year FE    &          No         &          No         &         Yes         &         Yes         \\
\addlinespace
State-Year FE       &          No         &          No         &         Yes         &         Yes         \\
\midrule
Clustering          &       gvkey         &       gvkey         &       gvkey         &naics4 Statecode         \\
R-squared (adj.)    &        0.24         &        0.65         &        0.63         &        0.63         \\
R-squared (within, adj)&        0.24         &        0.26         &        0.24         &        0.24         \\
Observations        &       60697         &       59485         &       57956         &       57956         \\
\bottomrule
\multicolumn{5}{l}{\footnotesize Standard errors in parentheses}\\
\multicolumn{5}{l}{\footnotesize \sym{*} \(p<0.1\), \sym{**} \(p<0.05\), \sym{***} \(p<0.01\)}\\
\end{tabular}
}

	\caption{The regressions above relate corporate R\&D to the entrepreneurship decisions of employees. The dependent variable is average yearly number of founders joining startups in years $t+1,t+2,t+3$. The independent variables are averages over $t,t-1,t-2$. Firm controls are employment, assets, intangible assets, investment, net income, cumulative citation-weighted patents, and the product of Tobin's Q and Assets (i.e., firm market value).}
	\label{table:RDandSpinoutFormation_absolute_founder2_l3f3}
\end{table}

\begin{table}[]
	\centering
	{
\def\sym#1{\ifmmode^{#1}\else\(^{#1}\)\fi}
\begin{tabular}{l*{4}{c}}
\toprule
                    &\multicolumn{1}{c}{(1)}&\multicolumn{1}{c}{(2)}&\multicolumn{1}{c}{(3)}&\multicolumn{1}{c}{(4)}\\
                    &\multicolumn{1}{c}{$\frac{\textrm{WSO4}}{\textrm{Assets}}$}&\multicolumn{1}{c}{$\frac{\textrm{WSO4}}{\textrm{Assets}}$}&\multicolumn{1}{c}{$\frac{\textrm{WSO4}}{\textrm{Assets}}$}&\multicolumn{1}{c}{$\frac{\textrm{WSO4}}{\textrm{Assets}}$}\\
\midrule
$\frac{\textrm{R\&D}}{\textrm{Assets}}$&        0.94\sym{***}&        0.52\sym{++} &        0.49\sym{+}  &        0.49\sym{**} \\
                    &      (0.17)         &      (0.32)         &      (0.36)         &      (0.21)         \\
\addlinespace
No FE               &         Yes         &          No         &          No         &          No         \\
\addlinespace
Firm FE             &          No         &         Yes         &         Yes         &         Yes         \\
\addlinespace
Year FE             &          No         &         Yes         &          No         &          No         \\
\addlinespace
Age FE              &          No         &          No         &         Yes         &         Yes         \\
\addlinespace
Industry-Year FE    &          No         &          No         &         Yes         &         Yes         \\
\addlinespace
State-Year FE       &          No         &          No         &         Yes         &         Yes         \\
\midrule
Clustering          &       gvkey         &       gvkey         &       gvkey         &naics4 Statecode         \\
R-squared (adj.)    &       0.010         &        0.27         &        0.22         &        0.22         \\
R-squared (within, adj)&       0.010         &     0.00088         &     0.00084         &     0.00084         \\
Observations        &       60687         &       59477         &       57948         &       57948         \\
\bottomrule
\multicolumn{5}{l}{\footnotesize Standard errors in parentheses}\\
\multicolumn{5}{l}{\footnotesize \sym{+} \(p<0.2\), \sym{++} \(p<0.15\), \sym{*} \(p<0.1\), \sym{**} \(p<0.05\), \sym{***} \(p<0.01\)}\\
\end{tabular}
}

	\caption{The regressions above relate corporate R\&D to the entrepreneurship decisions of employees. The dependent variable is the average yearly number of founders from the parent firm joining startups in years $t+1,t+2,t+3$, normalized by a trailing five-year moving average of assets. Independent variables are also normalized by assets.}
	\label{table:RDandSpinoutFormation_at_founder2_l3f3}
\end{table}

\begin{table}[]
	\centering
	{
\def\sym#1{\ifmmode^{#1}\else\(^{#1}\)\fi}
\begin{tabular}{l*{4}{c}}
\toprule
                    &\multicolumn{1}{c}{(1)}&\multicolumn{1}{c}{(2)}&\multicolumn{1}{c}{(3)}&\multicolumn{1}{c}{(4)}\\
                    &\multicolumn{1}{c}{WSO4}&\multicolumn{1}{c}{WSO4}&\multicolumn{1}{c}{WSO4}&\multicolumn{1}{c}{WSO4}\\
\midrule
log(R\&D)           &        1.52\sym{***}&        0.48\sym{**} &        1.29\sym{***}&        1.29\sym{**} \\
                    &      (0.13)         &      (0.20)         &      (0.46)         &      (0.53)         \\
\addlinespace
No FE               &         Yes         &          No         &          No         &          No         \\
\addlinespace
Firm FE             &          No         &         Yes         &         Yes         &         Yes         \\
\addlinespace
Year FE             &          No         &         Yes         &          No         &          No         \\
\addlinespace
Age FE              &          No         &          No         &         Yes         &         Yes         \\
\addlinespace
Industry-Year FE    &          No         &          No         &         Yes         &         Yes         \\
\addlinespace
State-Year FE       &          No         &          No         &         Yes         &         Yes         \\
\midrule
Clustering          &       gvkey         &       gvkey         &       gvkey         &naics4 Statecode         \\
pseudo R-squared    &        0.46         &        0.37         &        0.36         &        0.36         \\
Observations        &        7108         &         820         &         379         &         379         \\
\bottomrule
\multicolumn{5}{l}{\footnotesize Standard errors in parentheses}\\
\multicolumn{5}{l}{\footnotesize \sym{*} \(p<0.1\), \sym{**} \(p<0.05\), \sym{***} \(p<0.01\)}\\
\end{tabular}
}

	\caption{Poisson pseudo-Maximum Likelihood Regression. The dependent variable is average yearly number of founders joining startups in years $t+1,t+2,t+3$. The independent variables are in log terms and averages over $t,t-1,t-2$. Firm controls are (the log of) employment, assets, intangible assets, investment, net income, cumulative citation-weighted patents, and (the level of) Tobin's Q}
	\label{table:RDandSpinoutFormation_ppml_absolute_founder2_l3f3}
\end{table}

\subsubsection{Economic magnitude}

In each year $t$, I compute $\tilde{y}_{it}$, the expected number of founders per year starting firms in years $t+1,t+2,t+2$ by multiplying the R\&D in years $t,t-1,t-2$ by the average of the coefficient estimates from the first two regressions. I then plot this against the realizations of $y_{it}$.  \autoref{figure:founder2_founders_f3_Accounting} provides a visualization of the economic magnitude of the coefficient estimates. Based on this measure, the regression estimates are economically significant, conservatively accounting for about 75\% of the WSO spinouts observed in the data.\footnote{\autoref{figure:founder2_founders_f3_Accounting_industryYear} shows a similar accounting exercise using data at the industry-year level. focusing on NAICS industries 3 (manufacturing) and 5 (information), which are responsible for the vast majority of private sector R\&D spending. The pattern is clearly different by industry, with a very good fit for manufacturing industries but underestimated spinouts in information industries. In future work, I plan to study this question in a way that allows for variation across industries in spinout and noncompete-relevant parameters.} Further, Appendix tables \ref{table:startupLifeCycle_founder2founders_lemployeecount_founder2}, \ref{table:startupLifeCycle_founder2founders_lrevenue_founder2}, and \ref{table:startupLifeCycle_founder2founders_lpostvalusd_founder2} document that startups with a higher fraction of WSO4 founders tend to have roughly 35\% higher employee count, 45\% higher revenue, and 36\% higher valuation on a per-founder basis. This relationship holds after controlling for industry, state, time, cohort, and / or age factors,\footnote{I only control for two of the three (time, cohort, age) factors, to avoid the well-known multicollinearity problem.} and is statistically significant and robust across specifications. Combining these last two observations with the fact that about 8\% of startup founders are at WSOs suggests that R\&D-induced WSOs account for about 8\% of employment, revenue and valuation of startups in the dataset.\footnote{In Section \ref{sec:policy_analysis}, I conduct robustness checks with respect to the result of this calculation. For small changes, the magnitude of the main result (in percentage points consumption-equivalent welfare improvement) is roughly linear in the percentage just calculated.} 

\begin{figure}
	\centering
	\includegraphics[scale=0.8]{../empirics/figures/founder2_founders_wso4_f3_Accounting_noShares.pdf}
	\caption{Economic magnitude of regression estimates in Tables \ref{table:RDandSpinoutFormation_absolute_founder2_l3f3} and \ref{table:RDandSpinoutFormation_at_founder2_l3f3} (i.e. using the average value of the coefficient on R\&D in the fourth column). The figure compares the predictions of the regression for each year of R\&D ((based only on the R\&D spending, i.e. not other controls or fixed effects) to the number of WSO4 spinout founders in that year.}
	\label{figure:founder2_founders_f3_Accounting}
\end{figure}


\section{Calibration}\label{sec:calibration}

\subsection{Parameters}

The model has ten parameters, $\{\rho, \theta, \beta, \psi, \lambda, \chi, \hat{\chi}, \kappa_e, \kappa_c, \nu, \bar{L}_{RD}\}$. The two parameters $\theta, \psi$ are set externally. The elasticity parameter $\theta$ is set to 2, corresponding to a standard value used in the the literature. The entrant R\&D curvature parameter $\psi$ is set to a value of 0.5, which corresponds to a cost that scales quadratically with the innovation rate.\footnote{I consider robustness of the main result to the value of these and all other parameters in Section \ref{appendix:policyanalysis:ncacost}.}

The remaining eight parameters are calibrated. The parameter $\bar{L}_{RD}$ is calibrated to data on the share of employment in R\&D, from the NSF.\footnote{The parameter $\bar{L}_{RD}$ is only a normalization (a choice of units in which to measure the R\&D labor endowment) when it comes to determining the growth rate in the model; however, it does affect the model's implications for the employment share in firms of different ages. As this moment will be a target in the calibration, it is necessary to calibrate $\bar{L}_{RD}$.} The parameter $\beta$ is set to match the profit / GDP ratio. The final six parameters $\{\rho, \lambda, \chi, \hat{\chi}, \kappa_e, \kappa_c, \nu\}$ pertain to preferences ($\rho$) and to the innovation technolgoy and NCA usage (all others), and are chosen to match six moments from the data. One parameter, $\kappa_c$, is partially identified as $\kappa_C > \bar{\kappa}_c$ by the observation that $\tau^S > 0$. The remaining six parameters are exactly identified and the model reproduces the target moments exactly. I discuss the sources of identification in Section \ref{subsec:identification}. 

\subsection{Targets}

The targets of the calibration are displayed in \autoref{calibration_targets}. They consist of the labor productivity growth rate due to creative destruction and own-product innovation, the R\&D / GDP ratio, the real return on the corporate sector, the share of growth coming from older firms improving their own products, the employment share of new firms engaging in creative destruction, and the employment share of R\&D-induced WSOs. 

Matching the productivity growth rate, R\&D / GDP ratio and growth share of OI helps calibrate the efficiency of R\&D in generating aggregate productivity growth through OI and CD. The real return, profit / GDP ratio and employment share of entering firms determines the discount factor and the reward to innovation. Further, the employment share of young firms helps identify the size of each innovation: at a constant contribution to aggregate growth, a lower share of employment at entering firms implies each innovation by an entering firm is smaller. Finally, matching the employment share of entering WSOs allows the model to capture the rate at which R\&D by incumbents increases their likelihood of being replaced by a WSO.

Below, I discuss issues pertaining to the measurement of the target moments. In particular, in this section I discuss how the results in Section \ref{subsec:empirics:corpRDandspinouts} feed into the calibration of the model.

\paragraph{Growth rate due to CD and OI}

The growth rate is calibrated to the growth in labor productivity due to CD and OI, as calculated in the related papers \cite{garcia-macia_how_2019} and \cite{klenow_innovative_2020}. Their accounting procedure uses a structural model of growth through quality improvement and new variety creation (but with exogenous productivity growth) to infer the sources of growth.\footnote{The key identification assumptions are essentially that creative destruction, own-product innovation and new variety creation all have different implications for the cross-sectional and time-series joint distribution of firm and establishment employment, sales, and exit. An interesting question is whether their calibrated model would have difficulty matching the data if they augmented it with endogenous growth and additionally targeted moments relating to R\&D spending by firms of different ages and sizes.} 

\paragraph{Growth share of older firms}

The growth share of older firms improving their own products is calibrated to the growth share of OI as a fraction of OI and CD innovations, as estimated in \cite{garcia-macia_how_2019} and \cite{klenow_innovative_2020}. On average they find that, from 1982 to 2013\footnote{The end points are not exactly these in their data.}, roughly 65\% of CD + OI productivity growth was due to firms at least 6 years old. The computation of the corresponding model moment is described in \ref{appendix:calibration:growthShareOI}.

\paragraph{R\&D spending / GDP}

The data on R\&D spending is from the National Patterns of R\&D resources.\footnote{I take the average of business-funded R\&D business-performed R\&D.} In the data, about half of R\&D spending is wages for employees; in the model, the only input to R\&D is labor. I opt to match the model's aggregate R\&D intensity to that in the data, including costs other than labor. This means that the model captures the full cost of innovation, but overestimates the earnings of R\&D workers.\footnote{Since I am not considering the decision of whether to work in R\&D or production, the higher earnings of R\&D workers does not change the predictions of the model.} The computation of the corresponding model moment is described in \ref{appendix:calibration:rd/gdp}.




\paragraph{Real return}

The short-term risk-free real interest rate averages about 5\% in the United States from 1986-2006. However, the real interest rate in the model actually corresponds to the discount factor used to price an unlevered firm. Since there is no systemic risk in the model, these are the same; however, since the data exhibits systemic risk, unlevered firms require a higher return than 5\% in the data. 

To adjust for this, I use a back of the envelope calculation to calculate the asset beta from the equity betas and leverage ratios, and hence compute the hypothetical risk-premium on an unlevered investment. First, the real return on the S\&P 500 in the time period 1986-2006 averaged about 7\%. The average debt-value ratio of the S\&P 500 in the US is about 40\% during this period. Assuming that this corporate debt does not earn a risk premium, the entire risk premium accrues to the equity. If there were no leverage, the risk premium would be smaller in percentage terms, since it is accruing to a larger value investment. Quantitatively, we need to multiply the excess return by $E / (D + E)$, which in this case is $1 - 40\% = 60\%$. I arrive at a calibration value of about 6\% for the real interest rate in the model.

\paragraph{Profits \% GDP} 

The data on aggregate profits as a percent of GDP comes from the BEA (computed as an average during the sample period of 1986-2008).

\paragraph{Employment share of young firms}

The employment share target deserves some discussion. As discussed in \cite{klenow_innovative_2020}, adjustment costs mean that, in the data, it can take several years for a new product to displace an old one. However, in the model, entrants that replace incumbents reach their mature size immediately upon entry. If the model matches the amount of employment in firms of age < 1, it might underestimate the true impact on employment reallocation of each new cohort of firms.\footnote{In the data, because firms grow to achieve their mature size over the first five years (and beyond), so that the employment of an entering cohort of firms does not decrease over time (i.e., including firm exit) very rapidly in the data. If the data were in continuous time, the employment of the cohort would increase at first, then decrease. In the model, firms enter at their mature size, so the employment of a cohort decreases over time.} Given this, I match the employment share of firms age $\le 6$.

In addition, I restrict attention only to incumbents and young firms engaging in creative destruction. Because all entry in the model is creative destruction, including employment in entrants developing new varieties would overstate the rate of creative destruction. To do this, I turn to \cite{garcia-macia_how_2019} and \cite{klenow_innovative_2020}, which estimate the portion of growth coming from firms of different ages engaging in creative destruction, new variety creation, and own product improvement. They find that roughly 18\% of employment is in firms age $\le 6$ during the sample period, and that between 30\% and 70\% of the growth from these firms is due to creative destruction, the rest due to new variety creation. However, in their framework, as in mine, a given amount of growth from creative destruction requires significantly more employment, as it destroys a previous incumbent. Using a value of $\lambda = 1.2$, for example, creative destruction requires 6 times more employment than new variety creation to generate the same amount of growth. Taking this into account, I calculate an employment share of young firms of 13.34\% during the sample period. The computation of the corresponding model moment is described in \ref{appendix:calibration:entryRate}.
 
\paragraph{R\&D-induced spinout share of employment}

Finally, matching the employment share of spinouts is of course crucial so that the analysis accurately captures the burden such firms impose on the incumbents that spawn them. Figure \ref{figure:founder2_founders_f3_Accounting} documents that the regression coefficient is able to account for roughly 75\% of founder departures to WSOs in the data. Table \ref{table:GStable_founder2} shows that WSO founders account for roughly 8\% of all founders. Finally, regressions in the appendix (\ref{table:startupLifeCycle_founder2founders_lemployeecount_founder2} through \ref{table:startupLifeCycle_founder2founders_lpostvalusd_founder2}) show that a startup all of whose founders are from the same 4-digit industry has 35\% higher employment per founder, 40\% higher revenue per founder, 30-40\% higher valuation per founder. Putting these facts together, roughly 8\% of startup employment is in spinouts induced by corporate R\&D spending. However, as with the employment share of young firms, I want to restrict attention to firms engaging in creative destruction. The same kind of logic implies an adjustment factor between $1/.93$ and $1/.7$, which implies that between 8.5\% and 11\% of creative destruction startup employment consists of WSOs. I choose a value of 10\% for the calibration. The computation of the corresponding model moment is described in \ref{appendix:calibration:WSOempShare}.

\begin{table}[]
	\centering
	\captionof{table}{Calibration targets}\label{calibration_targets}
	\begin{tabular}{rcll}
		\toprule \toprule
		& Key parameter(s) & Target & Model \tabularnewline
		\midrule
		Profit (\% GDP) & $\beta$ & 8.5\% & 8.5\% 
		\tabularnewline
		R\&D emp. share & $\bar{L}_{RD}$ & 1\% & 1\% 
		\tabularnewline
		Real return & $\rho$ & 6\% & 6\% 
		\tabularnewline
		Growth rate (CD + OI) & $\mathbf{\lambda, \chi, \hat{\chi}}$ & 1.487\% & 1.487\%
		\tabularnewline		
		Age $\ge$ 6 growth share & $\chi, \hat{\chi}$  & 65\% & 65\%
		\tabularnewline
		Age $<$ 6 emp. share  & $\lambda, \hat{\chi}$ & 13.34\% & 13.34\%
		\tabularnewline
		Spinout emp. share &$\nu$  & 10\% & 10\%
		\tabularnewline
		R\&D spending (\% GDP) & $\chi, \hat{\chi}, \kappa_e$  & 1.35\% & 1.35\%
		\tabularnewline
		\bottomrule
	\end{tabular}
\end{table}

\normalsize

\subsection{Identification}\label{subsec:identification}

In this section, I make some comments regarding the identification of the model. The relationship between the model parameters and the model-generated moments is non-linear and many parameters influence multiple moments. \autoref{calibration_identificationSources} shows the elasticity of model moments to calibrated model parameters.\footnote{This is computed as the jacobian matrix of the mapping that takes log parameters to log model moments.} It suggests how identification occurs by showing which moments are sensitive to which parameters. All moments are influenced by all parameters. 

More to the point, what is relevant to identification is the inversion of this mapping; that is, the mapping from target moments to calibrated parameters. It is well-defined, at least locally, because the model is locally able to exactly reproduce the target moments. \autoref{calibration_sensitivity} shows the elasticity of calibrated parameters to moment targets.\footnote{This is calculated by inverting the matrix shown in the previous figure. This is feasible because the model is locally exactly identified by the target moments.} As the figure shows, the mapping is too complex to make simple pronouncements about which moments determine which parameters. In particular, it shows that conclusions based on \autoref{calibration_identificationSources} can be misleading. For example, while an increase in $\lambda$ causes a large increase in Growth Share OI, increasing the Growth Share OI moment target decreases the estimated $\lambda$. Given all the moments that need to be matched, the calibration prefers to match the higher Growth Share OI with a much higher $\chi$ and slightly lower $\lambda$. To complete the picture, \autoref{calibration_identificationSources_full} augments \autoref{calibration_identificationSources} with non-calibrated parameters included as both parameters and target moments. As before, \autoref{calibration_sensitivity_full} inverts this matrix to obtain the elasticity of calibrated parameters to moment targets and non-calibrated parameters. This gives the complete picture of how the model parameters are inferred. 

Based on this, I draw the following \textit{heuristic} conclusions about identification of the model. The discount rate $\rho$ is identified to simultaneously match the interest rate, growth rate and IES. The parameters $\lambda, \chi, \hat{\chi}$ are chosen to simultaneously match the growth rate $g$, the older firm share of growth and the young firm employment share. Intuitively, these three measures determine the size and frequency of each innovation, as well as its attribution to incumbents or entrants. The parameter $\kappa_e$ is distinguished from $\hat{\chi}$ by matching the R\&D / GDP ratio. Entrants have a high payoff to R\&D, which in equilibrium implies a counterfactually large R\&D / GDP ratio unless there is a non-R\&D cost associated with innovation to lower the return to entrant R\&D. Finally, the parameter $\nu$ is identified by matching employment share of WSOs. 

\begin{table}[]
	\centering
	\captionof{table}{Calibrated parameters}\label{calibration_parameters}
	\begin{tabular}{rlll}
		\toprule \toprule
		Parameter & Value & Description & Source \tabularnewline
		\midrule
		$\rho$ & 0.0303 & Discount rate  & Indirect inference \tabularnewline
		$\theta$ & 2 & $\theta^{-1} = $ IES & External calibration 
		\tabularnewline
		$\beta$ & 0.094 & $\beta^{-1} = $ EoS intermediate goods & Exactly identified \tabularnewline 
		$\psi$ & 0.5 & Entrant R\&D elasticity & External calibration \tabularnewline
		$\lambda$ & 1.084 & Quality ladder step size & Indirect inference 
		\tabularnewline
		$\chi$ & 21.217 & Incumbent R\&D productivity & Indirect inference 
		\tabularnewline
		$\hat{\chi}$ & 0.554 & Entrant R\&D productivity & Indirect inference \tabularnewline 
		$\kappa_e$ & 0.859 & Non-R\&D entry cost & Indirect inference \tabularnewline
		$\nu$ & 0.345 & Spinout generation rate  & Indirect inference\tabularnewline
		$\bar{L}_{RD}$ & 0.01 & R\&D labor allocation  & Exactly identified \tabularnewline
		\bottomrule
	\end{tabular}
\end{table}

\begin{figure}[]
	\includegraphics[scale = 0.43]{../code/julia/figures/simpleModel/identificationSources.pdf}
	\caption{Plot showing the elasticity of moments to model parameters. This illustrates how the model's equilibrium is affected by the various choices of parameters. These elasticities are computed by taking the jacobian matrix of the mapping from log parameters to log model moments.}
	\label{calibration_identificationSources}
\end{figure}

\begin{figure}[]
	\includegraphics[scale = 0.43]{../code/julia/figures/simpleModel/calibrationSensitivityFull.pdf}
	\caption{Same as \autoref{calibration_sensitivity}, but now including non-calibrated parameters. As before, this calculated by inverting the jacobian displayed in \autoref{calibration_identificationSources_full}.}
	\label{calibration_sensitivity_full}
\end{figure}

\section{Welfare effect of NCA enforcement and other policies}\label{sec:policy_analysis}

Now that the model has been calibrated, it can be used to conduct welfare analysis of the various policies studied in Section \ref{model:efficiency:policy_analysis}. 

\paragraph{Consumption-equivalent change in welfare} 

I compare welfare across BGPs in consumption-equivalent terms. For the purposes of this discussion, an \textit{allocation} is a set $A = \{ \tilde{C}_A, g_A \}$. Let $\tilde{W}(A)$ denote the initial welfare corresponding to a given allocation $A$ and consider an allocation $B$ such that $\tilde{W}(A) < \tilde{W}_B$. the CEV welfare improvement from allocation $A$ to allocation $B$ is the permanent increase in consumption in allocation $A$ that achieves the same welfare as allocation $B$. To make this precise, define the allocation $\alpha(A,B)$ as
\begin{align}
\alpha &= \{\hat{C}_{A,B}, g_A\} \\
\tilde{W} ( \alpha ) &= \tilde{W} ( B ) 
\end{align}

That is, allocation $\alpha$ has the same growth rate as allocation $A$ but a different consumption level $\hat{C}_{A,B}$ such that it provides the same welfare as allocation $B$. The consumption-equivalent percentage welfare improvement of allocation $B$ over $A$ can be calculated as
\begin{align}
100 \times \big(\frac{\hat{C}_{A,B}}{\tilde{C}_A} - 1 \big) 
\end{align}

For $\theta > 1$ (the case of interest in this paper), a $\frac{\xi}{\theta-1}\%$ CEV welfare improvement results from an $\xi\%$ decrease in the absolute value of $\tilde{W}$.\footnote{For $\theta < 1$, a $\frac{\xi}{1-\theta}\%$ CEV welfare improvement results from a $\xi\%$ increase in $\tilde{W}$. The case $\theta = 1$ corresponds to log utility, in which case
	\begin{align}
	\tilde{W} &= \frac{\rho \log(\tilde{C}) + g}{\rho^2} \label{eq:agg_welfare_log}
	\end{align}
	
	In this case, there is no simple correspondence to obtain CEV welfare changes, but they are easy to compute directly. Under the null policy, initial consumption is $\tilde{C}$ and growth is $g$. Under the new policy, initial consumption is $\tilde{C}^+$ and growth is $g^+$. The CEV welfare change is $\frac{\tilde{C}^* - \tilde{C}}{\tilde{C}}$, where $\tilde{C}^*$ is defined by 
	\begin{align}
	\frac{\rho\log(\tilde{C}^*) + g}{\rho^2} = \frac{\rho \log(\tilde{C}^+) + g^+}{\rho^2} \label{eq:agg_welfare_log_CEV}
	\end{align}}

\subsection{NCA cost $\kappa_c$}

\begin{table}
	\centering
	\captionof{table}{Effect of reducing $\kappa_c$}\label{reducing_kappa_c_table}
	\begin{tabular}{lrlll}
		\toprule \toprule
		Measure & Variable & $\kappa_c > \bar{\kappa}_c$ & $\kappa_c = 0$ & Chg. \tabularnewline
		\midrule
		Growth & $g$ & 1.487\% & 1.597\% & 7.4\% \tabularnewline
		Initial consumption & $\tilde{C}$  & 0.776 &  0.781 & 0.64\% \tabularnewline 
		\tabularnewline
		Welfare & $\tilde{W}$  &  & & 2.96\% (CEV terms)  \tabularnewline
		\bottomrule
	\end{tabular}
\end{table}

The first policy I study is the centerpiece of this paper: the enforcement of NCAs. \autoref{reducing_kappa_c_table} shows the effect on growth, initial consumption, and welfare of reducing $\kappa_c$ to zero, starting at $\kappa_c > \bar{\kappa}_c$. Both growth and initial consumption increase, leading to a 3\% increase in welfare in consumption-equivalent terms. \autoref{reducing_kappa_c_decomposition_table} displays the growth attribution under the high and low $\kappa_c$ regimes. When $\kappa_c = 0$, incumbents do a higher share of R\&D and growth increases.

The logic in Section \ref{model:efficiency:misallocationNCAs} explains the drivers of this result. The fact that growth increases is due to (\ref{cs:growth_misallocation_condition}) having a value of 0.23 in the calibration. This means that the marginal R\&D investment in own-product innovation is 4.5 times as likely to yield a given improvement in quality as the marginal investment in creative destruction. This is largely due to the business stealing and congestion externalities of creative destruction, which are $\frac{\lambda - 1}{\lambda} = .08$ and $(1-\psi) = 0.5$, respectively. While the high entry cost of creative destruction implies $\frac{1}{1-\kappa_e} = 7.1$, mitigating the effect somewhat, the net effect is still that entrants are on the margin only 27\% as productive as incumbents. In addition, this mitigating factor also means that creative destruction requires more resources. Finally, the magnitude of the effect is driven by the value of $\nu$.

\autoref{calibration_summaryPlot} shows graphically how the entire equilibrium varies with $\kappa_c$. Starting at value $\kappa_c \ge \bar{\kappa}_c$ and reducing $\kappa_c$, growth and welfare are at first constant, then jump down upon crossing $\bar{\kappa}_c$, and finally increase gradually until $\kappa_c = 0$. It also confirms the theoretical findings of Section \ref{model:efficiency:policy_analysis} regarding how equilibrium prices depend on $\kappa_c$. 

Before turning to the remaining policies, I want to note that I study the robustness of this result to variation in both target moments and uncalibrated parameters in Appendix \ref{appendix:policyanalysis:ncacost}. There I conclude that the result is robust to up to a 20\% or so standard deviation in the target moments (with zero correlation between moment uncertainty). Also, I show that the welfare result is reversed when calibrating the model to a 8\%, rather than 13.34\%, employment share of young firms. This occurs due to a much higher calibrated value of $\lambda$ (about 1.23) as well as a higher value of $\kappa_e$, both of which work to bring (\ref{cs:growth_misallocation_condition}) above 1. The reduction in welfare from $\kappa_c = 0$ is smaller in magnitude than the reduction in growth because of the very high entry cost of creative destruction.





\begin{table}
	\centering
	\captionof{table}{Decomposition of effect of reducing $\kappa_c$}\label{reducing_kappa_c_decomposition_table}
	\begin{tabular}{lrlll}
		\toprule \toprule
		Measure & Variable & $\kappa_c > \bar{\kappa}_c$ & $\kappa_c = 0$ & Chg. \tabularnewline
		\midrule
		Growth & $g$ & 1.487\% & 1.597\% & 7.4\% \tabularnewline
		\multicolumn{1}{r}{incumbents} &  & 81\% & 86.2\% & 6.42\% \tabularnewline
		\multicolumn{1}{r}{entrants} &  & 17.7\% & 13.8\% & -22\% \tabularnewline
		\multicolumn{1}{r}{spinouts} &  & 1.32\% & 0\% & -100\% \tabularnewline
		\tabularnewline
		R\&D & & & & 
		\tabularnewline
		\multicolumn{1}{r}{incumbents}  & $z / \bar{L}_{RD}$ & 67.7\% & 77.4\% & 14.3\% \tabularnewline 
		
		\multicolumn{1}{r}{entrants}  & $\hat{z} / \bar{L}_{RD}$ & 32.3\% & 22.5\% & -30.3\% \tabularnewline
		\bottomrule
	\end{tabular}
\end{table}



\begin{figure}[]
	\includegraphics[scale = 0.57]{../code/julia/figures/simpleModel/calibrationFixed_summaryPlot.pdf}
	\caption{Effect of varying $\kappa_c$ on equilibrium variables and welfare.}
	\label{calibration_summaryPlot}
\end{figure}

\subsection{R\&D subsidy (tax)}

As mentioned previously, the model can only set identify $\kappa > \bar{\kappa}_c$ in order to match the fact that there are spinouts. However, if $\kappa_c$ is much larger than $\bar{\kappa}_c$, then any changes to the incentives for NCAs induced by policy will have no observable effect on the equilibrium. Therefore, in order to be able to illustrate these effects, in this and all subsequent exercises I assume that $\kappa_c = 1.1 \bar{\kappa}_c$. 

\autoref{calibration_RDSubsidy_summaryPlot} shows how the equilibrium varies with the untargeted R\&D subsidy $T_{RD}$. Notice that growth (first row, third column) and welfare (third row, third column) both fall with $T_{RD}$, and jump down when the increase in $T_{RD}$ increases the use of NCAs. This confirms the results of Section \ref{model:efficiency:policy_analysis} that, in a setting with exogenous total R\&D, untargeted R\&D subsidies can (1) reallocate R\&D to creative destruction and (2) induce incumbents to use NCAs.

Of course, this result hinges on the fact that firms engaging in creative destruction are not worried about competition from their own employee spinouts. This is partly due to the assumption that innovation follows a memoryless random process. This means that entrants are always equally far from innovating until the point at which they succeed. If entrants build up a knowledge stock over time which yields a higher hazard rate of innovation, then they could be thought to be worried about spinouts if they are in addition large enough to take into account the effect of their behavior on their competition. 

\begin{figure}[]
	\includegraphics[scale = 0.57]{../code/julia/figures/simpleModel/calibrationFixed_RDSubsidy_summaryPlot.pdf}
	\caption{Summary of equilibrium for baseline parameter values and various values of $T_{RD}$. This assumes that $\kappa_c = 1.1 \bar{\kappa}_c$.}
	\label{calibration_RDSubsidy_summaryPlot}
\end{figure}

\subsection{CD tax (subsidy)}\label{subsec:cd_tax}

The plots in \autoref{calibration_entryTax_summaryPlot} show the impact of varying the creative destruction tax $T_e$ on equilibrium variables, growth and welfare. As expected based on the discussion of the previous paragraphs, growth and welfare increase in the entry tax $T_e$ due to a reallocation of R\&D to the incumbent. When the tax $T_e$ is high enough, incumbents begin to use noncompetes as their wage discount shrinks due to a shrinking employee value of spinout formation. Again, as in the previous case, this prediction hinges on entrants being atomistic and not having spinouts of their own. 

\begin{figure}[]
	\includegraphics[scale = 0.57]{../code/julia/figures/simpleModel/calibrationFixed_EntryTax_summaryPlot.pdf}
	\caption{Summary of equilibrium for baseline parameter values and various values of $T_e$. This assumes that $\kappa_c = 1.1 \bar{\kappa}_c$.}
	\label{calibration_entryTax_summaryPlot}
\end{figure}

\subsection{OI R\&D subsidy (tax)}\label{cs:oi_rd_subsidy}

\autoref{calibration_RDSubsidyTargeted_summaryPlot} show the effect of an R\&D subsidy targeted specifically at own-product innovation. Growth and welfare increase dramatically. It does so by increasing incumbent R\&D. For high values of $T_{RD,I}$, there is a switch from $\mathbbm{1}^{NCA} = 0$ to $\mathbbm{1}^{NCA} = 1$ and welfare jumps down slightly. Otherwise, it is monotonically increasing. 

The fact that welfare decreases slightly at the point where $T_{RD,I}$ is suffficiently high to induce the use of noncompetes suggests that the optimal policy would be to ban NCAs and implement large targeted R\&D subsidy. I consider this in the following section.

\begin{figure}[]
	\includegraphics[scale = 0.57]{../code/julia/figures/simpleModel/calibrationFixed_RDSubsidyTargeted_summaryPlot.pdf}
	\caption{Summary of equilibrium for baseline parameter values and various values of $T_{RD,I}$. This assumes that $\kappa_c = 1.1 \bar{\kappa}_c$.}
	\label{calibration_RDSubsidyTargeted_summaryPlot}
\end{figure}


\subsection{All policies}

\autoref{calibration_ALL_summaryPlot} shows the result of simultaneously varying $\kappa_c$ and $T_{RD,I}$. The highest growth rate is achieved with a high targeted R\&D subsidy of around 90\% and a high choice of $\kappa_c$ that essentially prohibits the use of NCAs. This point is also socially optimal, but less clearly so because innovation by spinouts implies higher creative destruction costs, reducing $\tilde{C}$. The welfare increase resulting from this policy is 9.05\% in consumption-equivalent terms.

Practically speaking, such a subsidy may be difficult to implement. A 90\% R\&D subsidy is extremely high. If one restricts the subidy to more reasonable levels, then it is optimal to, in addition, have $\kappa_c = 0$. This is true for all levels of the subidy below about 30\%. The optimal policy only shifts to a ban on NCAs when the allocation of R\&D is sufficiently corrected by a large enough targeted R\&D subsidy. 

\begin{figure}[]
	\includegraphics[scale = 0.46]{../code/julia/figures/simpleModel/calibrationFixed_ALL_summaryPlot.png}
	\caption{Selected equilibrium variables for various values of $T_{RD,I}$ and $\kappa_c$.}
	\label{calibration_ALL_summaryPlot}
\end{figure}

\section{Conclusion}
 
This paper investigates the effect of NCAs on growth and welfare. It makes several contributions. Empirically, I have shown that R\&D tends to predict employee spinout formation at the firm level, after controlling for firm level variables and various fixed effects. The relationship is statistically significant across three specifications and economically large enough to account for about 10\% of the employment in t, revenue and valuation on a per founder basis than non-wSO startups. These two findings confirm the findings of \cite{babina_entrepreneurial_2019} and \cite{muendler_employee_2012}, respectively, in a dataset of publicly traded parent firms and venture-capital financed startups. This is important as VC-funded startups are particularly important contributors to aggregate productivity growth. 

Theoretically, I extended a textbook model of endogenous growth with creative destruction to allow for within-product spinouts the Venture Source. It also finds that WSOs are significantly larger in employmenand non-competes. The augmented model offers some new theoretical insights about the effects of noncompete enforcement on growth and welfare. When calibrated to the firm-level relationship between R\&D and employee spinouts, findings from the growth accounting literature, and standard aggregate data, it suggests that reducing barriers to the usage of NCAs can significantly increase growth and welfare. It reaches this conclusion even in a model where there is a fixed supply of the input to R\&D due to an inferred equilibrium misallocation of R\&D labor to creative destruction, in turn resulting from strong inferred business stealing and congestion externalities of creative destruction.

The analysis also finds some counterintuitive implications of growth-enhancing policies. R\&D subsidies may have the counterintuitive effect of reducing growth by shifting R\&D labor to to firms without proprietary knowledge. R\&D subsidies targeted at own-product innovation avoid this, but for large enough values of any R\&D subsidy, incumbents may be induced to use noncompetes. Both of these effects are stronger to the extent that the society's resources for R\&D have an inelastic supply, so they may be more applicable in the medium to short term.

There are several possibilities for further work in this direction. On the empirical side, it is important to ascertain the extent to which within-industry spinouts, as identified here, actually compete with their spawning parent firms. For example, using my data, one could test whether funding announcements for spinout firms are associated with negative stock returns for parent firms. Short of this, one could at least endeavor to link the startups in Venture Source with their own reported information on their industry. Separately, one could perform a similar analysis considering employee mobility to competing incumbents. 

Theoretically, while the simplicity of the current model adds transparency to the theoretical analysis and calibration, it could be extended in several ways to add quantitative realism or to incorporate more sources of data. First, the model could be extended to allow for good-specific heterogeneity in the determinants to the use of noncompetes. This would allow the model to match data on the prevalence of the use of non-competes. The process of spinout formation could be modeled more explicitly. For example, the asymmetric information problem that leads to bilaterally inefficient spinouts could be modeled, and / or the the technology for spinout entry could be posited to require R\&D, analogous to the entrant technology. The cost of noncompete enforcement could be modeled instead as the expected duration of a noncompete contract, which could be modeled using perpetual youth. The incumbent R\&D technology could be made to have decreasing returns analogous entrants. The aggregate supply of R\&D labor could be made price-elastic. Finally parent firms can be made to compete neck-and-neck along the lines of \cite{aghion_competition_2005}. While several of these suggestions require adding a dynamic endogenous state variable to the the incumbent firm optimization problem, they may be worthwhile due to the additional realism they offer.
 

\bibliography{references.bib}

\appendix

\counterwithin{proposition}{section}
\counterwithin{proposition_corollary}{section}
\counterwithin{lemma}{section}
\counterwithin{lemma_corollary}{section}

\newpage
\section{Appendix of tables}

\setcounter{table}{0}
\renewcommand{\thetable}{\Alph{section}\arabic{table}}

% latex table generated in R 3.4.4 by xtable 1.8-4 package
% Thu Feb  6 14:38:22 2020
\begin{table}[!htb]
\centering
\begingroup\scriptsize
\begin{tabular}{p{4.5cm}llrllrll}
  \toprule
Industry & Startups & Individuals & State & Startups & Individuals & Year & Startups & Individuals \\ 
  \midrule
Business Applications Software & 1790 & 31218 & California & 8433 & 140958 & 1986 & 293 & 2103 \\ 
  Biotechnology Therapeutics & 1037 & 19264 & Massachussetts & 2217 & 37185 & 1987 & 353 & 2732 \\ 
  Communications Software & 996 & 14859 & New York & 1490 & 26450 & 1988 & 356 & 2877 \\ 
  Advertising/Marketing & 880 & 15211 & Texas & 1299 & 18452 & 1989 & 403 & 3293 \\ 
  Network/Systems Management Software & 671 & 13907 & Pennsylvania & 883 & 10759 & 1990 & 396 & 3222 \\ 
  Vertical Market Applications Software & 536 & 8401 & Washington & 784 & 12187 & 1991 & 422 & 3801 \\ 
  Online Communities & 467 & 6460 & Virginia & 606 & 8964 & 1992 & 537 & 4896 \\ 
  Application-Specific Integrated Circuits & 463 & 6475 & Colorado & 605 & 9337 & 1993 & 554 & 5322 \\ 
  Wired Communications Equipment & 458 & 6808 & Georgia & 562 & 7426 & 1994 & 689 & 6771 \\ 
  IT Consulting & 451 & 6378 & New Jersey & 557 & 7309 & 1995 & 876 & 8946 \\ 
  Drug Development Technologies & 400 & 5725 & Florida & 533 & 6524 & 1996 & 1191 & 13134 \\ 
  Healthcare Administration Software & 378 & 6500 & Illinois & 525 & 8054 & 1997 & 1141 & 13468 \\ 
  Fiberoptic Equipment & 362 & 4981 & North Carolina & 455 & 6333 & 1998 & 1513 & 19512 \\ 
  Therapeutic Devices (Minimally Invasive/Noninvasive) & 358 & 5635 & Maryland & 430 & 6223 & 1999 & 2557 & 32495 \\ 
  Business Support Services: Other & 341 & 4087 & Minnesota & 373 & 4661 & 2000 & 2003 & 24276 \\ 
  Procurement/Supply Chain & 325 & 4941 & Connecticut & 355 & 4614 & 2001 & 1067 & 13295 \\ 
  Multimedia/Streaming Software & 322 & 4460 & Ohio & 346 & 3876 & 2002 & 986 & 12946 \\ 
  Wireless Communications Equipment & 319 & 5045 & Utah & 249 & 3407 & 2003 & 1037 & 11922 \\ 
  Database Software & 318 & 6701 & Tennessee & 217 & 2828 & 2004 & 1110 & 13363 \\ 
  Specialty Retailers & 309 & 3354 & Oregon & 209 & 3071 & 2005 & 1222 & 13318 \\ 
  Entertainment & 295 & 3676 & Arizona & 207 & 2770 & 2006 & 1380 & 13829 \\ 
  Pharmaceuticals & 289 & 4282 & Michigan & 191 & 2460 & 2007 & 1506 & 13058 \\ 
  Therapeutic Devices (Invasive) & 285 & 3808 & Wisonsin & 140 & 1508 & 2008 & 1416 & 10504 \\ 
   \bottomrule
\end{tabular}
\endgroup
\caption{Statistics on startups covered by VS sample. Industry information uses VS industrial classification. Startups are counted by founding year, individuals by year they joined the firm.} 
\label{table:VS_summaryTable}
\end{table}


\begin{table}[!htb]
	\scriptsize
	\centering
	{
\def\sym#1{\ifmmode^{#1}\else\(^{#1}\)\fi}
\begin{tabular}{l*{4}{c}}
\toprule
                    &\multicolumn{1}{c}{(1)}         &\multicolumn{1}{c}{(2)}         &\multicolumn{1}{c}{(3)}         &\multicolumn{1}{c}{(4)}         \\
\midrule
$\frac{\text{WSO4 founders}}{\text{Total founders}}$&        0.19         &        0.32\sym{***}&        0.32\sym{***}&        0.30\sym{***}\\
                    &      (0.22)         &     (0.027)         &     (0.020)         &     (0.013)         \\
\addlinespace
Constant            &        2.44\sym{***}&        2.41\sym{***}&        2.41\sym{***}&        2.41\sym{***}\\
                    &     (0.073)         &   (0.00019)         &   (0.00015)         &   (0.00028)         \\
\addlinespace
State-Year FE       &          No         &         Yes         &         Yes         &          No         \\
\addlinespace
State-Age FE        &          No         &         Yes         &          No         &         Yes         \\
\addlinespace
State-Cohort FE     &          No         &          No         &         Yes         &         Yes         \\
\addlinespace
NAICS4-Year FE      &          No         &         Yes         &         Yes         &          No         \\
\addlinespace
NAICS4-Age FE       &          No         &         Yes         &          No         &         Yes         \\
\addlinespace
NAICS4-Cohort FE    &          No         &          No         &         Yes         &         Yes         \\
\addlinespace
No FE               &         Yes         &          No         &          No         &          No         \\
\midrule
Clustering          &statecode naics1\_4 year         &statecode naics1\_4         &statecode naics1\_4         &statecode naics1\_4         \\
R-squared (adj.)    &     0.00068         &        0.35         &        0.38         &        0.36         \\
R-squared (within, adj)&     0.00068         &      0.0028         &      0.0028         &      0.0024         \\
Observations        &       55767         &       54873         &       54654         &       54779         \\
\bottomrule
\multicolumn{5}{l}{\footnotesize Standard errors in parentheses}\\
\multicolumn{5}{l}{\footnotesize \sym{*} \(p<0.1\), \sym{**} \(p<0.05\), \sym{***} \(p<0.01\)}\\
\end{tabular}
}

	\caption{Dependent variable is the logarithm of the number of employees while the independent variable is the fraction of founders who most recently worked at a public firm in the same industry. The first column shows the raw regression. The following three columns control for state, industry, time, cohort and age factors. Specifically, each regression uses a subset of two of the three (year,age,cohort) effects, in all cases included interacted both with state and industry.} 
	\label{table:startupLifeCycle_founder2founders_lemployeecount_founder2}
\end{table}

\begin{table}[!htb]
	\scriptsize
	\centering
	{
\def\sym#1{\ifmmode^{#1}\else\(^{#1}\)\fi}
\begin{tabular}{l*{4}{c}}
\toprule
                    &\multicolumn{1}{c}{(1)}         &\multicolumn{1}{c}{(2)}         &\multicolumn{1}{c}{(3)}         &\multicolumn{1}{c}{(4)}         \\
\midrule
$\frac{\text{WSO4 founders}}{\text{Total founders}}$&       -0.13         &        0.45\sym{***}&        0.42\sym{***}&        0.39\sym{***}\\
                    &     (0.094)         &      (0.13)         &     (0.081)         &      (0.12)         \\
\addlinespace
State-Year FE       &          No         &         Yes         &         Yes         &          No         \\
\addlinespace
State-Age FE        &          No         &         Yes         &          No         &         Yes         \\
\addlinespace
State-Cohort FE     &          No         &          No         &         Yes         &         Yes         \\
\addlinespace
NAICS4-Year FE      &          No         &         Yes         &         Yes         &          No         \\
\addlinespace
NAICS4-Age FE       &          No         &         Yes         &          No         &         Yes         \\
\addlinespace
NAICS4-Cohort FE    &          No         &          No         &         Yes         &         Yes         \\
\midrule
Clustering          &State, Industry         &State, Industry         &State, Industry         &State, Industry         \\
R-squared (adj.)    &    0.000092         &        0.30         &        0.38         &        0.39         \\
R-squared (within, adj)&    0.000092         &      0.0030         &      0.0026         &      0.0022         \\
Observations        &       16948         &       15500         &       15531         &       15905         \\
\bottomrule
\multicolumn{5}{l}{\footnotesize Standard errors in parentheses}\\
\multicolumn{5}{l}{\footnotesize \sym{*} \(p<0.1\), \sym{**} \(p<0.05\), \sym{***} \(p<0.01\)}\\
\end{tabular}
}

	\caption{Dependent variable is the logarithm of annual revenue while the independent variable is the fraction of founders who most recently worked at a public firm in the same industry. The first column shows the raw regression. The following three columns control for state, industry, time, cohort and age factors. Specifically, each regression uses a subset of two of the three (year,age,cohort) effects, in all cases included interacted both with state and industry.} 
	\label{table:startupLifeCycle_founder2founders_lrevenue_founder2}
\end{table}

\begin{table}[!htb]
	\scriptsize
	\centering
	{
\def\sym#1{\ifmmode^{#1}\else\(^{#1}\)\fi}
\begin{tabular}{l*{4}{c}}
\toprule
                    &\multicolumn{1}{c}{(1)}         &\multicolumn{1}{c}{(2)}         &\multicolumn{1}{c}{(3)}         &\multicolumn{1}{c}{(4)}         \\
\midrule
$\frac{\text{WSO4 founders}}{\text{Total founders}}$&        0.46\sym{***}&        0.42\sym{***}&        0.36\sym{***}&        0.33\sym{***}\\
                    &     (0.065)         &     (0.058)         &     (0.069)         &     (0.074)         \\
\addlinespace
State-Year FE       &          No         &         Yes         &         Yes         &          No         \\
\addlinespace
State-Age FE        &          No         &         Yes         &          No         &         Yes         \\
\addlinespace
State-Cohort FE     &          No         &          No         &         Yes         &         Yes         \\
\addlinespace
NAICS4-Year FE      &          No         &         Yes         &         Yes         &          No         \\
\addlinespace
NAICS4-Age FE       &          No         &         Yes         &          No         &         Yes         \\
\addlinespace
NAICS4-Cohort FE    &          No         &          No         &         Yes         &         Yes         \\
\midrule
Clustering          &State, Industry         &State, Industry         &State, Industry         &State, Industry         \\
R-squared (adj.)    &      0.0042         &        0.28         &        0.29         &        0.26         \\
R-squared (within, adj)&      0.0042         &      0.0050         &      0.0035         &      0.0028         \\
Observations        &       26504         &       25174         &       25027         &       25337         \\
\bottomrule
\multicolumn{5}{l}{\footnotesize Standard errors in parentheses}\\
\multicolumn{5}{l}{\footnotesize \sym{*} \(p<0.1\), \sym{**} \(p<0.05\), \sym{***} \(p<0.01\)}\\
\end{tabular}
}

	\caption{Dependent variable is the logarithm of post-money valuation while the independent variable is the fraction of founders who most recently worked at a public firm in the same industry. The first column shows the raw regression. The following three columns control for state, industry, time, cohort and age factors. Specifically, each regression uses a subset of two of the three (year,age,cohort) effects, in all cases included interacted both with state and industry.} 
	\label{table:startupLifeCycle_founder2founders_lpostvalusd_founder2}
\end{table}


\begin{table}[]
	\centering
	\captionof{table}{2-digit NAICS codes summary}\label{}
	\begin{tabular}{rl}
		\toprule \toprule
		2-digit Code & Description \tabularnewline
		\midrule
		11  & Agriculture, Forestry, Fishing and Hunting \tabularnewline
		21  & Mining, Quarrying, and Oil and Gas Extraction\tabularnewline
		22  & Utilities\tabularnewline
		23  & Construction \tabularnewline
		31-33 & Manufacturing \tabularnewline
		42 & Wholesale trade \tabularnewline
		44-45 & Retail trade \tabularnewline
		48-49 & Transportation and warehousing \tabularnewline
		51 & Information \tabularnewline
		52 & Finance and insurance \tabularnewline
		53 & Real estate and Rental and Leasing \tabularnewline
		54 & Professional, Scientific, and Technical Services \tabularnewline
		55 & Management of Companies and Enterprises \tabularnewline
		56 & Administrative, Support, Waste Management, Remediation Service \tabularnewline
		61 & Educational services \tabularnewline
		62 & Health Care and Social Assistance \tabularnewline
		71 & Arts, Entertainment, Recreation \tabularnewline
		72 & Accomodation and Food Services \tabularnewline
		81 & Other Services (ecept public Admin.) \tabularnewline
		92 & Public Administration\tabularnewline
		\bottomrule
	\end{tabular}
\end{table}

\begin{table}[]
	\centering
	\captionof{table}{Alternative calibration}\label{calibration_2_parameters}
	\begin{tabular}{rlll}
		\toprule \toprule
		Parameter & Value & Description & Source \tabularnewline
		\midrule
		$\rho$ & 0.0303 & Discount rate  & Indirect inference \tabularnewline
		$\theta$ & 2 & $\theta^{-1} = $ IES & External calibration 
		\tabularnewline
		$\beta$ & 0.094 & $\beta^{-1} = $ EoS intermediate goods & Exactly identified \tabularnewline 
		$\psi$ & 0.5 & Entrant R\&D elasticity & External calibration \tabularnewline
		$\lambda$ & 1.083 & Quality ladder step size & Indirect inference 
		\tabularnewline
		$\chi$ & 21.3 & Incumbent R\&D productivity & Indirect inference 
		\tabularnewline
		$\hat{\chi}$ & 0.554 & Entrant R\&D productivity & Indirect inference \tabularnewline 
		$\kappa_e$ & 0.859 & Non-R\&D entry cost & Indirect inference \tabularnewline
		$\nu$ & 0.346 & Spinout generation rate  & Indirect inference\tabularnewline
		$\bar{L}_{RD}$ & 0.01 & R\&D labor allocation  & Normalization \tabularnewline
		\bottomrule
	\end{tabular}
\end{table}



\newpage
\section{Appendix of figures}

\setcounter{figure}{0}
\renewcommand{\thefigure}{\Alph{section}\arabic{figure}}

\begin{figure}[!htb]
	\centering
	\includegraphics[scale=0.95]{../empirics/figures/plots/industry_column_heatmap_naics2_founder2.pdf}
	\caption{Heatmap displaying the distribution of parent 2-digit NAICS code (row), conditional on child NAICS code (column). Darker hues indicate a higher density.}
	\label{figure:industry_column_heatmap_naics2_founder2}
\end{figure}

\begin{figure}[]
	\centering
	\includegraphics[scale=0.8]{../empirics/figures/founder2_founders_wso4_f3_Accounting_industryYear.pdf}
	\caption{Economic magnitude of regression estimates in Tables \ref{table:RDandSpinoutFormation_absolute_founder2_l3f3} and \ref{table:RDandSpinoutFormation_at_founder2_l3f3} (i.e. using the average value of the coefficient on R\&D in the fourth column). The figure compares predictions for WSO founder counts at the WSO4-year level to the actual counts, again using only the regression estimates and the level of R\&D in each sector-year. This plot considers only naics industries 3 (manufacturing) and 5 (which include software and most other tech companies), industries responsible for the vast majority (\textbf{[insert figure]}) of private sector R\&D in the United States. The solid (dotted) lines show the fit of an OLS regression of the prediction on the actual (weighted by R\&D spending in each industry-year).}
	\label{figure:founder2_founders_f3_Accounting_industryYear}
\end{figure}

\begin{figure}[]
	\includegraphics[scale = 0.43]{../code/julia/figures/simpleModel/calibrationSensitivity.pdf}
	\caption{Plot showing the elasticity of parameters to moments. It is computed by inverting the jacobian matrix of the mapping from log parameters to log model moments (whose entries comprise the previous figure). These elasticities, along with estimates of the noisiness of the moments used in the calibration, can be used to estimate confidence intervals for the parameters in the model, and thereby for the welfare comparison in question.}
	\label{calibration_sensitivity}
\end{figure}

\begin{figure}[]
	\includegraphics[scale = 0.43]{../code/julia/figures/simpleModel/identificationSourcesFull.pdf}
	\caption{Plot showing the elasticity of moments to model parameters, including parameters taken from the literature $\theta , \beta, \psi$. These non-calibrated parameters are added in as effective moments to be matched, allowing the sensitivity of calibrated parameters $\rho, \lambda, \chi, \hat{\chi}, \kappa_E, \nu$ to these parameters to be computed by simply inverting this matrix, as before.}
	\label{calibration_identificationSources_full}
\end{figure}

\begin{figure}[]
	\includegraphics[scale = 0.36]{../code/julia/figures/simpleModel/welfareComparisonParameterSensitivityFull.pdf}
	\caption{Sensitivity of welfare comparison to moments. This is $\nabla_p W$, where $W(p)$ maps log parameters to the log of the percentage change in BGP consumption which is equivalent to the change in welfare from changing $\kappa_c$ from $\kappa_c > \bar{\kappa_c}$ to $\kappa_c = 0$ (i.e. going from banning to frictionlessly enforcing NCAs).}
	\label{welfareComparisonParameterSensitivityFull}
\end{figure}


\section{Model}\label{appendix:model}

\subsection{Proofs of propositions}

\subsubsection{Proof of Proposition \ref{proposition:hjb_scaling}}\label{appendix:proofs:proposition:hjb_scaling}

\paragraph{Differentiability of $V(j,t|q)$}

I am assuming that the value of incumbency in line $j$ is differentiable in $t$ conditional on no innovations occurring. I believe this does lose some generality -- it excludes equilibria where sunspots could create jumps in the value function. However, as far as I can tell this assumption is made in every paper using this framework, starting from \cite{grossman_quality_1991} and including the models that form the direct foundation of this model such as the one in \cite{acemoglu_innovation_2015} and the related models in \cite{acemoglu_introduction_2009}. Extending the theoretical analysis of this broad class of models is outside the scope of this paper, so I adopt their assumption. 

\begin{proof}
	If $z > 0$ then the FOC holds with equality; otherwise, we can ignore that terms multiplied by $z$ in the incumbent's HJB. Hence, the incumbent HJB implies
	\begin{align}
	(r_t + \hat{\tau}) V(j,t|q) - \dot{V}(j,t|q) &= \tilde{\pi} q
	\end{align}
	
	where I used $\hat{\tau}_{jt} = \hat{\tau}$ on a symmetric BGP.
	
	Next, the static equilibrium implies that $C_t \propto Q_t$. Therefore $\frac{\dot{Q}_t}{Q_t} = g$ implies $\frac{\dot{C}_t}{C_t} = g$, and the Euler equation then implies $r_t = r$. Therefore, 
	\begin{align}
	(r + \hat{\tau}) V(j,t|q) - \dot{V}(j,t|q) &= \tilde{\pi} q
	\end{align}
	
	This differential equation has a constant solution equal to 
	\begin{align}
	V(j,t|q) &= \frac{\tilde{\pi} q}{r + \hat{\tau}} \\
	&= \tilde{V} q 
	\end{align}
	
	Nonconstant solutions correspond to positive or negative bubbles in the valuation of the incumbent, which I simply rule out. Below, I provide what may be a more rigorous proof, but I don't want to emphasize this. In every paper I know of using this type of model, the value of incumbents has simply been assumed to be the constant solution (or, as in \cite{acemoglu_innovation_2015}, the value function is directly assumed to be of the desired linear form).
\end{proof}

\begin{proof}

	The entrant optimization condition is (using $z_{jt} = z$)
	\begin{align}
	\hat{\chi} \hat{z}^{-\psi} V(j,t|\lambda q) &= \frac{q}{Q_t} \hat{w}_{RD,t}
	\end{align}
	
	Rearranging,
	\begin{align}
	\hat{\chi}^{-1} \hat{z}^{\psi} &= \frac{V(j,t|\lambda q)}{q} \frac{Q_t}{\hat{w}_{RD,t}}  \label{constant_vw_ratio}
	\end{align}
	
	The only term which varies with $j$ is $V(j,t|\lambda q)$. This implies that $V(j,t| \lambda q) =  \tilde{V}(t| \lambda q)$. In fact, the equation also shows that $V(t|\lambda q) / q$ is constant over $q$, i.e. that $\tilde{V}(t|\lambda q) = \tilde{V}(t) \lambda q$. Note, however, that it does not directly imply anything about $V(j,t|q)$: constant entrant innovation effort $\hat{z}_{jt} = \hat{z}$ implies that the value of incumbency tomorrow must be proportional to $q$, but it does not directly imply that the value of incumbency today is proportional to $q$. It makes sense intuitively that the same logic should imply that $V(j,t|q) = \bar{\tilde{V}}(t) q$: otherwise, the equilibrium we are on cannot have satisfied the (rational) expectations of previous entrants. Heuristically maybe this is enough, but I haven't found a rigorous proof along these lines. Instead, I show that $V(j,t|q) = \tilde{V}q$ by arguing that other solutions to the incumbent HJB contradict equilbrium requirements. Then the fact that at any time $V(j,t|q)$ has this form implies that innovators (entrants, incumbents, spinouts) expect this value in the future and therefore that they forecast their future payoffs using $V(j,t|\lambda q) = \tilde{V} \lambda q$.
	
	First, differentiating both sides with respect to $t$ and using $\frac{\dot{Q}_t}{Q_t} = g$ on the BGP yields
	\begin{align}
	- \frac{\dot{V}(t|\lambda q)}{V(t|\lambda q)} &= g - \frac{\dot{\hat{w}}_{RD,t}}{\hat{w}_{RD,t}} \label{appendix:eq:freeEntryDifferentiated}
	\end{align}
	
	The incumbent HJB implies
	\begin{align}
	(r_t + \hat{\tau}) V(j,t|q) - \dot{V}(j,t|q) &= \tilde{\pi} q
	\end{align}
	
	The static equilibrium implies that $C_t \propto Q_t$. Therefore $\frac{\dot{Q}_t}{Q_t} = g$ implies $\frac{\dot{C}_t}{C_t} = g$, and the Euler equation then implies $r_t = r$. Therefore, 
	\begin{align}
	(r + \hat{\tau}) V(j,t|q) - \dot{V}(j,t|q) &= \tilde{\pi} q
	\end{align}
	
	This differential equation has a constant solution equal to 
	\begin{align}
	V(j,t|q) &= \frac{\tilde{\pi} q}{r + \hat{\tau}} \\
	&= \tilde{V} q 
	\end{align}
	
	I want to show that this is the only solution which is compatible with equilibrium. Rearranging the original differential equation,
	\begin{align}
	\dot{V}(j,t|q) &= (r + \hat{\tau}) V(j,t|q) - \tilde{\pi} q \label{appendix:eq:hjbGeneral}
	\end{align}	
	
	Differentiating this expression again yields
	\begin{align}
	\ddot{V}(j,t|q) &= (r + \hat{\tau}) \dot{V}(j,t|q) \label{appendix:eq:hjbGeneralDifferentiated}
	\end{align}
	
	This means that if $\dot{V} < 0$ initially, then $\ddot{V} < 0$ initially as well, and similarly if $\dot{V} > 0$ initially then $\ddot{V} > 0$ initially as well. Hence, if in equilbrium $V(j,t|q)$ drifts locally, it must drift monotonically.
	
	If $\dot{V} < 0$ then (\ref{appendix:eq:hjbGeneral}) implies that $\dot{V}$ tends to $-\tilde{\pi}q$ as $t \to \infty$. This means that $V$ reaches a negative value in finite time with positive probability. This contradicts optimality since the incumbent is always free to choose $z = 0$ and earn flow profits $\tilde{\pi} q$; hence, this solution to the HJB is incompatible with equilibrium. 
	
	Next, rearrange the expression in the form
	\begin{align}
	\frac{\dot{V}(j,t|q)}{V(j,t|q)} &= r + \hat{\tau} - \frac{\tilde{\pi}q}{V(j,t|q)}
	\end{align}
	
	
	If $\dot{V} > 0$ intially, the second term on the RHS tends to zero and asymptotically $V$ grows at exponential rate $r + \hat{\tau}$. 
	
	First, suppose that $z > 0$. The FOC of the incumbent is
	\begin{align}
	\chi \Big( V(t|\lambda q) - V(j,t|q) \Big) &= \frac{q}{Q_t} \hat{w}_{RD,t} + \nu V(j,t|q) \nonumber \\
	&+ (1 - \mathbbm{1}^{NCA}_{jt}) (1- \kappa_e) \nu V(t|\lambda q) + \mathbbm{1}^{NCA}_{jt}  \kappa_c \nu V(j,t|q) \Big) 
	\end{align}
	
	Divide both sides by $q$, differentiate with respect to $t$, use $\frac{\dot{Q}_t}{Q_t} = g$ and  (\ref{appendix:eq:freeEntryDifferentiated}) to obtain
	\begin{align}
	-\frac{\dot{V}(j,t|q)}{V(j,t|q)} &= g - \frac{\dot{\hat{w}}_{RD,t}}{\hat{w}_{RD,t}}  \label{appendix:eq:freeEntryDifferentiatedImplication}
	\end{align}
	
	Using (\ref{appendix:eq:freeEntryDifferentiatedImplication}), this implies that with positive probability the R\&D wage grows to the point where $\hat{w}_{RD} \hat{z} > \tilde{Y}$, which contradicts equilbirium.
	
	Next, suppose that $z = 0$. In this case, the equilibrium R\&D allocation and growth rate are uniquely determined, so there is no harm in simply restricting attention to the case where $V(j,t|\bar{q}_{jt}) = \tilde{V}\bar{q}_{jt}, V(j,t|\lambda \bar{q}_{jt}) = \tilde{V} \bar{q}_{jt}$ for all $(j,t)$. However, below I have some discussion of how one would show that the value function has this form in any symmetric equilibrium.	
	
	We still know that $V(j,t|\lambda q) = \tilde{V}(t) \lambda q$, by (\ref{constant_vw_ratio}). However, we no longer have the incumbent FOC which helps to connect statements we make about $V(j,t|\lambda q) / q$ and $\hat{w}_{RD,t} / Q_t$ to statements about $V(j,t|q)$. However, the proof is simpler in this case because incumbents do no R\&D. This means that the value functions of the initial incumbents do not affect the rest of the equilibrium. More precisely, the BGP that arises in this model has exactly the same observable variables (except for the market price of initial incumbents, which may be a bubble in the model) as a BGP where $V(j,t|q) = \tilde{V}q$ and $V(j,t|\lambda q) = \tilde{V} \lambda q$ for all $(j,t,q)$, provided one can show that $\tilde{V}(t) = \tilde{V}$. 
	
	In principle, this equation could fail to hold because $\hat{w}_{RD,t} / Q_t$ fluctuates over time. When it grows (shrinks), $V(t)$ grows (shrinks) by the same proportion. To rule this out, suppose that $V(t') > V(t)$. With positive probability, the next innovation occurs at time $dt'$; alternatively, with positive probability the next innovation occurs at time $dt$. It cannot be true that $V(j,t|\lambda q) = V(j,t'|\lambda q) =  \frac{\tilde{\pi}\lambda q}{r + \hat{\tau}}$ is the value of the monopolist ex post. If $V(j,t|\lambda q) < \frac{\tilde{\pi}\lambda q}{r + \hat{\tau}}$ then by the logic above the value can go negative with positive probability, which violates optimality. 
	
	If $V(j,t|\lambda q) > \frac{\tilde{\pi}\lambda q}{r + \hat{\tau}}$ for some $(j,t)$ then this will violate the incumbent's transversality condition.\footnote{The incumbent's dynamic optimization problem has a transversality condition, because the dividend vs R\&D investment decision is an optimal control problem (the state being controlled is the quality of the incumbent).} It states that
	\begin{align}
	0 = \lim_{t' \to \infty} e^{-r(t'-t)} \mathbb{E}[\mathbbm{1}_{s(j,t) > t'} V(j,t',q)]
	\end{align}
	
	where $s(j,t)$ is the (random) time at which the current incumbent is displaced by an entrant. Because the time is exponentially distributed, this is the same as
	\begin{align}
	0 = \lim_{t' \to \infty} e^{-r(t'-t)} e^{-\hat{\tau} (t'-t)} V(j,t'|q)
	\end{align}
	
	However, we are in the case where $V(j,t'|q)$ grows at rate $r + \hat{\tau}$ asymptotically, i.e.
	\begin{align}
	\frac{\dot{V}(j,t|q)}{V(j,t|q)} &= r + \hat{\tau} - \frac{\tilde{\pi}q}{V(j,t|q)}
	\end{align}
	
	We have a situation with 
	\begin{align}
	\frac{\dot{X}}{X} &= r + \hat{\tau}, \quad  \text{i.e. } X(t') = e^{(r + \hat{\tau})(t' - t)} \\
	\frac{\dot{Y}}{Y} &= r + \hat{\tau} - \frac{\tilde{\pi}q}{Y}, \quad \text{i.e. } Y(t') = V(j,t'|q)
	\end{align}
	
	I want to show that
	\begin{align}
	\lim_{t \to \infty} \frac{Y}{X} > 0 
	\end{align}
	
	We know that
	\begin{align}
	\frac{d}{dt'}\Big(\frac{Y}{X} \Big) &= \frac{Y}{X} \Big( \frac{\dot{Y}}{Y} - \frac{\dot{X}}{X} \Big)
	\end{align}
	
	This yields
	\begin{align}
	\frac{d}{dt'} \Big(\frac{Y}{X} \Big) &= -\frac{\tilde{\pi}q}{Y} \frac{Y}{X}  \\
	&= - \frac{\tilde{\pi}q}{X} \\
	&= - \tilde{\pi} q e^{-(r +\hat{\tau}) (t' - t)}
	\end{align}
	
	using the definition of $X$. We know that $Y(t) = V(j,t|q) > \frac{\tilde{\pi} q}{r + \hat{\tau}}$ otherwise $\dot{V}(j,t|q) \le 0$ and it must continue to decline locally by (\ref{appendix:eq:hjbGeneralDifferentiated}). Since $X(t) = 1$, we have $\frac{Y(t)}{X(t)} > \frac{\tilde{\pi} q}{r + \hat{\tau}}$. Integrating, 
	\begin{align}
	\lim_{t' \to \infty} \frac{Y(t')}{X(t')} &=  \frac{Y(t)}{X(t)} + \lim_{t' \to \infty} \int_t^{t'} \frac{d}{ds} \Big(\frac{Y(s)}{X(s)} \Big) ds \\
	&= \frac{Y(t)}{X(t)} - \lim_{t' \to \infty} \int_t^{t'}  \tilde{\pi} q e^{-(r +\hat{\tau}) (s-t)} dt \\
	&> \frac{\tilde{\pi} q}{r + \hat{\tau}} - \underbrace{\lim_{t' \to \infty} \int_t^{t'}  \tilde{\pi} q e^{-(r +\hat{\tau}) (s-t)} dt}_{\mathclap{\frac{\tilde{\pi}q}{r + \hat{\tau}}}} \\ 
	&= 0
	\end{align}
	
	Therefore, the TVC is violated. I conclude that the only solution to the HJB compatible with equilibrium is 
	\begin{align}
	V(j,t|q) &= \frac{\tilde{\pi} q}{r + \hat{\tau}}
	\end{align}	
	
	which has the linear form required in the proposition.
	
\end{proof}

\subsection{Derivations for efficiency and theoretical policy analysis}

\subsubsection{CD tax (subsidy)}\label{appendix:model:efficiencyderivations:CDtax}

The incumbent HJB is given by 
\begin{align}
(r + \hat{\tau}) \tilde{V} = \tilde{\pi} + \max_{\substack{\mathbbm{1}^{NCA} \in \{0,1\} \\ z \ge 0}} \Big\{z &\Big( \overbrace{\chi (\lambda - 1) \tilde{V}}^{\mathclap{\mathbb{E}[\textrm{Benefit from R\&D}]}}-  \big( \overbrace{\hat{w}_{RD} - (1-\mathbbm{1}^{NCA})(1-(1+T_e)\kappa_e)\lambda \nu \tilde{V}}^{\mathclap{\text{R\&D wage}}}\big) \label{eq:hjb_incumbent_entryTax} \nonumber \\ 
&-  \underbrace{(1-\mathbbm{1}^{NCA}) \nu \tilde{V}}_{\mathclap{\text{Net cost from spinout formation}}} - \overbrace{\mathbbm{1}^{NCA} \kappa_{c} \nu \tilde{V}}^{\mathclap{\text{Direct cost of NCA}}}\Big) \Big\} 
\end{align}

The equilibrium conditions not shown in the main text are
\begin{align}
\hat{\tau} &= \hat{\chi} \hat{z}^{1-\psi} \\
z &= \bar{L}_{RD} - \hat{z} \label{eq:labor_resource_constraint_entryTax}\\ 
\tau &= \chi z \\
\tau^S &= (1-\mathbbm{1}^{NCA}) \nu z \\
g &= (\lambda - 1) (\tau + \tau^S + \hat{\tau}) \\
r &= \theta g + \rho \\
\tilde{V} &= \frac{\tilde{\pi}}{r + \hat{\tau}} \\ 
\hat{w}_{RD} &= \begin{cases}
\hat{\chi} \hat{z}^{-\psi} (1-(1+T_e)\kappa_e) \lambda \tilde{V} &\textrm{, if } \hat{z} > 0\\
\Big( \chi(\lambda -1) - \nu (\mathbbm{1}^{NCA}\kappa_c + (1-\mathbbm{1}^{NCA})\hat{\bar{\kappa}}_c(\kappa_e,\lambda;T_e))\Big) \tilde{V} &\textrm{, o.w.}
\end{cases} \label{eq:wage_rd_labor_entryTax}
\end{align}


\subsubsection{OI R\&D subsidy (tax)}\label{appendix:model:efficiencyderivations:OIRDtax}

The incumbent's HJB is given by
\begin{align}
(r + \hat{\tau}) \tilde{V} = \tilde{\pi} + \max_{\substack{\mathbbm{1}^{NCA} \in \{0,1\} \\ z \ge 0}} \Big\{z &\Big( \overbrace{\chi (\lambda - 1) \tilde{V}}^{\mathclap{\mathbb{E}[\textrm{Benefit from R\&D}]}}- (\underbrace{1-T_{RD,I}}_{\mathclap{\text{R\&D Subsidy}}}) \big( \overbrace{\hat{w}_{RD} - (1-\mathbbm{1}^{NCA})(1-\kappa_e)\lambda \nu \tilde{V}}^{\mathclap{\text{R\&D wage}}}\big) \label{eq:hjb_incumbent_RDsubsidyTargeted} \nonumber \\ 
&-  \underbrace{(1-\mathbbm{1}^{NCA}) \nu \tilde{V}}_{\mathclap{\text{Net cost from spinout formation}}} - \overbrace{x \kappa_{c} \nu \tilde{V}}^{\mathclap{\text{Direct cost of NCA}}}\Big) \Big\} 
\end{align}

The equilibrium conditions not shown in the main text are
\begin{align}
\hat{\tau} &= \hat{\chi} \hat{z}^{1-\psi} \\
z &= \bar{L}_{RD} - \hat{z} \label{eq:labor_resource_constraint_RDsubsidyTargeted}\\ 
\tau &= \chi z \\
\tau^S &= (1-\mathbbm{1}^{NCA}) \nu z \\
g &= (\lambda - 1) (\tau + \tau^S + \hat{\tau}) \\
r &= \theta g + \rho \\
\tilde{V} &= \frac{\tilde{\pi}}{r + \hat{\tau}} \\ 
\hat{w}_{RD} &= \hat{\chi} \hat{z}^{-\psi} (1-\kappa_e) \lambda \tilde{V} \label{eq:wage_rd_labor_RDsubsidyTargeted}
\end{align}



\subsubsection{All policies}\label{appendix:model:efficiencyderivations:allPolicies}

The R\&D labor supply indifference condition is
\begin{align}
\hat{w}_{RD} &= w_{RD,j} + (1-x_j) \nu (1-(\underbrace{1+T_e}_{\mathclap{\text{Entry tax}}})\kappa_e) \lambda \tilde{V} \label{eq:RD_worker_indifference_all}
\end{align}

The incumbent HJB is
\begin{align}
(r + \hat{\tau}) \tilde{V} = \tilde{\pi} + \max_{\substack{\mathbbm{1}^{NCA} \in \{0,1\} \\ z \ge 0}} \Big\{z &\Big( \overbrace{\chi (\lambda - 1) \tilde{V}}^{\mathclap{\mathbb{E}[\textrm{Benefit from R\&D}]}}-  (\underbrace{1 - T_{RD} - T_{RD,I}}_{\mathclap{\text{R\&D subsidies}}})\big( \overbrace{\hat{w}_{RD} - (1-\mathbbm{1}^{NCA})(1-(1+T_e)\kappa_e)\lambda \nu \tilde{V}}^{\mathclap{\text{R\&D wage}}}\big) \label{eq:hjb_incumbent_all} \nonumber \\ 
&-  \underbrace{(1-\mathbbm{1}^{NCA}) \nu \tilde{V}}_{\mathclap{\text{Net cost from spinout formation}}} - \overbrace{x \kappa_{c} \nu \tilde{V}}^{\mathclap{\text{Direct cost of NCA}}}\Big) \Big\} 
\end{align}

which can be rearranged to
\begin{align}
(r + \hat{\tau}) \tilde{V} = \tilde{\pi} + \max_{\substack{\mathbbm{1}^{NCA} \in \{0,1\} \\ z \ge 0}} \Big\{z &\Big( \overbrace{\chi (\lambda - 1) \tilde{V}}^{\mathclap{\mathbb{E}[\textrm{Benefit from R\&D}]}}- (1-T_{RD}- T_{RD,I})\hat{w}_{RD} \\
&-  \underbrace{(1-\mathbbm{1}^{NCA})(1 - (1-T_{RD} - T_{RD,I})(1-(1+T_e)\kappa_{e})\lambda)\nu \tilde{V}}_{\mathclap{\text{Net cost from spinout formation}}} - \overbrace{x \kappa_{c}\nu \tilde{V}}^{\mathclap{\text{Direct cost of NCA}}}\Big) \Big\} \label{eq:hjb_incumbent_all_2}
\end{align}

Define
\begin{align}
\bar{\bar{\kappa}}_c(\kappa_e,\lambda;T_{RD},T_{RD,I},T_e) = 1 - (1-T_{RD} - T_{RD,I})(1-(1+T_e)\kappa_e)\lambda  \label{eq:barkappa_all}
\end{align} 

If $z > 0$, the incumbent's optimal NCA policy is given by 
\begin{align}
x = \begin{cases}
1 & \textrm{if } \kappa_c < \bar{\bar{\kappa}}_c (\kappa_e, \lambda;T_{RD},T_{RD,I},T_e)\\
0 & \textrm{if } \kappa_c > \bar{\bar{\kappa}}_c (\kappa_e, \lambda;T_{RD},T_{RD,I},T_e)\\
\{0,1\} & \textrm{if } \kappa_c = \bar{\bar{\kappa}}_c (\kappa_e, \lambda;T_{RD},T_{RD,I},T_e)
\end{cases} \label{eq:nca_policy_all}
\end{align}


By the usual argument, $z > 0$ implies that the incumbent's FOC can be rearranged to
\begin{align}
\tilde{V} &= \frac{(1-T_{RD} - T_{RD,I})\hat{w}_{RD}}{\chi(\lambda -1) - \nu (x\kappa_c + (1-\mathbbm{1}^{NCA})(1 - (1-T_{RD} - T_{RD,I})(1-(1+T_e)\kappa_e)\lambda)) } \label{eq:hjb_incumbent_foc_all}
\end{align}

The free entry condition is
\begin{align}
\underbrace{\hat{\chi} \hat{z}^{-\psi}}_{\mathclap{\text{Marginal innovation rate}}} \overbrace{(1-(1+T_e)\kappa_e) \lambda \tilde{V}}^{\mathclap{\text{Payoff from innovation}}} &= (1-T_{RD})\underbrace{\hat{w}_{RD}}_{\mathclap{\text{MC of R\&D}}} \label{eq:free_entry_condition_all}
\end{align}

Substituting (\ref{eq:hjb_incumbent_foc_all}) into (\ref{eq:free_entry_condition_all}) to eliminate $\tilde{V}$ yields an expression for $\hat{z}$, 
\begin{align}
\hat{z} &= \Bigg( \frac{\Big(\frac{1-T_{RD} -T_{RD,I}}{1-T_{RD}} \Big)\hat{\chi} (1-(1+T_e)\kappa_{e}) \lambda}{\chi(\lambda -1) - \nu (x\kappa_c  + (1-\mathbbm{1}^{NCA})(1 - (1-T_{RD} - T_{RD,I})(1-(1+T_e)\kappa_e)\lambda)) } \Bigg)^{1/\psi} \label{eq:effort_entrant_all}
\end{align}

From here, the rest of the model can be solved using
\begin{align}
\hat{\tau} &= \hat{\chi} \hat{z}^{1-\psi} \\
z &= \bar{L}_{RD} - \hat{z} \label{eq:labor_resource_constraint_all}\\ 
\tau &= \chi z \\
\tau^S &= (1-\mathbbm{1}^{NCA}) \nu z \\
g &= (\lambda - 1) (\tau + \tau^S + \hat{\tau}) \\
r &= \theta g + \rho \\
\tilde{V} &= \frac{\tilde{\pi}}{r + \hat{\tau}} \\ 
\hat{w}_{RD} &= \begin{cases}
(1-T_{RD})^{-1}\hat{\chi} \hat{z}^{-\psi} (1-(1+T_e)\kappa_e) \lambda \tilde{V} &\textrm{, if } \hat{z} > 0\\
\Big( \chi(\lambda -1) - \nu (x\kappa_c + (1-\mathbbm{1}^{NCA})\bar{\bar{\kappa}}_c)\Big) \tilde{V} &\textrm{, o.w.}
\end{cases} \label{eq:wage_rd_labor_all}
\end{align}



\subsection{DRS incumbent innovation technology}

In this section I show why it is analytically convenient to have CRS innovation on the incumbent. Without it, the model must be solved numerically.

\begin{proposition}
	Suppose $z$ units of R\&D yields a Poisson rate
	\begin{align}
	\chi z^{1-\psi}  
	\end{align}
	for the incumbents and $\hat{z}$ units of R\&D yields a Poisson rate 
	\begin{align}
	\hat{\chi}\hat{z}^{1-\hat{\psi}}
	\end{align}
	for entrants.\footnote{Note that I have switched the notation so that $\psi$ with no hat refers to incumbents, so that it is consistent with the convention that hats go on variables related to entrants.}
	
	Consider $\psi \in [0,1)$. If $\psi = 0$, we have the baseline model, which admits a closed form solution. If $\psi = 0.5$, then $\tilde{V}$ has a closed form solution given parameters and $\hat{w}_{RD}$, but the model itself does not have a closed-form solution. For all other $\psi \in [0,1)$, there is no closed form solution for $\tilde{V}$ or the equilibrium given $\tilde{V}$.  
\end{proposition}

\begin{proof}
	The normalized incumbent HJB is now
	\begin{align}
	(r + \hat{\tau}) \tilde{V} &= \tilde{\pi} + \max_{\substack{\mathbbm{1}^{NCA} \in \{0,1\} \\ z \ge 0}} \Big\{  z \Big( z^{-\psi} \chi (\lambda - 1) \tilde{V} - \hat{w}_{RD} - (1-\mathbbm{1}^{NCA})(1 - (1-\kappa_e) \lambda)\nu \tilde{V} - \mathbbm{1}^{NCA} \kappa_c \nu \tilde{V}  \Big)   \Big\} \label{appendix:model:drsincumbent:hjb}
	\end{align} 
	
	The first order condition is
	\begin{align}
	(1-\psi) z^{-\psi} \chi (\lambda -1)\tilde{V} &= \hat{w}_{RD} + (1-\mathbbm{1}^{NCA}) (1-(1-\kappa_e)\lambda)\nu \tilde{V} + \mathbbm{1}^{NCA} \kappa_c \nu \tilde{V}
	\end{align}
	
	which implies
	\begin{align}
	z^{1-\psi} &= \Big( \frac{\hat{w}_{RD} + \zeta_1\tilde{V}}{C_2\tilde{V}} \Big)^{\frac{\psi -1}{\psi}} \\
	\zeta_1 &= (1-\mathbbm{1}^{NCA})(1-(1-\kappa_e)\lambda)\nu + x\kappa_c\nu \\
	\zeta_2 &= (1-\psi) \chi (\lambda-1)
	\end{align}
	
	Substituting into (\ref{appendix:model:drsincumbent:hjb}) yields
	\begin{align}
	(r + \hat{\tau}) \tilde{V} &= \tilde{\pi} + \Big( \frac{\hat{w}_{RD} + \zeta_1\tilde{V}}{C_2\tilde{V}} \Big)^{\frac{\psi -1}{\psi}} \zeta_2 \tilde{V} - \hat{w}_{RD} - \zeta_1 \tilde{V} 
	\end{align}
	
	This equation has no closed form expression for $\tilde{V}$ except in the quadratic cost case of $\psi = 0.5$, when $\frac{\psi - 1}{\psi} = -1$ and the above becomes
	\begin{align}
	(r + \hat{\tau}) \tilde{V} &= \tilde{\pi} +  \frac{1}{\hat{w}_{RD} + \zeta_1\tilde{V}} - \hat{w}_{RD} - \zeta_1 \tilde{V} 
	\end{align}
	
	Multiplying both sides by $\hat{w}_{RD} + \zeta_1 \tilde{V}$ yields a quadratic equation for $\tilde{V}$, which has solution
	\begin{align}
		\tilde{V} &= \frac{-b \pm \sqrt{b^2 - 4ac}}{2a}
	\end{align}
	
	However, the dependence of $\tilde{V}$ on $\hat{w}_{RD}$, given model parameters, is no longer linear. This means that $\hat{z}$ and $\hat{w}_{RD}$ are defined implicitly as the solution of two equation system, and the model must be solved numerically.	
\end{proof}

\subsection{Symmetric equilibria without $\mathbbm{1}^{NCA}_{jt} = x$}\label{appendix:model:proofs:proposition:mixedstrategyeq}

Consider the generalized growth accounting equation (it simplifies to the one in the main text when $\mathbbm{1}^{NCA}_{jt} = \mathbbm{1}^{NCA}$),
\begin{align}
	g_t &= (\lambda -1) \Big( \tau + \hat{\tau} + z \nu \int_{j : \mathbbm{1}^{NCA}_{jt} = 0} \frac{\bar{q}_{jt}}{Q_t} dj \Big)
\end{align}

Unless the integral term is constant, then $g_t$ is non-constant, even with constant $z_{jt},\hat{z}_{jt}$. The integral is equal to the product of the mass of incumbents which choose $\mathbbm{1}^{NCA}_{jt} = 0$ and the average relative quality of those incumbents $E[\frac{\bar{q}_{jt}}{Q_t} | \mathbbm{1}^{NCA}_{jt} = 0]$. 

One such construction is to suppose that a constant fraction $f \in (0,1)$ of entering firms choose $\mathbbm{1}^{NCA}_{jt} = 1$ and keep this choice throughout the entire life of the incumbent.\footnote{This choice could also alternate after each arrival of a good-specific Poisson process. Conceptually it is the same and it spans the same set of possible aggregate equilibrium outcomes. The notation is more complicated because there are inflows from new firms and from switching incumbents as well as outflows from switching incumbents.} The resulting problem is significantly simplified by the fact that the optimal $z_{jt}$ does not depend on the choice of $\mathbbm{1}^{NCA}_{jt}$ (where the incumbent is indifferent). Still, there is some additional complexity coming from two main sources. Comparing BGPs, if $f$ increases, so that entering firms are more likely to choose $\mathbbm{1}^{NCA}_{jt} = 1$, then $\mathbb{E}[\frac{\bar{q}_{jt}}{Q_t} | \mathbbm{1}^{NCA}_{jt} = 1]$ increases as well because new firms are of higher average quality than incumbents. And, for a given $f$, $\mathbbm{1}^{NCA}_{jt} = 1$ firms tend to get replaced less often and hence $\mathbb{E}[\frac{\bar{q}_{jt}}{Q_t} | \mathbbm{1}^{NCA}_{jt} = 1]$ is lower as it puts more weight on older, hence lower quality, incumbents. That is, even for $f = 1/2$ one has $\mathbb{E}[\frac{\bar{q}_{jt}}{Q_t} | \mathbbm{1}^{NCA}_{jt} = 1]$ < $\mathbb{E}[\frac{\bar{q}_{jt}}{Q_t} | \mathbbm{1}^{NCA}_{jt} = 0]$.\footnote{An alternative which sidesteps this issue is to use a Cobb-Douglas aggregator. This requires tracking limit pricing, but has the advantage that the relevant measure of aggregate quality is average of log quality, which means proportional improvements have a given effect on the aggregate regardless of the quality on which they improve.}

With the preceding discussion in mind, below I give a proof of Proposition \ref{proposition:mixedstrategyeq}.

\begin{proof}
	\textbf{[Finish up this proof, and get rid of first part about $m^x$ because second part about $m^x \Gamma^x$ is sufficient.]}
	
	The proof makes concrete the argument in the preceding paragraph. Relative to the baseline model, the only substantial modification is that one needs to derive an expression for the evolution over time of $\mathbb{E}[\frac{\bar{q}_{jt}}{Q_t} | \mathbbm{1}^{NCA}_{jt} = \mathbf{x}]$ for $\mathbf{x} \in \{0,1\}$, and set it equal to zero. This expression will involve the fraction $f \in (0,1)$. 
	
	It will be useful to work with the time-varying $\mathbb{E}[\bar{q}_{jt} | \mathbbm{1}^{NCA}_{jt} = \mathbf{x}]$ instead. To relieve cumbersome notation, let $\Gamma^{\mathbf{x}} = \mathbb{E}[\bar{q}_j | x_j = \mathbf{x}]$ and let $m^{\mathbf{x}}$ denote the mass of goods $j$ which have $x_j = \mathbf{x}$ (in general time-varying, but will be constant on this BGP). In general, $m^0,m^1$ satisfy
	\begin{align}
		\dot{m}^0 &= \overbrace{-(\hat{\tau} + \nu z) m^0}^{\mathclap{\text{outflow}}} + \overbrace{(\hat{\tau} + m^0 \nu z) (1-f)}^{\mathclap{\text{inflow}}}\\
		\dot{m}^1 &= \underbrace{-\hat{\tau} m^1 }_{\mathclap{\text{outflow}}} + \underbrace{(\hat{\tau} + m^0 \nu z) f}_{\mathclap{\text{inflow}}}
	\end{align}
	
	On the BGP, $\dot{m}^{\mathbf{x}} = 0$. Imposing this above yields
	\begin{align}
		m^0 &= \frac{\hat{\tau}}{\hat{\tau} + f \nu z} (1-f)   \label{appendix:model:mixedstrategyeq:m0}\\
		m^1 &= \frac{\hat{\tau} + m^0 \nu z}{\hat{\tau}} f   \label{appendix:model:mixedstrategyeq:m1}
	\end{align}
	
	Next, $m^0\Gamma^0,m^1\Gamma^1$ have their own evolution equations,
	\begin{align}
		m_{t+\Delta}^0\Gamma^0_{t+\Delta} &= \overbrace{(1 - \underbrace{(\hat{\tau} + \nu z) \Delta }_{\mathclap{\text{outflow from creative destruction}}} ) m^0_t  \Gamma_t^0}^{\mathclap{\text{Persisting incumbents}}} + \overbrace{\tau \Delta (\lambda - 1)m_t^0 \Gamma_t^0}^{\mathclap{\text{Improving incumbents}}} + \overbrace{(1-f) \lambda \Big( \underbrace{(\hat{\tau} + \nu z) \Delta m_t^0  \Gamma_t^0}_{\mathclap{\text{from } \mathbbm{1}^{NCA}_{jt} = 0}} +  \underbrace{\hat{\tau} \Delta m_t^1  \Gamma_t^1}_{\mathclap{\text{from } \mathbbm{1}^{NCA}_{jt} = 1}} \Big)}^{\mathclap{\text{Inflows}}} + o(\Delta)\\
		m_{t+\Delta}^1 \Gamma^1_{t+\Delta} &= \overbrace{(1 - \underbrace{\hat{\tau} \Delta }_{\mathclap{\text{outflow from creative destruction}}}) m^1_t  \Gamma_t^1}^{\mathclap{\text{Persisting incumbents}}}  + \overbrace{\tau \Delta (\lambda -1 ) m_t^1 \Gamma_t^1}^{\mathclap{\text{Improving incumbents}}}  + \overbrace{f \lambda \Big( (\hat{\tau} + \nu z) \Delta m_t^0  \Gamma_t^0 + \hat{\tau} \Delta m_t^1  \Gamma_t^1 \Big)}^{\mathclap{\text{Inflows}}} + o(\Delta) 
	\end{align}
	
	where $o(\Delta)$ has the usual meaning that $\lim_{\Delta \to 0} \frac{o(\Delta)}{\Delta} = 0$. Subtracting $m_t^{\mathbf{x}} \Gamma_t^{\mathbf{x}}$, dividing by $\Delta$ and taking the limit as $\Delta \to 0$, and finally using the fact that on the BGP $m_t^{\mathbf{x}} = m_{t+\Delta}^{\mathbf{x}} = m^{\mathbf{x}}$, yields
	\begin{align}
		m^0 \dot{\Gamma}_t^0 &= -(\hat{\tau} + \nu z) m^0 \Gamma_t^0 + \tau (\lambda - 1) m^0 \Gamma_t^0 + (1-f)\lambda \Big( (\hat{\tau} + \nu z) m^0 \Gamma_t^0 + \hat{\tau} m^1 \Gamma_t^1 \Big) \\
		m^1 \dot{\Gamma}_t^1 &= -\hat{\tau} m^1 \Gamma_t^1 + \tau (\lambda - 1) m^1 \Gamma_t^1 + f\lambda \Big( (\hat{\tau} + \nu z) m^0 \Gamma_t^0 + \hat{\tau} m^1 \Gamma_t^1 \Big)
	\end{align}
	
	Dividing by $m^x \Gamma_t^x$ yields
	\begin{align}
	\frac{\dot{\Gamma}_t^0}{\Gamma_t^0} &= -( \hat{\tau} + \nu z) + \tau (\lambda - 1) + (1-f)\lambda \Big( (\hat{\tau} + \nu z) + \hat{\tau} \frac{m^1 \Gamma_t^1 }{m^0 \Gamma_t^0}\Big) \\
	\frac{\dot{\Gamma}_t^1}{\Gamma_t^1} &= -\hat{\tau}  + \tau (\lambda - 1) + f\lambda \Big( (\hat{\tau} + \nu z) \big(\frac{m^1 \Gamma_t^1}{m^0 \Gamma_t^0}\big)^{-1} + \hat{\tau}  \Big)
	\end{align}
	
	In a BGP, we need $m^{\mathbf{x}} \mathbb{E} [\frac{\bar{q}_{jt}}{Q_t} | \mathbbm{1}^{NCA}_{jt} = \mathbf{x}]$ to be constant for each value of $x$. To show that here (given that we have shown that $m^0,m^1$ above are both in $(0,1)$ and hence valid steady state masses of firms in each state) it is sufficient to show that $\mathbb{E}[ \bar{q}_{jt} | \mathbbm{1}^{NCA}_{jt} = \mathbf{x}]$ grows at the same (geometric) rate independent of $\mathbf{x}$. Setting $\frac{\dot{\Gamma}_t^0}{\Gamma_t^0} = \frac{\dot{\Gamma}_t^1}{\Gamma_t^1}$ and multiplying both sides by $\frac{m^1 \Gamma_t^1}{m^0 \Gamma_t^0}$ yields a quadratic equation in $\frac{m^1 \Gamma_t^1}{m^0 \Gamma_t^0}$,
	\begin{align}
		[in progress]
	\end{align}
	
	[\textbf{Prove that the equation has only one positive solution}] The BGP value of $\frac{m^1 \Gamma_t^1}{m^0 \Gamma_t^0}$ is determined by the unique positive solution to this quadratic equation. Then using $Q_t = m^0 \Gamma_t^0 + m^1 \Gamma_t^1$ one can solve for $m^{\mathbf{x}} \Gamma_t^{\mathbf{x}}$ in terms of $Q_t$ and other equilibrium variables. In particular, one can compute the (constant) value of $\frac{m^0 \Gamma_t^0}{Q_t}$, which is necessary for computing the BGP growth rate from individual policies using the generalized accounting equation, which in this case can be written 
	\begin{align}
	g &= (\lambda - 1) (\tau + \hat{\tau} + z \nu  \frac{m^0 \Gamma^0_t}{Q_t} )
	\end{align}
	
	Other than this equation, the equations characterizing the BGP are the same as in the main text. The only differences in the equilibrium are through general equilibrium effects of the change in $g$: the BGP interest rate will be higher for lower $f$ due to higher growth, and this will tend to reduce R\&D wages as well. The incumbent value $\tilde{V}$ may increase or decrease.
\end{proof}

\subsubsection{Application to model with static heterogeneity}\label{appendix:model:heterogeneity}

\textbf{[Rewrite this - because evolution of individual variable is exogenous, it can be dynamic and still be tractable]} The above construction and derivation can be adapted to a richer model where there is heterogeneity in $\{\kappa_e, \kappa_c, \nu\}$ across goods $j$, inducing heterogeneity in chosen $\{z_j,x_j\}$. A tractable BGP in this setup only requires two things: that the state variable be constant throughout the life of the incumbent and that the Markov process by which goods $j$ move between states satisfy a monotone mixing condition. The latter condition essentially requires that there be no "absorbing subset" of states. This is similar to the standard necessary conditions for the existence of a stationary equilibrium in models with heterogeneous agents.

The density $\mu(\mathbbm{1}^{NCA})$ can be derived using the Kolmogorov forward equation,
\begin{align}
0 &= [in progress]
\end{align}
The system of difference equations for $m^{\textbf{x}} \Gamma_t^{\textbf{x}}$ are replaced by a functional difference equation, 
\begin{align}
\mu(\mathbbm{1}^{NCA}) \Gamma_{t+\Delta}^x &= (1- \text{CD}(\mathbbm{1}^{NCA}) \Delta) \mu(\mathbbm{1}^{NCA}) \Gamma_t^x + \text{OI}(\mathbbm{1}^{NCA}) \Delta (\lambda -1) \mu(\mathbbm{1}^{NCA}) \Gamma_t^x +  j^x \Delta  \lambda \int_{x' \in \mathbf{X}} \text{CD}(x') \Gamma_t^{x'} \mu(x') dx'
\end{align}
where $j^x$ is the density of the injection rate into state $x$ out of new incumbents. This can then be used to derive a functional differential equation,
\begin{align}
\frac{\dot{\Gamma}_{t}^x}{\Gamma_t^x} &= OI(\mathbbm{1}^{NCA}) (\lambda -1) - CD(\mathbbm{1}^{NCA}) + j^x \lambda (\mu(\mathbbm{1}^{NCA}) \Gamma_t^x)^{-1} \int_{x' \in \mathbf{X}} \text{CD}(x') \Gamma_t^{x'} \mu(x') dx'
\end{align}

Imposing the condition $\frac{\dot{\Gamma}_{t}^x}{\Gamma_t^x} = g$ for an unknown constant $g$ pins down the ratio $\frac{\int_{x' \in \mathbf{X}} \text{CD}(x') \Gamma_t^{x'} \mu(x') dx'}{\mu(\mathbbm{1}^{NCA}) \Gamma_t^x}$ for each $x$, determining the shape of the distribution $\Gamma_t^x$ (since $\mu(\mathbbm{1}^{NCA})$ is already determined by the KF equation). If the relevant functions are differentiable, the condition can also be derived by differentiating the expression for $\frac{\dot{\Gamma}_t^x}{\Gamma_t^x}$ with respect to $x$ and setting it equal to zero. This yields
\begin{align}
0 = \text{OI}'(\mathbbm{1}^{NCA}) (\lambda -1) - \text{CD}'(\mathbbm{1}^{NCA}) + \lambda \int_{x' \in \mathbf{X}} \text{CD}(x') \Gamma_t^{x'} \mu(x') dx' \Big(\frac{d}{dx} j^x \mu(\mathbbm{1}^{NCA}) \Gamma_t^x \Big)^{-1}
\end{align} 

where one would need to expand the last derivative using the product rule (recalling that all three terms depend on $x$). 

The scale of the distribution at time $t$ is determined by 
\begin{align}
\int_{x' \in \mathbf{X}} \Gamma_t^{x'} \mu(x') dx' = Q_t
\end{align}




\section{Calibration}\label{appendix:calibration}

\subsection{Computing model moments}

\subsubsection{Profits / GDP}\label{appendix:calibration:profits/gdp}

In the model, this ratio is simple to calculate using the solution to the static equilibrium as $\tilde{\pi} / \tilde{Y}$.

\subsubsection{R\&D / GDP}\label{appendix:calibration:rd/gdp}

In the model, the R\&D share is the ratio of the wage paid to R\&D workers to GDP. This is
\begin{align*}
\frac{\textrm{R\&D wage bill}}{\textrm{GDP}} &= \frac{w_{RD} z + \hat{w}_{RD} \hat{z}}{\tilde{Y}} \\ 
&= \frac{\hat{w}_{RD} (z + \hat{z}) + (w_{RD} - \hat{w}_{RD})z}{\tilde{Y}} \\
&= \frac{\hat{w}_{RD} (z + \hat{z}) - (1-\kappa_e) \lambda \tilde{V} \tau^S}{\tilde{Y}}
\end{align*}

where I used $w_{RD} - \hat{w}_{RD} = -(1-\mathbbm{1}^{NCA})(1-\kappa_e) \lambda \tilde{V} \nu$ and $\tau^S = (1-\mathbbm{1}^{NCA})\nu z$. 

\subsubsection{Growth share OI}\label{appendix:calibration:growthShareOI}

The model moment that corresponds here is the share of growth due to own innovation by incumbents of age >= 6. In the model, the fraction of OI growth due to incumbents in a given age group is exactly their fraction of employment: innovations arrive at the same rate for each incumbent, and their impact on aggregate growth is proportional to the incumbent's relative quality, which is proportional to employment. Hence old incumbents' share of growth due to own innovation is simply one minus the employment share calculated in the previous paragraph, $e^{((\hat{\tau}_I -1)g - (\hat{\tau} + (1-\mathbbm{1}^{NCA})z \nu))\cdot 6}$. Finally, the fraction of aggregate growth due to OI is $\hat{\tau}_i$, defined above. The fraction of growth due to incumbents of age at least 6 is the product of the two, 
\begin{align*}
\textrm{Age >= 6 share of OI} &= \hat{\tau}_I \frac{\ell(6)}{\ell(0)} \\
&= \hat{\tau}_I e^{((\hat{\tau}_I -1)g - (\hat{\tau} + (1-\mathbbm{1}^{NCA})z \nu))\cdot 6} 
\end{align*}


\subsubsection{Entry rate}\label{appendix:calibration:entryRate}

Let $\ell(a)$ denote the density of incumbent employment at age $a$ incumbents. Then $\ell(a)$ is characterized by 
\begin{align*}
\ell(a) &= \ell(0)e^{((\hat{\tau}_I -1)g - (\hat{\tau} + \tau^S))a}  \\
1 - \hat{z} &= \int_0^{\infty} \ell(a) da
\end{align*}

where $\hat{\tau}_I = \frac{\tau}{\tau + \hat{\tau} + \tau^S}$ is the fraction of innovations that are incumbents' own innovations. 

The intuition for this characterization of $\ell(a)$ has two parts. First, because all shocks are \textit{iid} across firms in equilibrium, the law of large numbers applied to each cohort of firms implies that we can consider directly the evolution of the cohort as a whole instead of explicitly analyzing the dynamics each individual firm in the cohort.  Second, the employment of a firm is proportional to its relative quality, $l_j \propto \tilde{q}_j = q_j / Q$, as long as it is the leader. When it is no longer the leader, its employment is zero forever. Putting these two together, $\ell(a)$ must decline at exponential rate $g$ due to the increase in $Q_t$ (obsolescence), increase at rate $\hat{\tau}_I g$ due to incumbents own innovations, and decline at rate $\hat{\tau} + \tau^S$ due to creative destruction.\footnote{The second equation imposes consistency with aggregate employment; it implies $\ell(0) = -((\hat{\tau}_I -1)g - (\hat{\tau} + \tau^S))(1-\hat{z})$. The calibration does not require this explicit calculation since it is based only on employment shares.} Note that the employment density is strictly decreasing in $a$. This is because there are no adjustment costs: firms achieve their optimal scale immediately upon entry, and subsequently become obsolete (on average) or lose the innovation race to an entrant. Finally, due to the constant exponential decay of $\ell(a)$, the share of incumbent employment in incumbents of strictly less than 6 years of age is given by 
\begin{align*}
\Xi_{[0,6)} &=  1 - \frac{\ell(6)}{\ell(0)} \\
&= 1 - e^{((\hat{\tau}_I -1)g - (\hat{\tau} + \tau^S))\cdot 6}
\end{align*}  


To calculate the share of employment in incumbents age < 6, I use as denominator the employment of intermediate goods firms in the economy (including R\&D by entrants). When bringing this to the data, it is equivalent to assuming that the age-employment distribution of final goods firms is the same as that of intermediate goods firms. Using this approach, the share of overall employment in incumbents of age < 6, including R\&D performed by non-producing entrants, is equal to the previously calculated $\Xi_{[0,6)}$ multiplied by the share of total labor in incumbents $1 - L_F - \hat{z}$, added to the R\&D labor used by entrants $\hat{z}$, divided by the share of total employment in intermediates $1 - L_F$, and finally multiplied by 2/3 which is the share of creative destruction that corresponds to new firms in the data.\footnote{Alternatively, one could assume that final goods firms have the same employment-age distribution as other intermediate goods firms. Then the formula would be
	
	\begin{align*}
	\textrm{Age < 6 share of employment} &= \frac{2}{3}(\Xi_{[0,6)} (1-\hat{z}) + \hat{z})
	\end{align*}
	
	This has only minor effects on the inferred parameters. They are listed in \autoref{calibration_2_parameters}.} This yields
\begin{align*}
\textrm{Age < 6 share of employment} &= \frac{2}{3} \frac{(\Xi_{[0,6)} (1 - L_F -\hat{z}) + \hat{z})}{1-L_F}
\end{align*}

The factor $2/3$ deserves some additional discussion. According to \cite{klenow_innovative_2020}, creative destruction by incumbents is responsible for half as much growth as creative destruction by entrants. In this interpretation of the model, both types of creative destruction use the same technology. Therefore, it follows that 2/3 of employment in young firms in the model represents employment in young firms in the data.

\subsubsection{Employment share of WSOs}\label{appendix:calibration:WSOempShare}

Because successfully innovating spinouts and entrants have identical expected growth dynamics, the BGP share of employment in firms started as spinouts is their share of new incumbents $\frac{\tau^S}{\tau^S+ (\frac{2}{3})\hat{\tau}}$, multiplied by the employment share of incumbents $\frac{1-L_F- (\frac{2}{3})\hat{z}}{1-L_F}$, 
\begin{align*}
\textrm{Spinout employment share} &= \frac{\tau^S}{\tau^S + \frac{2}{3}\hat{\tau}} \times \frac{1-L_F- (\frac{2}{3})\hat{z}}{1-L_F} 
\end{align*}

Again, the factor 2/3 is because this is the fraction of entrants in the model which the calibration maps to new firms in the data.

\section{Policy analysis}

\subsection{NCA cost $\kappa_c$}\label{appendix:policyanalysis:ncacost}

\paragraph{Robustness of welfare gain from NCA enforcement}

\autoref{welfareComparisonSensitivityFull} shows the sensitivity of the welfare comparison the moments targeted, including the externally calibrated parameters as pseudo-moments as before. It is computed as $\nabla_m \tilde{W}|_m = (J^{-1})^T \nabla_p W|_p$, where $J$ is the Jacobian of the mapping from log parameters to moments (so that $J^{-1}$ is the Jacobian of the inverse mapping), and $W$ is the mapping from parameters the log \% change (or raw \% change, in \autoref{levelsWelfareComparisonSensitivityFull})) in CEV welfare from reducing $\kappa_c$ from $\infty$ to $0$. That is, it is the gradient of the change in welfare to the change in target moments or uncalibrated parameters, taking as given the change in parameters required to continue matching the target moments. For reference, $\nabla_p W|_p$  for each definition of $W$ can be found in \autoref{welfareComparisonParameterSensitivityFull} and \autoref{levelsWelfareComparisonParameterSensitivityFull}.

To get a sense of what this means about robustness of the results, suppose that the log of each moment is assumed to have a standard deviation of $\sigma$ and that this uncertainty is uncorrelated across moments. The uncertainty propagates such that the standard deviation of the CEV welfare change is the square root of $(\nabla_m \tilde{W}|_m)^T \Sigma_m \nabla_m \tilde{W}|_m$, where $\Sigma_m = \sigma^2 I_{9\times 9}$. In this examples this yields 6.9$\sigma$ log points of uncertainty, or about 50\% of the result for $\sigma = 0.1$. I conclude that the result is in fact somewhat sensitive to the measurement of the moments. Because many decisions went into the computation of the moments I used (as well as their model counterparts), it is crucial in future work to find more rigorous ways to bring this and similar models to the data.


\begin{figure}[]
	\includegraphics[scale = 0.36]{../code/julia/figures/simpleModel/welfareComparisonSensitivityFull.pdf}
	\caption{Sensitivity of welfare comparison to moments. This is $(J^{-1})^T \nabla_p W$, where $W(p)$ maps log parameters to the log of the percentage change in BGP consumption which is equivalent to the change in welfare from changing $\kappa_c$ from $\infty$ to $0$ (i.e. going from banning to frictionlessly enforcing NCAs). The way to read this is the following. Looking at the column labeled \textit{E}, the chart says that a 1\% increase in the targeted employment share of young firms, which corresponds to a log change of about $0.01$, leads to a 4\% increase in the percentage CEV percentage welfare change. In this calibration it is about 1.42\%, so this is about $0.057$ percentage points.}
	\label{welfareComparisonSensitivityFull}
\end{figure}


\paragraph{When are NCAs bad for welfare?}

The sensitivity of the welfare improvement to the entry rate shown in (\ref{welfareComparisonSensitivityFull}) suggests that a calibration targeting a lower rate of creative destruction could have the opposite result. \autoref{calibration_lowEntry_summaryPlot} shows the analogue of \autoref{calibration_summaryPlot} if entry rate targeted is 8\% instead of 13.34\%. The model is again able to match the moments exactly; inferred parameter values are shown in \autoref{calibration_lowEntry_parameters}.

In this low entry calibration, growth and welfare fall when $\kappa_C$ is reduced to zero. The lower employment in young firms (while holding constant the fraction of growth coming from old firms) means that each entry by a young firm must have a higher effect on growth in order for the model to match the growth rate. Furthermore, as shown in \autoref{welfareComparisonParameterSensitivityFull}, the increase in $\lambda$ eliminates the overall welfare gain from reducing $\kappa_c$. As discussed previously, a higher value of $\lambda$ brings (\ref{cs:growth_misallocation_condition}) closer to unity, weakening the growth increase from reallocation of R\&D to OI. Closer inspection reveals that the new calibration also chooses a substantially higher value of $\kappa_e$. This works in the opposite direction, as a reallocation of R\&D to OI reduces the entry cost paid. However, on net, it is still the case that a reduction of $\kappa_c$ from $\bar{\kappa}_c$ to zero reduces both growth and welfare.


\begin{figure}[]
	\includegraphics[scale = 0.57]{../code/julia/figures/simpleModel/calibration_lowEntry_summaryPlot.pdf}
	\caption{Summary of equilibrium for baseline parameter values and various values of $\kappa_c$.}
	\label{calibration_lowEntry_summaryPlot}
\end{figure}

\begin{table}[]
	\centering
	\captionof{table}{Low entry rate calibration}\label{calibration_lowEntry_parameters}
	\begin{tabular}{rlll}
		\toprule \toprule
		Parameter & Value & Description & Source \tabularnewline
		\midrule
		$\rho$ & 0.0303 & Discount rate  & Indirect inference \tabularnewline
		$\theta$ & 2 & $\theta^{-1} = $ IES & External calibration 
		\tabularnewline
		$\beta$ & 0.094 & $\beta^{-1} = $ EoS intermediate goods & Exactly identified \tabularnewline 
		$\bar{L}_{RD}$ & 0.01 & R\&D labor allocation  & Exactly identified \tabularnewline
		$\psi$ & 0.5 & Entrant R\&D elasticity & External calibration \tabularnewline
		$\lambda$ & 1.23 & Quality ladder step size & Indirect inference 
		\tabularnewline
		$\chi$ & 5.861 & Incumbent R\&D productivity & Indirect inference 
		\tabularnewline
		$\hat{\chi}$ & 0.370 & Entrant R\&D productivity & Indirect inference \tabularnewline 
		$\kappa_e$ & 0.885 & Non-R\&D entry cost & Indirect inference \tabularnewline
		$\nu$ & 0.144 & Spinout generation rate  & Indirect inference\tabularnewline
		\bottomrule
	\end{tabular}
\end{table}









\end{document}