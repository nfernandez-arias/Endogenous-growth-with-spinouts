\documentclass[english,usenames,dvipsnames]{beamer}
\usetheme{Boadilla}
\beamertemplatenavigationsymbolsempty
\setbeamertemplate{footline}[frame number]
\setbeamercolor{alerted text}{fg=blue1}
%\setbeamercolor{frametitle}{fg=blue2}
\usepackage[utf8]{inputenc}
\usepackage{caption}
\usepackage{booktabs}
\usepackage{appendixnumberbeamer}
\usepackage{babel}
\usepackage{amsmath}
\usepackage{mathtools}
\usepackage[nocomma]{optidef}
\usepackage{hyperref}
\usepackage{geometry}
\usepackage{bbm}
\usepackage{tikz}
\usepackage{amsthm}
\usepackage{bm}
\usepackage{verbatim}
%\usepackage{palatino}
\definecolor{red1}{RGB}{155,50,0}
\definecolor{blue1}{RGB}{0,0,155}
\definecolor{blue2}{rgb}{0.22,0.37,1}
\definecolor{green1}{RGB}{34,139,35}

\setbeamertemplate{itemize items}[default]

\setbeamersize{text margin left=10mm,text margin right=10mm} 


\title{Endogenous Growth with Employee Spinouts, Noncompetes, and Creative Destruction}
\author{Nicolas Fernandez-Arias \\ Princeton University}
\date{Economics Graduate Student Conference \\ \today }


\newtheorem{proposition}{Proposition}
\newtheorem{proposition_corollary}{Corollary}[proposition]
%\newtheorem{lemma}{Lemma}
%\newtheorem{lemma_corollary}{Corollary}[lemma]

\begin{document}

\maketitle

\section{Introduction}


\begin{frame}{Background}\label{motivation_background}
	\begin{itemize}
		\item<+-> Long-run output per capita growth driven by \alert{\textbf{productivity}} growth \hyperlink{economic_growth_facts}{\beamergotobutton{growth accounting}}
		\medskip
		\item<+-> Decomposition (Garcia-Macia et al 2019, Klenow-Yi 2020)
		\begin{itemize}
			\item<.-> Own-product innovation $\approx 50\%$
			\item<.-> Creative destruction $\approx 25\%$
			\item<.-> New varieties $\approx 25\%$
		\end{itemize}
	\end{itemize}
\end{frame}

\begin{frame}{Motivation}\label{motivation}
\begin{itemize}
	\item<+-> \alert{\textbf{Own-product innovation R\&D}} leads to \alert{\textbf{more employee entrepreneurship}}
	\begin{itemize}
		\item \alert{\textbf{within-industry spinouts (WSOs)}}
		\hyperlink{spinouts_examples}{\beamergotobutton{examples}} \hyperlink{spinouts_facts_from_literature}{\beamergotobutton{stylized facts}} 
		\item evidence in this paper, Babina \& Howell 2019
		\item interpretation: R\&D workers \alert{\textbf{learn by doing}} 
	\end{itemize}
	\smallskip
	\item<+-> Competition from spinouts $\Rightarrow$ \alert{\textbf{disincentive to own-product innovation}}
	\smallskip
	\item<+-> \alert{\textbf{Non-compete agreements (NCAs)}} can prevent within-industry spinouts $\Rightarrow$ \alert{\textbf{economic tradeoff}}
	\begin{itemize}
		\item<.-> mitigate disincentive to own-product innovation R\&D
		\item<.-> \ldots but prevent socially valuable spinout innovation
		\item<.-> net effect not obvious
	\end{itemize}
	\smallskip
	\item<+-> Ongoing \alert{\textbf{policy debate}}
	\begin{itemize}
		\item states passing laws restricting enforcement of noncompetes (e.g., Hawaii in 2015, Massachussetts and Maryland in 2019) 
		\item emulation of Silicon Valley
	\end{itemize}
\end{itemize}
\end{frame}

\begin{frame}{This project}
\begin{itemize}
	\item Provide evidence that R\&D leads to within-industry employee spinouts
	\smallskip
	\item Develop GE model of endogenous growth with R\&D-induced WSOs and non-compete agreements
	\smallskip
	\item Calibrate model to match empirics, aggregate data
	\smallskip
	\item Quantify WSO contribution to productivity growth
	\smallskip
	\item Study effect of policy on aggregate growth and welfare
	\begin{itemize}
		\item reducing barriers to use of NCAs
		\item R\&D subsidies
	\end{itemize}
\end{itemize}
\end{frame}

\begin{frame}{Findings}\label{intro_findings}
	\begin{itemize}
		\item<+-> Empirics of R\&D and within-industry spinouts
		\begin{itemize}
			\item corporate R\&D associated with future WSO formation
			\item relationship is \alert{\textbf{economically significant}} and can \alert{\textbf{account for $\approx$ 10\%}} of employment in VC-funded startups   
		\end{itemize}
		\medskip
		\item<+-> Calibrated model and policy analysis
		\begin{itemize}
			\item WSOs \alert{\textbf{account for $\approx$ 10\%}} of productivity growth
			\item eliminating barriers to NCAs \alert{\textbf{increases welfare by 3.24\%}} (CE)
			\item optimal policy: \alert{\textbf{ban NCAs}} and \alert{\textbf{subsidize own-product R\&D}} 
			\item untargeted R\&D subsidies can \alert{\textbf{misallocate R\&D}} and \alert{\textbf{induce the use of NCAs}}
		\end{itemize}
		\medskip
		\item<+-> Bigger picture
		\begin{itemize}
			\item \alert{\textbf{allocation}} of R\&D matters, not just \alert{\textbf{amount}} of R\&D
		\end{itemize}
	\end{itemize}
\end{frame}

\begin{frame}{Related literature}
\begin{itemize}
\small
\item Firm dynamics and endogenous growth
\begin{itemize}
\item Romer 1986, Grossman \& Helpman 1991, Aghion \& Howitt 1992, Klette \& Kortum 2004, Acmemoglu \& Akcigit 2012, Akcigit \& Kerr 2017, Acemoglu \& Cao 2015
\end{itemize}
\smallskip
\item Models of employee spinouts
\begin{itemize}
\item Klepper 2002, Klepper \& Sleeper 2005, Anton \& Yao 1994/1995, Franco \& Filson 2006, Franco \& Mitchell 2008, Rauch 2015, Rossi-Hansberg \& Chatterjee 2012, Baslandze 2019
\end{itemize}
\smallskip
\item Empirics on employee mobility, spinouts
\begin{itemize}
\item Spawning of spinouts: Gompers et al. 2005, Klepper \& Sleeper 2005, Klepper 2007, Garmaise 2011, Baslandze 2019, Babina \& Howell 2019
\item Characteristics of spinouts: Muendler 2012
\item Effect on parent firms: Campbell et. al 2012, Wezel et al. 2006
\end{itemize}
\smallskip
\item Noncompetes
\begin{itemize}
	\item Garmaise 2009, Marx et al 2009, Samila-Sorenson 2011, Jeffers 2018, Shi 2018, Starr et al 2015 and 2017, Starr 2019, Balasubramanian et al 2020
\end{itemize}
\end{itemize}
\end{frame}

\section{Empirics of R\&D and employee spinouts}

\begin{frame}
	\tableofcontents[currentsection]
\end{frame}

\begin{frame}{Overview}
	\begin{itemize}
		\item Construct dataset matching incumbents to employee spinouts
		\smallskip
		\item Document empirical relationship between R\&D and subsequent employee spinout formation
		\begin{itemize}
			\item caveat: no definitive causality bc R\&D and spinout formation are endogenous 
		\end{itemize}
	\end{itemize}
\end{frame}

\begin{frame}{Consructing dataset}\label{empirics:constructing_dataset}
	\begin{itemize}
		\item<+-> \alert{\textbf{Venture Source:}} 
		\begin{itemize}
			\item deal and company data for US-based startups funded by VCs
			\item \alert{\textbf{employment biographies}} for founders / C-level / board members
			\item subsample of startups founded in US between 1987 and 2009
			\item $\approx$ 21,600 startups, 40,000 founders, 90,000 financing rounds
		\end{itemize}
		\medskip
		\item<+-> Merge with \alert{\textbf{Compustat:}}
		\begin{itemize}
			\item company data on publicly traded firms 
			\item R\&D spending and other time-varying firm controls
			\item match by parsing previous employer from Venture Source and string matching to company name \hyperlink{empirics:wso_heatmap}{\beamergotobutton{prevalence of WSOs}}
		\end{itemize}
		\medskip
		\item<+-> Merge with \alert{\textbf{NBER-USPTO patent data}}
		\begin{itemize}
			\item data on all USPTO-registered patents and their citations (also data on inventors, associated firms)
			\item crosswalk to Compustat firms
		\end{itemize}
	\end{itemize}
\end{frame}

\begin{frame}{Corporate R\&D is associated with spinout formation}\label{empirics:scatterplot}
	\begin{figure}[!htb]
		\centering
		\includegraphics[scale= 0.5]{../empirics/figures/scatterPlot_RD-FoundersWSO4_dIntersection.png}
		\caption{\footnotesize Binned scatterplot of average firm-level yearly within-industry spinout founder counts in $t+1,t+2,t+3$ versus average yearly R\&D spending in $t,t-1,t-2$. Both variables are demeaned, first at firm level then at the industry-state-age-year level.}
	\end{figure}
\end{frame}

\begin{frame}{Regression: R\&D predicts employee spinout formation}
	\begin{table}
		\tiny
		\centering
		{
\def\sym#1{\ifmmode^{#1}\else\(^{#1}\)\fi}
\begin{tabular}{l*{3}{c}}
\toprule
                    &\multicolumn{1}{c}{(1)}&\multicolumn{1}{c}{(2)}&\multicolumn{1}{c}{(3)}\\
                    &\multicolumn{1}{c}{WSO4}&\multicolumn{1}{c}{$\frac{\textrm{WSO4}}{\textrm{Assets}}$}&\multicolumn{1}{c}{WSO4}\\
\midrule
R\&D                &        0.28\sym{***}&                     &                     \\
                    &     (0.058)         &                     &                     \\
\addlinespace
$\frac{\textrm{R\&D}}{\textrm{Assets}}$&                     &        0.25\sym{**} &                     \\
                    &                     &      (0.12)         &                     \\
\addlinespace
log(R\&D)           &                     &                     &        1.06\sym{***}\\
                    &                     &                     &      (0.27)         \\
\addlinespace
Firm FE             &         Yes         &         Yes         &         Yes         \\
\addlinespace
Age FE              &         Yes         &         Yes         &         Yes         \\
\addlinespace
Industry-Year FE    &         Yes         &         Yes         &         Yes         \\
\addlinespace
State-Year FE       &         Yes         &         Yes         &          No         \\
\midrule
Clustering          &naics4 Statecode         &naics4 Statecode         &       gvkey         \\
R-squared (adj.)    &        0.63         &        0.28         &                     \\
R-squared (within, adj)&        0.26         &      0.0014         &                     \\
Observations        &       57956         &       57948         &         536         \\
\bottomrule
\multicolumn{4}{l}{\tiny Standard errors in parentheses}\\
\multicolumn{4}{l}{\tiny \sym{+} \(p<0.2\), \sym{++} \(p<0.15\), \sym{*} \(p<0.1\), \sym{**} \(p<0.05\), \sym{***} \(p<0.01\)}\\
\end{tabular}
}

		\caption{\scriptsize The dependent variable is average yearly number of founders joining WSOs in years $t+1,t+2,t+3$. The independent variables are averages over $t,t-1,t-2$. Firm controls include are employment, assets, intangible assets, investment, net income, cumulative citation-weighted patents, and Tobin's Q.}
	\end{table}
\end{frame}


\section{Model}

\begin{frame}
	\tableofcontents[currentsection]
\end{frame}

\begin{frame}{Model}
	\begin{itemize}	
		\item Quality ladder model of endogenous growth through creative destruction 
		\begin{itemize}
			\item builds on Grossman-Helpman 1991
		\end{itemize}
		\medskip
		\item \alert{\textbf{New:}} R\&D workers \textbf{\alert{acquire ability to form spinouts}} on the job
		\medskip
		\item \alert{\textbf{New:}} Non-compete agreements
		\begin{itemize}
			\item shut down formation of spinouts, \textbf{\alert{shrinking production possibilities frontier}}
			\item \textbf{\alert{may increase welfare}} due to effect on incumbent R\&D incentives
		\end{itemize}
	\end{itemize}
\end{frame}

\begin{frame}{Individual endowments and preferences}
	\begin{itemize}
		\item Continuous time, $t \ge 0$
		\item Rep. household has CRRA preferences 
		\begin{align*}
		U(\{C(t)\}_{t \ge 0}) &= \mathbb{E} \int_0^{\infty} e^{-\rho t} \frac{C(t)^{1-\theta} - 1}{1 - \theta} ds
		\end{align*}
		\item Labor endowment
		\begin{itemize}
			\item $\bar{L}_{RD}$ units R\&D labor
			\item $1 - \bar{L}_{RD}$ units of production labor
		\end{itemize}
	\end{itemize}
\end{frame}

\begin{frame}{Production technology and market structure}\label{model:production_overview}
	\begin{itemize}
		\item Continuum of intermediate goods $j \in [0,1]$ \hyperlink{intermediate_goods_production}{\beamergotobutton{details}}
		\begin{itemize}
			\item finite set of qualities $\{q_{jti}\}_{0 \le i \le I_{jt}}$
			\item produced with production labor 
			\item monopolistic competition across $j$
			\item two-stage Bertrand game within $j$ $\Rightarrow$ no limit pricing
		\end{itemize}
		\smallskip
		\item Final good \hyperlink{main:final_goods_production}{\beamergotobutton{details}}
		\begin{itemize}
			\item produced competitively with intermediate goods and production labor
			\item CES aggregator for intermediate goods
			\item Cobb-Douglas aggregator with production labor (Uzawa's theorem)
		\end{itemize}
	\end{itemize}
\end{frame}

\begin{frame}{Innovation overview}\label{model:innovation_overview}
	\begin{itemize}
		\item<+-> Sources of innovation \hyperlink{model:innovation_by_incumbents_and_entrants}{\beamergotobutton{details}}
		\begin{itemize}
			\item own-product innovation by incumbents  
			\item creative destruction by entrants 
			\item creative destruction by spinouts
		\end{itemize}
		\medskip
		\item<+-> Innovator on good $j$ becomes new incumbent $j$
		\begin{itemize}
			\item<+-> frontier quality improves by factor $\lambda > 1$
			\item<+-> no ``catch-up" innovation $\Rightarrow$ \alert{\textbf{tractability}} (standard)
		\end{itemize}
	\end{itemize}
\end{frame}

\begin{frame}{Noncompete agreements and spinouts}
	\begin{itemize}
		\item<+-> \alert{\textbf{NCA choice:}} each $t$ incumbent $j$ chooses NCA policy
		\begin{itemize}
			\item<.-> $\mathbbm{1}^{NCA}_{jt} = 1$ $\Leftrightarrow$ NCA is imposed during $dt$
			\item<+-> to use NCA, incumbent pays flow cost 
			\begin{align*}
				 \kappa_c \nu V(j,t | \bar{q}_{jt}) z_{jt}, \quad \kappa_c \ge 0, \nu \ge 0
			\end{align*}
			\vspace{-17pt} 
			\begin{itemize}
				\item<+-> $z_{jt}$ is flow R\&D effort; $\frac{\bar{q}_{jt}}{Q_t} z_{jt}$ is R\&D labor
				\item<+-> $V(j,t|\bar{q}_{jt})$ is \alert{\textbf{endogenous}} value of incumbency $\Rightarrow$ \alert{\textbf{tractability}} 
			\end{itemize}
		\end{itemize}
		\bigskip
		\item<+-> Given $\mathbbm{1}^{NCA}_{jt}$, R\&D generates spinout at rate
		\begin{align*}
		\tau^S(z_{jt},\mathbbm{1}^{NCA}_{jt}) &= \overbrace{(1-\mathbbm{1}_{jt}^{NCA})}^{\mathclap{= 0 \text{ when NCA is imposed}}} \nu z_{jt}
		\end{align*}
	\end{itemize}
\end{frame}

\begin{frame}{Creative destruction cost}
	\begin{itemize}
		\item Entrants and spinouts pay one-time creative destruction cost
		\begin{align*}
			\kappa_e V(j,t|\lambda \bar{q}_{jt}), \quad \kappa_e \in [0,1)
		\end{align*}
		\smallskip
		\item \alert{\textbf{Interpretation:}} non-R\&D costs of developing a new product vs. improving an existing product
	\end{itemize}
\end{frame}


\begin{frame}{Equilibrium}\label{definition:equilibrium}
	\hyperlink{model:firm_ownership}{\beamergotobutton{ownership of firms details}}
\begin{definition}
	\tiny
	A \emph{equilibrium} of this model consists of household consumption $C(t)$ and bond holdings $A(t)$; final good production $Y(t)$; frontier intermediate goods prices $p(j,t|q)$ and quantities $k(j,t|q)$; production wages $\bar{w}(t)$ and production labor allocation to final goods $L_{F}(t)$ and intermediate goods $\ell_I(j,t|q)$; R\&D wages paid by entrants $\hat{w}_{RD}(t)$, by incumbents using and not using noncompetes $w_{RD}(j,t|q,\mathbbm{1}^{NCA})$; R\&D labor allocations across incumbents $\ell_{RD}(j,t|q)$ and across entrants $\hat{\ell}_{RD}(j,t|q)$; and noncompete contract allocations $\mathbbm{1}^{NCA}(j,t|q)$ such that 
	\begin{enumerate}
		\item The final goods firm maximizes profits.
		\item Each incumbent $j$ optimally chooses production, R\&D labor demand, and the use of NCAs.
		\item Entrants optimize their R\&D labor demand.
		\item The representative household optimizes production and R\&D labor supply, consumption and savings.
		\item The competitive financial intermediary maximizes the discounted present value of profits remitted to the household.
		\item Markets clear (final goods, risk-free bonds in zero net supply).
	\end{enumerate}
\end{definition}
\end{frame}

\begin{frame}{Symmetric BGP equilibrium}\label{definition:symmetric_bgp}
	\hyperlink{characterizing_BGP}{\beamergotobutton{characterization}}
	\begin{definition}
		A \textbf{symmetric balanced growth path equilibrium} is an equilibrium in which there exists a growth rate $g > 0$ and R\&D efforts $z,\hat{z} \ge 0$ such that for all intermediate goods $j \in [0,1]$ and $t \ge 0$, 
		\begin{align*}
			\frac{\dot{Q}_t}{Q_t} &= \frac{\dot{C}(t)}{C(t)} = g, \\
			z_{jt} &= z, \\
			\hat{z}_{jet} &= \hat{z}.
		\end{align*}
	\end{definition}
\end{frame}

\begin{frame}{Equilibrium use of NCAs}\label{use_of_ncas}
	\begin{itemize}
		\item In eq., NCAs \alert{\textbf{maximize bilateral value}}: $\mathbbm{1}^{NCA}_{jt} \equiv 1$ iff
		\begin{align*}
		\overbrace{\kappa_c}^{\mathclap{(\nu \tilde{V})^{-1} (\text{Direct cost of NCA})}} &< \underbrace{1 - (1-\kappa_e) \lambda}_{\mathclap{(\nu\tilde{V})^{-1}\mathbb{E}[\text{Bilateral cost of spinout formation}]}} \eqqcolon \bar{\kappa}_c
		\end{align*}
		\hyperlink{use_of_ncas_details}{\beamergotobutton{details}}
	\end{itemize}
\end{frame}

\section{Calibration}


\begin{frame}
\tableofcontents[currentsection]
\end{frame}

\begin{frame}{Calibration overview}\label{calibration_overview}
	\begin{itemize}
		\item 11 parameters $\{\theta, \psi, \rho, \beta, \lambda , \chi, \hat{\chi}, \nu, \kappa_e, \kappa_c, \bar{L}_{RD} \}$
		\smallskip
		\begin{itemize}
			\item $\{\rho, \beta, \lambda ,\chi, \hat{\chi}, \nu, \kappa_e, \bar{L}_{RD} \}$ chosen to match moments, exactly identified \hyperlink{calibration_targets}{\beamergotobutton{targets}} \hyperlink{parameters}{\beamergotobutton{parameters}} 
			\item $\kappa_c > \bar{\kappa}_{c}$ set identified
			\item $\theta, \psi$ chosen from literature
		\end{itemize}
		\medskip
		\item Innovative growth accounting (Garcia-Macia et al 2019, Klenow \& Yi 2020)
		\begin{itemize}
			\item growth rate from own-product innovation and creative destruction
			\item share of growth from old vs. young firms
			\item employment share of young firms 
		\end{itemize}
		\medskip
		\item Other moments 
		\begin{itemize}
			\item share of employment in R\&D-induced within-industry spinouts	\hyperlink{economic_magnitude}{\beamergotobutton{details}} 
			\item r\&d share of GDP
		\end{itemize}
	\end{itemize}
\end{frame}

\section{Policy analysis}

\begin{frame}
\tableofcontents[currentsection]
\end{frame}

\begin{frame}{Reducing barriers to the use of noncompetes $\kappa_c$}\label{reducing_kappa_c_table}
	\begin{table}
		\centering
		\small
		\begin{tabular}{lclll}
			\toprule \toprule
			Measure & Variable & $\kappa_c > \bar{\kappa}_c$ & $\kappa_c = 0$ & Chg. \tabularnewline
			\midrule
			Growth & $g$ & 1.487\% & 1.696\% & 0.21 p.p. \tabularnewline
			Level & $\tilde{C}$  & 0.784 &  0.787 & 0.39\% \tabularnewline 
			\tabularnewline
			Welfare & $\tilde{W}$  &  & & \alert{\textbf{3.24\%}} (CEV terms)  \tabularnewline
			\bottomrule
		\end{tabular}
	\end{table}
	\hyperlink{welfare}{\beamergotobutton{definition of welfare}}
	\hyperlink{decomposition_growth_increase}{\beamergotobutton{decomposition}}
	\hyperlink{plots:reducing_kappa_c1}{\beamergotobutton{welfare decomposition plot}}
	\hyperlink{plots:reducing_kappa_c2}{\beamergotobutton{growth decomposition plot}}
	\hyperlink{robustness_to_moments}{\beamergotobutton{robustness to moments}} \hyperlink{robustness_to_parameters}{\beamergotobutton{robustness to parameters}}
	\hyperlink{efficiency}{\beamergotobutton{other efficiency}}
\end{frame}


\begin{frame}{Misallocation of R\&D}\label{misallocation_of_rd}
	\begin{itemize}
		\item <+-> Reallocation of R\&D labor to own-product innovation increases the growth rate iff \hyperlink{misallocation_of_rd:derivation}{\beamergotobutton{derivation}}
		\footnotesize
		\begin{multline*}
		1 > \alert{\overbrace{\frac{\lambda-1}{\lambda}}^{\mathclap{\text{Business stealing}}}} \times \underbrace{(1-\psi)}_{\mathclap{\text{Congestion}}}   \times \overbrace{\frac{1}{1-\kappa_{e}}}^{\mathclap{\text{Entry cost}}} \times \overbrace{\frac{\chi}{\chi + (1-\mathbbm{1}^{NCA})\nu}}^{\mathclap{\text{Spinout formation}}} \times \\ \overbrace{\frac{\chi(\lambda-1) -(1-\mathbbm{1}^{NCA}) (1-(1-\kappa_e)\lambda)\nu - \mathbbm{1}^{NCA} \kappa_c \nu}{\chi(\lambda-1)}}^{\mathclap{\text{Effective cost of R\&D}}}  \label{eq:RD_reallocation} 
		\end{multline*}
		\normalsize
		\item <+-> Key factor: \alert{\textbf{business stealing}} (.08 in calibration) produces \alert{\textbf{excessive creative destruction}} \hyperlink{other_factors}{\beamergotobutton{magnitude}}
	\end{itemize}
\end{frame}

\begin{frame}{R\&D subsidies reduce growth and welfare, induce NCAs}\label{RDsubsidy_table}
		\begin{table}
			\centering
			\small
			\caption*{Effect on growth, NCAs, consumption and welfare of untargeted R\&D subsidy}
			\begin{tabular}{lclllll}
				\toprule \toprule
				 &  & \multicolumn{4}{l}{R\&D Subsidy (\%)} \vspace{3pt} \tabularnewline
				Measure &Variable & 0 & 10 & 20 & 30 \tabularnewline
				\midrule
				Growth & $g$ & 1.49\% & 1.48\% & 1.46\% & 1.44\% \tabularnewline
				Level & $\tilde{C}$  & 0.784 &  0.784 & 0.783 & 0.783 \tabularnewline 
				NCAs & $\mathbbm{1}^{NCA}$ & 0 & 0 & 0 & \alert{\textbf{1}} \tabularnewline
				\tabularnewline
				$\Delta$ Welfare (CEV) & $\tilde{W}$  &  & -0.18\% & -0.36\% & \alert{\textbf{-0.73\%}} \tabularnewline
				\bottomrule
			\end{tabular}
		\end{table}
		\hyperlink{rd_subsidies:decomposition_growth_decrease}{\beamergotobutton{decomposition}}
		\hyperlink{plots:rd_subsidies1}{\beamergotobutton{welfare decomposition plot}}
		\hyperlink{plots:rd_subsidies2}{\beamergotobutton{growth decomposition plot}}
		\hyperlink{rd_subsidies:intuition}{\beamergotobutton{intuition}}
		\begin{itemize}
			\small
			\item $g \downarrow$: R\&D wage subsidized, cost to incumbent of future spinouts not subsidized $\Rightarrow$ GE reallocation to creative destruction
			\item $\mathbbm{1}^{NCA} \uparrow$: cost of future spinouts not subsidized $\Rightarrow$ incumbent prefers to compensate using wage
		\end{itemize}
\end{frame}

\begin{frame}{OI-targeted R\&D subsidies increase growth and welfare}\label{OI_RDsubsidy_table}
	\begin{table}
		\centering
		\caption*{Effect on growth, NCAs, consumption and welfare of targeted R\&D subsidy}
		\small
		\begin{tabular}{rllllll}
			\toprule \toprule
			 &  & \multicolumn{4}{l}{Targeted R\&D subsidy (\%)} \vspace{3pt} \tabularnewline
			Measure &Variable & 0 & 20 & 40 & 60 \tabularnewline
			\midrule
			Growth & $g$ & 1.49\% & 1.80\% & 2.01\% & 2.17\% \tabularnewline
			Level & $\tilde{C}$  & 0.784 &  0.787 & 0.789 & 0.792 \tabularnewline 
			NCAs & $\mathbbm{1}^{NCA}$ & 0 & 0 & 1 & 1 \tabularnewline
			\tabularnewline
			$\Delta$ Welfare (CEV) & $\tilde{W}$  &  & 4.47\% & 7.45\% & 9.63\% \tabularnewline
			\bottomrule
		\end{tabular}
	\end{table}
	\hyperlink{oi_rd_subsidies:decomposition_growth_decrease}{\beamergotobutton{decomposition}}
	\hyperlink{plots:oi_rd_subsidies1}{\beamergotobutton{welfare}} \hyperlink{plots:oi_rd_subsidies2}{\beamergotobutton{growth}}	
	\hyperlink{OI_RDsubsidy_intuition}{\beamergotobutton{intuition}}
	\begin{itemize}
		\small
		\item $g \uparrow$ intuition: R\&D wage for own-product innovation subsidized $\Rightarrow$ GE reallocation to R\&D
		\item $\mathbbm{1}^{NCA} \uparrow$ intuition: same as untargeted
	\end{itemize}	
\end{frame}

\begin{frame}{Optimal policy}\label{all_policies_overview}
	\begin{itemize}
		\item <+-> Optimal policy combines large targeted R\&D subsidy with ban on NCAs \hyperlink{plots:all_policies}{\beamergotobutton{plots}} 
		\begin{itemize}
			\item<+-> NCAs are \alert{\textbf{socially costly}} way to incentivize own-product innovation
			\item<.-> dominated by large targeted R\&D subsidies 	\hyperlink{intuition:all_policies}{\beamergotobutton{details}}
		\end{itemize}
	\end{itemize}
\end{frame}

\begin{frame}{Conclusion}
	\begin{itemize}
		\item R\&D spending is associated with future within-industry employee spinouts
		\medskip
		\item NCAs imply \alert{\textbf{economic tradeoff}} between \alert{\textbf{allocation of R\&D}} and \alert{\textbf{innovation by within-industry spinouts}}
		\medskip
		\item In a calibrated model, reducing NCA costs \alert{\textbf{increases growth }} ($+0.21$ p.p.), \alert{\textbf{consumption}} ($+0.4\%$) and \alert{\textbf{welfare}} ($+3.24\%$ in consumption-equivalent terms)
		\begin{itemize}
			\item NCAs work against \alert{\textbf{business stealing}} externality
		\end{itemize}
		\medskip
		\item Optimal policy is \alert{\textbf{large incumbent R\&D subsidies + ban on NCAs}} ($\kappa_c = \infty$)
		\begin{itemize}
			\item for smaller subsidies \alert{\textbf{allowing NCAs}} ($\kappa_c = 0$) is optimal
			\item subsidies to R\&D spending reduce growth and welfare
		\end{itemize}
	\end{itemize}
\end{frame}

\appendix

\section{Appendix}

\begin{frame}{Decomposing per-capita GDP growth in USA}\label{economic_growth_facts}\hyperlink{motivation_background}{\beamergotobutton{back}}
	\begin{table}
		\includegraphics[scale = 0.35]{figures/presentation/economic_growth_facts.png}
		\caption{Growth accounting (from Jones 2016, "The Facts of Economic Growth")}
	\end{table}
\end{frame}

\begin{frame}{Importance of firm entry in productivity growth}\label{motivation:importance_of_firm_entry}
	\hyperlink{motivation_background}{\beamergotobutton{back}}
	\begin{itemize}
		\item Firm entry contributes substantially to productivity growth
		\begin{itemize}
			\item 25\% of labor productivity growth in manufacturing (Baily, Bartelsman \& Haltiwanger 1996)
			\item 25\% of aggregate productivity growth (Akcigit \& Kerr 2017)
			\item 20-30\% of aggregate productivity growth (Garcia-Macia, Hsieh \& Klenow 2019)
			\item 40\% of aggregate productivity growth (Klenow \& Yi 2020)
		\end{itemize}
	\end{itemize}
\end{frame}

\begin{frame}{Importance of creative destruction in firm entry}\label{motivation:importance_of_creative_destruction}
	\hyperlink{motivation_background}{\beamergotobutton{back}}
	\begin{itemize}
		\item Creative destruction is significant part of new firm entry
		\begin{itemize}
			\item Garcia-Macia, Hsieh and Klenow 2018 finds approx. 70\% of productivity growth from entry is creative destruction, using firm-level employment data
			\item Klenow \& Yi 2020 estimate 30\% using plant-level sales and employment data
		\end{itemize}
	\end{itemize}
\end{frame}



\begin{frame}{Spinouts stylized facts}\label{spinouts_facts_from_literature}
\hyperlink{motivation}{\beamergotobutton{back}}
	\begin{itemize}
	\item Within-industry spinouts
		\begin{itemize}
		\item 15\% of entrants; larger at entry, grow faster, higher survival rates (Muendler et al. 2012, Brazilian data)
		\item Similar findings in my dataset
		\end{itemize}
	\end{itemize}
\end{frame}

\begin{frame}{Many examples of WSOs across industries and time}\label{spinouts_examples}
	\hyperlink{motivation}{\beamergotobutton{back}}
	\begin{itemize}
		\item Software
		\begin{itemize}
			\item Zoom (from Cisco WebEx); SalesForce (from Oracle); Adobe (from Xerox PARC)
		\end{itemize}
		\smallskip
		\item Semiconductors
		\begin{itemize}
			\item Fairchildren (Intel, AMD); Nvidia (from AMD); Qualcomm (from Linkabit)
		\end{itemize}
		\smallskip
		\item Hard drives
		\begin{itemize}
			\item see Franco-Filson 2006
		\end{itemize}
		\smallskip
		\item Pharmaceuticals
		\begin{itemize}
			\item Vertex (from Merck)
		\end{itemize}
		\item Automotive industry (Klepper 2007), steel manufacturing (Steel Dynamics from Nucorp), management consulting (Bain from BCG)
	\end{itemize}
\end{frame}


\begin{frame}{Spinouts of Fairchild Semiconductor}\label{fairchildren_early}
\hyperlink{motivation_background}{\beamergotobutton{back}}
\begin{figure}
	\includegraphics[scale=0.34]{../figures/fairchildren_early.png}
\end{figure}
\end{frame}

\subsection{Empirics}

\begin{frame}{Prevalence of WSOs}\label{empirics:wso_heatmap}
	\hyperlink{empirics:constructing_dataset}{\beamergotobutton{back}}
	\begin{figure}[]
		\centering
		\includegraphics[scale=0.45]{../empirics/figures/plots/industry_row_heatmap_naics2_founder2_ggplot2.png}
		\caption{Distribution of child 2-digit NAICS code (column) conditional on parent NAICS code (row) in sample of founders with titles CEO, CTO, Chief, President and / or Chairman. Darker hues indicate a higher density.}
		\label{figure:industry_row_heatmap_naics2_founder2}
	\end{figure}
\end{frame}


\subsection{Model}

\begin{frame}{Intermediate goods production}\label{intermediate_goods_production}
	\hyperlink{model:production_overview}{\beamergotobutton{back}}
	\begin{itemize}
		\item<+-> Equilibrium $\Rightarrow$ only \alert{\textbf{highest quality}} version of good $j$ is produced \hyperlink{two_stage_bertrand2}{\beamergotobutton{details}}
		\begin{itemize}
			\item referred to as \alert{\textbf{frontier}} good $j$, denoted $\bar{q}_{jt}$
			\item producer of frontier good $j$ is known as  \alert{\textbf{incumbent}} $j$ 
		\end{itemize}
		\medskip
		\item<+-> Production function
		\begin{align*}
			k_{jt} = Q_t \ell_{jt}
		\end{align*}
		\begin{itemize}
			\item average quality $Q_t = \int_0^1 \bar{q}_{jt} dj$
		\end{itemize}
		\medskip
		\item<.-> No storage of intermediate goods 
	\end{itemize}
\end{frame}

\begin{frame}{Final goods production}\label{main:final_goods_production}
	\hyperlink{model:production_overview}{\beamergotobutton{back}}
	\begin{itemize}
		\item<+-> Given only frontier goods $j$ used, production of final good follows
		\begin{align*}
			Y_t &= \frac{L_{Ft}^{\beta}}{1-\beta} \int_0^1 \bar{q}_{jt}^{\beta} k_{jt}^{1-\beta} dj 
		\end{align*}
		\begin{itemize}
			\item $L_{Ft}$: final goods labor
			\item $k_{jt}$: quantity of frontier good $j$
		\end{itemize} \hyperlink{definition:final_goods_production}{\beamergotobutton{Full final goods production function}} 
		\smallskip
		\item<.-> No storage of final good
	\end{itemize}
\end{frame}

\begin{frame}{Final goods production function}\label{definition:final_goods_production}\hyperlink{main:final_goods_production}{\beamergotobutton{back}}
	\begin{itemize}
		\item Final goods production function is given by
		\begin{align*}
		Y_t = F(L_{Ft},\{I_{jt}\},\{k_{jti}\}) &= \frac{L_{Ft}^{\beta}}{1-\beta} \int_0^1 \Big(\sum_{i = 0}^{I_{jt}} (\lambda^{i})^{\frac{\beta}{1-\beta}} k_{jti} \Big)^{1-\beta} dj
		\end{align*}
		\begin{itemize}
			\item \alert{\textbf{quality ladder}} assumption: $\{q_{jti}\}_{0 \le i \le I_{jt}} = \{\lambda^i\}_{0 \le i \le I_{jt}}$
		\end{itemize}
	\end{itemize}
\end{frame}


\subsection{Equilibrium}

\begin{frame}{Ownership of firms}\label{model:firm_ownership}
	\hyperlink{definition:equilibrium}{\beamergotobutton{back}}
	\begin{itemize}
		\item Household owns competitive financial intermediary
		\item Intermediary owns all firms
		\begin{itemize}
			\item intermediary maximizes DPV dividends paid to household
			\item firms maximize DPV of individual profits
			\item risk-free discount factor
		\end{itemize}
		\medskip
		\item Spinouts are sold to competitive financial intermediary at full private value
		\begin{itemize}
			\item i.e. intermediary takes spinout entry as given
		\end{itemize}
		\medskip
		\item Incumbent \alert{\textbf{assumed}} unable to buy spinout 
		\begin{itemize}
			\item consistent with finding that few firms acquire their employee spinouts (e.g., Babina \& Howell 2019)
			\item consistent with Klepper 1996 finding that spinouts result from disagreements
		\end{itemize}
	\end{itemize}
\end{frame}

\begin{frame}{Characterizing the symmetric BGP}\label{characterizing_BGP}
	\hyperlink{definition:symmetric_bgp}{\beamergotobutton{back}}
	\begin{itemize}
		\item<+-> Static equilibrium conditions given $\{\bar{q}_{jt}\}_{j \in [0,1]}$ \hyperlink{static_eq_conditions}{\beamergotobutton{details}}
		\begin{itemize}
			\item<.-> two-stage Bertrand competition within good $j$ $\Rightarrow$ only frontier good used, monopolistic competition pricing \hyperlink{two_stage_bertrand}{\beamergotobutton{details}}
		\end{itemize}
		\medskip
		\item<+-> Dynamic equilibrium conditions
		\begin{itemize}
			\item<.-> individual optimization \hyperlink{HJB_incumbent}{\beamergotobutton{incumbents}} \hyperlink{household_optimization}{\beamergotobutton{households}} 
			\item<.-> on symmetric BGP, $V(j,t|\bar{q}_{jt}) = \tilde{V} \bar{q}_{jt}$ and wages linear in $Q_t$  \hyperlink{proposition:hjb_scaling}{\beamergotobutton{Proposition}} 
			\item<.-> factors $Q_t$, $\bar{q}_{jt}$ drop out of equilibrium conditions 
			\item<.-> solve system of eq. recursively 
			\hyperlink{eq_innovation_and_growth}{\beamergotobutton{details}} 
		\end{itemize}
		\medskip
		\item<+-> Results 
		\begin{itemize}
			\item<.-> on symmetric BGP, $\mathbbm{1}^{NCA}_{jt} = \mathbbm{1}^{NCA} \in \{0,1\}$ for all $j,t$
			\item<.-> existence and uniqueness (except knife-edge) \hyperlink{existence_and_uniqueness}{\beamergotobutton{Proposition}} 
		\end{itemize}
	\end{itemize}
\end{frame}


\begin{frame}{Static equilibrium}\label{static_eq_conditions}
	\hyperlink{characterizing_BGP}{\beamergotobutton{back}}
	\begin{itemize}
		\small 
		\item Final goods optimization and monopolistic competition prices + quantities yield equilibrium wages and incumbent profits 
		\begin{align*}
			\bar{w}_t &= \tilde{\beta} Q_t \\ 
			\tilde{\beta} &= \beta^{\beta} (1-\beta)^{1-2\beta}  \\
			\pi_{jt} &= (1-\beta) \tilde{\beta} L_F \bar{q}_{jt}
		\end{align*}
		\small
		\item Labor allocation and output
		\begin{align*}
			L_{It} &= \Big( \frac{1-\beta}{\tilde{\beta}} \Big)^{1 / \beta} L_{Ft} \\
			L_{Ft} &= \frac{1 - \bar{L}_{RD}}{1 + \Big(\frac{1-\beta}{\tilde{\beta}}\Big)^{1/\beta}} \\
			Y_t &= \frac{(1-\beta)^{1-2\beta}}{\beta^{1-\beta}} Q_t L_{Ft} 
		\end{align*}
	\end{itemize}
\end{frame}

\begin{frame}{Two-stage Bertrand game}\label{two_stage_bertrand2}
	\hyperlink{characterizing_BGP}{\beamergotobutton{back}}
	\begin{itemize}
		\item For each good $j$ two-stage game each instant $t \ge 0$
		\begin{itemize}
			\item stage 1: pay fee $\varepsilon > 0$ units of final good
			\item stage 2: Bertrand competition within and across goods $j$
		\end{itemize}
		\item Bertrand competition $\Rightarrow$ only frontier good $j$ earns profits in stage 2 $\Rightarrow$ only frontier pay fee $\varepsilon$
		\item Monopolistic competition pricing (i.e. no limit pricing)
		\begin{align*}
		p_j &= \frac{\overline{w}}{(1-\beta) Q} \\
		\end{align*}
	\end{itemize}
\end{frame}

\begin{frame}{Two-stage Bertrand game}\label{two_stage_bertrand}
	\hyperlink{characterizing_BGP}{\beamergotobutton{back}}
	\begin{itemize}
		\item For each good $j$ two-stage game each instant $t \ge 0$
		\begin{itemize}
			\item stage 1: pay fee $\varepsilon > 0$ units of final good
			\item stage 2: Bertrand competition within and across goods $j$
		\end{itemize}
		\item Bertrand competition $\Rightarrow$ only frontier good $j$ earns profits in stage 2 $\Rightarrow$ only frontier pay fee $\varepsilon$
		\item Monopolistic competition pricing (i.e. no limit pricing)
		\begin{align*}
		p_j &= \frac{\overline{w}}{(1-\beta) Q} \\
		\end{align*}
	\end{itemize}
\end{frame}


\begin{frame}{Household optimization}\label{household_optimization}
	\hyperlink{characterizing_BGP}{\beamergotobutton{back}}
	\begin{itemize}
		\item Household solves
		\tiny
		\begin{maxi*}[1]<b>
			{\substack{\{C(t) \}_{t \ge 0} \\ \{A(t) \}_{t \ge 0} \\ \{ L(t)  \}_{t \ge 0} \\ \{\ell_{RD}(j,t|q,\mathbbm{1}^{NCA})\}_{j \in [0,1], t \ge 0} \\ \{\hat{\ell}_{RD}(j,t|q)\}_{j \in [0,1], t \ge 0}  }} {\int_0^{\infty} e^{-\rho t} \frac{C(t)^{1-\theta}-1}{1-\theta} dt}{}{}
			\addConstraint{ \dot{A}(t)}{ = -C(t) + r_tA(t) + \Pi_t + \bar{w}_tL(t)}  {\mkern-148mu\text{(Financial wealth law of motion)}}
			\addConstraint{ }{+ \int_0^1 w_{RD,jt} \ell_{RD,jt} dj} 
			\addConstraint{ }{+ \int_0^1 \big(1-\mathbbm{1}^{NCA}_{jt}\big) \nu (1-\kappa_e) V(j,t|\lambda \bar{q}_{jt}) \big(\frac{\bar{q}_{jt}}{Q_t} \big)^{-1} \ell_{RD,jt} dj}
			\addConstraint{ }{+ \int_0^1 \hat{w}_{RD}(t) \hat{\ell}_{RD,jt} dj,} 
			\addConstraint{A(0)}{= 0,} {\mkern-148mu\text{(Initial wealth)}} 
			\addConstraint{\lim_{t \to \infty} e^{-\int_0^{t} r_s ds }A(t)}{\ge 0,}  {\mkern-148mu\text{(No Ponzi-game)}} 
			\addConstraint{\int_0^1 (\ell_{RD,jt} + \hat{\ell}_{RD,jt}) dj}{ \le \bar{L}_{RD},} {\mkern-148mu\text{(R\&D labor endowment)}}
			\addConstraint{L(t)}{\le 1 - \bar{L}_{RD}.} {\mkern-148mu\text{(Production labor endowment)}}
		\end{maxi*}
	\end{itemize}
\end{frame}


\begin{frame}{Incumbent optimization}\label{HJB_incumbent}
\hyperlink{characterizing_BGP}{\beamergotobutton{back}}
\begin{itemize}
\item Incumbent value $V(j,t|\bar{q}_{jt})$ satisfies
\tiny
\begin{align*}
	(r_t + \overbrace{\hat{\tau}}^{\mathclap{\text{Creative destruction}}}) &V(j,t |q) - \dot{V}(j,t|q) = \overbrace{\tilde{\pi} q }^{\mathclap{\text{Flow profits}}}\nonumber \\_{}
	&+ \max_{\substack{\mathbbm{1}^{NCA} \in \{0,1\} \\ z \ge 0}} \Bigg\{ z \Big[  \overbrace{\chi \big( V(j,t|\lambda q) - V(j,t|q)\big)}^{\mathclap{\mathbb{E}[\text{Payoff from own-innovation}]}}  \nonumber \\
	&- \underbrace{\big(\frac{q}{Q_t}\big)}_{\mathclap{\text{scaling of R\&D cost}}} \Big( \overbrace{w_{RD,jt}(\mathbbm{1}^{NCA})}^{\mathclap{\text{R\&D wage depends on NCA}}} + \underbrace{\big(\frac{q}{Q_t}\big)^{-1}}_{\mathclap{\text{scaling of spinout formation rate}}} \overbrace{(1-\mathbbm{1}^{NCA}) \nu V(j,t|q)}^{\mathclap{\mathbb{E}[\text{Loss from spinout CD}]}} + \underbrace{\big(\frac{q}{Q_t}\big)^{-1}}_{\mathclap{\text{scaling of NCA cost}}}  \overbrace{\mathbbm{1}^{NCA} \kappa_c \nu V(j,t|q) }^{\mathclap{\text{NCA cost}}}\Big)  \Big] \Bigg\}.
\end{align*}
\end{itemize}
\end{frame}


\begin{frame}{Symmetric BGPs are linear}\label{proposition:hjb_scaling}
	\hyperlink{characterizing_BGP}{\beamergotobutton{back}}
	\small
	\begin{proposition}
		In a symmetric BGP, the value function of the incumbent is given by
		\begin{align*}
		V(j,t|q) &= \tilde{V} q,
		\end{align*}
		for some $\tilde{V} > 0$. Further, the equilibrium R\&D wages are given by 
		\begin{align*}
		\hat{w}_{RD,t} &= \hat{w}_{RD} Q_t, \\
		w_{RD}(j,t|q,\mathbbm{1}^{NCA}) &= \tilde{w}_{RD}(\mathbbm{1}^{NCA}) Q_t \textrm{, if $z > 0$}.
		\end{align*}
	\end{proposition}
\end{frame}

\begin{frame}{Equilibrium use of NCAs}\label{use_of_ncas_details}
	\hyperlink{use_of_ncas}{\beamergotobutton{back}}
	\begin{itemize}
		\item Incumbent must offer \alert{\textbf{total compensation}} equal to entrant
		\begin{align*}
		w_{RD}(\mathbbm{1}^{NCA}) + \underbrace{(1-\mathbbm{1}^{NCA}) \nu (1-\kappa_e) \lambda \tilde{V}}_{\mathclap{\text{Employee's value of future spinouts}}} &= \overbrace{\hat{w}_{RD}}^{\mathclap{\text{Entrant R\&D wage}}} 
		\end{align*}
		\begin{itemize}
			\item $\mathbbm{1}^{NCA}_{jt} = 0$ $\Rightarrow$ \alert{\textbf{wage discount}}
		\end{itemize}
		\item Incumbent minimizes \alert{\textbf{effective cost of R\&D}} given by
		\begin{align*}
		\hat{w}_{RD} + \mathbbm{1}^{NCA}_{jt} \overbrace{ \kappa_c \nu \tilde{V}}^{\mathclap{\text{Direct cost of NCA}}} + (1- \mathbbm{1}^{NCA}_{jt}) \overbrace{(1 - (1-\kappa_e) \lambda) \nu \tilde{V}}^{\mathclap{\mathbb{E}[\text{Loss of business}] - \mathbb{E}[\text{Wage discount}]}} 
		\end{align*}
	\end{itemize}
\end{frame}

\begin{frame}{Equilibrium innovation and growth}\label{eq_innovation_and_growth}
	\hyperlink{characterizing_BGP}{\beamergotobutton{back}} 
	\begin{itemize}
		\item Entrant optimization condition $\hat{z}^{-\psi} \hat{\chi} (1-\kappa_e) \lambda \tilde{V}= \hat{w}_{RD}$ yields
		\begin{align*}
		\hat{z} &= \Big( \frac{ \overbrace{\hat{\chi} (1-\kappa_e) \lambda}^{\mathclap{\propto \text{ private payoff to entrant innovation}}}}{\underbrace{\chi(\lambda -1)}_{\mathclap{\propto \text{ private payoff incumbent}}} - \Big[\underbrace{(1-\mathbbm{1}^{NCA}) (1- (1-\kappa_e) \lambda)\nu + \mathbbm{1}^{NCA} \kappa_c \nu \Big]}_{\mathclap{w_{RD}(\mathbbm{1}^{NCA}) + \mathbbm{1}^{NCA} \kappa_c  - \hat{w}_{RD}}} }\Big)^{1/\psi} \\
		z &= \underbrace{\bar{L}_{RD} - \hat{z}}_{\mathclap{\text{R\&D labor resource constraint}}} 
		\end{align*}
		with spinout formation rate $\tau^S = (1-\mathbbm{1}^{NCA}) z$
		\item Growth rate
		\begin{align*}
		g &= \frac{\dot{Q}_t}{Q_t} = (\lambda - 1) (\tau + \hat{\tau} + \tau^S)
		\end{align*}
	\end{itemize}
\end{frame}


\begin{frame}{Existence and uniqueness}\label{existence_and_uniqueness}\hyperlink{characterizing_BGP}{\beamergotobutton{back}} 
	\begin{proposition}\label{proposition:purstrategyeq:positiveOI}
		If $\theta \ge 1$, $\kappa_c \ne \bar{\kappa}_c$, and $\Big( \frac{\hat{\chi} (1-\kappa_{e}) \lambda}{\chi(\lambda-1) - \nu \min\{ 1-(1-\kappa_e) \lambda, \kappa_c \}} \Big)^{1/\psi} < \bar{L}_{RD}$, then:
		\begin{enumerate}
			\item There exists a unique symmetric BGP.
			\item On the symmetric BGP, $z > 0$ and $\mathbbm{1}^{NCA}_{jt} = x$
		\end{enumerate}
	\end{proposition}
	\begin{itemize}
		\item On the knife-edge $\kappa_c = \bar{\kappa}_c$ there is a continuum of symmetric BGPs
	\end{itemize}
\end{frame}

\subsection{Efficiency}

\begin{frame}{Welfare}\label{welfare}
	\hyperlink{reducing_kappa_c_table}{\beamergotobutton{back}}
	\begin{itemize}
		\item Social welfare
		\begin{align*}
		W &= \int_0^{\infty} e^{-\rho t} \frac{C(t)^{1-\theta} - 1}{1-\theta} dt
		\end{align*}
		\item $C(t) = \tilde{C} e^{gt}$ implies
		\begin{align*}
		W &= \overbrace{\frac{\tilde{C}^{1-\theta}}{(1-\theta)(\rho - g(1-\theta))}}^{\mathclap{\tilde{W}}} + \text{ Constant} 
		\end{align*}
		\item $\therefore$ Welfare determined by
		\begin{itemize}
			\item $g$: growth rate of consumption 
			\item $\tilde{C}$: level of consumption given each $Q_t$ 
		\end{itemize}
	\end{itemize}
\end{frame}

\subsection{Empirics}


\begin{frame}{Results of match to Compustat}\label{results_of_match}
	\hyperlink{economic_magnitude}{\beamergotobutton{back}}
	\begin{figure}
		\includegraphics[scale=0.24]{figures/presentation/table1_founder2.png}
	\end{figure}
\end{frame}

\begin{frame}{Founders explained by regression coefficient}\label{regs_economic_significance}
	\hyperlink{economic_magnitude}{\beamergotobutton{back}}
	\begin{figure}[!htb]
		\includegraphics[scale=0.3]{../empirics/figures/founder2_founders_wso4_f3_Accounting.pdf}
		\caption{\small Accounting for WSOs using regression estimated relationship between R\&D and founders leaving to WSOs.}
	\end{figure}
\end{frame}


\begin{frame}{Economic magnitude II: employment}\label{regs_startup_lifecycle_employment}
	\hyperlink{economic_magnitude}{\beamergotobutton{back}}
	\begin{table}[!htb]
		\Tiny
		\centering
		{
\def\sym#1{\ifmmode^{#1}\else\(^{#1}\)\fi}
\begin{tabular}{l*{4}{c}}
\toprule
                    &\multicolumn{1}{c}{(1)}         &\multicolumn{1}{c}{(2)}         &\multicolumn{1}{c}{(3)}         &\multicolumn{1}{c}{(4)}         \\
\midrule
$\frac{\text{WSO4 founders}}{\text{Total founders}}$&        0.19         &        0.32\sym{***}&        0.32\sym{***}&        0.30\sym{***}\\
                    &      (0.22)         &     (0.027)         &     (0.020)         &     (0.013)         \\
\addlinespace
Constant            &        2.44\sym{***}&        2.41\sym{***}&        2.41\sym{***}&        2.41\sym{***}\\
                    &     (0.073)         &   (0.00019)         &   (0.00015)         &   (0.00028)         \\
\addlinespace
State-Year FE       &          No         &         Yes         &         Yes         &          No         \\
\addlinespace
State-Age FE        &          No         &         Yes         &          No         &         Yes         \\
\addlinespace
State-Cohort FE     &          No         &          No         &         Yes         &         Yes         \\
\addlinespace
NAICS4-Year FE      &          No         &         Yes         &         Yes         &          No         \\
\addlinespace
NAICS4-Age FE       &          No         &         Yes         &          No         &         Yes         \\
\addlinespace
NAICS4-Cohort FE    &          No         &          No         &         Yes         &         Yes         \\
\addlinespace
No FE               &         Yes         &          No         &          No         &          No         \\
\midrule
Clustering          &statecode naics1\_4 year         &statecode naics1\_4         &statecode naics1\_4         &statecode naics1\_4         \\
R-squared (adj.)    &     0.00068         &        0.35         &        0.38         &        0.36         \\
R-squared (within, adj)&     0.00068         &      0.0028         &      0.0028         &      0.0024         \\
Observations        &       55767         &       54873         &       54654         &       54779         \\
\bottomrule
\multicolumn{5}{l}{\footnotesize Standard errors in parentheses}\\
\multicolumn{5}{l}{\footnotesize \sym{*} \(p<0.1\), \sym{**} \(p<0.05\), \sym{***} \(p<0.01\)}\\
\end{tabular}
}

		\caption{\footnotesize Dependent variable is logarithm of employee count divided by number of founders. Indepdendent variable is fraction of founders whose previous employer was in the same industry.} 
		\label{table:startupLifeCycle_founder2founders_lemployeecount_founder2}
	\end{table}
\end{frame}


\begin{frame}{Economic magnitude II: revenue}\label{regs_startup_lifecycle_revenue}
	\hyperlink{economic_magnitude}{\beamergotobutton{back}}
	\begin{table}[!htb]
		\Tiny
		\centering
		{
\def\sym#1{\ifmmode^{#1}\else\(^{#1}\)\fi}
\begin{tabular}{l*{4}{c}}
\toprule
                    &\multicolumn{1}{c}{(1)}         &\multicolumn{1}{c}{(2)}         &\multicolumn{1}{c}{(3)}         &\multicolumn{1}{c}{(4)}         \\
\midrule
$\frac{\text{WSO4 founders}}{\text{Total founders}}$&       -0.13         &        0.45\sym{***}&        0.42\sym{***}&        0.39\sym{***}\\
                    &     (0.094)         &      (0.13)         &     (0.081)         &      (0.12)         \\
\addlinespace
State-Year FE       &          No         &         Yes         &         Yes         &          No         \\
\addlinespace
State-Age FE        &          No         &         Yes         &          No         &         Yes         \\
\addlinespace
State-Cohort FE     &          No         &          No         &         Yes         &         Yes         \\
\addlinespace
NAICS4-Year FE      &          No         &         Yes         &         Yes         &          No         \\
\addlinespace
NAICS4-Age FE       &          No         &         Yes         &          No         &         Yes         \\
\addlinespace
NAICS4-Cohort FE    &          No         &          No         &         Yes         &         Yes         \\
\midrule
Clustering          &State, Industry         &State, Industry         &State, Industry         &State, Industry         \\
R-squared (adj.)    &    0.000092         &        0.30         &        0.38         &        0.39         \\
R-squared (within, adj)&    0.000092         &      0.0030         &      0.0026         &      0.0022         \\
Observations        &       16948         &       15500         &       15531         &       15905         \\
\bottomrule
\multicolumn{5}{l}{\footnotesize Standard errors in parentheses}\\
\multicolumn{5}{l}{\footnotesize \sym{*} \(p<0.1\), \sym{**} \(p<0.05\), \sym{***} \(p<0.01\)}\\
\end{tabular}
}

		\caption{\footnotesize Dependent variable is logarithm of revenue divided by number of founders. Indepdendent variable is fraction of founders whose previous employer was in the same industry.} 
		\label{table:startupLifeCycle_founder2founders_lrevenue_founder2}
	\end{table}
\end{frame}


\begin{frame}{Economic magnitude II: valuation}\label{regs_startup_lifecycle_valuation}
	\hyperlink{economic_magnitude}{\beamergotobutton{back}}
	\begin{table}[!htb]
		\Tiny
		\centering
		{
\def\sym#1{\ifmmode^{#1}\else\(^{#1}\)\fi}
\begin{tabular}{l*{4}{c}}
\toprule
                    &\multicolumn{1}{c}{(1)}         &\multicolumn{1}{c}{(2)}         &\multicolumn{1}{c}{(3)}         &\multicolumn{1}{c}{(4)}         \\
\midrule
$\frac{\text{WSO4 founders}}{\text{Total founders}}$&        0.46\sym{***}&        0.42\sym{***}&        0.36\sym{***}&        0.33\sym{***}\\
                    &     (0.065)         &     (0.058)         &     (0.069)         &     (0.074)         \\
\addlinespace
State-Year FE       &          No         &         Yes         &         Yes         &          No         \\
\addlinespace
State-Age FE        &          No         &         Yes         &          No         &         Yes         \\
\addlinespace
State-Cohort FE     &          No         &          No         &         Yes         &         Yes         \\
\addlinespace
NAICS4-Year FE      &          No         &         Yes         &         Yes         &          No         \\
\addlinespace
NAICS4-Age FE       &          No         &         Yes         &          No         &         Yes         \\
\addlinespace
NAICS4-Cohort FE    &          No         &          No         &         Yes         &         Yes         \\
\midrule
Clustering          &State, Industry         &State, Industry         &State, Industry         &State, Industry         \\
R-squared (adj.)    &      0.0042         &        0.28         &        0.29         &        0.26         \\
R-squared (within, adj)&      0.0042         &      0.0050         &      0.0035         &      0.0028         \\
Observations        &       26504         &       25174         &       25027         &       25337         \\
\bottomrule
\multicolumn{5}{l}{\footnotesize Standard errors in parentheses}\\
\multicolumn{5}{l}{\footnotesize \sym{*} \(p<0.1\), \sym{**} \(p<0.05\), \sym{***} \(p<0.01\)}\\
\end{tabular}
}

		\caption{\footnotesize Dependent variable is logarithm of valuation divided by number of founders. Indepdendent variable is fraction of founders whose previous employer was in the same industry.} 
		\label{table:startupLifeCycle_founder2founders_lpostvalusd_founder2}
	\end{table}
\end{frame}


\begin{frame}{Economic magnitude II: going out of business}\label{regs_startup_lifecycle_goingoutofbusiness}
	\hyperlink{economic_magnitude}{\beamergotobutton{back}}
	\begin{table}[!htb]
		\Tiny
		\centering
		{
\def\sym#1{\ifmmode^{#1}\else\(^{#1}\)\fi}
\begin{tabular}{l*{4}{c}}
\toprule
                    &\multicolumn{1}{c}{(1)}         &\multicolumn{1}{c}{(2)}         &\multicolumn{1}{c}{(3)}         &\multicolumn{1}{c}{(4)}         \\
\midrule
$\frac{\text{WSO4 founders}}{\text{Total founders}}$&       -0.16         &       -0.57\sym{***}&       -0.54\sym{***}&       -0.56\sym{***}\\
                    &      (0.23)         &      (0.12)         &      (0.11)         &     (0.099)         \\
\addlinespace
Constant            &        1.59\sym{***}&        1.61\sym{***}&        1.61\sym{***}&        1.61\sym{***}\\
                    &      (0.35)         &    (0.0013)         &    (0.0023)         &    (0.0016)         \\
\addlinespace
State-Year FE       &          No         &         Yes         &         Yes         &          No         \\
\addlinespace
State-Age FE        &          No         &         Yes         &          No         &         Yes         \\
\addlinespace
State-Cohort FE     &          No         &          No         &         Yes         &         Yes         \\
\addlinespace
NAICS4-Year FE      &          No         &         Yes         &         Yes         &          No         \\
\addlinespace
NAICS4-Age FE       &          No         &         Yes         &          No         &         Yes         \\
\addlinespace
NAICS4-Cohort FE    &          No         &          No         &         Yes         &         Yes         \\
\addlinespace
No FE               &         Yes         &          No         &          No         &          No         \\
\midrule
Clustering          &statecode naics1\_4 year         &statecode naics1\_4         &statecode naics1\_4         &statecode naics1\_4         \\
R-squared (adj.)    &   0.0000015         &       0.030         &       0.032         &       0.017         \\
R-squared (within, adj)&   0.0000015         &    0.000065         &    0.000054         &    0.000057         \\
Observations        &      251910         &      251460         &      251552         &      251710         \\
\bottomrule
\multicolumn{5}{l}{\footnotesize Standard errors in parentheses}\\
\multicolumn{5}{l}{\footnotesize \sym{*} \(p<0.1\), \sym{**} \(p<0.05\), \sym{***} \(p<0.01\)}\\
\end{tabular}
}

		\caption{\footnotesize Dependent variable is an indicator for the startup going out of business that year. Indepdendent variable is fraction of founders whose previous employer was in the same industry.} 
		\label{table:startupLifeCycle_founder2founders_goingoutofbusiness}
	\end{table}
\end{frame}

\begin{frame}{Economic magnitude II: M\&A and IPO}\label{regs_startup_lifecycle_successfullyexiting}
	\hyperlink{economic_magnitude}{\beamergotobutton{back}}
	\begin{table}[!htb]
		\Tiny
		\centering
		{
\def\sym#1{\ifmmode^{#1}\else\(^{#1}\)\fi}
\begin{tabular}{l*{4}{c}}
\toprule
                    &\multicolumn{1}{c}{(1)}         &\multicolumn{1}{c}{(2)}         &\multicolumn{1}{c}{(3)}         &\multicolumn{1}{c}{(4)}         \\
\midrule
$\frac{\text{WSO4 founders}}{\text{Total founders}}$&        2.52\sym{***}&        2.19\sym{***}&        2.03\sym{***}&        2.01\sym{***}\\
                    &     (0.056)         &      (0.14)         &      (0.18)         &      (0.18)         \\
\addlinespace
State-Year FE       &          No         &         Yes         &         Yes         &          No         \\
\addlinespace
State-Age FE        &          No         &         Yes         &          No         &         Yes         \\
\addlinespace
State-Cohort FE     &          No         &          No         &         Yes         &         Yes         \\
\addlinespace
NAICS4-Year FE      &          No         &         Yes         &         Yes         &          No         \\
\addlinespace
NAICS4-Age FE       &          No         &         Yes         &          No         &         Yes         \\
\addlinespace
NAICS4-Cohort FE    &          No         &          No         &         Yes         &         Yes         \\
\midrule
Clustering          &State, Industry         &State, Industry         &State, Industry         &State, Industry         \\
R-squared (adj.)    &     0.00046         &       0.035         &       0.033         &       0.035         \\
R-squared (within, adj)&     0.00046         &     0.00034         &     0.00027         &     0.00027         \\
Observations        &      240155         &      239696         &      239788         &      239959         \\
\bottomrule
\multicolumn{5}{l}{\footnotesize Standard errors in parentheses}\\
\multicolumn{5}{l}{\footnotesize \sym{*} \(p<0.1\), \sym{**} \(p<0.05\), \sym{***} \(p<0.01\)}\\
\end{tabular}
}

		\caption{\footnotesize Dependent variable is an indicator for the startup being acquired or having an IPO in that year. Indepdendent variable is fraction of founders whose previous employer was in the same industry.} 
		\label{table:startupLifeCycle_founder2founders_successfullyexiting}
	\end{table}
\end{frame}


\subsection{Calibration}

\begin{frame}{Calibration targets}\label{calibration_targets}
	\hyperlink{calibration_overview}{\beamergotobutton{back}} 
	\begin{table}[]
		\centering
		\captionof{table}{Calibration targets}
		\small
		\begin{tabular}{rcll}
			\toprule \toprule
			& Identified parameters & Target & Model \tabularnewline
			\midrule
			Profit (\% GDP) & $\beta$ & 8.5\% & 8.5\% 
			\tabularnewline
			R\&D emp. share & $\bar{L}_{RD}$ & 1\% & 1\% 
			\tabularnewline
			Interest rate & $\rho$ & 8.57\% & 8.57\% 
			\tabularnewline
			Growth rate (CD + OI) & $\mathbf{\lambda, \chi, \hat{\chi}}$ & 1.487\% & 1.487\%
			\tabularnewline		
			Age $\ge$ 6 growth share & $\chi, \hat{\chi}$  & 65\% & 65\%
			\tabularnewline
			Age $<$ 6 emp. share  & $\lambda, \hat{\chi}$ & 13.34\% & 13.34\%
			\tabularnewline
			Spinout emp. share &$\nu$  & 10\% & 10\%
			\tabularnewline
			R\&D spending (\% GDP) & $\chi, \hat{\chi}, \kappa_e$  & 1.35\% & 1.35\%
			\tabularnewline
			\bottomrule
		\end{tabular}
	\end{table}
\end{frame}


\begin{frame}{Parameters}\label{parameters}
	\hyperlink{calibration_overview}{\beamergotobutton{back}} 
	\begin{table}[]
		\footnotesize
		\centering
		\captionof{table}{Calibrated parameters}\label{calibration_parameters}
		\begin{tabular}{rlll}
			\toprule \toprule
			Parameter & Value & Description & Source \tabularnewline
			\midrule
			$\theta$ & 2 & $\theta^{-1} = $ IES & External
			\tabularnewline
			$\psi$ & 0.5 & Entrant R\&D curvature & External \tabularnewline
			$\rho$ & 0.056 & Discount rate  & Internal \tabularnewline
			$\beta$ & 0.094 & $\beta^{-1} = $ EoS intermediate goods & Internal \tabularnewline 
			$\lambda$ & 1.084 & Quality ladder step size & Internal
			\tabularnewline
			$\chi$ & 26.3 & Incumbent R\&D productivity & Internal
			\tabularnewline
			$\hat{\chi}$ & 0.461 & Entrant R\&D productivity & Internal \tabularnewline 
			$\kappa_e$ & 0.739 & Non-R\&D entry cost & Internal \tabularnewline
			$\nu$ & 0.431 & Spinout generation rate  & Internal \tabularnewline
			$\bar{L}_{RD}$ & 0.01 & R\&D labor allocation  & Internal \tabularnewline
			\bottomrule
		\end{tabular}
	\end{table}
	\hyperlink{identification}{\beamergotobutton{identification}} 
\end{frame}

\begin{frame}{Economic magnitude}\label{economic_magnitude}
	\hyperlink{calibration_overview}{\beamergotobutton{back}}
	\begin{itemize}
		\item WSO founders (assume all creative destruction) are approx. $7\%$ of founders  \hyperlink{results_of_match}{\beamergotobutton{details}}
		\item Corporate R\&D can account for approx. $90\%$ of founder departures to WSOs in the data \hyperlink{regs_economic_significance}{\beamergotobutton{details}}
		\item Per founder, WSOs are approx 35\% larger (employment, revenue, valuation) than other spinouts \hyperlink{regs_startup_lifecycle_employment}{\beamergotobutton{employment}} \hyperlink{regs_startup_lifecycle_revenue}{\beamergotobutton{revenue}} \hyperlink{regs_startup_lifecycle_valuation}{\beamergotobutton{valuation}}  \hyperlink{regs_startup_lifecycle_goingoutofbusiness}{\beamergotobutton{exit rate}} \hyperlink{regs_startup_lifecycle_successfullyexiting}{\beamergotobutton{M\&A and IPO}} 
		\item \alert{\textbf{Conclusion:}} R\&D induced spinouts account for roughly 8.5\% of startup employment / revenue / valuation 
		\begin{itemize}
			\item Creative destruction constitutes 30-70\% of growth from startups $\Rightarrow$ 72\% - 93\% of startup employment if $\lambda = 1.2$ (higher if $\lambda < 1.2$)
			\item $\Rightarrow$ R\&D-induced spinouts account for 8.6\% to 11.1\% of creative destruction startup employment / revenue / valuation
		\end{itemize}
	\end{itemize}
\end{frame}

\begin{frame}{Identification}\label{identification}\hyperlink{parameters}{\beamergotobutton{back}} 
	\begin{figure}
		\includegraphics[scale = 0.23]{../code/julia/figures/simpleModel/calibrationSensitivityFull.pdf}
		\caption{\small Elasticity of each calibrated parameter to each moment $r,g,OI,E,S,RD$ or external parameter $\theta, \beta, \psi$}
	\end{figure}
\end{frame}


\subsection{Policy}

\subsubsection{Welfare comparison robustness}

\begin{frame}{Robustness of welfare gain to moments}\label{robustness_to_moments}
\hyperlink{reducing_kappa_c_table}{\beamergotobutton{back}}
	\begin{figure}
		\includegraphics[scale = 0.21]{../code/julia/figures/simpleModel/WelfareComparisonSensitivityFull.pdf}
		\caption{Elasticity of CEV welfare improvement with respect to target moments and exactly or externally calibrated parameters.}
		\label{WelfareComparisonSensitivityFull}
	\end{figure}
\end{frame}

\begin{frame}{Robustness of welfare gain to parameters}\label{robustness_to_parameters}
	\hyperlink{reducing_kappa_c_table}{\beamergotobutton{back}}
	\begin{figure}
		\includegraphics[scale = 0.21]{../code/julia/figures/simpleModel/WelfareComparisonParameterSensitivityFull.pdf}
		\caption{Elasticity of CEV welfare improvement with respect to model parameters.}
		\label{WelfareComparisonSensitivityFull}
	\end{figure}
\end{frame}


\subsubsection{Noncompete policy}

\begin{frame}{Decomposition of growth increase}\label{decomposition_growth_increase}
	\hyperlink{reducing_kappa_c_table}{\beamergotobutton{back}}
	\begin{table}
		\centering
		\footnotesize
		\begin{tabular}{lclll}
			\toprule \toprule
			Measure & Variable & $\kappa_c > \bar{\kappa}_c$ & $\kappa_c = 0$ & Chg. \tabularnewline
			\midrule
			Growth & $g$ & 1.487\% & 1.696\% & $\phantom{-} 0.21$ p.p.\tabularnewline
			\multicolumn{1}{l}{\quad incumbents} & $(\lambda -1) \tau$  & 1.20\% & 1.47\% & $\phantom{-}$0.27 p.p. \tabularnewline
			\multicolumn{1}{l}{\quad entrants} & $(\lambda -1) \hat{\tau}$ & 0.26\% & 0.23\% & $-0.03$ p.p. \tabularnewline
			\multicolumn{1}{l}{\quad spinouts} & $(\lambda -1) \tau^S$ & 0.02\% & 0\% & $-0.02$ p.p. \tabularnewline
			\tabularnewline
			R\&D & & & & 
			\tabularnewline
			\multicolumn{1}{l}{\quad incumbents (\%)}  & $z / \bar{L}_{RD}$ & 54.0\% & 65.8\% & $\phantom{-} 11.8$ p.p. \tabularnewline 
			
			\multicolumn{1}{l}{\quad entrants (\%)}  & $\hat{z} / \bar{L}_{RD}$ & 46.0\% & 34.2\% & $-11.8$ p.p. \tabularnewline
			\bottomrule
		\end{tabular}
	\end{table}
\end{frame}


\begin{frame}{Reducing barriers to NCAs: welfare decomposition} \label{plots:reducing_kappa_c1} 
	\hyperlink{reducing_kappa_c_table}{\beamergotobutton{back}}
	\begin{figure}[]
		\includegraphics[scale = 0.19]{../code/julia/figures/simpleModel/calibrationFixed_welfareDecomp.pdf}
		\caption{Sufficiently reducing $\kappa_c$ increases welfare through higher growth and initial consumption.}
	\end{figure}
\end{frame}

\begin{frame}{Reducing barriers to NCAs: growth decomposition} \label{plots:reducing_kappa_c2} 
	\hyperlink{reducing_kappa_c_table}{\beamergotobutton{back}}
	\begin{figure}[]
		\includegraphics[scale = 0.19]{../code/julia/figures/simpleModel/calibrationFixed_growthDecomp.pdf}
		\caption{Sufficiently reducing $\kappa_c$ increases growth. Improvement in R\&D allocation outweighs reduction in spinout formation.}
	\end{figure}
\end{frame}


\begin{frame}{Efficiency}\label{efficiency} 
	\hyperlink{reducing_kappa_c_table}{\beamergotobutton{back}}
	\medskip
	\begin{itemize}
		\item Allocation of production labor
		\begin{itemize}
			\item final good vs. intermediate goods
			\item monopolistic competition
		\end{itemize}
		\medskip
		\item Use of NCAs 
		\begin{itemize}
			\item incumbent innovation vs. entrant and spinout innovation  \hyperlink{misallocation_of_nca}{\beamergotobutton{details}} 
		\end{itemize}
	\end{itemize}
\end{frame}

\begin{frame}{Misallocation of NCAs}\label{misallocation_of_nca} 
	\hyperlink{efficiency}{\beamergotobutton{back}}
	\begin{itemize}
		\item <+-> Suppose $z > 0$ and $\kappa_c > \bar{\kappa}_c > 0$, so that $\mathbbm{1}^{NCA} = 0$ 
		\begin{itemize}
			\item setting $\kappa_c' = \bar{\kappa}_c - \varepsilon \Rightarrow \mathbbm{1}^{NCA} = 1$ and $\tau^S = 0$
			\item setting $\kappa_c'' < \kappa_c'$ lowers incumbent's effective cost of R\&D
			\begin{itemize}
				\item more incumbent R\&D ($z \uparrow$), less entrant R\&D ($\hat{z} \downarrow$)
			\end{itemize}
		\end{itemize}
		\item <+-> Overall change in growth rate from reduction to $\kappa_c''$
		\begin{align*}
			\Delta g  &= (\lambda -1) (\Delta (\underbrace{\tau + \hat{\tau}}_{\mathclap{\chi z + \hat{\chi} \hat{z}^{1-\psi}}}) - \underbrace{\tau^S_0}_{\mathclap{ \nu z_0}})
		\end{align*}
		\begin{itemize}
			\item if (\ref{eq:RD_reallocation}) holds, $\Delta(\tau+ \hat{\tau}) > 0$ 
			\item growth can \alert{\textbf{increase}} or \alert{\textbf{decrease}}  
		\end{itemize}
	\end{itemize}
\end{frame}

\begin{frame}{Why reducing $\kappa_c$ increases growth in the calibration}\label{reducing_kappa_c_intuition}
	\hyperlink{reducing_kappa_c_table}{\beamergotobutton{back}}
	\begin{itemize}
		\item Business stealing, congestion
		\begin{itemize}
			\item large equilibrium difference in marginal effect on innovation of R\&D labor in OI vs CD
			\item reducing $\kappa_c$ reallocates R\&D to OI, increasing growth
		\end{itemize}
	\end{itemize}
\end{frame}

\begin{frame}{Other factors determining magnitude of growth increase}\label{other_factors}
	\hyperlink{reducing_kappa_c_table}{\beamergotobutton{back}}
	\begin{itemize}
		\item <+-> Rate of spinout formation $\nu$
		\begin{itemize}
			\item higher $\nu$ $\Rightarrow$ reduction in $\kappa_c$ implies proportionally larger decrease in effective cost of incumbent innovation
			\item proportionally more spinout innovation lost when $\kappa_c < \bar{\kappa}_c$ 
		\end{itemize}
		\medskip
		\item <+-> Scope for reductions in $\kappa_c$
		\begin{itemize}
			\item large $\bar{\kappa}_c = 1 - (1-\kappa_e)\lambda $ means more scope for reducing incumbent cost of R\&D through reduction in $\kappa_c$
		\end{itemize}
		\medskip
		\item <+-> Elasticity of R\&D spending 
		\begin{itemize}
			\item higher price-elasticity of R\&D spending $\Rightarrow$ more reallocation of R\&D
			\item result is quantitatively robust to DRS incumbent R\&D using standard parameter from literature 
		\end{itemize}
	\end{itemize}
\end{frame}

\begin{frame}{Welfare}\label{welfare_details}
	\hyperlink{reducing_kappa_c_table}{\beamergotobutton{back}}
	\begin{itemize}
		\item Initial consumption effects complicate general statements about welfare
		\begin{itemize}
			\item initial consumption affected by entry cost
			\item entry cost depends on incumbent value $\tilde{V}$ 
			\item incumbent value $\tilde{V}$ can increase or decrease as $\kappa_c \to 0$ b.c. $r \uparrow$ (as $g \uparrow$)
		\end{itemize}
		\medskip
		\item As long as $\tilde{V}$ at $\kappa_c = 0$ is larger than $\tilde{V}$ at $\bar{\kappa}_c$,
		\begin{itemize}
			\item $g(\kappa_c = 0) > g(\kappa_c = \bar{\kappa}_c)$ implies $\tilde{W}(\kappa_c = 0) > \tilde{W}(\kappa_c = \bar{\kappa}_c)$
			\smallskip
			\item follows from $\tilde{C} = \tilde{Y} - (\hat{\tau} + \tau^S)\kappa_e \tilde{V} - \kappa_c \nu z$
			\item this is the case in the calibration above
		\end{itemize}
	\end{itemize}
\end{frame}

\begin{frame}{Welfare}\label{welfare_varyingKappaC}
	\begin{itemize}
		\item Initial consumption effects complicate general statements about welfare \hyperlink{welfare_details}{\beamergotobutton{details}}
		\item Treating entry, NCA costs as transfers changes results are quantitatively similar (2.4\% increase in welfare) \hyperlink{welfare_costsAreTransfers}{\beamergotobutton{details}}
	\end{itemize}
\end{frame}

\begin{frame}{Derivation of R\&D misallocation condition}\label{misallocation_of_rd:derivation}
	\hyperlink{misallocation_of_rd}{\beamergotobutton{back}}
	\begin{itemize}
		\item Growth accounting
		\begin{align*}
			g &= (\lambda - 1) (\tau + \hat{\tau} + \tau^S)
		\end{align*}
		\item Compute $\frac{d}{d\hat{z}} g$ on $\bar{L}_{RD} = \hat{z} + z$ i.e. $dz /d\hat{z} = -1$
		\item Implies
		\begin{align}
			\frac{d}{d\hat{z}} g(z) &= (\lambda -1) \Big(\frac{d}{d\hat{z}} \hat{\tau} - \frac{d}{dz} \tau \Big)
		\end{align}
		\item Imposing $\frac{d}{d\hat{z}} g < 0$ and using equilibrium expressions for $\hat{\tau}$ and $\tau$ then yields the inequality
	\end{itemize}
\end{frame}

\subsubsection{R\&D subsidy}

\begin{frame}{Decomposition of growth decrease}\label{rd_subsidies:decomposition_growth_decrease}
	\hyperlink{RDsubsidy_table}{\beamergotobutton{back}}
	\begin{table}
		\centering
		\small
		\begin{tabular}{lclllll}
			\toprule \toprule
			&  & \multicolumn{4}{l}{R\&D Subsidy (\%)} \vspace{3pt} \tabularnewline
			Measure &Variable & 0 & 10 & 20 & 30 \tabularnewline
			\midrule
			Growth & $g$ & 1.49\% & 1.48\% & 1.46\% & 1.44\% \tabularnewline
			\multicolumn{1}{l}{\quad incumbents} & & 1.20\% & 1.19\% & 1.18\% & 1.17\% \tabularnewline
			\multicolumn{1}{l}{\quad entrants} & & 0.26\% & 0.27\% & 0.27\% & 0.27\% \tabularnewline
			\multicolumn{1}{l}{\quad spinouts} &  & 0.02\% & 0.02\% & 0.02\% & 0\% \tabularnewline
			\tabularnewline
			R\&D & &  &  &  & \tabularnewline
			\multicolumn{1}{l}{\quad incumbents} & $z / \bar{L}_{RD}$ & 54.0\% & 53.4\% & 52.8\% & 52.4\% \tabularnewline
			\multicolumn{1}{l}{\quad entrants} & $\hat{z} / \bar{L}_{RD}$ & 46.0\% & 46.6\% & 47.2\% & 47.2\% \tabularnewline
			\bottomrule
		\end{tabular}
	\end{table}
\end{frame}

\begin{frame}{R\&D subsidies: decomposing effect on welfare} \label{plots:rd_subsidies1} 
	\hyperlink{RDsubsidy_table}{\beamergotobutton{back}}
	\begin{figure}[]
		\includegraphics[scale = 0.19]{../code/julia/figures/simpleModel/calibrationFixed_RDSubsidy_welfareDecomp.pdf}
		\caption{Decomposition of the effect of an R\&D subsidy on welfare.}
		\label{calibration_RDSubsidy_welfareDecomp}
	\end{figure}
\end{frame}

\begin{frame}{R\&D subsidies: decomposing effect on growth} \label{plots:rd_subsidies2} 
	\hyperlink{RDsubsidy_table}{\beamergotobutton{back}}
	\begin{figure}[]
		\includegraphics[scale = 0.19]{../code/julia/figures/simpleModel/calibrationFixed_RDSubsidy_growthDecomp.pdf}
		\caption{Decomposition of the effect of an R\&D subsidy on the growth rate.}
		\label{calibration_RDSubsidy_growthDecomp}
	\end{figure}
\end{frame}

\begin{frame}{Intuition}\label{rd_subsidies:intuition}
	\hyperlink{RDsubsidy_table}{\beamergotobutton{back}}
	\begin{itemize}
		\item <+-> R\&D subsidy reduces cost of R\&D for entrants more than for incumbents
		\begin{itemize}
			\item future loss of business is not subsidized
		\end{itemize}
		\medskip
		\item <+-> Equilibrium reallocation of R\&D towards creative destruction
		\begin{itemize}
			\item reduces growth on the margin
		\end{itemize}
		\medskip
		\item <+-> Subsidies also induce the use of NCAs
		\begin{itemize}
			\item incumbents prefer to pay employees via (subsidized) wages instead of via future spinouts (cost not subsidized)
		\end{itemize}
	\end{itemize}
\end{frame}

\subsubsection{Targeted R\&D subsidy}

\begin{frame}{Decomposition of growth increase}\label{oi_rd_subsidies:decomposition_growth_decrease}
	\hyperlink{OI_RDsubsidy_table}{\beamergotobutton{back}}
	\begin{table}
		\centering
		\small
		\begin{tabular}{lclllll}
			\toprule \toprule
			&  & \multicolumn{4}{l}{Targeted R\&D Subsidy (\%)} \vspace{3pt} \tabularnewline
			Measure & Variable & 0 & 20 & 40 & 60 \tabularnewline
			\midrule
			Growth & $g$ & 1.49\% & 1.80\% & 2.01\% & 2.17\% \tabularnewline
			\multicolumn{1}{r}{\quad incumbents} & & 1.21\% & 1.56\% & 1.85\% & 2.05\% \tabularnewline
			\multicolumn{1}{r}{\quad entrants} & & 0.26\% & 0.21\% & 0.16\% & 0.11\% \tabularnewline
			\multicolumn{1}{r}{\quad spinouts} &  & 0.02\% & 0.02\% & 0.00\% & 0.00\% \tabularnewline
			\tabularnewline
			R\&D & &  &  &  & \tabularnewline
			\multicolumn{1}{r}{\quad incumbents} & $z / \bar{L}_{RD}$ & 54.0\% & 70.0\% & 82.9\% & 92.4\% \tabularnewline
			\multicolumn{1}{r}{\quad entrants} & $\hat{z} / \bar{L}_{RD}$ & 46.0\% & 30.0\% & 17.1\% & 7.6\% \tabularnewline
			\bottomrule
		\end{tabular}
	\end{table}
\end{frame}

\begin{frame}{Targeted R\&D subsidies: decomposing effect on welfare} \label{plots:oi_rd_subsidies1} 
	\hyperlink{OI_RDsubsidy_table}{\beamergotobutton{back}}
	\begin{figure}[]
		\includegraphics[scale = 0.19]{../code/julia/figures/simpleModel/calibrationFixed_RDSubsidyTargeted_welfareDecomp.pdf}
		\caption{Decomposition of the effect of a targeted R\&D subsidy on the welfare.}
		\label{calibration_OI_RDSubsidy_welfareDecomp}
	\end{figure}
\end{frame}

\begin{frame}{Targeted R\&D subsidies: decomposing effect on growth} \label{plots:oi_rd_subsidies2}
	\hyperlink{OI_RDsubsidy_table}{\beamergotobutton{back}}
	\begin{figure}[]
		\includegraphics[scale = 0.19]{../code/julia/figures/simpleModel/calibrationFixed_RDSubsidyTargeted_growthDecomp.pdf}
		\caption{Decomposition of the effect of a targeted R\&D subsidy on the growth rate.}
		\label{calibration_OI_RDSubsidy_growthDecomp}
	\end{figure}
\end{frame}

\begin{frame}{Intuition}\label{OI_RDsubsidy_intuition}
	\hyperlink{OI_RDsubsidy_table}{\beamergotobutton{back}}	
	\begin{itemize}
		\item <+-> Targeted R\&D subsidy corrects misallocation of R\&D
		\begin{itemize}
			\item tax on creative destruction R\&D isomorphic (except R\&D wage)
		\end{itemize}
		\smallskip
		\item <+-> Large subsidy induces use of NCAs
		\begin{itemize}
			\item same logic as untargeted subsidies
		\end{itemize}
		\smallskip
	\end{itemize}
\end{frame}


\subsubsection{Optimal policy}

\begin{frame}{Optimal policy} \label{plots:all_policies} 
	\hyperlink{all_policies_overview}{\beamergotobutton{back}}
	\begin{figure}[]
		\includegraphics[scale = 0.3]{../code/julia/figures/simpleModel/calibrationFixed_ALL_welfarePlot_contour.pdf}
		\caption{Summary of equilibrium for baseline parameter values and various values of $T_{RD,I}$ and $\kappa_c$. Optimum improves welfare by 11.42\%.}
		\label{calibration_ALL_summaryPlot}
	\end{figure}
\end{frame}

\begin{frame}{Intuition} \label{intuition:all_policies}
	\hyperlink{all_policies_overview}{\beamergotobutton{back}}
	\begin{itemize}
		\item <+-> NCAs are a socially costly way to provide incentive for incumbent R\&D
		\begin{itemize}
			\item sacrifice innovation by spinouts
		\end{itemize}
		\smallskip
		\item <+-> Dominated by large OI-targeted R\&D subsidies
		\smallskip
		\item <+-> Peak welfare improvement is 9.08\%
		\begin{itemize}
			\item modest improvement relative to only targeted R\&D subsidy
		\end{itemize}
		\item <+-> Still a role for NCAs if subsidies are not large
		\smallskip
	\end{itemize}
\end{frame}






\end{document}