\documentclass[english,usenames,dvipsnames]{beamer}
\usetheme{default}
\beamertemplatenavigationsymbolsempty
\setbeamertemplate{footline}[frame number]
\setbeamercolor{alerted text}{fg=blue2}
%\setbeamercolor{frametitle}{fg=blue2}
\usepackage[utf8]{inputenc}
\usepackage{caption}
\usepackage{booktabs}
\usepackage{appendixnumberbeamer}
\usepackage{babel}
\usepackage{amsmath}
\usepackage{hyperref}
\usepackage{geometry}
\usepackage{bbm}
\usepackage{amsthm}
\usepackage{verbatim}
%\usepackage{palatino}
\definecolor{red1}{RGB}{255,50,0}
\definecolor{blue1}{RGB}{80,80,255}
\definecolor{blue2}{rgb}{0.15,0.33,1}
\definecolor{green1}{RGB}{34,139,35}

\setbeamertemplate{itemize items}[default]


\title{Creating Creative Destruction: Endogenous Growth with Employee Spinouts}
\author{Nicolas Fernandez-Arias}
%\date{March 7, 2019}

\begin{document}

\maketitle


\begin{frame}{Notes from march presentation}
\begin{itemize}
	\footnotesize
	\item Redo introduction to more carefully define terms
	\item Also, improve theoretical motivation (i.e., argue why it is natural to want to augment the Grossman-Helpman model in this way -- endogenizing the knowledge spillovers in that model which are key for the welfare effects of it)
	\item Intensive margin regressions - are spinouts formed when there is R\&D of higher value than not? (discuss with Rowan)
	\item Try normalizing regressions by measures of cost, instead of assets (why? ask Benny)
	\item Hannah: try calibration with no non-WSO firms, calibrate and show that it's not the increased entry due to more non-WSO spinouts that's driving the welfare result
	\item Ernest: 
	\begin{itemize}
		\footnotesize
		\item Combine initial 2, maybe 3, motivation slides -- fewer facts (just say them) -- take out references to 'accounting' vs 'non-accounting' decompositions
		\item Fewer words
		\item Don't introduce WSO, OSO distinction until talking about their differences
		\item Write clear slide re: selection vs. learning from incumbents (what are my assumptions, when do they fail?)
		\item Put stuff in beginning about why this is a natural extension of quality ladders model exogenous knowledge spillovers
		\item Fewer words!
	\end{itemize}
\end{itemize}
\end{frame}

\begin{frame}{Notes}
\begin{itemize}
	\item Selection into Venture Source, Compustat -- need to address this carefully
	\item Augment data with data on the difference between venture-funded firms and regular firms, difference between public firms and private firms
	\item Make a picture of how the model works
	\item Product lines vs. firms vs. products -- be consistent in terminology
	\item Also, product vs firms -- this affects what curvature of R\&D technology one should use...I personally really like product interpretation better.
	\item Tell a clearer story of what frictions lead to these spinouts
	\item How good of a measure of competition is "same-industry spinouts" -- maybe a lot of measurement error
	\item Do lag regressions for accounting
	\item Be more clear about efficiency - -- planner wants spinouts and incumbents to do R\&D
\end{itemize}
\end{frame}



\section{Introduction}

\begin{frame}{Motivation}
\begin{itemize}
	\item Entry contributes significantly to productivity growth
	\begin{itemize}
		\item 25\% of labor productivity growth in manufacturing (Baily-Bartelsman-Haltiwanger 1996)
		\item ~25\% of aggregate productivity growth due to entrants (Akcigit-Kerr 2017)
	\end{itemize}
\end{itemize}
\end{frame}


\begin{frame}{Motivation}\label{motivation_spillovers}
\begin{itemize}
\item Knowledge spillovers important in productivity growth: Bloom 2013, others, etc.
\item Spinouts: employees found new firms using knowledge developed / learned at previous employer  \hyperlink{spinouts_facts_from_literature}{\beamergotobutton{stylized facts}}
\begin{itemize}
	\item Fairchild semiconductor / Silicon Valley; Detroit automakers
	\item Modern examples: Tinder $\rightarrow$ Bumble; Ableton $\rightarrow$ Bitwig Studio; 1010 Data $\rightarrow$ AltDG 
\end{itemize}
\end{itemize}
\end{frame}

\begin{frame}{Spinouts of Fairchild Semiconductor}
\begin{figure}
\includegraphics[scale=0.34]{../figures/fairchildren_early.png}
\caption{Source: Endeavor Insights}
\end{figure}
\end{frame}

\begin{frame}{Motivation}
\begin{itemize}
\item Non-compete agreement (NCA) enforcement policy debate
\begin{itemize}
\item Silicon Valley often attributed to California non-enforcement
\item States passing laws restricting enforcement of noncompetes (Hawaii in 2015, Massachussetts and Maryland in 2019, many others)
\end{itemize}
\end{itemize}
\end{frame}


\begin{frame}{NCA enforcement policy}
\begin{figure}
\includegraphics[scale=0.32,angle = -90]{./figures/bishara_map.png}
\caption{Map of non-compete enforcement policy in 2009, Bishara 2011.}
\end{figure}
\end{frame}


\begin{frame}{Theory}
\label{theory_big_picture}
\begin{itemize}
\item Schumpeter 1942, Arrow 1962, Romer 1986, Grossman-Helpman 1991, etc.: \textbf{\alert{underinvestment}} in knowledge due to \textbf{\alert{limited excludability}}
\item Patent literature: dynamic efficiency vs. static monopoly distortion tradeoff
\item Creative destruction by spinouts similar tradeoff
\begin{itemize}
\item Frictions (e.g. information asymmetry) cause breakdown in market for ideas (e.g. Anton \& Yao 1994/1995)
\item Employee cannot commit not to reduce bilateral value
\item Reduces incentive for R\&D, expands production possibilites
\item Growth and welfare effect unclear
\end{itemize}
\end{itemize}
\end{frame}

\begin{frame}{Related literature}
\begin{itemize}
\item Firm dynamics and endogenous growth
\begin{itemize}
\item Romer 1986, Grossman \& Helpman 1991, Aghion \& Howitt 1992, Klette \& Kortum 2004, Acmemoglu \& Akcigit 2012, Akcigit \& Kerr 2017
\end{itemize}
\item Models of employee spinouts
\begin{itemize}
\item Klepper 2002, Klepper \& Sleeper 2005, Anton \& Yao 1994/1995, Franco \& Filson 2006, Franco \& Mitchell 2008, Rauch 2015, Rossi-Hansberg \& Chatterjee 2012
\item Baslandze 2019
\end{itemize}
\item Empirics on employee mobility, spinouts
\begin{itemize}
\item Spawning of spinouts: Gompers et al. 2005, Garmaise 2011, Baslandze 2019, Babina \& Howell 2019
\item Effect on parent firms: Campbell et. al 2012, Wezel et al. 2006
\item Effect of non-compete enforcement: Garmaise 2009, Marx et al 2009, Samila-Sorenson 2011, Jeffers 2018, Shi 2018
\end{itemize}
\end{itemize}
\end{frame}

\begin{frame}{This project}
\begin{itemize}
\item Construct dataset linking Venture Source, Compustat and the NBER USPTO patent database
\item Document empirical facts relating corporate R\&D and spinout formation
\item Develop growth model with employee spinouts and creative destruction
\item Calibrate model to be consistent with micro estimates
\item Use model to study effect of increasing non-compete enforcement
\end{itemize}
\end{frame}

\section{Empirics}

\begin{frame}
\tableofcontents[currentsection]
\end{frame}


\begin{frame}{Venture Source}
\begin{itemize}
\item Venture Source
\begin{itemize}
\item Data on startups funded by VCs from 1986-2008: about 40,000 startups
\item Also includes funding by PEs and IPOs and acquisitions, and business status (e.g. product development, earning revenue, profitable)
\item \textbf{\alert{Key feature: }} employment biographies for founders / C-level / board members \\
\end{itemize}
\end{itemize}
\end{frame}

\begin{frame}{Merging with Compustat and patent data}
\begin{itemize}
\item Merge Venture Source with Compustat
\begin{itemize}
\item Founders are President, CEO, Chairman, CTOs and flagged "Founders" at startups
\end{itemize}
\item NBER-USPTO patent data
\begin{itemize}
\item Data on all USPTO-registered patents and their citations (also data on inventors, associated firms)
\item Merge to Compustat using gvkey
\end{itemize}
\end{itemize}
\end{frame}

\begin{frame}{Number of spinouts and non-spinouts}
\begin{figure}
\includegraphics[scale=0.45]{../figures/spinouts_entrants_counts.png}
\caption{\footnotesize Total spinout and non-spinout counts by year (only verified matches)}
\end{figure}
\end{frame}

\begin{frame}{Characteristics of spinout firms}
\begin{itemize}
	\item 17\% of entrants are spinouts
	\item 25\% of spinouts are within-industry spinouts (WSOs)
	\item Compared to non-spinouts, spinouts are
	\begin{itemize}
		\item 32\% more likely to attain revenue
		\item 31\% more likely to attain profitability
		\item 42\% more likely to IPO or be acquired
	\end{itemize}
\end{itemize}
\end{frame}


\begin{frame}{R\&D and spinout formations}
\begin{figure}
	\includegraphics[scale=0.45]{../figures/scatterPlot_RD-Spinouts-1yr-allFE.png}
	\caption{Relationship between R\&D and spinout counts at firm-year level after demeaning by parent firm, parent firm age, parent firm naics4-year, and parent firm state-year.}
\end{figure}
\end{frame}


\begin{frame}{Regression analysis}
\begin{itemize}
\item Specification
\begin{align*}
Y_{ijs[t+k]} &= \alpha_0 + \beta RD_{it} + \zeta X_{it} + \alpha_i + \xi_{a(i,t)} +  \gamma_{jt} + \sigma_{st} + \epsilon_{ijst}
\end{align*}
for firm $i$ in industry $j$, state $s$, and year $t$.
\item $RD_{it}$ is R\&D spending
\item $Y_{ijs[t+k]}$ is number of spinouts or WSO4 spinouts (weighted by founders) in $t+k$
\item $X_{it}$ are firm controls
\item $\alpha_i$, $\xi_{i,a(i,t)}$, $\gamma_{jt}$, $\sigma_{st}$ are firm, age, industry-year, and state-year fixed effects
\end{itemize}
\end{frame}

\begin{frame}{Regression results}
\begin{figure}
\includegraphics[scale=0.34]{./figures/regs_spinouts_accounting.png}
\end{figure}
\end{frame}

\begin{frame}{Regression results: WSOs}
\begin{figure}
\includegraphics[scale=0.34]{./figures/regs_spinouts_wso4_accounting.png}
\end{figure}
\end{frame}


\begin{frame}{Conclusion}
\begin{itemize}
\item Employee spinouts of publicly traded firms are a sizeable fraction of new Venture-funded firms
\item R\&D by publicly traded firms predicts future spinout formation 
\item \textbf{Rest of project: } endogenous growth model with employee spinouts and non-compete agreements, policy counterfactuals
\end{itemize}
\end{frame}


\section{Model}

\begin{frame}
\tableofcontents[currentsection]
\end{frame}

\begin{frame}{Model}
\begin{itemize}	
	\item Quality ladder model of endogenous growth through creative destruction 
	\begin{itemize}
		\item Builds on Grossman-Helpman 1991, Akcigit-Kerr 2017 
	\end{itemize}
	\item \textbf{New feature:} R\&D workers \textbf{\alert{learn on the job}} how to innovate, form within-industry spinouts and other industry spinouts
	\begin{itemize}
		\item Flow of knowledge \textbf{\alert{disincentives}} incumbent R\&D
		\item Stock of disseminated knowledge incentivizes incumbent innovation to \textbf{\alert{escape competition}}
	\end{itemize} 
\end{itemize}
\end{frame}

\begin{frame}{Model - Overview}
\begin{itemize}
	\item Individuals supply labor to production and R\&D
	\item Firms do R\&D to improve production technology, driving aggregate TFP growth
	\item R\&D by incumbents transfers knowledge to employees, generating new entrants with better entry technology
	\item Non-compete contracts
	\begin{itemize}
		\item Reduce flow of knowledge, \textbf{\alert{shrinking production possibilities frontier}}
		\item \textbf{\alert{May increase welfare}} due to effect on incumbent R\&D incentives
		\item Are relevant when expectation future spinouts \textbf{\alert{reduces bilateral value of firm-employee pair}} (e.g. DW loss of creative destruction, weakening of monopoly on R\&D)
	\end{itemize}
\end{itemize}
\end{frame}


\begin{frame}{Households}
\begin{itemize}
	\item Unit continuum of individuals are risk-neutral, discount rate $\rho > 0$, objective
	\begin{align*}
	\mathbb{E}_t \int_0^{\infty} e^{-\rho s} c_i(t+s) ds
	\end{align*}
	where $c_i(t)$ is household $i$ consumption of final good at time $t$
	\item One unit of labor, inelastic, used in final good or intermediate good production, or R\&D
\end{itemize}
\end{frame}

\begin{frame}{Final goods production}
\begin{itemize}
	\item \textbf{Final goods} production:
	\begin{align*}
	Y = F(L,\{q_j\},\{k_j\}) &= \frac{L^{\beta}}{1-\beta} \int_0^1 q_j^{\beta} k_j^{1-\beta}  dj 
	\end{align*}
	\item Labor $L$, intermediate goods $\{k_j\}$, qualities $\{q_j\}$, $j \in [0,1]$
	\item No storage $\rightarrow$ $Y = C - \overline{\kappa}$
\end{itemize}
\end{frame}

\begin{frame}{Intermediate goods production}
\begin{itemize}
	\item \textbf{Intermediate goods }production:
	\begin{align*}
	k_j = Q l_j
	\end{align*}
	where $Q = \int_0^1 q_j dj$ is average quality
\end{itemize}
\end{frame}

\begin{frame}{Incumbent innovation}
\begin{itemize}
\item Incumbent ($I$) discovers next innovation with Poisson intensity
\begin{align*}
\tau^I(z) &= \chi_I z^{1-\psi}
\end{align*}
where $z$ is units of R\&D, $\chi_I$ is R\&D productivity, and $\psi \in (0,1)$ measures decreasing returns to R\&D 
\item Scaling assumption: one unit of R\&D on quality $q$ requires $(q/Q)$ units of labor in R\&D
\end{itemize}
\end{frame}

\begin{frame}{Generation of spinout ideas}
\begin{itemize}
	\item Law of motion for mass of potential spinouts: $\dot{m}_j = \dot{m}_j^{\textrm{from $j$}} + \dot{m}_j^{\textrm{from $-j$}}$ where
	\begin{align*}
	\dot{m}_j^{\textrm{from $j$}}&= \theta \nu z_{I,j} (1-x_j)\\
	\dot{m}_j^{\textrm{from $-j$}} &= (1-\theta)\nu \int_0^1 \frac{q_j}{Q} z_{I,j'} dj'
	\end{align*}
	where $x_j = 1$ if the incumbent imposes a non-compete
	\item Rate $\theta \frac{Q}{q_j} \nu$ per flow unit of labor, R\&D worker obtains idea for spinout in line $j$
	\item Rate $(1-\theta) \nu$, idea generated in random line $j'$
	\item $\Rightarrow$ Individual state of product $j$ is $(q,m)$, aggregate state $d\mu(q,m,t)$
\end{itemize}
\end{frame}

\begin{frame}{Creative destruction}
\begin{itemize}
	\item For good $j$, unit mass of entrants (E), mass $m_j$ of spinouts (S), aggregates $z_E(j) = \int_0^1 z_E^e(j) de$, $z_S(j) = \int_0^{m} z_S^s(j) ds$
	\item Innovation technology
	\begin{align*}
	\textrm{Spinouts: }\tau^S(z,z_E+z_S) &= \chi_{S} z (z_E + z_S)^{-\psi} \\
	\textrm{Entrants: }\tau^E(z,z_E+z_S) &= \chi_{E} z (z_E + z_S)^{-\psi}
	\end{align*}
	\item Aggregate DRS due to lack of coordination: \textbf{\alert{duplication of research}} and \textbf{\alert{fishing out}} externalities
	\item Spinouts have \textbf{\alert{capacity constraint}} $z_S^s \le 1$ (normalization, can replace with DRS)
\end{itemize}
\end{frame}

\begin{frame}{Solving and closing the model}\label{closing_the_model}
\begin{itemize}
	\item Individual state of product is $(q,m)$, aggregate state $d\mu(q,m,t)$ time-varying in general 
	\item Incumbent, spinout and household HJBs \hyperlink{HJB_incumbent}{\beamergotobutton{incumbents}} \hyperlink{HJB_spinout}{\beamergotobutton{spinouts}} \hyperlink{HJB_household}{\beamergotobutton{households}}
	\item Guess and verify $V(q,m,t) = qV(m,t)$ and wages scaling with $Q_t$ to reduce complexity, and look for BGP \hyperlink{scaling_of_value_functions}{\beamergotobutton{more}}
	\item Static equilibrium conditions \hyperlink{static_eq_conditions}{\beamergotobutton{details}}
	\item Aggregation 
	\begin{itemize}
		\item Stationary density $\mu(m)$ and $E[q/Q|m]$ sufficient for BGP \hyperlink{aggregate_distribution_and_bgp}{\beamergotobutton{details}}
		\item Compute stationary density $\mu(m)$  \hyperlink{aggregation}{\beamergotobutton{details}}
		\item Compute stationary conditional mean $\gamma(m) = E_t[q/Q_t|m]$ \hyperlink{aggregation}{\beamergotobutton{details}}
	\end{itemize}
	\item Labor market clearing
\end{itemize}
\end{frame}

\begin{frame}{Household optimality}
\begin{itemize}
	\item Household labor supply decisions maximize value $U_{labor}$ of labor endowment
	\item Indifference condition: 
	\begin{align*}
	\bar{w} &= \overbrace{w(m)}^{\textrm{R\&D wage w/o NCA}} + \nu \Big(\overbrace{\theta \frac{Q_t}{q} W(m)}^{\textrm{Line $j$}} + \overbrace{(1-\theta) \mathcal{W}}^{\textrm{Other lines $-j$}} \Big) \\
	\bar{w} &= \underbrace{w^{NCA}}_{\textrm{R\&D wage w/ NCA}} + \underbrace{\nu (1-\theta) \mathcal{W}}_{\textrm{Only other lines $-j$}}
	\end{align*}
	for value of random spinout idea $j$
	\begin{align*}
	\mathcal{W} &= \int_0^{\infty} W(m) \tau(m) \mu(m) dm
	\end{align*}
\end{itemize}
\end{frame}

\begin{frame}{Incumbent optimal policies}
\begin{itemize}
	\item Incumbent optimal policies
	\small
	\begin{align*}
	x^*(m) &= 
	\begin{cases}
	1 & \textrm{if } w(m) - \theta \nu V'(m) > w^{NCA}(m) \\
	0 & \textrm{o.w.}
	\end{cases}\\ 
	z_I^*(m) &= \Bigg( \frac{x^*(m) w^{NCA}(m) + \Big(1-x^*(m)\Big) \Big(w(m) - \theta \nu V'(m)\Big)}{\chi_I (1-\psi) \Big(\lambda V(0) - V(m) \Big)} \Bigg)^{-1/\psi} 
	\end{align*}
	\normalsize
	\item Worker indifference condition implies $x = 1$ iff
	\begin{align*}
		\underbrace{- \theta \nu V'(m)}_{\textrm{$\mathbb{E}$[Harm to parent]}} > \underbrace{\nu (\theta W(m) + (1-\theta) \mathcal{W})}_{\textrm{$\mathbb{E}$[Benefit to employee]}}  
	\end{align*}
	so non-compete chosen to \textbf{\alert{maximize bilateral expected value}}
\end{itemize}
\end{frame}


\begin{frame}{Entrant optimal policies}
\begin{itemize}
	\item Spinouts earn rents and price out entrants
	\item Equilibrium entry
	\begin{align*}
	z_S(m) &= \min(m,Z_S) \\
	z_E(m) &= \max(0, Z_E - z_S(m))
	\end{align*} 
	where $Z_E,Z_S$ defined by
	\begin{align*}
	\chi_E Z_E^{-\psi} (1-\kappa) \lambda V(0) &= \bar{w} \\
	\chi_S Z_S^{-\psi} (1-\kappa) \lambda V(0) &= \bar{w} 
	\end{align*}
\end{itemize}
\end{frame}


\begin{frame}{Aggregate growth}
\begin{itemize}
	\item Growth rate of $Q(t)$
	\begin{align*}
	g_t &= (\lambda -1) \int_0^{\infty} \tau(m) \gamma(m) \mu(m) dm
	\end{align*}
	where $\tau(m)$ is the arrival rate of innovations in state $(m)$ at time $t$
\end{itemize}
\end{frame}


\section{Calibration}


\begin{frame}
\tableofcontents[currentsection]
\end{frame}

\begin{frame}{Identification}
\begin{center}
	\tiny
	\begin{tabular}{lll}
		\hline 
		\textbf{Parameter} &  \textbf{Description} & \textbf{Source / match / target} \\
		\hline 
		Chosen outside model & \\
		$\psi$ & Curvature R\&D technology & R\&D regressions from \\
		& & literature\\
		& & \\
		Calibrated directly &  \\
		$\chi_S / \chi_E$ & Spinouts / entrants R\&D prod. & Spinout / entrant probability \\
		& & of reaching profitablity \\
		$\theta$ & Fraction of spinouts that & WSO reg. coef. divided \\
		& are WSOs & by All Spinout reg. coef. \\
		$\rho$ & Discount factor & Interest rate \\
		$\beta$ & $\beta^{-1}$ EoS between intermediate goods & Profit / sales ratio \\
		& \\
		Indirect inference & \\
		$\lambda$ & Step size of innovations & Aggregate productivity growth \\
		$\chi_I$ & Incumbent R\&D prod. & Fraction of patents from \\
		& & existing firms \\
		$\kappa$ & Cost of creative destruction & Entry rate \\
		$\chi_E$ & Entrant R\&D prod. & R\&D employment / \\
		& & total employment \\
		$\nu$ & Spinout idea generation rate & Spinout entry rate implied \\
		& & by regressions \\
		\hline
	\end{tabular}
\end{center}
\end{frame}

\begin{frame}{Targeted moments}
\begin{center}
	\small
	\begin{tabular}{lll}
		\hline 
		\textbf{Moment}  & \textbf{Target} & \textbf{Model }\tabularnewline
		\hline 
		Growth rate $g$ & 0.02 & 0.01955 \tabularnewline
		Risk-adjusted interest rate $r$ & 0.05 & 0.05 \tabularnewline
		Profit / sales ratio & .109 & .109 \tabularnewline
		S/E profit. probability & 1.25 & 1.25 \tabularnewline
		\tabularnewline
		Incumbent innovation share & 0.9 & 0.5
		\tabularnewline
		Entry rate & 0.147 & 0.112
		\tabularnewline
		Spinout entry rate & 0.025 & 0.021
		\tabularnewline
		R\&D Labor allocation & 0.05 & 0.03
		\tabularnewline
		Agg R\&D / agg sales & 0.041 & 0.048
	\end{tabular}
\end{center}
\end{frame}


\begin{frame}{Parameters}
\begin{center}
	\small
	\begin{tabular}{lll}
		\hline 
		\textbf{Parameter} & \textbf{Value}\tabularnewline
		\hline 
		$\rho$ & 0.05\tabularnewline
		$\beta$ & 0.106\tabularnewline
		$\psi_I$ & 0.5\tabularnewline
		$\psi_E$ & 0.5\tabularnewline
		$\lambda$ & 1.0878\tabularnewline
		$\chi_I$ & 2.178 \tabularnewline
		$\chi_E$ & 0.654 \tabularnewline
		$\chi_S$ & 0.818 \tabularnewline
		$\kappa$ & 0.431 \tabularnewline
		$\nu$ & 0.409 \tabularnewline
		$\theta$ & 0.45
	\end{tabular}
	\label{calibration_parameters}
\end{center}
\end{frame}


\begin{frame}{Equilibrium innovation rates}
\begin{figure}
	\includegraphics[scale=0.45]{../code/julia/figures/plotsGR/innovation_rates_t.png}
	\caption{Model innovation hazard rates over time, conditional on no intervening innovations.}
\end{figure}
\end{frame}




\section{Counterfactual analysis}

\begin{frame}
\tableofcontents[currentsection]
\end{frame}



\begin{frame}{Counterfactual: full enforcement of non-competes}
\begin{itemize}
	\item Model calibrated assuming no availability of non-competes
	\item Counterfactual: allow use of non-competes
	\begin{itemize}
		\item Welfare increases by 1.23\% in current calibration
		\item Driven by 0.05 p.p. increase in BGP growth rate (1.7\% increase in growth rate)
		\item Incumbents hazard rate of innovation increases 5\%
		\item Entry rate by spinouts falls 40\%
	\end{itemize}
\end{itemize}
\end{frame}

\begin{frame}{Equilibrium knowledge transfer}
\begin{figure}
	\includegraphics[scale=0.45]{../code/julia/figures/plotsGR/effectiveRDWage_vs_t.png}
	\caption{In the calibration, unrestricted knowledge transfer to the employee harms the bilateral pair, creating a role for NCAs}
\end{figure}
\end{frame}



\appendix



\begin{frame}\label{spinouts_facts_from_literature}
\hyperlink{motivation_spillovers}{\beamergotobutton{back}}
\begin{itemize}
	\item All spinouts
	\begin{itemize}
		\item More patents / R\&D, sales growth, higher survival rates (Baslandze 2019) 
	\end{itemize}
	\item Within-industry spinouts (WSOs)
	\begin{itemize}
		\item 15\% of entrants; larger at entry, grow faster, higher survival rates (Muendler et al. 2012, Brazilian data)
	\end{itemize}
\end{itemize}
\end{frame}


\begin{frame}{Incumbent HJB}\label{HJB_incumbent}
\hyperlink{closing_the_model}{\beamergotobutton{back}}
\begin{itemize}
	\item Incumbent value $V(q,m,t)$ satisfies
	\tiny
	\begin{align*}
	(\rho + \tau_E& + \tau_S) V(q,m,t) = \pi(q,m,t) + V_t(q,m,t) + \overbrace{\bar{\sigma}V_m(q,m,t)}^{\textrm{Spinouts from $j'$}}  \nonumber \\ 
	&+ \max_{z \ge 0, x\in \{0,1\}} \Big\{ \underbrace{\chi_I z^{1-\psi_I} \Big[V(\lambda q,0,t) - V(q,m,t) \Big]}_{\textrm{EV of innovation}} \\
	&-(1-x) z \underbrace{\Big(\frac{q}{Q_t}\Big) \Big(  w(q,m,t) + \Big(\frac{q}{Q_t}\Big)^{-1} \theta \nu V_m(q,m,t) \Big)}_{\textrm{Eff. cost of R\&D w/o NCA}} - xz \underbrace{\Big( \frac{q}{Q_t} \Big) w^{NCA}(q,m,t)}_{\textrm{Cost of R\&D w/ NCA}} \Big\} 
	\end{align*}
\end{itemize}
\end{frame}


\begin{frame}{Spinout HJB}\label{HJB_spinout}
\hyperlink{closing_the_model}{\beamergotobutton{back}}
\begin{itemize}
	\item Spinout value $W(q,m,t)$ satisfies
	\footnotesize
	\begin{align*}
	(\rho  + \tau_E + \tau_S& + \tau_I)W(q,m,t) = W_t(q,m,t) + \bar{\sigma}W_m(q,m,t) \nonumber \\
	+& \max_{0 \le z \le 1} \Big\{ \underbrace{\chi_S z (z_E + z_S)^{-\psi_{SE}} (1-\kappa) V(\lambda q,0,t)}_{\textrm{Flow value of potential innovation}} - \underbrace{\Big(\frac{q}{Q_t}\Big) z w^{NCA}(q,m,t)}_{\textrm{R\&D cost}} \Big\} 
	\end{align*}
\end{itemize}
\end{frame}


\begin{frame}{Scaling of value functions}\label{scaling_of_value_functions}
\hyperlink{closing_the_model}{\beamergotobutton{back}}
\begin{itemize}
	\item Incumbent and spinout value functions $V(q,m,t),W(q,m,t)$, entrant and spinout behavior $z_E(q,m,t),\tau_E(q,m,t),z_S(q,m,t),\tau_S(q,m,t)$ 
	\item Guess and verify: BGP equilibrium with $V(q,m,t) = qV(m); W(q,m,t) = qW(m); z_E(q,m,t) = z_E(m)$ and similarly for $\tau_E,z_S,\tau_S$
	\item Worker indifference condition implies $w(q,m,t) = Q_t w(m)$ 
\end{itemize}
\end{frame}

\begin{frame}{Static equilibrium given $L_{RD}$}\label{static_eq_conditions}
\hyperlink{closing_the_model}{\beamergotobutton{back}}
\begin{itemize}
	\item Final goods wage $\bar{w}(t) = \Big(\frac{\beta}{1-\beta} (1-\beta)^{\frac{1-\beta}{\beta}} \Big)^{\beta} Q_t = C(\beta) Q_t$
	\item Final goods production labor allocation $L_F(t) = \frac{1 - L_{RD}(t)}{1 + \frac{1-\beta}{C(\beta)}}$
	\item Final goods production $Y(t) = \frac{\Big((1-\beta)C(\beta)^{-1}\Big)^{1-\beta}}{1-\beta} Q L_F(t)$
	\item Intermediate goods firm profits $\pi(q,m,t) = \pi(q,t) = \beta q L_F(t)$ 
\end{itemize}
\end{frame}

\begin{frame}{Aggregation}\label{aggregation}
\hyperlink{closing_the_model}{\beamergotobutton{back}}
\begin{itemize}
\item Stationary density $\mu(m)$ determined by KF equation
\begin{align*}
0 = - \frac{d}{dm} \Big( \sigma(m) \mu(m) \Big) - \tau(m) \mu(m)
\end{align*}
\item Solution
\begin{align*}
\mu(m) &= C_\mu e^{-\int_0^m \frac{\sigma'(m') + \tau(m')}{\sigma(m')}dm'} 
\end{align*}
\item Conditional mean $\gamma(m) = E[q/Q | m]$ given by 
\begin{align*}
\gamma(m) &= \tilde{\gamma}(s(m))
\end{align*}
where
\begin{align*}
\tilde{\gamma}(s) &= C_{\gamma} e^{-gs} \\
s(m) &= \int_0^m \frac{1}{\sigma(m')} dm' \\
1 &= \int_0^{\infty} \gamma(m) \mu(m) dm
\end{align*}
\end{itemize}
\end{frame}




\begin{frame}{Aggregate distribution and BGP}\label{aggregate_distribution_and_bgp}
\hyperlink{closing_the_model}{\beamergotobutton{back}}
\begin{itemize}
	\item Distribution $d\mu(q,m,t)$ shifts over time
	\begin{itemize}
		\item Marginal over $m$ constant over time
		\item Growth in $Q_t$ $\Rightarrow$ $\textrm{Pr}_t[q \le k | m]$ not constant
		\item Proportional growth in $q$, no exit for low $q/Q_t$ $\Rightarrow$ $\textrm{Pr}_t[q /Q_t\le k | m]$ not constant due to "fanning out"
		\item However, \textbf{\alert{conditional mean given $m$}} , $\gamma(m;t) := E_t[q/Q_t | m]$, constant over time
		\item Sufficient for BGP given policies derived above because
		\begin{itemize}
			\item Individually optimal $z$ (effective R\&D) depend on $m$ only
			\item R\&D labor demand given $(z,q)$ is $\frac{q}{Q}z$
			\item Growth contributions aggregate $g_t = (\lambda -1)\int_{q,m} \tau(m) \gamma(m) \mu(m) dm$
		\end{itemize}
	\end{itemize}
\end{itemize}
\end{frame}















\end{document}