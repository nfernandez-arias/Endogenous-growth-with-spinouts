\documentclass[english,usenames,dvipsnames]{beamer}
\usetheme{default}
\beamertemplatenavigationsymbolsempty
\setbeamertemplate{footline}[frame number]
\setbeamercolor{alerted text}{fg=blue1}
%\setbeamercolor{frametitle}{fg=blue2}
\usepackage[utf8]{inputenc}
\usepackage{caption}
\usepackage{booktabs}
\usepackage{appendixnumberbeamer}
\usepackage{babel}
\usepackage{amsmath}
\usepackage{hyperref}
\usepackage{geometry}
\usepackage{bbm}
\usepackage{amsthm}
\usepackage{verbatim}
%\usepackage{palatino}
\definecolor{red1}{RGB}{255,50,0}
\definecolor{blue1}{RGB}{80,80,255}
\definecolor{blue2}{rgb}{0.22,0.37,1}
\definecolor{green1}{RGB}{34,139,35}

\setbeamertemplate{itemize items}[default]


\title{Personal Wealth and Self-Employment}
\author{Aymeric Bellon \\ Discussant: Nicolas Fernandez-Arias}
\date{October 12, 2019}

\begin{document}
	
\maketitle

\begin{frame}{Summary}
\begin{itemize}
	\item Develop dataset of random wealth shocks due to proceeds to landowners from Barnett Shale drilling
	\item Study effect of wealth shocks of various types on labor-entrepreneurship choice
	\item Find evidence against financial frictions story, in favor of leisure hypothesis
\end{itemize}
\end{frame}

\begin{frame}{What I like about this paper}
\begin{itemize}
	\item Very interesting question about the determinants of entrepreneurship
	\item Impressive dataset drawn from several sources
	\item Compelling natural experiment
\end{itemize}
\end{frame}


\begin{frame}{Financial constraints and entrepreneurship}
\begin{itemize}
	\item Run regression with wealth shock interacted with proxies for credit constraints
	\begin{itemize}
		\item Credit score, debt-income ratio
	\end{itemize}
	\item Coefficient on interaction term is negative
	\item Endogeneity of credit score and debt / income weakens this interpretation
	\item Different reasons for credit constraints
	\begin{itemize}
		\item Low credit score, subprime, high debt-income: Constrained due to unobserved individual choices and characteristics, constraint signals urgent need for funds
		\item Constrained due to frictions in market for ideas -- different phenomenon in my view
	\end{itemize}
	\item Suggestion
	\begin{itemize}
		\item Collecting more individual-level variables to get a more complete picture of people with low credit scores?
		\item Random judge assignemnt + bankruptcy rulings? But not sure if possible given data available
	\end{itemize}
\end{itemize}
\end{frame}

\begin{frame}{High NPV projects}
\begin{itemize}
	\item Rate of entrepreneurship does not increase with education, suggesting these are not high NPV projects
	\item Larger "large payment" definition in this regression would be nice -- high education high NPV projects potentially require significantly more than \$50k in funding
	\item Can this be disciplined with data? How much does it cost to start the typical "high npv" business in question? 
\end{itemize}
\end{frame}

\begin{frame}{Interpreting the results}
\begin{itemize}
	\item Most people don't have high NPV ideas
	\item But most high NPV ideas might be associated with constrained people
	\item How to interpret / generalize from these results?
	\begin{itemize}
		\item As authors say, entrepreneurship is highly varied
		\item Random wealth shocks tend to create a specific kind of entrepreneur...
		\item But wealth / financial frictions may still be an important determinant of high NPV entrepreneurship
	\end{itemize}
\end{itemize}
\end{frame}

\begin{frame}{Leisure hypothesis}
\begin{itemize}
	\item For all sizes of windfalls, recipients tend to go back to work when they run out
	\item Even for large windfalls of $> $\$ 1 million 
	\item To me this is pretty compelling evidence that for most people in this sample, wealth is not enabling high NPV entrepreneurship projects
\end{itemize}
\end{frame}





\end{document}