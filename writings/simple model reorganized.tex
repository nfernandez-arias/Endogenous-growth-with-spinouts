\documentclass[11pt,english]{article}
\usepackage{lmodern}
\linespread{1.05}
%\usepackage{mathpazo}
%\usepackage{mathptmx}
%\usepackage{utopia}
\usepackage{microtype}



\usepackage{chngcntr}
\usepackage[nocomma]{optidef}

\usepackage[section]{placeins}
\usepackage[T1]{fontenc}
\usepackage[latin9]{inputenc}
\usepackage[dvipsnames]{xcolor}
\usepackage{geometry}

\usepackage{babel}
\usepackage{amsmath}
\usepackage{graphicx}
\usepackage{amsthm}
\usepackage{amssymb}
\usepackage{bm}
\usepackage{bbm}
\usepackage{amsfonts}

\usepackage{accents}
\newcommand\munderbar[1]{%
	\underaccent{\bar}{#1}}


\usepackage{svg}
\usepackage{booktabs}
\usepackage{caption}
\usepackage{blindtext}
%\renewcommand{\arraystretch}{1.2}
\usepackage{multirow}
\usepackage{float}
\usepackage{rotating}
\usepackage{mathtools}
\usepackage{chngcntr}

% TikZ stuff

\usepackage{tikz}
\usepackage{mathdots}
\usepackage{yhmath}
\usepackage{cancel}
\usepackage{color}
\usepackage{siunitx}
\usepackage{array}
\usepackage{gensymb}
\usepackage{tabularx}
\usetikzlibrary{fadings}
\usetikzlibrary{patterns}
\usetikzlibrary{shadows.blur}

\usepackage[font=small]{caption}
%\usepackage[printfigures]{figcaps}
%\usepackage[nomarkers]{endfloat}


%\usepackage{caption}
%\captionsetup{justification=raggedright,singlelinecheck=false}

\usepackage{courier}
\usepackage{verbatim}
\usepackage[round]{natbib}

\bibliographystyle{plainnat}

\definecolor{red1}{RGB}{128,0,0}
%\geometry{verbose,tmargin=1.25in,bmargin=1.25in,lmargin=1.25in,rmargin=1.25in}
\geometry{verbose,tmargin=1in,bmargin=1in,lmargin=1in,rmargin=1in}
\usepackage{setspace}

\usepackage[colorlinks=true, linkcolor={red!70!black}, citecolor={blue!50!black}, urlcolor={blue!80!black}]{hyperref}

\let\oldFootnote\footnote
\newcommand\nextToken\relax

\renewcommand\footnote[1]{%
	\oldFootnote{#1}\futurelet\nextToken\isFootnote}

\newcommand\isFootnote{%
	\ifx\footnote\nextToken\textsuperscript{,}\fi}

%\usepackage{esint}
\onehalfspacing

%\theoremstyle{remark}
%\newtheorem{remark}{Remark}
%\newtheorem{theorem}{Theorem}[section]
\newtheorem{assumption}{Assumption}
\newtheorem{proposition}{Proposition}
\newtheorem{proposition_corollary}{Corollary}[proposition]
\newtheorem{lemma}{Lemma}
\newtheorem{lemma_corollary}{Corollary}[lemma]


\theoremstyle{definition}
\newtheorem{definition}{Definition}

\begin{document}
	
\title{Endogenous Growth with Employee Spinouts, Creative Destruction, and Noncompete Agreements}

\author{Nicolas Fernandez-Arias} 
\date{\today \\ \small
	\href{https://drive.google.com/file/d/17bZL7-AUJKllRb78r9fIkZscnNdJwo1G/view?usp=sharing}{Click for most recent version}}

%\date{\today}

\maketitle


%\setcounter{tocdepth}{2}
%\tableofcontents

\begin{abstract}
	I study the effect of non-compete agreements (NCAs) on aggregate productivity growth. The central tradeoff I consider is that the availability of NCAs can prevent innovation by within-industry employee spinouts (firms whose founders previously worked in the same industry, hereafter just "spinouts") while increasing the incentive for own-product innovation by incumbent firms. To study this tradeoff, first I assemble a new dataset of venture capital funded startups matched to the previous employers of their founders. Regression analyses on these data reveal a statistically significant relationship between corporate R\&D and subsequent employee spinout formation. The relationship is economically significant, accounting for approximately 10\% of startup employment in the data. To study the implications of this microeconomic relationship, I extend a standard quality ladders model of endogenous growth to include R\&D-induced employee spinouts and noncompete contracts. I calibrate the model to match the micro estimates, aggregate moments and existing estimates from the literature. I then use the calibrated model as a laboratory to study the effect of reducing barriers to the use of NCAs. According to the calibrated model, reducing all barriers to the use of NCAs increases welfare by 3\% in consumption equivalent terms by improving the allocation of R\&D spending. Untargeted R\&D subsidies reduce growth and welfare by reallocating R\&D to creative destruction, which is less efficient at the margin in equilibrium. The optimal policy is a combination of large \textbf{[make precise]} R\&D subsidies targeted at own-product innovation and a ban on the use of NCAs. For small targeted R\&D subsidies \textbf{[make precise]}, eliminating barriers to the use of NCAs remains optimal.
\end{abstract}

\section{Introduction}

The entry of new firms contributes significantly to long-run growth in labor productivity. In turn, the entry of new firms often occurs on the basis of the diffusion of knowledge developed at incumbent firms. In particular, within-industry employee spinouts (hereafter spinouts) -- new firms whose founders previously worked at incumbent firms in the same industry -- typically rely to some degree on knowledge gained at those incumbents. The classic example of this phenomenon discussed in the literature is that of Fairchild Semiconductor, one of the first leading semiconductor firms of Silicon Valley. Fairchild was a spinout of Shockley Laboratories, another semiconductor firm, and in turn Fiarchild spawned some of the most well-known modern firms in the industry, such as Intel and AMD. 

Other examples of leading firms that were originally spinouts can easily be found in many industries and time periods. Continuing in the semiconductor industry, AMD in turn spawned NVIDIA, which is now a leading competitor of Intel and AMD. Qualcomm, the leading designer and manufacturer of networking chips, was founded by former employees of Linkabit, a previously leading company in digital communications. The case of the hard disk drive industry is studied in detail in \cite{franco_spin-outs_2006}. The case of the laser industry is considered in \cite{klepper_entry_2005}. In the software industry, Adobe was founded by former employees of the Xerox PARC research laboratory after they were unable to convince their boss to puruse a preliminary version of their product. The teleconferencing application Zoom was founded by a former vice president of engineering at Cisco systems who was working on teleconferencing technology there. SalesForce, a leading software company for HR management, was founded by a former employee of Oracle, another leading company in this area. In the pharmaceutical sector, Vertex Pharmaceuticals, which specializes in rational drug design, was founded by a scientist previously employed a Merck working on the same problem of rational drug design. In metal manufacturing, Steel Dynamics Inc (one of the top 5 steel producers in the US) was founded by former executives of Nucorp (the top steel producer). The early development of the automotive industry is discussed in detail in \cite{klepper_disagreements_2007}. Examples can also be found outside of the technology and manufacturing sectors. In management and business consulting, Bain \& Company was founded by a former vice president of Boston Consulting Group. In the financial sector, Bloomberg was founded by a former partner at Salomon Brothers who had set up internal computer systems. 

To avoid the possibility of competition by spinouts, incumbents may reduce their investment in R\&D and other forms of costly knowledge creation, as they require training potential future rivals. Alternatively, they may take steps to prevent the formation spinouts directly, mitigate the disincentive to their own innovative efforts but simultaneously preventing productivity-enhancing knowledge diffusion. The most salient example of this kind of effort is the non-compete agreement (NCA), a type of employment contracts. Such contracts preclude the employee from founding a competing firm after ceasing his or her current employment until a prespecified amount of time has passed. Thus, NCAs allow the employee to commit not to form spinouts. 

Noncompetes, therefore, inhibit innovation by spinouts while increasing the incentive for incumbents to innovate on their products. This observation motivates several questions. First, what is the effect of NCAs on aggregate productivity growth? Second, how does this depend on structural parameters that may be different in different locations, industries or time periods? Third, is it socially optimal to permit the free use of NCAs?

This paper takes a step towards a quantitative answer to these questions. To this end, first I construct and analyze a micro dataset of incumbent firms and startups and find a statistically significant and economically meaningful relationship between parent firm R\&D and subsequent employee startup formation. Using this finding as motivation, I develop a tractable model of endogenous growth which augments a standard quality ladders model (e.g., as described in \cite{acemoglu_introduction_2009}) to include R\&D-induced spinouts and NCAs. I then calibrate the model to capture the relationship between R\&D and employee entrepreneurship. I analzye policy counterfactuals using the calibrated model. First, I study the effect of reducing barriers to the use of NCAs. describe the model-implied optimal policy. I find that eliminiation of all barriers to NCAs can increase welfare by approximately 3\% in consumption-equivalent terms.  Next, I study how the effect of other policies differs from the predictions of the standard model. I find that R\&D subsidies can have the counterintuitive effect of reducing growth by misallocating R\&D labor to creative destruction instead of own-product innovation incumbents. They also induce incumbents to use noncompetes, further worsening the allocation. On the other hand, R\&D subsidies targeted at own-product innovation improve welfare (equivalently, taxes on R\&D targeted at creative destruction). The optimal policy involves very large targeted R\&D subsidies and a ban on NCAs. 

The model consists of a standard general equilibrium model of endogenous growth with creative destruction augmented to include employee entrepreneurship and NCAs. It assumes that, via a learning-by-doing type assumption, R\&D employees eventually gain the knowledge to form a competing spinout. This reduces the incentive for R\&D spending by the employers which fear being replaced. In equilibrium, NCAs are used exactly when they maximize the joint value of employment. This bilateral optimization occurs through the wage, as the employer is able to hire the employee at a lower wage if he does not require an NCA. In that case, R\&D employees effectively pay ex-ante for the damage they will cause. When this ex-ante implicit payment exceeds the employer's expected loss of profits due to future spinout formation, 

In order for the model to generate a role for NCAs, then, some friction in the employment relationship needs to be present so that, absent NCAs, bilaterally inefficient outcomes will occur. Specifically, it cannot be possible to synthesize a noncompete ex-post by, for example, buying an employee spinout only to shut it down (i.e., preventing the bilaterally inefficient outcome).\footnote{On the empirical side, the assumption is justified by the fact that , \cite{babina_entrepreneurial_2019} find that only 2-5\% of employee spinouts are bought out by their former employers.}  If this did occur, then in anticipation the employee would accept a lower wage and, on net, the firm would conduct R\&D as though it had imposed an NCA. I leave the particular friction unmodeled, simply assuming that there is no market in which WSOs can be sold to the incumbent firm that generated them. In reality, such a friction could relate to asymmetric information concerning the quality of the idea, disagreements between the employee and the employer concerning the idea's quality, or a simple lack of commitment power on the part of the employee (i.e., the employee cannot commit not to implement the idea even after selling it to his employer). In addition, antitrust law could prevent this type of ex-post buyout.

The result is a model in which WSOs expand the innovation possibilities frontier of the economy while having an ambiguous effect on equilibrium innovation and productivity growth. This is the case even though R\&D requires inelastically supplied labor because the threat of WSOs can worsen the allocation of R\&D across uses. Specifically, WSOs reduce own-product innovation, shifting R\&D labor into creative destruction, where it typically has a lower social return on the margin due to the business-stealing externality.\footnote{In the model, this relies on the fact that creative destruction involves creating a new product and hence is a different "innovation race" than own-product innovation.} Depending on the strength of this mechanism relative to the growth impact of WSOs, the freedom to use NCAs can increase or decrease the equilibrium growth rate. In turn, this depends on the model parameters. 

To discipline the model, I use microeconomic data on R\&D and spinout formation as well as aggregate data on productivity growth and the macroeconomy. First, I assemble a new dataset of parent firms and startups founded by their employees by combining Compustat data on publicly traded firms and private Venture Source data on VC-funded startups and their founders. Venture Source is the only dataset on startups with broad coverage of information on the most recent employer of the startup's key employees. Still, matching these datasets is somewhat challenging as there are is no common company identifier so it must be done by name only. This is non-trivial since companies go by different names. I solve this problem by using string matching techniques (e.g., regular expressions), Compsutat data on firm subsidiaries, and the merchant-mapper tool by Alternative Data Group, a startup that links credit card transactions data to firms using machine learning (itself a spinout of 1010 Data).\footnote{This component is in progress.} I define a startup as a spinout if its CEO, CTO, President, Chairman or Founder (1) was most recently employed at a firm in Compustat and (2) joined the startup in its first three years. Using this definition, I identify approximately 3,000 WSOs in the data. Finally, I match this dataset to the NBER-USPTO database, which contains information on all US patents.

However, R\&D spending is of course an endogenous variable: firms perform R\&D as a profit maximizing decision based on firm, industry, and aggregate economic conditions. A simple correlation of number of employee spinouts on lagged R\&D spending can therefore suffer from omitted variable bias unless such factors are controlled for. To control for this, I use firm, state-year, NAICS 4 digit industry-year (at 4-digit NAICS level), and firm age fixed effects, as well as firm-specific controls, such as employment, assets, Tobin's Q, and citation-weighted patents. Firm fixed effects control for unobservable firm-level factors that are time-invariant; firm age fixed effects control for the effect of the typical firm life cycle; and state-year and industry-year fixed effects attempt to control time-varying factors, such as shocks to investment opportunities or overall industry or state conditions. The resulting estimates vary little by specification and are typically statistically and economically significant. According to these estimates, R\&D can account for roughly 75\% of employee departures to WSOs in the data. I also consider the robustness of the results to this moment (and all other moments).  

I next calibrate the model using the estimates above as well as aggregate statistics and growth accounting estimates from \cite{garcia-macia_how_2019} and \cite{klenow_innovative_2020}. I also choose some parameters from the literature. The calibrated model can then be used to study the effect on productivity growth and welfare of reducing the cost of using NCAs to zero. As stated previously, I find that welfare rises 3\% in consumption-equivalent terms. I discuss how these results depend on the parameters and, via the calibration, on the value of the targeted moments. I also exhibit an alternative calibration with a smaller share of employment in young firms which returns the opposite conclusion and discuss why this results.

Finally, I study alternative policies that could improve welfare in this context. I consider R\&D subsidies, both overall and targeted specifically at own innovation by incumbent firms; a tax on creative destruction; and finally the combination of all studied policies. Two interesting findings emerge. First, untargeted R\&D subsidies (that is, that apply to creative destruction as well as own-product innovation) can have the unintended effect of shifting R\&D to entrants, and potentially even inducing the use of NCAs. The latter occurs because incumbents prefer to pay R\&D employees through subsidized wages rather than implicitly through future spinout formation (whose cost to the incumbent is not subsidized). The former is due to a similar reason. Second, targeted R\&D subsidies avoid this problem and, in combination with a ban on the use of NCAs, can achieve a first-best where the incumbent does enough R\&D while still allowing for spinouts to enter.\footnote{Not technically a first best, but as close as this model can come. I discuss this in detail in the body of the paper.} This might be a difficult policy to implement, and I discuss some of the potential barriers. I close with suggestions for future work. 

\paragraph{Related literature}

Some work has attempted to answer this question directly using empirical methods. Papers in this literature have typically used either cross-sectional and/or longitudinal variation in the state-level enforcement of non-competes.\footnote{Sometimes this variation is argued to be exogenous, either due to legislative error as in \cite{marx_mobility_2009} and \cite{marx_regional_2015}, or due to unexpected judicial precedent as in \cite{jeffers_impact_2018}. Often there is a control industry that is believed to be unaffected by the variation in CNC enforcement policy (e.g. CNCs typically are more difficult or impossible to enforce in the legal industry).} 

The results point to an important tradeoff between innovation by spinouts and investment by incumbent firms.  For example, \cite{stuart_liquidity_2003} find more local  entrepreneurship in response to local IPO (a "liquidity event") in regions not enforcing CNCs. \cite{marx_mobility_2009} finds that inventor mobility declines in response to an exogenous increase in non-compete enforcement. \cite{samila_venture_2010} finds that an increase in VC funding supply increases entrepreneurship more in states without non-compete restrictions, using an IV design. \cite{garmaise_ties_2011} finds that, in states where CNCs are more enforceable, managers are less mobile, have lower compensation, and invest less in their human capital, to the point of offsetting increased investments by the firm. 

On the other hand, \cite{conti_non-competition_2014} finds evidence that non-compete enforceability leads to incumbent firms pursuing riskier R\&D projects. \cite{colombo_does_2013} finds evidence that easier spinout formation -- proxied by access to finance -- leads to a reduction in incumbent firm knowledge investments.  Most recently, \cite{jeffers_impact_2018} uses data on influential state-level court precedents matched with LinkedIn data and finds that enforcement indeed reduces spinout formation while increasing capital investment by incumbent firms. Finally, \cite{marx_regional_2015} finds that CNC enforcement leads to inventor mobility out of the state, suggesting that cross-sectional differences in outcomes could be in part due to reallocation across states of the human capital inputs to entrepreneurship and innovation. 

The most directly related empirical paper is \cite{babina_entrepreneurial_2019}. They find evidence of a causal relationship from corporate R\&D spending to employee spinout formation.\footnote{To my knowledge, the papers are contemporaneous} My empirical analysis confirms their findings on a subset of firms particularly connected with productivity growth, VC-funded startups.  Together, they motivate the use of a model like the one developed in this paper.

This project also builds on a body of theoretical work that has considered this question in different frameworks. \cite{franco_spin-outs_2006} develops a model in which employees learn from their employers and use this knowledge to form spinouts. They emphasize the "paying for knowledge" effect, whereby employees implicitly pay for the knowledge they take from the parent firm through lower equilibrium wages. Importantly, they assume spinout firms do not steal business directly from their parents. The only effect of a spinout on the parent firm is a reduction in the price of the output good, which the parent firm is assumed not to take into account. Moreover, there is one good and it is produced competitively (individually decreasing returns to scale ensures that one producer does not dominate). This, combined with the "paying for knowledge" mechanism, ensures that the equilibrium allocation is Pareto efficient, even without resorting to elaborate labor contracts. In my case, by contrast, the equilibrium is generally not efficient due to creative destruction by entrants and spinouts as well as positive externalities of innovation.\footnote{This includes consumer surplus effects due to market power as well as innovation externalities in the vein of "standing on the shoulders of giants", which are standard in endogenous growth models}

\cite{franco_covenants_2008} studies a two-period, two-region model with employee spinouts in which the region which does not enforce CNCs initially lags but eventually overtakes the region in which CNCs are enforced. In the first period, entry is more valuable in the enforcing region. But in the second period, spinouts enter in the non-enforcing region, there is Cournot competition with parent firms in the product market, and output increases relative to the enforcing region. The analysis emphasizes how asymmetric information about whether an employee has learned leads some firms in the non-enforcing region to allow spinouts (assuming firms cannot commit to wage backloading). This can be taken as a rough microfoundation of my assumption that labor contracts are "simple" in  a non-enforcing region: just a wage, with no attempts at retention in the case of learning. Relative to this study, my analysis considers a fully dynamic model rather than two-period model, an important consideration as today's spinouts are tomorrow's incumbents. In addition, I emphasize the role of R\&D investment in spawning spinout firms and the potential additional disincentives to innovation that this can entail. My concession is that I study a model with only one enforcement regime.\footnote{I plan to pursue this idea in future work.} 

\cite{shi_restrictions_2018} uses a rich model of contracting disciplined by data on executive non-compete contracts to study the effect of non-competes on executive mobility and firm investment. She finds that the optimal policy is to somewhat restrict the permitted duration of CNCs. Her approach allows her to study the optimal contracting problem in more detail than in mine. However, she is mainly interested in an environment where the firm's productivity is embodied in the worker and where the concern is poaching, not spinout formation. Also, her calibration considers firm investment in capital expenditures, whereas I am interested in innovative investment in R\&D. Finally she performs her analysis in a partial equilibrium framework while I consider a fully specified general equilibrium model.

\cite{baslandze_spinout_2019}, the study closest to this paper, studies the effect of spinout entrepreneurship on entry and growth. She also uses a GE model of endogenous growth with employee spinouts, using Compustat and NBER-USPTO patent data to discipline the analysis. She finds the optimal policy is to ban NCAs. However she is focused in her framework on the harm from losing a valuable employee rather than the harm from competition with the parent firm. My paper focuses instead on creative destruction of the parent firm using knowledge rather than the loss of productivity from losing valuable employees. The other key difference is that I model the use of NCAs while her analysis assumes that they are used when available. To my knowledge, mine is the first general equilibrium model of endogenous growth to have this kind of feature.

\section{Empirics of R\&D and spinout formation}\label{sec:empirics}

In this section I document the empirical relationship between R\&D spending and employee spinout formation which motivates the model of Section \ref{sec:model} and informs the calibration and quantitative analysis in Sections \ref{sec:calibration} and \ref{sec:policy_analysis}. I construct a new micro dataset by matching data from Venture Source, Compustat, and the NBER-USPTO patent database.\footnote{The same kind of matching (Venture Source and Compustat) was previously done by \cite{gompers_entrepreneurial_2005} using data through 1999.} Using this dataset, I analyze the microeconomic relationship between firm-level R\&D spending and subsequent employee spinout formation using a regression analysis.

\subsection{Data}

\subsubsection{Sources}

\paragraph{VentureSource}

The data on startups comes from Venture Source (VS), a proprietary dataset containing information on venture capital (VC) firms and VC-funded startups.\footnote{When starting this project the data were owned by Dow Jones but they have since been sold to CB insights.} I use a subsample of the data for US-based startups founded between 1987 and 2009 which contain information on their founding year. The data cover 23,434 startups, 89,382 financing rounds, and 297,119 individual-firm pairs. For each financing round, the data contain information on valuation, amount raised, and status of the business at the time of the round, such as employment and revenue (with some missing data). Most importantly for this analysis, the data contain employment biographies for each of the startup's founders and key employees (C-level, high-ranking executives and managers) and board members. In this regard, Venture Source is unique among VC investment databases. Some summary information about the dataset is contained in \autoref{table:VS_summaryTable}. The dataset is described in detail in \cite{kaplan_how_2002} and \cite{kaplan_venture_2016}. 

\paragraph{Compustat}

The data on R\&D spending comes from Compustat, a comprehensive database of fundamental financial and market information on publicly traded companies. I consider a subsample consisting of all firms headquartered in the United States in operation at any point between 1986 and 2006, consisting of 20,534 firms. In addition to data on R\&D spending, the Compustat data contain information on industrial classification and time-varying firm characteristics and balance sheet information.

\paragraph{NBER-USPTO}

The data on patents comes from the NBER-USPTO database, which contains comprehensive information on all patents granted in the United States from 1976 to 2006 and comes linked to Compustat. I consider the subsample of patents assigned to US firms, consisting of 1,457,136 patents. 

\subsubsection{Construction of dataset}

The first step is to define which startup employees should be classified as founders. The Venture Source data contain information on high level employees and board members. For the purposes of this study, however, only those employees whose human capital is crucial to the value proposition of the startup should be considered founders. I  therefore restrict attention to employees who (1) have a job title related to the core operations of the firm (Founder, Chief, CEO, CTO or President) and (2) join the startup in its first three years since its founding date. When information on the individual's date of joining the startup is missing, I impute it as the founding date of the startup. \autoref{table:VS_founder2_titlesSummaryTable} shows a breakdown of the most frequent job titles.

% latex table generated in R 3.6.3 by xtable 1.8-4 package
% Wed Sep  2 15:53:15 2020
\begin{table}[]
\centering
\begingroup\normalsize
\begin{tabular}{rll}
  \toprule
Title & Individuals & Percentage \\ 
  \midrule
Chief Executive Officer & 10306 & 24.8 \\ 
  Chief Technology Officer & 8036 & 19.3 \\ 
  President \& CEO & 7806 & 18.8 \\ 
  Chief & 5400 & 13.0 \\ 
  President & 3969 & 9.5 \\ 
  Founder & 2634 & 6.3 \\ 
  Chairman \& CEO & 2385 & 5.7 \\ 
  President \& COO & 961 & 2.3 \\ 
  President \& Chairman & 104 & 0.2 \\ 
   \bottomrule
\end{tabular}
\endgroup
\caption{Most frequent titles among key founders in VS data.} 
\label{table:VS_founder2_titlesSummaryTable}
\end{table}


Next, I extract information on previous employment using the employment biography data in VS. This requires some effort as the data come in string form and also includes the prior job titles.\footnote{Because VS biographies are text fields, this requires some cleaning. The VS biographical data comes in a structured format, allowing parsing by regular expressions. Each prior job is represented in the format ``<position>, <employer>'' and different jobs are separated by ``;''. Job spells can be easily separated by splitting the string on the character ``;''. It is slightly more involved to separate positions from employer. It is not sufficient to simply separate on the right-most character ``,'' as <employer> can contain ``,''. However, in almost all cases, <employer> contains at most one ``,'' (e.g., in ``Microsoft, Inc.''), and in virtually all of these cases, the comma precedes one of a few strings (e.g. ``LLC'',``Inc'',``Corp''). Hence, I use a two-pass approach: first I split on the last ``,''; for employers that end up consisting only of corporate structure (e.g., ``LLC'', etc.), I split on the penultimate ``,'' instead. } The results of this procedure are summarized in \autoref{table:VS_previousEmployersNoPositionsSummaryTable}. The top twenty previous employers include several well-known technology firms such as Google, Microsoft, and IBM.

% latex table generated in R 3.6.3 by xtable 1.8-4 package
% Wed Sep  2 15:53:18 2020
\begin{table}[]
\centering
\begingroup\normalsize
\begin{tabular}{rlrl}
  \toprule
Employer & Count & Employer & Count \\ 
  \midrule
IBM & 197 & Stanford University & 53 \\ 
  Microsoft & 182 & Motorola & 53 \\ 
  Cisco Systems & 135 & Lucent Technologies & 49 \\ 
  Oracle & 124 & Accenture & 48 \\ 
  Sun Microsystems & 104 & AOL & 44 \\ 
  AT\&T & 98 & Nortel Networks & 44 \\ 
  Verizon & 93 & Texas Instruments & 44 \\ 
  Google & 86 & Broadcom & 44 \\ 
  Hewlett-Packard & 77 & Andersen Consulting & 41 \\ 
  Intel & 74 & Pfizer & 41 \\ 
   \bottomrule
\end{tabular}
\endgroup
\caption{Top 20 previous employers for founder2 founders in VS data.} 
\label{table:VS_previousEmployersNoPositionsSummaryTable}
\end{table}


I then match the data on each founder's previous employer to firm name variable in Compustat. This requires some effort to standardize the spelling of names. To do this, I use regular expressions, trimming e.g. ``Inc.", ``Corp.'' and variants thereof from each entry and converting to lower case. I look for exact matches to previous employers in the VS data. For previous employers in VS that do not match with any names in Compustat, I check against the business segment names, available from the Compustat Segments database.

Having linked startups to their founders' previous employers, the next step is to determine which of these startups compete with the parent firm. The best measure available for the product market of publicly traded firms is the self-reported NAICS code. While VS does not contain NAICS classifications, it does document industry using a classification that, for the most part, coincides (at least in name) with NAICS 4 or 5 digit categories. I therefore manually construct a crosswalk between the two classification schemes and use this to assign 4-digit NAICS codes to startups in VS. Then, I classify a founder-startup observation as a WSO whenever the startup is in the same 4-digit NAICS category as its parent.\footnote{VS does include a ``Competition" variable listing competitors. However, only 20\% of startups have this variable filled in: 30\% in the 90s, but dropping to around 10\% by the end of the sample.} 

Following this procedure, approximately 20\% of startup founders are matched to a public firm in Compustat. Of these, one third came from an employer in the same four digit NAICS industry as the startup.\footnote{The remainder were either most recently employed at a firm that is not publicly traded, and so is excluded from Compustat, or they are missed by the matching algorithm. This is likely due to (1) significantly different spellings or naming conventions between VS and Compustat, or (2) they worked at a subsidiary of a publicly traded firm, but the subsidiary is not named in the Compustat Segments database.} \autoref{figure:industry_row_heatmap_naics2_founder2} documents the joint distribution of parent industry and child industry, defined by 2-digit NAICS codes for easier visualization even though I use 4-digit NAICS industries to define WSOs. The raw joint distribution is too heavily concentrated to be easily visualized in this way, so instead I show the distribution of child industry (column) conditional on parent industry (row). The dark diagonal line reflects the prevalence of WSOs.\footnote{\autoref{figure:industry_column_heatmap_naics2_founder2} shows the joint distribution the other way around, with the probability of parent industry conditional on child industry.}\footnote{In \autoref{figure:industry_row_heatmap_naics2_founder2}, the dark vertical line at column 51 (Information) indicates that parent firms of all industries tend to spawn spinouts in that industry. Similar dark regions appear at columns 54 (Professional, Scientific and Technical Services), and 32 and 33 (Manufacturing). In \autoref{figure:industry_column_heatmap_naics2_founder2}, the dark horizontal lines at 51 and to a lesser extend 32, 33, 52 and 54 indicate that child firms of all industries tend to have founders from those industries.} The match is further documented in \autoref{table:GStable_founder2}, which corresponds roughly to Table 1 of \cite{gompers_entrepreneurial_2005}.\footnote{It is important to note that I do not replicate their findings. My algorithm finds fewer matches early on and more matches later in their sample period but always a smaller fraction.} 

\begin{figure}[]
	\centering
	\includegraphics[scale=0.7]{../empirics/figures/plots/industry_row_heatmap_naics2_founder2_ggplot2.png}
	\caption{Heatmap displaying the distribution of child 2-digit NAICS code (column), conditional on parent NAICS code (row). Darker hues indicate a higher number of founders. To facilitate visualization, counts are normalized to have mean zero and unit standard deviation at the Parent NAICS level.}
	\label{figure:industry_row_heatmap_naics2_founder2}
\end{figure}

\subsection{Corporate R\&D and spinout formation}\label{subsec:empirics:corpRDandspinouts}

\subsubsection{Preliminaries}

\begin{figure}[]
	\centering
	\includegraphics[scale= 0.8]{../empirics/figures/scatterPlot_RD-FoundersWSO4_dIntersection.png}
	\caption{Scatterplot of average yearly founder counts (restricted to same 4-digit NAICS industry) in $t+1,t+2,t+3$ versus average yearly R\&D spending in $t,t-1,t-2$. R\&D spending is measured in millions of 2012 dollars (using the R\&D deflator) and additionally deflated by productivity growth (also with 2012 as the base year). Finally, both R\&D and founder counts are demeaned at the firm and industry-state-age-year levels.}
	\label{figure:scatterPlot_RD-FoundersWSO4_dIntersection2}
\end{figure}

\autoref{figure:scatterPlot_RD-FoundersWSO4_dIntersection2} visualizes the relationship between corporate R\&D and spinout formation in a binned scatterplot at the parent firm-year level. On the horizontal axis is average yearly real effective R\&D spending by the parent firm in years $t-2,t-1,t$ and on the vertical axis is the yearly number of employees founding startups in years $t+1,t+2,t+3$.\footnote{Real effective R\&D spending is calculated from nominal R\&D spending by first deflating using the R\&D investment deflator, and then further deflated by productivity growth since 2012. This captures the assumption in the model that a given ratio of real R\&D to real GDP generates a given rate of spinout formation.}. Both variables are demeaned at the firm and  state-industry-age-year levels. The solid line shows the best fit of a straight line through all of the resulting points. It indicates the presence of a noisy but positive relationship between firm-level deviations in R\&D spending and subsequent firm-level deviations in employee entrepreneurship. 

\subsubsection{Regressions}

To properly assess the statistical significance of this relationship, as well as to control for confounding factors, I conduct a regression analysis. Regarding statistical significance, standard errors should be calculated taking into account the fact that regression errors may be correlated over time in a given firm, state, or industry. To account for this, I use multiway clustering of standard errors at the state and four-digit industry levels. Regarding confounders, the concern is that time-varying firm-level variables that are not constant at the state-industry-age-year level could be associated with both R\&D spending and spinout formation in both positive and negative ways. For example, investment opportunities specific to the firm's technological niche within its state-industry-age-year would induce R\&D spending and potentially within-industry spinout formation by its employees pursuing related ideas. On the other hand, poor management or other tribulations at the parent firm could reduce R\&D spending and simultaneously induce employee departures. To control for such factors, I include include time-varying firm controls (employment, cumulative patents, assets, intangible assets, net income, capital expenditures, and Tobin's Q). To control for unobservable factors, I include fixed effects at the firm, state-year, industry-year, and age of the firm. I do not pursue an instrumental variables identification strategy because there are problems with the usual instrument for firm-level R\&D spending. In particular, it may fail to satisfy the exclusion restriction because it alters economic conditions that affects the incentive for R\&D and startup formation separately.\footnote{The reason is that the standard instrument for firm-level R\&D spending is based on tax incentives for R\&D, as developed in \cite{bloom_identifying_2013}. Most of the variation in this instrument is based on state-level variables, which would also simultaneously affect the incentive for spinouts (most of which are in the same state as the parent firm) to conduct R\&D, either positively because they have access to R\&D subsidies, or negatively because the firm at which they work has the ability to pay them a higher wage for the same project due to increased R\&D subsidies. This would violate the exclusion restriction. In a working paper, \cite{babina_entrepreneurial_2019} argue that the exclusion restriction may be satisfied -- because R\&D tax incentives only reduce corporate taxes on profits, which startups don't have yet -- and use an IV approach based on these instruments. They find that the magnitude of the relationship between R\&D and spinout formation is five times larger in their IV estimation than when using OLS estimation. They conclude that the IV captures a local average treatment effect (LATE) that differs substantially from the ATE, which is what I am interested in here; and that they ultimately find their OLS estimates more reliable. Two other interpretations are that (1) the exclusion restriction is in fact violated (e.g., startups can carry forward their tax breaks) and (2) the omitted variable bias in the OLS regression leads to a downward, rather than an upward bias. All three interpretations suggest that IV estimation may not appropriate in this case.} 

\begin{table}[]
	\centering
	{
\def\sym#1{\ifmmode^{#1}\else\(^{#1}\)\fi}
\begin{tabular}{l*{3}{c}}
\toprule
                    &\multicolumn{1}{c}{(1)}&\multicolumn{1}{c}{(2)}&\multicolumn{1}{c}{(3)}\\
                    &\multicolumn{1}{c}{WSO4}&\multicolumn{1}{c}{$\frac{\textrm{WSO4}}{\textrm{Assets}}$}&\multicolumn{1}{c}{WSO4}\\
\midrule
R\&D                &        0.24\sym{***}&                     &                     \\
                    &     (0.052)         &                     &                     \\
\addlinespace
$\frac{\textrm{R\&D}}{\textrm{Assets}}$&                     &        0.26\sym{***}&                     \\
                    &                     &     (0.077)         &                     \\
\addlinespace
log(R\&D)           &                     &                     &        0.92\sym{***}\\
                    &                     &                     &      (0.29)         \\
\addlinespace
Firm FE             &         Yes         &         Yes         &         Yes         \\
\addlinespace
Age FE              &         Yes         &         Yes         &         Yes         \\
\addlinespace
Industry-Year FE    &         Yes         &         Yes         &         Yes         \\
\addlinespace
State-Year FE       &         Yes         &         Yes         &          No         \\
\midrule
Clustering          &naics4 Statecode         &naics4 Statecode         &       gvkey         \\
R-squared (adj.)    &        0.61         &        0.26         &                     \\
R-squared (within, adj)&        0.23         &      0.0019         &                     \\
Observations        &       56961         &       56953         &         470         \\
\bottomrule
\multicolumn{4}{l}{\footnotesize Standard errors in parentheses}\\
\multicolumn{4}{l}{\footnotesize \sym{*} \(p<0.1\), \sym{**} \(p<0.05\), \sym{***} \(p<0.01\)}\\
\end{tabular}
}

	\caption{The regressions above relate corporate R\&D to the entrepreneurship decisions of employees. The dependent variable is average yearly number of founders joining startups in years $t+1,t+2,t+3$. The independent variables are averages over $t,t-1,t-2$. Firm controls are employment, assets, intangible assets, investment, net income, cumulative citation-weighted patents, and the product of Tobin's Q and Assets (i.e., firm market value).}
	\label{table:RDandSpinoutFormation_headlingRegs}
\end{table}


Table \ref{table:RDandSpinoutFormation_headlingRegs} displays the results of three regression analyses. The first column is the regression
\begin{align}
	WSO_{it} &= \beta R\&D_{it} + \gamma X_{it} + \alpha_{i} + \xi_{j(i)t} + \sigma_{s(i)t} + \eta_{a(i,t)} + \epsilon_{it}.
\end{align}
As in the scatterplot, the dependent variable $WSO_{it}$ is again the (annualized) number of founders previously employed at firm $i$ joining startups in years $t+1,t+2,t+3$. $R\&D_{it}$ and the firm controls $X_{it}$ are calculated as moving averages over years $t,t-1,t-2$. The parameters $\alpha_i, \xi_{j(i)t}, \sigma_{s(i)t}, \eta_{a(i,t)}$ represent the firm, industry-year, state-year and age fixed effects, respectively. 

The result is statistically significant at the 1\% level. The robustness of the result to the inclusion of controls and firm, age, industry-year, and state-year fixed effects is encouraging. However, in this specification, industry-year, state-year and age fixed effects may not absorb many time-varying shocks due to what amounts to a misspecification problem. To be precise, such shocks are likely to affect firm outcomes more in absolute terms for larger firms but the fixed effects are restricted to be constant across firms in each industry-year, state-year and age group. Because firms do in fact vary greatly in size even within these categories, this inevitably leaves much firm-level variation in the error which, if correlated with $R\&D_{it}$, leads to an inconsistent coefficient estimate on R\&D spending.\footnote{This is the reason I control for market value in the levels regression rather than Tobin's Q}

To address this, I consider a specification where all regressors are divided by a trailing five-year moving average of firm assets. This amounts to the specification
\begin{align}
	\tilde{WSO}_{it} &= \beta \tilde{R\&D}_{it} + \gamma \tilde{X}_{it} + \alpha_{i} + \xi_{j(i)t} + \sigma_{s(i)t} + \eta_{a(i,t)} + \epsilon_{it},
\end{align}

where a $\tilde{}$ superscript denotes that the variable is normalized by a trailing five-year moving average of assets. The result is displayed in the second column of Table \ref{table:RDandSpinoutFormation_headlingRegs}. The estimate for the coefficient on R\&D spending is significant at the 1\% level and indistinguishable from the previous estimate. To estimate this regression, I use WLS weighting by firm level assets. The purpose of this is on efficiency and consistency grounds. To the extent that large firms conduct various statistically independent R\&D projects, the errors will be smaller for those firms; WLS weighting by assets therefore improves efficiency (and in fact it does reduce standard errors). Second, because I am interested in the aggregate implications of this mechanism, in the event that the effects are heterogeneous across firms it is preferable for the estimate to be driven by firms with more assets (which conduct more R\&D all else equal, by construction).\footnote{The estimates using OLS (unreported) are larger and have somewhat larger standard errors.}

Finally, the third column shows the results of a Poisson pseudo-Maximum Likelihood (PPML) estimation, a specification of 
\begin{align}
	WSO_{it} &= \exp \Big\{ \beta \log R\&D_{it} + \gamma \log X_{it} + \alpha_{i} + \xi_{j(i)t} + \sigma_{s(i)t} + \eta_{a(i,t)} \Big\} + \epsilon_{it}.
\end{align}
This specification can be thought of as a log-linear regression which can handle zeros in the dependent variable. It is also the specification in the model, as the rate of employee departures is assumed to depend on R\&D spending via a Poisson process.\footnote{However, even if the conditional distribution of spinout counts is actually not Poisson, a PPML regression consistently estimates the conditional mean} In this case, the coefficient on R\&D spending corresponds to an elasticity. The regression estimate is both statistically insignificant and statistically indistinguishable from 1. This is encouraging as a unit elasticity is the assumption in the model and in the previous two OLS specifications.\footnote{In the regression without fixed effects, the sample size is much smaller due to the fact that the specification now must take logarithms of RHS variables (this is the correct specification in the model). When fixed effects are added, the sample is additionally restricted because many of the induced categories have only one observation with non-missing data (due to the log specification).} 

Based on the regression analysis above, I conclude that the data are consistent with an effect of corporate R\&D spending on employee spinout formation in the same 4-digit NAICS industry. While the absence of truly exogenous variation in R\&D spending prevents any final judgments, the preceding analysis justifies considering the implications of interpreting the previous results causally, which is the implicit assumption of the calibration. In the next section, I consider the economic magnitude of the OLS estimates just calculated.

\subsubsection{Economic magnitude}

In each year $t$, I compute $\hat{WSO}_{it}$, the expected number of founders per year starting firms in years $t+1,t+2,t+2$ by multiplying average R\&D spending in years $t,t-1,t-2$ by the average of the coefficient estimates from the first two regressions. Aggregating across firms and years in the sample period, the regression estimates are economically significant, conservatively accounting for about 90\% of the WSO spinout founders observed in the data.\footnote{\autoref{figure:founder2_founders_f3_Accounting_industryYear} shows a similar accounting exercise using data at the industry-year level. focusing on NAICS industries 3 (manufacturing) and 5 (information), which are responsible for the vast majority of private sector R\&D spending. The pattern is clearly different by industry, with a very good fit for manufacturing industries but underestimated spinouts in information industries. In future work, I plan to study this question in a way that allows for variation across industries in spinout and noncompete-relevant parameters.} Further, Appendix tables \ref{table:startupLifeCycle_founder2founders_lemployeecount_founder2}, \ref{table:startupLifeCycle_founder2founders_lrevenue_founder2}, and \ref{table:startupLifeCycle_founder2founders_lpostvalusd_founder2} document that startups with a higher fraction of WSO4 founders tend to have roughly 35\% higher employee count, 45\% higher revenue, and 36\% higher valuation on a per-founder basis. This relationship holds after controlling for industry, state, time, cohort, and / or age factors,\footnote{I only control for two of the three (time, cohort, age) factors, to avoid the well-known multicollinearity problem.} and is statistically significant and robust across specifications. Combining these last two observations with the fact that about 7\% of startup founders are at WSOs suggests that R\&D-induced WSOs account for about 8.5\% of employment, revenue and valuation of startups in the dataset.

\section{Model}\label{sec:model}

The microeconomic relationship documented in the previous section motivates the study of a model of productivity growth in which R\&D by incumbent firms may lead to competition by employee spinouts as well as the possibility to use noncompete agreements to restrict such competition. In this section I build such a model, augmenting a standard quality ladders model of endogenous growth with R\&D-induced WSOs and NCAs. It builds mostly on \cite{grossman_quality_1991}, \cite{acemoglu_innovation_2015}, and \cite{akcigit_growth_2018}.

\subsection{Representative household}

I model a continuous time economy, starting at $t = 0$. The representative household has CRRA preferences over consumption streams of the final good $\{C(t)\}_{t \ge 0}$, given by
\begin{align}
U(\{C(t)\}_{t \ge 0}) &= \int_0^{\infty} e^{-\rho t} \frac{C(t)^{1-\theta} - 1}{1-\theta} dt. \label{preferences}
\end{align}
In each period $t \ge 0$, the household is endowed with $\bar{L}_{RD} \in (0,1)$ units of R\&D labor as well as $1 - \bar{L}_{RD}$ units of production labor which is used in the production of intermediate and final goods. The labor resource constraints are 
\begin{align}
L_{RD} &\le \bar{L}_{RD}, \label{labor_resource_constraint2} \\
L_{P} &\le 1 - \bar{L}_{RD}. \label{labor_resource_constraint} 
\end{align}
The household takes as given profits it receives from the ownership of all firms in the economy. The household also has access to an instantaneous risk-free bond that exists in zero net supply. Note that the above specification implies that the supply of R\&D labor is inelastic. This enables an analytical solution to the model. It also implies that productivity growth is determined by the allocation of R\&D rather than the amount.\textbf{[Justification]} 

\subsection{Final goods producer}

The final good is produced competitively using production labor and a continuum of intermediate goods $j\in [0,1]$ which, at any given time $t$, exist in a finite set of $I_{jt} \ge 1$ qualities $\{q_{jti}\}_{0 \le i \le I_{jt}}$. The production $Y(t)$ of the final good is given by
\begin{align}
Y(t) = F(L_{Ft},\{q_{jti}\},\{k_{jti}\}) &= \frac{L_{Ft}^{\beta}}{1-\beta} \int_0^1 \Big(\sum_{i = 0}^{I_{jt}} q_{jti}^{\frac{\beta}{1-\beta}} k_{jti} \Big)^{1-\beta} dj, \label{final_goods_production}
\end{align}
where $k_{jti} \ge 0$ is the quantity used of intermediate good $j$ of quality $q_{jti}$. This specification implies that different qualities of good $j$ are perfect substitutes. The use of a CES aggregator for intermediate goods with an elasticity that is different from one (i.e., Cobb-Douglas) simplifies the analysis by allowing me to abstract from limit pricing using a simple microfoundation taken from \cite{akcigit_growth_2018}. The exponent $\frac{\beta}{1-\beta}$ on $q_{jti}$ and later specification of the intermediate goods production and innovation technology together yield balanced growth. The model could be specified with $q_{jti}k_{jti}$ as the summand instead, a setting which can be interpreted as cost saving innovations. In that case, however, balanced growth requires a similar exponent in the production and innovation technology for intermediate goods. The present specification follows \cite{akcigit_growth_2018}. Several alternatives are discussed in \cite{acemoglu_introduction_2009}. 

Define $\bar{q}_{jt} = \max_{0 \le i \le I_{jt}} \{q_{jti}\}$ as the \emph{frontier} quality of good $j$. In equilibrium, the final goods production function admits a simpler representation 
\begin{align}
Y(t) = F(L_{Ft},\{\bar{q}_{jt}\},\{\bar{k}_{jt}\}) &= \frac{L_{Ft}^{\beta}}{1-\beta} \int_0^1 \bar{q}_{jt}^{\beta} \bar{k}_{jt}^{1-\beta} dj. \label{eq_final_goods_production}
\end{align}
There is no storage technology for the final good and its price is normalized to 1 in every period. 




\subsection{Intermediate goods production and innovation} \label{subsec:staticproduction}

\subsubsection{Incumbents}

Each quality $q_{jti}$ of each good $j$ is produced by a firm which has a monopoly on production of that quality of good $j$. Intermediate goods $j$ of any quality are produced according to the production function
\begin{align}
k_{jti} = H(\ell_{jti};Q) &= Q \ell_{jti}, \label{intermediate_goods_production}
\end{align}
where $\ell_{jti} \ge 0$ is the labor input and $Q_t = \int_0^1 \bar{q}_{jt} dj$ is the average frontier quality level in the economy. The producer which produces the frontier quality of good $j$ is denoted \emph{incumbent} $j$. There is no storage of intermediate goods. 

As alluded above, the scaling with average quality $Q_t$ is necessary for a balanced growth path (BGP) in this setting. It implies that improvements in the quality of good $j$ increases the productivity of all other goods $j' \ne j$ through a knowledge spillover externality. In this case, balanced growth requires the scaling to be linear in order to offset the fact that the production wage increases linearly with average quality $Q_t$.

\paragraph{Intermediate goods market structure} The following setup is drawn from \cite{akcigit_growth_2018}. Within each good $j$, intermediate goods producers play a two-stage Bertrand competition game at each time $t \ge 0$. In the first stage, participants bear a cost of $\varepsilon > 0$ units of the final good in exchange for a right to compete in the second stage. Then, in the second stage, they engage in Bertrand competition against each other and against producers of goods $j' \ne j$. Optimal pricing under Bertrand competition in the second stage implies that all producers not on the frontier will have zero profits. By backward induction, such producers do not pay the $\varepsilon$ entry cost. The incumbent therefore has a second-stage monopoly over good $j$ which implies she optimally prices her good at the monopolistic competition markup. I study the limit of this model as $\varepsilon \to 0$.\footnote{The purpose of this assumption is to simplify the model, following \cite{akcigit_growth_2018}. Without this setup, there is limit pricing in equilibrium: the markup charged by the technology leader in good $j$ depends on his gap relative to the next laggard.} 

\paragraph{Own-product innovation}\label{subsubsec:OI}

Incumbent $j$ can hire R\&D labor to improve the quality of the good she can produce. I refer to this as \textit{own-product innovation} or OI, following \cite{garcia-macia_how_2019} and \cite{klenow_innovative_2020}. By performing a flow of $z_{jt}$ units of R\&D, she receives a Poisson intensity of $\chi z_{jt}$ of innovating on good $j$, where $\chi > 0$ is an exogenous parameter representing the incumbent's R\&D productivity. This implies an arrival rate of incumbent innovations of 
\begin{align}
	\tau_{jt} &= \chi z_{jt}.
\end{align}
A successful own-product innovation improves the quality of the incumbent's product by an exogenous \emph{step size} $\lambda > 1$. Note that incumbent R\&D exhibits constant returns to scale. This is necessary to have a solution in closed form.\footnote{The model can be extended but a closed form solution will not exist for $V(j,t|q,\mathbbm{1}^{NCA})$.} When directed at a product of relative quality $\frac{\bar{q}_{jt}}{Q_t}$, the flow cost of $z_{jt}$ units of R\&D is $\frac{\bar{q}_{jt}}{Q_t} z_{jt}$ units of R\&D labor. This scaling assumption is natural because higher quality products require more human capital to improve. It implies that there are knowledge spillovers in R\&D as an increase in the quality of good $j$ improves the R\&D technology of all other goods $j' \ne j$. An assumption of this kind is also necessary for the existence of a balanced growth path, as it ensures that R\&D is allocated to all goods $j$ in equilibrium.\footnote{The cost of R\&D could scale up faster than $\frac{\bar{q}_{jt}}{Q_t}$, which would imply that higher quality products grow slower. In the current setup this would violate BGP since there is no stationary distribution of product quality. Adding a fixed cost, however, would induce such a stationary distribution, because it creates a lower exit barrier. For more discussion of this type of question, see \cite{gabaix_power_2009} or \cite{acemoglu_innovation_2015}.}

I assume that only incumbents can perform OI. That is, when an incumbent is overtaken by an entrant or spinout innovation (described in the next section), she loses access to the OI R\&D technology and therefore cannot use it to ``catch up'' to the frontier. One possible interpretation is that learning by doing means the current producer of a product has unique insights into how to improve on it. This assumption is standard in this class of models and significantly increases tractability.\footnote{Without this assumption, the incumbent problem would have an additional state variable (since falling away from the frontier is no longer an absorbing state) and an additional distribution would need to be tracked (the number of incumbents with the technology to produce each infra-frontier good $j$). Typically, papers which focus on catch up growth, such as \cite{aghion_competition_2005}, make simplifying assumptions analogous to mine in order to be able to compute the equilibrium in closed form. For a producer $n$ steps behind the frontier, the assumption of ``no catch up innovation'' is binding if the expected discounted present value of the cost of $n + 1$ innovations using the OI innovation technology is lower than the expected cost of one innovation using the freely available entrant technology (described in Section \ref{subsubsec:entrants}). For certain parameter values, this inequality will hold for small $n$.}  

\subsubsection{Spinouts and noncompete agreeements}\label{subsubsec:generation_of_spinouts}

A flow of R\&D spending on OI by an incumbent induces a positive probability of being overtaken by an employee spinout, unless the incumbent imposes a noncompete on R\&D labor. The decision of whether a noncompete is used is made instant-by-instant and prevents spinouts during the time it is used. To be precise, I assume that if incumbent $j$ conducts $z_{jt}$ units of R\&D effort, she faces a Poisson intensity of spawning a spinout given by 
\begin{align}
	\tau^S_{jt} &= (1-\mathbbm{1}^{NCA}_{jt}) \nu z_{jt}, \label{def:tau_S}
\end{align} 
where $\mathbbm{1}^{NCA}_{jt} = 1$ if and only if an NCA is used in that instant. 

Upon paying an entry cost (discussed in the next paragraph), an employee spinout from incumbent $j$ of quality $\bar{q}_{jt}$ is able to produce good $j$ with quality $\lambda \bar{q}_{jt}$. Therefore, it immediately becomes the new incumbent. The previous incumbent's profits go to zero forever after. The exogenous parameter $\nu \ge 0$ encodes the rate at which R\&D increases the likelihood of replacement by a WSO. As such, it determines the strength of the key mechanism documented in the preceding empirical section. Note that his specification amounts to assuming that the rate of spinout generation of a unit of R\&D labor is inversely proportional to the relative quality of the good to which it is applied. This lines up with the other scaling assumptions I have made regarding the innovation technology for goods of different qualities.\footnote{Because R\&D labor demand is $\frac{\bar{q}_{jt}}{Q_t} z_{jt}$, the factors cancel out and the rate of spinout formation is linear in $z_{jt}$. In this way, this is the specification of spinout generation that is analogous to the specification of the cost of R\&D in that the R\&D labor required for an innovation, inside or out of the firm, scales with the relative quality $\frac{\bar{q}_{jt}}{Q_t}$ of the product in question.}

Employee spinouts must pay an entry cost of $\kappa_{e} V(j,t|\lambda \bar{q}_{jt})$ units of the final good in order to begin producing, where $\kappa_e \in [0,1)$ is exogenous and $V(j,t|\lambda \bar{q}_{jt})$ denotes the equilibrium private value of incumbent $j$ at time $t$ with quality $\lambda \bar{q}_{jt}$. This cost represents non-R\&D expenditures required by creative destruction innovation but not by own-product innovation. Examples of such expenditures could be firm set-up or marketing costs for a new product or brand. The scaling with $V(j,t|\lambda q_{jt})$ assumes that the cost of entry is proportional to the value of the incumbency position obtained through entry, analogous to the scaling assumptions on the R\&D cost. In addition, it allows for an analytical solution to the model.\footnote{BGP could be obtained with a weaker assumption that 
	\begin{align*}
		\textrm{Entry cost}_{j,t,q} &\propto q,
\end{align*}
but this would require a numerical solution to the model.}

Incumbents also have to pay a cost in order to use an NCA. Specifically, when incumbent $j$ imposes an NCA on $z_j$ units of R\&D, she must pay a flow cost $\kappa_{c} \nu V(j,t|\bar{q}_j) z_j$ units of the final good. Given (\ref{def:tau_S}), incumbent $j$ overall pays
\begin{align}
	\textrm{NCA cost}_{jt} &= \tau^S_{jt} \kappa_c V(j,t|q). \label{def:nca_cost}
\end{align}
The NCA enforcement cost reflects the direct cost using an NCA. Even if there are no technical restrictions on what kinds of NCAs are valid, determining competition between businesses may be expensive. Moreover, many jurisdictions do, in fact, impose such restrictions, and resources can be invested to prove that the conditions of those restrictions do not apply. Overall, it seems plausible that investing resources increases the likelihood of a successful enforcement of an NCA.  Note also that a value of $\kappa_c = \infty$ can be interpreted as ban on the use of NCAs and a value $\kappa_c = 0$ can be interpreted as a complete relaxation of barriers to the use of NCAs.

As with the entry cost, the factor $V(j,t|q)$ implies that the cost of enforcing NCAs is proportional to the equilibrium value of the incumbent firm. The economic justification is that valuable incumbency positions require more resources to protect via NCAs. In the context of the model, this specification means that the cost of enforcing an NCA on a given unit of R\&D labor is proportional to both the value of the WSOs that labor will generate in the absence of an NCA, and the expected loss of incumbent value from an absence of NCAs. As with the entry cost, this assumption also improves model tractability by simplifying the analysis of the optimal noncompete policy (see Section \ref{subsubsec:dynamic_equilibrium_original_solution}).\footnote{As before, BGP can be obtained by only assuming that\begin{align*}
		\textrm{NCA cost}_{j,t,q} &\propto \tau^S_{jt} q,
\end{align*}
but this requries a numerical solution.}

Finally, it is important to note that I have assumed that spinout entry does not directly reduce the rate at which incumbent R\&D results in successful OI. Instead, spinouts enter when an additional, independent Poisson process with arrival rate $\nu z_j$ has an arrival. The interpretation is that spinouts in this model do not embody stolen ideas that otherwise would have been implemented by the parent firm. Rather, R\&D labor generates \textit{additional} innovations which the employee can use to offer a higher quality product than the incumbent firm. This is consistent with the finding in \cite{klepper_disagreements_2007} that spinouts often occur when the employee finds the idea more valuable than his employer. It is important to note that this assumption has important consequences for the private and social usefulness of NCAs. In particular, to the extent that spinouts take ideas that otherwise would be implemented inside the firm, they are less valuable both privately and socially. In the context of this model and calibration, the socially beneficial effects of NCA enforcement would be larger.\footnote{The validity of this assumption could be tested empirically with sufficiently exogenous variation in the enforceability of NCAs or in the use of NCAs. One could study, for example, whether R\&D generates fewer patents when R\&D managers and employees are not bound by NCAs. However, as noted in the introduction, interpreting the results of such regressions requires care to control for the effects of mobility across enforcement regions or whether the use of NCAs is correlated with firm characteristics in a way that could lead to a spurious correlation.}

\subsubsection{Entrants} \label{subsubsec:entrants}

For each frontier quality good $j$ there is a unit mass of entrants indexed by $e \in [0,1]$.\footnote{As is standard in this type of model, I assume for simplicity that entrants innovate only on frontier quality goods.} By performing a flow of $\hat{z}_{jet}$ units of R\&D, an entrant receives a Poisson intensity of $\hat{z}_{jet} \hat{\chi} \bar{\hat{z}}_{jt}^{-\psi}$ of innovating on good $j$, where $\bar{\hat{z}}_{jt} = \int_0^1 \hat{z}_{jet} de$ total entrant R\&D dedicated improving good $j$. From now I drop the bar notation when it is clear from context. Because each entrant is infinitesimal, their R\&D efforts have constant returns to scale at the individual level. The equilibrium therefore only pins down $\hat{z}_{jt}$; for simplicity, I assume that $\hat{z}_{jet} = \hat{z}_{jt}$. This assumption makes no difference to the aggregate equilibrium. Aggregating over $e \in [0,1]$, the arrival rate of entrant innovations on good $j$ is 
\begin{align}\label{model:entrantsInnovationTechnology}
	\hat{\tau}_{jt} &= \hat{\chi} \hat{z}_{jt}^{1-\psi}.
\end{align}
As with the other forms of innovation, the cost in terms of R\&D labor is proportional to the relative quality of the good $\frac{\bar{q}_{jt}}{Q_t}$ and a successful innovation yields a monopoly on the production of good $j$ with quality $\lambda \bar{q}_{jt}$. 

The parameter $\psi > 0$ introduces decreasing returns at the level of good $j$. It represents a \textit{congestion} externality in the entrant innovation technology. This is similar to the congestion externality present in search and matching models. Intuitively, due to a lack coordination, entrants attempt similar approaches to solve the same problem. This duplication of effort reduces the overall returns to entrant R\&D when considered at the level of good $j$. 

Finally, as with spinouts, entrants must also pay an entry cost of $\kappa_{e} V(j,t|\lambda \bar{q}_{jt})$ in units of the final good in order to enter once they have successfully innovated on a good of quality $\bar{q}_{jt}$. The interpretation is the same as with spinouts as both types of firms are engaging in creative destruction and not own-product innovation. 


\subsection{Competitive financial intermediary}\label{model:financial_intermediary}

The representative household owns a competitive financial intermediary which in turn owns all firms in the economy and remits their profits back to the representative household. Individual firms in the economy maximize profits subject to the household's risk-free discount rate (there is no collusion due to common ownership). When the representative household receives a shock in good $j$ that allows it to form a spinout, it sells the spinout to the financial intermediary at full private value (i.e. discounting the spinouts profits at the same risk-free discount rate). The financial intermediary takes the entry of the spinout as given, and therefore is willing to pay this value even though the entry of the spinout reduces the value of an existing incumbent.\footnote{The purpose of this construction is to avoid having to assume that the representative household does not take into account the loss of value of the incumbents it owns when spinouts enter.}


\subsection{Equilibrium}\label{subsec:decentralized_equilibrium}

\subsubsection{Definition of equilibrium}

The model involves idiosyncratic risk for each good $j$. In each realization of a given equilibrium, the price at time $t$ of the frontier intermediate good $j$, the R\&D wage paid by incumbent $j$, and the allocation of production and R\&D labor to the frontier intermediate good $j$ all depend on the stochastic realization of the frontier quality stochastic process $\bar{q}_{jt}$. To study this kind of equilibrium, I look for equilibrium objects which deterministic functions of $(j,t|q)$. A particular realization of the equilibrium is obtained by evaluating these functions at the particular realizations of $(j,t|\bar{q}_{jt})$. This approach follows \cite{acemoglu_introduction_2009}. This machinery is here for the sake of clarity and rigor; in a symmetric BGP, all aggregate variables evolve deterministically due to a law of large numbers applied to the continuum of goods $j \in [0,1]$. 

\theoremstyle{definition}
\begin{definition}
	A \emph{equilibrium} of this model consists of household consumption $C(t)$ and bond holdings $A(t)$; final good production $Y(t)$; frontier intermediate goods prices $p(j,t|q)$ and quantities $k(j,t|q)$; production wages $\bar{w}(t)$ and production labor allocation to final goods $L_{F}(t)$ and intermediate goods $\ell_I(j,t|q)$; R\&D wages paid by entrants $\hat{w}_{RD}(t)$, by incumbents using and not using noncompetes $w_{RD}(j,t|q,\mathbbm{1}^{NCA})$; R\&D labor allocations across incumbents $\ell_{RD}(j,t|q)$ and across entrants $\hat{\ell}_{RD}(j,t|q)$; and noncompete contract allocations $\mathbbm{1}^{NCA}(j,t|q)$ such that 
	\begin{enumerate}
		\item The final goods firm maximizes profits.
		\item Each incumbent $j$ optimizes R\&D spending and the use of noncompetes taking as given incumbent R\&D wages (i.e. with and without NCAs), the innovation rate by entrants ($\hat{\tau}(j,t|q) = \hat{\chi} \hat{z}(j,t|q)^{1-\psi}$), and the rate at which R\&D leads to creative destruction by employee spinouts.
		\item Entrants optimize their R\&D spending taking as given R\&D wages and the value of being an incumbent.
		\item The representative household optimizes labor supply, consumption and savings taking as given wages, the value of spinouts they might form through R\&D employment, the interest rate, and dividends from the competitive financial intermediary.
		\item The competitive financial intermediary maximizes the discounted present value of profits remitted to the household, taking as given prices and the innovation rate by entrants $\hat{\tau}(j,t|q)$ and spinouts $\tau^S(j,t|q) = (1 - \mathbbm{1}^{NCA}(j,t|q)) z(j,t|q) \nu$.
		\item Markets clear (final goods, intermediate goods, production labor, R\&D labor for incumbents and entrants at each intermediate good $j$, and risk-free bonds in zero net supply).
	\end{enumerate}
\end{definition}

For the sake of tractability, I will restrict attention to symmetric balanced growth paths equilibria. This requires two more definitions.

\theoremstyle{definition}
\begin{definition}
	A \emph{balanced growth path equilibrium} (BGP) is an equilibrium where there exist $g, C_0, Q_0 > 0$ such that
	\begin{align*}
		C(t) &= C_0 e^{gt}, \\
		Q_t &= Q_0 e^{gt}.
	\end{align*}
\end{definition}

\theoremstyle{definition}
\begin{definition}
	A \emph{symmetric balanced growth path equilibrium} (symmetric BGP) is a BGP where $z_{jt} = z$ and $\hat{z}_{jet} = \hat{z}$ for all $j,e \in [0,1], t \ge 0$. 
\end{definition}

Symmetric BGPs are a natural type of equilibrium given the symmetric setup of the model.\footnote{In principle I allow $\mathbbm{1}^{NCA}_{jt}$ to vary across $j$ and over time $t$; Propositions \ref{proposition:purstrategyeq:positiveOI} and \ref{proposition:purstrategyeq:zeroOI} below show that $\mathbbm{1}^{NCA}_{jt} = x$ in a symmetric BGP except on a knife-edge in the parameter space.}\footnote{One could relax the assumption that $z_{jt} = z$ as long as $\int_0^1 z_{jt} \frac{\bar{q}_{jt}}{Q_t}dj$ is constant on the BGP. This induces a continuum of BGPs which have the same aggregate variables (except higher moments of the quality distribution, which are irrelevant for the equilibrium), since this term appears in the growth accounting equation analogous to (\ref{eq:growth_accounting}) and the R\&D labor market clearing condition (\ref{eq:RD_labor_market_clearing}). As such, this kind of multiplicity does not affect aggregate growth or prices and is a technical artefact of the assumed CRS R\&D technology for incumbents. I therefore assume that $z_{jt} = z$ because it simplifies the algebra.} It is the typical case studied, e.g. in \cite{grossman_quality_1991} and \cite{acemoglu_innovation_2015}.

\subsubsection{Static equilibrium}

The first step is to characterize the static equilibrium given a profile of frontier qualities $\{ \bar{q}_{j}\}$. Note that I have omitted the dependence on $t$ for the sake of clarity. In addition, since only the frontier quality is produced in equilibrium, in this section I will drop the $\bar{q}_{j}$ notation and refer to the frontier good's quality and quantity by $q_j$ and $k_j$, respectively. The following definition abuses notation slightly as it reuses some of the symbols in the definition of equilibrium. 


\theoremstyle{definition}
\begin{definition}
	Given a profile a frontier qualities $\{q_{j}\}$, a \emph{static equilibrium} consists of frontier intermediate goods prices $p_j$ and quantities $k_j$, production wages $\bar{w}$, and a production labor allocation to final goods $L_{F}$ and intermediate goods $\ell_j$ such that
	\begin{enumerate}
		\item The final goods firm maximizes profits taking as given intermediate goods qualities, intermediate goods prices, and the production wage.
		\item Intermediate goods firms maximize profits taking as given other intermediate goods prices and qualities and the production wage.
		\item The production labor and intermediate goods markets clear.
	\end{enumerate}
\end{definition}

Given this definition, one can directly compute the unique static equilibrium of the model given any profile of qualities. 

\begin{proposition}\label{proposition:static_equilibrium_existence_uniqueness}
	Given a profile of frontier qualities $\{q_j\}$, there exists a unique static equilibrium.
\end{proposition}

\begin{proof}
	Final goods producer optimization implies the inverse demand function for intermediate goods
	\begin{align*}
		p_j &= L_F^{\beta} q_j^{\beta} k_j^{-\beta}.
	\end{align*}
	The intermediate goods market structure implies that incumbent $j$ can effectively ignore lower quality producers of good $j$ when making pricing decisions. She therefore sets $k_j$ to solve
	\begin{align}
		\pi(q_j) = \max_{k_j \ge 0} \Big\{ L_F^{\beta} q_j^{\beta} k_j^{1-\beta} - \frac{\overline{w}}{Q} k_j \Big\}, \label{incumbent_profit}
	\end{align}
	where $\overline{w}$ is the equilibrium production goods wage and $Q = \int_0^1 q_j dj$ is average frontier good quality. The solution to this maximization problem yields expressions for intermediate goods pricing, production labor demand, and production, given by 
	\begin{align}
		k_j &= \Big[ \frac{(1-\beta) Q}{\overline{w}} \Big]^{1/\beta}L_F q_j,  \label{optimal_k}\\
		\ell_j &= k_j / Q, \label{optimal_l}\\
		p_j &= \frac{\overline{w}}{(1-\beta) Q}. \label{optimal_p}
	\end{align}
	
	Substituting (\ref{optimal_k}) into the first-order condition for final goods firm optimal production labor demand yields an expression for the equilibrium wage $\overline{w}$, given by 
	\begin{align}
		\overline{w} &= \tilde{\beta} Q, \label{wbar} \\
		\tilde{\beta} &= \beta^{\beta} (1-\beta)^{1-2\beta}. \label{def_cbeta}
	\end{align}
	
	Substituting (\ref{optimal_k}) and (\ref{wbar}) into the expression for profit in (\ref{incumbent_profit}) yields
	\begin{align}
		\pi_j &= \overbrace{(1-\beta) \tilde{\beta} L_F}^{\mathclap{\tilde{\pi}}} q_j. \label{profits_eq}
	\end{align}
	
	Substituting (\ref{optimal_k}) into (\ref{optimal_l}) and integrating $L_I = \int_0^1 \ell_j dj$ yields aggregate labor allocated to intermediate goods production 
	\begin{align}
		L_I &= \Big( \frac{1-\beta}{\tilde{\beta}} \Big)^{1 / \beta} L_F, \label{intermediate_goods_labor}
	\end{align}
	and substituting (\ref{intermediate_goods_labor}) into the labor resource constraint (\ref{labor_resource_constraint}) yields
	\begin{align}
		L_F &= \frac{1 - \bar{L}_{RD}}{1 + \Big(\frac{1-\beta}{\tilde{\beta}}\Big)^{1/\beta}}.
	\end{align}
	
	Output can be computed by substituting (\ref{optimal_k}) into (\ref{eq_final_goods_production}), yielding
	\begin{align}
		Y = \frac{(1-\beta)^{1-2\beta}}{\beta^{1-\beta}} Q L_F. \label{flow_output}
	\end{align}
\end{proof}

\subsubsection{Existence and uniqueness of symmetric BGP}\label{subsubsec:dynamic_equilibrium_original_solution}

The static equilibrium derived in the previous section implies, for each profile of qualities $\{\bar{q}_{jt}\}$, a level of final goods production $Y(t)$, allocations of production labor, intermediate goods prices, and an allocation of intermediate goods. Moreover, it also implies that the flow of profits to incumbent $j$ of quality $q$ is given by $\pi(j,t|q) = \tilde{\pi} q$, which in turn dictates the dividends the household receives from the financial intermediary. All that remains to characterize the set of symmetric BGPs is to find R\&D wages and labor allocations, consumption, and financial asset holdings that are consistent with household, incumbent, and entrant optimization. 

\paragraph{Household optimization}

I begin by deriving the equilibrium conditions that arise from the representative household's optimization problem. As noted in the definition of equilibrium, the household takes as given incumbent R\&D wages and NCA policies $w_{RD}(j,t|q, \mathbbm{1}^{NCA}), \mathbbm{1}^{NCA}(j,t|q)$, entrant R\&D wages $\hat{w}_{RD}(t)$, production wages $\bar{w}(t)$, interest rates $\{r_t\}$, and profits from the financial intermediary $\{\Pi_t\}$. To simplify notation, I use the subscript $jt$ to denote a particular realization of an equilibrium object. Specifically, I use $\mathbbm{1}^{NCA}_{jt}$ to denote $\mathbbm{1}^{NCA}(j,t|\bar{q}_{jt})$, $\ell_{RD,jt}$ to denote $\ell_{RD}(j,t|\bar{q}_{jt}, \mathbbm{1}^{NCA}(j,t|\bar{q}_{jt}))$, $\hat{\ell}_{RD,jt}$ to denote $\hat{\ell}_{RD}(j,t|\bar{q}_{jt})$, and $w_{RD,jt}$ to denote $w_{RD}(j,t|\bar{q}_{jt}, \mathbbm{1}^{NCA}_{jt})$. I also use $t$ subscripts to denote production wages and the interest rate. Given this, the representative household solves

\begin{maxi*}[1]<b>
	{\substack{\{C(t) \}_{t \ge 0} \\ \{A(t) \}_{t \ge 0} \\ \{ L(t)  \}_{t \ge 0} \\ \{\ell_{RD}(j,t|q,\mathbbm{1}^{NCA})\}_{j \in [0,1], t \ge 0} \\ \{\hat{\ell}_{RD}(j,t|q)\}_{j \in [0,1], t \ge 0}  }} {\int_0^{\infty} e^{-\rho t} \frac{C(t)^{1-\theta}-1}{1-\theta} dt}{}{}
	\addConstraint{ \dot{A}(t)}{ = -C(t) + r_tA(t) + \Pi_t + \bar{w}_tL(t)}  {\mkern-148mu\text{(Financial wealth law of motion)}}
	\addConstraint{ }{+ \int_0^1 \hat{w}_{RD}(t) \hat{\ell}_{RD,jt} dj + \int_0^1 w_{RD,jt} \ell_{RD,jt} dj} 
	\addConstraint{ }{+ \int_0^1 \big(1-\mathbbm{1}^{NCA}_{jt}\big) \nu  \big(\frac{\bar{q}_{jt}}{Q_t} \big)^{-1} \ell_{RD,jt} (1-\kappa_e) V(j,t|\lambda \bar{q}_{jt}) dj,}
	\addConstraint{A(0)}{= 0,} {\mkern-148mu\text{(Initial wealth)}} 
	\addConstraint{\lim_{t \to \infty} e^{-\int_0^{t} r_s ds }A(t)}{\ge 0,}  {\mkern-148mu\text{(No Ponzi-game)}} 
	\addConstraint{\int_0^1 (\ell_{RD,jt} + \hat{\ell}_{RD,jt}) dj}{ \le \bar{L}_{RD},} {\mkern-148mu\text{(R\&D labor endowment)}}
	\addConstraint{L(t)}{\le 1 - \bar{L}_{RD}.} {\mkern-148mu\text{(Production labor endowment)}}
\end{maxi*}
where $L(t) = L_I(t) + L_F(t)$ denotes production labor. There are also non-negativity constraints on consumption and labor supply. Thus, the household consumes out of profits remitted by the intermediary $\Pi_t$, returns from holdings of the risk free bond $r_t A(t)$, wages earned from production labor supply $\bar{w}_t L(t)$, wages earned from R\&D labor supply $\int_0^1 \big(w_{RD,jt} \ell_{RD,j}(t) + \hat{w}_{RD.t} \hat{\ell}_{RD,j}(t) \big) dj$, and earnings from sales of spinouts to the financial intermediary $\int_0^1 (1-\mathbbm{1}^{NCA}_{jt}) (\frac{\bar{q}_{jt}}{Q_t})^{-1} \nu (1-\kappa_e) V(j,t|\lambda \bar{q}_{jt}) \big)\ell_{RD,j}(t) dj$. 

Because the household's expected payoff from working at a given incumbent $j$ depends not only on the wage but also on the expected present value of spinouts formed producing good $j$, it is necessary to explicitly model the household's R\&D labor allocation for each individual goods $j$ as a function of its frontier quality realization $\bar{q}_{jt}$. While this is not strictly necessary for R\&D labor supplied to entrants, I present the problem symmetrically. 

The expression for the value of earnings from sales of spinouts follows because a spinout occurs in good $j$ with Poisson intensity $(1-\mathbbm{1}^{NCA}_{jt}) (\frac{\bar{q}_{jt}}{Q_t})^{-1} \nu \ell_{RD,j}(t)$ and is sold to the financial intermediary at the price of $(1-\kappa_e) V(j,t|\lambda \bar{q}_{jt})$.\footnote{To be more precise, the household forms, and sells, a spinout in each good $j$ upon the arrival of an independent Poisson process $N_{jt}$ with time-varying hazard rate $(1-\mathbbm{1}^{NCA}_{jt} ) \nu (\frac{\bar{q}_{jt}}{Q_t})^{-1} \ell_{RD,jt}$. The expected payoff of spinouts from good $j$ is therefore $(1-\mathbbm{1}^{NCA}_{jt} ) \nu (\frac{\bar{q}_{jt}}{Q_t})^{-1} \ell_{RD,jt} (1-\kappa_e) V(j,t|\lambda \bar{q}_{jt}) dj dt$. The budget constraint integrates over $j \in [0,1]$ and, heuristically, divides by $dt$.}



\begin{lemma}\label{lemma:RD_worker_indifference}
	In a symmetric BGP with $z > 0$, R\&D wages satisfy
	\begin{align}
	\hat{w}_{RD}(t) &\le w_{RD}(j,t|\bar{q}_{jt}) + (1-\mathbbm{1}^{NCA}(j,t|\bar{q}_{jt})) (\frac{\bar{q}_{jt}}{Q_t})^{-1} \nu (1-\kappa_e) V(j,t|\lambda \bar{q}_{jt}) \label{eq:RD_worker_indifference}
	\end{align}
	for all $t \ge 0$ and $j \in [0,1]$. If $\hat{z} > 0$, (\ref{eq:RD_worker_indifference}) holds with equality.
\end{lemma}

\begin{proof}
	Optimality dictates that the household supplies R\&D labor only to jobs which provide the highest compensation. Therefore, in order to be consistent with household optimal labor supply when $z > 0$, (\ref{eq:RD_worker_indifference}) must hold for all $t \ge 0$ and $j \in [0,1]$. If, in addition $\hat{z} > 0$, the household must be indifferent between supplying R\&D to incumbents and entrants and (\ref{eq:RD_worker_indifference}) must therefore hold with equality.
\end{proof}

Intuitively, in a symmetric BGP, the expected compensation received from each incumbent in exchange for a given amount of R\&D labor must be constant. Next, optimal savings gives rise to a standard Euler equation which holds in equilibrium for all $t \ge 0$, given by
\begin{align}
\frac{\dot{C}(t)}{C(t)} = \frac{1}{\theta} (r_t - \rho). \label{eq:euler0} 
\end{align}

\begin{lemma}\label{lemma:constant_interest_rate}
	In a symmetric BGP, $r_t = r$.
\end{lemma}

\begin{proof}
	In a symmetric BGP, $\frac{\dot{C}(t)}{C(t)} = g$ is constant. By (\ref{eq:euler0}), $r_t = \theta g + \rho \eqqcolon r$.
\end{proof}

Finally, there is a transversality condition on financial wealth. In this model it holds trivially, as $A(t) = 0$ in equilibrium. Therefore I omit it from the exposition. The typical transversality condition, which would be imposed here if the household traded claims on the competitive financial intermediary, manifests later in a necessary and sufficient condition for finiteness of household utility.

\paragraph{Entrant optimization}

I briefly consider the optimization problem of entrants in order to prove a result which is useful later. Entrant $e$ in good $j$ chooses $\hat{z}_{jet}$ to maximize profits, taking as given prices (i.e. the R\&D wage, the interest rate), good $j$ entrant R\&D $\hat{z}_{jt}$, and the value of being an incumbent in good $j$ at time $t$ with quality $\lambda \bar{q}_{jt}$, denoted $V(j,t|\lambda \bar{q}_{jt})$. The resulting optimization problem is
\begin{align}
	\max_{\hat{z}_{jet} \ge 0} \overbrace{\hat{\chi} \hat{z}_{jet} \hat{z}_{jt}^{-\psi}}^{\mathclap{\text{Innovation rate}}} \underbrace{(1-\kappa_e) V(j,t|\lambda \bar{q}_{jt})}_{\mathclap{\mathbb{E}[\text{Payoff from innovation}]}} -  \overbrace{\underbrace{\hat{z}_{jet} \frac{\bar{q}_{jt}}{Q_t}}_{\mathclap{\text{R\&D labor}}} \hat{w}_{RD}(t)}^{\mathclap{\text{Cost of R\&D}}}. \label{eq:entrant_optimization_problem}
\end{align}

\begin{lemma}
	In any equilibrium, $\hat{z}_{jt} > 0$. In a symmetric BGP, $\hat{z} > 0$.  
\end{lemma} 

\begin{proof}	
	The first order condition corresponding to (\ref{eq:entrant_optimization_problem}) is given by
	\begin{align*}
		\hat{\chi} \hat{z}_{jt}^{-\psi} (1-\kappa_e) V(j,t|\lambda \bar{q}_{jt}) = \frac{\bar{q}_{jt}}{Q_t} \hat{w}_{RD}(t).
	\end{align*}
	This requires $\hat{z}_{jt} > 0$ because $Q_t > 0$ and all other terms are finite. This trivially implies that $\hat{z} > 0$ in a symmetric BGP. 
\end{proof}

\paragraph{Incumbent optimization}

Next, I turn to equilibrium conditions stemming from incumbent and entrant optimization. Incumbent $j$ takes as given flow profits $\pi(j,t|q) = \tilde{\pi} q$ derived from the static equilibrium, the interest rate $r_t$, which she uses to discount future profits, the rate of creative destruction by entrants $\hat{\tau}(j,t|q)$, as well as the R\&D wage she must pay conditional on her NCA policy $w_{RD}(j,t|q , \mathbbm{1}^{NCA})$. The discounting is at the risk-free rate $r_t$ because the financial intermediary diversifies across incumbents and there is no aggregate uncertainty. 

On the equilibrium path, Lemma \ref{lemma:RD_worker_indifference} gives an expression for the value of the incumbent R\&D wage. However, to define the optimal NCA policy of the incumbent, it is necessary to specify the incumbent R\&D wage conditional on any choice of NCA policy. Because the R\&D labor market is competitive, the incumbent takes as given that she can hire as much R\&D labor as necessary provided that she offers total compensation equal to that offered by entrants. If she does not use an NCA, this compensation includes the expected value of spinouts formed by the household. 

\begin{lemma}\label{lemma:RD_worker_indifference1}
	In a symmetric BGP, the incumbent maximizes profits taking as given 
	\begin{align*}
		w_{RD}(j,t| q , \mathbbm{1}^{NCA}) + (1-\mathbbm{1}^{NCA}) (\frac{q}{Q_t})^{-1} \nu (1-\kappa_e) V(j,t|\lambda q) &= \hat{w}_{RD}(t),
	\end{align*}
\end{lemma}

\begin{proof}
	By the same argument as in Lemma \ref{lemma:RD_worker_indifference}, the expression on the LHS is equal to the expected flow compensation to a unit of R\&D labor required to attract any workers. The RHS is the equilibrium price of R\&D labor. Lemma \ref{lemma:RD_worker_indifference} already yields the result for the equilibrium choice of $\mathbbm{1}^{NCA}$. The off-equilibrium wage is simply that wage that would be required to actually hire R\&D labor in the equilibrium that obtains using the off-equilibrium choice of NCA policy. This is also equal to the equilibrium price of R\&D labor $\hat{w}_{RD}(t)$.  
\end{proof}

The above discussion implies that the value of an incumbent $V(j,t|q)$ must satisfy a Hamilton-Jacobi-Bellman equation,
\begin{align}
(r_t + \overbrace{\hat{\tau}}^{\mathclap{\text{Creative destruction}}}) &V(j,t |q) - \dot{V}(j,t|q) = \overbrace{\tilde{\pi} q }^{\mathclap{\text{Flow profits}}}\nonumber \\_{}
&+ \max_{\substack{\mathbbm{1}^{NCA} \in \{0,1\} \\ z \ge 0}} \Bigg\{ z \Big[  \overbrace{\chi \big( V(j,t|\lambda q) - V(j,t|q)\big)}^{\mathclap{\mathbb{E}[\text{Payoff from own-innovation}]}}  \nonumber \\
&- \underbrace{\big(\frac{q}{Q_t}\big)}_{\mathclap{\text{scaling of R\&D cost}}} \Big( \overbrace{w_{RD,jt}(\mathbbm{1}^{NCA})}^{\mathclap{\text{R\&D wage depends on NCA}}} + \underbrace{\big(\frac{q}{Q_t}\big)^{-1}}_{\mathclap{\text{scaling of spinout formation rate}}} \overbrace{(1-\mathbbm{1}^{NCA}) \nu V(j,t|q)}^{\mathclap{\mathbb{E}[\text{Loss from spinout CD}]}} + \underbrace{\big(\frac{q}{Q_t}\big)^{-1}}_{\mathclap{\text{scaling of NCA cost}}}  \overbrace{\mathbbm{1}^{NCA} \kappa_c \nu V(j,t|q) }^{\mathclap{\text{NCA cost}}}\Big)  \Big] \Bigg\}, \label{eq:hjb_incumbent_0}
\end{align}
where $\tilde{\pi}$ is defined in (\ref{profits_eq}). 

The first proposition shows that, in a symmetric BGP, the value function must have a linear form.

\begin{proposition}\label{proposition:hjb_scaling}
	In a symmetric BGP, the value function of the incumbent is given by
	\begin{align*}
		V(j,t|q) &= \tilde{V} q,
	\end{align*}
	for some $\tilde{V} > 0$.
\end{proposition}

\begin{proof}
	
	The interest rate is constant by Lemma \ref{lemma:constant_interest_rate}. Further, in a symmetric BGP, $\hat{z}_{jt} = \hat{z}$. Together these two facts imply that solutions to the inumbcent HJB either satisfy $V(j,t|q) = \tilde{V} q$ or have some kind of pathology, such as an explosive path that violates another equilibrium condition. The technical details of the proof are contained in Appendix \ref{appendix:proofs:proposition:hjb_scaling}.
\end{proof}

The above Proposition implies the following two corollaries.



\begin{proposition_corollary}
	In a symmetric BGP, there exist $\tilde{\hat{w}}_{RD}, \tilde{w}_{RD} > 0$ such that
	\begin{align*}
	\hat{w}_{RD}(t) &= \tilde{\hat{w}}_{RD} Q_t, \\
	w_{RD}(j,t|q,\mathbbm{1}^{NCA}) &= \tilde{w}_{RD}(\mathbbm{1}^{NCA}) Q_t.
	\end{align*}
\end{proposition_corollary}

\begin{proof}
	Dividing the entrant first-order condition by $\bar{q}_{jt}$ yields
	\begin{align}
	\hat{\chi} \hat{z}^{-\psi} \tilde{V} \lambda &= \frac{\hat{w}_{RD}(t)}{Q_t},
	\end{align}
	implying that $\frac{\hat{w}_{RD}(t)}{Q_t}$ must be constant, i.e. $\hat{w}_{RD}(t) = \tilde{\hat{w}}_{RD} Q_t$ for some $\tilde{\hat{w}}_{RD}$. Using this and $V(j,t | q) = \tilde{V}q$ in Lemma \ref{lemma:RD_worker_indifference1} yields $w_{RD}(j,t|q, \mathbbm{1}^{NCA}) = \tilde{w}_{RD}(\mathbbm{1}^{NCA}) Q_t$. 
\end{proof}

From now on, I drop the $\hspace{1mm} \tilde{} \hspace{1mm}$ superscript when referring to $\tilde{w}_{RD}$ and $\tilde{\hat{w}}_{RD}$ and the meaning is clear from context. The next proposition characterizes the equilibrium NCA policy of all incumbents in a symmetric BGP with $z > 0$. 

\begin{proposition}\label{proposition:optimalNCApolicy}
	In a symmetric BGP with $z > 0$, the equilibrium NCA policy of all incumbents is given by 
	\begin{align}
	\mathbbm{1}^{NCA}_{jt} = \mathbbm{1}^{NCA} = \begin{cases}
	1 & \textrm{if } \kappa_{c} < \bar{\kappa}_c, \\
	0 & \textrm{if } \kappa_{c} > \bar{\kappa}_c , \\
	\{0,1\} & \textrm{if } \kappa_c = \bar{\kappa}_c , 
	\end{cases} \label{eq_nca_policy}
	\end{align}
	where $\bar{\kappa}_c  = 1 - (1-\kappa_e)\lambda$.

\end{proposition}


\begin{proof}
	Appendix \ref{appendix:proofs:proposition:optimalNCApolicy}. 
\end{proof}

Proposition \ref{proposition:optimalNCApolicy} implies that, in a symmetric BGP, all incumbents choose the same NCA policy, except on the knife-edge $\kappa_c = \bar{\kappa}_c$ where incumbents are indifferent between $\mathbbm{1}^{NCA} = 0$ and $\mathbbm{1}^{NCA} = 1$. This results from the fact that the only heterogeneity between incumbents in a symmetric BGP is their quality, which does not affect the optimal policy. In Appendix \ref{appendix:model:heterogeneity} I show how the model can be extended to accomodate exogenous heterogeneity in $\kappa_c$ and $\kappa_e$, which would imply a fraction of incumbents would use NCAs. Absent a way to discipline the properties of this distribution, I have left this out of the main model.

The proof of Proposition \ref{proposition:optimalNCApolicy} follows from the following equation, which is obtained by substituting into the incumbent HJB the expressions for the equilibrium incumbent R\&D wage in Lemma \ref{lemma:RD_worker_indifference1}.

\begin{align}
	r \tilde{V} &= -\hat{\tau} \tilde{V} + \tilde{\pi} + \max_{\substack{\mathbbm{1}^{NCA} \in \{0,1\} \\ z \ge 0}} \Big\{z \Big( \overbrace{\chi (\lambda - 1) \tilde{V}}^{\mathclap{\mathbb{E}[\textrm{Benefit from R\&D}]}}- \hat{w}_{RD} -  \underbrace{(1-\mathbbm{1}^{NCA})(1 - (1-\kappa_{e})\lambda)\nu \tilde{V}}_{\mathclap{\text{Net cost of not using NCA}}} - \overbrace{\mathbbm{1}^{NCA} \kappa_{c} \nu \tilde{V}}^{\mathclap{\text{Direct cost of NCA}}}\Big) \Big\}. \label{eq:hjb_incumbent_workerIndiff}
\end{align}

Equation (\ref{eq:hjb_incumbent_workerIndiff}) has an intuitive economic interpretation. The left-hand side is the equilibrium flow payoff on an asset with value $\tilde{V}$. The RHS is the flow payoff of incumbency. The first term, $-\hat{\tau} \tilde{V}$, is the expected capital loss from creative destruction by entrants. The term $\chi(\lambda -1) \tilde{V}$ is the expected benefit per unit of R\&D effort. The factor $\lambda -1$ reflects the fact that the incumbent takes into account the opportunity cost of no longer producing with the obsolete technology. The term $-\hat{w}_{RD}$ reflects the cost of R\&D effort due to the contribution from the prevailing R\&D wage. 

The last two terms determine the optimal NCA policy. The first term $-(1-\mathbbm{1}^{NCA})(1 - (1-\kappa_e) \lambda) \nu \tilde{V}$ is the expected net cost to the incumbent from not using an NCA. It consists of two components. First, the term $-(1-\mathbbm{1}^{NCA})\nu \tilde{V}$ is the expected capital loss from creative destruction by employee spinouts. Second, the term $(1-\mathbbm{1}^{NCA})(1-\kappa_e)\lambda \nu \tilde{V}$ is the equilibrium R\&D wage discount when an NCA is not imposed. This term is positive because, in equilibrium, the R\&D worker accepts a lower wage in return for an expected future payoff from spinout formation. Finally, the last term $-\mathbbm{1}^{NCA} \kappa_c \nu \tilde{V}$ reflects is the direct cost of using an NCA. Incumbents choose the value of $\mathbbm{1}^{NCA}$ which maximizes the term multiplying $z$.

\begin{lemma}\label{lemma:hjb_incumbent_foc}
	In a symmetric BGP with $z > 0$, 
	\begin{align}
		\tilde{V} &= \frac{\hat{w}_{RD}}{\chi(\lambda - 1) - (1-\mathbbm{1}^{NCA}) (1- (1-\kappa_e)\lambda)\nu - \mathbbm{1}^{NCA} \kappa_{c} \nu}. 
	\end{align}
\end{lemma}

\begin{proof}
	The constant returns to scale in the incumbent innovation technology implies the incumbent must be indifferent if $z > 0$, i.e. the term multiplying $z$ in (\ref{eq:hjb_incumbent_workerIndiff}) must equal zero. Solving for $\tilde{V}$ yields Lemma \ref{lemma:hjb_incumbent_foc}.
\end{proof}

\begin{lemma_corollary}
	In a symmetric BGP with $z > 0$
	\begin{align*}
		\chi(\lambda - 1) - (1-\mathbbm{1}^{NCA}) (1- (1-\kappa_e)\lambda)\nu - \mathbbm{1}^{NCA} \kappa_{c} \nu > 0.
	\end{align*}
\end{lemma_corollary}

\begin{proof}
	Because $\hat{w}_{RD} > 0$ in a symmetric BGP, the inequality above is implied by $\tilde{V} > 0$. In turn, $\tilde{V} > 0$ follows from the fact that choosing $z = 0$ yields a value of $\frac{\tilde{\pi}q}{r + \hat{\tau}} > 0$.
\end{proof}

\paragraph{Equilibrium innovation and growth}

Finally, I use the results from the previous three sections to characterize the equilibrium R\&D allocation and resulting growth rate. Using Proposition \ref{proposition:hjb_scaling} and its corollary, the FOC for the entrant's optimization problem (\ref{eq:entrant_optimization_problem}) becomes 
\begin{align}
	\hat{\chi} \hat{z}^{-\psi} (1-\kappa_e) \lambda \tilde{V} = \hat{w}_{RD} \label{eq:free_entry_condition}
\end{align}
Substituting for $\tilde{V}$ by using Lemma \ref{lemma:hjb_incumbent_foc} yields an expression for entrant R\&D, 
\begin{align}
	\hat{z} &= \Big( \frac{\hat{\chi} (1-\kappa_{e}) \lambda}{\chi(\lambda-1) - (1-\mathbbm{1}^{NCA}) (1- (1-\kappa_e)\lambda)\nu - \mathbbm{1}^{NCA} \kappa_{c} \nu} \Big)^{1/\psi}. \label{eq:effort_entrant}
\end{align}
Market clearing for R\&D labor implies
\begin{align}
	z &= \bar{L}_{RD} - \hat{z}. \label{eq:zI_asFuncZe}
\end{align}
If (\ref{eq:zI_asFuncZe}) implies that $z < 0$ then $z = 0$ in equilibrium. I return to this case below. Suppose for now that (\ref{eq:zI_asFuncZe}) is consistent with $z > 0$. Then growth is determined by the growth accounting equation\footnote{A derivation can be found in Appendix \ref{appendix:model:growth_accounting_equation}.}
\begin{align}
g &= (\lambda - 1)(\tau + \tau^S + \hat{\tau}). \label{eq:growth_accounting}
\end{align}
The Euler equation determines the interest rate, 
\begin{align}
	g &= \frac{\dot{C}}{C} = \frac{1}{\theta} (r - \rho), \label{eq:euler} \\
	r &= \theta g + \rho \label{eq:interest_rate}.
\end{align}
Substituting the incumbent's FOC into the incumbent's HJB, and using the expression for the interest rate, yields the incumbent's value $\tilde{V}$,
\begin{align}
	 \tilde{V} &= \frac{\tilde{\pi}}{r + \hat{\tau}}. \label{eq:hjb_incumbent_gordon_formula}
\end{align}
Finally, the entrant optimality condition (\ref{eq:free_entry_condition}) determines the equilibrium value of $\hat{w}_{RD}$. If the above steps imply $z < 0$ or $\hat{z} \le 0$, then $z = 0$ and $\hat{z} = \bar{L}_{RD}$. The rest of the derivation is the same as in the case that $z > 0$.

Finally, in order for equilibrium to be well-defined, household utility must be finite. This is ensured by the following assumption.

\begin{assumption}\label{model:assumption:boundedUtility1}
	$\rho > (1-\theta) g$
\end{assumption} 

In Assumption \ref{model:assumption:boundedUtility1}, $g$ stands for the closed form expression for $g$. Namely, if $z > 0$ then 
\begin{align}
	g &= (\lambda - 1) \Big(  \big( \frac{\hat{\chi} (1-\kappa_e \lambda}{\chi(\lambda-1) - \nu \min \{1 - (1-\kappa_e) \lambda, \kappa_c \}} \big)^{(1-\psi)/\psi)} \\
	&+ \big(\chi + (1- \mathbbm{1}^{NCA}_{\kappa_c < \bar{\kappa}_c(\kappa_e,\lambda)})\nu \big) \big( \bar{L}_{RD} -  \frac{\hat{\chi} (1-\kappa_e \lambda}{\chi(\lambda-1) - \nu \min \{1 - (1-\kappa_e) \lambda, \kappa_c \}} \big)^{1/\psi} \big) \Big).
\end{align}
Otherwise,
\begin{align}
	g &= (\lambda -1) \hat{\chi} \bar{L}_{RD}^{1-\psi}.
\end{align}

\begin{lemma}
	Under Assumption \ref{model:assumption:boundedUtility1}, the household's utility is finite on a symmetric BGP with growth rate $g$.
\end{lemma}

\begin{proof}
	Using $C(t) = C(0)e^{gt}$ on the BGP, the household's utility is
	\begin{align}
		U = \mathcal{K} \int_0^{\infty} e^{-\rho t} e^{(1-\theta)gt} dt + \text{Constant}.
	\end{align}
	
	for some constant $\mathcal{K} > 0$. The integral $\int_0^{\infty} e^{-\rho t} e^{(1-\theta)gt} dt$ converges if and only if $\rho > (1-\theta)g$. 
\end{proof}

Note that $\theta \ge 1$ implies Assumption \ref{model:assumption:boundedUtility1}. This is the empirically relevant case which I consider in the calibration. The next two assumptions are used in the statement of the main proposition.

\begin{assumption}
	$\chi(\lambda - 1) - (1-\mathbbm{1}^{NCA}) (1- (1-\kappa_e)\lambda)\nu - \mathbbm{1}^{NCA} \kappa_{c} \nu > 0$. \label{ineq:vtilde_denom_positive}
\end{assumption}

\begin{assumption}
	$\Big( \frac{\hat{\chi} (1-\kappa_{e}) \lambda}{\chi(\lambda-1) - (1-\mathbbm{1}^{NCA}) (1- (1-\kappa_e)\lambda)\nu - \mathbbm{1}^{NCA} \kappa_{c} \nu} \Big)^{1/\psi} < \bar{L}_{RD}$. \label{ineq:zhat_market_clearing}
\end{assumption}

Assumptions \ref{ineq:vtilde_denom_positive} and \ref{ineq:zhat_market_clearing} ensure that $z > 0$ on the symmetric BGP. The term $\mathbbm{1}^{NCA}$ is as defined in (\ref{eq_nca_policy}). Now I can state the main proposition.

\begin{proposition}\label{proposition:BGPexistence_uniqueness}
	If Assumption \ref{model:assumption:boundedUtility1} holds and $\kappa_c \ne \bar{\kappa}_c$ (as defined in Proposition \ref{proposition:optimalNCApolicy}), there exists a unique symmetric BGP. The unique symmetric BGP has $z > 0$ if and only if
	\begin{enumerate}
		\item The expression for $\tilde{V}$ in Lemma \ref{lemma:hjb_incumbent_foc} is positive.
		\item The expression for $\hat{z}$ in (\ref{eq:effort_entrant}) satisfies $\hat{z} < \bar{L}_{RD}$.  
	\end{enumerate}
\end{proposition}

\begin{proof}
	Here I offer a sketch of the proof; details can be found in Appendix \ref{appendix:proofs:proposition:BGPexistence_uniqueness}. Existence follows because a symmetric BGP can be constructed following the derivation given in the main text. Uniqueness follows because the system can be solved recursively with only one possible solution at each step. The conditions for $z > 0$ are those discussed in the main text. The proof formalizes them, showing that they are together necessary and sufficient for $z > 0$. In particular, this involves showing that the failure of at least one of those conditions implies that $z = 0$ is in fact optimal for the incumbent. 
\end{proof}

The model has multiple equilibria on the knife-edge $\kappa_c = \bar{\kappa}_c$. These multiple equilibria can take various forms, which I discuss in Appendix \ref{appendix:model:multiplicity_of_equilibria}.

\section{Calibration}\label{sec:calibration}

\subsection{Parameters}

The model has 11 parameters, $\{\rho, \theta, \beta, \psi, \lambda, \chi, \hat{\chi}, \kappa_e, \kappa_c, \nu, \bar{L}_{RD}\}$. The two parameters $\theta, \psi$ are set externally. The elasticity parameter $\theta$ is set to 2, corresponding to a standard value used in the the literature. The entrant R\&D curvature parameter $\psi$ is set to a value of 0.5 (the same as in \cite{acemoglu_innovation_2015}), corresponding to a quadratic cost. I consider robustness of the main result to the value of these and all other parameters in Section \ref{appendix:policyanalysis:ncacost}.

The remaining nine parameters are calibrated to match moments. The parameter $\bar{L}_{RD}$ is calibrated to data on the share of employment in R\&D, from the NSF. The parameter $\beta$ is set to match the profit to GDP ratio. The final seven parameters $\{\rho, \lambda, \chi, \hat{\chi}, \kappa_e, \kappa_c, \nu\}$ pertain to preferences ($\rho$) and to the innovation technology and NCA usage (all others), and are chosen to match six moments from the data. One parameter, $\kappa_c$, is partially identified as $\kappa_c > \bar{\kappa}_c$ by the observation that $\tau^S > 0$. The remaining six parameters are exactly identified and the model reproduces the target moments exactly. I discuss the sources of identification in Section \ref{subsec:identification}. 

\subsection{Targets}

The targets of the calibration are displayed in \autoref{calibration_targets}. They consist of the labor productivity growth rate due to creative destruction and own-product innovation, the R\&D to GDP ratio, the real return on the corporate sector, the share of growth coming from older firms improving their own products, the employment share of new firms engaging in creative destruction, and the employment share of R\&D-induced WSOs. The last three targets are not directly observable in the data. I obtain values for them from previous work in the literature and from the empirical section of this paper. 

Matching the productivity growth rate, R\&D to GDP ratio and growth share of OI helps calibrate the efficiency of R\&D in generating aggregate productivity growth through OI and CD. The real return, profit to GDP ratio and employment share of entering firms determines the discount factor and the reward to innovation. Further, the employment share of young firms helps identify the size of each innovation: at a constant contribution to aggregate growth, a lower share of employment at entering firms implies each innovation by an entering firm is smaller. Finally, matching the employment share of entering WSOs allows the model to capture the rate at which R\&D by incumbents increases their likelihood of being replaced by a WSO.

Below, I discuss the calibration targets in more detail.

\paragraph{Growth rate due to CD and OI}

The growth rate is calibrated to the growth in labor productivity due to CD and OI, as calculated in \cite{garcia-macia_how_2019} and \cite{klenow_innovative_2020}. Their accounting procedure uses a structural model of firm dynamics and growth through own-product innovation, creative destruction and new variety creation (but with exogenous productivity growth) to infer the sources of growth from the economy-wide distribution of changes in sales and employment at the firm and establishment levels.\footnote{The key identification assumptions are essentially that creative destruction, own-product innovation and new variety creation all have different implications for the cross-sectional and time-series joint distribution of firm and establishment employment, sales, and exit.} 

\paragraph{Growth share of older firms}

The growth share of older firms improving their own products is calibrated to the growth share of OI as a fraction of OI and CD innovations, as estimated in \cite{garcia-macia_how_2019} and \cite{klenow_innovative_2020}. On average they find that, from 1982 to 2013\footnote{The end points are not exactly these in their data.}, roughly 65\% of CD + OI productivity growth was due to firms at least 6 years old. The computation of the corresponding model moment is described in \ref{appendix:calibration:growthShareOI}.

\paragraph{R\&D spending to GDP}

The data on R\&D spending is from the National Patterns of R\&D resources.\footnote{I take the average of business-funded R\&D business-performed R\&D.} In the data, about half of R\&D spending is wages for employees; in the model, the only input to R\&D is labor. I opt to match the model's aggregate R\&D intensity to that in the data, including costs other than labor. This means that the model captures the full cost of innovation, but likely overestimates the aggregate labor earnings of R\&D employees.\footnote{Since I am not considering the decision of whether to work in R\&D or production, the higher earnings of R\&D workers does not change the predictions of the model.} The computation of the corresponding model moment is described in \ref{appendix:calibration:rd/gdp}.

\paragraph{Interest rate}

The short-term risk-free real interest rate averages about 5\% in the United States from 1986-2006. However, the real interest rate in the model actually corresponds to the discount factor used to price an unlevered firm. Since there is no systemic risk in the model, these are the same; however, since the data exhibits systemic risk, unlevered firms require a higher return than 5\% in the data. 

To compute the return on an unlevered firm, I use a back of the envelope calculation to calculate the hypothetical risk-premium on an unlevered investment. First, the real return on the S\&P 500 in the time period 1986-2006 averaged about 7\%. If these firms were unlevered, the risk premium would be 2\%. However, since the average debt-value ratio of the S\&P 500 in the US is about 40\% during this period. Assuming for simplicity that this corporate debt does not earn a risk premium, the entire risk premium accrues to the equity. If there were no leverage, the risk premium would be smaller in percentage terms, since it is accruing to a larger value investment. To calculate the excess return on total corporate assets, I multiply the excess return on equity by the ratio of market value of equity to total assets, $E / (D + E)$, where $E,D$ are the market value of equity and debt, respectively. Here, $1 - 40\% = 60\%$. I arrive at a calibration value of about 6\% for the real interest rate.\footnote{To the extent that debt does receive a risk-premium that is still less than the risk-premium of equity, the interest rate should be calibrated to a value below 7\%.}

\paragraph{Profits to GDP} 

The data the profit to GDP ratio comes from the BEA. I use the average ratio during the sample period of 1984-2006. 

\paragraph{Employment share of young firms}

As discussed in \cite{klenow_innovative_2020}, adjustment costs mean that, in the data, it can take several years for a new product to displace an old one. However, in the model, entrants that replace incumbents reach their mature size immediately upon entry, simplifying the model significantly. However, this does mean that, if the model matches the amount of employment in firms of age < 1, it might underestimate the true impact on employment reallocation of each new cohort of firms. For this reason, I choose to target the employment share of firms age $\le 6$. 

In addition, I restrict attention only to incumbents and young firms engaging in creative destruction. Because all entry in the model is creative destruction, including employment in entrants developing new varieties would overstate the rate of creative destruction. To do this, I turn to \cite{garcia-macia_how_2019} and \cite{klenow_innovative_2020}, which estimate the portion of growth coming from firms of different ages engaging in creative destruction, new variety creation, and own product improvement. They find that roughly 18\% of employment is in firms age $\le 6$ during the sample period, and that between 30\% and 70\% of the growth from these firms is due to creative destruction, the rest due to new variety creation. However, in their framework, as in mine, a given amount of growth from creative destruction requires significantly more employment, as it destroys a previous incumbent. Using a value of $\lambda = 1.2$, for example, creative destruction requires 6 times more employment than new variety creation to generate the same amount of growth. Using $\lambda = 1.2$ as a benchmark value, I calculate an employment share of young firms of 13.34\% during the sample period. The calibrated value of $\lambda$ will be 1.084, so a choice of $\lambda = 1.2$ in the calculation of this target is conservative. A lower choice here implies a higher share of employment in young firms and, as discussed in the robustness analysis of Appendix \ref{appendix:policyanalysis:ncacost}, this would strengthen my main result. The computation of the corresponding model moment is described in \ref{appendix:calibration:entryRate}.

\paragraph{R\&D-induced spinout share of employment}

Finally, matching the employment share of spinouts is of course crucial so that the analysis accurately captures the burden such firms impose on the incumbents that spawn them. As discussed in the empirical section, R\&D by parent firms can account for 8.5\% of employment at startups in the Venture Source. However, as with the employment share of young firms, I want to restrict attention to firms engaging in creative destruction. The same kind of logic implies an adjustment factor between $1/.93$ and $1/.7$, which implies that between 9\% and 12\% of creative destruction startup employment consists of WSOs. I choose a value of 10\% for the calibration. The computation of the corresponding model moment is described in \ref{appendix:calibration:WSOempShare}.

\begin{table}[]
	\centering
	\captionof{table}{Calibration targets}\label{calibration_targets}
	\begin{tabular}{rcll}
		\toprule \toprule
		& Key parameter(s) & Target & Model \tabularnewline
		\midrule
		Profit (\% GDP) & $\beta$ & 8.5\% & 8.5\% 
		\tabularnewline
		R\&D emp. share & $\bar{L}_{RD}$ & 1\% & 1\% 
		\tabularnewline
		Real return & $\rho$ & 6\% & 6\% 
		\tabularnewline
		Growth rate & $\mathbf{\lambda, \chi, \hat{\chi}}$ & 1.487\% & 1.487\%
		\tabularnewline		
		Age $\ge$ 6 growth share & $\chi, \hat{\chi}$  & 65\% & 65\%
		\tabularnewline
		Age $<$ 6 emp. share  & $\lambda, \hat{\chi}$ & 13.34\% & 13.34\%
		\tabularnewline
		Spinout emp. share &$\nu$  & 10\% & 10\%
		\tabularnewline
		R\&D spending (\% GDP) & $\chi, \hat{\chi}, \kappa_e$  & 1.35\% & 1.35\%
		\tabularnewline
		\bottomrule
	\end{tabular}
\end{table}

\normalsize

\subsection{Identification}\label{subsec:identification}

In this section, I make some comments regarding the identification of the model. The relationship between the model parameters and the model-generated moments is non-linear and many parameters influence multiple moments. \autoref{calibration_identificationSources} shows the elasticity of model moments to calibrated model parameters.\footnote{This is computed as the jacobian matrix of the mapping that takes log parameters to log model moments.} It suggests how identification occurs by showing which moments are sensitive to which parameters. All moments are influenced by all parameters. 

More to the point, what is relevant to identification is the inversion of this mapping; that is, the mapping from target moments to calibrated parameters. It is well-defined, at least locally, because the model is locally able to exactly reproduce the target moments. \autoref{calibration_sensitivity} shows the elasticity of calibrated parameters to moment targets.\footnote{This is calculated by inverting the matrix shown in the previous figure. This is feasible because the model is locally exactly identified by the target moments.} As the figure shows, the mapping is too complex to make simple pronouncements about which moments determine which parameters. In particular, it shows that conclusions based on \autoref{calibration_identificationSources} can be misleading. For example, while an increase in $\lambda$ causes a large increase in Growth Share OI, increasing the Growth Share OI moment target decreases the estimated $\lambda$. Given all the moments that need to be matched, the calibration prefers to match the higher Growth Share OI with a much higher $\chi$ and slightly lower $\lambda$. To complete the picture, \autoref{calibration_identificationSources_full} augments \autoref{calibration_identificationSources} with non-calibrated parameters included as both parameters and target moments. As before, \autoref{calibration_sensitivity_full} inverts this matrix to obtain the elasticity of calibrated parameters to moment targets and non-calibrated parameters. This gives the complete picture of how the model parameters are inferred. 

Based on this, I draw the following \textit{heuristic} conclusions about identification of the model. The discount rate $\rho$ is identified to simultaneously match the interest rate, growth rate and IES. The parameters $\lambda, \chi, \hat{\chi}$ are chosen to simultaneously match the growth rate $g$, the older firm share of growth and the young firm employment share. Intuitively, these three measures determine the size and frequency of each innovation, as well as its attribution to incumbents or entrants. The parameter $\kappa_e$ is distinguished from $\hat{\chi}$ by matching the R\&D to GDP ratio. Entrants have a high payoff to R\&D, which in equilibrium implies a counterfactually large R\&D to GDP ratio unless there is a non-R\&D cost associated with innovation to lower the return to entrant R\&D. Finally, the parameter $\nu$ is identified by matching employment share of WSOs. 

\begin{table}[]
	\centering
	\captionof{table}{Calibrated parameters}\label{calibration_parameters}
	\begin{tabular}{rlll}
		\toprule \toprule
		Parameter & Value & Description & Source \tabularnewline
		\midrule
		$\rho$ & 0.0303 & Discount rate  & Indirect inference \tabularnewline
		$\theta$ & 2 & $\theta^{-1} = $ IES & External calibration 
		\tabularnewline
		$\beta$ & 0.094 & $\beta^{-1} = $ EoS intermediate goods & Exactly identified \tabularnewline 
		$\psi$ & 0.5 & Entrant R\&D elasticity & External calibration \tabularnewline
		$\lambda$ & 1.084 & Quality ladder step size & Indirect inference 
		\tabularnewline
		$\chi$ & 21.217 & Incumbent R\&D productivity & Indirect inference 
		\tabularnewline
		$\hat{\chi}$ & 0.554 & Entrant R\&D productivity & Indirect inference \tabularnewline 
		$\kappa_e$ & 0.859 & Non-R\&D entry cost & Indirect inference \tabularnewline
		$\nu$ & 0.345 & Spinout generation rate  & Indirect inference\tabularnewline
		$\bar{L}_{RD}$ & 0.01 & R\&D labor allocation  & Exactly identified \tabularnewline
		\bottomrule
	\end{tabular}
\end{table}

\begin{figure}[]
	\centering
	\includegraphics[scale = .64]{../code/julia/figures/simpleModel/identificationSources.pdf}
	\caption{Plot showing the elasticity of model moments to model parameters. These elasticities are computed by taking the jacobian matrix of the mapping from log parameters to log model moments. I use the symbol $\chi_e$ for $\hat{\chi}$ for better readability.}
	\label{calibration_identificationSources}
\end{figure}

\section{Welfare effect of NCA enforcement and other policies}\label{sec:policy_analysis}

In this section, I use the calibrated model as a laboratory for studying the effect of policies. I conduct a sequence of second-best analyses assuming the planner can control one or more parameters and/or Pigouvian taxes. All comparisons below are static comparisons between BGPs. Throughout, I assume that taxes (subsidies) are rebated to (financed by) the representative household in a lump-sum payment.  Because there is no labor-leisure choice, this does not create any additional distortions in the economy. 

The main exercise considers whether, and by how much, reducing barriers to the use of NCAs increases or decreases growth and welfare.  Next, I consider a planner who can use R\&D subsidies in order to illustrate how they can have an effect in this model even though the total supply of R\&D labor is fixed. I then consider a planner who can target R\&D subsidies to own-product innovation, a policy which can substitute for the enforcement of NCAs in this setting. Finally, I consider a planner who can simultaneously subsidize own-product innovation and control the enforcement of NCAs.

\subsection{Preliminaries}

\paragraph{Welfare}

Social welfare is simply the representative household's lifetime utility,\footnote{Technically this should be written in terms of $W_t$, the welfare at time $t$. I ignore this detail in the interest of expositional simplicity and without loss of generality since the model grows at constant rate so $W_t = e^{(1-\theta)gt}\tilde{W}$.} 
\begin{align}
	\tilde{W} = \int_0^{\infty} e^{-\rho t} \frac{C(t)^{1-\theta} - 1}{1-\theta} ds \label{eq:agg_welfare0}.
\end{align}
Using $C(t) = \tilde{C} e^{gt}$ on the BGP and integrating yields
\begin{align}
	\tilde{W} &= \frac{\tilde{C}^{1-\theta} }{(1-\theta)(\rho - g(1-\theta))} + \kappa(\rho,\theta), \label{eq:agg_welfare1}
\end{align}
where $\kappa(\rho,\theta)$ is a constant that depends only on preferences. Social welfare can thus be decomposed into a \textit{growth} channel ($g$) and an \textit{level} channel ($\tilde{C}$). Higher values for either imply higher welfare. The term $\tilde{C}$ is referred to as the level of consumption because it is equal to $C(t) / Q(t)$ on the BGP. It can be further decomposed using 
\begin{align}
	\tilde{C} &= \tilde{Y} - \overbrace{(\hat{\tau} + \tau^S) \kappa_e \lambda \tilde{V}}^{\mathclap{\text{Creative destruction cost}}} - \underbrace{\mathbbm{1}^{NCA} z \kappa_c \nu \tilde{V}}_{\mathclap{\text{NCA enforcement cost}}}, \label{eq:agg_consumption_decomposition}
\end{align}
so that steady-state consumption is flow output of the final good minus the final goods cost of creative destruction and of NCA enforcement.

\paragraph{Consumption-equivalent change in welfare} 

To make welfare comparisons quantitatively meaningful, I compare welfare across BGPs in consumption-equivalent (CE) terms. This is defined as follows. For a given equilibrium $\varepsilon$, let $\tilde{C}_{\varepsilon}, g_{\varepsilon}, \tilde{W}_{\varepsilon}$ denote the level of consumption given $Q_t$, the growth rate of consumption, and the time-0 present value of household utility, respectively. Following the derivation of welfare in (\ref{eq:agg_welfare1}), one can write
\begin{align}
	\tilde{W}_{\varepsilon} &= f(\tilde{C}_{\varepsilon}, g_{\varepsilon}) + \kappa(\rho, \theta),
\end{align}
where
\begin{align}
	f(x,y) &= \frac{x^{1-\theta}}{(1-\theta) (\rho - y(1-\theta))},
\end{align}

Now consider an equilibrium $\varepsilon'$ such that $\tilde{W}_{\varepsilon} < \tilde{W}_{\varepsilon'}$. The CE welfare improvement from equilibrium $\varepsilon$ to equilibrium $\varepsilon'$ is the permanent increase in consumption in $\varepsilon$ that would achieve the same utility as $\varepsilon'$. More precisely, define $\hat{C}_{\varepsilon, \varepsilon'}$ such that
\begin{align}
	f(\hat{C}_{\varepsilon, \varepsilon'}, g_{\varepsilon}) = f(\tilde{C}_{\varepsilon'} , g_{\varepsilon'} ).
\end{align}
The CE percentage welfare improvement of $\varepsilon'$ over $\varepsilon$ is then defined by  
\begin{align}
	100 \times \big(\frac{\hat{C}_{\varepsilon,\varepsilon'}}{\tilde{C}_{\varepsilon}} - 1 \big).
\end{align}
For $\theta > 1$ (the case of interest in this paper), a $\frac{\xi}{\theta-1}\%$ CE welfare improvement results from an $\xi\%$ decrease in the absolute value of $\tilde{W} - \kappa(\rho ,\theta)$.\footnote{For $\theta < 1$, a $\frac{\xi}{1-\theta}\%$ CE welfare improvement results from a $\xi\%$ increase in $\tilde{W} - \kappa(\rho, \theta)$. The case $\theta = 1$ corresponds to log utility, in which case
	\begin{align}
		\tilde{W} &= \frac{\rho \log(\tilde{C}) + g}{\rho^2} \label{eq:agg_welfare_log}
	\end{align}
	
	In this case, there is no simple correspondence to obtain CE welfare changes, but they are easy to compute directly. Under the null policy, initial consumption is $\tilde{C}$ and growth is $g$. Under the new policy, initial consumption is $\tilde{C}^+$ and growth is $g^+$. The CE welfare change is $\frac{\tilde{C}^* - \tilde{C}}{\tilde{C}}$, where $\tilde{C}^*$ is defined by 
	\begin{align}
		\frac{\rho\log(\tilde{C}^*) + g}{\rho^2} = \frac{\rho \log(\tilde{C}^+) + g^+}{\rho^2} \label{eq:agg_welfare_log_CE}
\end{align}}

\subsection{NCA cost $\kappa_c$}

\begin{table}
	\centering
	\captionof{table}{Effect of reduction in $\kappa_c$ on growth, level of consumption, and welfare}\label{reducing_kappa_c_table}
	\begin{tabular}{lclll}
		\toprule \toprule
		Measure & Variable & $\kappa_c > \bar{\kappa}_c$ & $\kappa_c = 0$ & Chg. \tabularnewline
		\midrule
		Growth & $g$ & 1.487\% & 1.597\% & 0.11 p.p. \tabularnewline
		Level & $\tilde{C}$  & 0.776 &  0.781 & 0.64\% \tabularnewline 
		\tabularnewline
		Welfare & $\tilde{W}$  &  & & 2.96\% (CE)  \tabularnewline
		\bottomrule
	\end{tabular}
\end{table}

The first policy I study is the effect of reducing $\kappa_c$. I interpret this as loosening restrictions on the use of NCAs. Note that, as discussed in Section \ref{sec:calibration}, the fact that there are spinouts in the data partially identifies $\kappa_c > \bar{\kappa}_c$. Intuitively, according to the model, the spinouts in the data result from the fact that the NCAs that could have been used to prevent them were either too costly to enforce or legally prohibited. 

Therefore, to study the effect of reducing barriers to the enforcement of NCAs, I compare the calibrated BGP ($\kappa_c > \bar{\kappa}_c$) to the BGP that obtains when $\kappa_c = 0$. I interpret the latter to a case when there are no legal barriers to the use of NCAs. One might imagine that there is some fundamental cost $\underline{\kappa}_c$ of using an NCA even in a jurisdiction that imposes no barriers to its use. For simplicity, I assume $\underline{\kappa}_c = 0$. 

\begin{table}
	\centering
	\captionof{table}{Decomposition of effect of reducing $\kappa_c$ on growth and R\&D}\label{reducing_kappa_c_decomposition_table}
	\begin{tabular}{lclll}
		\toprule \toprule
		Measure & Variable & $\kappa_c > \bar{\kappa}_c$ & $\kappa_c = 0$ & Chg. \tabularnewline
		\midrule
		\textbf{Growth} & $g$ & 1.487\% & 1.597\% & $\phantom{-} 0.11$ p.p.\tabularnewline
		\multicolumn{1}{l}{\quad incumbents} & $(\lambda -1) \tau$  & 1.21\% & 1.38\% & $\phantom{-}0.17$ p.p. \tabularnewline
		\multicolumn{1}{l}{\quad entrants} & $(\lambda -1) \hat{\tau}$ & 0.26\% & 0.22\% & $-0.04$ p.p. \tabularnewline
		\multicolumn{1}{l}{\quad spinouts} & $(\lambda -1) \tau^S$ & 0.02\% & 0\% & $-0.02$ p.p. \tabularnewline
		\tabularnewline
		\textbf{R\&D} & & & & 
		\tabularnewline
		\multicolumn{1}{l}{\quad incumbents (\%)}  & $z / \bar{L}_{RD}$ & 67.7\% & 77.4\% & $\phantom{-} 9.7$ p.p. \tabularnewline 
		
		\multicolumn{1}{l}{\quad entrants (\%)}  & $\hat{z} / \bar{L}_{RD}$ & 32.3\% & 22.5\% & $-9.7$ p.p. \tabularnewline
		\bottomrule
	\end{tabular}
\end{table}

The results are displayed in \autoref{reducing_kappa_c_table}. Compared to the BGP with $\kappa_c > \bar{\kappa}_c$, the BGP with $\kappa_c = 0$ has higher growth and initial consumption, implying a 2.96\% increase in welfare in consumption-equivalent terms. \autoref{reducing_kappa_c_decomposition_table} shows a decomposition of the sources of growth in the $\kappa_c > \bar{\kappa}_c$ and $\kappa_c = 0$ equilibria. When $\kappa_c = 0$, a higher share of R\&D labor is allocated to own-product innovation and there is no longer entry by spinouts. Intuitively, having access to free NCAs ($\kappa_c = 0$) makes own-product innovation effectively less costly than when NCAs are prohibitively expensive ($\kappa_c > \bar{\kappa}_c$) and incumbents face the risk of employee spinout formation. Holding prices constant at their $\kappa_c > \bar{\kappa}_c$ levels, a shift to $\kappa_c = 0$ induces incumbents demand more R\&D labor. To clear the market, the equilibrium compensation to R\&D labor must increase. The net effect is a shift in the allocation of R\&D labor to own-product innovation.

Finally, \autoref{calibration_smallSummaryPlot} visually decomposes the effect of reducing $\kappa_c$ on key model objects. The top row shows the allocation of R\&D and the growth rate that results. For high values of $\kappa_c$ the growth rate is low. As soon as $\kappa_c < \bar{\kappa}_c$, NCAs are used, eliminating spinout entry and reducing the BGP growth rate by a discrete amount. Note that there is no discrete change in the allocation of R\&D when $\kappa_c$ crosses this threshold because incumbents are indifferent between using and not using an NCA when $\kappa_c = \bar{\kappa}_c$. As $\kappa_c$ is reduced further, the cost of own-product innovation declines (in partial equilibrium), inducing a shift of R\&D labor as described in the previous paragraph. Next, the bottom row shows the entry and NCA costs paid and the level of consumption $\tilde{C}$ that results. Plots showing other equilibrium objects (the innovation rate, incumbent value, the interest rate, and the R\&D wage) are displayed in \autoref{calibration_summaryPlot}.

\begin{figure}[]
	\centering
	\includegraphics[scale = 0.64]{../code/julia/figures/simpleModel/calibrationFixed_smallSummaryPlot.pdf}
	\caption{Effect of varying $\kappa_c$ on key equilibrium variables. The top-left panel shows R\&D labor allocated to incumbents (own-product innovation) and entrants (creative destruction). The top-right panel shows the aggregate productivity growth rate. The bottom-left panel shows the entry costs paid by entrants and spinouts as well as the direct NCA enforcement cost. Finally, the bottom-right panel shows the level of consumption.}
	\label{calibration_smallSummaryPlot}
\end{figure}

\subsubsection{Equilibrium R\&D misallocation}

To understand how a higher share of R\&D labor allocated to own-product innovation increases growth by enough to outweigh the contribution of spinouts, consider the equilibrium marginal effect on innovation (and hence growth) of the two types of R\&D. For creative destruction, this is given by
\begin{align}
	\frac{d}{d\hat{z}} \hat{\tau} &= \frac{d}{d\hat{z}} \hat{\chi} \hat{z}^{1-\psi} \nonumber \\
								  &= (1-\psi) \hat{\chi} \hat{z}^{-\psi}. \label{eq:cd_marginal_growth}
\end{align}
For own-product innovation, this is given by 
\begin{align}
	\frac{d}{dz} \tau &= \frac{d}{dz} (\chi + (1-\mathbbm{1}^{NCA}) \nu) z \nonumber \\
					  &= \chi + (1-\mathbbm{1}^{NCA}) \nu. \label{eq:oi_marginal_growth}
\end{align}
A reallocation of R\&D labor to own-product innovation increases the BGP growth rate if and only if
\begin{align}
	1 > \frac{\frac{d}{d\hat{z}} \hat{\tau}}{ \frac{d}{dz} \tau }. \label{cs:growth_misallocation_condition0}
\end{align}
Using (\ref{eq:cd_marginal_growth}), (\ref{eq:oi_marginal_growth}), and the expression for equilibrium $\hat{z}$ given in (\ref{eq:effort_entrant}), the inequality (\ref{cs:growth_misallocation_condition0}) becomes
\begin{align}
	1 &> \overbrace{\frac{\lambda-1}{\lambda}}^{\mathclap{\text{Business stealing}}} \times \underbrace{(1-\psi)}_{\mathclap{\text{Congestion}}}   \times \overbrace{\frac{1}{1-\kappa_{e}}}^{\mathclap{\text{Entry cost}}} \times \overbrace{\frac{\chi}{\chi + (1-\mathbbm{1}^{NCA})\nu}}^{\mathclap{\text{Spinout formation}}} \times \nonumber \\ 
	&\overbrace{\frac{\chi(\lambda-1) -(1-\mathbbm{1}^{NCA}) (1-(1-\kappa_e)\lambda)\nu - \mathbbm{1}^{NCA} \kappa_c \nu}{\chi(\lambda-1)}}^{\mathclap{\text{Effective cost of R\&D}}}. \label{cs:growth_misallocation_condition} 
\end{align}

In the calibrated BGP, the RHS of (\ref{cs:growth_misallocation_condition}) has a value of 0.23. This means that a marginal unit of R\&D allocated to own-product innovation is about 4.35 times as productive in generating growth as a marginal unit of R\&D allocated to creative destruction. This explains why reallocation of R\&D to incumbents can increase growth. To gain further insight, I next discuss the economic intuition for each of the terms in (\ref{cs:growth_misallocation_condition}).

First, the term $\frac{\lambda - 1}{\lambda} < 1$ reflects the \textit{business stealing} externality. Innovation by entrants imposes a negative externality on the profits of the incumbent. This means that entrants can earn the required (private) return on R\&D with a lower marginal innovation rate per unit of additional R\&D. In the calibration, $\frac{\lambda-1}{\lambda} \approx 0.08$, so this effect is quite strong and is in fact the main driver of the growth increase from setting $\kappa_c = 0$.\footnote{In models such as \cite{aghion_competition_2005}, this effect is attenuated by the fact that incumbents engage in business-stealing as well by innnovating to increase their technological advantage and hence strengthen their monopoly position. This means R\&D by incumbents would have a negative externality on other incumbents in the same good $j$, making the situation more symmetric between incumbents and entrants.}

Next, the term $1-\psi < 1$ reflects the \textit{congestion} externality. Individual entrants impose a negative externality on the expected returns of other entrants. As with business-stealing, the congestion externality also tends to overallocate R\&D to entrants. In the calibration, $1-\psi = 0.5$.

The term $\frac{1}{1-\kappa_e} \ge 1$ reflects the additional \textit{entry cost} paid by entrants upon innovating. In equilibrium, since all investments yield an expected return equal to the interest rate $r$, a higher non-R\&D marginal cost of creative destruction must be equilibriated by a lower R\&D marginal cost of creative destruction. As a result, the presence of non-R\&D costs of creative destruction increases the equilibrium marginal effect on innovation of R\&D allocated to creative destruction. This reduces the extent of misallocation of R\&D, as it works against the net of the other terms on the RHS of equation (\ref{cs:growth_misallocation_condition}). Of course, the non-R\&D costs of creative destruction are social costs, too, and they are saved when R\&D is reallocated to own-product innovation. In the calibration,  $\frac{1}{1-\kappa_e} \approx 7.09$. 

Next, the term $\frac{\chi(\lambda-1) -(1-\mathbbm{1}^{NCA}) (1-(1-\kappa_e)\lambda)\nu - \mathbbm{1}^{NCA} \kappa_c \nu}{\chi(\lambda-1)}$ reflects the fact that entrants pay a different \textit{effective cost of R\&D} than the incumbent. When $\kappa_c > \bar{\kappa}_c$, so $\mathbbm{1}^{NCA} = 0$, the inequality $1 - (1-\kappa_e) \lambda > 0$ means that the expected harm from future spinouts exceeds the equilibrium wage discount from not using an NCA -- i.e., spinouts are bilaterally suboptimal. In equilibrium, incumbents pay a higher effective cost per unit of R\&D. When $\kappa_c < \bar{\kappa}_c$, so that $\mathbbm{1}^{NCA} = 1$, incumbents pay the entrant R\&D wage plus the NCA enforcement cost. If $\kappa_c > 0$, incumbents have a strictly higher effective cost of R\&D. All else equal, this means that entrants have a higher private return to R\&D than incumbents. As above, in equilibrium these private returns must equate; therefore, a higher cost of R\&D for incumbents requires a higher equilibrium marginal effect on innovation of R\&D labor.\footnote{Alternatively, if $1 - (1-\kappa_e) \lambda < 0$, spinouts are bilaterally efficient and, in equilibrium, incumbents benefit from spinouts \textit{ex ante}. As a consequence, incumbents have a lower effective cost of R\&D than entrants and the effect is reversed.}

Finally, the term $\frac{\chi}{\chi + (1-\mathbbm{1}^{NCA})\nu} \le 1$ reflects the contribution to the productivity of own-product innovation stemming from \textit{entry by WSOs}. If $\mathbbm{1}^{NCA} = 0$ and $\nu > 0$, the term is strictly less than 1. OI by incumbents has a positive growth externality (through spinout entry) hence, in equilibrium it generates a higher marginal effect on growth from OI. If $\mathbbm{1}^{NCA} = 1$ or $\nu = 0$ this term is equal to 1 and has no effect on the inequality, corresponding to $\tau^S = 0$.

\paragraph{Magnitude of overall effect on growth of reducing $\kappa_c$}

The preceding discussion shows why there is any scope for a reduction in $\kappa_c$ to increase growth and, therefore, welfare. Precisely, it shows that the derivative of the growth rate with respect to an R\&D reallocation is positive.  However, the overall magnitude of the increase in growth -- taking into account the negative growth effect of the elimination of spinout entry -- depends on a few more factors. Intuitively, this magnitude is the integral of the marginal effect of reallocation minus the lost growth from the fact that spinouts no longer enter when $\kappa_c = 0$. This integral, in turn, is increasing in (1) how sensitive the R\&D allocation is to a given change in $\kappa_c$; and (2) how large is $\bar{\kappa}_c$ as this determines the scope for reducing $\kappa_c$. The sensitivity of R\&D to a change in price is just the price-elasticity of R\&D demand by both entrants and incumbents: a higher price-elasticity implies a larger R\&D reallocation.\footnote{For this reason, curvature in the incumbent innovation function could dampen the reallocation somewhat.} The latter of course depends on the value $\bar{\kappa}_c = 1 - (1-\kappa_e) \lambda$. Finally, a third factor -- the rate of spinout formation $\nu$ -- determines the magnitude of the overall growth effect of reducing $\kappa_c$. A higher value of $\nu$ increases the amount of R\&D reallocation in response to a change in $\kappa_c$, which follows from the fact that $\kappa_c \nu$ is what appears in the incumbent problem. On the other hand, it also increases the social value of spinouts lost from a reduction in $\kappa_c$: the loss of growth is equal to $(\lambda - 1)\nu z\Big|_{\kappa_c = \bar{\kappa}_c}$. Hence, both the growth-increasing and growth-decreasing mechanisms are amplified by the same factor $\nu$. 

\subsubsection{Robustness} 

Before turning to the remaining policies, I note here that I study the robustness of this result to variation in both target moments and uncalibrated parameters in Appendix \ref{appendix:policyanalysis:ncacost}. First, I show that the welfare result is quantitatively similar when considering the NCA and entry costs as transfers. Next, I show that the result that setting $\kappa_c = 0$ increases growth is robust to a 14\% standard deviation of uncertainty in the calibration targets (assuming zero cross-target correlation). Finally, I show that the growth and welfare results are in fact reversed when the model is forced to match an 8\%, rather than 13.34\%, employment share of young firms. This occurs largely due to a higher calibrated value of $\lambda$ (about 1.23) as well as a higher value of $\kappa_e$, both of which work to bring the RHS of inequality (\ref{cs:growth_misallocation_condition}) much closer to 1.

\subsection{R\&D subsidy}

The first alternative policy I consider is a subsidy to R\&D spending. This is a natural class of policy to study due to its significant magnitude the United States, where all told the Federal government funds about 15\% of business-performed R\&D.\footnote{In the United States, the marginal R\&D subsidy rate is between 15 and 20\%, which is claimed via  deduction on corporate income taxes. The deduction can be carried forward twenty years. These R\&D subsidies are applied only to R\&D spending above a firm-specific base which is defined in reference to past levels of R\&D and firm sales. Taking this into account direct R\&D subsidies offer about a 5\% effective subsidy.  In addition, federal and local governments directly fund about 10\% of private business-performed R\&D.}

First, I briefly discuss the equilibrium that obtains in the presence of an R\&D subsidy; the details of the derivation are in \ref{appendix:model:efficiencyderivations:RDsubsidy}. Suppose that the planner subsidizes R\&D spending at rate $T_{RD}$ (tax if $T_{RD} < 0$). In this case, in a symmetric BGP the incumbent's HJB becomes
\begin{align}
	(r + \hat{\tau}) \tilde{V} = \tilde{\pi} + \max_{\substack{\mathbbm{1}^{NCA} \in \{0,1\} \\ z \ge 0}} \Big\{z &\Big( \overbrace{\chi (\lambda - 1) \tilde{V}}^{\mathclap{\mathbb{E}[\textrm{Benefit from R\&D}]}}- (\underbrace{1-T_{RD}}_{\mathclap{\text{R\&D Subsidy}}}) \big( \overbrace{\hat{w}_{RD} - (1-\mathbbm{1}^{NCA})(1-\kappa_e)\lambda \nu \tilde{V}}^{\mathclap{\text{Incumbent R\&D wage}}}\big) \label{eq:hjb_incumbent_RDsubsidy} \nonumber \\ 
	&-  \underbrace{(1-\mathbbm{1}^{NCA}) \nu \tilde{V}}_{\mathclap{\text{Loss of incumbency from spinout formation}}} - \overbrace{\mathbbm{1}^{NCA} \kappa_{c} \nu \tilde{V}}^{\mathclap{\text{Direct cost of NCA}}}\Big) \Big\}.
\end{align}
Given this HJB, equilibrium usage of NCAs is analogous to the model with no R\&D subsidies, although with a different threshold value, given by
\begin{align}
	\tilde{\bar{\kappa}}_c = 1 - (1 - T_{RD})(1-\kappa_e) \lambda.
\end{align}  
Note that the threshold is decreasing in the subsidy to R\&D spending. This means that a sufficiently large R\&D subsidy induces incumbents to use NCAs in equilibrium provided $\kappa_c < 1$. This is the relevant case, as $\kappa_c > 1$ means that NCAs are prohibitively expensive even when they require no wage premium in equilibrium.\footnote{The case $\kappa_c = 1$ means the incumbent is indifferent about NCAs when the R\&D is fully subsidized.} 

Using $z > 0$ and the incumbent FOC as before, the equilibrium R\&D allocation is
\begin{align}
	\hat{z} &= \Bigg( \frac{\hat{\chi} (1-\kappa_{e}) \lambda}{\chi(\lambda -1) - \nu (\mathbbm{1}^{NCA}\kappa_c + (1-\mathbbm{1}^{NCA})(1 - (1-T_{RD})(1-\kappa_e)\lambda)) } \Bigg)^{1/\psi}, \label{eq:effort_entrant_RDsubsidy} \\
	z &= \bar{L}_{RD} - \hat{z}.
\end{align}
Inspection of the expression for $\hat{z}$ shows that, when $\mathbbm{1}^{NCA} = 0$, $\hat{z}$ increases in $T_{RD}$. Intuitively, the increased R\&D subsidy reduces the wage expenses paid for R\&D by the same factor $1-\frac{1-T_{RD}^1}{1-T_{RD}^0}$ for both incumbents and entrants. However, the incumbent's effective cost of R\&D also includes the increased likelihood of creative destruction by an employee spinout. Therefore, her effective cost of R\&D is reduced by a factor $\tilde{\tau}_{RD} < 1-\frac{1-T_{RD}^1}{1-T_{RD}^0}$. In general equilibrium, an increase in $T_{RD}$ results in R\&D labor being reallocated to entrants. Furthermore, for large enough subsidies, $\kappa_c < \tilde{\bar{\kappa}}_c$ so the incumbent uses NCAs. This occurs because she prefers to pay employees using subsidized wages rather than through future employee spinouts, the cost of which -- i.e., the expected lost profit from being replaced by a spinout -- is not subsidized.

\begin{table}
	\centering
	\caption{Effect of R\&D subsidy on growth, NCAs, level of consumption and welfare}\label{rdsubsidy_table}
	\begin{tabular}{lclllll}
		\toprule \toprule
		&  & \multicolumn{4}{c}{R\&D Subsidy (\%)} \vspace{3pt} \tabularnewline
		Measure &Variable & \multicolumn{1}{c}{0} & \multicolumn{1}{c}{20} & \multicolumn{1}{c}{40} & \multicolumn{1}{c}{60} \tabularnewline
		\midrule
		Growth & $g$ & $\phantom{-}1.49\%$ & $\phantom{-}1.48\%$ & $\phantom{-}1.47\%$ & $\phantom{-}1.45\%$ \tabularnewline
		Level & $\tilde{C}$  & $\phantom{-}0.776$ &  $\phantom{-}0.776$ & $\phantom{-}0.776$ & $\phantom{-}0.776$ \tabularnewline 
		NCAs & $\mathbbm{1}^{NCA}$ & $\phantom{-}0$ & $\phantom{-}0$ & $\phantom{-}0$ & $\phantom{-}1$ \tabularnewline
		\tabularnewline
		$\Delta$ Welfare (CE) & $\tilde{W}$  &  & $- 0.16\%$ & $- 0.32\%$ & $- 0.89\%$ \tabularnewline
		\bottomrule
	\end{tabular}
\end{table}

Turning to the quantitative exercise, recall that the model can only set identify $\kappa > \bar{\kappa}_c$ in order to match the fact that there are spinouts. However, if $\kappa_c$ is much larger than $\bar{\kappa}_c$, then any changes to the incentives for NCAs induced by policy will have no observable effect on the equilibrium. In order to be able to illustrate these effects, in this and all subsequent exercises I assume that $\kappa_c = 1.1 \bar{\kappa}_c$.\footnote{In a model with a non-degenerate distribution of $\kappa_e$ and $\kappa_c$ across goods $j$, one could identify moments of the joint distribution of $\kappa_e$ and $\kappa_c$ by matching the fraction of workers who are bound by NCAs. However, one would still have to take a stand on the shape of this distribution in order to assess policy counterfactuals.}

\autoref{rdsubsidy_table} shows how the equilibrium growth rate $g$, consumption level $\tilde{C}$, usage of NCAs $\mathbbm{1}^{NCA}$, and welfare $\tilde{W}$ are affected by R\&D subsidies. The growth rate declines and the level of consumption is unaffected. When R\&D subsidies are raised to 60\%, incumbents use NCAs. Welfare declines overall at an increasing rate, particularly when NCAs are used and spinouts no longer enter.  \autoref{rdsubsidy_table_decomposition} decomposes the sources of growth and allocation of R\&D. As in the case of varying $\kappa_c$, the growth decrease results from a reallocation of R\&D. Because the reallocation is to entrants (creative destruction) and away from incumbents (own-product innovation) in this case, the gorwth rate decreases due to (\ref{cs:growth_misallocation_condition}). \autoref{calibration_RDSubsidy_smallSummaryPlot} shows how key equilibrium objects respond to R\&D subsidies. In particular, growth declines continuously with the R\&D subsidy until point at which $\kappa_c < \tilde{\bar{\kappa}}_c$, where there is a discrete fall in the growth rate as spinouts no longer enter. Plots describing how the entire equilibrium responds are displayed in \autoref{calibration_RDSubsidy_summaryPlot}. 


\begin{table}
	\centering
	\caption{Decomposition of effect of R\&D subsidies on growth and R\&D}\label{rdsubsidy_table_decomposition}
	\begin{tabular}{lclllll}
		\toprule \toprule
		&  & \multicolumn{4}{c}{R\&D Subsidy (\%)} \vspace{3pt} \tabularnewline
		Measure &Variable & \multicolumn{1}{c}{0} & \multicolumn{1}{c}{20} & \multicolumn{1}{c}{40} & \multicolumn{1}{c}{60} \tabularnewline
		\midrule
		\textbf{Growth} & $g$ & 1.49\% & 1.48\% & 1.47\% & 1.45\% \tabularnewline
		\multicolumn{1}{l}{\quad incumbents} & $(\lambda-1)\tau$ & 1.21\% & 1.20\% & 1.19\% & 1.19\% \tabularnewline
		\multicolumn{1}{l}{\quad entrants} &$(\lambda-1)\hat{\tau}$ & 0.26\% & 0.26\% & 0.26\% & 0.26\% \tabularnewline
		\multicolumn{1}{l}{\quad spinouts} & $(\lambda -1)\tau^S$ & 0.02\% & 0.02\% & 0.02\% & 0\% \tabularnewline
		\tabularnewline
		\textbf{R\&D} & &  &  &  & \tabularnewline 
		\multicolumn{1}{l}{\quad incumbents (\%)} & $z / \bar{L}_{RD}$ & 67.7\% & 67.4\% & 67.1\% & 66.8\% \tabularnewline
		\multicolumn{1}{l}{\quad entrants (\%)} & $\hat{z} / \bar{L}_{RD}$ & 32.3\% & 32.6\% & 32.9\% & 33.1\% \tabularnewline
		\bottomrule
	\end{tabular}
\end{table}


\begin{figure}[]
	\centering
	\includegraphics[scale = 0.64]{../code/julia/figures/simpleModel/calibrationFixed_RDSubsidy_smallSummaryPlot.pdf}
	\caption{Summary of equilibrium for baseline parameter values and various values of $T_{RD}$. This assumes that $\kappa_c = 1.1 \bar{\kappa}_c$. The top-left panel shows R\&D labor allocated to incumbents (own-product innovation) and entrants (creative destruction). The top-right panel shows the aggregate productivity growth rate. The bottom-left panel shows the entry costs paid by entrants and spinouts as well as the direct NCA enforcement cost. Finally, the bottom-right panel shows the level of consumption.}
	\label{calibration_RDSubsidy_smallSummaryPlot}
\end{figure}

Because the supply of R\&D labor is inelastic in this model, it is not surprising that untargeted R\&D subsidies are do not lead to higher growth rates. However, in a model without within-industry spinouts, there would be no effect. This exercise shows that the general equilibrium adjustment of R\&D wages can lead to counterproductive misallocation of R\&D in response to R\&D subsidies. Thus, to the extent that the supply of R\&D labor is not perfectly elastic, this mechanism will play a role. It implies that untargeted R\&D subsidies have a weaker positive effect (less ``bang for buck'') than in a standard model without within-industry spinouts and NCAs, as they tend to allocate R\&D to firms that are not worried about spinouts. To the extent that this is correlated with firms engaging in marginally inefficient forms of R\&D -- for example, if firms without worries about losing their monopoly are engaging in socially inefficient creative destruction -- it weakens the positive effect of R\&D subsidies.

\subsection{Targeted R\&D subsidy}\label{cs:oi_rd_subsidy}

Next, I consider what other policies can be used by the policymaker to complement or substitute the availability of NCAs. The fact that reducing $\kappa_c$ increases growth by reallocating R\&D to own-product innovation suggests that targeted R\&D subsidies can be a substitute to enforcing NCAs. Hence, suppose that the plannner can subsidize own-product R\&D while excluding creative destruction R\&D, denoting this subsidy $T_{RD,I} < 1$ (tax if $T_{RD,I} < 0$). The new equilibrium conditions are derived in Appendix \ref{appendix:model:efficiencyderivations:OIRDtax}. Here I simply discuss the equilibrium use of NCAs and the R\&D allocation. As before, NCAs are used in equilibrium as long as $\kappa_c$ is lower than a threshold, this time given by
\begin{align}
	\hat{\bar{\kappa}}_c = 1 - (1-T_{RD,I})(1-\kappa_e)\lambda, 
\end{align}
Therefore, just as with untargeted R\&D subsidies, large enough targeted R\&D subsidies have the unintended effect of inducing the use of NCAs. The equilibrium R\&D allocation is given by
\begin{align}
	\hat{z} &= \Bigg( \frac{(1-T_{RD,I})\hat{\chi} (1-\kappa_{e}) \lambda}{\chi(\lambda -1) - \nu (\mathbbm{1}^{NCA} \kappa_c + (1-\mathbbm{1}^{NCA})(1 - (1-T_{RD,I})(1-\kappa_e)\lambda)) } \Bigg)^{1/\psi}, \label{eq:effort_entrant_RDsubsidyTargeted_maintext} \\
	z &= \bar{L}_{RD} - \hat{z}.
\end{align}
Note that when $\mathbbm{1}^{NCA} = 0$, the term $T_{RD,I}$ appears in both the numerator and denominator with the same negative sign. Specifically, the numerator has the term $ - \hat{\chi} (1-\kappa_e) \lambda T_{RD,I}$ and the denominator has the term $-(1-\mathbbm{1}^{NCA}) \nu  (1-\kappa_e) \lambda T_{RD,I}$. The term in the numerator results from the fact that incumbents demand more R\&D labor in response to subsidies, which in equilibrium increases the wage and thus reduces incentives for entrants to do R\&D. The term in the denominator is analogous to the term in the case of untargeted R\&D subsidies. It results from the fact that  when $\mathbbm{1}^{NCA} = 0$, incumbents effectively do not receive the full R\&D subsidy because part of their cost of R\&D is the expected loss of business due to employee spinout formation. Hence, the second effect dampens the first effect, but does not overturn it. To see this formally, differentiate (\ref{eq:effort_entrant_RDsubsidyTargeted_maintext}) with respect to $T_{RD,I}$ and simplify, yielding
\begin{align}
	\frac{d\hat{z}^{\psi}}{dT_{RD,I}} &= \frac{-\chi(\lambda -1) + \nu }{g(T_{RD,I},\chi,\lambda,\kappa_c,\kappa_e)^2},
\end{align}
where $g$ is a function of $T_{RD,I}$ and certain model parameters. As $g^2 > 0$, it follows that $\frac{d\hat{z}^{\psi}}{dT_{RD,I}}$ is strictly negative if and only if 
\begin{align}
	\chi(\lambda -1) > \nu. \label{cs:targeted_rd_subsidy_improvement_condition}
\end{align}
The condition (\ref{cs:targeted_rd_subsidy_improvement_condition}) holds in the calibration. In fact, it is also a necessary condition for $\tau^S > 0$ in the symmetric BGP.\footnote{If it did not hold, there is no positive equilibrium wage at which incumbents would demand R\&D labor not bound by an NCA.}

\begin{table}
	\centering
	\caption{Effect of targeted R\&D subsidy on growth, NCAs, level of consumption and welfare}\label{oirdsubsidy_table}
	\begin{tabular}{lclllll}
		\toprule \toprule
		&  & \multicolumn{4}{c}{Targeted R\&D Subsidy (\%)} \vspace{3pt}  \tabularnewline
		Measure &Variable &  \multicolumn{1}{c}{0} & \multicolumn{1}{c}{20} & \multicolumn{1}{c}{40} & \multicolumn{1}{c}{60} \tabularnewline
		\midrule
		Growth & $g$ & 1.49\% & 1.64\% & 1.75\% & 1.79\% \tabularnewline
		NCA usage & $\mathbbm{1}^{NCA}$ & 0 & 0 & 0 & 1 \tabularnewline
		Initial cons. & $\tilde{C}$  & 0.776 &  0.780 & 0.783 & 0.787 \tabularnewline 
		\tabularnewline
		$\Delta$ Welfare (CE) & $\tilde{W}$  &  & 3.71\% & 6.36\% & 7.56\% \tabularnewline
		\bottomrule
	\end{tabular}
\end{table}

\begin{table}
	\centering
	\caption{Decomposition of effect of targeted R\&D subsidy on growth and R\&D}\label{oirdsubsidy_table_decomposition}
	\begin{tabular}{lclllll}
		\toprule \toprule
		&  & \multicolumn{4}{c}{R\&D Subsidy (\%)} \vspace{3pt} \tabularnewline
		Measure & Variable & \multicolumn{1}{c}{0} & \multicolumn{1}{c}{20} & \multicolumn{1}{c}{40} & \multicolumn{1}{c}{60}\tabularnewline
		\midrule
		\textbf{Growth} & $g$ & 1.49\% & 1.64\% & 1.75\% & 1.79\% 
		\tabularnewline
		\multicolumn{1}{l}{\quad incumbents} & $(\lambda -1) \tau$ & 1.21\% & 1.36\% & 1.49\% & 1.58\% \tabularnewline
		\multicolumn{1}{l}{\quad entrants} & $(\lambda - 1) \hat{\tau}$ & 0.26\% & 0.25\% & 0.24\% & 0.21\% \tabularnewline
		\multicolumn{1}{l}{\quad spinouts} & $(\lambda - 1) \tau^S$ & 0.02\% & 0.02\% & 0.02\% & 0.00\% \tabularnewline \tabularnewline
		\textbf{R\&D} & &  &  &  & \tabularnewline
		\multicolumn{1}{l}{\quad incumbents (\%)} & $z / \bar{L}_{RD}$ & 68\% & 75\% & 83\% & 90\% \tabularnewline
		\multicolumn{1}{l}{\quad entrants (\%)} & $\hat{z} / \bar{L}_{RD}$ & 32\% & 25\% & 18\% & 10\% \tabularnewline
		\bottomrule
	\end{tabular}
\end{table}

To quantify the effect of a targeted R\&D subsidy $T_{RD,I}$, I return to the calibrated model. \autoref{oirdsubsidy_table} shows how the equilibrium growth rate $g$, consumption level $\tilde{C}$, and welfare are affected by subsidies to incumbent R\&D. The growth rate and level of consumption increase. When R\&D subsidies are raised to 60\%, incumbents are induced to use NCAs, as discussed above. Welfare increases at a decreasing rate, particularly when NCAs are used and spinouts no longer enter. \autoref{oirdsubsidy_table_decomposition} decomposes the effect on growth and R\&D. As targeted R\&D subsidies are increased, R\&D is reallocated to incumbents. Recalling the derivation above, this occurs because $\chi(\lambda -1 ) - \nu > 0$. Because inequality (\ref{cs:growth_misallocation_condition}) holds, this increases growth. For R\&D subsidies of 60\%, the overall increase in the growth rate is attenuated by the reduction in innovation by employee spinouts. 

\autoref{calibration_RDSubsidyTargeted_smallSummaryPlot} plots the effect of targeted R\&D subsidies on the R\&D allocation, growth rate, entry and NCA costs, and the level of consumption. Note how there is a discrete decline in the growth rate for large targeted R\&D subsidies as they induce NCAs. This reduces the growth-maximizing R\&D subsidy. Consumption increases as the R\&D subsidy is increased due to a continuous decline in entry and NCA costs. When NCAs begin to be used in equilibrium, there is a very small discrete fall in consumption as the rising NCA cost is mostly offset by a declining cost of entry by spinouts. Additional plots showing how other equilibrium objects are affected by targeted R\&D subsidies can be found in \autoref{calibration_RDSubsidyTargeted_summaryPlot}.

Before turning to the optimal policy analysis, note that in this model a tax on R\&D allocated to creative destruction would serve a very similar role to the targeted R\&D subsidy studied in this section. In fact, if the tax is denoted $T_{RD,E}$ (negative if subsidy), the R\&D allocation in that case would be
\begin{align}
	\hat{z} &= \Bigg( \frac{(1+T_{RD,E})^{-1} \hat{\chi} (1-\kappa_{e}) \lambda}{\chi(\lambda -1) - \nu (\mathbbm{1}^{NCA} \kappa_c + (1-\mathbbm{1}^{NCA})(1 - (1-\kappa_e)\lambda)) } \Bigg)^{1/\psi}, \label{eq:effort_entrant_RDtaxtargeted_maintext} \\
	z &= \bar{L}_{RD} - \hat{z}.
\end{align}
In this case, the R\&D allocation is more sensitive in that case because there is no dampening secondary effect and there is no change in the NCA usage threshold $\hat{\bar{\kappa}}_c$. I have chosen to focus on the subsidy to incumbent R\&D for two reasons. First, it may be impossible to implement at tax on R\&D expenditures in reality as a firm can simply claim that it is not conducting R\&D. The regulator would need to check whether firms are engaging in creative destruction R\&D. Contrast that with the case of R\&D subsidies, which require the regulator only to evaluate the validity of firms' claimed R\&D expenses. Second, the equivalence of these two policies would no longer hold in a model with an elastic total supply of R\&D labor. In that case, a tax on R\&D would also work to reduce the total amount of R\&D, whereas a subsidy would increase it. 


\begin{figure}[]
	\centering
	\includegraphics[scale = 0.64]{../code/julia/figures/simpleModel/calibrationFixed_RDSubsidyTargeted_smallSummaryPlot.pdf}
	\caption{Summary of equilibrium for baseline parameter values and various values of $T_{RD,I}$. This assumes that $\kappa_c = 1.1 \bar{\kappa}_c$. The top-left panel shows R\&D labor allocated to incumbents (own-product innovation) and entrants (creative destruction). The top-right panel shows the aggregate productivity growth rate. The bottom-left panel shows the entry costs paid by entrants and spinouts as well as the direct NCA enforcement cost. Finally, the bottom-right panel shows the level of consumption.}
	\label{calibration_RDSubsidyTargeted_smallSummaryPlot}
\end{figure}


\subsection{NCA enforcement and R\&D subsidies}

NCA enforcement improves growth in this model by improving the allocation of R\&D spending. However, it is a costly way of doing so as it sacrifices the innovation from employee spinouts. This innovation is socially valuable because it costs \autoref{allpolicies_table_growth} shows that 

\autoref{allpolicies_table_welfare} shows the effect on welfare of various combinations of $\kappa_c$ and $T_{RD,I}$. For low values of $T_{RD,I}$, it is optimal to reduce $\kappa_c$, as in the baseline case. As $T_{RD,I}$ increases, the benefit of reducing $\kappa_c$ declines. 

\begin{table}
	\centering
	\caption{Growth rate for various values of $\kappa_c$ and $T_{RD,I}$}\label{allpolicies_table_growth}
	\begin{tabular}{lrllll}
		\toprule \toprule
		&  & \multicolumn{4}{c}{$\kappa_c / \bar{\kappa}_c$} \vspace{3pt} \tabularnewline
		& & \multicolumn{1}{c}{0} & \multicolumn{1}{c}{.5} & \multicolumn{1}{c}{1} & \multicolumn{1}{c}{1.5}\tabularnewline
		\midrule
		\multirow{6}{*}{$T_{RD,I}$ (\%)} & 0 & 1.60\% & 1.54\% & 1.49\% & 1.49\% \tabularnewline
		& 20 & 1.70\% & 1.67\% & 1.63\% & 1.64\% \tabularnewline
		& 40 & 1.77\% & 1.75\% & 1.73\% & 1.75\% \tabularnewline
		& 60 & 1.80\% & 1.80\% & 1.79\% & 1.82\% \tabularnewline 
		& 80 & 1.81\% & 1.81\% & 1.81\% & 1.84\% \tabularnewline 
		& 90 & 1.80\% & 1.80\% & 1.80\% & 1.83\% \tabularnewline 
		\bottomrule
	\end{tabular}
\end{table}

\begin{table}
	\centering
	\caption{Welfare improvement from varying $\kappa_c$ and targeted R\&D subsidy}\label{allpolicies_table_welfare}
	\begin{tabular}{lrllll}
		\toprule \toprule
		&  & \multicolumn{4}{c}{$\kappa_c / \bar{\kappa}_c$} \vspace{3pt} \tabularnewline
		 & & \multicolumn{1}{c}{0} & \multicolumn{1}{c}{.5} & \multicolumn{1}{c}{1} & \multicolumn{1}{c}{1.5}\tabularnewline
		\midrule
		\multirow{6}{*}{$T_{RD,I}$ (\%)} & 0 & 2.97\% & 1.51\% & 0\% & 0\% \tabularnewline
		 & 20 & 5.42\% & 4.51\% & 3.33\% & 3.70\% \tabularnewline
		& 40 & 7.17\% & 6.63\% & 5.96\% & 6.35\% \tabularnewline
		& 60 & 8.29\% & 7.99\% & 7.64\% & 8.08\% \tabularnewline 
		& 80 & 8.83\% & 8.65\% & 8.45\% & 8.95\% \tabularnewline 
		& 90 & 8.90\% & 8.73\% & 8.57\% & 9.08\% \tabularnewline 
		\bottomrule
	\end{tabular}
\end{table}




For very large targeted R\&D subsidies greater than 80\%, banning NCAs maximizes welfare. At that point, the misallocation of R\&D has been corrected and therefore NCAs are no longer socially valuable. Banning them allows socially valuable spinouts to enter. The largest improvement in welfare is achieved with a high targeted R\&D subsidy of around 90\% and a high choice of $\kappa_c$ that essentially prohibits the use of NCAs. This point is also socially optimal, but less clearly so because innovation by spinouts implies higher creative destruction costs, reducing $\tilde{C}$. The welfare increase resulting from this policy is 9.08\% in consumption-equivalent terms.

\autoref{calibration_ALL_welfarePlot} shows the result of simultaneously varying $\kappa_c$ and $T_{RD,I}$. 


\begin{figure}[]
	\centering
	\includegraphics[scale = 0.64]{../code/julia/figures/simpleModel/calibrationFixed_ALL_welfarePlot_contour.pdf}
	\caption{Summary of equilibrium for baseline parameter values and various values of $T_{RD,I}$ and $\kappa_c$. Optimum improves welfare by 9.08\%.}
	\label{calibration_ALL_welfarePlot}
\end{figure}


\section{Conclusion}
 
This paper investigates the effect of NCAs on growth and welfare. It makes several contributions. Empirically, I have shown that R\&D tends to predict employee spinout formation at the firm level, after controlling for firm level variables and various fixed effects. The relationship is statistically significant across three specifications and economically large enough to account for about 10\% of the employment in the startups data. These two findings the findings of \cite{babina_entrepreneurial_2019} and \cite{muendler_employee_2012}, respectively, in a dataset of publicly traded parent firms and venture-capital financed startups. This is important as VC-funded startups are particularly important contributors to aggregate productivity growth. 

Theoretically, I extended a textbook model of endogenous growth with creative destruction to allow for within-product spinouts the Venture Source. It also finds that WSOs are significantly larger in employmenand non-competes. The augmented model offers some new theoretical insights about the effects of noncompete enforcement on growth and welfare. When calibrated to the firm-level relationship between R\&D and employee spinouts, findings from the growth accounting literature, and standard aggregate data, it suggests that reducing barriers to the usage of NCAs can significantly increase growth and welfare. It reaches this conclusion even in a model where there is a fixed supply of the input to R\&D due to an inferred equilibrium misallocation of R\&D labor to creative destruction, in turn resulting from strong inferred business stealing and congestion externalities of creative destruction.

The analysis also finds some counterintuitive implications of growth-enhancing policies. R\&D subsidies may have the counterintuitive effect of reducing growth by shifting R\&D labor to to firms without proprietary knowledge. R\&D subsidies targeted at own-product innovation avoid this, but for large enough values of any R\&D subsidy, incumbents may be induced to use noncompetes. Both of these effects are stronger to the extent that the society's resources for R\&D have an inelastic supply, so they may be more applicable in the medium to short term.

There are several possibilities for further work in this direction. On the empirical side, it is important to ascertain the extent to which within-industry spinouts, as identified here, actually compete with their spawning parent firms. For example, using my data, one could test whether funding announcements for spinout firms are associated with negative stock returns for parent firms. Short of this, one could at least endeavor to link the startups in Venture Source with their own reported information on their industry. Separately, one could perform a similar analysis considering employee mobility to competing incumbents. 

Theoretically, while the simplicity of the current model adds transparency to the theoretical analysis and calibration, it could be extended in several ways to add quantitative realism or to incorporate more sources of data. First, the model could be extended to allow for good-specific heterogeneity in the determinants to the use of noncompetes. This would allow the model to match data on the prevalence of the use of non-competes. The process of spinout formation could be modeled more explicitly. For example, the asymmetric information problem that leads to bilaterally inefficient spinouts could be modeled, and / or the the technology for spinout entry could be posited to require R\&D, analogous to the entrant technology. The cost of noncompete enforcement could be modeled instead as the expected duration of a noncompete contract, which could be modeled using perpetual youth. The incumbent R\&D technology could be made to have decreasing returns analogous entrants. The aggregate supply of R\&D labor could be made price-elastic. Finally parent firms can be made to compete neck-and-neck along the lines of \cite{aghion_competition_2005}. While several of these suggestions require adding a dynamic endogenous state variable to the the incumbent firm optimization problem, they may be worthwhile due to the additional realism they offer.
 

\bibliography{references.bib}

\appendix

\counterwithin{proposition}{section}
\counterwithin{proposition_corollary}{section}
\counterwithin{lemma}{section}
\counterwithin{lemma_corollary}{section}

\newpage
\section{Appendix of tables}

\setcounter{table}{0}
\renewcommand{\thetable}{\Alph{section}\arabic{table}}

% latex table generated in R 3.6.3 by xtable 1.8-4 package
% Wed Nov 25 13:59:18 2020
\begin{table}[!htb]
\centering
\begingroup\scriptsize
\begin{tabular}{p{4.5cm}llrllrll}
  \toprule
Industry & Startups & Individuals & State & Startups & Individuals & Year & Startups & Individuals \\ 
  \midrule
Business Applications Software & 1890 & 24572 & California & 8921 & 110208 & 1987 & 353 & 2680 \\ 
  Biotechnology Therapeutics & 1089 & 14381 & Massachussetts & 2279 & 30536 & 1988 & 356 & 2831 \\ 
  Communications Software & 1036 & 13387 & New York & 1644 & 16896 & 1989 & 403 & 3247 \\ 
  Advertising/Marketing & 962 & 12004 & Texas & 1372 & 15324 & 1990 & 396 & 3159 \\ 
  Network/Systems Management Software & 688 & 10866 & Pennsylvania & 927 & 8906 & 1991 & 422 & 3751 \\ 
  Vertical Market Applications Software & 561 & 7065 & Washington & 827 & 9647 & 1992 & 537 & 4854 \\ 
  Online Communities & 550 & 3984 & Colorado & 637 & 7592 & 1993 & 554 & 5294 \\ 
  Application-Specific Integrated Circuits & 464 & 6202 & Virginia & 618 & 7670 & 1994 & 689 & 6735 \\ 
  IT Consulting & 461 & 5482 & Georgia & 579 & 6420 & 1995 & 876 & 8910 \\ 
  Wired Communications Equipment & 454 & 6651 & Illinois & 573 & 5981 & 1996 & 1191 & 13102 \\ 
  Drug Development Technologies & 412 & 5038 & New Jersey & 567 & 6560 & 1997 & 1141 & 13426 \\ 
  Healthcare Administration Software & 403 & 4978 & Florida & 559 & 5277 & 1998 & 1513 & 19471 \\ 
  Therapeutic Devices (Minimally Invasive/Noninvasive) & 374 & 4668 & North Carolina & 466 & 5326 & 1999 & 2557 & 32463 \\ 
  Fiberoptic Equipment & 364 & 4911 & Maryland & 437 & 5230 & 2000 & 2003 & 24251 \\ 
  Database Software & 357 & 4500 & Minnesota & 378 & 4093 & 2001 & 1067 & 13268 \\ 
  Business Support Services: Other & 341 & 3889 & Ohio & 375 & 2911 & 2002 & 986 & 12928 \\ 
  Multimedia/Streaming Software & 337 & 4066 & Connecticut & 367 & 3727 & 2003 & 1037 & 11912 \\ 
  Entertainment & 335 & 2555 & Utah & 255 & 2570 & 2004 & 1110 & 13345 \\ 
  Procurement/Supply Chain & 327 & 4744 & Oregon & 224 & 2415 & 2005 & 1222 & 13305 \\ 
  Wireless Communications Equipment & 322 & 4506 & Tennessee & 222 & 2134 & 2006 & 1380 & 13823 \\ 
  Specialty Retailers & 308 & 2915 & Arizona & 211 & 2280 & 2007 & 1506 & 13045 \\ 
  Pharmaceuticals & 301 & 3629 & Michigan & 209 & 1707 & 2008 & 1416 & 10469 \\ 
  Data Management Services & 292 & 4034 & Wisonsin & 142 & 1110 & 2009 & 1494 & 9460 \\ 
   \bottomrule
\end{tabular}
\endgroup
\caption{Statistics on startups covered by VS sample. Industry information uses VS industrial classification. Startups are counted by founding year, individuals by year they joined the firm.} 
\label{table:VS_summaryTable}
\end{table}


% latex table generated in R 3.6.3 by xtable 1.8-4 package
% Tue Apr 28 17:19:03 2020
\begin{table}[!htb]
\centering
\begingroup\scriptsize
\begin{tabular}{p{1.75cm}p{1.75cm}p{1.75cm}p{1.75cm}p{1.75cm}p{1.75cm}p{1.75cm}p{1.75cm}}
  \toprule
Year & Number of founders & Number of start-ups & Number of founders from public companies & Fraction from public companies (\%) & Fraction from public companies when bio. info available (\%) & Fraction from public companies in same 4-digit NAICS (\%) & Fraction from public companies in same 4-digit NAICS when bio. info available (\%) \\ 
  \midrule
1986 & 269 & 260 & 45 & 16.7 & 22.8 & 5.2 & 7.1 \\ 
  1987 & 356 & 316 & 47 & 13.2 & 16.5 & 5.1 & 6.3 \\ 
  1988 & 372 & 324 & 61 & 16.4 & 21.0 & 5.1 & 6.5 \\ 
  1989 & 479 & 376 & 82 & 17.1 & 21.0 & 5.2 & 6.4 \\ 
  1990 & 484 & 365 & 91 & 18.8 & 22.3 & 7.2 & 8.6 \\ 
  1991 & 565 & 387 & 89 & 15.8 & 18.7 & 6.7 & 8.0 \\ 
  1992 & 711 & 490 & 113 & 15.9 & 19.1 & 4.1 & 4.9 \\ 
  1993 & 827 & 519 & 154 & 18.6 & 21.3 & 8.0 & 9.1 \\ 
  1994 & 1046 & 647 & 189 & 18.1 & 20.8 & 5.9 & 6.8 \\ 
  1995 & 1364 & 825 & 243 & 17.8 & 20.0 & 6.0 & 6.7 \\ 
  1996 & 2000 & 1143 & 356 & 17.8 & 19.5 & 5.9 & 6.5 \\ 
  1997 & 2096 & 1076 & 393 & 18.8 & 20.5 & 7.2 & 7.9 \\ 
  1998 & 3044 & 1443 & 601 & 19.7 & 20.7 & 6.3 & 6.6 \\ 
  1999 & 5376 & 2436 & 1046 & 19.5 & 20.3 & 5.5 & 5.7 \\ 
  2000 & 4343 & 1866 & 892 & 20.5 & 21.7 & 5.9 & 6.3 \\ 
  2001 & 2513 & 985 & 461 & 18.3 & 19.9 & 7.7 & 8.3 \\ 
  2002 & 2499 & 913 & 496 & 19.8 & 21.6 & 8.4 & 9.1 \\ 
  2003 & 2369 & 933 & 455 & 19.2 & 21.5 & 8.4 & 9.5 \\ 
  2004 & 2615 & 1024 & 512 & 19.6 & 22.0 & 8.8 & 9.8 \\ 
  2005 & 2739 & 1105 & 528 & 19.3 & 22.1 & 8.5 & 9.8 \\ 
  2006 & 2997 & 1246 & 595 & 19.9 & 22.9 & 7.7 & 8.9 \\ 
  2007 & 3235 & 1400 & 543 & 16.8 & 20.2 & 6.3 & 7.6 \\ 
  2008 & 3067 & 1337 & 544 & 17.7 & 21.3 & 6.7 & 8.1 \\ 
  2009 & 3129 & 1387 & 516 & 16.5 & 19.8 & 5.4 & 6.5 \\ 
  2010 & 3567 & 1631 & 542 & 15.2 & 18.5 & 5.2 & 6.3 \\ 
  2011 & 4673 & 2040 & 749 & 16.0 & 19.2 & 5.5 & 6.6 \\ 
  2012 & 5093 & 2204 & 826 & 16.2 & 18.8 & 5.1 & 5.9 \\ 
  2013 & 5400 & 2272 & 905 & 16.8 & 18.7 & 4.1 & 4.6 \\ 
  2014 & 5607 & 2282 & 943 & 16.8 & 18.6 & 4.5 & 4.9 \\ 
  2015 & 5312 & 2136 & 882 & 16.6 & 18.0 & 4.6 & 5.0 \\ 
   \bottomrule
\end{tabular}
\endgroup
\caption{Summary of founders. Here, "founder" includes all individuals employed at startups inthe VentureSource database who (1) joined the startup within 3 year(s) of its founding year; and (2) have the title of CEO, CTO, CCEO, PCEO, PRE, PCHM, PCOO, FDR, CHF.} 
\label{table:GStable_founder2}
\end{table}


\begin{table}[!htb]
	\scriptsize
	\centering
	{
\def\sym#1{\ifmmode^{#1}\else\(^{#1}\)\fi}
\begin{tabular}{l*{4}{c}}
\toprule
                    &\multicolumn{1}{c}{(1)}         &\multicolumn{1}{c}{(2)}         &\multicolumn{1}{c}{(3)}         &\multicolumn{1}{c}{(4)}         \\
\midrule
$\frac{\text{WSO4 founders}}{\text{Total founders}}$&        0.19         &        0.32\sym{***}&        0.32\sym{***}&        0.30\sym{***}\\
                    &      (0.22)         &     (0.027)         &     (0.020)         &     (0.013)         \\
\addlinespace
Constant            &        2.44\sym{***}&        2.41\sym{***}&        2.41\sym{***}&        2.41\sym{***}\\
                    &     (0.073)         &   (0.00019)         &   (0.00015)         &   (0.00028)         \\
\addlinespace
State-Year FE       &          No         &         Yes         &         Yes         &          No         \\
\addlinespace
State-Age FE        &          No         &         Yes         &          No         &         Yes         \\
\addlinespace
State-Cohort FE     &          No         &          No         &         Yes         &         Yes         \\
\addlinespace
NAICS4-Year FE      &          No         &         Yes         &         Yes         &          No         \\
\addlinespace
NAICS4-Age FE       &          No         &         Yes         &          No         &         Yes         \\
\addlinespace
NAICS4-Cohort FE    &          No         &          No         &         Yes         &         Yes         \\
\addlinespace
No FE               &         Yes         &          No         &          No         &          No         \\
\midrule
Clustering          &statecode naics1\_4 year         &statecode naics1\_4         &statecode naics1\_4         &statecode naics1\_4         \\
R-squared (adj.)    &     0.00068         &        0.35         &        0.38         &        0.36         \\
R-squared (within, adj)&     0.00068         &      0.0028         &      0.0028         &      0.0024         \\
Observations        &       55767         &       54873         &       54654         &       54779         \\
\bottomrule
\multicolumn{5}{l}{\footnotesize Standard errors in parentheses}\\
\multicolumn{5}{l}{\footnotesize \sym{*} \(p<0.1\), \sym{**} \(p<0.05\), \sym{***} \(p<0.01\)}\\
\end{tabular}
}

	\caption{Dependent variable is the logarithm of the number of employees while the independent variable is the fraction of founders who most recently worked at a public firm in the same industry. The first column shows the raw regression. The following three columns control for state, industry, time, cohort and age factors. Specifically, each regression uses a subset of two of the three (year,age,cohort) effects, in all cases included interacted both with state and industry.} 
	\label{table:startupLifeCycle_founder2founders_lemployeecount_founder2}
\end{table}

\begin{table}[!htb]
	\scriptsize
	\centering
	{
\def\sym#1{\ifmmode^{#1}\else\(^{#1}\)\fi}
\begin{tabular}{l*{4}{c}}
\toprule
                    &\multicolumn{1}{c}{(1)}         &\multicolumn{1}{c}{(2)}         &\multicolumn{1}{c}{(3)}         &\multicolumn{1}{c}{(4)}         \\
\midrule
$\frac{\text{WSO4 founders}}{\text{Total founders}}$&       -0.14         &        0.41\sym{***}&        0.36\sym{***}&        0.34\sym{**} \\
                    &      (0.17)         &      (0.14)         &     (0.086)         &      (0.13)         \\
\addlinespace
Constant            &        1.01\sym{***}&        0.90\sym{***}&        0.94\sym{***}&        0.94\sym{***}\\
                    &      (0.15)         &    (0.0091)         &    (0.0045)         &    (0.0091)         \\
\addlinespace
State-Year FE       &          No         &         Yes         &         Yes         &          No         \\
\addlinespace
State-Age FE        &          No         &         Yes         &          No         &         Yes         \\
\addlinespace
State-Cohort FE     &          No         &          No         &         Yes         &         Yes         \\
\addlinespace
NAICS4-Year FE      &          No         &         Yes         &         Yes         &          No         \\
\addlinespace
NAICS4-Age FE       &          No         &         Yes         &          No         &         Yes         \\
\addlinespace
NAICS4-Cohort FE    &          No         &          No         &         Yes         &         Yes         \\
\addlinespace
No FE               &         Yes         &          No         &          No         &          No         \\
\midrule
Clustering          &statecode naics1\_4 year         &statecode naics1\_4         &statecode naics1\_4         &statecode naics1\_4         \\
R-squared (adj.)    &     0.00017         &        0.27         &        0.36         &        0.36         \\
R-squared (within, adj)&     0.00017         &      0.0025         &      0.0019         &      0.0017         \\
Observations        &       17838         &       16424         &       16429         &       16784         \\
\bottomrule
\multicolumn{5}{l}{\footnotesize Standard errors in parentheses}\\
\multicolumn{5}{l}{\footnotesize \sym{*} \(p<0.1\), \sym{**} \(p<0.05\), \sym{***} \(p<0.01\)}\\
\end{tabular}
}

	\caption{Dependent variable is the logarithm of annual revenue while the independent variable is the fraction of founders who most recently worked at a public firm in the same industry. The first column shows the raw regression. The following three columns control for state, industry, time, cohort and age factors. Specifically, each regression uses a subset of two of the three (year,age,cohort) effects, in all cases included interacted both with state and industry.} 
	\label{table:startupLifeCycle_founder2founders_lrevenue_founder2}
\end{table}

\begin{table}[!htb]
	\scriptsize
	\centering
	{
\def\sym#1{\ifmmode^{#1}\else\(^{#1}\)\fi}
\begin{tabular}{l*{4}{c}}
\toprule
                    &\multicolumn{1}{c}{(1)}         &\multicolumn{1}{c}{(2)}         &\multicolumn{1}{c}{(3)}         &\multicolumn{1}{c}{(4)}         \\
\midrule
$\frac{\text{WSO4 founders}}{\text{Total founders}}$&        0.46\sym{***}&        0.42\sym{***}&        0.36\sym{***}&        0.33\sym{***}\\
                    &     (0.065)         &     (0.058)         &     (0.069)         &     (0.074)         \\
\addlinespace
State-Year FE       &          No         &         Yes         &         Yes         &          No         \\
\addlinespace
State-Age FE        &          No         &         Yes         &          No         &         Yes         \\
\addlinespace
State-Cohort FE     &          No         &          No         &         Yes         &         Yes         \\
\addlinespace
NAICS4-Year FE      &          No         &         Yes         &         Yes         &          No         \\
\addlinespace
NAICS4-Age FE       &          No         &         Yes         &          No         &         Yes         \\
\addlinespace
NAICS4-Cohort FE    &          No         &          No         &         Yes         &         Yes         \\
\midrule
Clustering          &State, Industry         &State, Industry         &State, Industry         &State, Industry         \\
R-squared (adj.)    &      0.0042         &        0.28         &        0.29         &        0.26         \\
R-squared (within, adj)&      0.0042         &      0.0050         &      0.0035         &      0.0028         \\
Observations        &       26504         &       25174         &       25027         &       25337         \\
\bottomrule
\multicolumn{5}{l}{\footnotesize Standard errors in parentheses}\\
\multicolumn{5}{l}{\footnotesize \sym{*} \(p<0.1\), \sym{**} \(p<0.05\), \sym{***} \(p<0.01\)}\\
\end{tabular}
}

	\caption{Dependent variable is the logarithm of post-money valuation while the independent variable is the fraction of founders who most recently worked at a public firm in the same industry. The first column shows the raw regression. The following three columns control for state, industry, time, cohort and age factors. Specifically, each regression uses a subset of two of the three (year,age,cohort) effects, in all cases included interacted both with state and industry.} 
	\label{table:startupLifeCycle_founder2founders_lpostvalusd_founder2}
\end{table}


\begin{table}[]
	\centering
	\captionof{table}{2-digit NAICS codes summary}\label{}
	\begin{tabular}{rl}
		\toprule \toprule
		2-digit Code & Description \tabularnewline
		\midrule
		11  & Agriculture, Forestry, Fishing and Hunting \tabularnewline
		21  & Mining, Quarrying, and Oil and Gas Extraction\tabularnewline
		22  & Utilities\tabularnewline
		23  & Construction \tabularnewline
		31-33 & Manufacturing \tabularnewline
		42 & Wholesale trade \tabularnewline
		44-45 & Retail trade \tabularnewline
		48-49 & Transportation and warehousing \tabularnewline
		51 & Information \tabularnewline
		52 & Finance and insurance \tabularnewline
		53 & Real estate and Rental and Leasing \tabularnewline
		54 & Professional, Scientific, and Technical Services \tabularnewline
		55 & Management of Companies and Enterprises \tabularnewline
		56 & Administrative, Support, Waste Management, Remediation Service \tabularnewline
		61 & Educational services \tabularnewline
		62 & Health Care and Social Assistance \tabularnewline
		71 & Arts, Entertainment, Recreation \tabularnewline
		72 & Accomodation and Food Services \tabularnewline
		81 & Other Services (ecept public Admin.) \tabularnewline
		92 & Public Administration\tabularnewline
		\bottomrule
	\end{tabular}
\end{table}

\begin{table}[]
	\centering
	\captionof{table}{Alternative calibration}\label{calibration_2_parameters}
	\begin{tabular}{rlll}
		\toprule \toprule
		Parameter & Value & Description & Source \tabularnewline
		\midrule
		$\rho$ & 0.0303 & Discount rate  & Indirect inference \tabularnewline
		$\theta$ & 2 & $\theta^{-1} = $ IES & External calibration 
		\tabularnewline
		$\beta$ & 0.094 & $\beta^{-1} = $ EoS intermediate goods & Exactly identified \tabularnewline 
		$\psi$ & 0.5 & Entrant R\&D elasticity & External calibration \tabularnewline
		$\lambda$ & 1.083 & Quality ladder step size & Indirect inference 
		\tabularnewline
		$\chi$ & 21.3 & Incumbent R\&D productivity & Indirect inference 
		\tabularnewline
		$\hat{\chi}$ & 0.554 & Entrant R\&D productivity & Indirect inference \tabularnewline 
		$\kappa_e$ & 0.859 & Non-R\&D entry cost & Indirect inference \tabularnewline
		$\nu$ & 0.346 & Spinout generation rate  & Indirect inference\tabularnewline
		$\bar{L}_{RD}$ & 0.01 & R\&D labor allocation  & Normalization \tabularnewline
		\bottomrule
	\end{tabular}
\end{table}



\newpage
\section{Appendix of figures}

\setcounter{figure}{0}
\renewcommand{\thefigure}{\Alph{section}\arabic{figure}}

\begin{figure}[!htb]
	\centering
	\includegraphics[scale=0.95]{../empirics/figures/plots/industry_column_heatmap_naics2_founder2_ggplot2.png}
	\caption{Heatmap displaying the distribution of parent 2-digit NAICS code (row), conditional on child NAICS code (column). Darker hues indicate a higher density. The industries 49 and 99 are omitted because I find no spinouts in those industries.}
	\label{figure:industry_column_heatmap_naics2_founder2}
\end{figure}

\begin{figure}[]
	\centering
	\includegraphics[scale=0.8]{../empirics/figures/founder2_founders_wso4_f3_Accounting_industryYear.pdf}
	\caption{Economic magnitude of regression estimates in Tables \ref{table:RDandSpinoutFormation_absolute_founder2_l3f3} and \ref{table:RDandSpinoutFormation_at_founder2_l3f3} (i.e. using the average value of the coefficient on R\&D in the fourth column). The figure compares predictions for WSO founder counts at the WSO4-year level to the actual counts, again using only the regression estimates and the level of R\&D in each sector-year. This plot considers only naics industries 3 (manufacturing) and 5 (which include software and most other tech companies), industries responsible for the vast majority (\textbf{[insert figure]}) of private sector R\&D in the United States. The solid (dotted) lines show the fit of an OLS regression of the prediction on the actual (weighted by R\&D spending in each industry-year).}
	\label{figure:founder2_founders_f3_Accounting_industryYear}
\end{figure}

\begin{figure}[]
	\centering
	\includegraphics[scale = 0.56]{../code/julia/figures/simpleModel/calibrationSensitivity.pdf}
	\caption{Plot showing the elasticity of parameters to moments. It is computed by inverting the jacobian matrix of the mapping from log parameters to log model moments (whose entries comprise the previous figure). These elasticities, along with estimates of the noisiness of the moments used in the calibration, can be used to estimate confidence intervals for the parameters in the model, and thereby for the welfare comparison in question.}
	\label{calibration_sensitivity}
\end{figure}

\begin{figure}[]
	\centering
	\includegraphics[scale = 0.56]{../code/julia/figures/simpleModel/calibrationSensitivityFull.pdf}
	\caption{Same as \autoref{calibration_sensitivity}, but now including non-calibrated parameters. As before, this calculated by inverting the jacobian displayed in \autoref{calibration_identificationSources_full}.}
	\label{calibration_sensitivity_full}
\end{figure}

\begin{figure}[]
	\centering
	\includegraphics[scale = 0.56]{../code/julia/figures/simpleModel/identificationSourcesFull.pdf}
	\caption{Plot showing the elasticity of moments to model parameters, including parameters taken from the literature $\theta , \beta, \psi$. These non-calibrated parameters are added in as effective moments to be matched, allowing the sensitivity of calibrated parameters $\rho, \lambda, \chi, \hat{\chi}, \kappa_E, \nu$ to these parameters to be computed by simply inverting this matrix, as before.}
	\label{calibration_identificationSources_full}
\end{figure}

\begin{figure}[]
	\centering
	\includegraphics[scale = 0.7]{../code/julia/figures/simpleModel/welfareComparisonParameterSensitivityFull.pdf}
	\caption{Sensitivity of welfare comparison to moments. This is $\nabla_p W$, where $W(p)$ maps log parameters to the log of the percentage change in BGP consumption which is equivalent to the change in welfare from changing $\kappa_c$ from $\kappa_c > \bar{\kappa_c}$ to $\kappa_c = 0$ (i.e. going from banning to frictionlessly enforcing NCAs).}
	\label{welfareComparisonParameterSensitivityFull}
\end{figure}

\begin{figure}[]
	\centering
	\includegraphics[scale = 0.57]{../code/julia/figures/simpleModel/calibrationFixed_summaryPlot.pdf}
	\caption{Effect of varying $\kappa_c$ on equilibrium variables and welfare.}
	\label{calibration_summaryPlot}
\end{figure}

\begin{figure}[]
	\centering
	\includegraphics[scale = 0.57]{../code/julia/figures/simpleModel/calibrationFixed_RDSubsidy_summaryPlot.pdf}
	\caption{Summary of equilibrium for baseline parameter values and various values of the untargeted R\&D subusidy $T_{RD}$. This assumes that $\kappa_c = 1.1 \bar{\kappa}_c$.}
	\label{calibration_RDSubsidy_summaryPlot}
\end{figure}

\begin{figure}[]
	\centering
	\includegraphics[scale = 0.57]{../code/julia/figures/simpleModel/calibrationFixed_RDSubsidyTargeted_summaryPlot.pdf}
	\caption{Summary of equilibrium for baseline parameter values and various values of $T_{RD,I}$. This assumes that $\kappa_c = 1.1 \bar{\kappa}_c$.}
	\label{calibration_RDSubsidyTargeted_summaryPlot}
\end{figure}


\section{Model}\label{appendix:model}

\subsection{Proofs of propositions}

\subsubsection{Proof of Proposition \ref{proposition:hjb_scaling}}\label{appendix:proofs:proposition:hjb_scaling}

Below, I assume that the value of incumbency in line $j$ is differentiable in $t$ conditional on there being no innovations. This excludes by assumption equilibria in which there are stochastic jumps in the value function (e.g., in response to sunspots). To my knowledge, this restriction is standard in treatments of this type of model, starting from \cite{grossman_quality_1991} and including the models that form the direct foundation of this model such as the one in \cite{acemoglu_innovation_2015} and the related models in \cite{acemoglu_introduction_2009}. Extending the theoretical analysis of this broad class of models is outside the scope of this paper, so I adopt their assumption. 

\begin{proof}
	If $z > 0$ then the FOC holds with equality; otherwise, we can ignore that terms multiplied by $z$ in the incumbent's HJB. Hence, the incumbent HJB implies
	\begin{align}
	(r + \hat{\tau}) V(j,t|q) - \dot{V}(j,t|q) &= \tilde{\pi} q, 
	\label{hjb_incumbent_time_dependent}
	\end{align}
	where I used $\hat{\tau}_{jt} = \hat{\tau}$ on a symmetric BGP. This differential equation has a constant solution equal to 
	\begin{align}
	V(j,t|q) &= \frac{\tilde{\pi} }{r + \hat{\tau}}  q\\
	&= \tilde{V} q.
	\end{align}
	The incumbent HJB also has time-dependent solutions. Intuitively, one has to rule out equilibria  where the wage to consumption ratio shrinks or grows exponentially proportionally with $V(j,t|q)$. To see this rigorously differentiating the time-dependent incumbent HJB (\ref{hjb_incumbent_time_dependent}) yields
	\begin{align}
		\ddot{V}(j,t|q) &= (r + \hat{\tau}) \dot{V}(j,t|q) \label{appendix:eq:hjbGeneralDifferentiated}
	\end{align}
	This means that if $\dot{V} < 0$ initially, then $\ddot{V} < 0$ initially as well, and similarly if $\dot{V} > 0$ initially then $\ddot{V} > 0$ initially as well. Hence, if in equilbrium $V(j,t|q)$ drifts locally, it must drift monotonically. 
	
	Suppose first that $\dot{V}(j,t|q) < 0$. Then (\ref{appendix:eq:hjbGeneral}) implies that $\dot{V}(j,t|q)$ tends to $-\tilde{\pi}q$ as $t \to \infty$. This means that $V$ reaches a negative value in finite time with positive probability. This contradicts optimality since the incumbent is always free to choose $z = 0$ and earn flow profits $\tilde{\pi} q$.
	
	Next, suppose that $\dot{V} > 0$ intially, then 
	\begin{align}
		\frac{\dot{V}}{V} = r + \hat{\tau} - \frac{\tilde{\pi} q}{V}
	\end{align}
	so $V$ grows at a rate that asymptotically approaches the exponential rate $r + \hat{\tau}$. Suppose first that $z > 0$. Differentiating the FOC of the incumbent FOC yields
	\begin{align}
		-\frac{\dot{V}(j,t|q)}{V(j,t|q)} &= g - \frac{\dot{\hat{w}}_{RD,t}}{\hat{w}_{RD,t}}  \label{appendix:eq:freeEntryDifferentiatedImplication}
	\end{align}
	This implies that with positive probability $\hat{w}_{RD}(t) \hat{z} > \tilde{Y}$, which is incompatible with final goods market clearing. 
	
	
\end{proof}



\begin{proof}

	
	Note, however, that it does not directly imply anything about $V(j,t|q)$: constant entrant innovation effort $\hat{z}_{jt} = \hat{z}$ implies that the value of incumbency tomorrow must be proportional to $q$, but it does not directly imply that the value of incumbency today is proportional to $q$. It makes sense intuitively that the same logic should imply that $V(j,t|q) = \bar{\tilde{V}}(t) q$: otherwise, the equilibrium we are on cannot have satisfied the (rational) expectations of previous entrants. Heuristically maybe this is enough, but I haven't found a rigorous proof along these lines. Instead, I show that $V(j,t|q) = \tilde{V}q$ by arguing that other solutions to the incumbent HJB contradict equilbrium requirements. Then the fact that at any time $V(j,t|q)$ has this form implies that innovators (entrants, incumbents, spinouts) expect this value in the future and therefore that they forecast their future payoffs using $V(j,t|\lambda q) = \tilde{V} \lambda q$.
	
	First, differentiating both sides with respect to $t$ and using $\frac{\dot{Q}_t}{Q_t} = g$ on the BGP yields
	\begin{align}
	- \frac{\dot{V}(t|\lambda q)}{V(t|\lambda q)} &= g - \frac{\dot{\hat{w}}_{RD,t}}{\hat{w}_{RD,t}} \label{appendix:eq:freeEntryDifferentiated}
	\end{align}

	Rearranging the original differential equation,
	\begin{align}
	\dot{V}(j,t|q) &= (r + \hat{\tau}) V(j,t|q) - \tilde{\pi} q \label{appendix:eq:hjbGeneral}
	\end{align}	
	
	
	
	
	
	Next, rearrange the expression in the form
	\begin{align}
	\frac{\dot{V}(j,t|q)}{V(j,t|q)} &= r + \hat{\tau} - \frac{\tilde{\pi}q}{V(j,t|q)}
	\end{align}
	
	
	 
	
	First, suppose that $z > 0$. The FOC of the incumbent is
	\begin{align}
	\chi \Big( V(t|\lambda q) - V(j,t|q) \Big) &= \frac{q}{Q_t} \hat{w}_{RD,t} + \nu V(j,t|q) \nonumber \\
	&+ (1 - \mathbbm{1}^{NCA}_{jt}) (1- \kappa_e) \nu V(t|\lambda q) + \mathbbm{1}^{NCA}_{jt}  \kappa_c \nu V(j,t|q) \Big) 
	\end{align}
	
	Divide both sides by $q$, differentiate with respect to $t$, use $\frac{\dot{Q}_t}{Q_t} = g$ and  (\ref{appendix:eq:freeEntryDifferentiated}) to obtain
	\begin{align}
	-\frac{\dot{V}(j,t|q)}{V(j,t|q)} &= g - \frac{\dot{\hat{w}}_{RD,t}}{\hat{w}_{RD,t}}  \label{appendix:eq:freeEntryDifferentiatedImplication}
	\end{align}
	
	Using (\ref{appendix:eq:freeEntryDifferentiatedImplication}), this implies that with positive probability the R\&D wage grows to the point where $\hat{w}_{RD} \hat{z} > \tilde{Y}$, which contradicts equilbirium.
	
	Next, suppose that $z = 0$. The entrant optimality condition requires that $\frac{V(j,t|\lambda q)}{\hat{w}_{RD}(t)}$ be constant. This implies $V(j,t|) $ \textbf{[finish]}
	
	
	
	
\end{proof}

\subsubsection{Proof of Proposition \ref{proposition:optimalNCApolicy}}\label{appendix:proofs:proposition:optimalNCApolicy}

\begin{proof}
	Using the representation $V(j,t|q) = \tilde{V}q$ derived in Proposition \ref{proposition:hjb_scaling} in the incumbent HJB (\ref{eq:hjb_incumbent_0}) and dividing both sides by $q$ yields
	\begin{align*}
		(r + \hat{\tau}) &\tilde{V} = \tilde{\pi} + \max_{\substack{\mathbbm{1}^{NCA} \in \{0,1\} \\ z \ge 0}} \Bigg\{ z \Big( \chi (\lambda -1) \tilde{V}- w_{RD}(\mathbbm{1}^{NCA}) - (1-\mathbbm{1}^{NCA}) \nu \tilde{V} - \mathbbm{1}^{NCA} \kappa_c \nu \tilde{V} \Big)\Bigg\}.
	\end{align*}
	
	In any symmetric BGP with $z > 0$, Lemma \ref{lemma:RD_worker_indifference} determines the relationship between $w_{RD,jt}(\mathbbm{1}^{NCA}_{jt})$ and $\hat{w}_{RD,t}$.  Substituting in $w_{RD}(\mathbbm{1}^{NCA})$ using the indifference condition (\ref{eq:RD_worker_indifference}) derived in Lemma \ref{lemma:RD_worker_indifference} yields
	\begin{align*}
		(r + \hat{\tau}) \tilde{V} &= \tilde{\pi} + \max_{\substack{\mathbbm{1}^{NCA} \in \{0,1\} \\ z \ge 0}} \Big\{z \Big( \overbrace{\chi (\lambda - 1) \tilde{V}}^{\mathclap{\mathbb{E}[\textrm{Benefit from R\&D}]}}- \hat{w}_{RD} -  \underbrace{(1-\mathbbm{1}^{NCA})(1 - (1-\kappa_{e})\lambda)\nu \tilde{V}}_{\mathclap{\text{Net cost from spinout formation}}} - \overbrace{\mathbbm{1}^{NCA} \kappa_{c} \nu \tilde{V}}^{\mathclap{\text{Direct cost of NCA}}}\Big) \Big\}.
	\end{align*}
	
	Let $\bar{\kappa}_c (\kappa_e, \lambda) = 1 - (1-\kappa_e)\lambda$. If $z > 0$, the incumbent maximizes her flow payoff by choosing $\mathbbm{1}^{NCA} \in \{0,1\}$ which maximizes the term multiplying $z$. Therefore, $\mathbbm{1}^{NCA} = 1$ is strictly preferred iff $1 - (1-\kappa_e) \lambda > \kappa_c$, which is equation (\ref{eq_nca_policy}).
\end{proof}

\subsubsection{Proof of Proposition \ref{proposition:BGPexistence_uniqueness}}\label{appendix:proofs:proposition:BGPexistence_uniqueness}

\begin{proof}
	Suppose first that both conditions (1) and (2) hold. To see existence guess and verify that $V(j,t|q) = \tilde{V} q$ and solve for equilibrium variables as described in the last section. Assumption \ref{model:assumption:boundedUtility1} guarantees that household utility is finite and the obtained allocation is in fact an equilibrium. To show uniqueness, first notice that in any symmetric BGP, one has $V(j,t|\bar{q}_{jt}) = \tilde{V}\bar{q}_{jt}$ by Proposition \ref{proposition:hjb_scaling} and its corollary. Given this representation and $\kappa_c \ne \bar{\kappa}_c$, Proposition \ref{proposition:optimalNCApolicy} implies that all symmetric BGPs have $\mathbbm{1}^{NCA}_{jt} = \mathbbm{1}^{NCA}$ for the same value for $\mathbbm{1}^{NCA}$. Using these facts in the equilibrium conditions above one can recursively compute all equilibrium variables. 
	
	Now suppose either (1) or (2) does not hold. Then $z > 0$ implies excess demand for R\&D labor or implies that for $\tilde{V}$ , both of which are incompatible with equilibrium. In such cases, there exists a unique symmetric BGP with $z = 0$. Uniqueness is immediate, since we already know that $\hat{z} = \bar{L}_{RD}$ and $\tau^S = 0$ so we can compute all equilibrium variables uniquely following the steps in the main text. It remains to confirm that $z = 0$ is in fact optimal for the incumbent. First suppose that it is the first condition (1) which does not hold. Because $\hat{w}_{RD} > 0$ in any equilibrium, the denominator $\chi(\lambda -1) - (1 - \mathbbm{1}^{NCA})(1 - (1-\kappa_e) \lambda) \nu - \mathbbm{1}^{NCA} \kappa_c \nu $ is negative. This in turn implies that there is no positive value $\tilde{V}$ such that $z > 0$ satisfies the incumbent's FOC. Therefore, $z = 0$ is consistent with equilibrium. Next, suppose that it is the second condition (2) which does not hold. This means that if $\hat{w}_{RD} / \tilde{V}$ is such that incumbents are indifferent about R\&D, then $\hat{z} > \bar{L}_{RD}$. In order for the fact that $\hat{z} = \bar{L}_{RD}$ to be consistent with equilibrium, one must have $\hat{w}_{RD} / \tilde{V}$ increase. This means the incumbent strictly prefers $z = 0$.
\end{proof}

\subsection{Additional derivations}

\subsubsection{Growth accounting equation}\label{appendix:model:growth_accounting_equation}

Let $\Delta > 0$ and let $J_0(\Delta)$ ($J_1(\Delta)$) denote the indices $j\in [0,1]$ on which innovation occurs zero (one) times between $t$ and $t+\Delta$. By a law of large numbers applied to the continuum of random processes $\{\tilde{q}_{jt}\}_{j \in [0,1]}$ (see \cite{uhlig_law_1996} for more details), the set $J_1(\Delta)$ has measure $\mu_1 \Delta = (\tau + \tau^S + \hat{\tau})\Delta + o(\Delta)$. The set $J_0(\Delta)$ has measure $1 - \mu_1 \Delta + o(\Delta)$. 
\begin{align*}
	Q_{t+\Delta} = \int_0^1 \bar{q}_{j,t+\Delta} dj &= \int_{j \in J_0} \bar{q}_{jt} dj + \int_{j \in J_1} \lambda \bar{q}_{jt} dj + o(\Delta) \\
	&= (1 - \mu_1\Delta - o(\Delta)) Q_t + (\mu_1 \Delta + o(\Delta) ) \lambda Q_t + o(\Delta) \\
	&= (1 - \mu_1\Delta) Q_t + \mu_1\Delta \lambda Q_t + o(\Delta)
\end{align*}
where I used the fact that $\mathbb{E}[\bar{q}_{jt} | j \in J_0, t]  = \mathbb{E}[\bar{q}_{jt} | j \in J_1, t] = Q_t$, since innovations happen at the same rate regardless of $\bar{q}_{jt}$. It follows that
\begin{align*}
	\frac{\dot{Q}_t}{Q_t} = \frac{\lim_{\Delta \to 0} \frac{Q_{t+\Delta} - Q_t}{\Delta}}{Q_t} &= (\lambda - 1)\mu_1 
\end{align*}

\subsubsection{Multiplicity of equilibria}\label{appendix:model:multiplicity_of_equilibria}

\begin{proposition}\label{proposition:purestrategyeq:incumbents_indifferent}
	If Assumptions \ref{ineq:vtilde_denom_positive}, \ref{} and \ref{model:assumption:boundedUtility1} hold and $\kappa_c = \bar{\kappa}_c$ (as defined in Proposition \ref{proposition:optimalNCApolicy}), then:
	\begin{enumerate}
		\item There exist exactly two symmetric BGPs with $\mathbbm{1}^{NCA}_{jt} = \mathbbm{1}^{NCA}$: one with $\mathbbm{1}^{NCA}_{jt} = 0$ and one with $\mathbbm{1}^{NCA}_{jt} = 1$.
		\item Both such equilibria have the same R\&D labor allocations $z, \hat{z}$
		\item The equilibrium with $\mathbbm{1}^{NCA}_{jt} = 0$ has a higher growth rate $g$ 
	\end{enumerate} 
\end{proposition}

\begin{proof}
	The proof of the first part is essentially the same as that of the previous proposition. The only difference is that either choice $\mathbbm{1}^{NCA}_{jt} = 1$ or $\mathbbm{1}^{NCA}_{jt} = 0$ is valid under Proposition \ref{proposition:optimalNCApolicy}. Given the representation $V(j,t|q) = \tilde{V}q$ and the scaling of wages $\hat{w}_{RD,t} = \hat{w}_{RD}Qt$ and $w_{RD,j}(\mathbbm{1}^{NCA}) = w_RD(\mathbbm{1}^{NCA}) Q_t$, the derivation above uniquely determines uniquely the rest of the equilibrium conditional on $x$. This equilibrium has finite household utility as long as Assumption \ref{model:assumption:boundedUtility1} holds. 
	
	The second part follows from the fact that when $\kappa_c = \bar{\kappa}_c$, the expressions for equilibrium R\&D effort $\hat{z},z$ do not depend on $\mathbbm{1}^{NCA}$. The reason is that $\mathbbm{1}^{NCA}$ only affects $\hat{z},z$ through its effect on the incumbent's effective wage, but here is the incumbent is indifferent between $\mathbbm{1}^{NCA} = 1$ and $\mathbbm{1}^{NCA} = 0$ hence faces the same effective wage. Mathematically, (\ref{eq:effort_entrant}) has the expression $(1-\mathbbm{1}^{NCA})(1-(1-\kappa_e)\lambda)\nu - \mathbbm{1}^{NCA} \kappa_c \nu = (1-\mathbbm{1}^{NCA}) \bar{\kappa}_c \nu + \mathbbm{1}^{NCA} \kappa_c \nu$ in the denominator. Since $\kappa_c = \bar{\kappa}_c$, $\hat{z}$ is unaffected by $\mathbbm{1}^{NCA}$, which in turn implies $z$ is also unaffected.
	
	The last statement follows from the fact that $z,\hat{z}$ are the same in both equilibria, but $\tau^S = 0$ when $\mathbbm{1}^{NCA} = 1$ and $\tau^S = \nu z^I > 0$ when $\mathbbm{1}^{NCA} = 0$. By the growth accounting equation (\ref{eq:growth_accounting}), this implies $g$ is higher when $\mathbbm{1}^{NCA} = 0$. 
\end{proof}

On the knife-edge $\kappa_c = \bar{\kappa}_c$, there exist symmetric BGPs in which incumbents do not all choose the same NCA policy, as described in the next proposition. 

\begin{proposition}\label{proposition:mixedstrategyeq}
	If $\theta \ge 1$, $\kappa_c = \bar{\kappa}_c$, and $\Big( \frac{\hat{\chi} (1-\kappa_{e}) \lambda}{\chi(\lambda-1) - \kappa_{c} \nu} \Big)^{1/\psi} < \bar{L}_{RD}$, then for all $f \in (0,1)$ there exists a symmetric BGP in which, at any given time $t$, a fraction $f$ of incumbents $j$ have $\mathbbm{1}^{NCA}_{jt} = 1$.  
\end{proposition}

\begin{proof}
	Consider the generalized growth accounting equation (it simplifies to the one in the main text when $\mathbbm{1}^{NCA}_{jt} = \mathbbm{1}^{NCA}$),
	\begin{align}
		g_t &= (\lambda -1) \Big( \tau + \hat{\tau} + z \nu \int_{j : \mathbbm{1}^{NCA}_{jt} = 0} \frac{\bar{q}_{jt}}{Q_t} dj \Big) \label{eq:generalized_growth_accounting0}
	\end{align}
	
	Unless the integral term is constant, then $g_t$ is non-constant, even with constant $z_{jt},\hat{z}_{jt}$. The integral is equal to the product of the mass $m_t^0$ of goods whose incumbents choose $\mathbbm{1}^{NCA}_{jt} = 0$ and the average relative quality of those goods $\gamma_t^0 = E[\frac{\bar{q}_{jt}}{Q_t} | \mathbbm{1}^{NCA}_{jt} = 0]$. Relative to the baseline model, the only substantial modification is that one needs to derive an expression for the evolution over time of
	\begin{align}
		\Gamma_t^\mathbf{x} &= m_t^{\mathbf{x}} \gamma_t^{\mathbf{x}} Q_t, \quad \mathbf{x} \in \{0,1\},
	\end{align}
	and show that
	\begin{align}
		\frac{\dot{\Gamma}_t^0}{\Gamma_t^0} = \frac{\dot{\Gamma}_t^1}{\Gamma_t^1}.
	\end{align}

	In this proof I will show that the integral remains constant as long as the random process $\mathbbm{1}^{NCA}_{jt}$ follows a certain stationary Markov process with states $\{0,1\}$. One such Markov process for $\mathbbm{1}^{NCA}_{jt}$ is to assume that $\mathbbm{1}^{NCA}_{jt}$ resets every time there is a new incumbent. To take the simplest scenario for the sake of exposition, suppose that the transition probabilities do not depend on the state. Specifically, suppose that, conditional on a transition, the likelihood of transitioning to $\mathbbm{1}^{NCA} = 1$ is $p \in (0,1)$. The equilibrium quantities $\Gamma^0,\Gamma^1$ evolve according to
	\begin{align}
		\Gamma^0_{t+\Delta} &= \overbrace{(1 - \underbrace{(\hat{\tau} + \nu z) \Delta }_{\mathclap{\text{outflow from CD}}} )  \Gamma_t^0}^{\mathclap{\text{No innovations}}} + \overbrace{\tau \Delta (\lambda - 1) \Gamma_t^0}^{\mathclap{\text{Innovating incumbents}}} + \overbrace{(1-p) \lambda \Big( \underbrace{(\hat{\tau} + \nu z) \Delta \Gamma_t^0}_{\mathclap{\mathbbm{1}^{NCA}_{jt} = 0}} +  \underbrace{\hat{\tau} \Delta   \Gamma_t^1}_{\mathclap{\mathbbm{1}^{NCA}_{jt} = 1}} \Big)}^{\mathclap{\text{Inflows}}} + o(\Delta), \\
		 \Gamma^1_{t+\Delta} &= \overbrace{(1 - \underbrace{\hat{\tau} \Delta }_{\mathclap{\text{outflow from CD}}})   \Gamma_t^1}^{\mathclap{\text{No innovations}}}  + \overbrace{\tau \Delta (\lambda -1 ) \Gamma_t^1}^{\mathclap{\text{Innovating incumbents}}}  + \overbrace{p \lambda \Big( \underbrace{(\hat{\tau} + \nu z) \Delta \Gamma_t^0}_{\mathclap{\mathbbm{1}^{NCA}_{jt} = 0}} + \underbrace{\hat{\tau} \Delta \Gamma_t^1}_{\mathclap{\mathbbm{1}^{NCA}_{jt} = 1}} \Big)}^{\mathclap{\text{Inflows}}} + o(\Delta),
	\end{align}
	where $o(\Delta)$ has the usual meaning that $\lim_{\Delta \to 0} \frac{o(\Delta)}{\Delta} = 0$. Subtracting $\Gamma_t^{\mathbf{x}}$, dividing by $\Delta$, and taking the limit as $\Delta \to 0$  yields
	\begin{align}
		\dot{\Gamma}_t^0 &= -(\hat{\tau} + \nu z) \Gamma_t^0 + \tau (\lambda - 1) \Gamma_t^0 + (1-p)\lambda \Big( (\hat{\tau} + \nu z) \Gamma_t^0 + \hat{\tau} \Gamma_t^1 \Big) \\
		\dot{\Gamma}_t^1 &= -\hat{\tau} \Gamma_t^1 + \tau (\lambda - 1) \Gamma_t^1 + p\lambda \Big( (\hat{\tau} + \nu z) \Gamma_t^0 + \hat{\tau} \Gamma_t^1 \Big)
	\end{align}
	
	Dividing by $\Gamma_t^x$ yields
	\begin{align}
		\frac{\dot{\Gamma}_t^0}{\Gamma_t^0} &= -( \hat{\tau} + \nu z) + \tau (\lambda - 1) + (1-p)\lambda \Big( (\hat{\tau} + \nu z) + \hat{\tau} \frac{\Gamma_t^1 }{ \Gamma_t^0}\Big) \\
		\frac{\dot{\Gamma}_t^1}{\Gamma_t^1} &= -\hat{\tau}  + \tau (\lambda - 1) + p\lambda \Big( (\hat{\tau} + \nu z) \big(\frac{\Gamma_t^1}{\Gamma_t^0}\big)^{-1} + \hat{\tau}  \Big)
	\end{align}
	
	Setting $\frac{\dot{\Gamma}_t^0}{\Gamma_t^0} = \frac{\dot{\Gamma}_t^1}{\Gamma_t^1}$ and multiplying both sides by $\frac{\Gamma_t^1}{\Gamma_t^0}$ yields a quadratic equation in $\frac{ \Gamma_t^1}{\Gamma_t^0}$, given by
	\begin{align}
		0 = \overbrace{(1-p) \lambda \hat{\tau}}^{\mathclap{a}}\Big( \frac{\Gamma_t^1}{\Gamma_t^0}\Big)^2 + \overbrace{\big( (1-p) \lambda (\hat{\tau} + \nu z) - \nu z - p\lambda \hat{\tau} \big)}^{\mathclap{b}} \Big( \frac{\Gamma_t^1}{\Gamma_t^0}\Big) - \overbrace{p\lambda (\hat{\tau} + \nu z)}^{\mathclap{c}}.
	\end{align}
	Using $p \in (0,1)$ and the facts that $\lambda, \hat{\tau} > 0$ and $\nu, z \ge 0$ yields $-4ac > 0$. Then $\frac{\Gamma_t^1}{\Gamma_t^0} = \frac{-b \pm \sqrt{b^2 - 4ac}}{2a}$ implies that there is always exactly one strictly positive real solution for $\frac{\Gamma^0_t}{\Gamma^1_t}$. To see this, note that of course $b^2 - 4ac > 0$ so the all solutions are real. Then, $-4ac > 0$ implies $\sqrt{b^2 - 4ac} > |b|$. Regardless of whether $b$ is positive or negative, $-b + \sqrt{b^2 - 4ac} > 0$ and $b - \sqrt{b^2 - 4ac} < 0$. The positive root is the equilibrium value of $\frac{\Gamma_t^1}{\Gamma_t^0}$. Using
	\begin{align}
		\Gamma_t^1 + \Gamma_t^0 &= \int_{j : \mathbbm{1}^{NCA}_{jt} = 1} \bar{q}_{jt} dj + \int_{j : \mathbbm{1}^{NCA}_{jt} = 0} \bar{q}_{jt} dj \nonumber \\
		                        &= \int_0^1 \bar{q}_{jt} dj \nonumber  \\
		                        &= Q_t,
	\end{align}
	one has a linear system of two equations in two unknowns, $\Gamma_t^1$ and $\Gamma_t^0$. Given $\frac{\Gamma_t^0}{Q_t} = \int_{j : \mathbbm{1}^{NCA}_{jt} = 0} \frac{\bar{q}_{jt}}{Q_t} dj$, the BGP growth rate is given by (\ref{eq:generalized_growth_accounting0}) above.   
\end{proof}

\paragraph{Application to model with incumbent heterogeneity}\label{appendix:model:heterogeneity}

\textbf{[Think about whether putting this in makes sense given that it kind of requires also adding decreasing returns for the incumbent...]}

The above construction and derivation can be adapted to a richer model where there is heterogeneity in $\xi_{jt} = \{\kappa_{e,jt}, \kappa_{c,jt}, \nu_{jt}\}$ across goods $j$ and times $t$, inducing heterogeneity in chosen $\{z_{jt}, \hat{z}_{jt}, \mathbbm{1}^{NCA}_{jt}\}$. A symmetric BGP in this kind of setting has a more general definition.

\begin{definition}
	A symmetric BGP in the model with incumbent heterogeneity is a BGP where there exist functions $z(\kappa_e, \kappa_c, \nu), \hat{z}(\kappa_e, \kappa_c , \nu), \mathbbm{1}^{NCA} (\kappa_e , \kappa_c, \nu)$ such that 
	\begin{align}
		z_{jt} &= z(\kappa_{e,jt}, \kappa_{c,jt}, \nu_{jt}), \\
		\hat{z}_{jt} &= \hat{z}_{jt} (\kappa_{e,jt}, \kappa_{c,jt}, \nu_{jt}), \\
		\mathbbm{1}^{NCA}_{jt} &= \mathbbm{1}^{NCA} (\kappa_{e,jt}, \kappa_{c,jt}, \nu_{jt}),
	\end{align}
	where $\kappa_{e,jt}, \kappa_{c,jt}, \nu_{jt}$ denote particular realizations of the random process describing the evolution of $\kappa_e, \kappa_c, \nu$ in good $j$. 
\end{definition}

That is, a symmetric BGP is one in which all agents behave the same way when they are in the same exogenous state.\footnote{I include a symmetry condition on $\mathbbm{1}^{NCA}_{jt}$ here as with a dynamic state it is not natural to rule out the possibility of occasionally hitting the knife-edge.} As in the preceding result, a tractable BGP in this setup only requires that the state variable follow an exogenous stationary Markov process which satisfies a ``mixing'' condition. The latter condition essentially requires that there be no absorbing subset of states. This is similar to the standard necessary conditions for the existence of a stationary equilibrium in models with heterogeneous agents.

The density $\mu(\mathbbm{1}^{NCA})$ can be derived using a Kolmogorov Forward (KF) Equation. The nature of this equation depends on the nature of the supposed Markov process for $\xi_t = (\kappa_{e,jt}, \kappa_{c,jt}, \nu_{jt})$. To simplify the exposition, I consider an example in which (1) there is a discrete set of states, (2) transition probabilities do not depend on the current state, and (3) the state jumps each time there is a new incumbent. However, the logic could be extended to a case where $\xi_{jt}$ follows a diffusion with jumps, at the cost of additional mathematical machinery. Given these simplifying assumptions, we are in a similar situation as in the ``mixed strategy'' equilibrium of the original model, with the exception that $z_{jt}, \hat{z}_{jt}$ now depend on the state $\xi_{jt}$. It is easy to see how to modify the equations in this case. For terms involving inflows or outflows from a given state, the rate of the flow depends on the incumbent and entrant policies in the source state. 
\begin{align}
	\dot{\Gamma}_t (\xi) &= 
\end{align}


The system of difference equations for $m^{\textbf{x}} \Gamma_t^{\textbf{x}}$ are replaced by a functional difference equation, 
\begin{align}
	\mu(\mathbbm{1}^{NCA}) \Gamma_{t+\Delta}^x &= (1- \text{CD}(\mathbbm{1}^{NCA}) \Delta) \mu(\mathbbm{1}^{NCA}) \Gamma_t^x + \text{OI}(\mathbbm{1}^{NCA}) \Delta (\lambda -1) \mu(\mathbbm{1}^{NCA}) \Gamma_t^x +  j^x \Delta  \lambda \int_{x' \in \mathbf{X}} \text{CD}(x') \Gamma_t^{x'} \mu(x') dx'
\end{align}
where $j^x$ is the density of the injection rate into state $x$ out of new incumbents. This can then be used to derive a functional differential equation,
\begin{align}
	\frac{\dot{\Gamma}_{t}^x}{\Gamma_t^x} &= OI(\mathbbm{1}^{NCA}) (\lambda -1) - CD(\mathbbm{1}^{NCA}) + j^x \lambda (\mu(\mathbbm{1}^{NCA}) \Gamma_t^x)^{-1} \int_{x' \in \mathbf{X}} \text{CD}(x') \Gamma_t^{x'} \mu(x') dx'
\end{align}

Imposing the condition $\frac{\dot{\Gamma}_{t}^x}{\Gamma_t^x} = g$ for an unknown constant $g$ pins down the ratio $\frac{\int_{x' \in \mathbf{X}} \text{CD}(x') \Gamma_t^{x'} \mu(x') dx'}{\mu(\mathbbm{1}^{NCA}) \Gamma_t^x}$ for each $x$, determining the shape of the distribution $\Gamma_t^x$ (since $\mu(\mathbbm{1}^{NCA})$ is already determined by the KF equation). If the relevant functions are differentiable, the condition can also be derived by differentiating the expression for $\frac{\dot{\Gamma}_t^x}{\Gamma_t^x}$ with respect to $x$ and setting it equal to zero. This yields
\begin{align}
	0 = \text{OI}'(\mathbbm{1}^{NCA}) (\lambda -1) - \text{CD}'(\mathbbm{1}^{NCA}) + \lambda \int_{x' \in \mathbf{X}} \text{CD}(x') \Gamma_t^{x'} \mu(x') dx' \Big(\frac{d}{dx} j^x \mu(\mathbbm{1}^{NCA}) \Gamma_t^x \Big)^{-1}
\end{align} 

where one would need to expand the last derivative using the product rule (recalling that all three terms depend on $x$). 

The scale of the distribution at time $t$ is determined by 
\begin{align}
	\int_{x' \in \mathbf{X}} \Gamma_t^{x'} \mu(x') dx' = Q_t
\end{align}




\subsection{Policy analysis derivations}

\subsubsection{R\&D subsidy (tax)}\label{appendix:model:efficiencyderivations:RDsubsidy}

Suppose that the planner subsidizes R\&D spending at rate $T_{RD}$ (tax if $T_{RD} < 0$). In this case, in a symmetric BGP the incumbent's HJB becomes
\begin{align}
	(r + \hat{\tau}) \tilde{V} = \tilde{\pi} + \max_{\substack{\mathbbm{1}^{NCA} \in \{0,1\} \\ z \ge 0}} \Big\{z &\Big( \overbrace{\chi (\lambda - 1) \tilde{V}}^{\mathclap{\mathbb{E}[\textrm{Benefit from R\&D}]}}- (\underbrace{1-T_{RD}}_{\mathclap{\text{R\&D Subsidy}}}) \big( \overbrace{\hat{w}_{RD} - (1-\mathbbm{1}^{NCA})(1-\kappa_e)\lambda \nu \tilde{V}}^{\mathclap{\text{Incumbent R\&D wage}}}\big) \label{eq:hjb_incumbent_RDsubsidy_appendix} \nonumber \\ 
	&-  \underbrace{(1-\mathbbm{1}^{NCA}) \nu \tilde{V}}_{\mathclap{\text{Net cost from spinout formation}}} - \overbrace{\mathbbm{1}^{NCA} \kappa_{c} \nu \tilde{V}}^{\mathclap{\text{Direct cost of NCA}}}\Big) \Big\}. 
\end{align}
Then if $z > 0$, the incumbent's optimal NCA policy is given by 
\begin{align}
	x = \begin{cases}
		1 & \textrm{if } \kappa_{c} < \tilde{\bar{\kappa}}_c,  \\
		0 & \textrm{if } \kappa_{c} > \tilde{\bar{\kappa}}_c, \\
		\{0,1\} & \textrm{if } \kappa_c = \tilde{\bar{\kappa}}_c,
	\end{cases} \label{eq:nca_policy_RDsubsidy}
\end{align}
where $\tilde{\bar{\kappa}}_c = 1 - (1-T_{RD})(1-\kappa_e)\lambda$. Since the argument is the same as in the baseline model, I omit the details. Assuming $z > 0$, by the same logic as before one can obtain an expression for equilibrium $\hat{z}$, 
\begin{align}
	\hat{z} &= \Bigg( \frac{\hat{\chi} (1-\kappa_{e}) \lambda}{\chi(\lambda -1) - \nu (\mathbbm{1}^{NCA}\kappa_c + (1-\mathbbm{1}^{NCA})(1 - (1-T_{RD})(1-\kappa_e)\lambda)) } \Bigg)^{1/\psi}. \label{eq:effort_entrant_RDsubsidy_appendix}
\end{align}
The rest of the equilibrium allocation and prices can be computed in the same way as before (including how to account for the possibility of $z = 0$), with the one exception being that the equilibrium R\&D wage is now given by 
\begin{align}
	\hat{w}_{RD} &= (1-T_{RD})^{-1}\hat{\chi} \hat{z}^{-\psi} (1-\kappa_e) \lambda \tilde{V}. \label{eq:wage_rd_labor_RDsubsidy_appendix}
\end{align}

\subsubsection{Targeted R\&D subsidy (tax)}\label{appendix:model:efficiencyderivations:OIRDtax}

The incumbent's HJB is given by
\begin{align}
	(r + \hat{\tau}) \tilde{V} = \tilde{\pi} + \max_{\substack{\mathbbm{1}^{NCA} \in \{0,1\} \\ z \ge 0}} \Big\{z &\Big( \overbrace{\chi (\lambda - 1) \tilde{V}}^{\mathclap{\mathbb{E}[\textrm{Benefit from R\&D}]}}- (\underbrace{1-T_{RD,I}}_{\mathclap{\text{R\&D Subsidy}}}) \big( \overbrace{\hat{w}_{RD} - (1-\mathbbm{1}^{NCA})(1-\kappa_e)\lambda \nu \tilde{V}}^{\mathclap{\text{R\&D wage}}}\big) \label{eq:hjb_incumbent_RDsubsidyTargeted} \nonumber \\ 
	&-  \underbrace{(1-\mathbbm{1}^{NCA}) \nu \tilde{V}}_{\mathclap{\text{Net cost from spinout formation}}} - \overbrace{x \kappa_{c} \nu \tilde{V}}^{\mathclap{\text{Direct cost of NCA}}}\Big) \Big\}.
\end{align}
It can be rearranged to a form analogous to (\ref{eq:hjb_incumbent_workerIndiff}),
\begin{align}
	(r + \hat{\tau}) \tilde{V} = \tilde{\pi} + \max_{\substack{\mathbbm{1}^{NCA} \in \{0,1\} \\ z \ge 0}} \Big\{z &\Big( \overbrace{\chi (\lambda - 1) \tilde{V}}^{\mathclap{\mathbb{E}[\textrm{Benefit from R\&D}]}}- (1-T_{RD,I}) \hat{w}_{RD} \\
	&-  \underbrace{(1-\mathbbm{1}^{NCA})(1 - (1-T_{RD,I})(1-\kappa_{e})\lambda)\nu \tilde{V}}_{\mathclap{\text{Net cost from spinout formation}}} - \overbrace{\mathbbm{1}^{NCA} \kappa_{c} \nu \tilde{V}}^{\mathclap{\text{Direct cost of NCA}}}\Big) \Big\}.9n 000000  \label{eq:hjb_incumbent_RDsubsidyTargeted_2}
\end{align}
The non-compete policy is the same as before, with new threshold given by 
\begin{align}
	\hat{\bar{\kappa}}_c = 1 - (1-T_{RD,I})(1-\kappa_e)\lambda.
\end{align} 
Using the same approach as before one obtains an expression for $\hat{z}$, 
\begin{align}
	\hat{z} &= \Bigg( \frac{(1-T_{RD,I})\hat{\chi} (1-\kappa_{e}) \lambda}{\chi(\lambda -1) - \nu (\mathbbm{1}^{NCA} \kappa_c + (1-\mathbbm{1}^{NCA})(1 - (1-T_{RD,I})(1-\kappa_e)\lambda)) } \Bigg)^{1/\psi}. \label{eq:effort_entrant_RDsubsidyTargeted}
\end{align}
The remaining equilibrium conditions are
\begin{align}
\hat{\tau} &= \hat{\chi} \hat{z}^{1-\psi}, \\
z &= \bar{L}_{RD} - \hat{z}, \label{eq:labor_resource_constraint_RDsubsidyTargeted}\\ 
\tau &= \chi z ,\\
\tau^S &= (1-\mathbbm{1}^{NCA}) \nu z ,\\
g &= (\lambda - 1) (\tau + \tau^S + \hat{\tau}), \\
r &= \theta g + \rho ,\\
\tilde{V} &= \frac{\tilde{\pi}}{r + \hat{\tau}}, \\ 
\hat{w}_{RD} &= \hat{\chi} \hat{z}^{-\psi} (1-\kappa_e) \lambda \tilde{V}. \label{eq:wage_rd_labor_RDsubsidyTargeted}
\end{align}

\subsubsection{All policies}\label{appendix:model:efficiencyderivations:allPolicies}

The R\&D labor supply indifference condition is
\begin{align}
\hat{w}_{RD} &= w_{RD,j} + (1-x_j) \nu (1-(\underbrace{1+T_e}_{\mathclap{\text{Entry tax}}})\kappa_e) \lambda \tilde{V}. \label{eq:RD_worker_indifference_all}
\end{align}
The incumbent HJB is
\begin{align}
(r + \hat{\tau}) \tilde{V} = \tilde{\pi} + \max_{\substack{\mathbbm{1}^{NCA} \in \{0,1\} \\ z \ge 0}} \Big\{z &\Big( \overbrace{\chi (\lambda - 1) \tilde{V}}^{\mathclap{\mathbb{E}[\textrm{Benefit from R\&D}]}}-  (\underbrace{1 - T_{RD} - T_{RD,I}}_{\mathclap{\text{R\&D subsidies}}})\big( \overbrace{\hat{w}_{RD} - (1-\mathbbm{1}^{NCA})(1-\kappa_e)\lambda \nu \tilde{V}}^{\mathclap{\text{R\&D wage}}}\big) \label{eq:hjb_incumbent_all} \nonumber \\ 
&-  \underbrace{(1-\mathbbm{1}^{NCA}) \nu \tilde{V}}_{\mathclap{\text{Net cost from spinout formation}}} - \overbrace{x \kappa_{c} \nu \tilde{V}}^{\mathclap{\text{Direct cost of NCA}}}\Big) \Big\},
\end{align}
which can be rearranged to
\begin{align}
(r + \hat{\tau}) \tilde{V} = \tilde{\pi} + \max_{\substack{\mathbbm{1}^{NCA} \in \{0,1\} \\ z \ge 0}} \Big\{z &\Big( \overbrace{\chi (\lambda - 1) \tilde{V}}^{\mathclap{\mathbb{E}[\textrm{Benefit from R\&D}]}}- (1-T_{RD}- T_{RD,I})\hat{w}_{RD} \\
&-  \underbrace{(1-\mathbbm{1}^{NCA})(1 - (1-T_{RD} - T_{RD,I})(1-\kappa_{e})\lambda)\nu \tilde{V}}_{\mathclap{\text{Net cost from spinout formation}}} - \overbrace{x \kappa_{c}\nu \tilde{V}}^{\mathclap{\text{Direct cost of NCA}}}\Big) \Big\}. \label{eq:hjb_incumbent_all_2}
\end{align}
Define
\begin{align}
\bar{\bar{\kappa}}_c = 1 - (1-T_{RD} - T_{RD,I})(1-\kappa_e)\lambda. \label{eq:barkappa_all}
\end{align} 
If $z > 0$, the incumbent's optimal NCA policy is given by 
\begin{align}
x = \begin{cases}
1 & \textrm{if } \kappa_c < \bar{\bar{\kappa}}_c, \\
0 & \textrm{if } \kappa_c > \bar{\bar{\kappa}}_c, \\
\{0,1\} & \textrm{if } \kappa_c = \bar{\bar{\kappa}}_c.
\end{cases} \label{eq:nca_policy_all}
\end{align}
By the usual argument, $z > 0$ implies that the incumbent's FOC can be rearranged to
\begin{align}
\tilde{V} &= \frac{(1-T_{RD} - T_{RD,I})\hat{w}_{RD}}{\chi(\lambda -1) - \nu (x\kappa_c + (1-\mathbbm{1}^{NCA})(1 - (1-T_{RD} - T_{RD,I})(1-(1+T_e)\kappa_e)\lambda)) }. \label{eq:hjb_incumbent_foc_all}
\end{align}
The entrant optimality condition is
\begin{align}
\underbrace{\hat{\chi} \hat{z}^{-\psi}}_{\mathclap{\text{Marginal innovation rate}}} \overbrace{(1-(1+T_e)\kappa_e) \lambda \tilde{V}}^{\mathclap{\text{Payoff from innovation}}} &= (1-T_{RD})\underbrace{\hat{w}_{RD}}_{\mathclap{\text{MC of R\&D}}}. \label{eq:free_entry_condition_all}
\end{align}
Substituting (\ref{eq:hjb_incumbent_foc_all}) into (\ref{eq:free_entry_condition_all}) to eliminate $\tilde{V}$ yields an expression for $\hat{z}$, 
\begin{align}
\hat{z} &= \Bigg( \frac{\Big(\frac{1-T_{RD} -T_{RD,I}}{1-T_{RD}} \Big)\hat{\chi} (1-(1+T_e)\kappa_{e}) \lambda}{\chi(\lambda -1) - \nu (x\kappa_c  + (1-\mathbbm{1}^{NCA})(1 - (1-T_{RD} - T_{RD,I})(1-(1+T_e)\kappa_e)\lambda)) } \Bigg)^{1/\psi}. \label{eq:effort_entrant_all}
\end{align}
From here, the rest of the model can be solved using
\begin{align}
\hat{\tau} &= \hat{\chi} \hat{z}^{1-\psi}, \\
z &= \bar{L}_{RD} - \hat{z}, \label{eq:labor_resource_constraint_all}\\ 
\tau &= \chi z, \\
\tau^S &= (1-\mathbbm{1}^{NCA}) \nu z, \\
g &= (\lambda - 1) (\tau + \tau^S + \hat{\tau}), \\
r &= \theta g + \rho, \\
\tilde{V} &= \frac{\tilde{\pi}}{r + \hat{\tau}}, \\ 
\hat{w}_{RD} &= \begin{cases}
(1-T_{RD})^{-1}\hat{\chi} \hat{z}^{-\psi} (1 - \kappa_e) \lambda \tilde{V} &\textrm{, if } \hat{z} > 0, \\
\Big( \chi(\lambda -1) - \nu (x\kappa_c + (1-\mathbbm{1}^{NCA})\bar{\bar{\kappa}}_c)\Big) \tilde{V} &\textrm{, o.w.}
\end{cases} \label{eq:wage_rd_labor_all}
\end{align}



\section{Calibration}\label{appendix:calibration}

\subsection{Computing model moments}

\subsubsection{Profit to GDP}\label{appendix:calibration:profits/gdp}

In the model, this ratio is simple to calculate using the solution to the static equilibrium as $\tilde{\pi} / \tilde{Y}$.

\subsubsection{R\&D to GDP}\label{appendix:calibration:rd/gdp}

In the model, the R\&D share is the ratio of the wage paid to R\&D workers to GDP. This is
\begin{align*}
\frac{\textrm{R\&D wage bill}}{\textrm{GDP}} &= \frac{w_{RD} z + \hat{w}_{RD} \hat{z}}{\tilde{Y}} \\ 
&= \frac{\hat{w}_{RD} (z + \hat{z}) + (w_{RD} - \hat{w}_{RD})z}{\tilde{Y}} \\
&= \frac{\hat{w}_{RD} (z + \hat{z}) - (1-\kappa_e) \lambda \tilde{V} \tau^S}{\tilde{Y}}
\end{align*}

where I used $w_{RD} - \hat{w}_{RD} = -(1-\mathbbm{1}^{NCA})(1-\kappa_e) \lambda \tilde{V} \nu$ and $\tau^S = (1-\mathbbm{1}^{NCA})\nu z$. 

\subsubsection{Growth share OI}\label{appendix:calibration:growthShareOI}

The model moment that corresponds here is the share of growth due to own innovation by incumbents of age >= 6. In the model, the fraction of OI growth due to incumbents in a given age group is exactly their fraction of employment: innovations arrive at the same rate for each incumbent, and their impact on aggregate growth is proportional to the incumbent's relative quality, which is proportional to employment. Hence old incumbents' share of growth due to own innovation is simply one minus the employment share calculated in the previous paragraph, $e^{((\hat{\tau}_I -1)g - (\hat{\tau} + (1-\mathbbm{1}^{NCA})z \nu))\cdot 6}$. Finally, the fraction of aggregate growth due to OI is $\hat{\tau}_i$, defined above. The fraction of growth due to incumbents of age at least 6 is the product of the two, 
\begin{align*}
\textrm{Age >= 6 share of OI} &= \hat{\tau}_I \frac{\ell(6)}{\ell(0)} \\
&= \hat{\tau}_I e^{((\hat{\tau}_I -1)g - (\hat{\tau} + (1-\mathbbm{1}^{NCA})z \nu))\cdot 6} 
\end{align*}


\subsubsection{Entry rate}\label{appendix:calibration:entryRate}

Let $\ell(a)$ denote the density of incumbent employment at age $a$ incumbents. Then $\ell(a)$ is characterized by 
\begin{align*}
\ell(a) &= \ell(0)e^{((\hat{\tau}_I -1)g - (\hat{\tau} + \tau^S))a}  \\
1 - \hat{z} &= \int_0^{\infty} \ell(a) da
\end{align*}

where $\hat{\tau}_I = \frac{\tau}{\tau + \hat{\tau} + \tau^S}$ is the fraction of innovations that are incumbents' own innovations. 

The intuition for this characterization of $\ell(a)$ has two parts. First, because all shocks are \textit{iid} across firms in equilibrium, the law of large numbers applied to each cohort of firms implies that we can consider directly the evolution of the cohort as a whole instead of explicitly analyzing the dynamics each individual firm in the cohort.  Second, the employment of a firm is proportional to its relative quality, $l_j \propto \tilde{q}_j = q_j / Q$, as long as it is the leader. When it is no longer the leader, its employment is zero forever. Putting these two together, $\ell(a)$ must decline at exponential rate $g$ due to the increase in $Q_t$ (obsolescence), increase at rate $\hat{\tau}_I g$ due to incumbents own innovations, and decline at rate $\hat{\tau} + \tau^S$ due to creative destruction.\footnote{The second equation imposes consistency with aggregate employment; it implies $\ell(0) = -((\hat{\tau}_I -1)g - (\hat{\tau} + \tau^S))(1-\hat{z})$. The calibration does not require this explicit calculation since it is based only on employment shares.} Note that the employment density is strictly decreasing in $a$. This is because there are no adjustment costs: firms achieve their optimal scale immediately upon entry, and subsequently become obsolete (on average) or lose the innovation race to an entrant. Finally, due to the constant exponential decay of $\ell(a)$, the share of incumbent employment in incumbents of strictly less than 6 years of age is given by 
\begin{align*}
\Xi_{[0,6)} &=  1 - \frac{\ell(6)}{\ell(0)} \\
&= 1 - e^{((\hat{\tau}_I -1)g - (\hat{\tau} + \tau^S))\cdot 6}
\end{align*}  


To calculate the share of employment in incumbents age < 6, I use as denominator the employment of intermediate goods firms in the economy (including R\&D by entrants). When bringing this to the data, it is equivalent to assuming that the age-employment distribution of final goods firms is the same as that of intermediate goods firms. Using this approach, the share of overall employment in incumbents of age < 6, including R\&D performed by non-producing entrants, is equal to the previously calculated $\Xi_{[0,6)}$ multiplied by the share of total labor in incumbents $1 - L_F - \hat{z}$, added to the R\&D labor used by entrants $\hat{z}$, divided by the share of total employment in intermediates $1 - L_F$, and finally multiplied by 2/3 which is the share of creative destruction that corresponds to new firms in the data.\footnote{Alternatively, one could assume that final goods firms have the same employment-age distribution as other intermediate goods firms. Then the formula would be
	
	\begin{align*}
	\textrm{Age < 6 share of employment} &= \frac{2}{3}(\Xi_{[0,6)} (1-\hat{z}) + \hat{z})
	\end{align*}
	
	This has only minor effects on the inferred parameters. They are listed in \autoref{calibration_2_parameters}.} This yields
\begin{align*}
\textrm{Age < 6 share of employment} &= \frac{2}{3} \frac{(\Xi_{[0,6)} (1 - L_F -\hat{z}) + \hat{z})}{1-L_F}
\end{align*}

The factor $2/3$ deserves some additional discussion. According to \cite{klenow_innovative_2020}, creative destruction by incumbents is responsible for half as much growth as creative destruction by entrants. In this interpretation of the model, both types of creative destruction use the same technology. Therefore, it follows that 2/3 of employment in young firms in the model represents employment in young firms in the data.

\subsubsection{Employment share of WSOs}\label{appendix:calibration:WSOempShare}

Because successfully innovating spinouts and entrants have identical expected growth dynamics, the BGP share of employment in firms started as spinouts is their share of new incumbents $\frac{\tau^S}{\tau^S+ (\frac{2}{3})\hat{\tau}}$, multiplied by the employment share of incumbents $\frac{1-L_F- (\frac{2}{3})\hat{z}}{1-L_F}$, 
\begin{align*}
\textrm{Spinout employment share} &= \frac{\tau^S}{\tau^S + \frac{2}{3}\hat{\tau}} \times \frac{1-L_F- (\frac{2}{3})\hat{z}}{1-L_F} 
\end{align*}

Again, the factor 2/3 is because this is the fraction of entrants in the model which the calibration maps to new firms in the data.

\section{Policy analysis}

\subsection{NCA cost $\kappa_c$}\label{appendix:policyanalysis:ncacost}

\paragraph{Robustness of welfare gain from NCA enforcement}

\autoref{welfareComparisonSensitivityFull} shows the sensitivity of the welfare comparison the moments targeted, including the externally calibrated parameters as pseudo-moments as before. It is computed as $\nabla_m \tilde{W}|_m = (J^{-1})^T \nabla_p W|_p$, where $J$ is the Jacobian of the mapping from log parameters to moments (so that $J^{-1}$ is the Jacobian of the inverse mapping), and $W$ is the mapping from parameters the log \% change (or raw \% change, in \autoref{levelsWelfareComparisonSensitivityFull})) in CE welfare from reducing $\kappa_c$ from $\infty$ to $0$. That is, it is the gradient of the change in welfare to the change in target moments or uncalibrated parameters, taking as given the change in parameters required to continue matching the target moments. For reference, $\nabla_p W|_p$  for each definition of $W$ can be found in \autoref{welfareComparisonParameterSensitivityFull} and \autoref{levelsWelfareComparisonParameterSensitivityFull}.

To get a sense of what this means about robustness of the results, suppose that the log of each moment is assumed to have a standard deviation of $\sigma$ and that this uncertainty is uncorrelated across moments. The uncertainty propagates such that the standard deviation of the CE welfare change is the square root of $(\nabla_m \tilde{W}|_m)^T \Sigma_m \nabla_m \tilde{W}|_m$, where $\Sigma_m = \sigma^2 I_{9\times 9}$. In this examples this yields 6.9$\sigma$ log points of uncertainty, or about 50\% of the result for $\sigma = 0.1$. I conclude that the result is in fact somewhat sensitive to the measurement of the moments. Because many decisions went into the computation of the moments I used (as well as their model counterparts), it is crucial in future work to find more rigorous ways to bring this and similar models to the data.


\begin{figure}[]
	\centering
	\includegraphics[scale = 0.7]{../code/julia/figures/simpleModel/welfareComparisonSensitivityFull.pdf}
	\caption{Sensitivity of welfare comparison to moments. This is $(J^{-1})^T \nabla_p W$, where $W(p)$ maps log parameters to the log of the percentage change in BGP consumption which is equivalent to the change in welfare from changing $\kappa_c$ from $\infty$ to $0$ (i.e. going from banning to frictionlessly enforcing NCAs). The way to read this is the following. Looking at the column labeled \textit{E}, the chart says that a 1\% increase in the targeted employment share of young firms, which corresponds to a log change of about $0.01$, leads to a 4\% increase in the percentage CE percentage welfare change. In this calibration it is about 1.42\%, so this is about $0.057$ percentage points.}
	\label{welfareComparisonSensitivityFull}
\end{figure}


\paragraph{When are NCAs bad for welfare?}

The sensitivity of the welfare improvement to the entry rate shown in (\ref{welfareComparisonSensitivityFull}) suggests that a calibration targeting a lower rate of creative destruction could have the opposite result. \autoref{calibration_lowEntry_summaryPlot} shows the analogue of \autoref{calibration_summaryPlot} if entry rate targeted is 8\% instead of 13.34\%. The model is again able to match the moments exactly; inferred parameter values are shown in \autoref{calibration_lowEntry_parameters}.

In this low entry calibration, growth and welfare fall when $\kappa_C$ is reduced to zero. The lower employment in young firms (while holding constant the fraction of growth coming from old firms) means that each entry by a young firm must have a higher effect on growth in order for the model to match the growth rate. Furthermore, as shown in \autoref{welfareComparisonParameterSensitivityFull}, the increase in $\lambda$ eliminates the overall welfare gain from reducing $\kappa_c$. As discussed previously, a higher value of $\lambda$ brings (\ref{cs:growth_misallocation_condition}) closer to unity, weakening the growth increase from reallocation of R\&D to OI. Closer inspection reveals that the new calibration also chooses a substantially higher value of $\kappa_e$. This works in the opposite direction, as a reallocation of R\&D to OI reduces the entry cost paid. However, on net, it is still the case that a reduction of $\kappa_c$ from $\bar{\kappa}_c$ to zero reduces both growth and welfare.

\begin{table}[]
	\centering
	\captionof{table}{Low entry rate calibration}\label{calibration_lowEntry_parameters}
	\begin{tabular}{rlll}
		\toprule \toprule
		Parameter & Value & Description & Source \tabularnewline
		\midrule
		$\rho$ & 0.0303 & Discount rate  & Indirect inference \tabularnewline
		$\theta$ & 2 & $\theta^{-1} = $ IES & External calibration 
		\tabularnewline
		$\beta$ & 0.094 & $\beta^{-1} = $ EoS intermediate goods & Exactly identified \tabularnewline 
		$\bar{L}_{RD}$ & 0.01 & R\&D labor allocation  & Exactly identified \tabularnewline
		$\psi$ & 0.5 & Entrant R\&D elasticity & External calibration \tabularnewline
		$\lambda$ & 1.23 & Quality ladder step size & Indirect inference 
		\tabularnewline
		$\chi$ & 5.88 & Incumbent R\&D productivity & Indirect inference 
		\tabularnewline
		$\hat{\chi}$ & 0.370 & Entrant R\&D productivity & Indirect inference \tabularnewline 
		$\kappa_e$ & 0.885 & Non-R\&D entry cost & Indirect inference \tabularnewline
		$\nu$ & 0.145 & Spinout generation rate  & Indirect inference\tabularnewline
		\bottomrule
	\end{tabular}
\end{table}

\begin{table}
	\centering
	\captionof{table}{Effect of reducing $\kappa_c$}\label{reducing_kappa_c_table_lowEntry}
	\begin{tabular}{lrlll}
		\toprule \toprule
		Measure & Variable & $\kappa_c > \bar{\kappa}_c$ & $\kappa_c = 0$ & Chg. \tabularnewline
		\midrule
		Growth & $g$ & 1.487\% & 1.470\% & $-0.017$ p.p. \tabularnewline
		Level & $\tilde{C}$  & 0.776 &  0.785 & $1.16\%$ \tabularnewline 
		\tabularnewline
		Welfare & $\tilde{W}$  &  & & $\mathbf{-0.09\%}$ \textbf{(CE)}  \tabularnewline
		\bottomrule
	\end{tabular}
\end{table}


\begin{figure}[]
	\centering
	\includegraphics[scale = 0.64]{../code/julia/figures/simpleModel/lowEntry_smallSummaryPlot.pdf}
	\caption{Summary of equilibrium for baseline parameter values and various values of $\kappa_c$. The top-left panel shows R\&D labor allocated to incumbents (own-product innovation) and entrants \textsc{(creative destruction)}. The top-right panel shows the aggregate productivity growth rate. The bottom-left panel shows the entry costs paid by entrants and spinouts as well as the direct NCA enforcement cost. Finally, the bottom-right panel shows the level of consumption.}
	\label{calibration_lowEntry_smallSummaryPlot}
\end{figure}



\begin{figure}[]
	\centering
	\includegraphics[scale = 0.57]{../code/julia/figures/simpleModel/lowEntry_SummaryPlot.pdf}
	\caption{Summary of equilibrium for baseline parameter values and various values of $\kappa_c$.}
	\label{calibration_lowEntry_summaryPlot}
\end{figure}









\end{document}