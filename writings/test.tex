\documentclass[11pt,english]{article}
\usepackage{lmodern}
\linespread{1.05}
%\usepackage{mathpazo}
%\usepackage{mathptmx}
%\usepackage{utopia}
\usepackage{microtype}



\usepackage{chngcntr}
\usepackage[nocomma]{optidef}

\usepackage[section]{placeins}
\usepackage[T1]{fontenc}
\usepackage[latin9]{inputenc}
\usepackage[dvipsnames]{xcolor}
\usepackage{geometry}

\usepackage{babel}
\usepackage{amsmath}
\usepackage{graphicx}
\usepackage{amsthm}
\usepackage{amssymb}
\usepackage{bm}
\usepackage{bbm}
\usepackage{amsfonts}

\usepackage{accents}
\newcommand\munderbar[1]{%
	\underaccent{\bar}{#1}}


\usepackage{svg}
\usepackage{booktabs}
\usepackage{caption}
\usepackage{blindtext}
%\renewcommand{\arraystretch}{1.2}
\usepackage{multirow}
\usepackage{float}
\usepackage{rotating}
\usepackage{mathtools}
\usepackage{chngcntr}

% TikZ stuff

\usepackage{tikz}
\usepackage{mathdots}
\usepackage{yhmath}
\usepackage{cancel}
\usepackage{color}
\usepackage{siunitx}
\usepackage{array}
\usepackage{gensymb}
\usepackage{tabularx}
\usetikzlibrary{fadings}
\usetikzlibrary{patterns}
\usetikzlibrary{shadows.blur}

\usepackage[font=small]{caption}
%\usepackage[printfigures]{figcaps}
%\usepackage[nomarkers]{endfloat}


%\usepackage{caption}
%\captionsetup{justification=raggedright,singlelinecheck=false}

\usepackage{courier}
\usepackage{verbatim}
\usepackage[round]{natbib}

\bibliographystyle{plainnat}

\definecolor{red1}{RGB}{128,0,0}
%\geometry{verbose,tmargin=1.25in,bmargin=1.25in,lmargin=1.25in,rmargin=1.25in}
\geometry{verbose,tmargin=1in,bmargin=1in,lmargin=1in,rmargin=1in}
\usepackage{setspace}

\usepackage[colorlinks=true, linkcolor={red!70!black}, citecolor={blue!50!black}, urlcolor={blue!80!black}]{hyperref}

\let\oldFootnote\footnote
\newcommand\nextToken\relax

\renewcommand\footnote[1]{%
	\oldFootnote{#1}\futurelet\nextToken\isFootnote}

\newcommand\isFootnote{%
	\ifx\footnote\nextToken\textsuperscript{,}\fi}

%\usepackage{esint}
\onehalfspacing

%\theoremstyle{remark}
%\newtheorem{remark}{Remark}
%\newtheorem{theorem}{Theorem}[section]
\newtheorem{assumption}{Assumption}
\newtheorem{proposition}{Proposition}
\newtheorem{proposition_corollary}{Corollary}[proposition]
\newtheorem{lemma}{Lemma}
\newtheorem{lemma_corollary}{Corollary}[lemma]

\begin{document}
	
\title{Creating Creative Destruction: Endogenous Growth with Employee Spinouts and Non-compete Agreements}

\author{Nicolas Fernandez-Arias} 
\date{\today \\ \small
	\href{https://drive.google.com/file/d/1gu4CT1ft4LY4MsKKgluxb8Gu_YoP8DLD/view?usp=sharing}{Click for most recent version}}
\maketitle


%\setcounter{tocdepth}{2}
%\tableofcontents

\begin{abstract}
	I study the effect of non-compete agreements (NCAs) on aggregate productivity growth. I first develop an augmented quality ladders model of endogenous growth with NCAs and within-industry employee spinouts (WSOs). I then assemble a new dataset of VC-funded startups matched to the previous employers of their founding members. I find a statistically and economically significant relationship between corporate R\&D and employee spinout formation which quantitatively can account for 75\% of employee departures to WSOs in the data. I calibrate the model to match the micro estimates, aggregate moments and estimates from the literature. According to the calibrated model, reducing all barriers to the use of NCAs increases welfare by 3\% in consumption equivalent terms. Blanket R\&D subsidies can reduce growth and welfare by misallocating R\&D to entrants engaging in creative destruction. The optimal policy is a combination of R\&D subsidies targeted at own-product innovation and a ban on the use of NCAs.
\end{abstract}

\section{Introduction}

Knowledge spillovers and firm entry are both major contributors to aggregate productivity growth. Firm entry in turn is often the result of knowledge spillovers from existing firms. In particular, within-industry employee spinouts (WSOs) -- new firms founded by former employees of incumbent firms in the same industry -- often take advantage of knowledge gained at previous employers. Figure \ref{fairchild_spinouts} shows the many direct and indirect spinouts of Fairchild Semiconductor, one of the first leading semiconductor firms of Silicon Valley -- itself a spinout of Shockley Laboratories, another semiconductor firm. Although Fairchild was founded in the 1950s, its list of spinouts includes some of the most well-known modern firms in the industry, such as Intel and AMD. 

\begin{figure}	\phantomsection
	\center
	\includegraphics[scale = 0.77]{../figures/fairchildren_early.png}
	\caption{Direct and indirect spinouts of Fairchild Semiconductor}
	\label{fairchild_spinouts}
\end{figure}


To avoid the possibility of competition from such firms, incumbents may reduce their investment in R\&D and other forms of costly knowledge creation. Alternatively, they may take steps to prevent WSOs directly, mitigate the disincentive to R\&D restricting productivity-enhancing knowledge spillovers. The most salient example of this kind of effort is the non-compete agreement (NCA), an employment contract which precludes the employee from founding a competing firm after ceasing his or her current employment until a prespecified amount of time has passed. Given the aforementioned tradeoff, it is not clear what the effect of NCAs is on aggregate productivity growth, nor is it clear how the answer to this question depends on structural parameters that may be different in different locations, industries or time periods. Further, from a normative perspective, it is natural to ask whether it is socially optimal to permit the free use of NCAs.

This paper is an attempt to provide quantitative answers to these questions. To do so, I first develop a tractable model of endogenous growth which augments the standard quality ladders framework in \cite{acemoglu_introduction_2009} to include WSOs and NCAs. I then construct and analyze a micro dataset of incumbent firms and startups, providing evidence for a causal relationship between parent firm R\&D and subsequent startup formation and find suggestive evidence of an economically meaningful causal relationship. The model is then calibrated using aggregate statistics and the microeconomic relationship between R\&D and employee entrepreneurship. Using the calibrated model, I study the effect of varying the barriers to enforcement of NCAs and describe the model-implied optimal policy. I find that eliminiation of all barriers to NCAs can increase welfare by approximately 3\% in consumption-equivalent terms. R\&D subsidies can have the counterintuitive effect of reducing growth by misallocating R\&D labor to entrants instead of incumbents.\footnote{This stems in part from an assumption of a fixed stock of R\&D labor. In a fuller model, this mechanism would simply dull the growth-enhancing effect of R\&D subsidies. I plant to extend the model in that direction.} R\&D subsidies targeted at own-product innovation can work well in tandem with a ban on NCAs, but are difficult if not impossible to implement in practice.

The model consists of a standard general equilibrium model of endogenous growth with creative destruction, augmented to include employee entrepreneurship and NCAs. The model takes as given that, absent NCAs, R\&D employees eventually gain the knowledge to form a competing WSO. Given the R\&D wage, this reduces the incentive for R\&D. However, in equilibrium, R\&D workers accept a lower wage, as they internalize the expected future profits WSOs they form. In this sense they "pay" ex-ante for the damage they will cause. Whether their payment is sufficient to fully mitigate the disincentive therefore depends on whether the value gained by the founder of the WSO is larger than the value lost by the incumbent. In fact, NCAs are used only if WSO formation does not maximize the ex-ante joint value of the employment relationship. 

In order for the model to generate a role for NCAs, there must be some other contracting friction or in the employer-employee relationship.  Otherwise, the incumbent could offer to buy any WSOs and shut them down. Ex-ante, the employee accepts a lower wage and the firm conducts R\&D as though it had imposed an NCA (and pays a higher wage). private information concerning the quality of the idea, disagreements between the employee and the employer concerning the idea's quality, and the lack of commitment power on the part of the employee (i.e., the employee cannot commit not to implement the idea even after notionally selling it to his employer). I leave the particular friction unmodeled, simply assuming that there is no market in which WSOs can be sold to the incumbent firm that generated them.\footnote{In future work, I plan to explore this area further.}

The result is a model in which WSOs expand the innovation possibilities frontier of the economy while having an ambiguous effect on equilibrium innovation and productivity growth. The freedom to use NCAs, which prevent WSOs, also may increase or decrease the equilibrium growth rate. To go further, some discipline needs to be imposed on the model parameters.

To do so, I assemble a dataset of parent firms and startups founded by their employees by combining Compustat data on publicly traded firms and Venture Source data on VC-funded startups and their founders. While the Venture Source data has limitations, it is the only dataset of startups with information on the most recent employer of the startup's key employees. Matching these data is somewhat challenging since there are no company identifiers across datasets: the match must be done by name only. This is non-trivial since companies go by different names. I solve this problem by using string matching techniques (e.g., regular expressions), Compsutat data on subsidiaries and finally the merchant-mapper tool by Alternative Data Group, a startup that links credit card transactions data to firms using machine learning (itself a spinout of 1010 Data). I define a startup as a spinout if its CEO, CTO, President, Chairman or Founder (1) was most recently employed at a firm in Compustat and (2) joined the startup in its first three years. Using this definition, I identify approximately 3,000 WSOs in the data. Finally, I match this dataset to data on all US patents taken from the NBER-USPTO patent database.

Figures \ref{figure:scatterPlot_RD-Founders_dIntersection} and \ref{figure:scatterPlot_RD-FoundersWSO4_dIntersection} illustrate the primary motivation for this paper: firm-level R\&D is associated with subsequent employee entrepreneurship. To be precise, the x-axis shows firm-level average R\&D spending over periods $t,t-1,t-2$ and the y-axis shows firm-level average yearly number of employee founders from that firm in $t+1,t+2,t+3$. Both of these variables are then purged sequentially of their firm- and state-industry-age-year means. Figure \ref{figure:scatterPlot_RD-FoundersWSO4_dIntersection} confirms this when restricting attention to employee entrepreneurship in the same 4-digit industry of the initial employer. 

\begin{figure}[]
	\centering
	\includegraphics[scale= 0.5]{../empirics/figures/scatterPlot_RD-Founders_dIntersection.pdf}
	\caption{Scatterplot of average yearly founder counts in $t+1,t+2,t+3$ versus average yearly R\&D spending in $t,t-1,t-2$.}
	\label{figure:scatterPlot_RD-Founders_dIntersection}
\end{figure}

\begin{figure}[]
	\centering
	\includegraphics[scale= 0.5]{../empirics/figures/scatterPlot_RD-FoundersWSO4_dIntersection.pdf}
	\caption{Scatterplot of average yearly founder counts (restricted to same 4-digit NAICS industry) in $t+1,t+2,t+3$ versus average yearly R\&D spending in $t,t-1,t-2$.}
	\label{figure:scatterPlot_RD-FoundersWSO4_dIntersection}
\end{figure}

However, a simple regression of number of employee spinouts on lagged R\&D spending suffers from omitted variable bias, as factors such as changes in demand or changes in technological investment opportunities likely affect both variables in the same direction. To control for this, I use firm, state-year, NAICS 4 digit industry-year (at 4-digit NAICS level), and firm age fixed effects, as well as firm-specific controls, such as employment, assets, Tobin's Q, and citation-weighted patents. The resulting estimates vary by specification obut are typically statistically and economically significant. According to these estimates, R\&D can account for roughly 75\% of employee departures to WSOs in the data. 

I next calibrate the model using the estimates above as well as aggregate statistics and growth accounting estimates from \cite{garcia-macia_how_2019} and \cite{klenow_innovative_2020}. I also choose some parameters from the literature. I use the calibrated model to study the effect on productivity growth and welfare of varying the cost of using NCAs. As stated previously, I find that welfare rises 3\% in consumption-equivalent terms. I discuss how these results depend on the parameters and, via the calibration, on the value of the targeted moments. I also exhibit an alternative calibration which returns the opposite conclusion. 

Finally, I discuss other policies that could improve welfare in this context. I consider R\&D subsidies, both overall and targeted specifically at own innovation by incumbent firms; a tax on creative destruction; and the interaction of these policies. Two interesting findings emerge. First, untargeted R\&D subsidies (that is, that apply to creative destruction as well as own-product innovation) can have the unintended effect of shifting R\&D to entrants, and potentially even inducing the use of NCAs. The latter occurs because incumbents prefer to pay R\&D employees through subsidized wages rather than implicitly through future spinout formation (whose cost to the incumbent is not subsidized). The former is due to a similar reason. Second, targeted R\&D subsidies avoid this problem and, in combination with a ban on the use of NCAs, can achieve a sort of "first best" where the incumbent does enough R\&D while still allowing for spinouts to enter. This might be a difficult policy to implement, and I discuss some of the potential barriers. I close with suggestions for future work. 

\paragraph{Related literature}

Some work has attempted to answer this question directly using empirical methods. Papers in this literature have typically used either cross-sectional and/or longitudinal variation in the state-level enforcement of non-competes.\footnote{Sometimes this variation is argued to be exogenous, either due to legislative error as in \cite{marx_mobility_2009} and \cite{marx_regional_2015}, or due to unexpected judicial precedent as in \cite{jeffers_impact_2018}. Often there is a control industry that is believed to be unaffected by the variation in CNC enforcement policy (e.g. law firms are typically exempt from CNC restrictions).} The results are inconclusive and suggest an important tradeoff between entry of spinouts and investment by incumbent firms. \cite{stuart_liquidity_2003} find more local  entrepreneurship in response to local IPO (a "liquidity event") in regions not enforcing CNCs. \cite{marx_mobility_2009} finds that inventor mobility declines in response to an increase in non-compete enforcement. \cite{samila_venture_2010} finds that an increase in VC funding supply increases entrepreneurship more in states without non-compete restrictions, using an IV design. \cite{garmaise_ties_2011} finds that, in states where CNCs are more enforceable, managers are less mobile, have lower compensation, and invest less in their human capital, to the point of offsetting increased investments by the firm. On the other hand, \cite{conti_non-competition_2014} finds evidence that non-compete enforceability leads to incumbent firms pursuing riskier R\&D projects. \cite{colombo_does_2013} finds evidence that easier spinout formation -- proxied by access to finance -- leads to a reduction in incumbent firm knowledge investments.  Most recently, \cite{jeffers_impact_2018} uses data on influential state-level court precedents matched with LinkedIn data and finds that enforcement indeed reduces spinout formation while increasing capital investment by incumbent firms. Finally, \cite{marx_regional_2015} finds that CNC enforcement leads to inventor mobility out of the state, suggesting that differences in outcomes could be in part due to reallocation. 

Theoretical work has also explored this question. As mentioned previously, \cite{franco_spin-outs_2006} develops a model in which employees learn from their employers and use this knowledge to form spinouts. They emphasize the "paying for knowledge" effect, whereby employees implicitly pay for the knowledge they take from the parent firm through lower equilibrium wages. Importantly, they assume spinout firms do not steal business from their parents: the only effect of a spinout on the parent firm is a reduction in the price of the output good, which the parent firm is assumed not to take into account. This, combined with the "paying for knowledge" mechanism, ensures that the competitive equilibrium allocation is Pareto efficient, even without resorting to elaborate labor contracts.

\cite{franco_covenants_2008} studies a two-period, two-region model with employee spinouts in which the region which does not enforce CNCs initially lags but eventually overtakes the region in which CNCs are enforced. In the first period, entry is more valuable in the enforcing region. But in the second period, spinouts enter in the non-enforcing region, there is Cournot competition with parent firms in the product market, and output increases relative to the enforcing region. The analysis emphasizes how asymmetric information about whether an employee has learned leads some firms in the non-enforcing region to allow spinouts (assuming firms cannot commit to wage backloading). This can be taken as a rough microfoundation of my assumption that labor contracts are "simple" in  a non-enforcing region: just a wage, with no attempts at retention in the case of learning. Relative to this study, my analysis considers a fully dynamic model rather than two-period model. In addition, I emphasize the role of R\&D investment in spawning spinout firms.

\cite{shi_restrictions_2018} uses a rich model of contracting disciplined by data on executive non-compete contracts to study the effect of non-competes on executive mobility and firm investment. She finds that the optimal policy is to somewhat restrict the permitted duration of CNCs. Her approach allows her to study the optimal contracting problem in more detail than in mine. However, she is mainly interested in an environment where the firm's productivity is embodied in the worker and where the concern is poaching, not spinout formation. Also, her calibration considers firm investment in capital expenditures, whereas I am interested in innovative investment in R\&D.

\cite{baslandze_spinout_2019}, the study closest to this paper, studies the effect of spinout entrepreneurship on entry and growth. She also uses a GE model of endogenous growth with employee spinouts, using Compustat and NBER-USPTO patent data to discipline the analysis. She finds the optimal policy is to ban NCAs. However she is focused in her framework on the harm from losing a valuable employee rather than the harm from competition with the parent firm. My paper focuses instead on creative destruction of the parent firm using knowledge rather than the loss of productivity from losing valuable employees.  The other key difference is that I model the use of NCAs while her analysis assumes that they are used when available. To my knowledge, mine is the first general equilibrium model of endogenous growth to have this kind of feature.

\cite{babina_entrepreneurial_2019} find evidence of a causal relationship from corporate R\&D spending to employee spinout formation. My empirical analysis confirms their findings on a subset of firms particularly connected with productivity growth, VC-funded startups. Together, they motivate the use of a model like the one developed in this paper.


\section{Model}\label{sec:model}

\subsection{Individual endowments and preferences}

The model is in continuous time, starting at $t = 0$. The representative household has CRRA preferences over consumption, given by\footnote{There is no expectation operator use there is no aggregate uncertainty in this setting (more on this in later sections).}
\begin{align}
U_t &= \int_0^{\infty} e^{-\rho s} \frac{C(t+s)^{1-\theta} - 1}{1-\theta} ds \label{preferences}
\end{align}

In each period $t \ge 0$, the household is endowed with $\bar{L}_{RD} \in (0,1)$ units of R\&D labor as well as $1 - \bar{L}_{RD}$ units of production labor which is used in the production of intermediate and final goods. The labor resource constraints are 
\begin{align}
L_{RD} &\le \bar{L}_{RD} \label{labor_resource_constraint2} \\
L_{\text{production}} &\le 1 - \bar{L}_{RD} \label{labor_resource_constraint} 
\end{align}

\subsection{Production of intermediate and final goods} \label{subsec:staticproduction}

\paragraph{Intermediate goods} There is a continuum of intermediate goods $j\in [0,1]$ which at any given time $t$ exist in a finite set of qualities $\{q_{jti}\}_{0 \le i \le I_{jt}}$. Define $\bar{q}_{jt} = \max_{0 \le i \le I_{jt}} \{q_{jti}\}$ as the \emph{frontier} quality of good $j$, and refer to the producer of this good as \emph{incumbent} $j$. Intermediate goods $j$ of any quality are produced according to the production function
\begin{align}
k_{jti} = H(\ell_{jti};Q) &= Q \ell_{jti} \label{intermediate_goods_production}
\end{align}
where $\ell_{jti} \ge 0$ is the labor input and $Q_t = \int_0^1 \bar{q}_{jt} dj$ is the average frontier quality level in the economy.\footnote{The linear scaling with the aggregate economy $Q$ is to ensure a balanced growth path (BGP), given that the total quantity of labor stays the same over time. It is analogous to assuming a constant marginal cost in a model where the final good, rather than labor, is the input of intermediate goods production. \textbf{[Note about alternative formulation with no scaling in $Q_t$ here -- see Acemoglu textbook for equivalence results.]}} Each quality of good $j$ is produced by a firm which has a monopoly on the production of that quality of good $j$. There is no storage of intermediate goods.

\paragraph{Final good}

The final good $Y$ is produced competitively using production labor and intermediate goods. Its production technology is given by \footnote{Intermediate goods are aggregated in a CES form with an elasticity of substitution greater than 1, rather than the Cobb-Douglas form in e.g., \cite{grossman_quality_1991} and \cite{baslandze_spinout_2019}. This reduces the complexity of the firm problem. In those models, Cobb-Douglas implies that expenditure on each intermediate good is constant in quality. This requires limit pricing to be explicitly modeled, otherwise increasing the price always increases profits and the firm problem is not well-defined. To model limit pricing, one must track the gap between leader and follow in each good $j$, adding a state variable to the firm problem and to the aggregation of the model. In the current setup, by contrast, expenditure is decreasing in the price of the intermediate good, so even if one can abstract from limit pricing (as is the case in this model, see next footnote), intermediate goods firms have a constant optimal markup. Reducing the complexity in this way allows me to add more features to the model while maintaining a transparent analysis.}
\begin{align}
Y_t = F(L_{Ft},\{q_{jti}\},\{k_{jti}\}) &= \frac{L_{Ft}^{\beta}}{1-\beta} \int_0^1 \Big(\sum_{i = 0}^{I_{jt}} q_{jti}^{\frac{\beta}{1-\beta}} k_{jti} \Big)^{1-\beta} dj \label{final_goods_production}
\end{align}

where $k_{jti} \ge 0$ is the quantity used of intermediate good $j$ of quality $q_{jti}$. This specification assumes that different qualities of good $j$ are perfect substitutes in final goods production, and the intermediate goods production technology (\ref{intermediate_goods_production}) has constant returns to scale. Bertrand competition in the intermediate goods market then implies that in equilibrium only the highest quality producer of good $j$ will produce a positive amount.\footnote{Usually, this implies that the leader will use limit pricing unless each quality improvement is sufficiently large. To avoid this complication, I use an assumption that ensures monopolistic competition pricing regardless of the step size, borrowed from \cite{akcigit_growth_2018}.} This allows for a simpler representation,
\begin{align}
Y = F(L_F,\{\bar{q}_j\},\{\bar{k}_j\}) &= \frac{L_F^{\beta}}{1-\beta} \int_0^1 \bar{q}_j^{\beta} \bar{k}_j^{1-\beta} dj  \label{eq_final_goods_production}
\end{align}

Finally, there is no storage technology for the final good and its price is normalized to 1 in every period. 

\subsection{Innovation}\label{subsec:innovation}

There are three types of innovation in this economy. Incumbents can expend R\&D to improve on their own product. I refer to this as \textit{own-product innovation} or OI, following \cite{garcia-macia_how_2019} and \cite{klenow_innovative_2020}. The other two types of innovation involve \textit{creative destruction}, or CD. R\&D by incumbents can lead to the formation of employee spinouts, which in the model are simply new higher-quality incumbents producing the same good. In addition, entrants can perform targeted R\&D on each intermediate good. 

An innovator on the frontier quality of good $j$ becomes the new incumbent of good $j$ and holds a perpetual patent on the production of good $j$ of quality $\bar{q}_{jt} = \lambda \lim_{t' \uparrow t} \bar{q}_{jt'}$, where $\lambda > 1$ is the exogenous quality ladder step size. That is, the new frontier quality improves on the previous frontier quality by a factor $\lambda$. Importantly, the perpetual patent \textit{does not} prevent entrants or WSOs from leapfrogging the incumbent with an even better quality product.

\paragraph{No catch-up innovation}

I assume that only incumbents can perform OI. When an incumbent is overtaken by an entrant or spinout, she loses access to the OI R\&D technology and therefore cannot use it to "catch up" to the frontier. One possible interpretation is that learning by doing means the current producer of a product has unique insights into how to improve on it. From a mathematical standpoint, the assumption dramatically simplifies the analysis.\footnote{This is standard in quality ladders models with OI and is usually not made explicit. Without it, the incumbent problem would have an additional state variable (since falling away from the frontier is no longer an absorbing state) and an additional distribution would need to be tracked (the number of incumbents with the technology to produce each infra-frontier good $j$. The setting is so intractable that many papers which focus on this mechanism, such as \cite{aghion_competition_2005}, also make simplifying assumptions analogous to mine. For a producer $n$ steps behind the frontier, the assumption has bite if the expected discounted present value of the cost of $n + 1$ innovations using the OI innovation technology is lower than the expected cost of one innovation using the freely available entrant technology (described in Section \ref{subsubsec:entrants}). For certain parameter values, this inequality will hold for small $n$.}  

\paragraph{Scaling of cost of R\&D}

A given amount of R\&D applied to improving a product leads to a given Poisson intensity of innovating on a product, regardless of the quality of the product. However, when directed at a higher quality product, a given amount of R\&D requires more R\&D labor input. Specifically, one unit of R\&D costs $\frac{\bar{q}_{jt}}{Q_t}$ units of R\&D labor when directed at a product with frontier quality $\bar{q}_{jt}$. This scaling assumption is economically natural, as higher quality products require more human capital to improve. It is also necessary for the existence of a balanced growth path in this model.\footnote{The cost of R\&D could scale up faster than $\frac{\bar{q}_{jt}}{Q_t}$, which would imply that higher quality products grow slower. In the current setup this would violate BGP since there is no stationary distribution of product quality. Adding a fixed cost, however, would induce such a stationary distribution, because it creates a lower exit barrier. For more discussion of this type of question, see \cite{gabaix_power_2009} or \cite{acemoglu_innovation_2015}.}

\subsubsection{Own-product innovation by incumbents} \label{subsubsec:OI}

In return for $z_{jt}$ units of R\&D, the incumbent receives a Poisson intensity of $\chi z_{jt}$ of innovating on good $j$, where $\chi > 0$ is an exogenous parameter representing the incumbent's R\&D productivity. This implies an arrival rate of incumbent innovations of 
\begin{align}
	\tau_{jt} &= \chi z_{jt}
\end{align}

Note that incumbent R\&D exhibits constant returns to scale. This is necessary to have analytical solutions, but the model can be easily extended to the more realistic case of decreasing returns and solved numerically.


\subsubsection{Creative destruction by entrants} \label{subsubsec:entrants}

For each good $j$ there is a unit mass (normalization) of entrants indexed by $e \in [0,1]$.\footnote{For expositional simplicity I assume this, but it is actually an easily provable equilibrium result.} In return for $\hat{z}_{jet}$ units of R\&D, an entrant receives a Poisson intensity of $\hat{z}_{jet} \hat{\chi} \hat{z}_{jt}^{-\psi}$ of innovating on good $j$, where $\hat{z}_j = \int_0^1 \hat{z}_{je} de$ denotes aggregate entrant effort on improving good $j$. Aggregating over $e$, this implies an arrival rate of entrant innovations to good $j$ of 
\begin{align}\label{model:entrantsInnovationTechnology}
\hat{\tau}_{jt} &= \hat{\chi} \hat{z}_{jt}^{1-\psi}
\end{align}

The choice $\psi > 0$ introduces decreasing returns at the level of good $j$. It represents a \textit{congestion} externality in the entrant innovation technology. This is similar to the congestion externality present in search and matching models. Intuitively, due to a lack coordination, entrants attempt similar approaches to solve the same problem. This duplication of effort reduces the overall returns to entrant R\&D when considered at the level of good $j$.

\subsubsection{Creative destruction by employee spinouts}\label{subsubsec:generation_of_spinouts}

R\&D by incumbents leads to knowledge spillovers in the form of employee spinouts from R\&D labor. Incumbents can prevent spinouts by imposing a noncompete agreement on their R\&D labor, a decision which can be made instant-by-instant. The representative household takes this into account whether a noncompete is imposed when allocating its R\&D labor (also instant-by-instant, as it is a perfectly competitive market) and therefore demands a higher wage when a noncompete is imposed. I will return to this in the next section.

To make this precise, when incumbent $j$ conducts $z_{jt}$ units of R\&D effort, she faces a Poisson intensity of spawning a WSO, given by 
\begin{align}
	\tau^S_{jt} &= (1-\mathbbm{1}^{NCA}_{jt}) \nu z_{jt} \label{def:tau_S}
\end{align} 
where $\mathbbm{1}^{NCA}_{jt} = 1$ if and only if an NCA is used in that instant. 

Spinouts from incumbents of quality $q$ have the ability to produce the same good with quality $\lambda q$. They immediately become the new incumbent and, recalling the "no catch-up" assumption in Section \ref{subsec:innovation}, the previous incumbent's profits go to zero forever after. The exogenous parameter $\nu \ge 0$ is a reduced form encoding the rate at which corporate R\&D increases the likelihood of replacement by a WSO.

In the context of this model, this specification amounts to assuming that the rate of spinout generation of a unit of R\&D labor is inversely proportional to the relative quality of the good to which it is applied, $\frac{\bar{q}_{jt}}{Q_t}$. Because R\&D labor demand is $\frac{\bar{q}_{jt}}{Q_t} z_{jt}$, the factors cancel out and the rate of spinout formation is linear in $z_{jt}$. In this way, this is the specification of spinout generation that is analogous to the specification of the cost of R\&D.

\paragraph{No idea stealing} WSO entry does not directly reduce the rate at which incumbent R\&D results in successful OI. Instead, WSOs happen when additional, independent Poisson process with rate $\nu z_j$ has an arrival. The interpretation is that WSOs in this model do not steal ideas that otherwise would have been implemented by the parent firm. Rather, when unbound by NCAs, R\&D labor generates \textit{additional} innovations which the household uses to displace the incumbent firm.\footnote{This assumption has important consequences for the private and social usefulness of NCAs. If spinouts steal ideas, they are less bilaterally valuable and NCAs are more useful privately. For the same reason, WSOs are less socially valuable and therefore so are NCAs. This is an interesting topic for further research.}

\paragraph{Direct cost of using NCAs}\label{paragraph:nca_cost}

When incumbent $j$ imposes an NCA on $z_j$ units of R\&D, she must pay a flow cost $\kappa_{c} \nu V(j,t|\bar{q}_j) z_j$ units of the final good, where $V(j,t|q)$ is the value of incumbent $j$ at time $t$ given quality $q$. Given (\ref{def:tau_S}), incumbent $j$ overall pays
\begin{align}
	\textrm{NCA cost}_{jt} &= \tau^S_{jt} \kappa_c V(j,t|q) \label{def:nca_cost}
\end{align}

The NCA enforcement cost reflects the direct cost using an NCA. Even if there are no technical restrictions on what kinds of NCAs are valid, determining competition between businesses may be expensive. Moreover, many jurisdictions do, in fact, impose such restrictions, and resources can be invested to prove that the conditions of those restrictions do not apply. Overall, it seems plausible that investing resources increases the likelihood of a successful enforcement of an NCA.  Note also that a value of $\kappa_c = \infty$ can be interpreted as ban on the use of NCAs.

The factor $V(j,t|q)$ means that the cost of enforcing NCAs increases in the equilibrium value of the incumbent firm. The economic justification is that valuable incumbency positions require more resources to protect via NCAs. In the context of the model, this specification means that the cost of enforcing an NCA on a given unit of R\&D labor is proportional to both the value of the WSOs that labor will generate in the absence of an NCA, and the expected loss of incumbent value from an absence of NCAs. In addition to economics, the assumption improves model tractability by simplifying the analysis of the optimal noncompete policy (see Section \ref{subsubsec:dynamic_equilibrium_original_solution}).\footnote{BGP only requires that\begin{align*}
	\textrm{NCA cost}_{j,t,q} &\propto \tau^S_{jt} q
	\end{align*}}

\subsubsection{Entry cost}

In addition to the R\&D costs of innovation, entrants and spinouts must pay an entry cost of $\kappa_{e} V(j,t|\lambda q)$ units of the final good in order to enter once they have successfully innovated, for exogenous $\kappa_e \in [0,1)$. This reduced form represents non-R\&D expenditures required by CD innovation but not by OI innovation. Examples of such expenditures could be firm set-up or marketing costs for a new product or brand.

The scaling with $V(j,t|\lambda q)$ parallels the scaling of the NCA cost in Section \ref{paragraph:nca_cost}. This is economically reasonable given the interpretation above. It is also important for matching the data, in particular the R\&D / GDP ratio and the entry rate.\footnote{This relates to \cite{comin_rd_2004}, which finds in a similar model that the model-implied expected payoff to aggregate R\&D is much higher than aggregate R\&D spending in the data. This is inconsistent with free entry. His interpretation is that R\&D is in fact responsible for only a small part of innovation, which reduces the model-implied payoff of aggregate R\&D spending. Instead, my interpretation, via this model, is that there are additional non-R\&D expenditures associated with innovation. The calibration sets these expenditures so that the return to innovative investment is equal to the market rate.} In addition, as before, scaling with $V(j,t|\lambda q)$ enables a simple analytical solution to the model, for the same reason as the NCA cost assumption. It could easily be extended in the same way.\footnote{As in Section \ref{paragraph:nca_cost}, BGP only requires that 
\begin{align*}
	\textrm{Entry cost}_{j,t,q} &\propto q
\end{align*}}

\subsubsection{Competitive financial intermediary}

The representative household owns a competitive financial intermediary which in turn owns all firms in the economy and remits their profits back to the representative household. Individual firms in the economy maximize profits subject to the household's risk-free discount rate (there is no collusion due to common ownership). When the representative household receives a shock in good $j$ that allows it to form a spinout, it sells the spinout to the financial intermediary at full private value (i.e. discounting the spinouts profits at the same risk-free discount rate). The financial intermediary takes the entry of the spinout as given, and therefore is willing to pay this value even though the entry of the spinout reduces the value of an existing incumbent.\footnote{The purpose of this construction is to avoid having to assume that the representative household does not take into account the loss of value of the incumbents it owns when spinouts enter.}\footnote{An alternative setup instead of a representative household is a continuum of households each consisting of a continuum of agents who fully insure one other against idiosyncratic risk. What is essential is that households be fully insured (to avoid having to deal with modeling financial frictions) and that they not internalize the business-stealing effects of the spinouts.} 

\subsection{Equilibrium}\label{subsec:decentralized_equilibrium}

I will solve for a BGP equlibrium with a constant growth rate of output ($g_t = g$) as well as constant innovation effort by incumbents ($z_{jt} = z$) and entrants ($\hat{z}_{jet} = \hat{z})$. I will refer to this class of BGP as a \textit{symmetric BGP}.\footnote{In principle I allow $x_{jt}$ to vary across $j$ and over time $t$; Propositions \ref{proposition:purstrategyeq:positiveOI} and \ref{proposition:purstrategyeq:zeroOI} below show that $x_{jt} = x$ in a symmetric BGP except on a knife-edge in the parameter space.}\footnote{One could relax the assumption that $z_{jt} = z$ as long as $\int_0^1 z_{jt} \frac{\bar{q}_{jt}}{Q_t}dj$ is constant on the BGP. This induces a continuum of BGPs which have the same aggregate variables (except higher moments of the quality distribution, which are irrelevant for the equilibrium), since this term appears in the growth accounting equation analogous to (\ref{eq:growth_accounting}) and the R\&D labor market clearing condition (\ref{eq:RD_labor_market_clearing}). As such, this kind of multiplicity does not affect aggregate growth or prices and is a technical artefact of the assumed CRS R\&D technology for incumbents. I therefore assume that $z_{jt} = z$ because it simplifies the algebra.} Symmetric BGPs are a natural type of equilibrium given the symmetric setup of the model. In this case, there always exists a unique symmetric BGP, except on a knife-edge in which case there is a continuum of symmetric BGPs.

To do so, first I characterize the static equilibrium given a profile of frontier qualities $\{ \bar{q}_{j}\}$. Next, using the assumption that $\hat{z}_{jet} = \hat{z}, z_{jt} = z$, I prove that there exists $\tilde{V} > 0$ such that $V(j,t|\bar{q}_{jt}) = \tilde{V} \bar{q}_{jt}$ is the value of incumbent $j$ of quality $\bar{q}_{jt}$ at time $t$. This in turn implies that R\&D wages scale linearly with aggregate productivity $Q_t$ as well. Given the linear scaling with $\bar{q}_{jt}$ and $Q_t$, respectively, factors $\bar{q}_{jt}$ and $Q_t$ drop out of the equilibrium equations. The resulting system is straightforward to solve recursively.

\subsubsection{Static equilibrium}

In this section, I omit the dependence on $t$ of all equilibrium variables. In addition, since only the frontier quality is produced in equilibrium, I will drop the bar notation and refer to the frontier good's quality and quantity by $q_j$ and $k_j$, respectively.

Final goods producer optimization implies the following inverse demand function for intermediate goods, 
\begin{align*}
p_j &= L_F^{\beta} q_j^{\beta} k_j^{-\beta}	
\end{align*}

To continue computing the equilibrium of the model, the market structure for intermediate goods must be specified. 

\paragraph{Intermediate goods market structure} The following setup is drawn from \cite{akcigit_growth_2018}. Within each good $j$, intermediate goods producers play a two-stage Bertrand competition game. In the first stage, participants bear a cost of $\varepsilon > 0$ units of the final good in exchange for a right to compete in the second stage. Then, in the second stage, the engage in Bertrand competition. Optimal pricing under Bertrand competition in the second stage implies that all producers not on the frontier will earn zero profits. By backward induction, they do not pay the entry cost in equilibrium, and the leader has a second-stage monopoly over good $j$.\footnote{Without this assumption, there is limit pricing, and the markup charged by the technology leader in good $j$ would depend on his gap relative to the next laggard, e.g. \cite{baslandze_spinout_2019} or \cite{aghion_competition_2005}, only equating to the monopolistic competition markup for large enough gaps.} Different good $j$ incumbents compete against each other under monopolistic competition.

This market structure implies that the incumbent for each good $j$ can effectively ignore lower quality producers of good $j$. She maximizes profits according to
\begin{align}
\pi(q_j) = \max_{k_j \ge 0} \Big\{ L_F^{\beta} q_j^{\beta} k_j^{1-\beta} - \frac{\overline{w}}{Q} k_j \Big\} \label{incumbent_profit}
\end{align}

where $\overline{w}$ is the equilibrium final goods / intermediate goods wage.
This yields optimal pricing, labor demand and production of intermediate goods,
\begin{align}
k_j &= \Big[ \frac{(1-\beta) Q}{\overline{w}} \Big]^{1/\beta}L_F q_j  \label{optimal_k}\\
\ell_j &= k_j / Q \label{optimal_l}\\
p_j &= \frac{\overline{w}}{(1-\beta) Q} \label{optimal_p}
\end{align}

Substituting (\ref{optimal_k}) into the first-order condition for final goods firm optimal labor demand yields a closed form expression for the equilibrium wage $\overline{w}$:
\begin{align}
\overline{w} &= \tilde{\beta} Q \label{wbar} \\
\tilde{\beta} &= \beta^{\beta} (1-\beta)^{1-2\beta} \label{def_cbeta}
\end{align}

Substituting (\ref{optimal_k}) and (\ref{wbar}) into the expression for profit in (\ref{incumbent_profit}) yields
\begin{align}
\pi_j &= \overbrace{(1-\beta) \tilde{\beta} L_F}^{\mathclap{\tilde{\pi}}} q_j \label{profits_eq}
\end{align}

Substituting (\ref{optimal_k}) into (\ref{optimal_l}) and integrating $L_I = \int_0^1 l_j dj$ yields aggregate labor allocated to intermediate goods production,
\begin{align}
L_I &= \Big( \frac{1-\beta}{\tilde{\beta}} \Big)^{1 / \beta} L_F \label{intermediate_goods_labor}
\end{align}

and substituting (\ref{intermediate_goods_labor}) into the labor resource constraint (\ref{labor_resource_constraint}) yields
\begin{align}
L_F &= \frac{1 - \bar{L}_{RD}}{1 + \Big(\frac{1-\beta}{\tilde{\beta}}\Big)^{1/\beta}}
\end{align}

Output can be computed by substituting (\ref{optimal_k}) into (\ref{final_goods_production}), 
\begin{align}
Y = \frac{(1-\beta)^{1-2\beta}}{\beta^{1-\beta}} Q L_F \label{flow_output}
\end{align}

\subsubsection{Dynamic equilibrium}\label{subsubsec:dynamic_equilibrium_original_solution}

\paragraph{Household optimization}

The household takes as given for all $t \ge 0$ incumbent R\&D wages and NCA policies $\{w_{RD,jt}, x_{jt}\}$, entrant R\&D wages $\{\hat{w}_{RD,t}\}$, the production wage $\{\bar{w}_t\}$, interest rates $\{r_t\}$, and profits from the financial intermediary $\{\Pi_t\}$. 

The household problem is\footnote{There are also non-negativity constraints on all control variables $\{C(t), \ell_{RD,j} (t), \hat{\ell}_{RD,j}(t), L(t) \}_{t\ge 0}$, ommitted for clarity.}\footnote{The problem is formulated as a stochastic optimal control problem because it involves allocating R\&D labor to different goods $j$ depending on the random processes $\{\bar{q}_{jt}\}_{t \ge 0}, \{w_{RD,jt}\}_{t\ge 0}, \{x_{jt}\}_{t\ge 0}, \{V(j,t|\bar{q}_{jt})\}_{t\ge 0}$ for $j \in [0,1]$, However, in a symmetric BGP, there will be no uncertainty in the household's consumption stream because equilibrium compensation (including the value of WSOs generated) will be the same in all goods $j$ to satisfy market clearing. See Lemma \ref{lemma:RD_worker_indifference}.}

\begin{maxi*}[1]<b>
	{\substack{\{C(t) \}_{t \ge 0} \\ \{ L(t)  \}_{t \ge 0} \\ \{\ell_{RD,j}(t)\}_{j \in [0,1], t \ge 0} \\ \{\hat{\ell}_{RD,j}(t)\}_{j \in [0,1], t \ge 0}  }} {\mathbb{E} \int_0^{\infty} e^{-\rho t} \frac{C(t)^{1-\theta}-1}{1-\theta} dt}{}{}
	\addConstraint{ C(t)}{ \le \Pi_t + \bar{w}_tL(t)} 
	\addConstraint{ }{+ \int_0^1 \big( w_{RD,jt} + (1-x_{jt})(\frac{\bar{q}_{jt}}{Q_t})^{-1} \nu (1-\kappa_e) V(j,t|\lambda \bar{q}_{jt}) \big) \ell_{RD,j}(t) dj}
	\addConstraint{ }{+ \int_0^1 \hat{w}_{RD,t} \hat{\ell}_{RD,j}(t) dj}
	\addConstraint{\int_0^1 (\ell_{RD,j}(t) + \hat{\ell}_{RD,j}(t))dj}{ \le \bar{L}_{RD}} 
	\addConstraint{L(t)}{\le 1 - \bar{L}_{RD}}
\end{maxi*}

\normalsize


where $L(t) = L_I(t) + L_F(t)$ denotes production labor, $\ell_{RD,j}(t)$ denotes R\&D labor supplied to incumbent $j$, and $\hat{\ell}_{RD,j}(t)$ denotes R\&D labor supplied to entrants attempting innovation on good $j$.\footnote{Because the household's effective "taste" for working at a given incumbent $j$ depends on the expected DPV of WSOs formed after working there, it is necessary to explicitly model the household's R\&D labor allocation across goods $j$. This is not necessary for R\&D labor supplied to entrants, since they are all identical. However, I present the problem symmetrically.} The household consumes out of profits remitted by the intermediary $\Pi_t$ (which it takes as given), wages earned from production labor supply $\bar{w}_t L(t)$, wages earned from R\&D labor supply $\int_0^1 \big(w_{RD,jt} \ell_{RD,j}(t) + \hat{w}_{RD.t} \hat{\ell}_{RD,j}(t) \big) dj$, and earnings from sales of WSOs to the financial intermediary $ \int_0^1 (1-x_{jt}) (\frac{\bar{q}_{jt}}{Q_t})^{-1} \nu (1-\kappa_e) V(j,t|\lambda \bar{q}_{jt}) \big)\ell_{RD,j}(t) dj$.\footnote{To be more rigorous one would define a Poisson process for each $j$ denoted $N_{jt}$ which keeps track of the cumulative number of spinouts in line $j$. Then the budget constraint would have $\int_0^1 w_{RD,jt}\ell_{RD,j}(t) dj$ and a separate term $\int_0^1 (1-\kappa_e) V(j,t|\lambda \bar{q}_{jt}) dN_{jt} dj$, where arrival rate of $N_{jt}$ is $(\frac{\bar{q}_{jt}}{Q_t})^{-1} \nu \ell_{RD,j}(t)$. By the law of large numbers, this would yield the same expression as given above.} The last term is expected value of spinouts formed in good $j$, which occurs with intensity $(1-x_{jt}) (\frac{\bar{q}_{jt}}{Q_t})^{-1} \nu \ell_{RD,j}(t)$ and has value $(1-\kappa_e) V(j,t|\lambda \bar{q}_{jt})$, given that the entry cost $\kappa_e V(j,t|\lambda \bar{q}_{jt})$ must be paid to enter.\footnote{Technically this assumes that $\kappa_e < 1$. When I discuss entry taxes in Section \ref{subsec:cd_tax} and $\kappa_e + T_e \ge 1$, the value will be $\max\{0,(1-\kappa_e - T_e) \lambda \tilde{V} \}$.} 

Household optimization yields two types equilibrium conditions: an indifference condition on R\&D wages and an Euler equation. The former amounts to the condition that, if the worker supplies R\&D labor to all incumbents at all times, the expected compensation received from each incumbent is equal to the wage earned when supplying R\&D labor to entrants. The former is equal to the wage paid by the incumbent plus, if $x = 0$ (NCA not used), the expected future value of spinouts formed by the employee as a result of their employment at the incumbent. 


\begin{lemma}\label{lemma:RD_worker_indifference}
	In a symmetric BGP with $z > 0$, R\&D wages satisfy
	\begin{align}
	\hat{w}_{RD,t} &= w_{RD,jt} + (1-x_{jt}) (\frac{\bar{q}_{jt}}{Q_t})^{-1} \nu (1-\kappa_e) V(j,t|\lambda \bar{q}_{jt}) \label{eq:RD_worker_indifference}
	\end{align}
	for all $t \ge 0$ and $j \in [0,1]$.
\end{lemma}

\begin{proof}
	\textbf{[Probably just say it's proved in text.]}
	First note that $\hat{z} > 0$ in any symmetric BGP due to the Inada conditions on the entrant innovation technology given in  (\ref{model:entrantsInnovationTechnology}). If in addition $z > 0$, then $\ell_{RD,j}(t) > 0, \hat{\ell}_{RD,j}(t) > 0$ for all $j,t$. Optimality dictates that the household supplies R\&D labor only to jobs which provide the highest compensation. Therefore, in order to be consistent with household optimal labor supply, (\ref{eq:RD_worker_indifference}) must hold for all $t \ge 0$ and $j \in [0,1]$, 
\end{proof}

To derive the second condition -- the Euler equation --  technically one needs to add a market for a instantaneous risk-free bond, which in equilibrium is in zero net supply. I have not done this explicitly to avoid complicating the household's stated problem. Denote the interest rate for this bond by $r_t$, which in principle can be time-varying. The resulting household problem is standard\footnote{Because the household takes as given its ownership of the financial intermediary and this is the only asset that exists in positive net supply in the economy, there is no transversality condition associated with household optimization. However, this choice of modeling device does not actually expand the set of equilibria because any equilibrium which would have violated the household's transversality condition in the standard setup will violate finiteness of household utility in this set up.}  and gives rise to the Euler equation at each time $t \ge 0$, 
\begin{align}
\frac{\dot{C}(t)}{C(t)} = \frac{1}{\theta} (r_t - \rho) \label{eq:euler0} 
\end{align}



\paragraph{Incumbent optimization}

For all $t \ge 0$, incumbent $j$ takes as given the R\&D wage it must pay conditional on each choice of NCA policy, $w_{RD,jt}(\mathbbm{1}^{NCA})$, as well as the interest rate $r_t$ which it uses to discount future profits and the rate of creative destruction by entrants $\hat{\tau}(j,t|\bar{q}_{jt})$. I have introduced the conditional incumbent R\&D wages $w_{RD,jt}(\mathbbm{1}^{NCA})$ because at this point it is necessary to specify what wage the incumbent would have to pay given an off-equilibrium choice of $\mathbbm{1}^{NCA}_{jt}$, so that the equilibrium choice can be shown to be the optimal policy. To be consistent with the microfoundations of the perfect competition model, I assume that the incumbent expects to be able to hire as much R\&D labor as necessary at the going market compensation for R\&D labor, and none if a lower compensation is offered. This implies a result analogous to Lemma \ref{lemma:RD_worker_indifference} to $w_{RD,jt}(\mathbbm{1}^{NCA})$.

\begin{lemma}\label{lemma:RD_worker_indifference1}
	The incumbent takes as given the R\&D wage it must pay conditional on $\mathbbm{1}^{NCA}_{jt} = \mathbbm{1}^{NCA}$,
	\begin{align*}
		w_{RD,jt}(\mathbbm{1}^{NCA}) &= \hat{w}_{RD,t} - (1-\mathbbm{1}^{NCA}) (\frac{\bar{q}_{jt}}{Q_t})^{-1} \nu (1-\kappa_e) V(j,t|\lambda \bar{q}_{jt})
	\end{align*}
\end{lemma}

\begin{proof}
	Follows from the discussion.
\end{proof}

The value of an incumbent must satisfy a Hamilton-Jacobi-Bellman equation,
\begin{align}
(r_t + \overbrace{\hat{\tau}}^{\mathclap{\text{Creative destruction}}}) &V(j,t |q) - \dot{V}(j,t|q) = \overbrace{\tilde{\pi} q }^{\mathclap{\text{Flow profits}}}\nonumber \\_{}
&+ \max_{\substack{\mathbbm{1}^{NCA} \in \{0,1\} \\ z \ge 0}} \Bigg\{ z \Big[  \overbrace{\chi \big( V(j,t|\lambda q) - V(j,t|q)\big)}^{\mathclap{\mathbb{E}[\text{Payoff from own-innovation}]}}  \nonumber \\
&- \underbrace{\big(\frac{q}{Q_t}\big)}_{\mathclap{\text{scaling of R\&D cost}}} \Big( \overbrace{w_{RD,jt}(\mathbbm{1}^{NCA})}^{\mathclap{\text{R\&D wage depends on NCA}}} + \underbrace{\big(\frac{q}{Q_t}\big)^{-1}}_{\mathclap{\text{scaling of spinout formation rate}}} \overbrace{(1-\mathbbm{1}^{NCA}) \nu V(j,t|q)}^{\mathclap{\mathbb{E}[\text{Loss from spinout CD}]}} + \underbrace{\big(\frac{q}{Q_t}\big)^{-1}}_{\mathclap{\text{scaling of NCA cost}}}  \overbrace{\mathbbm{1}^{NCA} \kappa_c \nu V(j,t|q) }^{\mathclap{\text{NCA cost}}}\Big)  \Big] \Bigg\} \label{eq:hjb_incumbent_0}
\end{align}
where $\tilde{\pi}$ is defined in (\ref{profits_eq}). The discounting is at the risk-free rate $r_t$ because the financial intermediary diversifies across incumbents and there is no aggregate risk.

The first proposition shows that, in a symmetric BGP, the value function must have a linear form.

\begin{proposition}\label{proposition:hjb_scaling}
	In a symmetric BGP, the value function of the incumbent is given by
	\begin{align*}
		V(j,t|q) &= \tilde{V} q
	\end{align*}
	for some $\tilde{V} > 0$.
\end{proposition}

The result follows from two facts. First, in a BGP the interest rate is constant, by the Euler equation (\ref{eq:euler0}). Second, by definition $\hat{z}_{jt} = \hat{z}$ in a symmetric BGP. Together these imply that solutions to the incumbent HJB either satisfy $V(j,t|q) = \tilde{V} q$. The technical details of the proof are contained in Appendix \ref{appendix:proofs:proposition:hjb_scaling}.\footnote{In particular, I attempt to rule out non-constant solutions to $V(j,t|q)$, rather than simply assuming them away as in \cite{grossman_quality_1991} and \cite{acemoglu_innovation_2015}.} 

The above implies the following corollary.

\begin{proposition_corollary}
	In a symmetric BGP, the equilibrium R\&D wages are given by 
	\begin{align*}
	\hat{w}_{RD,t} &= \hat{w}_{RD} Q_t \\
	w_{RD,jt}(\mathbbm{1}^{NCA}) &= w_{RD}(\mathbbm{1}^{NCA}_{jt}) Q_t \textrm{, if $z > 0$}
	\end{align*}
\end{proposition_corollary}

\begin{proof}
	The entrant's free entry condition is
	\begin{align}
	\hat{\chi} \hat{z}^{-\psi} V(j,t|\lambda \bar{q}_{jt}) &= \frac{\bar{q}_{jt}}{Q_t} \hat{w}_{RD,t}
	\end{align}
	
	where $V(j,t|\lambda \bar{q}_{jt})$ is the value that the entrant will have as the new incumbent if he successfully innovates in the next instant. By the previous formula, $V(j,t | \lambda \bar{q}_{jt}) = \tilde{V} \lambda \bar{q}_{jt}$. Substituting yields
	\begin{align}
	\hat{\chi} \hat{z}^{-\psi} \tilde{V} \lambda &= \frac{\hat{w}_{RD,t}}{Q_t}
	\end{align}
	
	implying that $\frac{\hat{w}_{RD,t}}{Q_t}$ must be constant, i.e. $\hat{w}_{RD,t} = \hat{w}_{RD} Q_t$ for some $\hat{w}_{RD}$. Using this and $V(j,t | \bar{q}_{jt}) = \tilde{V}\bar{q}_{jt}$ in Lemma \ref{lemma:RD_worker_indifference1} yields $w_{RD,jt}(x) = w_{RD}(x) Q_t$. 
\end{proof}

The next proposition characterizes the equilibrium NCA policy of all incumbents in a symmetric BGP with $z > 0$. 

\begin{proposition}\label{proposition:optimalNCApolicy}
	In a symmetric BGP with $z > 0$, the equilibrium NCA policy of all incumbents is given by 
	\begin{align}
	\mathbbm{1}^{NCA}_{jt} = \mathbbm{1}^{NCA} = \begin{cases}
	1 & \textrm{if } \kappa_{c} < \bar{\kappa}_c (\kappa_e, \lambda) \\
	0 & \textrm{if } \kappa_{c} > \bar{\kappa}_c (\kappa_e, \lambda)\\
	\{0,1\} & \textrm{if } \kappa_c = \bar{\kappa}_c (\kappa_e, \lambda) 
	\end{cases} \label{eq_nca_policy}
	\end{align}
	where $\bar{\kappa}_c (\kappa_e, \lambda) = 1 - (1-\kappa_e)\lambda$.

\end{proposition}

Note that on the knife-edge $\kappa_c = \bar{\kappa}_c$, the incumbent is indifferent between $x = 0$ and $x = 1$. The proof entails some algebraic manipulations using the previous two lemmas. 

\begin{proof}
	Using the representation $V(j,t|q) = \tilde{V}q$ derived in Proposition \ref{proposition:hjb_scaling} in the incumbent HJB (\ref{eq:hjb_incumbent_0}) and dividing both sides by $q$ yields
	\begin{align}
	(r + \hat{\tau}) &\tilde{V} = \tilde{\pi} + \max_{\substack{\mathbbm{1}^{NCA} \in \{0,1\} \\ z \ge 0}} \Bigg\{ z \Big( \chi (\lambda -1) \tilde{V}- w_{RD}(\mathbbm{1}^{NCA}) - (1-\mathbbm{1}^{NCA}) \nu \tilde{V} - \mathbbm{1}^{NCA} \kappa_c \nu \tilde{V} \Big)\Bigg\} \label{eq:hjb_incumbent_1}
	\end{align}
	
	In any symmetric BGP with $z > 0$, Lemma \ref{lemma:RD_worker_indifference} determines the relationship between $w_{RD,jt}(x_{jt})$ and $\hat{w}_{RD,t}$.  Substituting in $w_{RD}(x)$ using the indifference condition (\ref{eq:RD_worker_indifference}) derived in Lemma \ref{lemma:RD_worker_indifference} yields
	\begin{align}
	(r + \hat{\tau}) \tilde{V} &= \tilde{\pi} + \max_{\substack{\mathbbm{1}^{NCA} \in \{0,1\} \\ z \ge 0}} \Big\{z \Big( \overbrace{\chi (\lambda - 1) \tilde{V}}^{\mathclap{\mathbb{E}[\textrm{Benefit from R\&D}]}}- \hat{w}_{RD} -  \underbrace{(1-\mathbbm{1}^{NCA})(1 - (1-\kappa_{e})\lambda)\nu \tilde{V}}_{\mathclap{\text{Net cost from spinout formation}}} - \overbrace{\mathbbm{1}^{NCA} \kappa_{c} \nu \tilde{V}}^{\mathclap{\text{Direct cost of NCA}}}\Big) \Big\} \label{eq:hjb_incumbent_workerIndiff}
	\end{align}

	
	Let $\bar{\kappa}_c (\kappa_e, \lambda) = 1 - (1-\kappa_e)\lambda$. If $z > 0$, the incumbent maximizes her flow payoff by choosing $\mathbbm{1}^{NCA} \in \{0,1\}$ which maximizes the term multiplying $z$. Therefore, $\mathbbm{1}^{NCA} = 1$ is strictly preferred iff $1 - (1-\kappa_e) \lambda > \kappa_c$, which is equation (\ref{eq_nca_policy}).
\end{proof}

Equation (\ref{eq:hjb_incumbent_workerIndiff}) has an intuitive economic interpretation. The left-hand side is the equilibrium flow payoff on an asset with value $\tilde{V}$ plus the expected loss upon creative destruction by an entrant. The RHS is the flow payoff of incumbency, absent creative destruction by entrants. The term $\chi(\lambda -1) \tilde{V}$ is the expected benefit per unit of R\&D effort. Notice the factor $\lambda -1$, which takes into account the opportunity cost of no longer producing with the obsolete technology. The term $-\hat{w}_{RD}$ reflects the cost of R\&D effort due to the contribution from the prevailing R\&D wage. The term $-(1-\mathbbm{1}^{NCA})(1 - (1-\kappa_e) \lambda) \nu \tilde{V}$ represents the expected net harm to the incumbent due to spinouts from the employee. Expanding this term, the term $-(1-\mathbbm{1}^{NCA})\nu \tilde{V}$ reflects the direct harm from creative destruction by spinouts. The second term $(1-\mathbbm{1}^{NCA})(1-\kappa_e)\lambda \nu \tilde{V}$ reflects the reduction in R\&D wage accepted by the R\&D employee in return for being free to open spinouts. Finally, the term $-\mathbbm{1}^{NCA} \kappa_c \nu \tilde{V}$ reflects the direct cost of enforcing NCAs.

\paragraph{Entry, aggregation and market clearing}

As before, suppose first that $z > 0$. The incumbent's FOC implies that, in equilibrium, the term multiplying $z$ in (\ref{eq:hjb_incumbent_workerIndiff}) must equal zero,
\begin{align*}
	0 &= \chi(\lambda-1)\tilde{V}- \hat{w}_{RD} - (1-\mathbbm{1}^{NCA})(1 - (1-\kappa_e)\lambda) \nu \tilde{V} - \mathbbm{1}^{NCA} \kappa_c \nu \tilde{V}
\end{align*}

Solving for $\tilde{V}$ yields
\begin{align}
	\tilde{V} &= \frac{\hat{w}_{RD}}{\chi(\lambda - 1) - (1-\mathbbm{1}^{NCA}) (1- (1-\kappa_e)\lambda)\nu - \mathbbm{1}^{NCA} \kappa_{c} \nu} \label{eq:hjb_incumbent_foc}
\end{align}

If the denominator is negative then $\tilde{V} < 0$ for positive wage $\hat{w}_{RD}$, which contradicts optimality as $z = 0$ yields a positive value. Suppose therefore that the denominator is positive. Entrant innovation satisfies a free entry condition,\footnote{The original condition is 
	\begin{align*}
		\hat{\chi} \hat{z}^{-\psi} (1-\kappa_e)  V(j,t|\lambda q) = \frac{q}{Q_t} \hat{w}_{RD,t}
	\end{align*}
	Using $V(j,t|q) = \tilde{V} q$ and $\hat{w}_{RD,t} = \hat{w}_{RD} Q_t$ yields (\ref{eq:free_entry_condition}).}
\begin{align}
	\underbrace{\hat{\chi} \hat{z}^{-\psi}}_{\mathclap{\text{Marginal innovation rate}}} \overbrace{(1-\kappa_e) \lambda \tilde{V}}^{\mathclap{\text{Payoff from innovation}}} &= \underbrace{\hat{w}_{RD}}_{\mathclap{\text{MC of R\&D}}} \label{eq:free_entry_condition}
\end{align}

Substituting $\tilde{V}$ using (\ref{eq:hjb_incumbent_foc}) yields an expression for entrant R\&D effort, 
\begin{align}
	\hat{z} &= \Big( \frac{\hat{\chi} (1-\kappa_{e}) \lambda}{\chi(\lambda-1) - (1-\mathbbm{1}^{NCA}) (1- (1-\kappa_e)\lambda)\nu - \mathbbm{1}^{NCA} \kappa_{c} \nu} \Big)^{1/\psi} \label{eq:effort_entrant}
\end{align}

Market clearing for R\&D labor requires
\begin{align}
	\bar{L}_{RD} &= \int_0^1 \frac{q_j}{Q} (z + \hat{z}) dj = z + \hat{z} \label{eq:RD_labor_market_clearing}
\end{align}
 
which implies
\begin{align}
	z &= \bar{L}_{RD} - \hat{z} \label{eq:zI_asFuncZe}
\end{align}

Growth is determined by the growth accounting equation\footnote{To see this, let $\Delta > 0$ and let $J_0(\Delta)$ ($J_1(\Delta)$) denote the indices $j\in [0,1]$ on which innovation occurs zero (one) times between $t$ and $t+\Delta$. By the law of large numbers, the set $J_1(\Delta)$ has measure $\mu_1 \Delta = (\tau + \tau^S + \hat{\tau})\Delta + o(\Delta)$. The set $J_0(\Delta)$ has measure $1 - \mu_1 \Delta + o(\Delta)$. 
		\begin{align*}
			Q_{t+\Delta} = \int_0^1 \bar{q}_{j,t+\Delta} dj &= \int_{j \in J_0} \bar{q}_{jt} dj + \int_{j \in J_1} \lambda \bar{q}_{jt} dj + o(\Delta) \\
			&= (1 - \mu_1\Delta - o(\Delta)) Q_t + (\mu_1 \Delta + o(\Delta) ) \lambda Q_t + o(\Delta) \\
			&= (1 - \mu_1\Delta) Q_t + \mu_1\Delta \lambda Q_t + o(\Delta)
 	\end{align*}
 where I used the fact that $\mathbb{E}[\bar{q}_{jt} | j \in J_0, t]  = \mathbb{E}[\bar{q}_{jt} | j \in J_1, t] = Q_t$, since innovations happen at the same rate regardless of $\bar{q}_{jt}$. It follows that
\begin{align*}
	\frac{\dot{Q}_t}{Q_t} = \frac{\lim_{\Delta \to 0} \frac{Q_{t+\Delta} - Q_t}{\Delta}}{Q_t} &= (\lambda - 1)\mu_1 
	\end{align*}}
\begin{align}
g &= (\lambda - 1)(\tau + \tau^S + \hat{\tau}) \label{eq:growth_accounting}
\end{align}

The Euler equation determines the interest rate, 
\begin{align}
	g &= \frac{\dot{C}}{C} = \frac{1}{\theta} (r - \rho) \label{eq:euler} \\
	\therefore r &= \theta g + \rho \label{eq:interest_rate}
\end{align}

Substituting the incumbent's FOC into the incumbent's HJB, and using the expression for the interest rate, yields the incumbent's value $\tilde{V}$,
\begin{align}
	 \tilde{V} &= \frac{\tilde{\pi}}{r + \hat{\tau}} \label{eq:hjb_incumbent_gordon_formula}
\end{align}

Finally, the free entry condition (\ref{eq:free_entry_condition}) determines the equilibrium value of $\hat{w}_{RD}$. 

If $\hat{w}_{RD}$ is negative or if (\ref{eq:effort_entrant}) implies that $\hat{z} > \bar{L}_{RD}$ then the assumption that $z > 0$ in a symmetric equilibrium leads to a contradiction, and instead equilibrium has $\hat{z} = \bar{L}_{RD}$ and $z = \tau = \tau^S = 0$. Then $g = (\lambda - 1) \hat{\tau}$. The interest rate is derived via the Euler equation (\ref{eq:interest_rate}), the incumbent value from (\ref{eq:hjb_incumbent_gordon_formula}), and the wage from (\ref{eq:free_entry_condition}). This implies the following lemma.

\begin{assumption}\label{model:assumption:zPositive0}
	$\chi(\lambda -1) > \nu \min \{ 1 - (1-\kappa_e) \lambda, \kappa_c \}$ 
\end{assumption}

\begin{assumption}\label{model:assumption:zPositive}
	$\Big( \frac{\hat{\chi} (1-\kappa_{e}) \lambda}{\chi(\lambda-1) - \nu \min\{ 1-(1-\kappa_e) \lambda, \kappa_c \}} \Big)^{1/\psi} < \bar{L}_{RD}$
\end{assumption}


\begin{lemma}\label{model:lemma:zge0condition}
	If a symmetric BGP exists, then Assumptions \ref{model:assumption:zPositive0} and \ref{model:assumption:zPositive} hold if and only if $z > 0$. 
\end{lemma}

\begin{proof}
	Follows from the discussion.
\end{proof}

The next assumption provides conditions under which household utility in the BGP is finite. This is necessary for the equilibrium to be well-defined.\footnote{In a standard setup, Assumption \ref{model:assumption:boundedUtility1} would be a consequence of the household's transversality condition.}

\begin{assumption}\label{model:assumption:boundedUtility1}
	$\rho > (1-\theta) g$
\end{assumption} 

In Assumption \ref{model:assumption:boundedUtility1}, $g$ stands for the closed form expression for $g$. Namely, if Assumptions \ref{model:assumption:zPositive0} and \ref{model:assumption:zPositive} hold then 
\begin{align}
g &= (\lambda - 1) \Big(  \big( \frac{\hat{\chi} (1-\kappa_e \lambda}{\chi(\lambda-1) - \nu \min \{1 - (1-\kappa_e) \lambda, \kappa_c \}} \big)^{(1-\psi)/\psi)} \\
&+ \big(\chi + (1- \mathbbm{1}^{NCA}_{\kappa_c < \bar{\kappa}_c(\kappa_e,\lambda)})\nu \big) \big( \bar{L}_{RD} -  \frac{\hat{\chi} (1-\kappa_e \lambda}{\chi(\lambda-1) - \nu \min \{1 - (1-\kappa_e) \lambda, \kappa_c \}} \big)^{1/\psi} \big) \Big) 
\end{align}

and otherwise
\begin{align}
g &= (\lambda -1) \hat{\chi} \bar{L}_{RD}^{1-\psi}
\end{align}

\begin{lemma}
	Under Assumption \ref{model:assumption:boundedUtility1}, the household's utility is finite on a symmetric BGP with growth rate $g$.
\end{lemma}

\begin{proof}
	Using $C(t) = C(0)e^{gt}$ on the BGP, the household's utility is
	\begin{align}
		U = \mathcal{K} \int_0^{\infty} e^{-\rho t} e^{(1-\theta)gt} dt + \text{Constant}
	\end{align}
	
	for some constant $\mathcal{K} > 0$. The integral $\int_0^{\infty} e^{-\rho t} e^{(1-\theta)gt} dt$ converges if and only if $\rho > (1-\theta)g$. 
\end{proof}




\begin{lemma_corollary}
	Under Assumption \ref{model:assumption:boundedUtility1}, the household's utility is finite on a symmetric BGP. 
\end{lemma_corollary}

Note that $\theta \ge 1$ implies Assumption \ref{model:assumption:boundedUtility1}. This is the empirically relevant case which I consider in the calibration.

I can now state some propositions concerning existence and uniqueness of the symmetric BGP. 

\begin{proposition}\label{proposition:BGPexistence}
	Under Assumption \ref{model:assumption:boundedUtility1}, there exists a symmetric BGP.
\end{proposition}

\begin{proof}
	To see this, one does not even need Proposition \ref{proposition:hjb_scaling}. Simply guess and verify that $V(j,t|q) = \tilde{V} q$ by solving for equilibrium variables as described in the last section. Assumption \ref{model:assumption:boundedUtility1} guarantees that household utility is finite and therefore a symmetric BGP exists.
\end{proof}

\begin{proposition}\label{proposition:purstrategyeq:positiveOI}
	If Assumptions \ref{model:assumption:zPositive0}, \ref{model:assumption:zPositive} and \ref{model:assumption:boundedUtility1} hold and $\kappa_c \ne \bar{\kappa}_c$ (as defined in Proposition \ref{proposition:optimalNCApolicy}), then:
	\begin{enumerate}
		\item There exists a unique symmetric BGP.
		\item On the symmetric BGP, $z > 0$ and $\mathbbm{1}^{NCA}_{jt} = \mathbbm{1}^{NCA}$
	\end{enumerate}
\end{proposition}

\begin{proof}
	Existence follows from Proposition \ref{proposition:BGPexistence}. To show uniqueness, first notice that in any symmetric BGP, one has $V(j,t|\bar{q}_{jt}) = \tilde{V}\bar{q}_{jt}$ by Proposition \ref{proposition:hjb_scaling} and its corollary. Given this representation and $\kappa_c \ne \bar{\kappa}_c$, Proposition \ref{proposition:optimalNCApolicy} implies that all symmetric BGPs have $\mathbbm{1}^{NCA}_{jt} = \mathbbm{1}^{NCA}$ for the same value for $\mathbbm{1}^{NCA}$. Together, these facts uniquely pin down the solution to the system of equations described in the derivation of the model. Hence there is a unique symmetric BGP. Finally, by Lemma \ref{model:lemma:zge0condition}, the symmetric BGP has $z > 0$. 
\end{proof}

\begin{proposition}\label{proposition:purestrategyeq:incumbents_indifferent}
	If Assumptions \ref{model:assumption:zPositive0}, \ref{model:assumption:zPositive} and \ref{model:assumption:boundedUtility1} hold and $\kappa_c = \bar{\kappa}_c$ (as defined in Proposition \ref{proposition:optimalNCApolicy}), then:
	\begin{enumerate}
		\item There exist exactly two symmetric BGPs with $\mathbbm{1}^{NCA}_{jt} = \mathbbm{1}^{NCA}$: one with $\mathbbm{1}^{NCA}_{jt} = 0$ and one with $\mathbbm{1}^{NCA}_{jt} = 1$.
		\item Both such equilibria have the same R\&D labor allocations $z, \hat{z}$
		\item The equilibrium with $\mathbbm{1}^{NCA}_{jt} = 0$ has a higher growth rate $g$ 
	\end{enumerate} 
\end{proposition}

\begin{proof}
	The proof of the first part is essentially the same as that of the previous proposition. The only difference is that either choice $\mathbbm{1}^{NCA}_{jt} = 1$ or $\mathbbm{1}^{NCA}_{jt} = 0$ is valid under Proposition \ref{proposition:optimalNCApolicy}. Given the representation $V(j,t|q) = \tilde{V}q$ and the scaling of wages $\hat{w}_{RD,t} = \hat{w}_{RD}Qt$ and $w_{RD,j}(\mathbbm{1}^{NCA}) = w_RD(\mathbbm{1}^{NCA}) Q_t$, the derivation above uniquely determines uniquely the rest of the equilibrium conditional on $x$. This equilibrium has finite household utility as long as Assumption \ref{model:assumption:boundedUtility1} holds. 
	
	The second part follows from the fact that when $\kappa_c = \bar{\kappa}_c$, the expressions for equilibrium R\&D effort $\hat{z},z$ do not depend on $\mathbbm{1}^{NCA}$. The reason is that $\mathbbm{1}^{NCA}$ only affects $\hat{z},z$ through its effect on the incumbent's effective wage, but here is the incumbent is indifferent between $\mathbbm{1}^{NCA} = 1$ and $\mathbbm{1}^{NCA} = 0$ hence faces the same effective wage. Mathematically, (\ref{eq:effort_entrant}) has the expression $(1-\mathbbm{1}^{NCA})(1-(1-\kappa_e)\lambda)\nu - \mathbbm{1}^{NCA} \kappa_c \nu = (1-\mathbbm{1}^{NCA}) \bar{\kappa}_c \nu + \mathbbm{1}^{NCA} \kappa_c \nu$ in the denominator. Since $\kappa_c = \bar{\kappa}_c$, $\hat{z}$ is unaffected by $\mathbbm{1}^{NCA}$, which in turn implies $z$ is also unaffected.
	
	The last statement follows from the fact that $z,\hat{z}$ are the same in both equilibria, but $\tau^S = 0$ when $\mathbbm{1}^{NCA} = 1$ and $\tau^S = \nu z^I > 0$ when $\mathbbm{1}^{NCA} = 0$. By the growth accounting equation (\ref{eq:growth_accounting}), this implies $g$ is higher when $\mathbbm{1}^{NCA} = 0$. 
\end{proof}

\begin{proposition}\label{proposition:purstrategyeq:zeroOI}
	If Assumption \ref{model:assumption:boundedUtility1} holds but either of Assumptions \ref{model:assumption:zPositive0} or \ref{model:assumption:zPositive} do not, then:
	\begin{enumerate}
		\item There is a unique symmetric BGP (modulo irrelevant incumbent choice of $\mathbbm{1}^{NCA}_{jt}$)
		\item This BGP has $z = 0$ and $\hat{z} = \bar{L}_{RD}$.
	\end{enumerate} 
\end{proposition}

\begin{proof}
	By Lemma \ref{model:lemma:zge0condition}, $z = 0$ in a symmetric BGP. Then $\hat{z} = \bar{L}_{RD}$ by R\&D labor market clearing (\ref{eq:zI_asFuncZe}). The rest of the equilibrium is pinned down given $x$ by the derivation given the representation $V(j,t|q) = \tilde{V}q$ and the scaling of wages $\hat{w}_{RD,t} = \hat{w}_{RD}Qt$ and $w_{RD,j}(x) = w_RD(x) Q_t$, which Lemma \ref{proposition:hjb_scaling} proves holds on any symmetric BGP. This shows that the equilibrium does not depend on the choice of $x$ by incumbents in a symmetric equilibrium with $z = 0$. Hence, the symmetric BGP is unique modulo an irrelevant choice by the incumbent of whether to use $x_{jt} = 0$ or $x_{jt} = 1$. 
\end{proof}

Finally, there is a technical possibility of "mixed strategy" equilibria on the knife-edge $\kappa_c = \bar{\kappa}_c$ where both choices of $x$ occur in equilibrium. This result is included for completeness but I will not study this case further in this paper.

\begin{proposition}\label{proposition:mixedstrategyeq}
	If $\theta \ge 1$, $\kappa_c = \bar{\kappa}_c$, and $\Big( \frac{\hat{\chi} (1-\kappa_{e}) \lambda}{\chi(\lambda-1) - \kappa_{c} \nu} \Big)^{1/\psi} < \bar{L}_{RD}$, then for all $f \in (0,1)$ there exists a symmetric BGP in which, at any given time $t$, a fraction $f$ of incumbents $j$ have $x_{jt} = 1$.  
\end{proposition}

\begin{proof}
	See Appendix \ref{appendix:model:proofs:proposition:mixedstrategyeq}.
\end{proof}


\section{Efficiency and theoretical policy analysis}\label{model:efficiency:efficiency}


%\setcounter{secnumdepth}{3}

Below I discuss theoretically the efficiency of the decentralized equilibrium and how this depends on parameters. To do this, first I characterize social welfare. It is driven by two channels: the level of consumption given $Q_t$, and the growth rate of $Q_t$. 

I discuss three sources of misallocation in the decentralized equilibrium, corresponding to the three allocative margins in a symmetric BGP: production labor, R\&D labor, and NCAs. The conclusion is that, while there is a negative social externality from the use of NCAs (as growth in this class of models has positive externalities), the overuse of NCAs can help to mitigate the harm from the business stealing externality, which induces the economy to spend too many resources on R\&D aimed at creative destruction. 

I then use these theoretical insights to conduct several theoretical policy analyses, the key one being the effect of varying the cost of NCAs $\kappa_c$. The signs and magnitudes of the growth and welfare implications depend on the parameters of the model, which I discipline in the following two sections, before turning to a quantitative policy analysis.

\subsection{Preliminaries}

\subsubsection{Welfare}

Social welfare is simply the representative household's lifetime utility,\footnote{Technically this should be written in terms of $W_t$, the welfare at time $t$. I ignore this detail in the interest of expositional simplicity and without loss of generality since the model grows at constant rate so $W_t = e^{(1-\theta)gt}\tilde{W}$.} 
\begin{align}
	\tilde{W} = \int_0^{\infty} e^{-\rho t} \frac{C(t)^{1-\theta} - 1}{1-\theta} ds \label{eq:agg_welfare0}
\end{align}

Using $C(t) = \tilde{C} e^{gt}$ on the BGP and integrating yields
\begin{align}
	\tilde{W} &= \frac{\tilde{C}^{1-\theta} }{(1-\theta)(\rho - g(1-\theta))} + \text{Constant}
\end{align}

This illustrates that social welfare can be decomposed into two channels: a \textit{growth} channel via $g$ and a \textit{steady-state consumption} channel via $\tilde{C}$. Higher values for either imply higher welfare. In turn, $\tilde{C}$ can be decomposed using 
\begin{align}
	\tilde{C} &= \tilde{Y} - \overbrace{(\hat{\tau} + \tau^S) \kappa_e \lambda \tilde{V}}^{\mathclap{\text{Creative destruction cost}}} - \underbrace{x z \kappa_c \nu \tilde{V}}_{\mathclap{\text{NCA enforcement cost}}} \label{eq:agg_consumption_decomposition}
\end{align}

so that steady-state consumption is flow output of the final good minus the final goods cost of creative destruction and of NCA enforcement.

\subsection{Efficiency of decentralized equilibrium}

As mentioned previously, the decentralized equilibrium is inefficient. The model has three factors of production: production labor, R\&D labor and intermediate goods, which are used in the production of the final good. In this section I show that these margins are all, in general, inefficient in the decentralized economy. 

One caveat is important. The social planer's first-best allocation is not defined in this setting because primitives of the model depend on the endogenous value of being an incumbent.\footnote{As discussed previously, the model could be modified in a straightforward way to not have this feature, but the solution would no longer be in closed form.} However, this caveat only applies to the mapping from the R\&D allocation and NCA usage to $\tilde{C} = \frac{C(t)}{Q(t)}$. Hence, I can still make precise statements regarding the allocation of production labor and the effect of the allocation of R\&D and NCAs on the BGP growth rate. I study the overall effect on welfare of such forms of reallocation by studying policy counterfactuals on the calibrated model in Section \ref{sec:policy_analysis}.

\subsubsection{Misallocation of production labor: monopoly distortion}

The allocation of production labor in the economy is distorted by the monopoly power of producers in the intermediate goods market. As is standard, this monopoly power induces pricing higher than marginal cost and hence to an underallocation of production labor to intermediate goods production. This reduces $\tilde{C}$. I mention this for completeness only; from now on, I will ignore this source of inefficiency as it is not the focus of this analysis.\footnote{In this setting with exogenous total supply of R\&D, a subsidy to intermediate goods production would correct this externality and have no effect on equilibrium growth.}\footnote{In a model with limit pricing, average markups would depend on the distance between leaders and followers in good $j$. In that case, this distortion interacts with the distortion to R\&D and would need to be considered.}

\subsubsection{Misallocation of R\&D labor}\label{model:efficiency:misallocationRD}

The decentralized allocation of R\&D labor is also, in general, not efficient. Because total R\&D spending is exogenous, any inefficiency must be due to a misallocation of R\&D \textit{between} OI by incumbents and CD by entrants.

To isolate the determinants of the degree of equilibrium misallocation, first consider the equilibrium marginal effects on the innovation rate from more incumbent OI and entrant CD, respectively. If the marginal effect of entrant CD on innovation is lower, then equilibrium innovation, and therefore growth, would increase after a reallocation of R\&D labor to incumbent OI. The marginal effect of OI, including the induced innovation by spinouts, is equal to $\chi + (1-\mathbbm{1}^{NCA}) \nu$. The marginal effect of CD by entrants is
\begin{align}
\frac{d}{d\hat{z}} \hat{\tau} &= (1-\psi) \hat{\chi} \hat{z}^{-\psi} \label{eq:marginal_effect_effort_entrant}
\end{align}
%
Substituting the expression for $\hat{z}$ in (\ref{eq:effort_entrant}), dividing by $\chi + (1-x)\nu$, and rearranging yields 
\begin{align}
	\frac{\frac{d}{d\hat{z}} \hat{\tau}}{\chi + (1-x)\nu} &= \overbrace{\frac{\lambda-1}{\lambda}}^{\mathclap{\text{Business stealing}}} \times \underbrace{(1-\psi)}_{\mathclap{\text{Congestion}}}  \times \overbrace{\frac{\chi(\lambda-1) -(1-\mathbbm{1}^{NCA}) (1-(1-\kappa_e)\lambda)\nu - \mathbbm{1}^{NCA} \kappa_c \nu}{\chi(\lambda-1)}}^{\mathclap{\text{Effective cost of R\&D}}} \nonumber \\
	&\times \underbrace{\frac{\chi}{\chi + (1-\mathbbm{1}^{NCA})\nu}}_{\mathclap{\text{Spinout formation}}} \times  \overbrace{\frac{1}{1-\kappa_{e}}}^{\mathclap{\text{Entry cost}}}  \label{cs:growth_misallocation_condition}
\end{align}

If the RHS is less than 1, then reallocating R\&D labor from entrants to incumbents increases the BGP growth rate. This depends on the value of the five factors on the RHS, which I discuss below. 

\subparagraph{Business stealing}

The term $\frac{\lambda - 1}{\lambda} < 1$, reflects the \textit{business stealing} externality.\footnote{This is sometimes referred to as \emph{Arrow's replacement effect}, which emphasizes the fact that incumbents, unlike entrants, take into account the fact that they replace their monopoly. This is somewhat backwards, since the replacement of one's own profits is not an externality. For this reason, I prefer the business-stealing terminology.} Innovation by entrants imposes a negative externality on the profits of the incumbent. This means that entrants can earn the required (private) return on R\&D with a lower innovation rate per marginal cost than incumbents. In the calibration, $\lambda \approx 1.1$ so $\frac{\lambda-1}{\lambda} \approx 0.09$, so this effect can be strong in magnitude.\footnote{In models such as \cite{aghion_competition_2005}, this effect is attenuated by the fact that incumbents engage in neck-and-neck competition within each good $j$. This means R\&D by incumbents has a negative externality on other incumbents in the same good $j$, making the situation more symmetric between incumbents and entrants. I plan to explore this question further in later work.}


\paragraph{Congestion}

The term $1-\psi < 1$ reflects the \textit{congestion} externality. Individual entrants impose a negative externality on the expected returns of other entrants. As with business-stealing, the congestion externality also tends to overallocate R\&D to entrants. To give a sense of magnitude, in the calibration $\psi = 0.5$ so $1-\psi = 0.5$.

\paragraph{Effective cost of R\&D} 

The term $\frac{\chi(\lambda-1) -(1-\mathbbm{1}^{NCA}) (1-(1-\kappa_e)\lambda)\nu - \mathbbm{1}^{NCA} \kappa_c \nu}{\chi(\lambda-1)}$ reflects the fact that entrants pay a different effective cost of R\&D than the incumbent. As long as $(1 - (1-\kappa_e) \lambda > 0$, incumbents pay a higher effective cost because they either internalize the harm from future WSOs or must pay a cost to enforce NCAs to prevent them. All else equal, this means entrants have a higher private return to R\&D than incumbents. In equilibrium these private returns must equate; therefore, there is more more entry and it has a lower marginal effect on growth. Alternatively, if $1 - (1-\kappa_e) \lambda < 0$, incumbents benefit from spinouts \textit{ex ante} because they are bilaterally efficient, and as a consequence they pay a lower effective cost of R\&D. This has the opposite effect of increasing the equilibrium marginal effect on growth of entrant R\&D.


\paragraph{Spinout formation}

The term $\frac{\chi}{\chi + (1-\mathbbm{1}^{NCA})\nu} \le 1$ reflects the contribution to the productivity of OI stemming from entry by WSOs. If $\mathbbm{1}^{NCA} = 0$ and $\nu > 0$, the term is strictly less than 1. OI by incumbents has a positive growth externality (through spinout entry) hence, in equilibrium it generates a higher marginal effect on growth from OI. If $\mathbbm{1}^{NCA} = 1$ or $\nu = 0$ this term is equal to 1 and has no effect on the inequality, corresponding to $\tau^S = 0$.


\paragraph{Entry cost}

Finally, the term $\frac{1}{1-\kappa_e} \ge 1$ reflects the additional entry cost paid by entrants upon innovating. All else equal, this implies entrants have a lower private return from R\&D spending. In equilibrium, the returns to R\&D spending must be the same therefore $\hat{z}$ declines. This tends to reduce the extent of misallocation, as it works against the net of the other terms on the RHS of equation (\ref{cs:growth_misallocation_condition}), although this does come at the cost of reduced steady-state consumption (which I consider in Section \ref{sec:policy_analysis}).


\subsubsection{Misallocation of NCAs}\label{model:efficiency:misallocationNCAs}

Using NCAs is always suboptimal for growth in a partial equilibrium sense, since it throws away innovations. The easiest way to see this is to consider an exogenous shift from $\mathbbm{1}^{NCA} = 1$ to $\mathbbm{1}^{NCA} = 0$ while holding $z,\hat{z}$ constant. This increases the growth rate by $(\lambda -1) z \nu$ because spinout entrepreneurship is no longer prevented and because we have assumed that spinouts represent truly new innovations, rather than ideas stolen from the incumbent.

What matters in the context of this paper, however, is the general equilibrium effects of the availability of NCAs. Viewed in isolation, the decentralized equilibrium tends to overuse NCAs because incumbents and their employees do not internalize the positive growth externalities that spinouts have on the rest of the economy.\footnote{These positive externalities are: the increase in steady state consumption for the representative household, the increase in the productivity of all other intermediate goods producers, and finally the fact that entrants are now able to innovate on a machine of higher quality.} Because in this model NCAs are used by either all or no incumbents, this inefficiency can only manifest as an expansion of the region in parameter space where NCAs are used but, were they not used, aggregate growth would be higher. This can be most clearly seen by considering a case where the cost of NCAs such that incumbents are nearly indifferent between using and not using NCAs, i.e. $\kappa_c = \bar{\kappa}_c - \varepsilon$ for $\varepsilon > 0$ small enough. By Proposition \ref{proposition:purstrategyeq:positiveOI}, there are is a unique symmetric BGPs with $\mathbbm{1}^{NCA}_{jt} = 1$. As $\varepsilon \to 0$, incumbent R\&D effort with and without an NCA converges to the same value $z$. Hence, even if the RHS of inequality (\ref{cs:growth_misallocation_condition}) is less than 1, exogenously requiring $\mathbbm{1}^{NCA}_{jt} = 0$ implies barely any growth-enhancing reallocation of R\&D, while increasing innnovation by spinouts by a discrete amount. 

However, when the decentralized equilibrium misallocates R\&D labor, it does not follow from the above discussion that making the use of NCAs less costly will further reduce growth. These direct negative externalities are part of a correction of distortion: the misallocation of R\&D. In the example just given, while exogenously forcing incumbents to not use NCAs is beneficial, it may be even more beneficial to even further reduce the barriers to the usage of NCAs. I analyze this question below in Section \ref{subsubsec:ncacost}.

\subsection{Effect of NCA enforcement and other policies}\label{model:efficiency:policy_analysis}

To make more concrete the ideas developed in the previous section, in this section I conduct a sequence of theoretical second-best analyses assuming the planner can control one or more parameters and/or Pigouvian taxes. The central question is whether reducing barriers to the use of NCAs increases or decreasing growth.\footnote{I defer the discussion of overall welfare to the quantitative analysis in Section \ref{sec:policy_analysis}, as the effect on $\tilde{C}$ of the policies I study depend in very complicated ways on parameters and as such the theoretical analysis generates more heat than light.} In addition, I also study other policies which may substitute or complement NCA enforceability policy, such as R\&D subsidies. While such policies can technically be studied in a standard quality ladders model, their \textit{interaction} with the endogenous use of NCAs is novel and yields some new theoretical insights. 

I will find that the effect of NCA policy and adjacent policies on the growth rate depends significantly on the parameters of the model. For this reason, in the following two sections I use data to calibrate the model.\footnote{In the empirical section, I obtain empirical discipline on $\nu$, which determines the magnitude of the effect of a reduction of NCA costs. In the calibration section, I use aggregate data, as well as some growth attribution estimates and standard parameters from the literature, to discipline the other parameters of the model, determining the sign.}

\paragraph{Policies considered} 

I study planners who can control:

\begin{enumerate}
	\item Cost of NCAs: $\kappa_c$ 
	\item R\&D subsidy (tax): $T_{RD}$
	\item Creative destruction tax (subsidy): $T_e$
	\item OI R\&D subsidy (tax): $T_{RD,I}$
	\item All of the above: $\{\kappa_c, T_{RD}, T_{RD,I}, T_e\}$
\end{enumerate}

\paragraph{Comparative statics}

All comparisons below are static comparisons between BGPs. I often use language like "as [a certain parameter or tax] increases..." or "as [parameter] crosses [a threshold], [equilibrium variable] jumps...". This does not refer to a transition path of the economy but to as static comparison of the BGPs for each value of the parameter or tax.\footnote{That being said, in this model it is the case that the economy immediately jumps to the unique new BGP following a parameter or tax change, provided it is assumed that the pre- and post-change equilibria are symmetric.}

\paragraph{Public finance} 

In cases of taxes (subsidies), I assume that they are rebated to (financed by) the representative household in a lump-sum payment. Because there is no labor-leisure choice, this does not create any additional distortions in the economy.\footnote{In general, however, this is a matter of first-order importance and an interesting avenue for further research. Policies should be evaluated in terms of "bang for buck."}

\subsubsection{NCA cost $\kappa_c$}\label{subsubsec:ncacost}

To analyze this question using the model, consider a planner who controls the parameter $\kappa_c$. I will consider the effect of a reduction in $\kappa_c$ on growth, starting from a value $\kappa_c > \bar{\kappa}_c$. I interpret this as a policymaker changing the extent of restrictions on NCAs so that they are more less difficult to use, or simply allowing their free use in cases where they are not currently allowed. Of course, NCAs require costs for enforcement even if they are fully endorsed by the legal system: a contract must be written and, in the case of infringement, it must be established that the employee is, in fact, competing with their previous employer. Therefore, it is reasonable to suppose a minimum NCA cost $\munderbar{\kappa}_c \ge 0$ such that $\kappa_c \ge \munderbar{\kappa}_c$. For simplicity, in the analysis below I assume $\munderbar{\kappa}_c = 0$.

Suppose first that $\mathbbm{1}^{NCA} = 0$ and $z > 0$, with $\kappa_c > \bar{\kappa}_c$. Initially, a marginal reduction in $\kappa_c$ has no effect on the equilibrium as NCAs $\mathbbm{1}^{NCA} = 0$ so the cost is not being paid by any agents. Upon crossing the threshold $\bar{\kappa}_c$, there is a shift to $\mathbbm{1}^{NCA} = 1$ and growth decreases by a discrete amount through a reduction in employee spinout formation. The allocation of R\&D is unchanged at this point because incumbents are indifferent between using and not using NCAs. The reduction in the growth rate induces via general equilibrium a desire to save, which lowers the equilibrium interest rate. This increases the value of the incumbent, so to keep the labor market in equilibrium, the R\&D wage rises discretely.

A further reduction in $\kappa_c$ reallocates R\&D labor from the entrant to the incumbent, using equations (\ref{eq:effort_entrant}) and (\ref{eq:zI_asFuncZe}). This reallocation of R\&D labor decreases the BGP growth rate if and only if the marginal effect on growth of incumbent R\&D is higher than the marginal efect on growth of entrant R\&D. As discussed in Section \ref{model:efficiency:misallocationRD}, this occurs whenever inequality (\ref{cs:growth_misallocation_condition}) holds with sufficient slack. 

The reallocation of R\&D occurs through a change in the ratio $\hat{w}_{RD} / \tilde{V}$. When $\kappa_c$ decreases, the ratio increases so that the incumbent's FOC continues to hold. This ratio then feeds into the reduction in entrant R\&D with sensitivity given by the return-elasticity of entrant R\&D spending. Intuitively, the increase in $\kappa_c$ makes R\&D more expensive for incumbents, reducing $z$ to zero in partial equilibrium. To clear the labor market, $\hat{w}_{RD}$ must decline to induce more R\&D. Because incumbents pay for R\&D not just through wages but implicitly through future WSOs, in the new equilibrium, their effective cost of R\&D is higher relative to entrants, whose only R\&D cost is the R\&D wage. As a result, incumbents employs a smaller share of the R\&D labor in equilibrium.

Finally note that if $z = 0$ initially, then decreasing $\kappa_{c}$ to values below $\bar{\kappa}_c$ has no effect, as the incumbent is on a corner solution. However, it is possible that for low enough $\kappa_c$, one has $z > 0$ in equilibrium. Thereafter, the effect of a reduction in $\kappa_c$ is the same as described above.

\paragraph{Does eliminating barriers to NCAs increase growth?}

The answer to this question depends on four main factors. The first three factors determine the sign of a maximal reduction in barriers to the use of NCAs. The fourth factor determines the magnitude.

\subparagraph{Extent of R\&D misallocation} The first factor is whether inequality (\ref{cs:growth_misallocation_condition}) holds, and by how much. The smaller is the RHS compared to 1, the more severe is the decentralized misallocation of R\&D spending and therefore the greater the growth increase from a marginal reallocation of R\&D. If the inequality does not hold, the reallocation of R\&D reduces growth and a reduction in $\kappa_c$ is unequivocally bad for welfare. 

\subparagraph{Elasticity of R\&D spending}

The second factor is how much (general equilibrium) reallocation will actually result from a given (partial equilibrium) change in the ratio of private returns to R\&D of incumbents and entrants. The latter is the mechanism by which a reduction in $\kappa_c$ affects the reallocation of R\&D. This depends on the price-elasticity of R\&D spending for both parties. The higher are these elasticities, the more reallocation there is in response to a given change in $\kappa_c$. In this model, the incumbent's elasticity is infinite, as she has constant returns to scale, while the entrant's elasticity is finite due to decreasing returns to scale. This would be more realistic if the incumbent's elasticity were finite, so I consider the case of decreasing returns to incumbent R\&D in a robustness check \textbf{[not yet in this draft]}. Experiments with this model using a very low entrant elasticity suggest that this will not change the results qualitatively, though it may reduce their magnitude somewhat.\footnote{If the incumbent had decreasing returns to R\&D, the increase in incumbent R\&D spending would help to bring the FOC back into alignment and there would be less required reallocation.} 

\subparagraph{Scope for reductions in $\kappa_c$}

Finally, the third factor is simply how large of a reduction in $\kappa_c$ is reasonably under the control of the policy maker. This has two components. If the model is to generate spinouts on the BGP (which is one of the calibrating assumptions), then it must set identify $\kappa_c \ge \bar{\kappa}_c = 1 - (1-\kappa_e) \lambda$. Therefore, the smaller is $(1-\kappa_e) \lambda$, the larger is the model-implied direct cost of using NCAs. In addition, it depends on the value of $\munderbar{\kappa}_c$, as discussed above. Essentially, this is the question of what fraction of the cost $\kappa_c$ should be interpreted as due to restrictions on contracting and how much is due to the direct costs of implementing an employment contract.\footnote{One might imagine subsidizing the use of NCAs (i.e. setting $\kappa_c < 0$), but as they are simply pieces of paper that can be produced even without actually doing R\&D, this would not be incentive compatible in reality. Alternatively, one could imagine directly subsidizing the R\&D spending of the incumbents. In the model when $\mathbbm{1}^{NCA} = 1$ these are equivalent. Later I consider this type of policy and find that it is actually part of the optimal policy.} 

\subparagraph{Rate of spinout formation}

The final factor is the rate of spinout formation $\nu$. A higher value amplifies the effect of reductions in $\kappa_c$. Mathematically, this occurs because $\kappa_c$ is the cost of NCAs per attempted spinout so it appears in the model in the form $\kappa_c \nu$. Hence a higher $\nu$ means that the cost of R\&D is more sensitive to changes in $\kappa_c$. Also, when $\kappa_c$ crosses the $\bar{\kappa}_c$ threshold, the effect on spinout entry is also amplified in the same way. Intuitively, a higher spinout formation rate just means that spinouts are more significant both to overall growth and to the incumbents who wish to avoid being replaced by them and hence their incentives for R\&D spending. Whether reducing barriers to NCAs is good or bad for growth, a higher rate of spinout formation amplifies the effect. 

Since all of these factors depend on parameters. 
 

\subsubsection{RD subsidy (tax)}

R\&D subsidies are a natural policy to study due to their ubiquity and significant magnitude throughout the developed world and, in particular, the United States (around 20\%). Suppose that the planner subsidizes R\&D spending at rate $T_{RD}$ (tax if $T_{RD} < 0$). In this case, in a symmetric BGP the incumbent's normalized HJB becomes
\begin{align}
(r + \hat{\tau}) \tilde{V} = \tilde{\pi} + \max_{\substack{\mathbbm{1}^{NCA} \in \{0,1\} \\ z \ge 0}} \Big\{z &\Big( \overbrace{\chi (\lambda - 1) \tilde{V}}^{\mathclap{\mathbb{E}[\textrm{Benefit from R\&D}]}}- (\underbrace{1-T_{RD}}_{\mathclap{\text{R\&D Subsidy}}}) \big( \overbrace{\hat{w}_{RD} - (1-\mathbbm{1}^{NCA})(1-\kappa_e)\lambda \nu \tilde{V}}^{\mathclap{\text{Incumbent R\&D wage, by Lemma \ref{lemma:RD_worker_indifference1} }}}\big) \label{eq:hjb_incumbent_RDsubsidy} \nonumber \\ 
&-  \underbrace{(1-\mathbbm{1}^{NCA}) \nu \tilde{V}}_{\mathclap{\text{Net cost from spinout formation}}} - \overbrace{\mathbbm{1}^{NCA} \kappa_{c} \nu \tilde{V}}^{\mathclap{\text{Direct cost of NCA}}}\Big) \Big\} 
\end{align}

Define
\begin{align}
\tilde{\bar{\kappa}}_c = \tilde{\bar{\kappa}}_c(\kappa_e,\lambda;T_{RD}) = 1 - (1-T_{RD})(1-\kappa_e)\lambda
\end{align} 

Then if $z > 0$, the incumbent's optimal NCA policy is given by 
\begin{align}
x = \begin{cases}
1 & \textrm{if } \kappa_{c} < \tilde{\bar{\kappa}}_c  \\
0 & \textrm{if } \kappa_{c} > \tilde{\bar{\kappa}}_c \\
\{0,1\} & \textrm{if } \kappa_c = \tilde{\bar{\kappa}}_c 
\end{cases} \label{eq:nca_policy_RDsubsidy}
\end{align}

Since the argument is the same as in Section \ref{subsubsec:dynamic_equilibrium_original_solution}, I omit the details. Assuming $z > 0$, by the same logic as before one can obtain an expression for equilibrium $\hat{z}$, 
\begin{align}
\hat{z} &= \Bigg( \frac{\hat{\chi} (1-\kappa_{e}) \lambda}{\chi(\lambda -1) - \nu (\mathbbm{1}^{NCA}\kappa_c + (1-\mathbbm{1}^{NCA})(1 - (1-T_{RD})(1-\kappa_e)\lambda)) } \Bigg)^{1/\psi} \label{eq:effort_entrant_RDsubsidy}
\end{align}

The rest of the equilibrium allocation and prices can be computed in the same way as before (including how to account for the possibility of $z = 0$), with the one exception being that the equilibrium R\&D wage is now given by 
\begin{align}
\hat{w}_{RD} &= (1-T_{RD})^{-1}\hat{\chi} \hat{z}^{-\psi} (1-\kappa_e) \lambda \tilde{V} \label{eq:wage_rd_labor_RDsubsidy}
\end{align}

\paragraph{Effect on growth}

First suppose $\mathbbm{1}^{NCA} = 0$ and consider a small increase in $T_{RD}$ from $T_{RD}^0$ to $T_{RD}^1 > T_{RD}^0$. If $\mathbbm{1}^{NCA} = 0$ after the increase in $T_{RD}$, then by (\ref{eq:effort_entrant_RDsubsidy}), $\hat{z}$ increases; and by the labor resource constraint $z$ decreases. If (\ref{cs:growth_misallocation_condition}) holds, this reduces growth. Intuitively, the increased R\&D subsidy reduces the wage expenses paid for R\&D by the same factor $1-\frac{1-T_{RD}^1}{1-T_{RD}^0}$ for both incumbents and entrants. However, the incumbent's effective cost of R\&D also includes the increased likelihood of creative destruction by an employee spinout. Therefore, her effective cost of R\&D is reduced by a factor $\tilde{\tau}_{RD} < 1-\frac{1-T_{RD}^1}{1-T_{RD}^0}$. In general equilibrium, R\&D labor is reallocated to entrants and growth declines.

If the increase in $T_{RD}$ is large enough, $\mathbbm{1}^{NCA}$ changes from $\mathbbm{1}^{NCA} = 0$ to $\mathbbm{1}^{NCA} = 1$ and therefore $\tau^S$ jumps to zero, reducing growth further. Intuitively, higher R\&D subsidies mean the incumbent prefers to pay for the R\&D with wages, which receive a subsidy, rather than implicitly through future spinouts, the cost of which is not subsidized. Incumbents therefore opt to use NCAs, bringing spinout entry to zero and reducing growth by a discrete jump. In addition, there are no indirect effects on growth through changes in $\hat{z}$,$z$, as these variables do not jump: according to (\ref{eq:nca_policy_RDsubsidy}), the transition from $\mathbbm{1}^{NCA} = 0$ to $\mathbbm{1}^{NCA} =1$ occurs at the value of $T_{RD}$ such that $\kappa_c$ is equal to the term multiplying $(1-\mathbbm{1}^{NCA})$, implying that $\hat{z}$, and therefore $z$, does not jump.

Finally, if $T_{RD}$ is increased even further, there is no change in the equilibrium allocation. The only change is the wage of R\&D labor, which by (\ref{eq:wage_rd_labor_RDsubsidy}) increases to equilibriate the R\&D labor market.

\subsubsection{Creative destruction tax (subsidy)}

Suppose that the planner taxes entry at rate $T_e$ (subsidy if $T_e < 0$). Specifically, the planner taxes the entry fixed cost $\kappa_e \lambda \tilde{V} q$ at rate $T_e$ so that a firm entering with quality $\lambda q$ perceives a total cost of $(1+T_e) \kappa_e \lambda \tilde{V}q$ units of the final good. Economically, this can be interpreted as a tax on non-R\&D expenses related to the development of new versions of products currently not sold by the firm in question.\footnote{\textbf{[Put this footnote earlier in model exposition]} Because the tax is proportional to these expenses, rather than a fixed tax on entry, it does not induce any reallocation of R\&D towards higher quality goods. This property is not only analytically convenient -- it is necessary for a BGP to exist. In the baseline model, the expected growth rate of normalized frontier quality $\tilde{\bar{q}}_j = \frac{\bar{q}_j}{Q}$ is constant for all $j \in [0,1]$ and there is no exit of low quality firms (and subsequent injection of "average quality" firms). Running this stochastic process forward in time, the distribution of $\tilde{\bar{q}}_j$ spreads out, i.e. its variance and higher order measures of dispersion increase, which implies that there is no stationary distribution of $\tilde{\bar{q}}_j$. A BGP continues to exist, however, because only the mean of $\tilde{\bar{q}}_j$, $\mathbb{E}[\tilde{\bar{q}}_j] = 1$, is relevant for aggregate variables. This is why, e.g. the growth accounting equation (\ref{eq:growth_accounting}) can be written so simply. If, instead, growth is faster for higher $\tilde{\bar{q}}_j$, as is the case with a fixed entry fee, there is again no stationary distribution of $\tilde{\bar{q}}_j$, as before. However, in addition, there is no BGP, because aggregate variables such as the growth rate and the R\&D wage now depend on the entire distribution of $\tilde{\bar{q}}_j$, which is not stationary.}

In this setup, the R\&D labor supply indifference condition becomes
\begin{align}
\hat{w}_{RD} &= w_{RD}(\mathbbm{1}^{NCA}) + (1-\mathbbm{1}^{NCA}) \nu (1-(1+T_e)\kappa_e) \lambda \tilde{V} \label{eq:RD_worker_indifference_entryTax}
\end{align}

Substituting this into the incumbent's HJB and using the same argument as before, this implies that if $z > 0$, the allocation of NCAs is 
\begin{align}
\mathbbm{1}^{NCA} = \begin{cases}
1 & \textrm{if } \kappa_{c} < \hat{\bar{\kappa}}_c  \\
0 & \textrm{if } \kappa_{c} > \hat{\bar{\kappa}}_c \\
\{0,1\} & \textrm{if } \kappa_c = \hat{\bar{\kappa}}_c 
\end{cases} \label{eq:nca_policy_entryTax}
\end{align}

where
\begin{align}
\hat{\bar{\kappa}}_c = \hat{\bar{\kappa}}_c(\kappa_e,\lambda;T_e) = 1 - (1-(1+T_e)\kappa_e)\lambda  \label{eq:barkappa_entryTax}
\end{align}

If $(1 + T_e) \kappa_e > 1$ then $\hat{z} = 0$ and $z = \bar{L}_{RD}$. Otherwise, the free entry condition is now
\begin{align}
\underbrace{\hat{\chi} \hat{z}^{-\psi}}_{\mathclap{\text{Marginal innovation rate}}} \overbrace{(1-(1+T_e)\kappa_e) \lambda \tilde{V}}^{\mathclap{\text{Payoff from innovation}}} &= \underbrace{\hat{w}_{RD}}_{\mathclap{\text{MC of R\&D}}} \label{eq:free_entry_condition_entryTax}
\end{align}

Substituting the incumbent FOC into (\ref{eq:free_entry_condition_entryTax}) to eliminate $\tilde{V}$ yields an expression for $\hat{z}$, 
\begin{align}
\hat{z} &= \Bigg( \frac{\hat{\chi} (1-(1+T_e)\kappa_{e}) \lambda}{\chi(\lambda -1) - \nu (\mathbbm{1}^{NCA}\kappa_c + (1-\mathbbm{1}^{NCA})(1 - (1-(1+T_e)\kappa_e)\lambda)) } \Bigg)^{1/\psi} \label{eq:effort_entrant_entryTax}
\end{align}

From here, the rest of the model (including the case where $(1-\kappa_e)\lambda < 1$ and $\hat{z} = 0$) can be solved in a similar way as before (details in Appendix \ref{appendix:model:efficiencyderivations:CDtax}). 

\paragraph{Effect on growth}

Suppose that $\mathbbm{1}^{NCA} = 1$ and the tax is increased from $T_e$ to $T_e' > T_e$. Then (\ref{eq:effort_entrant_entryTax}) implies that $\hat{z}$ falls, (\ref{eq:labor_resource_constraint_entryTax}) implies that $z$ increase to keep $L_{RD} = \bar{L}_{RD}$. Following the same logic as Section \ref{model:efficiency:misallocationRD}, if (\ref{cs:growth_misallocation_condition}) holds, then growth increases. Intuitively, when $\mathbbm{1}^{NCA} = 1$ the only effect of the entry tax is to reduce the misallocation of R\&D labor to entrants. 

However, if $\mathbbm{1}^{NCA} = 0$, the situation changes, for two reasons. First, as can be seen readily in (\ref{eq:effort_entrant_entryTax}), the effect of $T_e$ on $\hat{z}$ is ambiguous, since the denominator now decreases in $T_e$ as well as the numerator. Intuitively, an increase in $T_e$ reduces the value of future spinouts, requiring incumbents to compensate workers with higher wages in equilibrum. However, the expected harm to incumbents from WSOs per unit of $z$ is unchanged. This follows from the assumption that potential WSOs arise as a by-product of working in R\&D rather than as a result of intentional side projects by R\&D workers. The net effect is that incumbents' effective cost of R\&D increases and R\&D labor is reallocated to the entrant. The mechanism in the previous paragraph is still present, however; the logic here only serves to attenuate the increase in growth from an increase in $T_e$ when (\ref{cs:growth_misallocation_condition}) holds with $\mathbbm{1}^{NCA} = 0$.

Second, by (\ref{eq:nca_policy_entryTax}) and (\ref{eq:barkappa_entryTax}), a sufficiently large increase in $T_e$ induces a change from $\mathbbm{1}^{NCA} = 0$ to $\mathbbm{1}^{NCA} = 1$. Intuitively, as mentioned in the previous paragraph, a higher $T_e$ means it is relatively more expensive for incumbents to compensate their employees with future spinouts as they are less valuable but cause the same harm to the incumbent. For a high enough $T_e$, incumbents prefer to use NCAs and pay their employees with wages directly. Using the logic of Section \ref{model:efficiency:misallocationNCAs}, this switch implies a reduction in growth.


\subsubsection{OI R\&D subsidy (tax)}

Suppose that the plannner can subsidize R\&D spent on improving a product while excluding R\&D aiming at creative destruction. In the model, this corresponds to a targeted subsidy to R\&D spending by incumbents, of magnitude $T_{RD,I}$ (tax if $T_{RD,I} < 0$). In practice, this policy may be difficult to implement for the same reason as the CD tax. Firms may not be expected to be truthful regarding the purpose of their R\&D or the effect of their R\&D on their competitors' profits. It may not even be possible to tell in advance whether R\&D will result in creative CD, OI, or even new varieties of products. Furthermore, innovation to improve existing products can be a form of creative destruction. Nevertheless, it is still useful as a theoretical benchmark.

In this case, the incumbent HJB can be rearranged to a form analogous to (\ref{eq:hjb_incumbent_workerIndiff}),
\begin{align}
(r + \hat{\tau}) \tilde{V} = \tilde{\pi} + \max_{\substack{x \in \{0,1\} \\ z \ge 0}} \Big\{z &\Big( \overbrace{\chi (\lambda - 1) \tilde{V}}^{\mathclap{\mathbb{E}[\textrm{Benefit from R\&D}]}}- (1-T_{RD,I}) \hat{w}_{RD} \\
&-  \underbrace{(1-\mathbbm{1}^{NCA})(1 - (1-T_{RD,I})(1-\kappa_{e})\lambda)\nu \tilde{V}}_{\mathclap{\text{Net cost from spinout formation}}} - \overbrace{\mathbbm{1}^{NCA} \kappa_{c} \nu \tilde{V}}^{\mathclap{\text{Direct cost of NCA}}}\Big) \Big\} \label{eq:hjb_incumbent_RDsubsidyTargeted_2}
\end{align}

The non-compete policy is the same as with untargeted R\&D subsidies. That is, define
\begin{align}
\tilde{\bar{\kappa}}_c = \tilde{\bar{\kappa}}_c(\kappa_e,\lambda;T_{RD}) = 1 - (1-T_{RD,I})(1-\kappa_e)\lambda
\end{align} 

Then $z > 0$ implies that the incumbent's optimal NCA policy is given by 
\begin{align}
\mathbbm{1}^{NCA} = \begin{cases}
1 & \textrm{if } \kappa_{c} < \tilde{\bar{\kappa}}_c  \\
0 & \textrm{if } \kappa_{c} > \tilde{\bar{\kappa}}_c \\
\{0,1\} & \textrm{if } \kappa_c = \tilde{\bar{\kappa}}_c 
\end{cases} \label{eq:nca_policy_RDsubsidyTargeted}
\end{align}

Using the same approach as before one obtains an expression for $\hat{z}$, 
\begin{align}
\hat{z} &= \Bigg( \frac{(1-T_{RD,I})\hat{\chi} (1-\kappa_{e}) \lambda}{\chi(\lambda -1) - \nu (x\kappa_c + (1-x)(1 - (1-T_{RD,I})(1-\kappa_e)\lambda)) } \Bigg)^{1/\psi} \label{eq:effort_entrant_RDsubsidyTargeted}
\end{align}

The rest of the equilibrium allocation and prices can be computed in a similar way as before (details in Appendix \ref{appendix:model:efficiencyderivations:OIRDtax}). 






\paragraph{Effect on growth}

If $\mathbbm{1}^{NCA} = 1$, increasing $T_{RD,I}$ reduces $\hat{z}$ by (\ref{eq:effort_entrant_RDsubsidyTargeted}) and will increase growth if the condition (\ref{cs:growth_misallocation_condition}) holds. Intuitively, a subsidy to incumbent R\&D causes the R\&D wage to increase, reducing R\&D by the entrant in equilibrium. 

If $\mathbbm{1}^{NCA} = 0$, increasing $T_{RD,I}$ has a more complicated effect on $\hat{z}$ because it reduces the denominator as well. This follows from the same reasoning as in the case of the untargeted R\&D subsidy: incumbents pay partially through future spinouts and so not all of their costs are subsidized at rate $T_{RD,I}$. From this economic interpretation, it follows immediately that the net effect is still to reduce incumbent R\&D expenses relative to those of the entrant and hence to lower $\hat{z}$ and increase $z$, and this is confirmed in the numerical analysis of the next subsection.

Finally, (\ref{eq:nca_policy_RDsubsidyTargeted}) implies that if the increase in $T_{RD,I}$ is sufficiently large, it will induce the use of NCAs by incumbents. As in the case of the untargeted R\&D subsidy, targeted R\&D subsidies do not reduce the harm to the incumbent's profits due to future employee spinouts. At a certain point, the incumbent prefers the higher but tax-deductible wages of an NCA contract. This switch unambiguously reduces growth.

The last observation implies that even targeted R\&D subsidies are unable to achieve the socially valuable outcome high spinout entry and high incumbent R\&D. In order to achieve this result, it is necessary to pair the targeted R\&D subsidy with an increase in legal barriers to NCAs $\kappa_c$ or an increase in the tax on NCA usage $T_{NCA}$. 

\subsubsection{All policies}

The BGP of the model when the planner can use all of the above policies simultaneously is derived in Appendix \ref{appendix:model:efficiencyderivations:allPolicies}. I discuss the growth and welfare implications of this case in detail in the quantitative analysis of Section \ref{sec:policy_analysis}. As a prelude, notice that with a combination of OI R\&D subsidies and an increase in $\kappa_c$, the social planner can achieve something like a "first best"\footnote{As noted previously, there is no well-defined first-best here. In fact, there is no well-defined second-best either because there is no well-defined equilibrium R\&D wage in a model where the social planner makes all R\&D decisions, and hence there is no well-defined value of the incumbent, which is necessary for computing consumption. Again, the model can easily be modified to allow this type of analysis, but the current one is more transparent.} where incumbents do the socially optimal amount of R\&D while still allowing for growth-enhancing spinouts to innovate.


\section{Empirics}\label{sec:empirics}

In this section I describe the empirics which are part of the calibration and quantitative analysis of the following two sections. I consider the microeconomic relationship between firm-level R\&D spending and employee spinout formation using a dataset I construct out of Venture Source, Compustat, and the NBER-USPTO patent database. I find a statistically robust and economically large relationship between R\&D spending and employee spinouts in the next few years. The results of this section help me quantify the role of R\&D in employee spinout formation, which allows me to discipline the parameter $\nu$ in the calibration of the next section. As discussed previously, this parameter determines the magnitude of the effect on growth of a reduction in $\kappa_c$. 

\subsection{Data}

\subsubsection{Sources}

\paragraph{VentureSource}

The data on startups comes from Venture Source (VS), a proprietary dataset containing information on venture capital (VC) firms and VC-funded startups.\footnote{When starting this project the data were owned by Dow Jones but they have since been sold to CB insights.} I use a subsample of the data for US-based startups founded between 1986 and 2008 which contain information on their founding year. The data cover 23,434 startups, 89,382 financing rounds, and 297,119 individual-firm pairs. For each financing round, the data contain information on valuation, amount raised, and status of the business at the time of the round -- employment, revenue, net income, burn rate -- albeit with substantial missing data. Most importantly for this analysis, the data contain employment biographies for each of the startup's founders and key employees (C-level, high-ranking executives and managers) and board members. In this regard, Venture Source is unique among VC investment databases. Some summary information about the dataset is contained in \autoref{table:VS_summaryTable}. The dataset is described in detail in \cite{kaplan_how_2002} and \cite{kaplan_venture_2016}. 

\paragraph{Compustat}

The data on R\&D spending comes from Compustat, a comprehensive database of fundamental financial and market information on publicly traded companies. I consider a subsample consisting of all firms headquartered in the United States in operation at any point between 1986 and 2006, consisting of 20,534 firms. In addition to data on R\&D spending, the Compustat data contain information on industrial classification and time-varying firm variables such as market value, tangible and intangible assets, employees, sales, etc.

\paragraph{NBER-USPTO}

The NBER-USPTO database contains comprehensive information on all patents granted in the United States from 1976 to 2006, and is linked to Compustat. I consider the subsample of patents assigned to US firms, consisting of 1,457,136 patents. 

\subsubsection{Construction of dataset}

\paragraph{Classifying founders}

The Venture Source data contain information on high level employees and board members. For the purposes of this study, however, not all of these employees should be considered founders of the startup in question. In particular, only those employees whose human capital is crucial to the value proposition of the startup should be considered founders. 

To this end, I first restrict attention to employees who join a startup in its first three years. When information on the individual's date of joining the startup is missing, I impute it as the founding date of the startup. I also conduct robustness exercises where I exclude these individual-startup observations. 

Next, I only consider employees whose job titles relate to the core operations of the firm. \autoref{table:VS_titlesSummaryTable} shows a breakdown of the 20 most frequent titles. Nearly 35\% of named employees are outside board members (e.g. VC investors). For the purpose of this study, I define a \textit{founder} as an employee with the title Founder, Chief, CEO, CTO or President.\footnote{I explore how the analysis changes with a different definition of founder in Section \textbf{XYZ}.}

% latex table generated in R 3.6.3 by xtable 1.8-4 package
% Tue Jul  7 15:40:11 2020
\begin{table}[!htb]
\centering
\begingroup\footnotesize
\begin{tabular}{rll}
  \toprule
Title & Individuals & Percentage \\ 
  \midrule
Board member (outsider) & 103367 & 34.6 \\ 
  Vice President & 59149 & 19.8 \\ 
  Chief Executive Officer & 24230 & 8.1 \\ 
  Chief Technology Officer & 13971 & 4.7 \\ 
  Chief Financial Officer & 11621 & 3.9 \\ 
  Director & 10988 & 3.7 \\ 
  Chief & 10846 & 3.6 \\ 
  President \& CEO & 9088 & 3.0 \\ 
  Senior Vice President & 8700 & 2.9 \\ 
  Founder & 8471 & 2.8 \\ 
  Chief Operating Officer & 6777 & 2.3 \\ 
  President & 5441 & 1.8 \\ 
  Chairman & 5029 & 1.7 \\ 
  Executive Vice President & 4920 & 1.6 \\ 
  Chairman \& CEO & 2755 & 0.9 \\ 
  Manager & 2357 & 0.8 \\ 
  Chief Scientific Officer & 1461 & 0.5 \\ 
  Controller & 1137 & 0.4 \\ 
  President \& COO & 1134 & 0.4 \\ 
  General Counsel & 1056 & 0.4 \\ 
   \bottomrule
\end{tabular}
\endgroup
\caption{Top 20 most frequent titles among founders in VS data.} 
\label{table:VS_titlesSummaryTable}
\end{table}


\paragraph{Extracting information on the most recent employer}

In order to relate the activities of employers to the entrepreneurship behavior of their employees, I link the Compustat data to the VS data using information in employee biographies. Because VS biographies are text fields, this requires matching entries by name to firm names in Compustat.  

The VS biographical data comes in a structured format, allowing parsing by regular expressions. Each prior job is represented in the format ``<position>, <employer>'' and different jobs are separated by ``;''. Job spells can be easily separated by splitting the string on the character ``;''. It is slightly more involved to separate positions from employer. It is not sufficient to simply separate on the right-most character ``,'' as <employer> can contain ``,''. However, in almost all cases, <employer> contains at most one ``,'' (e.g., in ``Microsoft, Inc.''), and in virtually all of these cases, the comma precedes one of a few strings (e.g. ``LLC'',``Inc'',``Corp''). Hence, I use a two-pass approach: first I split on the last ``,''; for employers that end up consisting only of corporate structure (e.g., ``LLC'', etc.), I split on the penultimate ``,'' instead. 

The above procedure yields a dataset containing, for each individual, all of his or her previous positions and employers. However, because an individual can take various jobs over the years at an individual startup, there are individuals whose most recent employer coincides with their startup. I exclude these cases by comparing the previous employer with the \texttt{EntityName} text field. Because these are both text fields with potentially different formatting, this entails two steps. First, I bring both fields to a common format, eliminating endings such as ``Inc.'', ``Corp.'' etc which may vary across them, and converting to lower case. I then exclude observations where the strings either exactly coincide or one contains the other. 

The results of this procedure are summarized in \autoref{table:VS_previousEmployersSummaryTable}. The top previous employers include several well-known technology firms such as Microsoft, IBM, Google, and Oracle. However, notice that many of the top previous employers are VC firms (e.g., New Enterprise Associates, Sequoia Capital, Kleiner Perkins), and the top employer is the unaffiliated category "Individual Investor." This is largely because many of the individuals affiliated with startups are investors turned board members: about 33\% of the individual-startup observations are outside board members, As my focus is on the flow of knowledge from previous employers to new startups, I restrict attention to individuals with knowledge-related and/or executive titles. Finally, notice that Stanford University is also listed in the top 20. Several other prominent research universities are also in the top 50. 

% latex table generated in R 3.6.3 by xtable 1.8-4 package
% Wed Nov 25 14:00:10 2020
\begin{table}[]
\centering
\begingroup\normalsize
\begin{tabular}{rlrll}
  \toprule
Employer & Count & Position & Count & Percentage \\ 
  \midrule
IBM & 174 & Executive & 3645 & 9.3 \\ 
  Microsoft & 168 & CEO & 2611 & 6.7 \\ 
  Cisco Systems & 122 & President \& CEO & 2325 & 6.0 \\ 
  Oracle & 109 & President & 1445 & 3.7 \\ 
  Verizon & 101 & CTO & 1439 & 3.7 \\ 
  Sun Microsystems & 94 & Founder & 1319 & 3.4 \\ 
  Google & 79 & Cofounder & 771 & 2.0 \\ 
  AT\&T & 76 & Board Member, Institutional Investor & 557 & 1.4 \\ 
  Intel & 70 & VP & 543 & 1.4 \\ 
  Hewlett-Packard & 64 & Chairman \& CEO & 501 & 1.3 \\ 
  Stanford University & 56 & COO & 467 & 1.2 \\ 
  Lucent Technologies & 46 & Founder \& CEO & 456 & 1.2 \\ 
  AOL & 44 & Chief Executive Officer & 346 & 0.9 \\ 
  Motorola & 42 & President \& COO & 309 & 0.8 \\ 
  Andersen Consulting & 40 & Partner & 288 & 0.7 \\ 
  Nortel Networks & 40 & SVP & 277 & 0.7 \\ 
  MIT & 40 & Director & 266 & 0.7 \\ 
  McKinsey \& Company & 39 & VP, Engineering & 261 & 0.7 \\ 
  Texas Instruments & 38 & Managing Director & 245 & 0.6 \\ 
  Apple & 37 & Chairman & 240 & 0.6 \\ 
   \bottomrule
\end{tabular}
\endgroup
\caption{Top 20 previous employers and previous positions for founder2 founders in VS data.} 
\label{table:VS_previousEmployersSummaryTable}
\end{table}


\paragraph{Linking to Compustat}

The data on prior employers is matched to the variable \texttt{conml} in Compustat. To do this, first I standardize names as before, using regular expressions to trim e.g. ``Inc.", ``Corp.'' and variants thereof from each entry and converting to lower case. I look for exact matches to previous employers in the VS data. For previous employers in VS that do not match with any names in Compustat, I check against the business segment names, available from the Compustat Segments database. 

\paragraph{Defining WSOs}

This study emphasizes the importance of competition between spinouts and their parent firms. The best measure I have for the product market of publicly traded firms is their self reported NAICS code. While VS does not contain NAICS classifications for its startups, it does document their industry using a classification that, for the most part, coincides with NAICS 4 or 5 digit categories. I manually construct a crosswalk between the two classification schemes and use this to assign 4-digit NAICS codes to startups in VS.\footnote{An alternative would be to us VS's "Competition" variable, which documents directly the competitors of the startup observation. However, only 20\% of startups have this variable filled in: 30\% in the 90s, but dropping to around 10\% by the end of the sample.} Then, I classify a founder-startup observation as a WSO whenever the startup is in the same 4-digit NAICS category as its parent. 

\paragraph{Evaluating the match}

\autoref{table:GStable_founder2} documents the quality of this match. It corresponds roughly to Table 1 of \cite{gompers_entrepreneurial_2005}.\footnote{The numbers are different. I find a similar number of founders from public companies, but a substantially smaller fraction. I suspect this is due to startups being added to the data retroactively since the time of that article \textbf{[ask VS]}.} About 20\% of founders have a most recent previous employer that matches to a public firm in Compustat. Nearly half of those came from an employer in the same four digit NAICS industry as the startup.

% latex table generated in R 3.6.3 by xtable 1.8-4 package
% Sat Sep 26 15:59:48 2020
\begin{sidewaystable}[!htb]
\centering
\begingroup\tiny
\begin{tabular}{p{1.75cm}p{1.75cm}p{1.75cm}p{1.75cm}p{1.75cm}p{1.75cm}p{1.75cm}p{1.75cm}}
  \toprule
Year & Number of founders & Number of start-ups & Number of founders from public companies & Fraction from public companies (\%) & Fraction from public companies when bio. info available (\%) & Fraction from public companies in same 4-digit NAICS (\%) & Fraction from public companies in same 4-digit NAICS when bio. info available (\%) \\ 
  \midrule
1986 & 269 & 216 & 45 & 16.7 & 22.8 & 5.2 & 7.1 \\ 
  1987 & 356 & 280 & 43 & 12.1 & 15.1 & 3.9 & 4.9 \\ 
  1988 & 372 & 281 & 58 & 15.6 & 19.9 & 4.6 & 5.8 \\ 
  1989 & 479 & 341 & 75 & 15.7 & 19.2 & 4.2 & 5.1 \\ 
  1990 & 478 & 329 & 85 & 17.8 & 21.1 & 6.3 & 7.5 \\ 
  1991 & 540 & 356 & 81 & 15.0 & 17.9 & 6.3 & 7.5 \\ 
  1992 & 674 & 450 & 100 & 14.8 & 17.9 & 3.3 & 3.9 \\ 
  1993 & 778 & 490 & 137 & 17.6 & 20.3 & 6.7 & 7.7 \\ 
  1994 & 999 & 611 & 167 & 16.7 & 19.3 & 4.9 & 5.7 \\ 
  1995 & 1326 & 772 & 224 & 16.9 & 19.0 & 5.2 & 5.8 \\ 
  1996 & 1926 & 1077 & 319 & 16.6 & 18.1 & 4.9 & 5.3 \\ 
  1997 & 1986 & 1036 & 345 & 17.4 & 19.0 & 5.9 & 6.5 \\ 
  1998 & 2895 & 1390 & 541 & 18.7 & 19.6 & 5.2 & 5.5 \\ 
  1999 & 5189 & 2388 & 975 & 18.8 & 19.6 & 5.0 & 5.2 \\ 
  2000 & 4084 & 1832 & 786 & 19.2 & 20.4 & 5.2 & 5.5 \\ 
  2001 & 2245 & 948 & 384 & 17.1 & 18.7 & 6.3 & 6.9 \\ 
  2002 & 2113 & 884 & 385 & 18.2 & 20.1 & 7.3 & 8.0 \\ 
  2003 & 1979 & 903 & 344 & 17.4 & 19.8 & 7.5 & 8.5 \\ 
  2004 & 2098 & 988 & 365 & 17.4 & 20.1 & 6.8 & 7.9 \\ 
  2005 & 2278 & 1068 & 400 & 17.6 & 20.7 & 6.5 & 7.7 \\ 
  2006 & 2492 & 1212 & 432 & 17.3 & 20.5 & 6.3 & 7.5 \\ 
  2007 & 2817 & 1366 & 388 & 13.8 & 17.0 & 4.9 & 6.1 \\ 
  2008 & 2710 & 1307 & 422 & 15.6 & 19.1 & 5.4 & 6.6 \\ 
   \bottomrule
\end{tabular}
\endgroup
\caption{\scriptsize Summary of founders. Here, "founder" includes all individuals employed at startups inthe VentureSource database who (1) joined the startup within 3 year(s) of its founding year; and (2) have the title of CEO, CTO, CCEO, PCEO, PRE, PCHM, PCOO, FDR, CHF.} 
\label{table:GStable_founder2}
\end{sidewaystable}


To get a sense of the importance of within-industry spinouts in the data, \autoref{figure:industry_row_heatmap_naics2_founder2} and \autoref{figure:industry_column_heatmap_naics2_founder2} document the joint distribution of parent industry and child industry, defined by 2-digit NAICS codes. The raw joint distribution is too heavily concentrated to be easily visualized in this way, so instead I show the distribution of child industry (parent industry) conditional on parent industry (child industry), displayed in \autoref{figure:industry_row_heatmap_naics2_founder2} (\autoref{figure:industry_column_heatmap_naics2_founder2}). The dark diagonal lines in both figures reflects the prevalence of WSOs.\footnote{In \autoref{figure:industry_row_heatmap_naics2_founder2}, the dark vertical line at column 51 (Information) indicates that parent firms of all industries tend to spawn spinouts in that industry. Similar dark regions appear at columns 54 (Professional, Scientific and Technical Services), and 32 and 33 (Manufacturing). In \autoref{figure:industry_column_heatmap_naics2_founder2}, the dark horizontal lines at 51 and to a lesser extend 32, 33, 52 and 54 indicate that child firms of all industries tend to have founders from those industries.}

\begin{figure}[!htb]
	\centering
	\includegraphics[scale=0.65]{../empirics/figures/plots/industry_row_heatmap_naics2_founder2.pdf}
	\caption{Heatmap displaying the distribution of child 2-digit NAICS code (column), conditional on parent NAICS code (row). Darker hues indicate a higher density.}
	\label{figure:industry_row_heatmap_naics2_founder2}
\end{figure}

\subsection{Corporate R\&D and spinout formation}\label{subsec:empirics:corpRDandspinouts}

In this section, I consider the determinants of spinout formation. This is relevant to the general equilibrium consequences of spinouts. If employee spinout formation -- and, in particular, WSO4 formation -- is a consequence of parent firm decisions, then it could affect the parent firm decision making process, altering the general equilibrium consequences of facilitating WSO4 spinout formation by, e.g., prohibiting non-compete agreements. 

In particular, I focus on whether R\&D expenditures tend to produce employee spinouts. As discussed in the introduction, this is theoretically plausible because (1) employees undertaking innnovation must be trained, exposing and them to the firm's existing knowledge stock, and (2) such employees also may develop new ideas which may not be implementable within the firm.  

The purpose of this section is to provide discipline on the parametrization of the model used in the quantitative analysis. That model hypothesizes that R\&D by parent firms leads to a flow of employees into starting new spinout firms, and assumes that parent firms internalize this \textit{causal} relationship when making R\&D decisions. Therefore, the validity of my quantitative experiments depends crucially on whether the relationship established in this section is in fact causal. 

This presents a challenge, as many variables can be thought to simultaneously affect both corporate R\&D and employee entrepreneurship. For example, an innovative incumbent may have high R\&D expenses and also hire the most innovative employees who then start WSOs which they would have started regardless of being hired to do R\&D at the incumbent. Furthermore, there may be time-varying shocks to innovative investment opportunities, either at the aggregate level, or at the industry, state, or even industry-state level. 

I address this challenge in the standard way by using a regression analysis. I control for observable confounders and using multiple fixed effects to absorb contaminating variation from shocks which are not directly observable. Based on these regressions, I obtain suggestive evidence of a statistically and economically significant causal effect of corporate R\&D on WSO formation.

\subsubsection{Preliminaries}

\autoref{figure:scatterPlot_RD-Founders} shows a scatterplot illustrating the relationship between R\&D spending in years $t-2,t-1,t$ and employee entrepreneurship in years $t+1,t+2,t+3$. The dashed line shows the fit of a straight line through all of the points. The solid line shows the fit of a line only through firm-year observations with nonzero number of employee founders. The graph shows a positive relationship. \autoref{figure:scatterPlot_RD-Founders_dIntersection} shows the relationship between deviations from firm and State-industry-age-year means. The positive relationship remains. Finally, \autoref{figure:scatterPlot_RD-FoundersWSO4_dIntersection} shows that the same positive relationship holds when considering only WSO4 spinouts. 


\subsubsection{Regressions}

\autoref{table:RDandSpinoutFormation_absolute_founder2_l3f3} displays the results of a regression analysis relating employee entrepreneurship to parent firm R\&D spending. The dependent variable $Y_{it}$ is again the (annualized) number of founders previously employed at firm $i$ joining startups in years $t+1,t+2,t+3$. The independent variables $X_{it}$ are moving averages over years $t,t-1,t-2$. 

The first three columns consider the effect of R\&D on the number of founders leaving the parent firm. These regressions find a positive coefficient which is statistically significant at the 1\% level, even after including firm and year fixed effects. The magnitude of the coefficient indicates that in 2014, three billion dollars of R\&D over three years leads to on average 1.5 founders leaving to found new firms in the next three years.\footnote{The reason for the dependence on the year 2014 is that the specification assumes that the amount of R\&D that leads to a founder leaving grows at the rate of aggregate productivity growth.} 

The robustness of the result to the inclusion of age, industry-year, and State-year fixed effects is encouraging. However, in this context such fixed effects may not absorb much contaminating variation due to what amounts to a misspecification problem: shocks are likely to affect firm outcomes more in absolute terms for larger firms. Because firms vary in size, the fixed effect -- which must be constant in absolute terms for all firms -- leaves much firm-level variation unabsorbed. This is the same reason for the inclusion of Tobin's Q $\times$ Assets, rather than Tobin's Q. 

To address this, \autoref{table:RDandSpinoutFormation_at_founder2_l3f3} displays the results of a regression analysis where all variables are normalized by a trailing 5-year moving average of firm assets. Normalizing in this way improves the specification of the fixed effects, although it is not a panacea as it leaves unaddressed the problem of variation in firm R\&D / asset ratios. The magnitudes of the estimates of the coefficient on the measure of R\&D are strikingly similar. This is particularly true once the full battery of fixed effects is included. Normalizing by assets throws away any variation in absolute firm levels of R\&D, reducing power substantially. In spite of this, the most stringent estimate is significant at the 10\% level.

Finally \autoref{table:RDandSpinoutFormation_ppml_absolute_founder2_l3f3} shows the results of a Poisson pseudo-Maximum Likelihood estimation. This can be thought of as a log-linear regression which is more robust to zeros in the dependent variable.\footnote{Specifically, for any group (as defined by fixed effects), a group's observations are dropped iff the dependent variable equals zero for all observations in that group. In log-linear regression, all zero observations are dropped.} Robustness analysis for each of these specifications (i.e. varying controls, fixed effects, and clustering) is contained in the appendix, specifically robustness of the OLS regression with levels is analyzed in \ref{figure:speccheck2_levels_reghdfe} and  \ref{figure:speccheck2_levels_wso4_reghdfe}; of the asset-normalized OLS regressions in figures \ref{figure:speccheck2_at_reghdfe} and \ref{figure:speccheck2_at_wso4_reghdfe}; and of the PPML regressions in figures \ref{figure:speccheck2_levels_ppmlhdfe} and \ref{figure:speccheck2_levels_wso4_ppmlhdfe}.



\begin{table}[!htb]
	\scriptsize
	\centering
	{
\def\sym#1{\ifmmode^{#1}\else\(^{#1}\)\fi}
\begin{tabular}{l*{8}{c}}
\toprule
                    &\multicolumn{1}{c}{(1)}&\multicolumn{1}{c}{(2)}&\multicolumn{1}{c}{(3)}&\multicolumn{1}{c}{(4)}&\multicolumn{1}{c}{(5)}&\multicolumn{1}{c}{(6)}&\multicolumn{1}{c}{(7)}&\multicolumn{1}{c}{(8)}\\
                    &\multicolumn{1}{c}{Founders}&\multicolumn{1}{c}{Founders}&\multicolumn{1}{c}{Founders}&\multicolumn{1}{c}{Founders}&\multicolumn{1}{c}{WSO4}&\multicolumn{1}{c}{WSO4}&\multicolumn{1}{c}{WSO4}&\multicolumn{1}{c}{WSO4}\\
\midrule
R\&D                &        0.34\sym{**} &        0.73\sym{***}&        0.73\sym{***}&        0.63\sym{***}&        0.19\sym{***}&        0.32\sym{***}&        0.31\sym{***}&        0.28\sym{***}\\
                    &      (0.13)         &      (0.24)         &      (0.23)         &      (0.14)         &     (0.045)         &     (0.067)         &     (0.065)         &     (0.050)         \\
\addlinespace
No FE               &         Yes         &          No         &          No         &          No         &         Yes         &          No         &          No         &          No         \\
\addlinespace
Firm FE             &          No         &         Yes         &         Yes         &         Yes         &          No         &         Yes         &         Yes         &         Yes         \\
\addlinespace
Year FE             &          No         &         Yes         &          No         &          No         &          No         &         Yes         &          No         &          No         \\
\addlinespace
Age FE              &          No         &          No         &         Yes         &          No         &          No         &          No         &         Yes         &          No         \\
\addlinespace
Industry-Age FE     &          No         &          No         &          No         &         Yes         &          No         &          No         &          No         &         Yes         \\
\addlinespace
Industry-Year FE    &          No         &          No         &         Yes         &          No         &          No         &          No         &         Yes         &          No         \\
\addlinespace
State-Year FE       &          No         &          No         &         Yes         &          No         &          No         &          No         &         Yes         &          No         \\
\addlinespace
Industry-State-Year FE&          No         &          No         &          No         &         Yes         &          No         &          No         &          No         &         Yes         \\
\midrule
r2\_a                &        0.24         &        0.69         &        0.69         &        0.75         &        0.21         &        0.65         &        0.64         &        0.61         \\
r2\_a\_within         &        0.24         &        0.26         &        0.26         &        0.24         &        0.21         &        0.25         &        0.23         &        0.15         \\
N                   &       65009         &       63732         &       62211         &       37810         &       65009         &       63732         &       62211         &       37810         \\
\bottomrule
\multicolumn{9}{l}{\footnotesize Standard errors in parentheses}\\
\multicolumn{9}{l}{\footnotesize \sym{*} \(p<0.1\), \sym{**} \(p<0.05\), \sym{***} \(p<0.01\)}\\
\end{tabular}
}

	\caption{The regressions above relate corporate R\&D to the entrepreneurship decisions of employees. The dependent variable is average yearly number of founders joining startups in years $t+1,t+2,t+3$. The independent variables are averages over $t,t-1,t-2$. Firm controls are employment, assets, intangible assets, investment, net income, cumulative citation-weighted patents, and the product of Tobin's Q and Assets (i.e., firm market value). Standard errors are clustered by firm in columns (1)-(3) and (5)-(7). In columns (4) and (8), standard errors are multway clustered by State and 4-digit NAICS industry.}
	\label{table:RDandSpinoutFormation_absolute_founder2_l3f3}
\end{table}

\begin{table}[!htb]
	\scriptsize
	\centering
	{
\def\sym#1{\ifmmode^{#1}\else\(^{#1}\)\fi}
\begin{tabular}{l*{8}{c}}
\toprule
                    &\multicolumn{1}{c}{(1)}&\multicolumn{1}{c}{(2)}&\multicolumn{1}{c}{(3)}&\multicolumn{1}{c}{(4)}&\multicolumn{1}{c}{(5)}&\multicolumn{1}{c}{(6)}&\multicolumn{1}{c}{(7)}&\multicolumn{1}{c}{(8)}\\
                    &\multicolumn{1}{c}{$\frac{\textrm{Founders}}{\textrm{Assets}}$}&\multicolumn{1}{c}{$\frac{\textrm{Founders}}{\textrm{Assets}}$}&\multicolumn{1}{c}{$\frac{\textrm{Founders}}{\textrm{Assets}}$}&\multicolumn{1}{c}{$\frac{\textrm{Founders}}{\textrm{Assets}}$}&\multicolumn{1}{c}{$\frac{\textrm{WSO4}}{\textrm{Assets}}$}&\multicolumn{1}{c}{$\frac{\textrm{WSO4}}{\textrm{Assets}}$}&\multicolumn{1}{c}{$\frac{\textrm{WSO4}}{\textrm{Assets}}$}&\multicolumn{1}{c}{$\frac{\textrm{WSO4}}{\textrm{Assets}}$}\\
\midrule
$\frac{\textrm{R\&D}}{\textrm{Assets}}$&        1.71\sym{***}&        1.12         &        1.15         &        0.28         &        0.82\sym{***}&        0.62\sym{**} &        0.57         &        0.77         \\
                    &      (0.34)         &      (0.68)         &      (0.71)         &      (1.05)         &      (0.17)         &      (0.32)         &      (0.36)         &      (0.88)         \\
\addlinespace
NAICS4-State-Age-Year FE&          No         &          No         &          No         &         Yes         &          No         &          No         &          No         &         Yes         \\
\addlinespace
NAICS4-Year FE      &          No         &          No         &         Yes         &          No         &          No         &          No         &         Yes         &          No         \\
\addlinespace
State-Year FE       &          No         &          No         &         Yes         &          No         &          No         &          No         &         Yes         &          No         \\
\addlinespace
Firm FE             &          No         &         Yes         &         Yes         &         Yes         &          No         &         Yes         &         Yes         &         Yes         \\
\addlinespace
Age FE              &          No         &          No         &         Yes         &          No         &          No         &          No         &         Yes         &          No         \\
\addlinespace
Year FE             &          No         &         Yes         &          No         &          No         &          No         &         Yes         &          No         &          No         \\
\addlinespace
No FE               &         Yes         &          No         &          No         &          No         &         Yes         &          No         &          No         &          No         \\
\midrule
r2\_a                &       0.014         &        0.26         &        0.22         &        0.25         &      0.0088         &        0.27         &        0.21         &        0.40         \\
r2\_a\_within         &       0.014         &      0.0028         &      0.0025         &      0.0014         &      0.0088         &      0.0014         &      0.0012         &      0.0046         \\
N                   &       60687         &       59477         &       58201         &       23665         &       60687         &       59477         &       58201         &       23665         \\
\bottomrule
\multicolumn{9}{l}{\footnotesize Standard errors in parentheses}\\
\multicolumn{9}{l}{\footnotesize \sym{*} \(p<0.1\), \sym{**} \(p<0.05\), \sym{***} \(p<0.01\)}\\
\end{tabular}
}

	\caption{The regressions above relate corporate R\&D to the entrepreneurship decisions of employees. The dependent variable is the average yearly number of founders from the parent firm joining startups in years $t+1,t+2,t+3$, normalized by a trailing five-year moving average of assets. Independent variables are also normalized by assets. Standard errors are clustered at the firm level.}
	\label{table:RDandSpinoutFormation_at_founder2_l3f3}
\end{table}

\begin{table}[!htb]
	\scriptsize
	\centering
	{
\def\sym#1{\ifmmode^{#1}\else\(^{#1}\)\fi}
\begin{tabular}{l*{8}{c}}
\toprule
                    &\multicolumn{1}{c}{(1)}&\multicolumn{1}{c}{(2)}&\multicolumn{1}{c}{(3)}&\multicolumn{1}{c}{(4)}&\multicolumn{1}{c}{(5)}&\multicolumn{1}{c}{(6)}&\multicolumn{1}{c}{(7)}&\multicolumn{1}{c}{(8)}\\
                    &\multicolumn{1}{c}{Founders}&\multicolumn{1}{c}{Founders}&\multicolumn{1}{c}{Founders}&\multicolumn{1}{c}{Founders}&\multicolumn{1}{c}{WSO4}&\multicolumn{1}{c}{WSO4}&\multicolumn{1}{c}{WSO4}&\multicolumn{1}{c}{WSO4}\\
\midrule
log(R\&D)           &        0.81\sym{***}&        0.49\sym{***}&        0.50\sym{**} &        0.50\sym{***}&        1.54\sym{***}&        0.55\sym{***}&        1.28\sym{***}&        1.28\sym{***}\\
                    &     (0.098)         &      (0.15)         &      (0.22)         &      (0.13)         &      (0.13)         &      (0.20)         &      (0.44)         &      (0.49)         \\
\addlinespace
No FE               &         Yes         &          No         &          No         &          No         &         Yes         &          No         &          No         &          No         \\
\addlinespace
Firm FE             &          No         &         Yes         &         Yes         &         Yes         &          No         &         Yes         &         Yes         &         Yes         \\
\addlinespace
Year FE             &          No         &         Yes         &          No         &          No         &          No         &         Yes         &          No         &          No         \\
\addlinespace
Age FE              &          No         &          No         &         Yes         &         Yes         &          No         &          No         &         Yes         &         Yes         \\
\addlinespace
Industry-Year FE    &          No         &          No         &         Yes         &         Yes         &          No         &          No         &         Yes         &         Yes         \\
\addlinespace
State-Year FE       &          No         &          No         &         Yes         &         Yes         &          No         &          No         &         Yes         &         Yes         \\
\midrule
Clustering          &       gvkey         &       gvkey         &       gvkey         &naics4 Statecode         &       gvkey         &       gvkey         &       gvkey         &naics4 Statecode         \\
pseudo R-squared    &        0.44         &        0.49         &        0.54         &        0.54         &        0.45         &        0.38         &        0.38         &        0.38         \\
Observations        &        7434         &        2416         &        1335         &        1335         &        7434         &         898         &         436         &         436         \\
\bottomrule
\multicolumn{9}{l}{\footnotesize Standard errors in parentheses}\\
\multicolumn{9}{l}{\footnotesize \sym{*} \(p<0.1\), \sym{**} \(p<0.05\), \sym{***} \(p<0.01\)}\\
\end{tabular}
}

	\caption{Poisson pseudo-Maximum Likelihood Regression. The dependent variable is average yearly number of founders joining startups in years $t+1,t+2,t+3$. The independent variables are in log terms and averages over $t,t-1,t-2$ Firm controls are employment, assets, intangible assets, investment, net income, cumulative citation-weighted patents, and the product of Tobin's Q and Assets (i.e., firm market value). Standard errors are clustered by firm in columns (1)-(3) and (5)-(7). In columns (4) and (8), standard errors are multi-way clustered by State and 4-digit NAICS industry.}
	\label{table:RDandSpinoutFormation_ppml_absolute_founder2_l3f3}
\end{table}

\subsubsection{Economic magnitude}

\textbf{[Remake figure with only WSO spinouts]} \textbf{[Make my aggregation assumptions more explicit]} In each year $t$, I compute $\tilde{y}_{it}$, the expected number of founders per year starting firms in years $t+1,t+2,t+2$ by multiplying the R\&D in years $t,t-1,t-2$ by the relevant coefficient estimate. I then plot this against the realizations of $y_{it}$.  \autoref{figure:founder2_founders_f3_Accounting} provides a visualization of the economic magnitude of the coefficient estimates. The left column is for all founders and the right column is for founders of firms in the same 4-digit NAICS industry as the parent. 

Based on this measure, the regression estimates are economically significant, accounting for roughly all of the WSO spinouts observed in the data. Moreover, in the Appendix, tables \ref{table:startupLifeCycle_founder2founders_lemployeecount_founder2}, \ref{table:startupLifeCycle_founder2founders_lrevenue_founder2}, and \ref{table:startupLifeCycle_founder2founders_lpostvalusd_founder2} document that startups with a higher fraction of WSO4 founders tend to have roughly 35\% higher employee count, revenue, and valuation on a per-founder basis. This relationship holds after controlling for industry, state, time, cohort and life-cycle factors, and is highly statistically significant and robust across specifications. It implies that R\&D-induced WSOs account for about 12-13\% of employment, revenue and valuation of startups in the dataset.

\begin{figure}[!htb]
	\includegraphics[scale=0.5]{../empirics/figures/founder2_founders_f3_Accounting.pdf}
	\caption{Economic magnitude of regression estimates in Tables \ref{table:RDandSpinoutFormation_absolute_founder2_l3f3} and \ref{table:RDandSpinoutFormation_at_founder2_l3f3}. The first row of figures compares the predicted number of employee founders (dotted lines) to the observed number of employee founders (solid lines). The left figure considers all founders, the right figure only founders of firms in the same 4-digit NAICS industry as their previous employers. The bottom row shows the percentage explained in each year.}
	\label{figure:founder2_founders_f3_Accounting}
\end{figure}

\section{Calibration}\label{sec:calibration}

\subsection{Parameters}

The model has ten parameters given by $\{\rho, \theta, \beta, \psi, \lambda, \chi, \hat{\chi}, \kappa_e, \kappa_c, \nu, \bar{L}_{RD}\}$. The parameter $\bar{L}_{RD}$ is calibrated to data from the NSF.\footnote{The parameter $\bar{L}_{RD}$ is only a normalization (a choice of units in which to measure the R\&D labor endowment) when it comes to determining the growth rate in the model; however, it does affect the model's implications for the employment share in firms of different ages. As this moment will be a target in the calibration, it is necessary to calibrate $\bar{L}_{RD}$.} The elasticity parameters $\{\theta, \psi\}$ are chosen to match estimates from the literature. The remaining seven parameters $\{\rho, \lambda, \chi, \hat{\chi}, \kappa_e, \kappa_c, \nu\}$ pertain to preferences ($\rho$) and to the technology for innovation and NCA usage (all others), and are chosen to match six moments from the data. One parameter, $\kappa_c$, is partially identified as $\kappa_C > \bar{\kappa}_c$ by the observation that $\tau^S > 0$. The remaining six parameters are exactly identified and the model reproduces the target moments exactly. I discuss the sources of identification in Section \ref{subsec:identification}.

\subsection{Targets}

The targets of the calibration are displayed in \autoref{calibration_targets} and consist of the labor productivity growth rate, the R\&D / GDP ratio, the share of growth coming from OI, the real interest rate, the employment share of entering firms, and the employment share of R\&D-induced WSOs. Matching the productivity growth rate, R\&D / GDP ratio and growth share of OI helps calibrate the efficiency of R\&D in generating aggregate productivity growth through OI and CD. The interest rate, profit / GDP ratio and employment share of entering firms determines the discount factor and the reward to innovation, which is in turn determined by the flow payoff to innovation (profit / GDP) and the expected duration of an incumbency position (employment share of entering firms). Finally, matching the employment share of entering WSOs allows the model to capture the rate at which R\&D by incumbents increases their likelihood of being replaced by a WSO. This calibrates the magnitude of the disincentive to OI R\&D.

Below, I discuss issues pertaining to the measurement of the target moments. In particular, in this section I discuss how the results in Section \ref{subsec:empirics:corpRDandspinouts} are used to calibrate the model.

\paragraph{Growth rate}

The growth rate is calibrated to the growth in labor productivity due to CD and OI, as calculated in \cite{klenow_innovative_2020}. [\textbf{Insert description of how they identify}]

\paragraph{R\&D spending / GDP}

The data on R\&D spending is from the National Patterns of R\&D resources.\footnote{I take the average of business-funded R\&D business-performed R\&D.} In the data, about half of R\&D spending is wages for employees; in the model, the only input to R\&D is labor. I opt to match the model's aggregate R\&D intensity to that in the data, including costs other than labor. This means that the model captures the full cost of innovation. The computation of the corresponding model moment is described in \ref{appendix:calibration:rd/gdp}.

\paragraph{Growth share of older firms}

The growth share of OI is calibrated to the growth share of OI as a fraction of OI and CD innovations, as estimated in \cite{garcia-macia_how_2019} and \cite{klenow_innovative_2020} using models similar to this one. On average they find that, from 1982 to 2013\footnote{The end points are not exactly these in their data.}, roughly 65\% of CD + OI productivity growth was due to firms at least 6 years old. The computation of the corresponding model moment is described in \ref{appendix:calibration:growthShareOI}.


\paragraph{Interest rate}

The short-term risk-free real interest rate averages about 5\% in the United States from 1986-2006. However, the real interest rate in the model actually corresponds to the discount factor used to price an unlevered firm. Since there is no systemic risk in the model, these are the same; however, since the data exhibits systemic risk, unlevered firms require a higher return than 5\% in the data. 

To adjust for this, I use a back of the envelope calculation to calculate the asset beta from the equity betas and leverage ratios, and hence compute the hypothetical risk-premium on an unlevered investment. First, the real return on the S\&P 500 in the time period 1986-2006 averaged about 7\%. The average debt-value ratio of the S\&P 500 in the US is about 40\% during this period. Assuming that this corporate debt does not earn a risk premium, the entire risk premium accrues to the equity. If there were no leverage, the risk premium would be smaller in percentage terms, since it is accruing to a larger value investment. Quantitatively, we need to multiply the excess return by $E / (D + E)$, which in this case is $1 - 40\% = 60\%$. I arrive at a calibration value of about 6\% for the real interest rate in the model.

\paragraph{Profits \% GDP} 

The data on aggregate profits as a percent of GDP comes from the BEA (computed as an average during the sample period of 1986-2008). In the model, this ratio is simple to calculate using the solution to the static equilibrium as $\tilde{\pi} / \tilde{Y}$.



\paragraph{Entry rate}

The entry rate target deserves some discussion. The purpose of including entry in the model is to capture the rate at which incumbent profits are destroyed due to creative destruction. As discussed in \cite{klenow_innovative_2020}, adjustment costs mean that, in the data, it can take several years for a new product to displace an old one. However, in the model, entrants that replace incumbents reach their mature size immediately upon entry. If the model matches the amount of employment in firms of age < 1, it might underestimate the true impact on employment reallocation of each new cohort of firms.\footnote{In the data, because firms grow to achieve their mature size over the first five years (and beyond), so that the employment of an entering cohort of firms does not decrease over time (i.e., including firm exit) very rapidly in the data. If the data were in continuous time, the employment of the cohort would increase at first, then decrease. In the model, firms enter at their mature size, so the employment of a cohort decreases over time.} Given this, I match the employment share of firms age $\le 6$ engaging in creative destruction.

This is not a readily available moment. To attempt to estimate it, I turn to \cite{garcia-macia_how_2019} and \cite{klenow_innovative_2020}, which estimate the portion of growth coming from firms of different ages engaging in creative destruction, new variety creation, and own product improvement. They find that roughly 18\% of employment is in firms age $\le 6$, and that between 30\% and70\% of the growth from these firms is due to creative destruction, the rest due to new variety creation. However, in their framework, as in mine, a given amount of growth from creative destruction requires significantly more employment, as it destroys a previous incumbent. Using a value of $\lambda = 1.2$, for example, creative destruction requires 6 times more employment than new variety creation to generate the same amount of growth. Taking this into account, I calculate an employment share of young firms of 13.34\% during the sample period. The computation of the corresponding model moment is described in \ref{appendix:calibration:entryRate}.


 
\paragraph{R\&D-induced spinout share of employment}

Finally, matching the employment share of spinouts is of course crucial so that the analysis accurately captures the burden such firms impose on the incumbents that spawn them. Table \ref{table:GStable_founder2} shows that WSO founders account for roughly 10\% of all founders. Unreported regressions find that WSOs are approximately 30\% larger than non-spinouts in terms of employment, valuation, and revenue, but also in terms of number of founders. 

As will be discussed below, care must be taken to only match the employment share of spinouts which can be attributed to R\&D, since spinouts in the model correspond to those in the data which are "induced" by R\&D at the incumbent. I discipline this using micro data on spinouts. The computation of the corresponding model moment is described in \ref{appendix:calibration:WSOempShare}.

Specifically, Figure \ref{figure:founder2_founders_f3_Accounting} documents that the regression coefficient is able to account for roughly 75\% of founder departures to WSOs in the data \textbf{[Update figure]}. Table \ref{table:GStable_founder2} shows that WSO founders account for roughly 8\% of all founders. Finally, regressions in the appendix (\ref{table:startupLifeCycle_founder2founders_lemployeecount_founder2} through \ref{table:startupLifeCycle_founder2founders_lpostvalusd_founder2}) show that a startup all of whose founders are from the same 4-digit industry has 35\% higher employment per founder, 40\% higher revenue per founder, 30-40\% higher valuation per founder. Putting these facts together, roughly 8\% of startup employment is in spinouts induced by corporate R\&D spending. 

However, as with the entry rate, I need to account for the fact that I am calibrating the model to CD and OI, and not new varieties which do occur in the startup data. Using the same type of logic implies an adjustment factor between $1/.93$ and $1/.7$, which implies that between 8.5\% and 11\% of creative destruction startup employment consists of WSOs. I choose a value of 10\% for the calibration.

\begin{table}[]
	\centering
	\captionof{table}{Calibration targets}\label{calibration_targets}
	\begin{tabular}{rcll}
		\toprule \toprule
		& Key parameter(s) & Target & Model \tabularnewline
		\midrule
		Profit (\% GDP) & $\beta$ & 8.5\% & 8.5\% 
		\tabularnewline
		R\&D emp. share & $\bar{L}_{RD}$ & 1\% & 1\% 
		\tabularnewline
		Interest rate & $\rho$ & 6\% & 6\% 
		\tabularnewline
		Growth rate (CD + OI) & $\mathbf{\lambda, \chi, \hat{\chi}}$ & 1.487\% & 1.487\%
		\tabularnewline		
		Age $\ge$ 6 growth share & $\chi, \hat{\chi}$  & 65\% & 65\%
		\tabularnewline
		Age $<$ 6 emp. share  & $\lambda, \hat{\chi}$ & 13.34\% & 13.34\%
		\tabularnewline
		Spinout emp. share &$\nu$  & 10\% & 10\%
		\tabularnewline
		R\&D spending (\% GDP) & $\chi, \hat{\chi}, \kappa_e$  & 1.35\% & 1.35\%
		\tabularnewline
		\bottomrule
	\end{tabular}
\end{table}

\normalsize

\subsection{Identification}\label{subsec:identification}

\autoref{calibration_identificationSources} shows the elasticity of model moments to calibrated model parameters.\footnote{This is computed as the jacobian matrix of the mapping that takes log parameters to log model moments.} It suggests how identification occurs by showing which moments are sensitive to which parameters. All moments are influenced by all parameters. Therefore, there is no one-to-one correspondence between moments and parameters used to identify them. 

To get a better sense of how identification is actually occurring in the model, \autoref{calibration_sensitivity} shows the elasticity of calibrated parameters to moment targets.\footnote{This is calculated by inverting the matrix shown in the previous figure. This is feasible because the model is locally exactly identified by the target moments.} This provides a more complete picture of the identification, since it takes into account the interaction of the various sensitivities of moments to parameters when it inverts the matrix. And it shows that conclusions based on \autoref{calibration_identificationSources} can be misleading. For example, while an increase in $\lambda$ causes a large increase in Growth Share OI, increasing the Growth Share OI moment target decreases the estimated $\lambda$. Given all the moments that need to be matched, the calibration prefers to match the higher Growth Share OI with a much higher $\chi$ and slightly lower $\lambda$.

\autoref{calibration_identificationSources_full} augments \autoref{calibration_identificationSources} with non-calibrated parameters included as both parameters and target moments. As before, \autoref{calibration_sensitivity_full} inverts this matrix to obtain the elasticity of calibrated parameters to moment targets and non-calibrated parameters. This gives the complete picture of how the model parameters are inferred. Based on this picture, I draw the following conclusions:

\begin{enumerate}
	\item The discount rate $\rho$ is identified to simultaneously match the interest rate, growth rate and IES
	\item The parameters $\lambda, \chi_I, \chi_E$ are determined in a complicated way, as can be seen from the fact that their profiles of sensitivities have similar shapes (modulo flipping the vertical axis). Still, they are not exactly the same: the effect of E on $\chi_E$ is larger in proportion to the effect of OI on $\chi_E$ than it is for $\lambda, \chi_I$. In addition, there are other subtle differences. An increase in $r$ has the same effect on $\chi_I$ as an increase in OI or E, but a different effect on $\lambda, \chi_E$. An increase in $g$ has the same sign effect as OI or E on $\chi_E$, but a different sign effect for $\lambda, \chi_I$. Finally, it can be seen from the $\chi_E$ panel that the externally set parameters $\beta, \psi$ have a large effect on $\chi_E$.
	\item The parameter $\kappa_E$ is distinguished from $\chi_E$ by matching the R\&D / GDP ratio.
	\item The parameter $\nu$ is identified by matching employment share of WSOs. 
\end{enumerate}

\begin{table}[]
	\centering
	\captionof{table}{Calibrated parameters}\label{calibration_parameters}
	\begin{tabular}{rlll}
		\toprule \toprule
		Parameter & Value & Description & Source \tabularnewline
		\midrule
		$\rho$ & 0.0303 & Discount rate  & Indirect inference \tabularnewline
		$\theta$ & 2 & $\theta^{-1} = $ IES & External calibration 
		\tabularnewline
		$\beta$ & 0.094 & $\beta^{-1} = $ EoS intermediate goods & Exactly identified \tabularnewline 
		$\psi$ & 0.5 & Entrant R\&D elasticity & External calibration \tabularnewline
		$\lambda$ & 1.084 & Quality ladder step size & Indirect inference 
		\tabularnewline
		$\chi$ & 21.217 & Incumbent R\&D productivity & Indirect inference 
		\tabularnewline
		$\hat{\chi}$ & 0.554 & Entrant R\&D productivity & Indirect inference \tabularnewline 
		$\kappa_e$ & 0.859 & Non-R\&D entry cost & Indirect inference \tabularnewline
		$\nu$ & 0.345 & Spinout generation rate  & Indirect inference\tabularnewline
		$\bar{L}_{RD}$ & 0.01 & R\&D labor allocation  & Exactly identified \tabularnewline
		\bottomrule
	\end{tabular}
\end{table}

\begin{figure}[]
	\includegraphics[scale = 0.43]{../code/julia/figures/simpleModel/identificationSources.pdf}
	\caption{Plot showing the elasticity of moments to model parameters. This illustrates how the model's equilibrium is affected by the various choices of parameters. These elasticities are computed by taking the jacobian matrix of the mapping from log parameters to log model moments.}
	\label{calibration_identificationSources}
\end{figure}

\begin{figure}[]
	\includegraphics[scale = 0.43]{../code/julia/figures/simpleModel/calibrationSensitivityFull.pdf}
	\caption{Same as \autoref{calibration_sensitivity}, but now including non-calibrated parameters. As before, this calculated by inverting the jacobian displayed in \autoref{calibration_identificationSources_full}.}
	\label{calibration_sensitivity_full}
\end{figure}

\section{Welfare effect of NCA enforcement and other policies}\label{sec:policy_analysis}

Now that the model has been calibrated, it can be used to conduct welfare analysis of the various policies studied in Section \ref{model:efficiency:policy_analysis}.

\paragraph{Consumption-equivalent change in welfare} 

I will compare welfare across BGPs in consumption-equivalent terms. For the purposes of this discussion, an \textit{allocation} is a set $A = \{ \tilde{C}_A, g_A \}$. Let $\tilde{W}(A)$ denote the normalized welfare corresponding to a given allocation $A$ and consider an allocation $B$ such that $\tilde{W}(A) < \tilde{W}_B$. the CEV welfare improvement from allocation $A$ to allocation $B$ is the permanent increase in consumption in allocation $A$ that achieves the same welfare as allocation $B$. To make this precise, define the allocation $\alpha(A,B)$ as
\begin{align}
\alpha &= \{C_{\alpha}, g_A\} \\
\tilde{W} ( \alpha ) &= \tilde{W} ( B ) 
\end{align}

That is, allocation $\alpha$ has the same growth rate as allocation $A$ but a different consumption level $C_{\alpha}$ such that it provides the same welfare as allocation $B$. The consumption-equivalent percentage welfare improvement of allocation $B$ over $A$ can be calculated as
\begin{align}
100 \times \big(\frac{\hat{C}_A}{\tilde{C}_A} - 1 \big) 
\end{align}

For $\theta > 1$ (the case of interest in this paper), a $\frac{\xi}{\theta-1}\%$ CEV welfare improvement results from an $\xi\%$ decrease in the absolute value of $\tilde{W}$. \footnote{For $\theta < 1$, a $\frac{\xi}{1-\theta}\%$ CEV welfare improvement results from a $\xi\%$ increase in $\tilde{W}$. The case $\theta = 1$ corresponds to log utility, in which case
	\begin{align}
	\tilde{W} &= \frac{\rho \log(\tilde{C}) + g}{\rho^2} \label{eq:agg_welfare_log}
	\end{align}
	
	In this case, there is no simple correspondence to obtain CEV welfare changes, but they are easy to compute directly. Under the null policy, initial consumption is $\tilde{C}$ and growth is $g$. Under the new policy, initial consumption is $\tilde{C}^+$ and growth is $g^+$. The CEV welfare change is $\frac{\tilde{C}^* - \tilde{C}}{\tilde{C}}$, where $\tilde{C}^*$ is defined by 
	\begin{align}
	\frac{\rho\log(\tilde{C}^*) + g}{\rho^2} = \frac{\rho \log(\tilde{C}^+) + g^+}{\rho^2} \label{eq:agg_welfare_log_CEV}
	\end{align}}

\subsection{NCA cost $\kappa_c$}

\begin{table}
	\centering
	\captionof{table}{Effect of reducing $\kappa_c$}\label{reducing_kappa_c_table}
	\begin{tabular}{lrlll}
		\toprule \toprule
		Measure & Variable & $\kappa_c > \bar{\kappa}_c$ & $\kappa_c = 0$ & Chg. \tabularnewline
		\midrule
		Growth & $g$ & 1.487\% & 1.597\% & 7.4\% \tabularnewline
		Initial consumption & $\tilde{C}$  & 0.776 &  0.781 & 0.64\% \tabularnewline 
		\tabularnewline
		Welfare & $\tilde{W}$  &  & & 2.96\% (CEV terms)  \tabularnewline
		\bottomrule
	\end{tabular}
\end{table}

\autoref{reducing_kappa_c_table} shows the effect on growth, initial consumption, and welfare of reducing $\kappa_c$ to zero, starting at $\kappa_c > \bar{\kappa}_c$. Both growth and initial consumption increase, leading to a 3\% increase in welfare in consumption-equivalent terms. \autoref{reducing_kappa_c_decomposition_table} displays the growth attribution under the high and low $\kappa_c$ regimes. When $\kappa_c = 0$, incumbents do a higher share of R\&D. Because (\ref{cs:growth_misallocation_condition}) holds, growth increases. In addition, the reduction in entry costs and NCA costs leads to an increase in initial consumption.  

\autoref{calibration_summaryPlot} shows graphically how the entire equilibrium varies with $\kappa_c$. As $\kappa_c$ increases in $[0,\bar{\kappa}_c)$, welfare decreases (third row, third panel). This is driven by the changes in the growth rate (first row, third panel). The movements in the growth rate in turn drive movements in the interest rate, via the Euler equation (second row, second panel). Both R\&D wages paid by incumbents and entrants decline and then jump downwards at the $\bar{\kappa}_c$ threshold (second row, third panel). The growth rate is driven by the changes in the innovation rate (first row, second panel). The incumbent reduces innovation gradually, while the entrant increases gradually, but by less due to lower marginal returns to R\&D in equilibrium, as inequality (\ref{cs:growth_misallocation_condition}) holds. Spinouts increase the growth rate by a discrete amount as the threshold $\bar{\kappa}_c$ is crossed. Finally, the incumbent value decreases continuously until the threshold, where it jumps downwards (second row, first panel). 

\begin{table}
	\centering
	\captionof{table}{Decomposition of effect of reducing $\kappa_c$}\label{reducing_kappa_c_decomposition_table}
	\begin{tabular}{lrlll}
		\toprule \toprule
		Measure & Variable & $\kappa_c > \bar{\kappa}_c$ & $\kappa_c = 0$ & Chg. \tabularnewline
		\midrule
		Growth & $g$ & 1.487\% & 1.597\% & 7.4\% \tabularnewline
		\multicolumn{1}{r}{incumbents} &  & 81\% & 86.2\% & 6.42\% \tabularnewline
		\multicolumn{1}{r}{entrants} &  & 17.7\% & 13.8\% & -22\% \tabularnewline
		\multicolumn{1}{r}{spinouts} &  & 1.32\% & 0\% & -100\% \tabularnewline
		\tabularnewline
		R\&D & & & & 
		\tabularnewline
		\multicolumn{1}{r}{incumbents}  & $z / \bar{L}_{RD}$ & 67.7\% & 77.4\% & 14.3\% \tabularnewline 
		
		\multicolumn{1}{r}{entrants}  & $\hat{z} / \bar{L}_{RD}$ & 32.3\% & 22.5\% & -30.3\% \tabularnewline
		\bottomrule
	\end{tabular}
\end{table}

To validate this result, I study its robustness to variations in the target moments in \ref{appendix:policyanalysis:ncacost}. I conclude that the result is robust to up to a 10\% or so standard deviation in the target moments (with zero correlation between moment uncertainty). Also, I show that the welfare result is reversed when calibrating the model to a 9\%, rather than 13.34\%, employment share of entering firms. This occurs due to a much higher calibrated value of $\lambda$ (about 1.22), reducing the slack with which (\ref{cs:growth_misallocation_condition}) holds.  

\begin{figure}[]
	\includegraphics[scale = 0.57]{../code/julia/figures/simpleModel/calibrationFixed_summaryPlot.pdf}
	\caption{Effect of varying $\kappa_c$ on equilibrium variables and welfare.}
	\label{calibration_summaryPlot}
\end{figure}

\subsection{R\&D subsidy (tax)}

As numerical exercises show welfare effects are driven by productivity growth, rather than consumption at a given level of productivity, typically the effect of increasing $T_{RD}$ will be to reduce welfare, provided of course that (\ref{cs:growth_misallocation_condition}) holds. \autoref{calibration_RDSubsidy_summaryPlot} shows how the equilibrium varies with the $T_{RD}$. For this exercise, I set $\kappa_c = 1.2 \tilde{\bar{\kappa}}_c(\kappa_e,\lambda;T_{RD} = 0)$. 

\begin{figure}[]
	\includegraphics[scale = 0.57]{../code/julia/figures/simpleModel/calibrationFixed_RDSubsidy_summaryPlot.pdf}
	\caption{Summary of equilibrium for baseline parameter values and various values of $T_{RD}$. This assumes that $\kappa_c = 1.2 \tilde{\bar{\kappa}}_c(\kappa_e,\lambda;T_{RD} = 0)$.}
	\label{calibration_RDSubsidy_summaryPlot}
\end{figure}

Notice that growth (first row, third column) and welfare (third row, third column) both fall with $T_{RD}$, and jump down when the increase in $T_{RD}$ increases the use of NCAs.


\subsection{CD tax (subsidy)}\label{subsec:cd_tax}


The plots in \autoref{calibration_entryTax_summaryPlot} show the impact of varying $T_e$ on equilibrium variables, growth and welfare, when $\kappa_c = 1.2 \hat{\tilde{\kappa}}_c(\kappa_e,\lambda;T_e = 0)$. As expected based on the discussion of the previous paragraphs, growth and welfare increase in the entry tax $T_e$ due to a reallocation of R\&D to the incumbent. When the tax $T_e$ is high enough (around 16\% below), incumbents begin to use noncompetes.  

\begin{figure}[]
	\includegraphics[scale = 0.57]{../code/julia/figures/simpleModel/calibrationFixed_EntryTax_summaryPlot.pdf}
	\caption{Summary of equilibrium for baseline parameter values and various values of $T_e$. This assumes that $\kappa_c = 1.2 \hat{\bar{\kappa}}_c(\kappa_e,\lambda;T_e = 0)$.}
	\label{calibration_entryTax_summaryPlot}
\end{figure}

\subsection{OI R\&D subsidy (tax)}\label{cs:oi_rd_subsidy}

In this calibraiton, an increase in $T_{RD,I}$ significantly increases growth and welfare, as shown in \autoref{calibration_RDSubsidyTargeted_summaryPlot}. It does so by increasing incumbent R\&D. For high values of $T_{RD,I}$, there is a switch from $x = 0$ to $x = 1$ and welfare jumps down slightly. Otherwise, it is monotonically increasing. 

\begin{figure}[]
	\includegraphics[scale = 0.57]{../code/julia/figures/simpleModel/calibrationFixed_RDSubsidyTargeted_summaryPlot.pdf}
	\caption{Summary of equilibrium for baseline parameter values and various values of $T_{RD,I}$. This assumes that $\kappa_c = 1.2 \tilde{\bar{\kappa}}_c(\kappa_e,\lambda;T_{RD,I} = 0)$.}
	\label{calibration_RDSubsidyTargeted_summaryPlot}
\end{figure}


\subsection{All policies}

The planner wants to reallocate R\&D to the incumbent and ensure that $\mathbbm{1}^{NCA} = 0$ so that spinout potential is not wasted. This can be accomplished by raising both $T_{RD,I}$, which reallocates R\&D labor to the incumbent, and $\kappa_c$ \textbf{[figure needs to be updated]}, which prevents NCAs so that $\mathbbm{1}^{NCA} = 0$. \autoref{calibration_ALL_summaryPlot} confirms this intuition (second row, third column). The improvement in CEV welfare is more than 7\%, which is significantly more than any of the other improvements could achieve on their own. This is driven by the fact that $T_{NCA}$ raises the threshold at which $T_{RD,I}$ induces a switch to $\mathbbm{1}^{NCA} = 1$ (first row, third column).

\begin{figure}[]
	\includegraphics[scale = 0.46]{../code/julia/figures/simpleModel/calibrationFixed_ALL_summaryPlot.png}
	\caption{Summary of equilibrium for various values of $T_{RD,I}$ and $T_{NCA}$. This assumes the planner chooses $\kappa_c = \underline{\kappa}_c = \frac{1}{2} \bar{\bar{\kappa}}_c(\kappa_e,\lambda;T_{RD},T_{RD,I},T_e))$.}
	\label{calibration_ALL_summaryPlot}
\end{figure}

\section{Conclusion}

The question of whether noncompete agreements are good or bad for growth is a difficult but important question. This paper is an attempt to take a step towards answering it. Empirically, I have shown that R\&D tends to predict employee spinout formation at the firm level, after controlling for firm level variables and various fixed effects. The relationship is statistically robust and economically large enough to account for around 10\% of the employment in the Venture Source. Theoretically, I have shown that a relatively standard model of endogenous growth with creative desruction is able to speak to this question, and when calibrated to relatively standard data, suggests that reducing barriers to the usage of NCAs can significantly increase growth and welfare.

There are several possibilities for further work in this direction. \textbf{[In progress]}

\bibliography{references.bib}

\appendix

\counterwithin{proposition}{section}
\counterwithin{proposition_corollary}{section}
\counterwithin{lemma}{section}
\counterwithin{lemma_corollary}{section}

\section{Appendix of tables}

\setcounter{table}{0}
\renewcommand{\thetable}{\Alph{section}\arabic{table}}

% latex table generated in R 3.4.4 by xtable 1.8-4 package
% Thu Feb  6 14:38:22 2020
\begin{table}[!htb]
\centering
\begingroup\scriptsize
\begin{tabular}{p{4.5cm}llrllrll}
  \toprule
Industry & Startups & Individuals & State & Startups & Individuals & Year & Startups & Individuals \\ 
  \midrule
Business Applications Software & 1790 & 31218 & California & 8433 & 140958 & 1986 & 293 & 2103 \\ 
  Biotechnology Therapeutics & 1037 & 19264 & Massachussetts & 2217 & 37185 & 1987 & 353 & 2732 \\ 
  Communications Software & 996 & 14859 & New York & 1490 & 26450 & 1988 & 356 & 2877 \\ 
  Advertising/Marketing & 880 & 15211 & Texas & 1299 & 18452 & 1989 & 403 & 3293 \\ 
  Network/Systems Management Software & 671 & 13907 & Pennsylvania & 883 & 10759 & 1990 & 396 & 3222 \\ 
  Vertical Market Applications Software & 536 & 8401 & Washington & 784 & 12187 & 1991 & 422 & 3801 \\ 
  Online Communities & 467 & 6460 & Virginia & 606 & 8964 & 1992 & 537 & 4896 \\ 
  Application-Specific Integrated Circuits & 463 & 6475 & Colorado & 605 & 9337 & 1993 & 554 & 5322 \\ 
  Wired Communications Equipment & 458 & 6808 & Georgia & 562 & 7426 & 1994 & 689 & 6771 \\ 
  IT Consulting & 451 & 6378 & New Jersey & 557 & 7309 & 1995 & 876 & 8946 \\ 
  Drug Development Technologies & 400 & 5725 & Florida & 533 & 6524 & 1996 & 1191 & 13134 \\ 
  Healthcare Administration Software & 378 & 6500 & Illinois & 525 & 8054 & 1997 & 1141 & 13468 \\ 
  Fiberoptic Equipment & 362 & 4981 & North Carolina & 455 & 6333 & 1998 & 1513 & 19512 \\ 
  Therapeutic Devices (Minimally Invasive/Noninvasive) & 358 & 5635 & Maryland & 430 & 6223 & 1999 & 2557 & 32495 \\ 
  Business Support Services: Other & 341 & 4087 & Minnesota & 373 & 4661 & 2000 & 2003 & 24276 \\ 
  Procurement/Supply Chain & 325 & 4941 & Connecticut & 355 & 4614 & 2001 & 1067 & 13295 \\ 
  Multimedia/Streaming Software & 322 & 4460 & Ohio & 346 & 3876 & 2002 & 986 & 12946 \\ 
  Wireless Communications Equipment & 319 & 5045 & Utah & 249 & 3407 & 2003 & 1037 & 11922 \\ 
  Database Software & 318 & 6701 & Tennessee & 217 & 2828 & 2004 & 1110 & 13363 \\ 
  Specialty Retailers & 309 & 3354 & Oregon & 209 & 3071 & 2005 & 1222 & 13318 \\ 
  Entertainment & 295 & 3676 & Arizona & 207 & 2770 & 2006 & 1380 & 13829 \\ 
  Pharmaceuticals & 289 & 4282 & Michigan & 191 & 2460 & 2007 & 1506 & 13058 \\ 
  Therapeutic Devices (Invasive) & 285 & 3808 & Wisonsin & 140 & 1508 & 2008 & 1416 & 10504 \\ 
   \bottomrule
\end{tabular}
\endgroup
\caption{Statistics on startups covered by VS sample. Industry information uses VS industrial classification. Startups are counted by founding year, individuals by year they joined the firm.} 
\label{table:VS_summaryTable}
\end{table}


\begin{table}[!htb]
	\scriptsize
	\centering
	{
\def\sym#1{\ifmmode^{#1}\else\(^{#1}\)\fi}
\begin{tabular}{l*{4}{c}}
\toprule
                    &\multicolumn{1}{c}{(1)}         &\multicolumn{1}{c}{(2)}         &\multicolumn{1}{c}{(3)}         &\multicolumn{1}{c}{(4)}         \\
\midrule
$\frac{\text{WSO4 founders}}{\text{Total founders}}$&        0.19         &        0.32\sym{***}&        0.32\sym{***}&        0.30\sym{***}\\
                    &      (0.22)         &     (0.027)         &     (0.020)         &     (0.013)         \\
\addlinespace
Constant            &        2.44\sym{***}&        2.41\sym{***}&        2.41\sym{***}&        2.41\sym{***}\\
                    &     (0.073)         &   (0.00019)         &   (0.00015)         &   (0.00028)         \\
\addlinespace
State-Year FE       &          No         &         Yes         &         Yes         &          No         \\
\addlinespace
State-Age FE        &          No         &         Yes         &          No         &         Yes         \\
\addlinespace
State-Cohort FE     &          No         &          No         &         Yes         &         Yes         \\
\addlinespace
NAICS4-Year FE      &          No         &         Yes         &         Yes         &          No         \\
\addlinespace
NAICS4-Age FE       &          No         &         Yes         &          No         &         Yes         \\
\addlinespace
NAICS4-Cohort FE    &          No         &          No         &         Yes         &         Yes         \\
\addlinespace
No FE               &         Yes         &          No         &          No         &          No         \\
\midrule
Clustering          &statecode naics1\_4 year         &statecode naics1\_4         &statecode naics1\_4         &statecode naics1\_4         \\
R-squared (adj.)    &     0.00068         &        0.35         &        0.38         &        0.36         \\
R-squared (within, adj)&     0.00068         &      0.0028         &      0.0028         &      0.0024         \\
Observations        &       55767         &       54873         &       54654         &       54779         \\
\bottomrule
\multicolumn{5}{l}{\footnotesize Standard errors in parentheses}\\
\multicolumn{5}{l}{\footnotesize \sym{*} \(p<0.1\), \sym{**} \(p<0.05\), \sym{***} \(p<0.01\)}\\
\end{tabular}
}

	\caption{Dependent variable is the logarithm of the number of employees while the independent variable is the fraction of founders who most recently worked at a public firm in the same industry. The first column shows the raw regression. The following three columns control for state, industry, time, cohort and age factors. Specifically, each regression uses a subset of two of the three (year,age,cohort) effects, in all cases included interacted both with state and industry.} 
	\label{table:startupLifeCycle_founder2founders_lemployeecount_founder2}
\end{table}

\begin{table}[!htb]
	\scriptsize
	\centering
	{
\def\sym#1{\ifmmode^{#1}\else\(^{#1}\)\fi}
\begin{tabular}{l*{4}{c}}
\toprule
                    &\multicolumn{1}{c}{(1)}         &\multicolumn{1}{c}{(2)}         &\multicolumn{1}{c}{(3)}         &\multicolumn{1}{c}{(4)}         \\
\midrule
$\frac{\text{WSO4 founders}}{\text{Total founders}}$&       -0.13         &        0.45\sym{***}&        0.42\sym{***}&        0.39\sym{***}\\
                    &     (0.094)         &      (0.13)         &     (0.081)         &      (0.12)         \\
\addlinespace
State-Year FE       &          No         &         Yes         &         Yes         &          No         \\
\addlinespace
State-Age FE        &          No         &         Yes         &          No         &         Yes         \\
\addlinespace
State-Cohort FE     &          No         &          No         &         Yes         &         Yes         \\
\addlinespace
NAICS4-Year FE      &          No         &         Yes         &         Yes         &          No         \\
\addlinespace
NAICS4-Age FE       &          No         &         Yes         &          No         &         Yes         \\
\addlinespace
NAICS4-Cohort FE    &          No         &          No         &         Yes         &         Yes         \\
\midrule
Clustering          &State, Industry         &State, Industry         &State, Industry         &State, Industry         \\
R-squared (adj.)    &    0.000092         &        0.30         &        0.38         &        0.39         \\
R-squared (within, adj)&    0.000092         &      0.0030         &      0.0026         &      0.0022         \\
Observations        &       16948         &       15500         &       15531         &       15905         \\
\bottomrule
\multicolumn{5}{l}{\footnotesize Standard errors in parentheses}\\
\multicolumn{5}{l}{\footnotesize \sym{*} \(p<0.1\), \sym{**} \(p<0.05\), \sym{***} \(p<0.01\)}\\
\end{tabular}
}

	\caption{Dependent variable is the logarithm of annual revenue while the independent variable is the fraction of founders who most recently worked at a public firm in the same industry. The first column shows the raw regression. The following three columns control for state, industry, time, cohort and age factors. Specifically, each regression uses a subset of two of the three (year,age,cohort) effects, in all cases included interacted both with state and industry.} 
	\label{table:startupLifeCycle_founder2founders_lrevenue_founder2}
\end{table}

\begin{table}[!htb]
	\scriptsize
	\centering
	{
\def\sym#1{\ifmmode^{#1}\else\(^{#1}\)\fi}
\begin{tabular}{l*{4}{c}}
\toprule
                    &\multicolumn{1}{c}{(1)}         &\multicolumn{1}{c}{(2)}         &\multicolumn{1}{c}{(3)}         &\multicolumn{1}{c}{(4)}         \\
\midrule
$\frac{\text{WSO4 founders}}{\text{Total founders}}$&        0.46\sym{***}&        0.42\sym{***}&        0.36\sym{***}&        0.33\sym{***}\\
                    &     (0.065)         &     (0.058)         &     (0.069)         &     (0.074)         \\
\addlinespace
State-Year FE       &          No         &         Yes         &         Yes         &          No         \\
\addlinespace
State-Age FE        &          No         &         Yes         &          No         &         Yes         \\
\addlinespace
State-Cohort FE     &          No         &          No         &         Yes         &         Yes         \\
\addlinespace
NAICS4-Year FE      &          No         &         Yes         &         Yes         &          No         \\
\addlinespace
NAICS4-Age FE       &          No         &         Yes         &          No         &         Yes         \\
\addlinespace
NAICS4-Cohort FE    &          No         &          No         &         Yes         &         Yes         \\
\midrule
Clustering          &State, Industry         &State, Industry         &State, Industry         &State, Industry         \\
R-squared (adj.)    &      0.0042         &        0.28         &        0.29         &        0.26         \\
R-squared (within, adj)&      0.0042         &      0.0050         &      0.0035         &      0.0028         \\
Observations        &       26504         &       25174         &       25027         &       25337         \\
\bottomrule
\multicolumn{5}{l}{\footnotesize Standard errors in parentheses}\\
\multicolumn{5}{l}{\footnotesize \sym{*} \(p<0.1\), \sym{**} \(p<0.05\), \sym{***} \(p<0.01\)}\\
\end{tabular}
}

	\caption{Dependent variable is the logarithm of post-money valuation while the independent variable is the fraction of founders who most recently worked at a public firm in the same industry. The first column shows the raw regression. The following three columns control for state, industry, time, cohort and age factors. Specifically, each regression uses a subset of two of the three (year,age,cohort) effects, in all cases included interacted both with state and industry.} 
	\label{table:startupLifeCycle_founder2founders_lpostvalusd_founder2}
\end{table}


\begin{table}[]
	\centering
	\captionof{table}{2-digit NAICS codes summary}\label{}
	\begin{tabular}{rl}
		\toprule \toprule
		2-digit Code & Description \tabularnewline
		\midrule
		11  & Agriculture, Forestry, Fishing and Hunting \tabularnewline
		21  & Mining, Quarrying, and Oil and Gas Extraction\tabularnewline
		22  & Utilities\tabularnewline
		23  & Construction \tabularnewline
		31-33 & Manufacturing \tabularnewline
		42 & Wholesale trade \tabularnewline
		44-45 & Retail trade \tabularnewline
		48-49 & Transportation and warehousing \tabularnewline
		51 & Information \tabularnewline
		52 & Finance and insurance \tabularnewline
		53 & Real estate and Rental and Leasing \tabularnewline
		54 & Professional, Scientific, and Technical Services \tabularnewline
		55 & Management of Companies and Enterprises \tabularnewline
		56 & Administrative, Support, Waste Management, Remediation Service \tabularnewline
		61 & Educational services \tabularnewline
		62 & Health Care and Social Assistance \tabularnewline
		71 & Arts, Entertainment, Recreation \tabularnewline
		72 & Accomodation and Food Services \tabularnewline
		81 & Other Services (ecept public Admin.) \tabularnewline
		92 & Public Administration\tabularnewline
		\bottomrule
	\end{tabular}
\end{table}

\begin{table}[]
	\centering
	\captionof{table}{3-digit breakdown of NAICS 31-33}\label{}
	\begin{tabular}{rl}
		\toprule \toprule
		3-digit code & Description \tabularnewline
		\midrule
		311 & Food \tabularnewline 
		312 & Beverage and tobacco  \tabularnewline 
		313-316 & Textiles, apparel and leather  \tabularnewline
		321-323 & Wood, paper manufacturing and printing \tabularnewline  
		324 & Petroleum and coal products \tabularnewline
		325 & Chemical \tabularnewline
		326 & Plastics and rubber \tabularnewline
		327 & Nonmetallic mineral products \tabularnewline 
		331-332 & Primary and fabricated metal  \tabularnewline
		333 & Machinery \tabularnewline
		334 & Computer and electronic product \tabularnewline
		335 & Electrical equipment \tabularnewline
		336 & Transportation equipment \tabularnewline
		337 & Furniture \tabularnewline
		339 & Misc \tabularnewline
		\bottomrule
	\end{tabular}
\end{table}

\begin{table}[]
	\centering
	\captionof{table}{\textbf{[in progress]} 3-digit breakdown of NAICS 51-56}\label{}
	\begin{tabular}{rl}
		\toprule \toprule
		3-digit code & Description \tabularnewline
		\midrule
		511 & Publishing (non-internet) \tabularnewline 
		512 & Motion picture and sound recording \tabularnewline 
		515 & Broadcasting (non-internet) \tabularnewline
		517 & Telecommunications \tabularnewline  
		518 & Data processing, hosting and related services \tabularnewline
		519 & Other information services \tabularnewline
		 & Plastics and rubber \tabularnewline
		327 & Nonmetallic mineral products \tabularnewline 
		331-332 & Primary and fabricated metal  \tabularnewline
		333 & Machinery \tabularnewline
		334 & Computer and electronic product \tabularnewline
		335 & Electrical equipment \tabularnewline
		336 & Transportation equipment \tabularnewline
		337 & Furniture \tabularnewline
		339 & Misc \tabularnewline
		\bottomrule
	\end{tabular}
\end{table}





\begin{table}[!htb]
	\centering
	\captionof{table}{Alternative calibration}\label{calibration_2_parameters}
	\begin{tabular}{rlll}
		\toprule \toprule
		Parameter & Value & Description & Source \tabularnewline
		\midrule
		$\rho$ & 0.0339 & Discount rate  & Indirect inference \tabularnewline
		$\theta$ & 2 & $\theta^{-1} = $ IES & External calibration 
		\tabularnewline
		$\beta$ & 0.094 & $\beta^{-1} = $ EoS intermediate goods & Exactly identified \tabularnewline 
		$\psi$ & 0.5 & Entrant R\&D elasticity & External calibration \tabularnewline
		$\lambda$ & 1.170 & Quality ladder step size & Indirect inference 
		\tabularnewline
		$\chi$ & 1.80 & Incumbent R\&D productivity & Indirect inference 
		\tabularnewline
		$\hat{\chi}$ & 0.115 & Entrant R\&D productivity & Indirect inference \tabularnewline 
		$\kappa_e$ & 0.737 & Non-R\&D entry cost & Indirect inference \tabularnewline
		$\nu$ & 0.04766 & Spinout generation rate  & Indirect inference\tabularnewline
		$\bar{L}_{RD}$ & 0.05 & R\&D labor allocation  & Normalization \tabularnewline
		\bottomrule
	\end{tabular}
\end{table}



\newpage
\section{Appendix of figures}

\setcounter{figure}{0}
\renewcommand{\thefigure}{\Alph{section}\arabic{figure}}

\begin{figure}[!htb]
	\centering
	\includegraphics[scale=0.95]{../empirics/figures/plots/industry_column_heatmap_naics2_founder2.pdf}
	\caption{Heatmap displaying the distribution of parent 2-digit NAICS code (row), conditional on child NAICS code (column). Darker hues indicate a higher density.}
	\label{figure:industry_column_heatmap_naics2_founder2}
\end{figure}

\begin{figure}[!htb]
	\centering
	\includegraphics[scale=.97]{../empirics/figures/plots/industry_row_heatmap_naics3_founder2.pdf}
	\caption{Heatmap displaying the distribution of child 3-digit NAICS code (column), conditional on parent NAICS code (row). Darker hues indicate a higher density.}
	\label{figure:industry_row_heatmap_naics3_founder2}
\end{figure}

\begin{figure}[!htb]
	\centering
	\includegraphics[scale=.97]{../empirics/figures/plots/industry_column_heatmap_naics3_founder2.pdf}
	\caption{Heatmap displaying the distribution of parent 3-digit NAICS code (row), conditional on child NAICS code (column). Darker hues indicate a higher density.}
	\label{figure:industry_column_heatmap_naics3_founder2}
\end{figure}

\begin{figure}[!htb]
	\centering
	\includegraphics[scale=1]{../empirics/figures/plots/industry_row_heatmap_naics4_founder2.pdf}
	\caption{Heatmap displaying the distribution of child 4-digit NAICS code (column), conditional on parent NAICS code (row). Darker hues indicate a higher density.}
	\label{figure:industry_row_heatmap_naics4_founder2}
\end{figure}

\begin{figure}[!htb]
	\centering
	\includegraphics[scale=1]{../empirics/figures/plots/industry_column_heatmap_naics4_founder2.pdf}
	\caption{Heatmap displaying the distribution of parent 4-digit NAICS code (row), conditional on child NAICS code (column). Darker hues indicate a higher density.}
	\label{figure:industry_column_heatmap_naics4_founder2}
\end{figure}

\begin{figure}[!htb]
	\centering
	\includegraphics[scale= 1.2]{../empirics/figures/plots/RDandSpinoutFormation_speccheck2_levels_wso4_reghdfe.pdf}
	\caption{Robustness of WSO4 founders regression result to different sets of controls, fixed effects, and clustering assumptions. The regressions always control for employment and cumulative patent citations, always include firm and year fixed effects, and always cluster by firm. The above plots show the robustness to controlling for assets, intangible assets, capital expenditures, net income, market value, }
	\label{figure:speccheck2_levels_wso4_reghdfe}
\end{figure}

\begin{figure}[!htb]
	\centering
	\includegraphics[scale= 1.2]{../empirics/figures/plots/RDandSpinoutFormation_speccheck2_at_wso4_reghdfe.pdf}
	\caption{Robustness of asset-normalized WSO4 founders regression result to different sets of controls, fixed effects, and clustering assumptions. The regressions always control for employment and cumulative patent citations, always include firm and year fixed effects, and always cluster by firm. The above plots show the robustness to controlling for assets, intangible assets, capital expenditures, net income, market value, }
	\label{figure:speccheck2_at_wso4_reghdfe}
\end{figure}

\begin{figure}[!htb]
	\centering
	\includegraphics[scale= 1.2]{../empirics/figures/plots/RDandSpinoutFormation_speccheck2_levels_wso4_ppmlhdfe.pdf}
	\caption{Robustness of WSO4 founders Poisson pseudo-Maximum Likelihood regression result to different sets of controls, fixed effects, and clustering assumptions. The regressions always control for log employment and log cumulative patent citations, always include firm and year fixed effects, and always cluster by firm. The above plots show the robustness to controlling for log assets, log intangible assets, log capital expenditures, log net income, log market value, }
	\label{figure:speccheck2_levels_wso4_ppmlhdfe}
\end{figure}

\begin{figure}[]
	\includegraphics[scale = 0.43]{../code/julia/figures/simpleModel/calibrationSensitivity.pdf}
	\caption{Plot showing the elasticity of parameters to moments. It is computed by inverting the jacobian matrix of the mapping from log parameters to log model moments (whose entries comprise the previous figure). These elasticities, along with estimates of the noisiness of the moments used in the calibration, can be used to estimate confidence intervals for the parameters in the model, and thereby for the welfare comparison in question.}
	\label{calibration_sensitivity}
\end{figure}

\begin{figure}[]
	\includegraphics[scale = 0.43]{../code/julia/figures/simpleModel/identificationSourcesFull.pdf}
	\caption{Plot showing the elasticity of moments to model parameters, including parameters taken from the literature $\theta , \beta, \psi$. These non-calibrated parameters are added in as effective moments to be matched, allowing the sensitivity of calibrated parameters $\rho, \lambda, \chi, \hat{\chi}, \kappa_E, \nu$ to these parameters to be computed by simply inverting this matrix, as before.}
	\label{calibration_identificationSources_full}
\end{figure}

\begin{figure}[!htb]
	\includegraphics[scale = 0.36]{../code/julia/figures/simpleModel/levelsWelfareComparisonSensitivityFull.pdf}
	\caption{Sensitivity of welfare comparison to moments. This is $(J^{-1})^T \nabla_p W$, where $W(p)$ maps log parameters to the percentage change in BGP consumption which is equivalent to the change in welfare from changing $\kappa_c$ from $\infty$ to $0$ (i.e. going from banning to frictionlessly enforcing NCAs). In contrast with the elasticity of the previous figure, this is a semi-elasticity. In particular it can allow for the change in welfare to be negative. The way to read this is the following. Looking at the column labeled \textit{E}, the chart says that a 1\% increase in the targeted employment share of young firms, which corresponds to a log change of about $0.01$, leads to an increase in the \% welfare improvement of approximately $6 \times 0.01 = 0.06$ percentage points.}
	\label{levelsWelfareComparisonSensitivityFull}
\end{figure}

\begin{figure}[]
	\includegraphics[scale = 0.36]{../code/julia/figures/simpleModel/welfareComparisonParameterSensitivityFull.pdf}
	\caption{Sensitivity of welfare comparison to moments. This is $\nabla_p W$, wahere $W(p)$ maps log parameters to the log of the percentage change in BGP consumption which is equivalent to the change in welfare from changing $\kappa_c$ from $\infty$ to $0$ (i.e. going from banning to frictionlessly enforcing NCAs).}
	\label{welfareComparisonParameterSensitivityFull}
\end{figure}

\begin{figure}[]
	\includegraphics[scale = 0.36]{../code/julia/figures/simpleModel/levelsWelfareComparisonParameterSensitivityFull.pdf}
	\caption{Sensitivity of welfare comparison to moments. This is $\nabla_p W$, wahere $W(p)$ maps log parameters to the percentage change in BGP consumption which is equivalent to the change in welfare from changing $\kappa_c$ from $\infty$ to $0$ (i.e. going from banning to frictionlessly enforcing NCAs).}
	\label{levelsWelfareComparisonParameterSensitivityFull}
\end{figure}

\section{Model}\label{appendix:model}

\subsection{Proofs of propositions}

\subsubsection{Proof of Proposition \ref{proposition:hjb_scaling}}\label{appendix:proofs:proposition:hjb_scaling}

\begin{proof}
	If $z > 0$ then the FOC holds with equality; otherwise, we can ignore that terms multiplied by $z$ in the incumbent's HJB. Hence, the incumbent HJB implies
	\begin{align}
	(r_t + \hat{\tau}) V(j,t|q) - \dot{V}(j,t|q) &= \tilde{\pi} q
	\end{align}
	
	where I used $\hat{\tau}_{jt} = \hat{\tau}$ on a symmetric BGP.
	
	Next, the static equilibrium implies that $C_t \propto Q_t$. Therefore $\frac{\dot{Q}_t}{Q_t} = g$ implies $\frac{\dot{C}_t}{C_t} = g$, and the Euler equation then implies $r_t = r$. Therefore, 
	\begin{align}
	(r + \hat{\tau}) V(j,t|q) - \dot{V}(j,t|q) &= \tilde{\pi} q
	\end{align}
	
	This differential equation has a constant solution equal to 
	\begin{align}
	V(j,t|q) &= \frac{\tilde{\pi} q}{r + \hat{\tau}} \\
	&= \tilde{V} q 
	\end{align}
	
	Nonconstant solutions correspond to positive or negative bubbles in the valuation of the incumbent, which I simply rule out. Below, I provide what may be a more rigorous proof, but I don't want to emphasize this. In every paper I know of using this type of model, the value of incumbents has simply been assumed to be the constant solution (or, as in \cite{acemoglu_innovation_2015}, the value function is directly assumed to be of the desired linear form).
\end{proof}

\begin{proof}

	The entrant optimization condition is (using $z_{jt} = z$)
	\begin{align}
	\hat{\chi} \hat{z}^{-\psi} V(j,t|\lambda q) &= \frac{q}{Q_t} \hat{w}_{RD,t}
	\end{align}
	
	Rearranging,
	\begin{align}
	\hat{\chi}^{-1} \hat{z}^{\psi} &= \frac{V(j,t|\lambda q)}{q} \frac{Q_t}{\hat{w}_{RD,t}}  \label{constant_vw_ratio}
	\end{align}
	
	The only term which varies with $j$ is $V(j,t|\lambda q)$. This implies that $V(j,t| \lambda q) =  \tilde{V}(t| \lambda q)$. In fact, the equation also shows that $V(t|\lambda q) / q$ is constant over $q$, i.e. that $\tilde{V}(t|\lambda q) = \tilde{V}(t) \lambda q$. Note, however, that it does not directly imply anything about $V(j,t|q)$: constant entrant innovation effort $\hat{z}_{jt} = \hat{z}$ implies that the value of incumbency tomorrow must be proportional to $q$, but it does not directly imply that the value of incumbency today is proportional to $q$. It makes sense intuitively that the same logic should imply that $V(j,t|q) = \bar{\tilde{V}}(t) q$: otherwise, the equilibrium we are on cannot have satisfied the (rational) expectations of previous entrants. Heuristically maybe this is enough, but I haven't found a rigorous proof along these lines. Instead, I show that $V(j,t|q) = \tilde{V}q$ by arguing that other solutions to the incumbent HJB contradict equilbrium requirements. Then the fact that at any time $V(j,t|q)$ has this form implies that innovators (entrants, incumbents, spinouts) expect this value in the future and therefore that they forecast their future payoffs using $V(j,t|\lambda q) = \tilde{V} \lambda q$.
	
	First, differentiating both sides with respect to $t$ and using $\frac{\dot{Q}_t}{Q_t} = g$ on the BGP yields
	\begin{align}
	- \frac{\dot{V}(t|\lambda q)}{V(t|\lambda q)} &= g - \frac{\dot{\hat{w}}_{RD,t}}{\hat{w}_{RD,t}} \label{appendix:eq:freeEntryDifferentiated}
	\end{align}
	
	The incumbent HJB implies
	\begin{align}
	(r_t + \hat{\tau}) V(j,t|q) - \dot{V}(j,t|q) &= \tilde{\pi} q
	\end{align}
	
	The static equilibrium implies that $C_t \propto Q_t$. Therefore $\frac{\dot{Q}_t}{Q_t} = g$ implies $\frac{\dot{C}_t}{C_t} = g$, and the Euler equation then implies $r_t = r$. Therefore, 
	\begin{align}
	(r + \hat{\tau}) V(j,t|q) - \dot{V}(j,t|q) &= \tilde{\pi} q
	\end{align}
	
	This differential equation has a constant solution equal to 
	\begin{align}
	V(j,t|q) &= \frac{\tilde{\pi} q}{r + \hat{\tau}} \\
	&= \tilde{V} q 
	\end{align}
	
	I want to show that this is the only solution which is compatible with equilibrium. Rearranging the original differential equation,
	\begin{align}
	\dot{V}(j,t|q) &= (r + \hat{\tau}) V(j,t|q) - \tilde{\pi} q \label{appendix:eq:hjbGeneral}
	\end{align}	
	
	Differentiating this expression again yields
	\begin{align}
	\ddot{V}(j,t|q) &= (r + \hat{\tau}) \dot{V}(j,t|q) \label{appendix:eq:hjbGeneralDifferentiated}
	\end{align}
	
	This means that if $\dot{V} < 0$ initially, then $\ddot{V} < 0$ initially as well, and similarly if $\dot{V} > 0$ initially then $\ddot{V} > 0$ initially as well. Hence, if in equilbrium $V(j,t|q)$ drifts locally, it must drift monotonically.
	
	If $\dot{V} < 0$ then (\ref{appendix:eq:hjbGeneral}) implies that $\dot{V}$ tends to $-\tilde{\pi}q$ as $t \to \infty$. This means that $V$ reaches a negative value in finite time with positive probability. This contradicts optimality since the incumbent is always free to choose $z = 0$ and earn flow profits $\tilde{\pi} q$; hence, this solution to the HJB is incompatible with equilibrium. 
	
	Next, rearrange the expression in the form
	\begin{align}
	\frac{\dot{V}(j,t|q)}{V(j,t|q)} &= r + \hat{\tau} - \frac{\tilde{\pi}q}{V(j,t|q)}
	\end{align}
	
	
	If $\dot{V} > 0$ intially, the second term on the RHS tends to zero and asymptotically $V$ grows at exponential rate $r + \hat{\tau}$. 
	
	First, suppose that $z > 0$. The FOC of the incumbent is
	\begin{align}
	\chi \Big( V(t|\lambda q) - V(j,t|q) \Big) &= \frac{q}{Q_t} \hat{w}_{RD,t} + \nu V(j,t|q) \nonumber \\
	&+ (1 - \mathbbm{1}^{NCA}_{jt}) (1- \kappa_e) \nu V(t|\lambda q) + \mathbbm{1}^{NCA}_{jt}  \kappa_c \nu V(j,t|q) \Big) 
	\end{align}
	
	Divide both sides by $q$, differentiate with respect to $t$, use $\frac{\dot{Q}_t}{Q_t} = g$ and  (\ref{appendix:eq:freeEntryDifferentiated}) to obtain
	\begin{align}
	-\frac{\dot{V}(j,t|q)}{V(j,t|q)} &= g - \frac{\dot{\hat{w}}_{RD,t}}{\hat{w}_{RD,t}}  \label{appendix:eq:freeEntryDifferentiatedImplication}
	\end{align}
	
	Using (\ref{appendix:eq:freeEntryDifferentiatedImplication}), this implies that with positive probability the R\&D wage grows to the point where $\hat{w}_{RD} \hat{z} > \tilde{Y}$, which contradicts equilbirium.
	
	Next, suppose that $z = 0$. In this case, the equilibrium R\&D allocation and growth rate are uniquely determined, so there is no harm in simply restricting attention to the case where $V(j,t|\bar{q}_{jt}) = \tilde{V}\bar{q}_{jt}, V(j,t|\lambda \bar{q}_{jt}) = \tilde{V} \bar{q}_{jt}$ for all $(j,t)$. However, below I have some discussion of how one would show that the value function has this form in any symmetric equilibrium.	
	
	We still know that $V(j,t|\lambda q) = \tilde{V}(t) \lambda q$, by (\ref{constant_vw_ratio}). However, we no longer have the incumbent FOC which helps to connect statements we make about $V(j,t|\lambda q) / q$ and $\hat{w}_{RD,t} / Q_t$ to statements about $V(j,t|q)$. However, the proof is simpler in this case because incumbents do no R\&D. This means that the value functions of the initial incumbents do not affect the rest of the equilibrium. More precisely, the BGP that arises in this model has exactly the same observable variables (except for the market price of initial incumbents, which may be a bubble in the model) as a BGP where $V(j,t|q) = \tilde{V}q$ and $V(j,t|\lambda q) = \tilde{V} \lambda q$ for all $(j,t,q)$, provided one can show that $\tilde{V}(t) = \tilde{V}$. 
	
	In principle, this equation could fail to hold because $\hat{w}_{RD,t} / Q_t$ fluctuates over time. When it grows (shrinks), $V(t)$ grows (shrinks) by the same proportion. To rule this out, suppose that $V(t') > V(t)$. With positive probability, the next innovation occurs at time $dt'$; alternatively, with positive probability the next innovation occurs at time $dt$. It cannot be true that $V(j,t|\lambda q) = V(j,t'|\lambda q) =  \frac{\tilde{\pi}\lambda q}{r + \hat{\tau}}$ is the value of the monopolist ex post. If $V(j,t|\lambda q) < \frac{\tilde{\pi}\lambda q}{r + \hat{\tau}}$ then by the logic above the value can go negative with positive probability, which violates optimality. 
	
	If $V(j,t|\lambda q) > \frac{\tilde{\pi}\lambda q}{r + \hat{\tau}}$ for some $(j,t)$ then this will violate the incumbent's transversality condition.\footnote{The incumbent's dynamic optimization problem has a transversality condition, because the dividend vs R\&D investment decision is an optimal control problem (the state being controlled is the quality of the incumbent).} It states that
	\begin{align}
	0 = \lim_{t' \to \infty} e^{-r(t'-t)} \mathbb{E}[\mathbbm{1}_{s(j,t) > t'} V(j,t',q)]
	\end{align}
	
	where $s(j,t)$ is the (random) time at which the current incumbent is displaced by an entrant. Because the time is exponentially distributed, this is the same as
	\begin{align}
	0 = \lim_{t' \to \infty} e^{-r(t'-t)} e^{-\hat{\tau} (t'-t)} V(j,t'|q)
	\end{align}
	
	However, we are in the case where $V(j,t'|q)$ grows at rate $r + \hat{\tau}$ asymptotically, i.e.
	\begin{align}
	\frac{\dot{V}(j,t|q)}{V(j,t|q)} &= r + \hat{\tau} - \frac{\tilde{\pi}q}{V(j,t|q)}
	\end{align}
	
	We have a situation with 
	\begin{align}
	\frac{\dot{X}}{X} &= r + \hat{\tau}, \quad  \text{i.e. } X(t') = e^{(r + \hat{\tau})(t' - t)} \\
	\frac{\dot{Y}}{Y} &= r + \hat{\tau} - \frac{\tilde{\pi}q}{Y}, \quad \text{i.e. } Y(t') = V(j,t'|q)
	\end{align}
	
	I want to show that
	\begin{align}
	\lim_{t \to \infty} \frac{Y}{X} > 0 
	\end{align}
	
	We know that
	\begin{align}
	\frac{d}{dt'}\Big(\frac{Y}{X} \Big) &= \frac{Y}{X} \Big( \frac{\dot{Y}}{Y} - \frac{\dot{X}}{X} \Big)
	\end{align}
	
	This yields
	\begin{align}
	\frac{d}{dt'} \Big(\frac{Y}{X} \Big) &= -\frac{\tilde{\pi}q}{Y} \frac{Y}{X}  \\
	&= - \frac{\tilde{\pi}q}{X} \\
	&= - \tilde{\pi} q e^{-(r +\hat{\tau}) (t' - t)}
	\end{align}
	
	using the definition of $X$. We know that $Y(t) = V(j,t|q) > \frac{\tilde{\pi} q}{r + \hat{\tau}}$ otherwise $\dot{V}(j,t|q) \le 0$ and it must continue to decline locally by (\ref{appendix:eq:hjbGeneralDifferentiated}). Since $X(t) = 1$, we have $\frac{Y(t)}{X(t)} > \frac{\tilde{\pi} q}{r + \hat{\tau}}$. Integrating, 
	\begin{align}
	\lim_{t' \to \infty} \frac{Y(t')}{X(t')} &=  \frac{Y(t)}{X(t)} + \lim_{t' \to \infty} \int_t^{t'} \frac{d}{ds} \Big(\frac{Y(s)}{X(s)} \Big) ds \\
	&= \frac{Y(t)}{X(t)} - \lim_{t' \to \infty} \int_t^{t'}  \tilde{\pi} q e^{-(r +\hat{\tau}) (s-t)} dt \\
	&> \frac{\tilde{\pi} q}{r + \hat{\tau}} - \underbrace{\lim_{t' \to \infty} \int_t^{t'}  \tilde{\pi} q e^{-(r +\hat{\tau}) (s-t)} dt}_{\mathclap{\frac{\tilde{\pi}q}{r + \hat{\tau}}}} \\ 
	&= 0
	\end{align}
	
	Therefore, the TVC is violated. I conclude that the only solution to the HJB compatible with equilibrium is 
	\begin{align}
	V(j,t|q) &= \frac{\tilde{\pi} q}{r + \hat{\tau}}
	\end{align}	
	
	which has the linear form required in the proposition.
	
	\paragraph{Note on the differentiability of $V(j,t|q)$}
	
	I am assuming that the value of incumbency in line $j$ is differentiable in $t$ conditional on no innovations occurring. I believe this does lose some generality -- it excludes equilibria where sunspots could create jumps in the value function. However, as far as I can tell this assumption is made (implicitly) in every single paper using this framework, starting from \cite{grossman_quality_1991} and including the models that form the direct foundation of this model such as the one in \cite{acemoglu_innovation_2015} and the related models in \cite{acemoglu_introduction_2009}. Extending the theoretical analysis of this broad class of models is outside the scope of this paper, so I adopt their assumption. 
	
\end{proof}

\subsection{Derivations for efficiency and theoretical policy analysis}

\subsubsection{CD tax (subsidy)}\label{appendix:model:efficiencyderivations:CDtax}

The incumbent HJB is given by 
\begin{align}
(r + \hat{\tau}) \tilde{V} = \tilde{\pi} + \max_{\substack{x \in \{0,1\} \\ z \ge 0}} \Big\{z &\Big( \overbrace{\chi (\lambda - 1) \tilde{V}}^{\mathclap{\mathbb{E}[\textrm{Benefit from R\&D}]}}-  \big( \overbrace{\hat{w}_{RD} - (1-x)(1-(1+T_e)\kappa_e)\lambda \nu \tilde{V}}^{\mathclap{\text{R\&D wage}}}\big) \label{eq:hjb_incumbent_entryTax} \nonumber \\ 
&-  \underbrace{(1-x) \nu \tilde{V}}_{\mathclap{\text{Net cost from spinout formation}}} - \overbrace{x \kappa_{c} \nu \tilde{V}}^{\mathclap{\text{Direct cost of NCA}}}\Big) \Big\} 
\end{align}

The equilibrium conditions not shown in the main text are
\begin{align}
\hat{\tau} &= \hat{\chi} \hat{z}^{1-\psi} \\
z &= \bar{L}_{RD} - \hat{z} \label{eq:labor_resource_constraint_entryTax}\\ 
\tau &= \chi z \\
\tau^S &= (1-x) \nu z \\
g &= (\lambda - 1) (\tau + \tau^S + \hat{\tau}) \\
r &= \theta g + \rho \\
\tilde{V} &= \frac{\tilde{\pi}}{r + \hat{\tau}} \\ 
\hat{w}_{RD} &= \begin{cases}
\hat{\chi} \hat{z}^{-\psi} (1-(1+T_e)\kappa_e) \lambda \tilde{V} &\textrm{, if } \hat{z} > 0\\
\Big( \chi(\lambda -1) - \nu (x\kappa_c + (1-x)\hat{\bar{\kappa}}_c(\kappa_e,\lambda;T_e))\Big) \tilde{V} &\textrm{, o.w.}
\end{cases} \label{eq:wage_rd_labor_entryTax}
\end{align}


\subsubsection{OI R\&D subsidy (tax)}\label{appendix:model:efficiencyderivations:OIRDtax}

The incumbent's HJB is given by
\begin{align}
(r + \hat{\tau}) \tilde{V} = \tilde{\pi} + \max_{\substack{x \in \{0,1\} \\ z \ge 0}} \Big\{z &\Big( \overbrace{\chi (\lambda - 1) \tilde{V}}^{\mathclap{\mathbb{E}[\textrm{Benefit from R\&D}]}}- (\underbrace{1-T_{RD,I}}_{\mathclap{\text{R\&D Subsidy}}}) \big( \overbrace{\hat{w}_{RD} - (1-x)(1-\kappa_e)\lambda \nu \tilde{V}}^{\mathclap{\text{R\&D wage}}}\big) \label{eq:hjb_incumbent_RDsubsidyTargeted} \nonumber \\ 
&-  \underbrace{(1-x) \nu \tilde{V}}_{\mathclap{\text{Net cost from spinout formation}}} - \overbrace{x \kappa_{c} \nu \tilde{V}}^{\mathclap{\text{Direct cost of NCA}}}\Big) \Big\} 
\end{align}

The equilibrium conditions not shown in the main text are
\begin{align}
\hat{\tau} &= \hat{\chi} \hat{z}^{1-\psi} \\
z &= \bar{L}_{RD} - \hat{z} \label{eq:labor_resource_constraint_RDsubsidyTargeted}\\ 
\tau &= \chi z \\
\tau^S &= (1-x) \nu z \\
g &= (\lambda - 1) (\tau + \tau^S + \hat{\tau}) \\
r &= \theta g + \rho \\
\tilde{V} &= \frac{\tilde{\pi}}{r + \hat{\tau}} \\ 
\hat{w}_{RD} &= \hat{\chi} \hat{z}^{-\psi} (1-\kappa_e) \lambda \tilde{V} \label{eq:wage_rd_labor_RDsubsidyTargeted}
\end{align}



\subsubsection{All policies}\label{appendix:model:efficiencyderivations:allPolicies}

The R\&D labor supply indifference condition is
\begin{align}
\hat{w}_{RD} &= w_{RD,j} + (1-x_j) \nu (1-(\underbrace{1+T_e}_{\mathclap{\text{Entry tax}}})\kappa_e) \lambda \tilde{V} \label{eq:RD_worker_indifference_all}
\end{align}

The incumbent HJB is
\begin{align}
(r + \hat{\tau}) \tilde{V} = \tilde{\pi} + \max_{\substack{x \in \{0,1\} \\ z \ge 0}} \Big\{z &\Big( \overbrace{\chi (\lambda - 1) \tilde{V}}^{\mathclap{\mathbb{E}[\textrm{Benefit from R\&D}]}}-  (\underbrace{1 - T_{RD} - T_{RD,I}}_{\mathclap{\text{R\&D subsidies}}})\big( \overbrace{\hat{w}_{RD} - (1-x)(1-(1+T_e)\kappa_e)\lambda \nu \tilde{V}}^{\mathclap{\text{R\&D wage}}}\big) \label{eq:hjb_incumbent_all} \nonumber \\ 
&-  \underbrace{(1-x) \nu \tilde{V}}_{\mathclap{\text{Net cost from spinout formation}}} - \overbrace{x \kappa_{c} \nu \tilde{V}}^{\mathclap{\text{Direct cost of NCA}}}\Big) \Big\} 
\end{align}

which can be rearranged to
\begin{align}
(r + \hat{\tau}) \tilde{V} = \tilde{\pi} + \max_{\substack{x \in \{0,1\} \\ z \ge 0}} \Big\{z &\Big( \overbrace{\chi (\lambda - 1) \tilde{V}}^{\mathclap{\mathbb{E}[\textrm{Benefit from R\&D}]}}- (1-T_{RD}- T_{RD,I})\hat{w}_{RD} \\
&-  \underbrace{(1-x)(1 - (1-T_{RD} - T_{RD,I})(1-(1+T_e)\kappa_{e})\lambda)\nu \tilde{V}}_{\mathclap{\text{Net cost from spinout formation}}} - \overbrace{x \kappa_{c}\nu \tilde{V}}^{\mathclap{\text{Direct cost of NCA}}}\Big) \Big\} \label{eq:hjb_incumbent_all_2}
\end{align}

Define
\begin{align}
\bar{\bar{\kappa}}_c(\kappa_e,\lambda;T_{RD},T_{RD,I},T_e) = 1 - (1-T_{RD} - T_{RD,I})(1-(1+T_e)\kappa_e)\lambda  \label{eq:barkappa_all}
\end{align} 

If $z > 0$, the incumbent's optimal NCA policy is given by 
\begin{align}
x = \begin{cases}
1 & \textrm{if } \kappa_c < \bar{\bar{\kappa}}_c (\kappa_e, \lambda;T_{RD},T_{RD,I},T_e)\\
0 & \textrm{if } \kappa_c > \bar{\bar{\kappa}}_c (\kappa_e, \lambda;T_{RD},T_{RD,I},T_e)\\
\{0,1\} & \textrm{if } \kappa_c = \bar{\bar{\kappa}}_c (\kappa_e, \lambda;T_{RD},T_{RD,I},T_e)
\end{cases} \label{eq:nca_policy_all}
\end{align}


By the usual argument, $z > 0$ implies that the incumbent's FOC can be rearranged to
\begin{align}
\tilde{V} &= \frac{(1-T_{RD} - T_{RD,I})\hat{w}_{RD}}{\chi(\lambda -1) - \nu (x\kappa_c + (1-x)(1 - (1-T_{RD} - T_{RD,I})(1-(1+T_e)\kappa_e)\lambda)) } \label{eq:hjb_incumbent_foc_all}
\end{align}

The free entry condition is
\begin{align}
\underbrace{\hat{\chi} \hat{z}^{-\psi}}_{\mathclap{\text{Marginal innovation rate}}} \overbrace{(1-(1+T_e)\kappa_e) \lambda \tilde{V}}^{\mathclap{\text{Payoff from innovation}}} &= (1-T_{RD})\underbrace{\hat{w}_{RD}}_{\mathclap{\text{MC of R\&D}}} \label{eq:free_entry_condition_all}
\end{align}

Substituting (\ref{eq:hjb_incumbent_foc_all}) into (\ref{eq:free_entry_condition_all}) to eliminate $\tilde{V}$ yields an expression for $\hat{z}$, 
\begin{align}
\hat{z} &= \Bigg( \frac{\Big(\frac{1-T_{RD} -T_{RD,I}}{1-T_{RD}} \Big)\hat{\chi} (1-(1+T_e)\kappa_{e}) \lambda}{\chi(\lambda -1) - \nu (x\kappa_c  + (1-x)(1 - (1-T_{RD} - T_{RD,I})(1-(1+T_e)\kappa_e)\lambda)) } \Bigg)^{1/\psi} \label{eq:effort_entrant_all}
\end{align}

From here, the rest of the model can be solved using
\begin{align}
\hat{\tau} &= \hat{\chi} \hat{z}^{1-\psi} \\
z &= \bar{L}_{RD} - \hat{z} \label{eq:labor_resource_constraint_all}\\ 
\tau &= \chi z \\
\tau^S &= (1-x) \nu z \\
g &= (\lambda - 1) (\tau + \tau^S + \hat{\tau}) \\
r &= \theta g + \rho \\
\tilde{V} &= \frac{\tilde{\pi}}{r + \hat{\tau}} \\ 
\hat{w}_{RD} &= \begin{cases}
(1-T_{RD})^{-1}\hat{\chi} \hat{z}^{-\psi} (1-(1+T_e)\kappa_e) \lambda \tilde{V} &\textrm{, if } \hat{z} > 0\\
\Big( \chi(\lambda -1) - \nu (x\kappa_c + (1-x)\bar{\bar{\kappa}}_c)\Big) \tilde{V} &\textrm{, o.w.}
\end{cases} \label{eq:wage_rd_labor_all}
\end{align}



\subsection{DRS incumbent innovation technology}

In this section I show why it is analytically convenient to have CRS innovation on the incumbent. Without it, the model must be solved numerically.

\begin{proposition}
	Suppose $z$ units of R\&D yields a Poisson rate
	\begin{align}
	\chi z^{1-\psi}  
	\end{align}
	for the incumbents and $\hat{z}$ units of R\&D yields a Poisson rate 
	\begin{align}
	\hat{\chi}\hat{z}^{1-\hat{\psi}}
	\end{align}
	for entrants.\footnote{Note that I have switched the notation so that $\psi$ with no hat refers to incumbents, so that it is consistent with the convention that hats go on variables related to entrants.}
	
	Consider $\psi \in [0,1)$. If $\psi = 0$, we have the baseline model, which admits a closed form solution. If $\psi = 0.5$, then $\tilde{V}$ has a closed form solution given parameters and $\hat{w}_{RD}$, but the model itself does not have a closed-form solution. For all other $\psi \in [0,1)$, there is no closed form solution for $\tilde{V}$ or the equilibrium given $\tilde{V}$.  
\end{proposition}

\begin{proof}
	The normalized incumbent HJB is now
	\begin{align}
	(r + \hat{\tau}) \tilde{V} &= \tilde{\pi} + \max_{\substack{x \in \{0,1\} \\ z \ge 0}} \Big\{  z \Big( z^{-\psi} \chi (\lambda - 1) \tilde{V} - \hat{w}_{RD} - (1-x)(1 - (1-\kappa_e) \lambda)\nu \tilde{V} - x \kappa_c \nu \tilde{V}  \Big)   \Big\} \label{appendix:model:drsincumbent:hjb}
	\end{align} 
	
	The first order condition is
	\begin{align}
	(1-\psi) z^{-\psi} \chi (\lambda -1)\tilde{V} &= \hat{w}_{RD} + (1-x) (1-(1-\kappa_e)\lambda)\nu \tilde{V} + x \kappa_c \nu \tilde{V}
	\end{align}
	
	which implies
	\begin{align}
	z^{1-\psi} &= \Big( \frac{\hat{w}_{RD} + \zeta_1\tilde{V}}{C_2\tilde{V}} \Big)^{\frac{\psi -1}{\psi}} \\
	\zeta_1 &= (1-x)(1-(1-\kappa_e)\lambda)\nu + x\kappa_c\nu \\
	\zeta_2 &= (1-\psi) \chi (\lambda-1)
	\end{align}
	
	Substituting into (\ref{appendix:model:drsincumbent:hjb}) yields
	\begin{align}
	(r + \hat{\tau}) \tilde{V} &= \tilde{\pi} + \Big( \frac{\hat{w}_{RD} + \zeta_1\tilde{V}}{C_2\tilde{V}} \Big)^{\frac{\psi -1}{\psi}} \zeta_2 \tilde{V} - \hat{w}_{RD} - \zeta_1 \tilde{V} 
	\end{align}
	
	This equation has no closed form expression for $\tilde{V}$ except in the quadratic cost case of $\psi = 0.5$, when $\frac{\psi - 1}{\psi} = -1$ and the above becomes
	\begin{align}
	(r + \hat{\tau}) \tilde{V} &= \tilde{\pi} +  \frac{1}{\hat{w}_{RD} + \zeta_1\tilde{V}} - \hat{w}_{RD} - \zeta_1 \tilde{V} 
	\end{align}
	
	Multiplying both sides by $\hat{w}_{RD} + \zeta_1 \tilde{V}$ yields a quadratic equation for $\tilde{V}$, which has solution
	\begin{align}
		\tilde{V} &= \frac{-b \pm \sqrt{b^2 - 4ac}}{2a}
	\end{align}
	
	However, the dependence of $\tilde{V}$ on $\hat{w}_{RD}$, given model parameters, is no longer linear. This means that $\hat{z}$ and $\hat{w}_{RD}$ are defined implicitly as the solution of two equation system, and the model must be solved numerically.	
\end{proof}

\subsection{Symmetric equilibria without $x_{jt} = x$}\label{appendix:model:proofs:proposition:mixedstrategyeq}

Consider the generalized growth accounting equation (it simplifies to the one in the main text when $\mathbbm{1}^{NCA}_{jt} = \mathbbm{1}^{NCA}$),
\begin{align}
	g_t &= (\lambda -1) \Big( \tau + \hat{\tau} + z \nu \int_{j : \mathbbm{1}^{NCA}_{jt} = 0} \frac{\bar{q}_{jt}}{Q_t} dj \Big)
\end{align}

Unless the integral term is constant, then $g_t$ is non-constant, even with constant $z_{jt},\hat{z}_{jt}$. The integral is equal to the product of the mass of incumbents which choose $\mathbbm{1}^{NCA}_{jt} = 0$ and the average relative quality of those incumbents $E[\frac{\bar{q}_{jt}}{Q_t} | \mathbbm{1}^{NCA}_{jt} = 0]$. 

One such construction is to suppose that a constant fraction $f \in (0,1)$ of entering firms choose $\mathbbm{1}^{NCA}_{jt} = 1$ and keep this choice throughout the entire life of the incumbent. The resulting problem is significantly simplified by the fact that the optimal $z_{jt}$ does not depend on the choice of $\mathbbm{1}^{NCA}_{jt}$ (where the incumbent is indifferent). Still, there is some additional complexity coming from two main sources. Comparing BGPs, if $f$ increases, so that entering firms are more likely to choose $\mathbbm{1}^{NCA}_{jt} = 1$, then $\mathbb{E}[\frac{\bar{q}_{jt}}{Q_t} | \mathbbm{1}^{NCA}_{jt} = 1]$ increases as well because new firms are of higher average quality than incumbents. And, for a given $f$, $\mathbbm{1}^{NCA}_{jt} = 1$ firms tend to get replaced less often and hence $\mathbb{E}[\frac{\bar{q}_{jt}}{Q_t} | \mathbbm{1}^{NCA}_{jt} = 1]$ is lower as it puts more weight on older, hence lower quality, incumbents. That is, even for $f = 1/2$ one has $\mathbb{E}[\frac{\bar{q}_{jt}}{Q_t} | \mathbbm{1}^{NCA}_{jt} = 1]$ < $\mathbb{E}[\frac{\bar{q}_{jt}}{Q_t} | \mathbbm{1}^{NCA}_{jt} = 0]$.\footnote{An alternative which sidesteps this issue is to use a Cobb-Douglas aggregator. This requires tracking limit pricing, but has the advantage that the relevant measure of aggregate quality is average of log quality, which means proportional improvements have a given effect on the aggregate regardless of the quality on which they improve.}

With the preceding discussion in mind, below I give a proof of Proposition \ref{proposition:mixedstrategyeq}.

\begin{proof}
	\textbf{[Finish up this proof, and get rid of first part about $m^x$ because second part about $m^x \Gamma^x$ is sufficient.]}
	
	The proof makes concrete the argument in the preceding paragraph. Relative to the baseline model, the only substantial modification is that one needs to derive an expression for the evolution over time of $\mathbb{E}[\frac{\bar{q}_{jt}}{Q_t} | \mathbbm{1}^{NCA}_{jt} = \mathbf{x}]$ for $\mathbf{x} \in \{0,1\}$, and set it equal to zero. This expression will involve the fraction $f \in (0,1)$. 
	
	It will be useful to work with the time-varying $\mathbb{E}[\bar{q}_{jt} | \mathbbm{1}^{NCA}_{jt} = \mathbf{x}]$ instead. To relieve cumbersome notation, let $\Gamma^{\mathbf{x}} = \mathbb{E}[\bar{q}_j | x_j = \mathbf{x}]$ and let $m^{\mathbf{x}}$ denote the mass of goods $j$ which have $x_j = \mathbf{x}$ (in general time-varying, but will be constant on this BGP). In general, $m^0,m^1$ satisfy
	\begin{align}
		\dot{m}^0 &= \overbrace{-(\hat{\tau} + \nu z) m^0}^{\mathclap{\text{outflow}}} + \overbrace{(\hat{\tau} + m^0 \nu z) (1-f)}^{\mathclap{\text{inflow}}}\\
		\dot{m}^1 &= \underbrace{-\hat{\tau} m^1 }_{\mathclap{\text{outflow}}} + \underbrace{(\hat{\tau} + m^0 \nu z) f}_{\mathclap{\text{inflow}}}
	\end{align}
	
	On the BGP, $\dot{m}^{\mathbf{x}} = 0$. Imposing this above yields
	\begin{align}
		m^0 &= \frac{\hat{\tau}}{\hat{\tau} + f \nu z} (1-f)   \label{appendix:model:mixedstrategyeq:m0}\\
		m^1 &= \frac{\hat{\tau} + m^0 \nu z}{\hat{\tau}} f   \label{appendix:model:mixedstrategyeq:m1}
	\end{align}
	
	Next, $m^0\Gamma^0,m^1\Gamma^1$ have their own evolution equations,
	\begin{align}
		m_{t+\Delta}^0\Gamma^0_{t+\Delta} &= \overbrace{(1 - \underbrace{(\hat{\tau} + \nu z) \Delta }_{\mathclap{\text{outflow from creative destruction}}} ) m^0_t  \Gamma_t^0}^{\mathclap{\text{Persisting incumbents}}} + \overbrace{\tau \Delta (\lambda - 1)m_t^0 \Gamma_t^0}^{\mathclap{\text{Improving incumbents}}} + \overbrace{(1-f) \lambda \Big( \underbrace{(\hat{\tau} + \nu z) \Delta m_t^0  \Gamma_t^0}_{\mathclap{\text{from } \mathbbm{1}^{NCA}_{jt} = 0}} +  \underbrace{\hat{\tau} \Delta m_t^1  \Gamma_t^1}_{\mathclap{\text{from } \mathbbm{1}^{NCA}_{jt} = 1}} \Big)}^{\mathclap{\text{Inflows}}} + o(\Delta)\\
		m_{t+\Delta}^1 \Gamma^1_{t+\Delta} &= \overbrace{(1 - \underbrace{\hat{\tau} \Delta }_{\mathclap{\text{outflow from creative destruction}}}) m^1_t  \Gamma_t^1}^{\mathclap{\text{Persisting incumbents}}}  + \overbrace{\tau \Delta (\lambda -1 ) m_t^1 \Gamma_t^1}^{\mathclap{\text{Improving incumbents}}}  + \overbrace{f \lambda \Big( (\hat{\tau} + \nu z) \Delta m_t^0  \Gamma_t^0 + \hat{\tau} \Delta m_t^1  \Gamma_t^1 \Big)}^{\mathclap{\text{Inflows}}} + o(\Delta) 
	\end{align}
	
	where $o(\Delta)$ has the usual meaning that $\lim_{\Delta \to 0} \frac{o(\Delta)}{\Delta} = 0$. Subtracting $m_t^{\mathbf{x}} \Gamma_t^{\mathbf{x}}$, dividing by $\Delta$ and taking the limit as $\Delta \to 0$, and finally using the fact that on the BGP $m_t^{\mathbf{x}} = m_{t+\Delta}^{\mathbf{x}} = m^{\mathbf{x}}$, yields
	\begin{align}
		m^0 \dot{\Gamma}_t^0 &= -(\hat{\tau} + \nu z) m^0 \Gamma_t^0 + \tau (\lambda - 1) m^0 \Gamma_t^0 + (1-f)\lambda \Big( (\hat{\tau} + \nu z) m^0 \Gamma_t^0 + \hat{\tau} m^1 \Gamma_t^1 \Big) \\
		m^1 \dot{\Gamma}_t^1 &= -\hat{\tau} m^1 \Gamma_t^1 + \tau (\lambda - 1) m^1 \Gamma_t^1 + f\lambda \Big( (\hat{\tau} + \nu z) m^0 \Gamma_t^0 + \hat{\tau} m^1 \Gamma_t^1 \Big)
	\end{align}
	
	Dividing by $m^x \Gamma_t^x$ yields
	\begin{align}
	\frac{\dot{\Gamma}_t^0}{\Gamma_t^0} &= -( \hat{\tau} + \nu z) + \tau (\lambda - 1) + (1-f)\lambda \Big( (\hat{\tau} + \nu z) + \hat{\tau} \frac{m^1 \Gamma_t^1 }{m^0 \Gamma_t^0}\Big) \\
	\frac{\dot{\Gamma}_t^1}{\Gamma_t^1} &= -\hat{\tau}  + \tau (\lambda - 1) + f\lambda \Big( (\hat{\tau} + \nu z) \big(\frac{m^1 \Gamma_t^1}{m^0 \Gamma_t^0}\big)^{-1} + \hat{\tau}  \Big)
	\end{align}
	
	In a BGP, we need $m^{\mathbf{x}} \mathbb{E} [\frac{\bar{q}_{jt}}{Q_t} | \mathbbm{1}^{NCA}_{jt} = \mathbf{x}]$ to be constant for each value of $x$. To show that here (given that we have shown that $m^0,m^1$ above are both in $(0,1)$ and hence valid steady state masses of firms in each state) it is sufficient to show that $\mathbb{E}[ \bar{q}_{jt} | \mathbbm{1}^{NCA}_{jt} = \mathbf{x}]$ grows at the same (geometric) rate independent of $\mathbf{x}$. Setting $\frac{\dot{\Gamma}_t^0}{\Gamma_t^0} = \frac{\dot{\Gamma}_t^1}{\Gamma_t^1}$ and multiplying both sides by $\frac{m^1 \Gamma_t^1}{m^0 \Gamma_t^0}$ yields a quadratic equation in $\frac{m^1 \Gamma_t^1}{m^0 \Gamma_t^0}$,
	\begin{align}
		[in progress]
	\end{align}
	
	[\textbf{Prove that the equation has only one positive solution}] The BGP value of $\frac{m^1 \Gamma_t^1}{m^0 \Gamma_t^0}$ is determined by the unique positive solution to this quadratic equation. Then using $Q_t = m^0 \Gamma_t^0 + m^1 \Gamma_t^1$ one can solve for $m^{\mathbf{x}} \Gamma_t^{\mathbf{x}}$ in terms of $Q_t$ and other equilibrium variables. In particular, one can compute the (constant) value of $\frac{m^0 \Gamma_t^0}{Q_t}$, which is necessary for computing the BGP growth rate from individual policies using the generalized accounting equation, which in this case can be written 
	\begin{align}
	g &= (\lambda - 1) (\tau + \hat{\tau} + z \nu  \frac{m^0 \Gamma^0_t}{Q_t} )
	\end{align}
	
	Other than this equation, the equations characterizing the BGP are the same as in the main text. The only differences in the equilibrium are through general equilibrium effects of the change in $g$: the BGP interest rate will be higher for lower $f$ due to higher growth, and this will tend to reduce R\&D wages as well. The incumbent value $\tilde{V}$ may increase or decrease.
\end{proof}

\subsubsection{Application to model with static heterogeneity}

The above construction and derivation can be adapted to a more complicated model where there is heterogeneity in $\{\kappa_e, \kappa_c, \nu\}$ across goods $j$, inducing heterogeneity in chosen $\{z_j,x_j\}$. A tractable BGP in this setup only requires two things: that the state variable be constant throughout the life of the incumbent and that the Markov process by which goods $j$ move between states satisfy a monotone mixing condition. The former ensures that the incumbent's HJB has no additional state variables. The latter condition essentially requires that there be no "absorbing subset" of states. This is similar to the standard necessary conditions for the existence of a stationary equilibrium in models with heterogeneous agents.

The density $\mu(x)$ can be derived using the Kolmogorov forward equation,
\begin{align}
0 &= [in progress]
\end{align}
The system of difference equations for $m^{\textbf{x}} \Gamma_t^{\textbf{x}}$ are replaced by a functional difference equation, 
\begin{align}
\mu(x) \Gamma_{t+\Delta}^x &= (1- \text{CD}(x) \Delta) \mu(x) \Gamma_t^x + \text{OI}(x) \Delta (\lambda -1) \mu(x) \Gamma_t^x +  j^x \Delta  \lambda \int_{x' \in \mathbf{X}} \text{CD}(x') \Gamma_t^{x'} \mu(x') dx'
\end{align}
where $j^x$ is the density of the injection rate into state $x$ out of new incumbents. This can then be used to derive a functional differential equation,
\begin{align}
\frac{\dot{\Gamma}_{t}^x}{\Gamma_t^x} &= OI(x) (\lambda -1) - CD(x) + j^x \lambda (\mu(x) \Gamma_t^x)^{-1} \int_{x' \in \mathbf{X}} \text{CD}(x') \Gamma_t^{x'} \mu(x') dx'
\end{align}

Imposing the condition $\frac{\dot{\Gamma}_{t}^x}{\Gamma_t^x} = g$ for an unknown constant $g$ pins down the ratio $\frac{\int_{x' \in \mathbf{X}} \text{CD}(x') \Gamma_t^{x'} \mu(x') dx'}{\mu(x) \Gamma_t^x}$ for each $x$, determining the shape of the distribution $\Gamma_t^x$ (since $\mu(x)$ is already determined by the KF equation). If the relevant functions are differentiable, the condition can also be derived by differentiating the expression for $\frac{\dot{\Gamma}_t^x}{\Gamma_t^x}$ with respect to $x$ and setting it equal to zero. This yields
\begin{align}
0 = \text{OI}'(x) (\lambda -1) - \text{CD}'(x) + \lambda \int_{x' \in \mathbf{X}} \text{CD}(x') \Gamma_t^{x'} \mu(x') dx' \Big(\frac{d}{dx} j^x \mu(x) \Gamma_t^x \Big)^{-1}
\end{align} 

where one would need to expand the last derivative using the product rule (recalling that all three terms depend on $x$). 

The scale of the distribution at time $t$ is determined by 
\begin{align}
\int_{x' \in \mathbf{X}} \Gamma_t^{x'} \mu(x') dx' = Q_t
\end{align}




\section{Calibration}\label{appendix:calibration}

\subsection{Computing model moments}

\subsubsection{R\&D / GDP}\label{appendix:calibration:rd/gdp}

In the model, the R\&D share is the ratio of the wage paid to R\&D workers to GDP. This is
\begin{align*}
\frac{\textrm{R\&D wage bill}}{\textrm{GDP}} &= \frac{w_{RD} z + \hat{w}_{RD} \hat{z}}{\tilde{Y}} \\ 
&= \frac{\hat{w}_{RD} (z + \hat{z}) + (w_{RD} - \hat{w}_{RD})z}{\tilde{Y}} \\
&= \frac{\hat{w}_{RD} (z + \hat{z}) - (1-\kappa_e) \lambda \tilde{V} \tau^S}{\tilde{Y}}
\end{align*}

where I used $w_{RD} - \hat{w}_{RD} = -(1-x)(1-\kappa_e) \lambda \tilde{V} \nu$ and $\tau^S = (1-x)\nu z$. 

\subsubsection{Growth share OI}\label{appendix:calibration:growthShareOI}

The model moment that corresponds here is the share of growth due to own innovation by incumbents of age >= 6. In the model, the fraction of OI growth due to incumbents in a given age group is exactly their fraction of employment: innovations arrive at the same rate for each incumbent, and their impact on aggregate growth is proportional to the incumbent's relative quality, which is proportional to employment. Hence old incumbents' share of growth due to own innovation is simply one minus the employment share calculated in the previous paragraph, $e^{((\hat{\tau}_I -1)g - (\hat{\tau} + (1-x)z \nu))\cdot 6}$. Finally, the fraction of aggregate growth due to OI is $\hat{\tau}_i$, defined above. The fraction of growth due to incumbents of age at least 6 is the product of the two, 
\begin{align*}
\textrm{Age >= 6 share of OI} &= \hat{\tau}_I \frac{\ell(6)}{\ell(0)} \\
&= \hat{\tau}_I e^{((\hat{\tau}_I -1)g - (\hat{\tau} + (1-x)z \nu))\cdot 6} 
\end{align*}


\subsubsection{Entry rate}\label{appendix:calibration:entryRate}

Let $\ell(a)$ denote the density of incumbent employment at age $a$ incumbents. Then $\ell(a)$ is characterized by 
\begin{align*}
\ell(a) &= \ell(0)e^{((\hat{\tau}_I -1)g - (\hat{\tau} + (1-x)z \nu))a}  \\
1 + \bar{L}_{RD} - \hat{z} &= \int_0^{\infty} \ell(a) da
\end{align*}

where $\hat{\tau}_I = \frac{\tau}{\tau + \hat{\tau} + \tau^S}$ is the fraction of innovations that are incumbents' own innovations. 

The intuition for this characterization of $\ell(a)$ has two parts. First, because all shocks are \textit{iid} across firms in equilibrium, the law of large numbers applied to each cohort of firms implies that we can consider directly the evolution of the cohort as a whole instead of explicitly analyzing the dynamics each individual firm in the cohort.  Second, the employment of a firm is proportional to its relative quality, $l_j \propto \tilde{q}_j = q_j / Q$, as long as it is the leader. When it is no longer the leader, its employment is zero forever. Putting these two together, $\ell(a)$ must decline at exponential rate $g$ due to the increase in $Q_t$ (obsolescence), increase at rate $\hat{\tau}_I g$ due to incumbents own innovations, and decline at rate $\hat{\tau} + \tau^S$ due to creative destruction.\footnote{The second equation imposes consistency with aggregate employment; it implies $\ell(0) = -((\hat{\tau}_I -1)g - (\hat{\tau} + \tau^S))(1 + \bar{L}_{RD})$. The calibration does not require this explicit calculation since it is based only on employment shares.} Note that the employment density is strictly decreasing in $a$. This is because there are no adjustment costs: firms achieve their optimal scale immediately upon entry, and subsequently become obsolete (on average) or lose the innovation race to an entrant. Finally, due to the constant exponential decay of $\ell(a)$, the share of incumbent employment in incumbents of strictly less than 6 years of age is given by 
\begin{align*}
\Xi_{[0,6)} &=  1 - \frac{\ell(6)}{\ell(0)} \\
&= 1 - e^{((\hat{\tau}_I -1)g - (\hat{\tau} + \tau^S))\cdot 6}
\end{align*}  


The share of overall employment in incumbents of age < 6, including R\&D performed by non-producing entrants, is equal to the previously calculated $\Xi_{[0,6)}$, multiplied by the share of total labor in incumbents, $1 - \hat{z}$, added to the R\&D labor used by entrants $\hat{z}$, 
\begin{align*}
\textrm{Age < 6 share of employment} &= \frac{2}{3}(\Xi_{[0,6)} (1-\hat{z}) + \hat{z})
\end{align*}

The factor 2/3 follows from interpreting entrants in the model as either new firms or incumbents engaging in creative destruction. According to KH 2020, creative destruction by incumbents is responsible for half as much growth as creative destruction by entrants. In this interpretation of the model, both types of creative destruction use the same technology. Therefore, 2/3 of employment in young firms in the model represents employment in young firms in the data.\footnote{Note that the formula in the text assumes that final goods firms have the same employment-age distribution as intermediate goods incumbents. Alternatively, one could assume that final goods firms have the same employment-age distribution as the rest of the firms in the economy, i.e. including entrants. Then the formula would be
	\begin{align*}
	\textrm{Age < 6 share of employment} &= \frac{2}{3} \frac{(\Xi_{[0,6)} (1 - L_F -\hat{z}) + \hat{z})}{1-L_F}
	\end{align*}
	
	This has only minor effects on the inferred parameters. They are listed in \autoref{calibration_2_parameters} \textbf{[update table with new calibration]}}

\subsubsection{Employment share of WSOs}\label{appendix:calibration:WSOempShare}

Because successfully innovating spinouts and entrants have identical expected growth dynamics, the BGP share of employment in firms started as spinouts is their share of new incumbents $\frac{\tau^S}{\tau^S+ (\frac{2}{3})\hat{\tau}}$, multiplied by the employment share of incumbents $\frac{1-L_F- (\frac{2}{3})\hat{z}}{1-L_F}$, 
\begin{align*}
\textrm{Spinout employment share} &= \frac{\tau^S}{\tau^S + \frac{2}{3}\hat{\tau}} \times \frac{1-L_F- (\frac{2}{3})\hat{z}}{1-L_F} 
\end{align*}

Again, the factor 2/3 is because this is the fraction of entrants in the model which the calibration maps to new firms in the data.

\section{Policy analysis}

\subsection{NCA cost $\kappa_c$}\label{appendix:policyanalysis:ncacost}

\paragraph{Robustness of welfare gain from NCA enforcement}

\autoref{welfareComparisonSensitivityFull} shows the sensitivity of the welfare comparison the moments targeted, including the externally calibrated parameters as pseudo-moments as before. It is computed as $\nabla_m \tilde{W}|_m = (J^{-1})^T \nabla_p W|_p$, where $J$ is the Jacobian of the mapping from log parameters to moments (so that $J^{-1}$ is the Jacobian of the inverse mapping), and $W$ is the mapping from parameters the log \% change (or raw \% change, in \autoref{levelsWelfareComparisonSensitivityFull})) in CEV welfare from reducing $\kappa_c$ from $\infty$ to $0$. That is, it is the gradient of the change in welfare to the change in target moments or uncalibrated parameters, taking as given the change in parameters required to continue matching the target moments. For reference, $\nabla_p W|_p$  for each definition of $W$ can be found in \autoref{welfareComparisonParameterSensitivityFull} and \autoref{levelsWelfareComparisonParameterSensitivityFull}.

Suppose that the log of each moment is assumed to have a standard deviation of $\sigma = 0.05$, and that this uncertainty is statistically independent across moments. The uncertainty propagates such that the standard deviation of the CEV welfare change is the square root of $(\nabla_m \tilde{W}|_m)^T \Sigma_m \nabla_m \tilde{W}|_m$, where $\Sigma_m = \sigma^2 I_{9\times 9}$. In this examples this yields 0.31 log points (0.45 percentage points). Also, in both cases the result is linear in $\sigma$. Hence with $\sigma = 0.1$, the result is 0.62 log points (0.90 percentage points), etc. 

The estimated welfare improvement is about 1.4\%. Taking the uncertainty into account, the $2\sigma$ uncertainty region excludes zero for $\sigma \le .112$. This suggests the result is quite sensitive to the moments and non-calibrated parameters used in the calibration. 


\begin{figure}[]
	\includegraphics[scale = 0.36]{../code/julia/figures/simpleModel/welfareComparisonSensitivityFull.pdf}
	\caption{Sensitivity of welfare comparison to moments. This is $(J^{-1})^T \nabla_p W$, where $W(p)$ maps log parameters to the log of the percentage change in BGP consumption which is equivalent to the change in welfare from changing $\kappa_c$ from $\infty$ to $0$ (i.e. going from banning to frictionlessly enforcing NCAs). The way to read this is the following. Looking at the column labeled \textit{E}, the chart says that a 1\% increase in the targeted employment share of young firms, which corresponds to a log change of about $0.01$, leads to a 4\% increase in the percentage CEV percentage welfare change. In this calibration it is about 1.42\%, so this is about $0.057$ percentage points.}
	\label{welfareComparisonSensitivityFull}
\end{figure}


\paragraph{When are NCAs bad for welfare?}

The sensitivity of the welfare improvement to the entry rate shown in (\ref{welfareComparisonSensitivityFull}) suggests that a calibration targeting a lower rate of creative destruction could have the opposite result. \autoref{calibration_lowEntry_summaryPlot} shows the analogue of \autoref{calibration_summaryPlot} if entry rate targeted is 4\% instead of 8.35\%. The model is again able to match the moments exactly; inferred parameter values are shown in \autoref{calibration_lowEntry_parameters}.

As expected, growth and welfare fall when $\kappa_C$ is reduced so that $x = 1$. Mathematically, this results from the much higher inferred value of $\lambda$: it is 1.29 in this calibration, versus 1.08 in the original calibration. Intuitively, the lower rate of entry means that each entry must have a higher effect on growth in order for the model to match the growth rate. Furthermore, as shown in \autoref{welfareComparisonParameterSensitivityFull}, the increase in $\lambda$ significantly reduces the welfare gain from reducing $\kappa_C$. A higher value of $\lambda$ weakens inequality (\ref{cs:growth_misallocation_condition}), reducing the effect of the incumbent disincentive on growth. While the inequality still holds, the extent of misallocation is weaker. Therefore, inducing further misallocation via an increase in $\kappa_C$ has a smaller negative effect on growth. The present exercise shows that these local relationships are to some extent global and in fact strong enough to switch the sign of the welfare comparison.

\textbf{[Add description of changes in $\kappa_e$]}

\begin{figure}[]
	\includegraphics[scale = 0.57]{../code/julia/figures/simpleModel/calibration_lowEntry2_summaryPlot.pdf}
	\caption{Summary of equilibrium for baseline parameter values and various values of $\kappa_c$.}
	\label{calibration_lowEntry_summaryPlot}
\end{figure}

\begin{table}[]
	\centering
	\captionof{table}{Low entry rate calibration}\label{calibration_lowEntry_parameters}
	\begin{tabular}{rlll}
		\toprule \toprule
		Parameter & Value & Description & Source \tabularnewline
		\midrule
		$\rho$ & 0.0303 & Discount rate  & Indirect inference \tabularnewline
		$\theta$ & 2 & $\theta^{-1} = $ IES & External calibration 
		\tabularnewline
		$\beta$ & 0.094 & $\beta^{-1} = $ EoS intermediate goods & Exactly identified \tabularnewline 
		$\bar{L}_{RD}$ & 0.01 & R\&D labor allocation  & Exactly identified \tabularnewline
		$\psi$ & 0.5 & Entrant R\&D elasticity & External calibration \tabularnewline
		$\lambda$ & 1.23 & Quality ladder step size & Indirect inference 
		\tabularnewline
		$\chi$ & 5.861 & Incumbent R\&D productivity & Indirect inference 
		\tabularnewline
		$\hat{\chi}$ & 0.370 & Entrant R\&D productivity & Indirect inference \tabularnewline 
		$\kappa_e$ & 0.885 & Non-R\&D entry cost & Indirect inference \tabularnewline
		$\nu$ & 0.144 & Spinout generation rate  & Indirect inference\tabularnewline
		\bottomrule
	\end{tabular}
\end{table}









\end{document}