\documentclass[12pt,english]{article}
\usepackage{lmodern}
\usepackage[T1]{fontenc}
\usepackage[latin9]{inputenc}
\usepackage{geometry}
\usepackage{amsthm}
\usepackage{courier}
\usepackage{verbatim}
\geometry{verbose,tmargin=1in,bmargin=1in,lmargin=1in,rmargin=1in}
\usepackage{setspace}
%\usepackage{esint}
\onehalfspacing
\usepackage{babel}
\usepackage{amsmath}

\theoremstyle{remark}
\newtheorem*{remark}{Remark}
\begin{document}
	
\title{Is there a BGP - updated 7-18}
\author{Nicolas Fernandez-Arias}
\maketitle

\section{Equations and Unknowns}

If the model I am interested in with total labor supply $L = 1$ admits a constant $\gamma_t(m)$, defined below, then so should the model with any other labor supply $L'$. Hence, I take as given $L^F$, compute the implied $\bar{w}$, and attempt to find a BGP. 

The \textbf{exogenous variables }are:
\begin{enumerate}
	\item The final goods wage $\bar{w}$
	\item The rate at which flow profits scale with the quality of the good, $\pi$ (i.e. flow profits are $\pi q $)
\end{enumerate}

The \textbf{endogenous variables }are \big(for $m \ge 0$ \big):
\begin{enumerate}
	\item Incumbent value function $V(m)$
	\item Incumbent policy $z_I(m)$
	\item Spinout value function $W(m)$
	\item Aggregate spinout R\&D effort on a good in state $m$, $z_S(m)$
	\item Aggregate non-spinout entrant R\&D effort on a good in state $m$, $z_E(m)$
	\item Arrival rate due to entrants and spinouts $\tau_{SE}(m)$
	\item Total arrival rate of innovations $\tau(m)$
	\item Drift rate $a(m)$
	\item Wage $w(m)$
	\item Stationary distribution $\mu(m)$ 
	\item Stationary $\gamma(m) = E[\tilde{q}|m]$
	\item Growth rate $g$
\end{enumerate}

The\textbf{ system of equations} satisfied in a recursive equilibrium is:\footnote{Technically, I need to add some transversality conditions to the HJBs. Perhaps I can just add boundary conditions like $\lim_{m\to \infty} V'(m) = \lim_{m \to \infty} W(m) = 0$. Is this a guess of form of eq.? Is this okay? Try it, try w/o it, ask Ezra.}\footnote{Note: boundary condition can just be written $V(M) = \frac{\pi + z_I(m)\chi_I \phi(z_I(m))\lambda V(0)}{\rho + \tau(m)}$ and $W(M) = 0$, where $M$ is the free-entry threshold. In equilibrium, $V(m)$ is continuous (by economic reasoning, i.e. since it solves a maximization problem) on $[0,\infty)$ and differentiable on $[0,M)$. So we can solve numerically in the same way. Finally, the existence of $M$ is guaranteed as long as $w(m)$ is increasing. I haven't proven this, but I conjecture (and it's very reasonable) that this will be true in equilibrium, since if $w(m)$ is exogenously given and constant, everything holds as well, and (\ref{Indifference}) wants to then make $w(m)$ increasing.} 
\begin{align}
(\rho + \tau_{SE}(m))V(m) &=  \pi + a(m) V'(m) \label{HJB_V}\\
						  &+ z_I(m)\chi_I \phi(z_I(m)) (\lambda V(0) - V(m)) \nonumber \\
						  &- z_I(m)w(m) \nonumber \\ 
 V(M) &= \frac{\pi + z_I(m)\chi_I \phi(z_I(m))\lambda V(0)}{\rho + \tau(M)} \label{HJB_V_Boundary} \\
(\rho + \tau(m)) W(m) &= a(m) W'(m) \label{HJB_W}\\
						&+ z_S(m)\chi_S\eta(z_S(m) + z_E(m))\lambda V(0) \nonumber \\
						&- z_S(m)w(m)  \nonumber	\\	
\lim_{m \to \infty} W(m) &= 0 \label{HJB_W_Boundary} \\ 										  
w(m) - \nu V'(m) &= z_I(m)\chi_I\phi'(z_I(m)) + \chi_I \phi(z_I(m)) (\lambda V(0) - V(m)) \label{HJB_V_FOC}\\
w(m) &= \chi_E\eta(z_S(m) + z_E(m)) \lambda V(0) \label{E_FreeEntry}\\ 
\Big(\forall m: z_S(m) = \xi m \Big) w(m) &\le \chi_S \eta(z_S(m) + z_E(m)) \lambda V(0) \label{HJB_W_Optimality1}\\
\Big(\forall m: z_S(m) < \xi m \Big) w(m) &= \chi_S \eta(z_S(m) + z_E(m)) \lambda V(0) \label{HJB_W_Optimality2}\\
\bar{w} &= w(m) + \nu W(m) \label{Indifference}\\
0 &= - a'(m)\mu(m) - a(m)\mu'(m) - \tau(m)\mu(m) \label{KF}\\
1 &= \int_{0}^{\infty} \mu(m) dm \label{KF_integral}\\
g &= (\lambda - 1) \int_{0}^{\infty} \tau(m) \mu(m) dm \label{growth_aggregation} \\
\gamma'(m) &= \frac{-g }{a(m)} \gamma(m) \label{gamma_firstorder}\\
1 &= \int_{0}^{\infty} \mu(m) \gamma(m) dm \label{gamma_integral}\\ 
\gamma(0) &= \lambda \times \frac{\int_{0}^{\infty} \mu(m) \tau(m) \gamma(m)  dm}{\int_{0}^{\infty} \mu(m) \tau(m) dm} \label{gamma_consistency}\\
a(m) &= \big( z_I(m) + z_S(m) + z_E(m) \big) \nu \label{a_def}\\
\tau_{SE}(m) &= \big(z_S(m) \chi_S + z_E(m) \chi_E \big) \eta \big(z_S(m) + z_E(m)\big) \label{tau_SE_def}\\
\tau(m) &= z_I(m) \chi_I \phi(z_I(m)) + \tau_{SE}(m)  \label{tau_def}
\end{align}

\section{Solving the system}

\subsection{Preliminary ramblings}

First, equations (\ref{a_def}),(\ref{tau_SE_def}), and (\ref{tau_def}) are just definitions and hence take care of one equation each. 

The point is this: given $w(m)$, the first few equations pin down the value functions $V(m),W(m)$, policies $z_I(m),z_E(m),z_S(m)$, arrival rates $\tau_{SE}(m),\tau(m)$ and drift $a(m)$. Through the KF equation (first order and integral), this pins down $\mu(m)$. Then the growth aggregation equation pins down $g$. Then, the first-order and integral equations pin down $\gamma(m)$. Finally, (\ref{Indifference}) pins down $w(m)$. We are left with an extra equation, (\ref{gamma_consistency}). 

The only hope is that, given the solutions to the other equations, this equation, which looks to be non-trivial, becomes redundant. Not sure how one would prove this. All that is left is to (1) try to solve it on a computer, or (2) think about the economics of the situation. Why should there not be a constant expected relative quality given $m$? 

If we start with $\gamma_0(m)$ that satisfies (\ref{gamma_firstorder}) and (\ref{gamma_integral}), how does it evolve? First, assume for simplicity that growth is constant and equal to $g$. At all times we will have 
\begin{align*}
	\gamma_t(0) = \lambda \times \frac{\int_{0}^{\infty} \gamma_t(m) \tau_t(m) \mu_t(m) dm}{\int_{0}^{\infty} \tau_t(m) \mu_t(m) dm}
\end{align*}

The evolution is as follows. At each $t \ge 0$, there is a boundary, $b(t) \in \mathbf{R}$, determined by $a(m)$. To the right of $b(t)$, we will have $\gamma_t(m) = \gamma_0(m)$. To the left of $b(t)$, will have $\gamma_t(m) = e^{-g(t-t')} \gamma_{t'} (m')$, where $t' < t$ is the time when 

The boundary is determined by the evolution of the frontier. The movement of the frontier is easy to calculate because at any time the rate of drift in the $m$-direction of a given good $j$ depends only on its current state $m$. Hence, goods cannot ``pass'' each other in $m$-space, and we only need to track the movement through $m$-space of the goods which 

It will be very difficult to show anything theoretically. Everything is endogenous and simultaneous, and I do not have closed forms. The distribution $\mu(m)$ is determined by $a(m)$, which is determined by the wage $w(m)$. The conditional expectation $\gamma(m)$ is determined by $\mu(m)$ and $g/a(m)$. The growth rate $g$ in turn is determined by the integral of $\mu(m)\tau(m)$. Finally $\tau(m)$ is determined by the Nash Equilibrium given 

\paragraph{Derivation of $\gamma(m)$ given $a(m),\tau(m)$}
The differential equation for $\mu(m)$ has solutions of the form
\begin{align*}
	\mu(m) &= C_{\mu}\exp\Bigg\{ - \int_0^m \frac{a'(m') + \tau(m')}{a(m')} dm'\Bigg\}
\end{align*}

The differential equation for $\gamma(m)$ has solutions of the form
\begin{align*}
	\gamma(m) &= C_{\gamma} \exp\Bigg\{ -g \int_0^m a(m')^{-1} dm' \Bigg\}
\end{align*}

Given these functional forms, let's check whether any of the equations determining $\gamma$ or $\mu$ are redundant. Equation (\ref{gamma_integral}) - the Law of Iterated Expectations - becomes
\begin{align*}
	1 &= \int_{0}^{\infty} \mu(m) \gamma(m) dm \\
	  &= C_{\mu} C_{\gamma} \int_0^{\infty} \exp\Big\{ - \int_0^m \frac{a'(m') + \tau(m')}{a(m')} dm' \Big\} \exp\Big\{ -g \int_{0}^{m} a(m')^{-1}  \Big\} dm \\
	  &= C_{\mu} C_{\gamma} \int_0^{\infty} \exp\Big\{ - \int_0^m \frac{g + a'(m') + \tau(m')}{a(m')} dm' \Big\} dm
\end{align*}

Doesn't look like anything can be done. Ok, how about (\ref{gamma_consistency})? Substituting in the above functional form, we see that $C_{\gamma} = \gamma(0)$ hence (\ref{gamma_consistency}) becomes
\begin{align*}
	C_{\gamma} &=  \lambda \times \frac{\int_{0}^{\infty} \gamma(m) \tau(m) \mu(m) dm}{\int_{0}^{\infty}\tau(m) \mu(m) dm}
\end{align*}

\paragraph{Simple case of constant drift / innovation arrival ($a(m) \equiv a, \tau(m) \equiv \tau$)}

What happens if we set $\chi_E \le \chi_S$? Then, in actuality, the equilibrium of the model has constant $a(m) \equiv a$ and $\tau(m) \equiv \tau$, since spinouts are redundant. We also have in this case $w(m) \equiv \bar{w}$ in equilibrium. Moreover, we may even be able to get closed forms for everything. What happens with $\gamma_t(m)$ in this case? Is there a $\gamma_0(m)$ which would remain constant if we ran the model forward? If not, why not?

In this case, the necessary conditions for a constant $\gamma_t(m) = \gamma(m)$ are 
\begin{align*}
	\mu(m) &= (\tau / a) e^{-(\tau/a)m} \\
	\gamma(m) &= \lambda e^{-(g/a)m} \\
	1 &= \int_0^{\infty} \gamma(m) \mu(m) dm
\end{align*}

Substituting the first two into the last yields
\begin{align*}
	1 &= (\lambda \tau / a)  \int_0^{\infty} e^{-(g/a)m} e^{-(\tau/a)m} dm \\ 
	  &= (\lambda \tau / a)  \int_0^{\infty} e^{-((g + \tau)/a)m} dm
\end{align*}

This imposes a relationship between $\lambda$,$\tau$, and $a$.

We also have (\ref{growth_aggregation}), using the fact that $\tau(m)$ is constant, 
\begin{align*}
	g &= (\lambda-1) \tau 
\end{align*}

Substituting this in we get 
\begin{align*}
	1 &= (\lambda \tau / a) \int_0^{\infty} e^{\lambda \tau / a} m dm \\
	  &= 1
\end{align*}

so that there exists a BGP. So in this case, it works!

\subsection{Solving the general case}
The basic idea is that we can make a change of variables $m \to m'$ so that the drift in the $m'$ direction is constant. Hence, we can assume without loss of generality that $a(m) = 1$. Although, we do have to slightly modify the equations - essentially now we replace (\ref{a_def}) with $a(m) = 1$, because $1$ is what goes into the KF equation, etc. 

This implies that
\begin{align*}
	\mu(m) &= C_{\mu} \exp \Big\{  -\int_{0}^{m} \tau(m') dm'\Big\} \\
	1 &= \int_{0}^{\infty} \mu(m) dm \\
	\gamma(m) &= C_{\gamma} e^{-gm} \\
	1 &= \int_{0}^{\infty}\gamma(m) \mu(m) dm \\ 
	C_{\gamma} &= \lambda \times \frac{\int_{0}^{\infty} \mu(m) \tau(m) \gamma(m) dm}{\int_{0}^{\infty} \mu(m) \tau(m)dm} \\
	g &= (\lambda -1) \int_{0}^{\infty} \tau(m) \gamma(m) \mu(m) dm
\end{align*}

Substitute out the functional equations into the integrals to arrive at 4 equations in 3 1-dimensional unknowns:
\begin{align*}
	C_{\mu}^{-1} &= \int_{0}^{\infty} \exp \Big\{ -\int_0^m \tau(m') dm'\Big\} dm \\
	C_{\gamma}^{-1} &= \int_{0}^{\infty} e^{-gm}  C_{\mu} \exp \Big\{ - \int_0^m \tau(m') dm'\Big\} dm \\
	C_{\gamma} &= \lambda \times  \frac{ C_{\gamma} C_{\mu} \int_{0}^{\infty} \exp \Big\{  -\int_{0}^{m} \tau(m') dm'\Big\}  \tau(m) e^{-gm} dm}{C_{\mu} \int_{0}^{\infty} \exp \Big\{  -\int_{0}^{m} \tau(m') dm'\Big\}  \tau(m) dm} \\
	g &= (\lambda -1) C_{\gamma} C_{\mu} \int_{0}^{\infty} \tau(m) e^{-gm} \exp \Big\{  -\int_{0}^{m} \tau(m') dm'\Big\}   dm
\end{align*}

We can further simplify a bit: 
\begin{align*}
C_{\mu}^{-1} &= \int_{0}^{\infty} \exp \Big\{- \int_0^m \tau(m') dm'\Big\} dm \\
C_{\gamma}^{-1} &= \int_{0}^{\infty} e^{-gm}  C_{\mu} \exp \Big\{-\int_0^m \tau(m') dm'\Big\} dm \\
1 &= \lambda \times  \frac{ \int_{0}^{\infty} \exp \Big\{  -\int_{0}^{m} \tau(m') dm'\Big\}  \tau(m) e^{-gm} dm}{\int_{0}^{\infty} \exp \Big\{  -\int_{0}^{m} \tau(m') dm'\Big\}  \tau(m) dm} \\
g &= (\lambda -1) C_{\gamma} C_{\mu} \int_{0}^{\infty} \tau(m) e^{-gm} \exp \Big\{  -\int_{0}^{m} \tau(m') dm'\Big\}   dm
\end{align*}

Given $\tau(m)$, the first equation yields $C_{\mu}$. The third equation yields $g$. The second equation then yields $C_{\gamma}$. Hence, the fourth equation cannot impose any additional restrictions - it must be redundant in order for this system to have a solution. Note that we know this even though there is nothing linear about this system, since I can prove that given $\tau(m)$ there is one choice of $C_{\mu}, g, C_{\gamma}$ consistent with the first three equations. 

Furthermore, we have 
\begin{align*}
	\int_{0}^{\infty} \tau(m) e^{-\int_0^m \tau(m')dm'} dm &= 1 
\end{align*}

This follows because 
\begin{align*}
	\frac{d}{dm} \big[ -e^{-\int_0^m \tau(m')dm'} \big] &= \tau(m)e^{-\int_0^m \tau(m')dm'}
\end{align*}

Since $\lim_{m \to \infty} e^{-\int_0^m \tau(m')dm'} = 0$ and $e^{-\int_0^0 \tau(m')dm'} = 1$, this implies the first equation. Similarly, can get
\begin{align*}
	\int_0^{\infty} \tau(m)e^{-\int_0^m g + \tau(m')dm'} dm &= 1 - g C_{\gamma}^{-1} C_{\mu}^{-1}
\end{align*}

To see this, note that
\begin{align*}
		\frac{d}{dm} \big[ -e^{-\int_0^m g + \tau(m')dm'} \big] &= (g +\tau(m))e^{-\int_0^m g + \tau(m')dm'}
\end{align*}

This means that 
\begin{align*}
	1 &= \int_0^{\infty} (g + \tau(m)) e^{-\int_0^m g+ \tau(m')dm'} dm \\
	1 - g\int_0^{\infty} e^{-\int_0^m g + \tau(m') dm'} dm &= \int_0^{\infty} \tau(m) e^{-\int_0^m g+ \tau(m')dm'} dm \\ 
	1 - g C_{\gamma}^{-1} C_{\mu}^{-1} &= \int_0^{\infty} \tau(m) e^{-\int_0^m g+ \tau(m')dm'} dm
\end{align*}
after substituting in the second equation.

Substituting these relationships into the system, it becomes:
\begin{align*}
	C_{\mu}^{-1} &= \int_0^{\infty} e^{-\int_0^m \tau(m')dm'}dm \\
	C_{\gamma}^{-1} C_{\mu}^{-1} &= \int_0^{\infty} e^{-\int_0^m g + \tau(m') dm'} dm \\
	1 &= \lambda (1 - gC_{\gamma}^{-1} C_{\mu}^{-1}) \\ 
	g &= (\lambda -1) C_{\gamma}C_{\mu} (1 - gC_{\gamma}^{-1} C_{\mu}^{-1}) 
\end{align*}

Now define a new variable $C = (C_{\gamma}C_{\mu})^{-1}$. The system is now
\begin{align*}
	C &= (C_{\gamma}C_{\mu})^{-1} \\
	C_{\mu}^{-1} &= \int_0^{\infty} e^{-\int_0^m \tau(m')dm'} dm \\
	C &= \int_0^{\infty} e^{-\int_0^m g + \tau(m') dm'} dm \\
	1 &= \lambda (1 - gC) \\
	g &= (\lambda -1) C^{-1}(1 - gC)
\end{align*}

It can be easily shown now that the last two equations are redundant. The last equation boils down to 
\begin{align*}
	g &= (\lambda -1) (C^{-1} - g) \\
	g (1 + \lambda - 1) &= (\lambda -1) C^{-1} \\ 
	g &= \frac{\lambda -1}{\lambda} C^{-1}  
\end{align*}

The fourth equation also amounts to the same:
\begin{align*}
	1 &= \lambda(1 - gC) \\
	1 &= \lambda - \lambda g C\\
	g &= \frac{\lambda -1}{\lambda } C^{-1}
\end{align*}

Hence we drop one of the equations. We now can use the second and third or second and fourth equations to determine $C,g$. I still have to show that there exists a $g$ such that
\begin{align*}
	g &= \frac{\lambda -1}{\lambda} \Big( \int_{0}^{\infty} e^{-\int_0^m g + \tau(m') dm'} dm  \Big)
\end{align*}

Both LHS and RHS are continuous. As $g$ increases from $0$ to $+\infty$, the RHS goes from some positive number to 0. By the intermediate value theorem there exists some $g$ solving the equation (there could be several). Then $C$ is computed from $g$ according to $C = \frac{\lambda - 1}{\lambda} g^{-1}$. 

Once we know that there exist $g,C$ solving the system, we know that $C_{\mu}$ is well-defined by 
\begin{align*}
	C_{\mu}^{-1} &=\int_0^{\infty} e^{-\int_0^m \tau(m')dm'} dm
\end{align*}

and $C_{\gamma}$ is well-defined by $C_{\gamma} = C^{-1} C_{\mu}^{-1}$. 

\paragraph{Case 1: Constant $\tau(m) \equiv \tau$}

The system becomes
\begin{align*}
	C_{\mu}^{-1} &= \int_{0}^{\infty} e^{-\tau m} dm \\
	C_{\gamma}^{-1} &= C_{\mu} \int_{0}^{\infty} e^{-(g+\tau)m}  dm \\
	1 &= \lambda \times  \frac{ \int_{0}^{\infty} e^{-(g+\tau)m} dm}{\int_{0}^{\infty} e^{-\tau m}dm} \\
	g &= (\lambda -1) \tau C_{\gamma} C_{\mu} \int_{0}^{\infty} e^{-(g+\tau)m} dm
\end{align*}

The first two equations immediately yield $C_{\mu} = \tau$ and $C_{\gamma} = \frac{g+\tau}{\tau}$. By substitution, the fourth equation yields $g = (\lambda - 1)\tau$. Substituting into the third equation yields $1 = 1$, i.e. we have a solution.

\paragraph{Case 2: Non-constant $\tau(m)$}

First, suppose we have 
\begin{align*}
	\tau(m) =  \begin{cases} 
	0 & m < m^*  \\
	\tau^* & m \ge m^*
	\end{cases}
\end{align*}

The first equation becomes
\begin{align*}
	\mu(m) &= C_{\mu} e^{-\max(0,\tau^*(m-m^*))} \\
	\Leftrightarrow	\mu(m) &=  \begin{cases} 
	C_{\mu} & m < m^*  \\
	C_{\mu}e^{-\tau^*(m-m^*)} & m \ge m^*
	\end{cases}
\end{align*}

Plugging this into the second equation yields a closed-form expression for $C_{\mu}$: 
\begin{align*}
	C_{\mu}^{-1} &= \int_{0}^{m^*} 1 dm + \int_{m^*}^{\infty} e^{-\tau^*(m-m^*)} dm \\ 
				 &= m^* + \int_{0}^{\infty} e^{-\tau^* m'} dm' \\
				 &= m^* + \frac{1}{\tau^*}
\end{align*}

Putting this together, we have a closed form expression for $\mu(m)$:
\begin{align*}
	\mu(m) &= \Big(m^* + \frac{1}{\tau^*} \Big)e^{-\max(0,\tau^*(m-m^*))} \\
	\Leftrightarrow	\mu(m) &=  \begin{cases} 
	m^* + \frac{1}{\tau^*} & m < m^*  \\
	\Big(m^* + \frac{1}{\tau^*}\Big)e^{-\tau^*(m-m^*)} & m \ge m^*
	\end{cases}	
\end{align*}

Next, plugging $\mu(m)$ and the third equation into the fourth equation yields a closed form expression for $C_{\gamma}$ (involving $g$): 
\begin{align*}
	C_{\gamma}^{-1} &= \int_0^{\infty} e^{-gm} \mu(m) dm \\
				    &= \big(m^* + \frac{1}{\tau^*}\big)\Big(\int_{0}^{m^*} e^{-gm} dm + \int_{m^*}^{\infty} e^{-gm} e^{-\tau^*(m-m^*)}dm \Big) \\
				    &= \big(m^* + \frac{1}{\tau^*}\big)\Big( \frac{1}{g} (1-e^{-gm^*})  + e^{\tau m^*} (\frac{1}{g+\tau^*} )e^{-(g+\tau^*)m^*} \Big) 
\end{align*}

This yields a closed form expression for $\gamma(m)$, also involving $g$. Divide both sides of the fourth equation by $C_{\gamma}$ to obtain 
\begin{align*}
	1 = \lambda \times \frac{\int_0^{\infty} \mu(m) \tau(m) e^{-gm} dm}{\int_{0}^{\infty} \mu(m) \tau(m) dm}
\end{align*}

Plugging the expression for $\mu(m)$ into this equation yields an equation which implicitly defines $g$, given $m^*,\tau^*$. Provided there is a $g$ which solves the equation, call it $g_1(m^*,\tau^*)$. Similarly, plugging the expressions for $\mu(m),\gamma(m)$ into the fifth equation, 
\begin{align*}
	g = (\lambda -1) \int_{0}^{\infty} \tau(m) \gamma(m) \mu(m) dm
\end{align*}

yields a separate equation implicitly defining $g$ in terms of $m^*,\tau^*$. Provided there is a solution $g$, call it $g_2(m^*,\tau^*)$. The system has a solution if and only if (1) both equations have solutions, and (2) $g_1(m^*,\tau^*) = g_2(m^*,\tau^*)$. 

\paragraph{KF equation for joint distribution of $(q,m)$}

This approach would have me write out the KF equation for the evolution of the joint distribution of $(q,m)$ and see if it gives me any insight.

KF equations: for $m > 0$ have
\begin{align*}
	\partial_t \mu_t(\tilde{q},m) = -\partial_{\tilde{q}} \big[ a^{\tilde{q}} (\tilde{q},m) \mu_t(\tilde{q},m)  \big] -\partial_{m} \big[ a^{m} (\tilde{q},m) \mu_t(\tilde{q},m)  \big] - \tau(m)\mu_t(\tilde{q},m)
\end{align*}

As before, changing variables allows us to transform the problem into one in which $a^m(\tilde{q},m) \equiv 1$. Further, we have $a^{\tilde{q}} (\tilde{q},m) = - g\tilde{q}$ is independent of $m$. The KF equation becomes
\begin{align*}
	\partial_t \mu_t(\tilde{q},m) = -\partial_{\tilde{q}} \big[ -g\tilde{q} \mu_t(\tilde{q},m)  \big] -\partial_{m} \mu_t(\tilde{q},m) - \tau(m)\mu_t(\tilde{q},m)
\end{align*}

Next, multiply both sides by $\tilde{q}$ and integrate both sides over $(0,\infty)$ to get the following:
\begin{align*}
	\int_0^{\infty} \partial_t \mu_t(\tilde{q},m) d\tilde{q} &= g \int_0^{\infty} \tilde{q} \mu_t(\tilde{q},m) d\tilde{q} + g \int_{0}^{\infty} \tilde{q}^{2} \partial_{\tilde{q}}\mu_t(\tilde{q},m) d\tilde{q}  \\
	& - \int_{0}^{\infty} \partial_m \tilde{q}\mu_t(\tilde{q},m) d\tilde{q} - \tau(m)  \int_0^{\infty} \tilde{q}\mu_t(\tilde{q},m) d\tilde{q}
\end{align*}

Except for the second term on the RHS, all of the above are equal to some constant in $\tilde{q}$ (not necessarily in $m$) multiplied by $E_t[\tilde{q}|m]$ (after suitably interchanging the order of differentiation and integration). For the second term, we integrate by parts:
\begin{align*}
	\int_0^{\infty} \tilde{q}^2 \partial_{\tilde{q}} \mu_t (\tilde{q},m) d\tilde{q} &= \tilde{q}^2 \mu_t(\tilde{q},m) \big |^{\infty}_0 - \int_0^{\infty} 2 \tilde{q} \mu_t (\tilde{q},m) d\tilde{q}\\
			   &= 0 - \int_0^{\infty} 2 \tilde{q} \mu_t (\tilde{q},m) d\tilde{q}
\end{align*}

Plugging all of this into the RHS, get
\begin{align*}
	\frac{d}{dt} E_{\mu_t}[\tilde{q}|m] &= -g E_{\mu_t} [\tilde{q}|m] - \partial_m E_{\mu_t}[\tilde{q}|m] - \tau(m)E_{\mu_t}[\tilde{q},m]
\end{align*}

Setting the LHS equal to 0 yields a necessary condition for the conditional mean of $\tilde{q}$ given $m$ to be constant over time. Using notation $\gamma_t (m) = E_{\mu_t} [\tilde{q}|m]$ as before, have
\begin{align*}
	\gamma'(m) = -(g + \tau(m)) \gamma(m)
\end{align*}

This directly contradicts the condition I derived when working directly with the object $\gamma(m)$, 
\begin{align*}
	\gamma'(m) = -g \gamma(m) 
\end{align*}

What explains this discrepancy? One potential explanation for the discrepancy is that in the latter case I am truly  talking about the expected mean. Whereas in the former case I am not controlling for the fact that $\tau$ leads to the total mass shrinking over time. So I am pretty sure that they are actually the same condition if I made this change.

The more troubling issue for this approach is that it doesn't really give me anything new. It gives me a necessary condition analogous to the necessary condition I could already derive. The hard part - equation (\ref{gamma_consistency}) - I am not even discussing directly because that would require me to understand how to think about the KF equation when it comes to the injection site...

And this equation is correct - if start with $\mu_0(\tilde{q},m)$ satisfying the consistency condition, then for every $m > 0$ there is a time step $\Delta(m) > 0$ such that everything continues to hold at time $\Delta(m)$. 

In any case, this approach does not really work. The reason is that $\mu_t(\tilde{q},m)$ is in general not differentiable everywhere (see discussion above). The KF equation only holds at a steady state. 










\pagebreak









\paragraph{Differential equation for evolution of $\gamma_t(m)$ - deprecated}
First, we need to think about how $m$ moves. Essentially, there is some function $a(m)$ such that $m$ satisfies the differential equation 
\begin{align*}
\dot{m} = a(m)
\end{align*} 

For a given $m_0 > 0$ and for small enough $\Delta$ (i.e. depending on $m_0$), the differential equation above tells us $m^{\Delta}(m_0)$, i.e. what the state $m'$ would have been $\Delta$ time units ago. Not sure how to write this in closed form, but anyway. Have: 
\begin{align*}
\gamma_{t_0+\Delta} (m_0) &= e^{-g \Delta} \gamma_{t_0}(m^\Delta(m_0)) \\
\gamma_{t_0 + \Delta} (m_0) - \gamma_{t_0} (m_0) &= e^{-g \Delta} \gamma_{t_0}(m^{\Delta}(m_0)) - \gamma_{t_0}(m_0) \\
\frac{\gamma_{t+\Delta} - \gamma_{t_0}(m_0)}{\Delta} &= \frac{e^{-g\Delta} \gamma_{t_0}(m^{\Delta}(m_0)) - \gamma_{t_0}(m_0)}{\Delta}
\end{align*}

Letting $F(t,t_0,m_0) = e^{-g(t-t_0)}\gamma_{t_0}(m^{t-t_0}(m_0))$, we can see that the RHS is just 
\begin{align*}
\frac{F(t_0+\Delta,t_0,m_0) - F(t_0,t_0,m_0)}{\Delta}
\end{align*}

Finally, take the limit as $\Delta \to 0$ on both sides of the equation. This yields 
\begin{align*}
\partial_t \gamma_t(m) \Big|_{t = t_0,m=m_0} &= -g \gamma_t(m) - \partial_m \gamma_t(m) \frac{dm}{dt} \Big|_{t = t_0,m=m_0}
\end{align*}

Furthermore, we pin down $\gamma_t(0)$ by (is this right? ask John maybe??)
\begin{align*}
\gamma_t(0) = \lambda \times \frac{\int_{0}^{\infty} \gamma_t(m) \tau_t(m) \mu_t(m) dm}{\int_{0}^{\infty} \tau_t(m) \mu_t(m) dm}
\end{align*}

As a sanity check, notice that setting $\partial_t \gamma_t(m) = 0$ yields the necessary condition for a constant $\gamma$ derived above:
\begin{align*}
\gamma'(m) = -\frac{g}{a(m)} \gamma(m) 
\end{align*}

\end{document}