\documentclass[12pt,english]{article}
\usepackage{lmodern}
\usepackage[T1]{fontenc}
\usepackage[latin9]{inputenc}
\usepackage{geometry}
\usepackage{amsthm}
\usepackage{verbatim}
\geometry{verbose,tmargin=1in,bmargin=1in,lmargin=1in,rmargin=1in}
\usepackage{setspace}
%\usepackage{esint}
\onehalfspacing
\usepackage{babel}
\usepackage{amsmath}

\theoremstyle{remark}
\newtheorem*{remark}{Remark}
\begin{document}

\title{Computer Algorithm: permanent non-competes}
\author{Nicolas Fernandez-Arias}
\maketitle
	
	
\begin{enumerate}
	\item Guess $L^F$
	\begin{enumerate}
		\item Guess $g$
		\begin{enumerate}
			\item Guess $w(q,m)$, the wage paid to labor not bound by non-competes\footnote{Labor bound by non-competes will receive wage $\overline{w}$, which can be written in closed form as function of parameters}
			\begin{enumerate}
				\item Guess $M(q)$, which implies guess for $L^E(q,m) = \xi \min(m,M(q))$
				
				\item Guess $x(q,m)$, the non-compete policy of the incumbent\footnote{Entrants are infinitesimal relative to their industry $j$ hence not take into account their effect on industry $j$ aggregates; hence, they perceive no cost from leaking knowledge and endogenously do not require their employees to sign non-competes}.
				
				\item Using these guesses, solve for $V(q,m),W(q,m)$ using Moll's algorithm. For now, just set boundary conditions on initial (really, final) guess and hope it works, come back to this if it doesn't. Boundary on $V(q,m,T)$ comes from $V(0,m) = 0$ and $V_m(q,m) = 0$ for $m > M(q)$. Boundary on $W(q,m,T)$ comes, again, from $W(0,m) = 0$ and $W(q,m) = 0$ for $m > M(q)$.  
				
				\item Check that $x(q,m)$ is consistent with incumbent optimality: $x(q,m) = 1$ exactly when $|V_m(q,m)| < W(q,m)$ and zero $x(q,m) = 0$ otherwise. If not, return to (1aiB) and guess a new value $x(q,m)$.\footnote{I think this will converge, but it's not entirely obvious since there is a strategic interaction - the guess $x(q,m)$ affects $W(q,m)$ which then affects the optimal $x(q,m)$.} 
				
				\item Check that $M(q)$, and hence $L^E(q,m)$, is consistent with entrant optimality: that is, 
				\begin{align*}
				M(q) = \sup \big\{m: \chi_E \phi (L^I(q,m) +L^E(q,m) (V((1+\lambda)q,0) - W(q,m)) > w(q,m) \big\}
				\end{align*}
				If not, return to (1aiA) and guess a new value for $M(q)$.\footnote{Again, not 100\% sure if this will converge...}
			\end{enumerate}
			\item Check that $w(q,m) = \overline{w} - \nu W(q,m)$. If not, return to (1ai) and make a new guess.
		\end{enumerate} 
		\item Now, given $L^I(q,m),L^E(q,m)$ computed above, compute stationary distribution of $(q,m)$ and then aggregate to compute growth rate $g^*$. If not equal to guess $g$, return to (1a) and guess a new value of $g$.
	\end{enumerate}
	\item Finally, the above allocation gives us functions $L^I(L^F),L^{R\&D}(L^F)$. Check that $L^I + L^{R\&D} + L^F = 1$; otherwise, go back to (1) and guess a new value of $L^F$. 
\end{enumerate}




\end{document}