\documentclass[12pt,english]{article}
\usepackage{lmodern}
\usepackage[T1]{fontenc}
\usepackage[latin9]{inputenc}
\usepackage{geometry}
\usepackage{amsthm}
\usepackage{courier}
\usepackage{verbatim}
\geometry{verbose,tmargin=1in,bmargin=1in,lmargin=1in,rmargin=1in}
\usepackage{setspace}
%\usepackage{esint}
\onehalfspacing
\usepackage{babel}
\usepackage{amsmath}
\usepackage[open]{bookmark}
\usepackage[toc,page]{appendix}
\renewcommand{\baselinestretch}{1}



\theoremstyle{remark}
\newtheorem*{remark}{Remark}
\begin{document}

	
\title{Empirics for ``A Model of Endogenous Growth with Employee Spinouts''}
\author{Nicolas Fernandez-Arias}
\maketitle

\section{Introduction}

In this document I describe the preliminary tests of the "R\&D leads to spinouts" hypothesis. 

\begin{enumerate}
	\item OLS regression of state-level start-up rates on state-level R\&D expenditures
	\item IV regression of the above, using ``effective state-level cost of R\&D capital" as instrument (arguably exogenous based on Bloom et al "Identifying Knowledge Spillovers")
	\item OLS regression of industry-level start-up rates on industry-level R\&D spending
	\item IV regression of the above, using ``effective industry-level cost of R\&D capital" by looking at distribution of industry R\&D over states (using NBER patent data)
\end{enumerate}

Of these, I have so far implemented the first two. Below I describe the results.

\subsection{State-level regressions}











\end{document}