\documentclass[12pt,english]{article}
\usepackage{lmodern}
%\usepackage[T1]{fontenc}
\usepackage[latin9]{inputenc}
\usepackage{geometry}
\usepackage{amsthm}
\usepackage{verbatim}
\geometry{verbose,tmargin=1in,bmargin=1in,lmargin=1in,rmargin=1in}
\usepackage{setspace}
%\usepackage{esint}
\onehalfspacing
\usepackage{babel}
\usepackage{amsmath}

\theoremstyle{remark}
\newtheorem*{remark}{Remark}
\begin{document}
	
	\title{NBER USPTO-Compustat Aggregation: State-level R\&D Spending}
	\author{Nicolas Fernandez-Arias}
	\maketitle

\section{Introduction}

This document describes how to use NBER Patent Database and Compustat to construct location-time measures of R\&D spending by incumbent firms. Jupyter notebooks with the steps executed on the raw data can also be found in the main folder. 

Section \ref{section_data} describes the raw datasets used. Section \ref{section_algorithm_state} describes the construction of state-year level measures. Section \ref{section_algorithm_MSA} describes the construction of MSA-year level measures. 


\section{Data}\label{section_data}
Below I list the datasets and their key linking variables.

\begin{enumerate}
	\item \textbf{PATENTS:} \texttt{pat76\_06\_assg.dta}
	\begin{itemize}
		\item Patent \#
		\item appyear
		\item pdpass (unique assignee \#) 
	\end{itemize}
	\item \textbf{INVENTORS:} \texttt{inventor.dta}
	\begin{itemize}
		\item Patent \#
		\item Inventor
	\end{itemize}
	\item \textbf{ASSIGNEES:} \texttt{assignee.dta}
	\begin{itemize}
		\item pdpass
		\item Standard name
		\item I think this is unnecessary - pdpass is already in pat76\_06\_ass.dta
	\end{itemize}
	\item \textbf{DYNASS (dynamic match to Compustat firm id, gvkey):} \texttt{dynass.dta}
	\begin{itemize}
		\item pdpass
		\item pdpcoi, gvkeyi for all spells - we want the one corresponding to our year of interest
	\end{itemize}
	\item \textbf{Assocations of pdpco and gvkey:} \texttt{pdpcohdr.dta}
	\begin{itemize}
		\item pdpco
		\item gvkey
		\item Is this dataset necessary? I want to essentially look at all firms in Compustat that do R\&D and / or have received patents.
	\end{itemize}
	\item \textbf{COMPUSTAT:} \texttt{compustat\_annual\_13.dta}
	\begin{itemize}
		\item gvkey
		\item year
		\end{itemize}
	\item \textbf{CRUNCHBASE\_OBJECTS:} \texttt{crunchbase\_objects.dta}
	\begin{itemize}
		\item Firm name
	\end{itemize}
\end{enumerate}

\section{State-year R\&D aggregation}\label{section_algorithm_state}

The algorithm involves constructing a measure of the distribution of R\&D spending for each \texttt{gvkey,year} pair (e.g., in wide format, a variable for \texttt{weight<StateAbbrev>} for each state). This is merged by \texttt{gvkey,year} with \texttt{compustat\_annual\_13.dta}. The variable \texttt{xrd} is then split, according to \texttt{weight<StateAbbrev>}, into variables \texttt{xrd<StateAbbrev>}. Finally, these are aggregated by \texttt{StateAbbrev,year} into state-year level R\&D measures. 

\subsection{Geographic distribution of R\&D for each \texttt{gkvey,year}}\label{subsection_RDdistribution_gvkeyyear}

\subsection{Aggregation}\label{subsection_aggregation_state}

\section{MSA-year R\&D aggregation}\label{section_algorithm_MSA}

The steps are the same as in the state-level case except for the addition of one step where the dataset is merged with a dictionary relating cities to MSAs. 

\end{document}


