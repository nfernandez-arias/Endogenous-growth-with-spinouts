\documentclass[12pt,english]{article}
\usepackage{lmodern}
\usepackage[T1]{fontenc}
\usepackage[latin9]{inputenc}
\usepackage{geometry}
\usepackage{amsthm}
\usepackage{courier}
\usepackage{verbatim}
\geometry{verbose,tmargin=1in,bmargin=1in,lmargin=1in,rmargin=1in}
\usepackage{setspace}
%\usepackage{esint}
\onehalfspacing
\usepackage{babel}
\usepackage{amsmath}

\theoremstyle{remark}
\newtheorem*{remark}{Remark}
\begin{document}
	
\title{Algorithm 6-25-2018}
\author{Nicolas Fernandez-Arias}
\maketitle

\section{Outline}

\begin{enumerate}
	\item Iterate on $L_{RD}$. Guess $L^0_{RD}$.  
	\item Compute implied $\bar{w},\pi$.
	\item Iterate on wage $w^0(m)$. 
	\begin{enumerate}
		\item Guess $w^0(m) \equiv w^0$.
		\item Iterate on $z_E,z_S$ to find Nash equilibrium of R\&D race game.
		\begin{enumerate}
			\item Guess $z_E^0(m) \equiv z_E^0 > 0$
			\item Guess $z_S^0(m) \equiv z_S^0 > 0$ small.
			\item Calculate $V,z_I(m)$ by solving HJB (Moll's method, implicit)
			\item Compute
			\begin{align*}
			z_E^1(m) &= z_E^0(m) \cdot \frac{\chi_E \eta \big(z_E^0(m) + z_S^0(m)\big)V(0)}{w^0(m)} \\
			z_S^1(m) &= \min\Bigg\{\xi m,z_S^0(m) \cdot \frac{\chi_S \eta \big(z_E^0(m) + z_S^0(m)\big)V(0)}{w^0(m)}\Bigg\}
			\end{align*}
			\item Check convergence. If converged, go to (c).
			\item If not converged, have new guesses
			\begin{align*}
			z_E(m) &= \alpha_E z_E^1(m) + (1-\alpha_E) z_E^0(m) \\
			z_S(m) &= \alpha_S z_S^1(m) + (1-\alpha_S) z_S^0(m) 
			\end{align*}
			and also threshold $m$ such that
			\begin{align*}
				\forall m' > m, z_s(m) < \xi m
			\end{align*}
			and go back to (iii). 
		\end{enumerate} 
		\item Compute $W$ directly from HJB (Moll's method not required). Use boundary condition $W'(\bar{m}) = 0$ and assume R\&D intensity equal to $\xi$ (given NE from 3(b), individual spinouts either want to be at maximum scale or are indifferent)
		\item Compute $w^1(m) =  \bar{w} - \nu W(m)$.
		\item Check convergence. If converged, go to 4. 
		\item If not converged, update guess 
		\begin{align*}
			w(m) = \alpha_w w^1(m) + (1-\alpha_w) w^0 (m) 
		\end{align*} 
		and go back to 3(b).
	\end{enumerate}
	\item Compute $\mu$ by solving KF equation (in detail below) 
	\item Use $\mu$ and aggregate policy functions from 3(b) to compute R\&D labor demand $L^1_{RD}$.
	\item Check convergence. If converged, go to 4. 
	\item If not converged, new guess
	\begin{align*}
	L_{RD} &= \alpha_L L^1_{RD} + (1-\alpha_L) L^0_{RD} 
	\end{align*}
	and go back to 2. 
\end{enumerate}

\section{Details}

\subsection{HJB equations}

\subsubsection{Incumbent}

Current algorithm is inspired by Moll's HACT. Given flow profits $\pi$, non-R\&D wage $\bar{w}$, and R\&D wages $w(m)$, the algorithm proceeds as follows:

\begin{enumerate}
	\item Guess terminal value $V_0(m) = \pi / \rho$
	\begin{itemize}
		\item This is NPV of a perpetual monopoly on a given product line, assuming no investment in R\&D (and hence no innovation)
		\item Trying to follow spirit of HACT appendix, i.e. "value of staying put."
	\end{itemize}
	\item Semi-implicit method:
	\begin{enumerate}
		\item Roughly, solving time-dependent version of HJB backwards, given terminal condition, and hoping that in the infinite past it converges to the time-independent solution
		\item There are no state constraints in this problem - the only constraint is that $z_I \ge 0$, but this should be guaranteed by the fact that the marginal return of innovation effort goes to $+\infty$ as $z_I \to 0$ from above. 
		\item Discretize HJB. Since sign of drift is known (always positive), upwind scheme is easy: always use same finite difference approximation to derivative terms
		\item Given: $V_n(m)$ guess, calculate optimal incumbent R\&D policy $z_I^{*,n}(m)$ using $V_n(m)$ (this is why method is only semi-implicit, following HACT notes)
		\item Perform implicit update step. Essentially, the 
	\end{enumerate}
\end{enumerate}

\subsubsection{Potential Spinout}

\subsection{KF equation}

Solving the KF equation is easy. A given line $j$ starts in the state $m = 0$. Subsequently $m$ grows deterministically until a shock occurs, resetting $m$ to zero. Hence, there is no point mass at 0 individually, and there will be no point mass at 0 in the cross section. 

Let $\tau(m)$ denote the aggregate rate of arrival of innovations. Let $a(m)$ denote the rate of drift. Both of these depend on R\&D by the incumbent, spinouts and other entrants. The KF equation for $m > 0$ is 
\begin{align*}
	0 &= -a'(m)\mu(m) - a(m)\mu'(m) - \tau(m)\mu(m) \\
	\mu'(m) &= -\frac{a'(m) + \tau(m)}{a(m)} \mu(m)
\end{align*}
which has solution 
\begin{align*}
	\mu(m) = C\exp \Bigg\{ - \int_0^m \frac{a'(m') + \tau(m')}{a(m')} dm'    \Bigg\}
\end{align*}
with $C$ chosen so that $\int_0^{\infty} = 1$, which we will approximate by numerically integrating until $\bar{m}$. If chosen large enough, this should be fine, since in equilibrium the density decays exponentially. 

Furthermore, in equilibrium there is a bijection between $m$, the current mass of spinouts, and $s$, the amount of time that has elapsed since the last innovation on that product line. By construction, particles drift at constant rate 1 in $s$-space. And $\hat{\tau}(s) = \tau(m(s))$. Hence, the density $\hat{\mu(s)}$ is determined by the equations below:
\begin{align*}
	\hat{\mu}(s) &= Ce^{-\int_0^s \hat{\tau}(s')ds'} \\
	\int_0^{\infty} \hat{\mu}(s') ds' &= 1
\end{align*}









\end{document}