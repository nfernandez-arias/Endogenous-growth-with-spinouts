\documentclass[12pt,english]{article}
\usepackage{lmodern}
\usepackage[T1]{fontenc}
\usepackage[latin9]{inputenc}
\usepackage{geometry}
\usepackage{amsthm}
\usepackage{verbatim}
\geometry{verbose,tmargin=1in,bmargin=1in,lmargin=1in,rmargin=1in}
\usepackage{setspace}
%\usepackage{esint}
\onehalfspacing
\usepackage{babel}
\usepackage{amsmath}

\theoremstyle{remark}
\newtheorem*{remark}{Remark}
\begin{document}

\title{Efficiency effects of creative destruction in my model}
\author{Nicolas Fernandez-Arias}
\maketitle

If the model had no creative destruction, it would probably be essentially efficient (except for monopoly distortion of production of course). 

In my model I add creative destruction. However, it is not the business stealing itself that is causing the inefficiency. This is compensated for. Compared to the model with no creative destruction (and entrant technology the same), the incumbent certainly receives net more compensation, since he is not being harmed eventually by business stealing. Hence he has a stronger incentive to innovate and innovates more in equilibrium. But this is not an \textit{efficiency} improvement; in the counterfactual economy where all spinouts create new products, the social planner wants to dedicate more resources to R\&D because it has been assumed to be more productive (since it generates fundamentally new spillover ideas rather than incremental improvements).

Rather, it is the fact that creative destruction means more of a monopoly on R\&D is lost by educating spinouts. This is also the mechanism in standard models. The question is then: does the fact that firms can now restrict these spinouts by doing less R\&D help things? 

Roughly speaking, let's think of it this way. From the planner's perspective there is an optimal amount of R\&D by each agent in the economy, given his R\&D productivity etc. Take as given that spinouts are somehow more productive than incumbents so that the social planner wants some in equilibrium. When we decentralize and allow everyone to do everything they want, the entrants have a force guiding them to do too much R\&D because they don't realize they're stealing business from incumbents, and the incumbents (and everyone else) do too little R\&D because they don't internalize the fact that they are creating knowledge spillovers. The net effect of these is that there can be varying degrees of efficiency. 

Relative to this benchmark, my model is simply a different form of knowledge spillovers. Instead of spilling over the instant of discovery, they spill over as R\&D is performed. The effect is the same: firms have a disincentive to perform R\&D. 















\end{document}