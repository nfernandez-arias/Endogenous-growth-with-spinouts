\documentclass[12pt,english]{article}
\usepackage{lmodern}
\usepackage[T1]{fontenc}
\usepackage[latin9]{inputenc}
\usepackage{geometry}
\usepackage{amsthm}
\usepackage{courier}
\usepackage{verbatim}
\geometry{verbose,tmargin=1in,bmargin=1in,lmargin=1in,rmargin=1in}
\usepackage{setspace}
%\usepackage{esint}
\onehalfspacing
\usepackage{babel}
\usepackage{amsmath}

\theoremstyle{remark}
\newtheorem*{remark}{Remark}
\begin{document}
	
\title{Numerical Algorithm for computing BGP in ``Endogenous Growth with Creative Destruction and Employee Spinouts''}
\author{Nicolas Fernandez-Arias}
\maketitle

\section{Outline}

\begin{enumerate}
	\item Iterate on $L_{RD}$. Guess $L^0_{RD}$.  
	\item Compute implied $\bar{w},\pi$.
	\item Iterate on wage $w^0(m)$. 
	\begin{enumerate}
		\item Guess $w^0(m) \equiv w^0$.
		\item Iterate on $z_E,z_S$ to find Nash equilibrium of R\&D race game.
		\begin{enumerate}
			\item Guess $z_E^0(m) \equiv z_E^0 > 0$
			\item Guess $z_S^0(m) \equiv z_S^0 > 0$ small.
			\item Calculate $V,z_I(m)$ by solving HJB (Moll's method, implicit)
			\item Compute
			\begin{align*}
			z_E^1(m) &= z_E^0(m) \cdot \frac{\chi_E \eta \big(z_E^0(m) + z_S^0(m)\big)V(0)}{w^0(m)} \\
			z_S^1(m) &= \min\Bigg\{\xi m,z_S^0(m) \cdot \frac{\chi_S \eta \big(z_E^0(m) + z_S^0(m)\big)V(0)}{w^0(m)}\Bigg\}
			\end{align*}
			\item Check convergence. If converged, go to (c).
			\item If not converged, have new guesses
			\begin{align*}
			z_E(m) &= \alpha_E z_E^1(m) + (1-\alpha_E) z_E^0(m) \\
			z_S(m) &= \alpha_S z_S^1(m) + (1-\alpha_S) z_S^0(m) 
			\end{align*}
			and also threshold $m$ such that
			\begin{align*}
				\forall m' > m, z_s(m) < \xi m
			\end{align*}
			and go back to (iii). 
		\end{enumerate} 
		\item Compute $W$ directly from HJB (Moll's method not required). Use boundary condition $W'(\bar{m}) = 0$ and assume R\&D intensity equal to $\xi$ (given NE from 3(b), individual spinouts either want to be at maximum scale or are indifferent)
		\item Compute $w^1(m) =  \bar{w} - \nu W(m)$.
		\item Check convergence. If converged, go to 4. 
		\item If not converged, update guess 
		\begin{align*}
			w(m) = \alpha_w w^1(m) + (1-\alpha_w) w^0 (m) 
		\end{align*} 
		and go back to 3(b).
	\end{enumerate}
	\item Compute $\mu$ by solving KF equation (in detail below) 
	\item Use $\mu$ and aggregate policy functions from 3(b) to compute R\&D labor demand $L^1_{RD}$.
	\item Check convergence. If converged, go to 4. 
	\item If not converged, new guess
	\begin{align*}
	L_{RD} &= \alpha_L L^1_{RD} + (1-\alpha_L) L^0_{RD} 
	\end{align*}
	and go back to 2. 
\end{enumerate}

\section{Details}

\subsection{HJB equations}

\subsubsection{Incumbent}

\paragraph{Equation to be solved}The HJB to be solved is: 

\begin{align*}
	(\rho + \tau^{SE}(m)) V(m) &= \pi + \nu (a^{SE}(m))V'(m) \\
							   &+ \max_{z_I \ge 0} \Big\{ z_I \chi_I \phi(z_I) \big[ \lambda V(0) - V(m) \big] - z_I w(m) + z_I \nu  V'(m) \Big\}
\end{align*}
where
\begin{align*}
	\tau^{SE}(m) &\equiv (\chi_S z_S(m) + \chi_E z_E(m)) \hat{\phi}(a_{SE}(m)) \\
	a^{SE}(m) &\equiv z_S(m) + z_E(m) \\
	\phi(m) &= m^{-p} \\
	p &\in (-1,0) \\
	\nu &> 0
\end{align*}

where $\hat{\phi}$ is the innovation production function for entrants and spinouts. \footnote{Some assumptions are being imposed here, they could be changed but that might require changing my algorithm.} 

\paragraph{Algorithm} Given flow profits $\pi$, non-R\&D wage $\bar{w}$, R\&D wages $w(m)$, etc., the algorithm proceeds as follows:

\begin{enumerate}
	\item Guess terminal value $V_0(m) = \pi / \rho$
	\begin{itemize}
		\item This is NPV of a perpetual monopoly on a given product line, assuming no investment in R\&D (and hence no innovation)
		\item Another reasonable guess would be $V_0(m) = \pi / (\rho - g)$ for some growth rate $g$
	\end{itemize}
	\item Implement semi-implicit method:\footnote{Roughly, solving time-dependent version of HJB backwards, given terminal condition, and hoping that in the infinite past it converges to the time-independent solution.}
	\begin{enumerate}
		\item \textbf{Discretize HJB} Since the direction of a product line's drift in $m$-space does not depend on the policy function of the incumbent, upwind scheme is easy to implement. Discretization results in
		\begin{align*}
			\frac{V_i^{n+1} - V_i^{n}}{\Delta_t} + (\rho + \tau^{SE}_i)V_i^{n+1} &= \pi + \nu a_I^{SE} \frac{V_{i+1}^{n+1} - V_i^{n+1}}{\Delta_m} \\
			          &+ z^{I,n}_i \Big( \chi_I \phi(z^{I,n}_i) \big[ \lambda V^{n+1}_1 - V^{n+1}_i \big] - w_i + \nu \frac{V_{i+1}^{n+1} - V_i^{n+1}}{\Delta_m} \Big)
		\end{align*}
		\item \textbf{State constraints} There are no state constraints in this problem - the only constraint is that $z_I \ge 0$, which I can insure by hard-coding the constraint at the moment of optimization.\footnote{I am not using a closed form FOC for optimization, so I don't need to worry about cases in which the FOC doesn't hold, which is the only reason state constraints might not be trivial to handle}
		\item \textbf{Calculating optimal policy $z_I^{*,n}(m)$:} Given result of previous iteration $V_n(m)$ (or initial guess if $n = 0$), calculate optimal incumbent R\&D policy $z_I^{*,n}(m)$ by maximizing RHS side of HJB in (a), with $n$ instead of $n+1$ superscripts:
		\begin{align*}
			z_I^{*,n} (m) &= \arg \max_{z \ge 0}  z \Big( \chi_I \phi(z) \big[ \lambda V^n_1 - V^n_i \big] - w_i + \frac{V_{i+1}^n - V_i^n}{\Delta_m} \Big)
		\end{align*} 
		\item \textbf{Constructing matrix $A^n$ for update step} This step is just a matter of carefully translating the above into matrix notation. For $i < I_{\max}$, 
		\begin{align*}
			A_{i,1}^n &= z^{I,n}_i \chi_I \phi(z^{I,n}_i) \lambda \\
			A_{i,i+1}^n &= \nu \frac{a^{SE}_i + z^{I,n}_i}{\Delta_m} \\
			A_{i,i}^n &= -\nu \frac{a^{SE}_i + z^{I,n}_i}{\Delta_m} - z^{I,n}_i \chi_I \phi(z^{I,n}_i) - \tau^{SE}_i 
		\end{align*} 
		At $i = I_{\max}$, I am not sure what to do. So far I have set essentially set the right-derivative of $V$ to be zero at $I_{\max}$.
		\begin{align*}
			A_{I_{\max},1} &= z^{I,n}_{I_{\max}} \chi_I \phi(z^{I,n}_{I_{\max}}) \lambda \\
			A_{I_{\max},I_{\max}} &= - z^{I,n}_{I_{\max}} \chi_I \phi(z^{I,n}_{I_{\max}}) - a^{SE}_{I_{\max}}
		\end{align*}
		\item \textbf{Final equation} Following the HACT numerical appendix, the system can be written 
		\begin{align*}
			B^n V^{n+1} &= b^n \\
			B^n &\equiv \Big( \frac{1}{\Delta_t} + \rho \Big) I - A^n \\
			b^n &\equiv u^n + \frac{1}{\Delta_t} V^{n} \\
			u^n_i &\equiv \pi - z^{I,n}_i w_i 
		\end{align*}
		which I solve in Matlab.
	\end{enumerate}
\end{enumerate}

\paragraph{Troubleshooting}

Given a guess that $w(m)$ is constant and some generic weakly increasing guesses for $\tau^{SE}(m), a^{SE}(m)$, which is what I expect in equilibrium, I get the strange result that the algorithm seems to converge but $V$ is not decreasing. Intuitively, the equilibrium $V$ should really be strictly decreasing; and in partial equilibrium, constant $w(m)$ should ensure this too. 

Two questions:

\begin{enumerate}
	\item Could this be arising from what I do at the boundary? 
	\item Could this be because the implicit method doesn't work in this case? I've tried changing it essentially by replacing the superscript on $V(0)$ in the HJB update step from $n+1$ to $n$, but this seems to introduce new problems of its own. 
\end{enumerate}

There also seems to be another issue near the edge of the state space in my current version of the algorithm, 


 


\subsubsection{Potential Spinout}

\subsection{KF equation}

Solving the KF equation is easy. A given line $j$ starts in the state $m = 0$. Subsequently $m$ grows deterministically until a shock occurs, resetting $m$ to zero. Hence, there is no point mass at 0 individually, and there will be no point mass at 0 in the cross section. 

Let $\tau(m)$ denote the aggregate rate of arrival of innovations. Let $a(m)$ denote the rate of drift. Both of these depend on R\&D by the incumbent, spinouts and other entrants. The KF equation for $m > 0$ is 
\begin{align*}
	0 &= -a'(m)\mu(m) - a(m)\mu'(m) - \tau(m)\mu(m) \\
	\mu'(m) &= -\frac{a'(m) + \tau(m)}{a(m)} \mu(m)
\end{align*}
which has solution 
\begin{align*}
	\mu(m) = C\exp \Bigg\{ - \int_0^m \frac{a'(m') + \tau(m')}{a(m')} dm'    \Bigg\}
\end{align*}
with $C$ chosen so that $\int_0^{\infty} = 1$, which we will approximate by numerically integrating until $\bar{m}$. If chosen large enough, this should be fine, since in equilibrium the density decays exponentially. 

Furthermore, in equilibrium there is a bijection between $m$, the current mass of spinouts, and $s$, the amount of time that has elapsed since the last innovation on that product line. By construction, particles drift at constant rate 1 in $s$-space. And $\hat{\tau}(s) = \tau(m(s))$. Hence, the density $\hat{\mu(s)}$ is determined by the equations below:
\begin{align*}
	\hat{\mu}(s) &= Ce^{-\int_0^s \hat{\tau}(s')ds'} \\
	\int_0^{\infty} \hat{\mu}(s') ds' &= 1
\end{align*}

\subsection{Aggregation}

(Section not written yet) I have proved elsewhere that the steps below ``work''. 






\end{document}