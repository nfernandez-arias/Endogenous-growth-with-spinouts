\documentclass[12pt,english]{article}
\usepackage{lmodern}
\usepackage[T1]{fontenc}
\usepackage[latin9]{inputenc}
\usepackage{geometry}
\usepackage{amsthm}
\usepackage{courier}
\usepackage{verbatim}
\geometry{verbose,tmargin=1in,bmargin=1in,lmargin=1in,rmargin=1in}
\usepackage{setspace}
%\usepackage{esint}
\onehalfspacing
\usepackage{babel}
\usepackage{amsmath}

\theoremstyle{remark}
\newtheorem*{remark}{Remark}
\begin{document}
	
\title{Equations and variables of the model: is there a BGP}
\author{Nicolas Fernandez-Arias}
\maketitle

If the model I am interested in with total labor supply $L = 1$ admits a constant $\gamma_t(m)$, defined below, then so should the model with any other labor supply $L'$. Hence, I take as given $L^F$, compute the implied $\bar{w}$, and attempt to find a BGP. 

The \textbf{exogenous variables }are:
\begin{enumerate}
	\item The final goods wage $\bar{w}$
	\item The rate at which flow profits scale with the quality of the good, $\pi$ (i.e. flow profits are $\pi q $)
\end{enumerate}

The \textbf{endogenous variables }are \big(for $m \ge 0$ \big):
\begin{enumerate}
	\item Incumbent value function $V(m)$
	\item Incumbent policy $z_I(m)$
	\item Spinout value function $W(m)$
	\item Aggregate spinout R\&D effort on a good in state $m$, $z_S(m)$
	\item Aggregate non-spinout entrant R\&D effort on a good in state $m$, $z_E(m)$
	\item Arrival rate due to entrants and spinouts $\tau_{SE}(m)$
	\item Total arrival rate of innovations $\tau(m)$
	\item Drift rate $a(m)$
	\item Wage $w(m)$
	\item Stationary distribution $\mu(m)$ 
	\item Stationary $\gamma(m) = E[\tilde{q}|m]$
	\item Growth rate $g$
\end{enumerate}

The\textbf{ system of equations} satisfied in a recursive equilibrium is:\footnote{Technically, I need to add some transversality conditions to the HJBs. Perhaps I can just add boundary conditions like $\lim_{m\to \infty} V'(m) = \lim_{m \to \infty} W(m) = 0$. Is this a guess of form of eq.? Is this okay? Try it, try w/o it, ask Ezra.}
\begin{align}
(\rho + \tau_{SE}(m))V(m) &=  \pi + a(m) V'(m) \label{HJB_V}\\
						  &+ z_I(m)\chi_I \phi(z_I(m)) (\lambda V(0) - V(m)) \nonumber \\
						  &- z_I(m)w(m) \nonumber \\ 
\lim_{m \to \infty} V'(m) &= \label{HJB_V_Boundary} 0 \\
(\rho + \tau(m)) W(m) &= a(m) W'(m) \label{HJB_W}\\
						&+ z_S(m)\chi_S\eta(z_S(m) + z_E(m))\lambda V(0) \nonumber \\
						&- z_S(m)w(m)  \nonumber	\\	
\lim_{m \to \infty} W(m) &= 0 \label{HJB_W_Boundary} \\ 										  
w(m) - \nu V'(m) &= z_I(m)\chi_I\phi'(z_I(m)) + \chi_I \phi(z_I(m)) (\lambda V(0) - V(m)) \label{HJB_V_FOC}\\
w(m) &= \chi_E\eta(z_S(m) + z_E(m)) \lambda V(0) \label{E_FreeEntry}\\ 
\Big(\forall m: z_S(m) = \xi m \Big) w(m) &\le \chi_S \eta(z_S(m) + z_E(m)) \lambda V(0) \label{HJB_W_Optimality1}\\
\Big(\forall m: z_S(m) < \xi m \Big) w(m) &= \chi_S \eta(z_S(m) + z_E(m)) \lambda V(0) \label{HJB_W_Optimality2}\\
\bar{w} &= w(m) + \nu W(m) \label{Indifference}\\
0 &= - a'(m)\mu(m) - a(m)\mu'(m) - \tau(m)\mu(m) \label{KF}\\
1 &= \int_{0}^{\infty} \mu(m) dm \label{KF_integral}\\
g &= (\lambda - 1) \int_{0}^{\infty} \tau(m) \mu(m) dm \label{growth_aggregation} \\
\gamma'(m) &= \frac{-g }{a(m)} \gamma(m) \label{gamma_firstorder}\\
1 &= \int_{0}^{\infty} \mu(m) \gamma(m) dm \label{gamma_integral}\\ 
\gamma(0) &= \lambda \times \frac{\int_{0}^{\infty} \mu(m) \tau(m) \gamma(m)  dm}{\int_{0}^{\infty} \mu(m) \tau(m) dm} \label{gamma_consistency}\\
a(m) &= \big( z_I(m) + z_S(m) + z_E(m) \big) \nu \label{a_def}\\
\tau_{SE}(m) &= \big(z_S(m) \chi_S + z_E(m) \chi_E \big) \eta \big(z_S(m) + z_E(m)\big) \label{tau_SE_def}\\
\tau(m) &= z_I(m) \chi_I \phi(z_I(m)) + \tau_{SE}(m)  \label{tau_def}
\end{align}

First, equations (\ref{a_def}),(\ref{tau_SE_def}), and (\ref{tau_def}) are just definitions and hence take care of one equation each. 

The point is this: given $w(m)$, the first few equations pin down the value functions $V(m),W(m)$, policies $z_I(m),z_E(m),z_S(m)$, arrival rates $\tau_{SE}(m),\tau(m)$ and drift $a(m)$. Through the KF equation (first order and integral), this pins down $\mu(m)$. Then the growth aggregation equation pins down $g$. Then, the first-order and integral equations pin down $\gamma(m)$. Finally, (\ref{Indifference}) pins down $w(m)$. We are left with an extra equation, (\ref{gamma_consistency}). 

The only hope is that, given the solutions to the other equations, this equation, which looks to be non-trivial, becomes redundant. Not sure how one would prove this. All that is left is to (1) try to solve it on a computer, or (2) think about the economics of the situation. Why should there not be a constant expected relative quality given $m$? 

If we start with $\gamma_0(m)$ that satisfies (\ref{gamma_firstorder}) and (\ref{gamma_integral}), how does it evolve? First, assume for simplicity that growth is constant and equal to $g$. At all times we will have 
\begin{align*}
	\gamma_t(0) = \lambda \times \frac{\int_{0}^{\infty} \gamma_t(m) \tau_t(m) \mu_t(m) dm}{\int_{0}^{\infty} \tau_t(m) \mu_t(m) dm}
\end{align*}

The evolution is as follows. At each $t \ge 0$, there is a boundary, $b(t) \in \mathbf{R}$, determined by $a(m)$. To the right of $b(t)$, we will have $\gamma_t(m) = \gamma_0(m)$. To the left of $b(t)$, will have $\gamma_t(m) = e^{-g(t-t')} \gamma_{t'} (m')$, where $t' < t$ is the time when 

The boundary is determined by the evolution of the frontier. The movement of the frontier is easy to calculate because at any time the rate of drift in the $m$-direction of a given good $j$ depends only on its current state $m$. Hence, goods cannot ``pass'' each other in $m$-space, and we only need to track the movement through $m$-space of the goods which 

It will be very difficult to show anything theoretically. Everything is endogenous and simultaneous, and I do not have closed forms. The distribution $\mu(m)$ is determined by $a(m)$, which is determined by the wage $w(m)$. The conditional expectation $\gamma(m)$ is determined by $\mu(m)$ and $g/a(m)$. The growth rate $g$ in turn is determined by the integral of $\mu(m)\tau(m)$. Finally $\tau(m)$ is determined by the Nash Equilibrium given 

\paragraph{Derivation of $\gamma(m)$ given $a(m),\tau(m)$}
The differential equation for $\mu(m)$ has solutions of the form
\begin{align*}
	\mu(m) &= C_{\mu}\exp\Bigg\{ - \int_0^m \frac{a'(m') + \tau(m')}{a(m')} dm'\Bigg\}
\end{align*}

The differential equation for $\gamma(m)$ has solutions of the form
\begin{align*}
	\gamma(m) &= C_{\gamma} \exp\Bigg\{ -g \int_0^m a(m')^{-1} dm' \Bigg\}
\end{align*}

Given these functional forms, let's check whether any of the equations determining $\gamma$ or $\mu$ are redundant. Equation (\ref{gamma_integral}) - the Law of Iterated Expectations - becomes
\begin{align*}
	1 &= \int_{0}^{\infty} \mu(m) \gamma(m) dm \\
	  &= C_{\mu} C_{\gamma} \int_0^{\infty} \exp\Big\{ - \int_0^m \frac{a'(m') + \tau(m')}{a(m')} dm' \Big\} \exp\Big\{ -g \int_{0}^{m} a(m')^{-1}  \Big\} dm \\
	  &= C_{\mu} C_{\gamma} \int_0^{\infty} \exp\Big\{ - \int_0^m \frac{g + a'(m') + \tau(m')}{a(m')} dm' \Big\} dm
\end{align*}

Doesn't look like anything can be done. Ok, how about (\ref{gamma_consistency})? Substituting in the above functional form, we see that $C_{\gamma} = \gamma(0)$ hence (\ref{gamma_consistency}) becomes
\begin{align*}
	C_{\gamma} &=  \lambda \times \frac{\int_{0}^{\infty} \gamma(m) \tau(m) \mu(m) dm}{\int_{0}^{\infty}\tau(m) \mu(m) dm}
\end{align*}

\paragraph{Simple case of constant drift / innovation arrival ($a(m) \equiv a, \tau(m) \equiv \tau$)}

What happens if we set $\chi_E \le \chi_S$? Then, in actuality, the equilibrium of the model has constant $a(m) \equiv a$ and $\tau(m) \equiv \tau$, since spinouts are redundant. We also have in this case $w(m) \equiv \bar{w}$ in equilibrium. Moreover, we may even be able to get closed forms for everything. What happens with $\gamma_t(m)$ in this case? Is there a $\gamma_0(m)$ which would remain constant if we ran the model forward? If not, why not?

In this case, the necessary conditions for a constant $\gamma_t(m) = \gamma(m)$ are 
\begin{align*}
	\mu(m) &= (\tau / a) e^{-(\tau/a)m} \\
	\gamma(m) &= \lambda e^{-(g/a)m} \\
	1 &= \int_0^{\infty} \gamma(m) \mu(m) dm
\end{align*}

Substituting the first two into the last yields
\begin{align*}
	1 &= (\lambda \tau / a)  \int_0^{\infty} e^{-(g/a)m} e^{-(\tau/a)m} dm \\ 
	  &= (\lambda \tau / a)  \int_0^{\infty} e^{-((g + \tau)/a)m} dm
\end{align*}

This imposes a relationship between $\lambda$,$\tau$, and $a$.

We also have (\ref{growth_aggregation}), using the fact that $\tau(m)$ is constant, 
\begin{align*}
	g &= (\lambda-1) \tau 
\end{align*}

Substituting this in we get 
\begin{align*}
	1 &= (\lambda \tau / a) \int_0^{\infty} e^{\lambda \tau / a} m dm \\
	  &= 1
\end{align*}

so that there exists a BGP. So in this case, it works!

\paragraph{Idea for how to do General Case}
The basic idea is that we can make a change of variables $m \to m'$ so that the drift in the $m'$ direction is constant. Hence, we can assume without loss of generality that $a(m) = 1$. Although, we do have to slightly modify the equations - essentially now we replace (\ref{a_def}) with $a(m) = 1$, because $1$ is what goes into the KF equation, etc. 

This implies that
\begin{align*}
	\mu(m) &= C_{\mu} \exp \Big\{  -\int_{0}^{m} \tau(m') dm'\Big\} \\
	1 &= \int_{0}^{\infty} \mu(m) dm \\
	\gamma(m) &= C_{\gamma} e^{-gm} \\
	1 &= \int_{0}^{\infty}\gamma(m) \mu(m) dm \\ 
	C_{\gamma} &= \lambda \times \frac{\int_{0}^{\infty} \mu(m) \tau(m) \gamma(m) dm}{\int_{0}^{\infty} \mu(m) \tau(m)dm} \\
	g &= (\lambda -1) \int_{0}^{\infty} \tau(m) \gamma(m) \mu(m) dm
\end{align*}

Substitute out the functional equations into the integrals to arrive at 4 equations in 3 1-dimensional unknowns:
\begin{align*}
	C_{\mu}^{-1} &= \int_{0}^{\infty} \exp \Big\{ \tau(m') dm'\Big\} dm \\
	C_{\gamma}^{-1} &= \int_{0}^{\infty} e^{-gm}  \exp \Big\{ \tau(m') dm'\Big\} dm \\
	C_{\gamma} &= \lambda \times  \frac{ C_{\gamma} C_{\mu} \int_{0}^{\infty} \exp \Big\{  -\int_{0}^{m} \tau(m') dm'\Big\}  \tau(m) e^{-gm} dm}{C_{\mu} \int_{0}^{\infty} \exp \Big\{  -\int_{0}^{m} \tau(m') dm'\Big\}  \tau(m) dm} \\
	g &= (\lambda -1) C_{\gamma} C_{\mu} \int_{0}^{\infty} \tau(m) e^{-gm} \exp \Big\{  -\int_{0}^{m} \tau(m') dm'\Big\}   dm
\end{align*}

We can further simplify a bit: 
\begin{align*}
C_{\mu}^{-1} &= \int_{0}^{\infty} \exp \Big\{ \tau(m') dm'\Big\} dm \\
C_{\gamma}^{-1} &= \int_{0}^{\infty} e^{-gm}  \exp \Big\{ \tau(m') dm'\Big\} dm \\
1 &= \lambda \times  \frac{ \int_{0}^{\infty} \exp \Big\{  -\int_{0}^{m} \tau(m') dm'\Big\}  \tau(m) e^{-gm} dm}{\int_{0}^{\infty} \exp \Big\{  -\int_{0}^{m} \tau(m') dm'\Big\}  \tau(m) dm} \\
g &= (\lambda -1) C_{\gamma} C_{\mu} \int_{0}^{\infty} \tau(m) e^{-gm} \exp \Big\{  -\int_{0}^{m} \tau(m') dm'\Big\}   dm
\end{align*}

Given $\tau(m)$, the first equation yields $C_{\mu}$. The third equation yields $g$. The second equation then yields $C_{\gamma}$. Hence, the fourth equation cannot impose any additional restrictions - it must be redundant in order for this system to have a solution. Note that we know this even though there is nothing linear about this system, since I can prove that given $\tau(m)$ there is one choice of $C_{\mu}, g, C_{\gamma}$ consistent with the first three equations. 


No idea how to show that this has a solution in the case of non-constant $\tau$. 

Perhaps one way to get intuition would be to consider a discrete-state, discrete-time case and see if I can get closed forms. Intuitively, when $\tau(m_i)$ is increased, agents are more likely to die at that point, hence $\mu(m_j)$ is decreased for all $j > i$. So that $\mu$ continues to integrate to 1, $\mu(m_j)$ has to increase for $j \le i$ in the same proportion. Also, the growth rate has decreased somewhat, etc. It gets complicated.

The other angle of attack is to think more intuitively about why I should expect things to work out in this model. What is the economics? The average quality of the new lines has to be some weighted average quality of the lines over $m$, weighted by the fraction of innovating lines coming from each state. Essentially, innovations come from frontier firms and from laggards, and this affects the average quality of ``new'' products. But intuitively, what would stop this process from working out? Products are born at $m = 0$ with some average $q$, and drift down in $q$-space at rate $g$ as they move through $m$ (having normalized $a = 1$), and then are brought back to $m = 0$ with a jump of size $\lambda$. It is precisely this process that creates the decay at rate $g$, but crucially, the fact that this process creates the decay rate $g$ does not depend at all on this. It depends solely on the rate of arrivals. So it functions, chugging along regardless of anything, generating the rate $g$, and then the shape of $\gamma$ just is what it is.

Maybe I should think about the evolution of the distribution of $(q,m)$. Now, we have a function $\gamma_t(m)$, and in order to have a BGP, require that $\gamma_t(m)$ grow at rate $g$ over time. If we suppose that in the next time step, there is a BGP, then there will be constant growth rate. This growth has to come from the lines that enter at $m=0$ in the next time step. The rest of the distribution *will* evolve as necessary, given how it already was. 


















\pagebreak









\paragraph{Differential equation for evolution of $\gamma_t(m)$ - deprecated}
First, we need to think about how $m$ moves. Essentially, there is some function $a(m)$ such that $m$ satisfies the differential equation 
\begin{align*}
\dot{m} = a(m)
\end{align*} 

For a given $m_0 > 0$ and for small enough $\Delta$ (i.e. depending on $m_0$), the differential equation above tells us $m^{\Delta}(m_0)$, i.e. what the state $m'$ would have been $\Delta$ time units ago. Not sure how to write this in closed form, but anyway. Have: 
\begin{align*}
\gamma_{t_0+\Delta} (m_0) &= e^{-g \Delta} \gamma_{t_0}(m^\Delta(m_0)) \\
\gamma_{t_0 + \Delta} (m_0) - \gamma_{t_0} (m_0) &= e^{-g \Delta} \gamma_{t_0}(m^{\Delta}(m_0)) - \gamma_{t_0}(m_0) \\
\frac{\gamma_{t+\Delta} - \gamma_{t_0}(m_0)}{\Delta} &= \frac{e^{-g\Delta} \gamma_{t_0}(m^{\Delta}(m_0)) - \gamma_{t_0}(m_0)}{\Delta}
\end{align*}

Letting $F(t,t_0,m_0) = e^{-g(t-t_0)}\gamma_{t_0}(m^{t-t_0}(m_0))$, we can see that the RHS is just 
\begin{align*}
\frac{F(t_0+\Delta,t_0,m_0) - F(t_0,t_0,m_0)}{\Delta}
\end{align*}

Finally, take the limit as $\Delta \to 0$ on both sides of the equation. This yields 
\begin{align*}
\partial_t \gamma_t(m) \Big|_{t = t_0,m=m_0} &= -g \gamma_t(m) - \partial_m \gamma_t(m) \frac{dm}{dt} \Big|_{t = t_0,m=m_0}
\end{align*}

Furthermore, we pin down $\gamma_t(0)$ by (is this right? ask John maybe??)
\begin{align*}
\gamma_t(0) = \lambda \times \frac{\int_{0}^{\infty} \gamma_t(m) \tau_t(m) \mu_t(m) dm}{\int_{0}^{\infty} \tau_t(m) \mu_t(m) dm}
\end{align*}

As a sanity check, notice that setting $\partial_t \gamma_t(m) = 0$ yields the necessary condition for a constant $\gamma$ derived above:
\begin{align*}
\gamma'(m) = -\frac{g}{a(m)} \gamma(m) 
\end{align*}

\end{document}