\documentclass[12pt,english]{article}
\usepackage{lmodern}
\usepackage[T1]{fontenc}
\usepackage[latin9]{inputenc}
\usepackage{geometry}
\usepackage{amsthm}
\usepackage{courier}
\usepackage{verbatim}
\geometry{verbose,tmargin=1in,bmargin=1in,lmargin=1in,rmargin=1in}
\usepackage{setspace}
%\usepackage{esint}
\onehalfspacing
\usepackage{babel}
\usepackage{amsmath}

\theoremstyle{remark}
\newtheorem*{remark}{Remark}
\begin{document}
	
\title{Funding proposal for VentureSource Data}
\author{Nicolas Fernandez-Arias}
\maketitle

\section{Description of projects}

\subsection{Endogenous growth with creative destruction by employee spinouts}

\subsubsection{Overview}

This project consists of developing an model of endogenous growth with a new channel, 
calibrating it, and using it to analyze the effect on growth and welfare of various comparative statics and policies. 

In my framework, which builds on the endogenous growth with quality ladders framework in Grossman \& Helpman 1991, long-term growth is driven by profit-motivated investment in R\&D by firms in the economy. The key mechanism that distinguishes my framework from the existing literature is that a firm's current R\&D effort, which consists of hiring R\&D employees, leads to the formation of with-industry spinouts (WSOs) by the same R\&D employees. These spinouts can then compete against their original employers for a chance to obtain a (temporary) monopoly on the next step of the quality ladder. 

In addition to calibrating based on various aggregate moments, the key distinguishing feature of my analysis will be to investigate directly the central hypothesis in my model: that more R\&D spending by firms leads to more competing spinouts. In order to do this, I would like to use the following datasets:

\begin{enumerate}
	\item Compustat
	\item Bloom et al. 2013, "Identifying technology spillovers..." company-specific instruments for R\&D spending
	\item NBER-USPTO patent database on patents, their applicants, and citations
	\item VentureSource database of VC-funded firms
\end{enumerate}

\subsubsection{Model}

The model I am using is, technically, most closely related to the model in Akcigit \& Kerr, "Growth through Heterogeneous Innovations" (forthcoming in the JPE). The differences are:
\begin{enumerate}
	\item Workers are risk-neutral in my model
	\item In my model, the R\&D input is labor rather than the final good (and made some scaling assumptions on this to preserve a BGP)
	\item I have dropped ``external innovation'' by incumbents (and hence any explicit role for ``firm size'' - my analysis is at the product level)
	\item I have changed the innovation technology of entrants
	\item I have added a new class of potential entrant - an employee spinout - which can only form upon learning by working as an R\&D employee in that product line
\end{enumerate}

The resulting problem for either an incumbent or a spinout in a given product line $j$ then depends on a one-dimensional state variable, $m(j)$, which can be thought of as a measure of how widely disseminated the knowledge of ``how to compete in an race to develop the next step on the quality ladder for product line $j$". As is standard in this type of model, once there is a victory, the state of product $j$ resets -- in this case, $m$ is reset to $m = 0$. 

The resulting model exhibits a BGP with a constant growth rate $g$ in output, wages, etc; and constant distributions of products over states $\mu(m)$, constant $\gamma(m) = E[q_t / Q_t | m]$,\footnote{The joint distribution over $(q/Q_t,m)$ is not constant over time, but $E[q/Q_t | m] = \gamma(m)$ is. Only $\gamma(m)$ is necessary for computing aggregates.} and constant normalized R\&D wages by state of product $w(m) / Q_t$.\footnote{In other words, the R\&D wage for a product in state $m$ is $w(m) * Q_t$ for some constant function $w(m)$.} 


\subsection{How is the spawning of spinouts affected by changes in non-compete enforceability?}

A related project that will at least become an empirical paper, and ideally be useful in a more detailed model-based analysis as well, is to study how this process is mediated by the enforceability of non-competition agreements. 

Papers such as Jeffers, "The Impact of Restricting Labor Mobility on Corporate Investment and Entrepreneurship" and others (e.g. Garmaise, Marx, etc.) have identified plausibly exogenous state-level changes in non-compete enforceability and explored, using triple-diff identification strategies, their impact on average departure and/or entrepreneurship rates by employees of various types, as well as on net capital investment by firms. 

However, no study I am aware of has looked inside the box, so to speak, and studied how specific activities by the firm - in my case, R\&D - themselves create the conditions leading to employee departures. Or obtained an evidence of how this relationship is conditioned by the enforceability of noncompetes. 

In addition, studying this question with VentureSource data, as opposd to LinkedIn data as was in the case of Jeffers, offers a complementary picture. As pointed out in her paper, LinkedIn helps her focus on "growth entrepreneurship" as opposedto "new establishments or new subsistence entrepreneurship that census-type data capture." Since VentureSource focuses exclusively on venture-capital funded enterprises, it serves to focus the analysis even further. 


\section{Role for Venture Source data}

In both projects, the role of Venture Source data is that it allows me to implement the following empirical strategy: 

\begin{enumerate}
	\item Identify the within-industry spinouts that a particular firm produces
	\item Relate this number to firm-level R\&D spending
	\item Leverage firm-level instruments for R\&D for causal identification
\end{enumerate}

This causal identification is crucial. If I only use aggregate moments to calibrate the framework I described above, I will essentially be assuming that any relationship R\&D spending and spinout formation is causal. Equivalently, if I wrote the model to nest both possibilities -- e.g., adding an intercept term\footnote{And also allowed for firm-level shocks to R\&D and spinout formation to be correlated}  -- I would not be able to separately identify the slope and intercept. However this distinction is key for all comparative statics and policy analyses I am considering. 

\subsection{Description of data set}

The VentureSource contains information on VC-funded companies going back to 1986. 

\subsubsection{Features and coverage}

Venture Source contains:
\begin{itemize}
	\item Firm information: description, industry, products, revenue, ...
	\item Employee bios (for founders and some C-level): previous employment, education?
	\item Funding information: dates, valuations, VC firms, funding details
\end{itemize}

The coverage continues to be solid into the present day, for both founders and CTOs. 

\subsubsection{Cost}

The cost structure is
\begin{enumerate}
	\item \$10,000: One-year academic license containing only basic firm information (name, location, industry) and founder bios.
	\item \$15,000: Above plus valuations at funding dates (useful for extending my analysis)
	\item \$17,000: Above plus product and revenue data (not much coverage)
	\item \$20,000: Full US archive
	\item \$30,000: Full global archive (coverage outside US is not as good)
	\item For multiple-year contracts, 5\% is deducted from the first year cost.
\end{enumerate}

The bare minimums for the projects described in this document is (1), but it would be useful to have (2) so that I can have a measure of the success of these firms. 

\subsection{Proposed methodology}

\subsubsection{Endogenous growth with spinouts}

My goal is to identify the causal effect of firm R\&D on spinout formation. My approach consists of combining two empirical strategies:
\begin{enumerate}
	\item Instrumenting R\&D to get exogenous variation in firm-level R\&D spending
	\item Identifying within-industry spinouts by employment histories of their founders
\end{enumerate}

The first I will accomplish by following the paper by Bloom et al., "Identifying Technology Spillovers and Product-Market Rivalry", which develops firm-specific instruments for R\&D spending based on state-level changes in tax incentives for R\&D. This also involves using Compustat data - for firm-year-level R\&D expenditures - and NBER-USPTO patent database data - for inferring in which states a multi-state firm's R\&D occurs in, to develop their specific index of effective total R\&D subsidies. 

The second I will accomplish by following Gompers, Lerner and Scharfstein, "Entrepreneurial Spawning: Public Corporations and the Genesis of New Ventures, 1986 to 1999," which also uses the VentureSource dataset. 

\subsubsection{How is the spawning of spinouts affected by changes in non-compete enforceability?}

This second project broadly speaking consists of analyzing the extent to which the results of the above analysis depends on the enforceability of non-competes.

At least one additional complexity emerges. A simple test of how changes in non-compete enforceability affect the difference in the relationship between R\&D spending and WSO formation (i.e. a Diff-in-Diff) is not robust to different state-level trends in this coefficient. Hence I may incorporate a diff-in-diff-in-diff, by using an industry not affected, or less affected, by changes in non-compete enforceability as a control. Usually, the legal industry is used as this control, since it is often not subject to the same restrictions on non-compete agreements. However, it is unclear how prominent in my data this industry will be as my datasets focus on public and VC-funded firms. Hence, to inform this component, I plan to refer to survey evidence on prevalence of non-compete agreements in Starr et al., "Noncompetes in the U.S. Labor Force". 

\end{document}